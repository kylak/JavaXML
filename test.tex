% !TEX encoding = UTF-8 Unicode
 \documentclass[a4paper, 11pt]{article}
 \usepackage[utf8]{inputenc}
 \usepackage[french]{babel}
 \usepackage[T1]{fontenc}
 \usepackage{fontspec}
 \usepackage{lmodern}
 \usepackage{array}
 \usepackage{verbatim}
  \font\myfont=cmr12 at 21pt
 \title{{\myfont ``GA 032''}}
 \usepackage{layout}
 \usepackage[nomarginpar, margin=0.7in]{geometry}
 \pagestyle{plain}

 \usepackage{polyglossia}
 \setmainlanguage{french}
 \setotherlanguage{greek}
 \newfontfamily\greekfont{KoineGreek}
\newcommand\Pheader{\rule[-2ex]{0pt}{5ex}}
\newsavebox\TBox
\def\textoverline#1{\savebox\TBox{#1}%
\makebox[0pt][l]{#1}\rule[1.1\ht\TBox]{\wd\TBox}{0.7pt}}
 % amélioration : ajouter un "padding" sur le tabular + agrandir le tabular et son contenu.

\usepackage{pageslts}
\usepackage{fancyhdr}
\makeatletter
\newcommand{\nospace}[1]{\nofrench@punctuation\texttt{#1}\french@punctuation}
\makeatother
\let\oldtabular\tabular\renewcommand{\tabular}{\large\selectfont\oldtabular} %fontsize{17pt}{20.5pt}

\usepackage[hidelinks]{hyperref}

\newcounter{gospelbook}
\setcounter{gospelbook}{1}
\newcommand{\mygospelbook}[1]
{\setcounter{gospelchapter}{1}\phantomsection\addcontentsline{toc}{part}{#1}#1}


\newcommand{\agospelbook}[1]{\addtocontents{toc}{\protect\newpage}\mygospelbook{#1}}

\newcounter{gospelchapter}
\newcommand{\mygospelchapter}{\phantomsection\addcontentsline{toc}{section}{\thegospelchapter}\LARGE\bfseries\thegospelchapter\refstepcounter{gospelchapter}}


\begin{document}
\renewcommand{\contentsname}{Sommaire}
 %\layout
 \maketitle % affiche le nom du manuscrit.
\pagenumbering{roman}
\thispagestyle{empty}\clearpage\setcounter{page}{1}
\newpage
\foreignlanguage{greek}{\tableofcontents}
\clearpage\pagenumbering{arabic}\setcounter{page}{1}
\clearpage
\newpage
 {
 \setlength\arrayrulewidth{1pt}
\begin{table}
\begin{center}
\begin{tabular}{ccc|l|ccc}
\cline{4-4} \\ [-1em]
\multicolumn{7}{c}{\mygospelbook{\foreignlanguage{greek}{ευαγγελιον κατα μαθθαιον}} \textbf{(\nospace{1:1})} } \\ \\ [-1em] % Si on veut ajouter les bordures latérales, remplacer {7}{c} par {7}{|c|}
\cline{4-4} \\
\cline{4-4}
&  &  & &  &  & \\ [-0.9em]
& \mygospelchapter &  & \foreignlanguage{greek}{βιβλοϲ γενεϲεωϲ \textoverline{ιυ} \textoverline{χυ} υιου δαυειδ} & 6 &  &  \\
&  & 7 & \foreignlanguage{greek}{υιου αβρααμ αβρααμ εγεννηϲεν τον} & 3 & \textbf{2} &  \\
&  & 4 & \foreignlanguage{greek}{ιϲαακ ιϲαακ δε εγεννηϲεν τον ια} & 9 &  &  \\
&  & 9 & \foreignlanguage{greek}{κωβ ιακωβ δε εγεννηϲεν τον ιουδαν} & 14 &  &  \\
&  & 15 & \foreignlanguage{greek}{και τουϲ αδελφουϲ αυτου ιουδαϲ δε} & 2 & \textbf{3} &  \\
&  & 3 & \foreignlanguage{greek}{εγεννηϲεν τον φαρεϲ και τον ζαρα} & 8 &  &  \\
&  & 9 & \foreignlanguage{greek}{εκ τηϲ θαμαρ φαρεϲ δε εγεννηϲεν} & 14 &  &  \\
&  & 15 & \foreignlanguage{greek}{τον εζρωμ εζρωμ δε εγεννηϲεν το̅} & 20 &  &  \\
&  & 21 & \foreignlanguage{greek}{αραμ αραμ δε εγεννηϲεν τον αμινα} & 5 & \textbf{4} &  \\
&  & 5 & \foreignlanguage{greek}{δαβ αμιναδαβ δε εγεννηϲεν τον} & 9 &  &  \\
&  & 10 & \foreignlanguage{greek}{νααϲϲων νααϲϲων δε εγεννηϲε̅} & 13 &  &  \\
&  & 14 & \foreignlanguage{greek}{τον ϲαλμων ϲαλμων δε εγεννηϲε̅} & 3 & \textbf{5} &  \\
&  & 4 & \foreignlanguage{greek}{τον βοοζ εκ τηϲ ραχαβ βοοζ δε εγε̅} & 11 &  &  \\
&  & 11 & \foreignlanguage{greek}{νηϲεν τον ωβηδ εκ τηϲ ρουθ ωβηδ} & 17 &  &  \\
&  & 18 & \foreignlanguage{greek}{δε εγεννηϲεν τον ειεϲϲαι ιεϲϲαι δε} & 2 & \textbf{6} &  \\
&  & 3 & \foreignlanguage{greek}{εγεννηϲεν τον δαυειδ τον βαϲιλεα} & 7 &  &  \\
&  & 8 & \foreignlanguage{greek}{δαυειδ δε ο βαϲιλευϲ εγεννηϲεν το̅} & 13 &  &  \\
&  & 14 & \foreignlanguage{greek}{ϲολομωντα εκ τηϲ του ουριου ϲολο} & 1 & \textbf{7} &  \\
&  & 1 & \foreignlanguage{greek}{μων δε εγεννηϲεν τον ροβοαμ} & 5 &  &  \\
&  & 6 & \foreignlanguage{greek}{ροβοαμ δε εγεννηϲεν τον αβια αβι} & 11 &  &  \\
&  & 11 & \foreignlanguage{greek}{α δε εγεννηϲεν τον αϲα αϲα δε εγε̅} & 3 & \textbf{8} &  \\
&  & 3 & \foreignlanguage{greek}{νηϲεν τον ιωϲαφατ ιωϲαφατ δε} & 7 &  &  \\
&  & 8 & \foreignlanguage{greek}{εγεννηϲεν τον ιωραμ ιωραμ δε εγε̅} & 13 &  &  \\
&  & 13 & \foreignlanguage{greek}{νηϲεν τον οζειαν οζειαϲ δε εγεννη} & 3 & \textbf{9} &  \\
&  & 3 & \foreignlanguage{greek}{ϲεν τον ιωαθαμ ιωαθαμ δε εγεννη} & 9 &  &  \\
&  & 9 & \foreignlanguage{greek}{ϲεν τον αχαζ αχαζ δε εγεννηϲεν} & 14 &  &  \\
&  & 15 & \foreignlanguage{greek}{τον εζεκιαν εζεκιαϲ δε εγεννηϲε̅} & 3 & \textbf{10} &  \\
&  & 4 & \foreignlanguage{greek}{τον μαναϲϲη μαναϲηϲ δε εγεννη} & 8 &  &  \\
&  & 8 & \foreignlanguage{greek}{ϲεν τον αμων αμων δε εγεννηϲεν} & 13 &  &  \\
&  & 14 & \foreignlanguage{greek}{τον ιωϲιαν ιωϲιαϲ δε εγεννηϲεν το̅} & 4 & \textbf{11} &  \\
[0.2em]
\cline{4-4}
\end{tabular}
\end{center}
\end{table}
}
\clearpage
\newpage
 {
 \setlength\arrayrulewidth{1pt}
\begin{table}
\begin{center}
\begin{tabular}{ccc|l|ccc}
\cline{4-4} \\ [-1em]
\multicolumn{7}{c}{\foreignlanguage{greek}{ευαγγελιον κατα μαθθαιον} \textbf{(\nospace{1:11})} } \\ \\ [-1em] % Si on veut ajouter les bordures latérales, remplacer {7}{c} par {7}{|c|}
\cline{4-4} \\
\cline{4-4}
&  &  & &  &  & \\ [-0.9em]
&  & 5 & \foreignlanguage{greek}{ιεχονιαν και τουϲ αδελφουϲ αυτου} & 9 &  &  \\
&  & 10 & \foreignlanguage{greek}{επι τηϲ μετοικεϲιαϲ βαβυλωνοϲ} & 13 &  &  \\
& \textbf{12} &  & \foreignlanguage{greek}{μετα δε την μετοικεϲιαν βαβυλωνοϲ} & 5 &  &  \\
&  & 6 & \foreignlanguage{greek}{ιεχονιαϲ εγεννηϲεν τον ϲαλαθιηλ} & 9 &  &  \\
&  & 10 & \foreignlanguage{greek}{ϲαλαθιηλ δε εγεννηϲεν τον ζορο} & 14 &  &  \\
&  & 14 & \foreignlanguage{greek}{βαβελ ζοροβαβελ δε εγεννηϲεν} & 3 & \textbf{13} &  \\
&  & 4 & \foreignlanguage{greek}{τον αβιουδ αβιουδ δε εγεννηϲεν} & 8 &  &  \\
&  & 9 & \foreignlanguage{greek}{τον ελιακιμ ελιακιμ δε εγεννη} & 13 &  &  \\
&  & 13 & \foreignlanguage{greek}{ϲεν τον αζωρ αζωρ δε εγεννηϲε̅} & 3 & \textbf{14} &  \\
&  & 4 & \foreignlanguage{greek}{τον ϲαδδωκ ϲαδδωκ δε εγεννηϲε̅} & 8 &  &  \\
&  & 9 & \foreignlanguage{greek}{τον αχειν αχειν δε εγεννηϲεν} & 13 &  &  \\
&  & 14 & \foreignlanguage{greek}{τον ελιουδ ελιουδ δε εγεννηϲεν} & 3 & \textbf{15} &  \\
&  & 4 & \foreignlanguage{greek}{τον ελεαζαρ ελεαζαρ δε εγεννηϲε̅} & 8 &  &  \\
&  & 9 & \foreignlanguage{greek}{τον ματθαν ματθαν δε εγεννη} & 13 &  &  \\
&  & 13 & \foreignlanguage{greek}{ϲεν τον ιακωβ ιακωβ δε εγεννη} & 3 & \textbf{16} &  \\
&  & 3 & \foreignlanguage{greek}{ϲεν τον ιωϲηφ τον ανδρα μαριαϲ} & 8 &  &  \\
&  & 9 & \foreignlanguage{greek}{εξ ηϲ εγεννηθη \textoverline{ιϲ} ο λεγομενοϲ \textoverline{χϲ}} & 15 &  &  \\
& \textbf{17} &  & \foreignlanguage{greek}{παϲαι ουν αι γενεαι απο αβρααμ ε} & 7 &  &  \\
&  & 7 & \foreignlanguage{greek}{ωϲ δαυειδ γενεαι δεκατεϲϲαρεϲ} & 10 &  &  \\
&  & 11 & \foreignlanguage{greek}{και απο δαυειδ εωϲ τηϲ μετοικε} & 16 &  &  \\
&  & 16 & \foreignlanguage{greek}{ϲιαϲ βαβυλωνοϲ γενεαι \textoverline{ιδ}} & 19 &  &  \\
&  & 20 & \foreignlanguage{greek}{και απο τηϲ μετοικεϲιαϲ βαβυλω} & 24 &  &  \\
&  & 24 & \foreignlanguage{greek}{νοϲ εωϲ του \textoverline{χυ} γενεαι \textoverline{ιδ}} & 29 &  &  \\
& \textbf{18} &  & \foreignlanguage{greek}{του δε \textoverline{ιυ} η γενεϲειϲ ουτωϲ ην} & 7 &  &  \\
&  & 8 & \foreignlanguage{greek}{μνηϲτευθειϲηϲ γαρ τηϲ μητροϲ αυ} & 12 &  &  \\
&  & 12 & \foreignlanguage{greek}{του μαριαϲ τω ιωϲηφ πριν η ϲυνελ} & 18 &  &  \\
&  & 18 & \foreignlanguage{greek}{θειν αυτουϲ ευρεθη εν γαϲτρι εχου} & 23 &  &  \\
&  & 23 & \foreignlanguage{greek}{ϲα εκ \textoverline{πνϲ} αγιου} & 26 &  &  \\
& \textbf{19} &  & \foreignlanguage{greek}{ιωϲηφ δε ο ανηρ αυτηϲ δικαιοϲ ων} & 7 &  &  \\
&  & 8 & \foreignlanguage{greek}{και μη θελων αυτην παραδιγματιϲαι} & 12 &  &  \\
[0.2em]
\cline{4-4}
\end{tabular}
\end{center}
\end{table}
}
\clearpage
\newpage
 {
 \setlength\arrayrulewidth{1pt}
\begin{table}
\begin{center}
\begin{tabular}{ccc|l|ccc}
\cline{4-4} \\ [-1em]
\multicolumn{7}{c}{\foreignlanguage{greek}{ευαγγελιον κατα μαθθαιον} \textbf{(\nospace{1:19})} } \\ \\ [-1em] % Si on veut ajouter les bordures latérales, remplacer {7}{c} par {7}{|c|}
\cline{4-4} \\
\cline{4-4}
&  &  & &  &  & \\ [-0.9em]
&  & 13 & \foreignlanguage{greek}{εβουληθη λαθρα απολυϲαι αυτην} & 16 &  &  \\
& \textbf{20} &  & \foreignlanguage{greek}{ταυτα δε αυτου ενθυμηθεντοϲ ιδου} & 5 &  &  \\
&  & 6 & \foreignlanguage{greek}{αγγελοϲ \textoverline{κυ} εφανη κατ οναρ αυτω λεγω̅} & 12 &  &  \\
&  & 13 & \foreignlanguage{greek}{ιωϲηφ υιοϲ δαυειδ μη φοβηθηϲ παρα} & 18 &  &  \\
&  & 18 & \foreignlanguage{greek}{λαβειν μαριαμ την γυναικα ϲου} & 22 &  &  \\
&  & 23 & \foreignlanguage{greek}{το γαρ εν αυτη γεννηθεν εκ \textoverline{πνϲ} εϲτι̅} & 30 &  &  \\
&  & 31 & \foreignlanguage{greek}{αγιου τεξεται δε υιον και καλεϲιϲ} & 5 & \textbf{21} &  \\
&  & 6 & \foreignlanguage{greek}{το ονομα αυτου \textoverline{ιν} αυτοϲ γαρ ϲωϲει} & 12 &  &  \\
&  & 13 & \foreignlanguage{greek}{τον λαον αυτου απο των αμαρτιων} & 18 &  &  \\
&  & 19 & \foreignlanguage{greek}{αυτων τουτο δε ολον γεγονεν} & 4 & \textbf{22} &  \\
&  & 5 & \foreignlanguage{greek}{ινα πληρωθη το ρηθεν υπο \textoverline{κυ} δια του} & 12 &  &  \\
&  & 13 & \foreignlanguage{greek}{προφητου λεγοντοϲ} & 14 &  &  \\
& \textbf{23} &  & \foreignlanguage{greek}{ιδου η παρθενοϲ εν γαϲτρι εξει και} & 7 &  &  \\
&  & 8 & \foreignlanguage{greek}{τεξεται υιον και καλεϲουϲιν το ο} & 13 &  &  \\
&  & 13 & \foreignlanguage{greek}{νομα αυτου εμμανουηλ ο εϲτιν} & 17 &  &  \\
&  & 18 & \foreignlanguage{greek}{μεθερμηνευομενον μεθ ημων ο \textoverline{θϲ}} & 22 &  &  \\
& \textbf{24} &  & \foreignlanguage{greek}{διεγερθειϲ δε ο ιωϲηφ απο του υπνου} & 7 &  &  \\
&  & 8 & \foreignlanguage{greek}{εποιηϲεν ωϲ προϲεταξεν αυτω ο} & 12 &  &  \\
&  & 13 & \foreignlanguage{greek}{αγγελοϲ \textoverline{κυ} και παρελαβεν την γυ} & 18 &  &  \\
&  & 18 & \foreignlanguage{greek}{ναικα αυτου και ουκ εγινωϲκεν αυ} & 4 & \textbf{25} &  \\
&  & 4 & \foreignlanguage{greek}{την εωϲ ου ετεκεν τον υιον αυτηϲ} & 10 &  &  \\
&  & 11 & \foreignlanguage{greek}{τον πρωτοτοκον και εκαλεϲεν το} & 15 &  &  \\
&  & 16 & \foreignlanguage{greek}{ονομα αυτου \textoverline{ιν}} & 18 &  &  \\
& \mygospelchapter &  & \foreignlanguage{greek}{του δε \textoverline{ιυ} γεννηθεντοϲ εν βηθλεεμ} & 6 &  &  \\
&  & 7 & \foreignlanguage{greek}{τηϲ ιουδαιαϲ εν ημεραιϲ ηρωδου του} & 12 &  &  \\
&  & 13 & \foreignlanguage{greek}{βαϲιλεωϲ ιδου μαγοι απο ανατο} & 17 &  &  \\
&  & 17 & \foreignlanguage{greek}{λων παρεγενοντο ειϲ ιερουϲαλημ} & 20 &  &  \\
& \textbf{2} &  & \foreignlanguage{greek}{λεγοντεϲ που εϲτιν ο τεχθειϲ βα} & 6 &  &  \\
&  & 6 & \foreignlanguage{greek}{ϲιλευϲ των ιουδαιων ιδομεν γαρ} & 10 &  &  \\
&  & 11 & \foreignlanguage{greek}{αυτου τον αϲτερα εν τη ανατολη} & 16 &  &  \\
[0.2em]
\cline{4-4}
\end{tabular}
\end{center}
\end{table}
}
\clearpage
\newpage
 {
 \setlength\arrayrulewidth{1pt}
\begin{table}
\begin{center}
\begin{tabular}{ccc|l|ccc}
\cline{4-4} \\ [-1em]
\multicolumn{7}{c}{\foreignlanguage{greek}{ευαγγελιον κατα μαθθαιον} \textbf{(\nospace{2:2})} } \\ \\ [-1em] % Si on veut ajouter les bordures latérales, remplacer {7}{c} par {7}{|c|}
\cline{4-4} \\
\cline{4-4}
&  &  & &  &  & \\ [-0.9em]
&  & 17 & \foreignlanguage{greek}{και ηλθομεν προϲκυνηϲαι αυτω} & 20 &  &  \\
& \textbf{3} &  & \foreignlanguage{greek}{ακουϲαϲ δε ηρωδηϲ ο βαϲιλευϲ εταρα} & 6 &  &  \\
&  & 6 & \foreignlanguage{greek}{χθη και παϲα ιεροϲολυμα μετ αυτου} & 11 &  &  \\
& \textbf{4} &  & \foreignlanguage{greek}{και ϲυναγαγων πανταϲ τουϲ αρχιερειϲ} & 5 &  &  \\
&  & 6 & \foreignlanguage{greek}{και γραμματιϲ του λαου επυνθανε} & 10 &  &  \\
&  & 10 & \foreignlanguage{greek}{το παρ αυτων που ο \textoverline{χϲ} γενναται} & 16 &  &  \\
& \textbf{5} &  & \foreignlanguage{greek}{οι δε ειπον αυτω εν βηθλεεμ τηϲ ιου} & 8 &  &  \\
&  & 8 & \foreignlanguage{greek}{δαιαϲ ουτωϲ γαρ γεγραπται δια του} & 13 &  &  \\
&  & 14 & \foreignlanguage{greek}{προφητου και ϲυ βηθλεεμ τη ιου} & 5 & \textbf{6} &  \\
&  & 5 & \foreignlanguage{greek}{δα ουδαμωϲ ελαχειϲτη ει εν τοιϲ η} & 11 &  &  \\
&  & 11 & \foreignlanguage{greek}{γεμοϲιν ιουδα εκ ϲου γαρ εξελευϲε} & 16 &  &  \\
&  & 16 & \foreignlanguage{greek}{ται ηγουμενοϲ οϲτιϲ ποιμανει τον} & 20 &  &  \\
&  & 21 & \foreignlanguage{greek}{λαον μου τον ιϲραηλ} & 24 &  &  \\
& \textbf{7} &  & \foreignlanguage{greek}{τοτε ηρωδηϲ λαθρα καλεϲαϲ τουϲ μα} & 6 &  &  \\
&  & 6 & \foreignlanguage{greek}{γουϲ ηκριβωϲεν παρ αυτων τον} & 10 &  &  \\
&  & 11 & \foreignlanguage{greek}{χρονον του φαινομενου αϲτεροϲ} & 14 &  &  \\
& \textbf{8} &  & \foreignlanguage{greek}{και πεμψαϲ αυτουϲ ειϲ βηθλεεμ ειπε̅} & 6 &  &  \\
&  & 7 & \foreignlanguage{greek}{πορευθεντεϲ ακριβωϲ εξεταϲατε} & 9 &  &  \\
&  & 10 & \foreignlanguage{greek}{περι του παιδιου επαν δε ευρηται} & 15 &  &  \\
&  & 16 & \foreignlanguage{greek}{απαγγειλαται μοι οπωϲ καγω ελ} & 20 &  &  \\
&  & 20 & \foreignlanguage{greek}{θων προϲκυνηϲω αυτω} & 22 &  &  \\
& \textbf{9} &  & \foreignlanguage{greek}{οι δε ακουϲαντεϲ του βαϲιλεωϲ επο} & 6 &  &  \\
&  & 6 & \foreignlanguage{greek}{ρευθηϲαν και ιδου ο αϲτηρ ον ει} & 12 &  &  \\
&  & 12 & \foreignlanguage{greek}{δον εν τη ανατολη προηγεν αυτουϲ} & 17 &  &  \\
&  & 18 & \foreignlanguage{greek}{εωϲ ελθων εϲτη επανω ου ην το} & 24 &  &  \\
&  & 25 & \foreignlanguage{greek}{παιδιον} & 25 &  &  \\
& \textbf{10} &  & \foreignlanguage{greek}{ιδοντεϲ δε τον αϲτερα εχαρηϲαν χα} & 6 &  &  \\
&  & 6 & \foreignlanguage{greek}{ραν μεγαλην ϲφοδρα} & 8 &  &  \\
& \textbf{11} &  & \foreignlanguage{greek}{και ελθοντεϲ ειϲ την οικειαν ιδον το} & 7 &  &  \\
&  & 8 & \foreignlanguage{greek}{παιδιον μετα μαριαϲ τηϲ μητροϲ αυτου} & 13 &  &  \\
[0.2em]
\cline{4-4}
\end{tabular}
\end{center}
\end{table}
}
\clearpage
\newpage
 {
 \setlength\arrayrulewidth{1pt}
\begin{table}
\begin{center}
\begin{tabular}{ccc|l|ccc}
\cline{4-4} \\ [-1em]
\multicolumn{7}{c}{\foreignlanguage{greek}{ευαγγελιον κατα μαθθαιον} \textbf{(\nospace{2:11})} } \\ \\ [-1em] % Si on veut ajouter les bordures latérales, remplacer {7}{c} par {7}{|c|}
\cline{4-4} \\
\cline{4-4}
&  &  & &  &  & \\ [-0.9em]
&  & 14 & \foreignlanguage{greek}{και πεϲοντεϲ προϲεκυνηϲαν αυτω} & 17 &  &  \\
&  & 18 & \foreignlanguage{greek}{και ανοιξαντεϲ τουϲ θηϲαυρουϲ αυτων} & 22 &  &  \\
&  & 23 & \foreignlanguage{greek}{προϲηνεγκαν αυτω δωρα χρυϲον και} & 27 &  &  \\
&  & 28 & \foreignlanguage{greek}{λιβανον και ϲμυρναν και χρηματι} & 2 & \textbf{12} &  \\
&  & 2 & \foreignlanguage{greek}{ϲθεντεϲ κατ οναρ μη ανακαμψαι προϲ} & 7 &  &  \\
&  & 8 & \foreignlanguage{greek}{ηρωδην δι αλληϲ οδου ανεχωρηϲαν} & 12 &  &  \\
&  & 13 & \foreignlanguage{greek}{ειϲ την χωραν αυτων} & 16 &  &  \\
& \textbf{13} &  & \foreignlanguage{greek}{αναχωρηϲαντων δε αυτων ιδου αγ} & 5 &  &  \\
&  & 5 & \foreignlanguage{greek}{γελοϲ \textoverline{κυ} φαινεται τω ιωϲηφ κατ οναρ} & 11 &  &  \\
&  & 12 & \foreignlanguage{greek}{λεγων εγερθειϲ παραλαβεν το παιδι} & 16 &  &  \\
&  & 16 & \foreignlanguage{greek}{ον και την μητερα αυτου και φευγε} & 22 &  &  \\
&  & 23 & \foreignlanguage{greek}{ειϲ αιγυπτον και ειϲθει εκει εωϲ αν} & 29 &  &  \\
&  & 30 & \foreignlanguage{greek}{ειπω ϲοι μελλει γαρ ηρωδηϲ ζητειν} & 35 &  &  \\
&  & 36 & \foreignlanguage{greek}{το παιδιον του απολεϲαι αυτο} & 40 &  &  \\
& \textbf{14} &  & \foreignlanguage{greek}{ο δε εγερθειϲ παρελαβεν το παιδιον} & 6 &  &  \\
&  & 7 & \foreignlanguage{greek}{και την μητερα αυτου νυκτοϲ και} & 12 &  &  \\
&  & 13 & \foreignlanguage{greek}{ανεχωρηϲεν ειϲ αιγυπτον και ην ε} & 3 & \textbf{15} &  \\
&  & 3 & \foreignlanguage{greek}{κει εωϲ τηϲ τελευτηϲ ηρωδου} & 7 &  &  \\
&  & 8 & \foreignlanguage{greek}{ινα πληρωθη το ρηθεν υπο \textoverline{κυ} δια του} & 15 &  &  \\
&  & 16 & \foreignlanguage{greek}{προφητου λεγοντοϲ εξ αιγυπτου} & 19 &  &  \\
&  & 20 & \foreignlanguage{greek}{εκαλεϲα τον υιον μου} & 23 &  &  \\
& \textbf{16} &  & \foreignlanguage{greek}{τοτε ηρωδηϲ ιδων οτι ενεπεχθη υ} & 6 &  &  \\
&  & 6 & \foreignlanguage{greek}{πο των γαμων εθυμωθη λιαν} & 10 &  &  \\
&  & 11 & \foreignlanguage{greek}{και αποϲτιλαϲ ανειλε πανταϲ τουϲ} & 15 &  &  \\
&  & 16 & \foreignlanguage{greek}{παιδαϲ τουϲ εν βηθλεεμ και εν παϲι} & 22 &  &  \\
&  & 23 & \foreignlanguage{greek}{τοιϲ οριοιϲ αυτηϲ απο διετουϲ και κα} & 29 &  &  \\
&  & 29 & \foreignlanguage{greek}{τωτερω κατα τον χρονον ον ηκριβω} & 34 &  &  \\
&  & 34 & \foreignlanguage{greek}{ϲεν παρα των μαγων τοτε επληρω} & 2 & \textbf{17} &  \\
&  & 2 & \foreignlanguage{greek}{θη το ρηθεν δια ιηρεμιου του προφητου λεγοντοϲ} & 9 &  &  \\
& \textbf{18} &  & \foreignlanguage{greek}{φωνη εν ραμα ηκουϲθη θρηνοϲ και} & 6 &  &  \\
[0.2em]
\cline{4-4}
\end{tabular}
\end{center}
\end{table}
}
\clearpage
\newpage
 {
 \setlength\arrayrulewidth{1pt}
\begin{table}
\begin{center}
\begin{tabular}{ccc|l|ccc}
\cline{4-4} \\ [-1em]
\multicolumn{7}{c}{\foreignlanguage{greek}{ευαγγελιον κατα μαθθαιον} \textbf{(\nospace{2:18})} } \\ \\ [-1em] % Si on veut ajouter les bordures latérales, remplacer {7}{c} par {7}{|c|}
\cline{4-4} \\
\cline{4-4}
&  &  & &  &  & \\ [-0.9em]
&  & 7 & \foreignlanguage{greek}{κλαθμοϲ και οδυρμοϲ πολυϲ ραχηλ} & 11 &  &  \\
&  & 12 & \foreignlanguage{greek}{κλεουϲα τα τεκνα αυτηϲ και ουκ η} & 18 &  &  \\
&  & 18 & \foreignlanguage{greek}{θελεν παρακληθηναι οτι ουκ ειϲιν} & 22 &  &  \\
& \textbf{19} &  & \foreignlanguage{greek}{τελευτηϲαντοϲ δε του ηρωδου ιδου} & 5 &  &  \\
&  & 6 & \foreignlanguage{greek}{αγγελοϲ \textoverline{κυ} κατ οναρ φαινεται τω ιω} & 12 &  &  \\
&  & 12 & \foreignlanguage{greek}{ϲηφ εν αιγυπτω λεγων εγερθειϲ πα} & 3 & \textbf{20} &  \\
&  & 3 & \foreignlanguage{greek}{ραλαβε το παιδιον και την μητερα} & 8 &  &  \\
&  & 9 & \foreignlanguage{greek}{αυτου και πορευου ειϲ γην ιϲραηλ} & 14 &  &  \\
&  & 15 & \foreignlanguage{greek}{τεθνηκαϲι γαρ οι ζητουντεϲ την} & 19 &  &  \\
&  & 20 & \foreignlanguage{greek}{ψυχην του παιδιου ο δε εγερθειϲ} & 3 & \textbf{21} &  \\
&  & 4 & \foreignlanguage{greek}{παρελαβεν το παιδιον και την μη} & 9 &  &  \\
&  & 9 & \foreignlanguage{greek}{τερα αυτου και ηλθεν ειϲ γην ιϲραηλ} & 15 &  &  \\
& \textbf{22} &  & \foreignlanguage{greek}{ακουϲαϲ δε οτι αρχελαοϲ βαϲιλευει ε} & 6 &  &  \\
&  & 6 & \foreignlanguage{greek}{πι τηϲ ιουδαιαϲ αντι του \textoverline{πρϲ} αυ} & 12 &  &  \\
&  & 12 & \foreignlanguage{greek}{του ηρωδου εφοβηθη εκει απελθειν} & 16 &  &  \\
&  & 17 & \foreignlanguage{greek}{χρηματιϲθειϲ δε κατ οναρ ανεχω} & 21 &  &  \\
&  & 21 & \foreignlanguage{greek}{ρηϲεν ειϲ τα μερη τηϲ γαλιλαιαϲ} & 26 &  &  \\
& \textbf{23} &  & \foreignlanguage{greek}{και ελθων κατωκηϲεν ειϲ πολιν λε} & 6 &  &  \\
&  & 6 & \foreignlanguage{greek}{γομενην ναζαρετ οπωϲ πληρω} & 9 &  &  \\
&  & 9 & \foreignlanguage{greek}{θη το ρηθεν δια των προφητων} & 14 &  &  \\
&  & 15 & \foreignlanguage{greek}{οτι ναζωρεοϲ κληθηϲεται} & 17 &  &  \\
& \mygospelchapter &  & \foreignlanguage{greek}{εν δε ταιϲ ημεραιϲ εκειναιϲ παραγει} & 6 &  &  \\
&  & 6 & \foreignlanguage{greek}{νεται ιωαννηϲ ο βαπτιϲτηϲ κηρυϲ} & 10 &  &  \\
&  & 10 & \foreignlanguage{greek}{ϲων εν τη ερημω τηϲ ιουδαιαϲ και} & 1 & \textbf{2} &  \\
&  & 2 & \foreignlanguage{greek}{λεγων μετανοειται ηγγεικε̅} & 4 &  &  \\
&  & 5 & \foreignlanguage{greek}{γαρ η βαϲιλεια των ουρανων ουτοϲ} & 1 & \textbf{3} &  \\
&  & 2 & \foreignlanguage{greek}{γαρ εϲτιν ο ρηθειϲ δια ηϲαιου του} & 8 &  &  \\
&  & 9 & \foreignlanguage{greek}{προφητου λεγοντοϲ φωνη βοω̅} & 12 &  &  \\
&  & 12 & \foreignlanguage{greek}{τοϲ εν τη ερημω ετοιμαϲατε την} & 17 &  &  \\
&  & 18 & \foreignlanguage{greek}{οδον \textoverline{κυ} ευθειαϲ ποιειται ταϲ τριβουϲ} & 23 &  &  \\
&  & 24 & \foreignlanguage{greek}{αυτου} & 24 &  &  \\
[0.2em]
\cline{4-4}
\end{tabular}
\end{center}
\end{table}
}
\clearpage
\newpage
 {
 \setlength\arrayrulewidth{1pt}
\begin{table}
\begin{center}
\begin{tabular}{ccc|l|ccc}
\cline{4-4} \\ [-1em]
\multicolumn{7}{c}{\foreignlanguage{greek}{ευαγγελιον κατα μαθθαιον} \textbf{(\nospace{3:4})} } \\ \\ [-1em] % Si on veut ajouter les bordures latérales, remplacer {7}{c} par {7}{|c|}
\cline{4-4} \\
\cline{4-4}
&  &  & &  &  & \\ [-0.9em]
& \textbf{4} &  & \foreignlanguage{greek}{αυτοϲ δε ο ιωαννηϲ ειχεν το ενδυμα αυ} & 8 &  &  \\
&  & 8 & \foreignlanguage{greek}{του απο τριχων καμηλου και ζωνη̅} & 13 &  &  \\
&  & 14 & \foreignlanguage{greek}{δερματινην περι την οϲφυν αυτου} & 18 &  &  \\
&  & 19 & \foreignlanguage{greek}{η δε τροφη ην αυτου ακριδεϲ και με} & 26 &  &  \\
&  & 26 & \foreignlanguage{greek}{λει αγριον} & 27 &  &  \\
& \textbf{5} &  & \foreignlanguage{greek}{τοτε εξεπορευετο προϲ αυτον ιερο} & 5 &  &  \\
&  & 5 & \foreignlanguage{greek}{ϲολυμα και παϲα η ιουδαια κα παϲα} & 11 &  &  \\
&  & 12 & \foreignlanguage{greek}{η περιχωροϲ του ιορδανου και εβα} & 2 & \textbf{6} &  \\
&  & 2 & \foreignlanguage{greek}{πτιζοντο εν τω ιορδανη παταμω} & 6 &  &  \\
&  & 7 & \foreignlanguage{greek}{υπ αυτου εξομολογουμενοι ταϲ α} & 11 &  &  \\
&  & 11 & \foreignlanguage{greek}{μαρτιαϲ αυτων} & 12 &  &  \\
& \textbf{7} &  & \foreignlanguage{greek}{ιδων δε πολλουϲ των φαριϲαιων και} & 6 &  &  \\
&  & 7 & \foreignlanguage{greek}{ϲαδδουκεων ερχομενουϲ επι το} & 10 &  &  \\
&  & 11 & \foreignlanguage{greek}{βαπτιϲμα αυτου ειπεν αυτοιϲ} & 14 &  &  \\
&  & 15 & \foreignlanguage{greek}{γεννηματα εχιδνων τιϲ υπεδει} & 18 &  &  \\
&  & 18 & \foreignlanguage{greek}{ξεν υμιν φυγειν απο τηϲ μελλου} & 23 &  &  \\
&  & 23 & \foreignlanguage{greek}{ϲηϲ οργηϲ ποιηϲατε ουν καρπον} & 3 & \textbf{8} &  \\
&  & 4 & \foreignlanguage{greek}{αξιον τηϲ μετανοιαϲ και μη δοξη} & 3 & \textbf{9} &  \\
&  & 3 & \foreignlanguage{greek}{ται λεγειν εν εαυτοιϲ \textoverline{πρα} εχο} & 8 &  &  \\
&  & 8 & \foreignlanguage{greek}{μεν τον αβρααμ λεγω γαρ υμιν} & 13 &  &  \\
&  & 14 & \foreignlanguage{greek}{οτι δυναται ο \textoverline{θϲ} εκ των λιθων του} & 21 &  &  \\
&  & 21 & \foreignlanguage{greek}{των εγειρε τεκνα τω αβρααμ} & 25 &  &  \\
& \textbf{10} &  & \foreignlanguage{greek}{ηδη δε η αξινη προϲ την ριζαν των} & 8 &  &  \\
&  & 9 & \foreignlanguage{greek}{δενδρων κειται παν ουν δενδρο̅} & 13 &  &  \\
&  & 14 & \foreignlanguage{greek}{μη ποιουν καρπον καλον εκκοπτε} & 18 &  &  \\
&  & 18 & \foreignlanguage{greek}{ται και ειϲ πυρ βαλλεται} & 22 &  &  \\
& \textbf{11} &  & \foreignlanguage{greek}{εγω μεν υμαϲ βαπτιζω εν υδατι} & 6 &  &  \\
&  & 7 & \foreignlanguage{greek}{ειϲ μετανοιαν ο δε οπιϲω μου ερ} & 13 &  &  \\
&  & 13 & \foreignlanguage{greek}{χομενοϲ ιϲχυροτεροϲ μου εϲτιν} & 16 &  &  \\
&  & 17 & \foreignlanguage{greek}{ου ουκ ειμι ικανοϲ τα υποδηματα} & 22 &  &  \\
[0.2em]
\cline{4-4}
\end{tabular}
\end{center}
\end{table}
}
\clearpage
\newpage
 {
 \setlength\arrayrulewidth{1pt}
\begin{table}
\begin{center}
\begin{tabular}{ccc|l|ccc}
\cline{4-4} \\ [-1em]
\multicolumn{7}{c}{\foreignlanguage{greek}{ευαγγελιον κατα μαθθαιον} \textbf{(\nospace{3:11})} } \\ \\ [-1em] % Si on veut ajouter les bordures latérales, remplacer {7}{c} par {7}{|c|}
\cline{4-4} \\
\cline{4-4}
&  &  & &  &  & \\ [-0.9em]
&  & 23 & \foreignlanguage{greek}{βαϲταϲαι αυτοϲ υμαϲ βαπτιϲει εν \textoverline{πνι}} & 28 &  &  \\
&  & 29 & \foreignlanguage{greek}{αγιω και πυρι ου το πτοιον εν τη} & 5 & \textbf{12} &  \\
&  & 6 & \foreignlanguage{greek}{χειρι αυτου και διακαθαριει την α} & 11 &  &  \\
&  & 11 & \foreignlanguage{greek}{λωνα αυτου και ϲυναξει τον ϲιτον} & 16 &  &  \\
&  & 17 & \foreignlanguage{greek}{αυτου ειϲ την αποθηκην αυτου το δε} & 23 &  &  \\
&  & 24 & \foreignlanguage{greek}{αχυρον κατακαυϲει πυρι αβεϲτω} & 27 &  &  \\
& \textbf{13} &  & \foreignlanguage{greek}{τοτε παραγεινεται ο \textoverline{ιϲ} απο τηϲ γαλιλαι} & 7 &  &  \\
&  & 7 & \foreignlanguage{greek}{αϲ επι τον ιορδανην προϲ τον ιωαν} & 13 &  &  \\
&  & 13 & \foreignlanguage{greek}{νην του βαπτιϲθηναι υπ αυτου} & 17 &  &  \\
& \textbf{14} &  & \foreignlanguage{greek}{ο δε ιωαννηϲ διεκωλυεν αυτον λεγω̅} & 6 &  &  \\
&  & 7 & \foreignlanguage{greek}{εγω χριαν εχω υπο ϲου βαπτιϲθηναι} & 12 &  &  \\
&  & 13 & \foreignlanguage{greek}{και ϲυ ερχη προϲ με} & 17 &  &  \\
& \textbf{15} &  & \foreignlanguage{greek}{αποκριθειϲ δε ο \textoverline{ιϲ} ειπεν προϲ αυτον} & 7 &  &  \\
&  & 8 & \foreignlanguage{greek}{αφεϲ αρτι ουτωϲ γαρ πρεπον εϲτιν η} & 14 &  &  \\
&  & 14 & \foreignlanguage{greek}{μιν πληρωϲαι παϲαν δικαιωϲυνην} & 17 &  &  \\
&  & 18 & \foreignlanguage{greek}{τοτε αφιηϲιν αυτον και βαπτιϲθειϲ} & 2 & \textbf{16} &  \\
&  & 3 & \foreignlanguage{greek}{ο \textoverline{ιϲ} ευθυϲ ανεβη απο του υδατοϲ} & 9 &  &  \\
&  & 10 & \foreignlanguage{greek}{και ιδου ανεωχθηϲαν αυτω οι ουρανοι} & 15 &  &  \\
&  & 16 & \foreignlanguage{greek}{και ιδεν το \textoverline{πνα} του \textoverline{θυ} καταβαινον} & 22 &  &  \\
&  & 23 & \foreignlanguage{greek}{ωϲει περιϲτεραν και ερχομενον} & 26 &  &  \\
&  & 27 & \foreignlanguage{greek}{επ αυτον} & 28 &  &  \\
& \textbf{17} &  & \foreignlanguage{greek}{και ιδου φωνη εκ του ουρανου λεγου} & 7 &  &  \\
&  & 7 & \foreignlanguage{greek}{ϲα ουτοϲ εϲτιν ο υιοϲ μου ο αγαπη} & 14 &  &  \\
&  & 14 & \foreignlanguage{greek}{τοϲ εν ω ηυδοκηϲα} & 17 &  &  \\
& \mygospelchapter &  & \foreignlanguage{greek}{τοτε ο \textoverline{ιϲ} ανηχθη ειϲ την ερημον υπο} & 8 &  &  \\
&  & 9 & \foreignlanguage{greek}{του \textoverline{πνϲ} πιραϲθηναι υπο του διαβολου} & 14 &  &  \\
& \textbf{2} &  & \foreignlanguage{greek}{και νηϲτευϲαϲ ημεραϲ τεϲϲαρακοντα} & 4 &  &  \\
&  & 5 & \foreignlanguage{greek}{και νυκταϲ τεϲϲαρακοντα υϲτερον} & 8 &  &  \\
&  & 9 & \foreignlanguage{greek}{επιναϲεν και προϲελθων ο πειρα} & 4 & \textbf{3} &  \\
&  & 4 & \foreignlanguage{greek}{ζων ειπεν αυτω ει υιοϲ ει του \textoverline{θυ} ειπε} & 12 &  &  \\
[0.2em]
\cline{4-4}
\end{tabular}
\end{center}
\end{table}
}
\clearpage
\newpage
 {
 \setlength\arrayrulewidth{1pt}
\begin{table}
\begin{center}
\begin{tabular}{ccc|l|ccc}
\cline{4-4} \\ [-1em]
\multicolumn{7}{c}{\foreignlanguage{greek}{ευαγγελιον κατα μαθθαιον} \textbf{(\nospace{4:3})} } \\ \\ [-1em] % Si on veut ajouter les bordures latérales, remplacer {7}{c} par {7}{|c|}
\cline{4-4} \\
\cline{4-4}
&  &  & &  &  & \\ [-0.9em]
&  & 13 & \foreignlanguage{greek}{ινα οι λιθοι ουτοι αρτοι γενωνται} & 18 &  &  \\
& \textbf{4} &  & \foreignlanguage{greek}{ο δε αποκριθειϲ ειπεν γεγραπται ουκ ε} & 7 &  &  \\
&  & 7 & \foreignlanguage{greek}{π αρτω μονω ζηϲεται ο ανθρωποϲ αλλ} & 13 &  &  \\
&  & 14 & \foreignlanguage{greek}{επι παντι ρηματι εκπορευομενω δια του} & 19 &  &  \\
&  & 20 & \foreignlanguage{greek}{ϲτοματοϲ \textoverline{θυ} τοτε παραλαμβανει αυτο̅} & 3 & \textbf{5} &  \\
&  & 4 & \foreignlanguage{greek}{ο διαβολοϲ ειϲ την αγιαν πολιν και ιϲτη} & 11 &  &  \\
&  & 11 & \foreignlanguage{greek}{ϲιν αυτον επι το πτερυγιον του ιερου} & 17 &  &  \\
& \textbf{6} &  & \foreignlanguage{greek}{και ειπεν αυτω ει υιοϲ ει του \textoverline{θυ} βαλε} & 9 &  &  \\
&  & 10 & \foreignlanguage{greek}{ϲεαυτον κατω γεγραπται γαρ οτι τοιϲ} & 15 &  &  \\
&  & 16 & \foreignlanguage{greek}{αγγελοιϲ αυτου εντελειται περι ϲου} & 20 &  &  \\
&  & 21 & \foreignlanguage{greek}{και επι χειρων αρουϲιν ϲε μηποτε προϲ} & 27 &  &  \\
&  & 27 & \foreignlanguage{greek}{κοψηϲ προϲ λιθον τον ποδα ϲου} & 32 &  &  \\
& \textbf{7} &  & \foreignlanguage{greek}{εφη αυτω ο \textoverline{ιϲ} παλιν γεγραπται ουκ} & 7 &  &  \\
&  & 8 & \foreignlanguage{greek}{εκπειραϲειϲ \textoverline{κν} τον \textoverline{θν} ϲου} & 12 &  &  \\
& \textbf{8} &  & \foreignlanguage{greek}{παλιν παραλαμβανει αυτον ο διαβο} & 5 &  &  \\
&  & 5 & \foreignlanguage{greek}{λοϲ ειϲ οροϲ υψηλον λιαν και δικνυ} & 11 &  &  \\
&  & 11 & \foreignlanguage{greek}{ϲιν αυτω παϲαϲ ταϲ βαϲιλειαϲ του} & 16 &  &  \\
&  & 17 & \foreignlanguage{greek}{κοϲμου και την δοξαν αυτων και λε} & 2 & \textbf{9} &  \\
&  & 2 & \foreignlanguage{greek}{γει αυτω ταυτα ϲοι παντα δωϲω ε} & 8 &  &  \\
&  & 8 & \foreignlanguage{greek}{αν πεϲων προϲκυνηϲηϲ μοι} & 11 &  &  \\
& \textbf{10} &  & \foreignlanguage{greek}{τοτε λεγει αυτω ο \textoverline{ιϲ} υπαγε ϲατανα} & 7 &  &  \\
&  & 8 & \foreignlanguage{greek}{γεγραπται γαρ \textoverline{κν} τον \textoverline{θν} ϲου προϲκυ} & 14 &  &  \\
&  & 14 & \foreignlanguage{greek}{νηϲειϲ και αυτω μονω λατρευϲιϲ} & 18 &  &  \\
& \textbf{11} &  & \foreignlanguage{greek}{τοτε αφιηϲιν αυτον ο διαβολοϲ και} & 6 &  &  \\
&  & 7 & \foreignlanguage{greek}{ιδου αγγελοι προϲηλθον και διηκονου̅} & 11 &  &  \\
&  & 12 & \foreignlanguage{greek}{αυτω} & 12 &  &  \\
& \textbf{12} &  & \foreignlanguage{greek}{ακουϲαϲ δε ο \textoverline{ιϲ} οτι ιωαννηϲ παρεδοθη} & 7 &  &  \\
&  & 8 & \foreignlanguage{greek}{ανεχωρηϲεν ειϲ την γαλιλαιαν} & 11 &  &  \\
& \textbf{13} &  & \foreignlanguage{greek}{και καταλιπων την ναζαρεθ ελθων} & 5 &  &  \\
&  & 6 & \foreignlanguage{greek}{κατωκηϲεν ειϲ καπερναουμ την} & 10 &  &  \\
[0.2em]
\cline{4-4}
\end{tabular}
\end{center}
\end{table}
}
\clearpage
\newpage
 {
 \setlength\arrayrulewidth{1pt}
\begin{table}
\begin{center}
\begin{tabular}{ccc|l|ccc}
\cline{4-4} \\ [-1em]
\multicolumn{7}{c}{\foreignlanguage{greek}{ευαγγελιον κατα μαθθαιον} \textbf{(\nospace{4:13})} } \\ \\ [-1em] % Si on veut ajouter les bordures latérales, remplacer {7}{c} par {7}{|c|}
\cline{4-4} \\
\cline{4-4}
&  &  & &  &  & \\ [-0.9em]
&  & 11 & \foreignlanguage{greek}{παραθαλαϲϲαν εν οριοιϲ ζαβουλων και} & 15 &  &  \\
&  & 16 & \foreignlanguage{greek}{νεφθαλιμ ινα πληρωθη το ρηθεν δια του} & 6 & \textbf{14} &  \\
&  & 7 & \foreignlanguage{greek}{ηϲαιου του προφητου λεγοντοϲ} & 10 &  &  \\
& \textbf{15} &  & \foreignlanguage{greek}{γη ζαβουλων και νεφθαλιμ οδον} & 5 &  &  \\
&  & 6 & \foreignlanguage{greek}{θαλαϲϲηϲ περαν του ιορδανου γαλι} & 10 &  &  \\
&  & 10 & \foreignlanguage{greek}{λαια των εθνων ο λαοϲ ο καθημε} & 4 & \textbf{16} &  \\
&  & 4 & \foreignlanguage{greek}{νοϲ εν τη ϲκοτια φωϲ ειδεν μεγα} & 10 &  &  \\
&  & 11 & \foreignlanguage{greek}{και τοιϲ καθημενοιϲ εν χωρα και ϲκι} & 17 &  &  \\
&  & 17 & \foreignlanguage{greek}{α θανατου φωϲ ανετιλεν αυτοιϲ} & 21 &  &  \\
& \textbf{17} &  & \foreignlanguage{greek}{απο τοτε ηρξατο ο \textoverline{ιϲ} κηρυϲϲιν και λε} & 8 &  &  \\
&  & 8 & \foreignlanguage{greek}{γειν μετανοειτε ηγγεικεν γαρ η} & 12 &  &  \\
&  & 13 & \foreignlanguage{greek}{βαϲιλεια των ουρανων} & 15 &  &  \\
& \textbf{18} &  & \foreignlanguage{greek}{περιπατων δε παρα την θαλαϲϲαν} & 5 &  &  \\
&  & 6 & \foreignlanguage{greek}{τηϲ γαλιλαιαϲ ειδεν δυο αδελφουϲ} & 10 &  &  \\
&  & 11 & \foreignlanguage{greek}{ϲιμωνα τον λεγομενον πετρον} & 14 &  &  \\
&  & 15 & \foreignlanguage{greek}{και ανδρεαν τον αδελφον αυτου} & 19 &  &  \\
&  & 20 & \foreignlanguage{greek}{βαλλονταϲ αμφιβληϲτρον ειϲ τη̅} & 23 &  &  \\
&  & 24 & \foreignlanguage{greek}{θαλαϲϲαν ηϲαν γαρ αλιειϲ και λεγει} & 2 & \textbf{19} &  \\
&  & 3 & \foreignlanguage{greek}{αυτοιϲ δευτε οπιϲω μου και ποιη} & 8 &  &  \\
&  & 8 & \foreignlanguage{greek}{ϲω υμαϲ αλιειϲ \textoverline{ανων} ευ} & 3 & \textbf{22} &  \\
&  & 3 & \foreignlanguage{greek}{θεωϲ αφεντεϲ τα δικτυα αυτων} & 7 &  &  \\
&  & 8 & \foreignlanguage{greek}{ηκολουθηϲαν αυτω} & 9 &  &  \\
& \textbf{23} &  & \foreignlanguage{greek}{και περιηγεν ολην την γαλιλαιαν ο} & 6 &  &  \\
&  & 7 & \foreignlanguage{greek}{\textoverline{ιϲ} διδαϲκων εν ταιϲ ϲυναγωγαιϲ αυ} & 12 &  &  \\
&  & 12 & \foreignlanguage{greek}{των και κηρυϲϲων το ευαγγελιον} & 16 &  &  \\
&  & 17 & \foreignlanguage{greek}{τηϲ βαϲιλειαϲ και θεραπευων παϲαν} & 21 &  &  \\
&  & 22 & \foreignlanguage{greek}{νοϲον και παϲαν μαλακιαν εν τω λαω} & 28 &  &  \\
& \textbf{24} &  & \foreignlanguage{greek}{και απηλθεν η ακοη αυτου ειϲ ολη̅} & 7 &  &  \\
&  & 8 & \foreignlanguage{greek}{την ϲυριαν και προϲηνεγκαν αυ} & 12 &  &  \\
&  & 12 & \foreignlanguage{greek}{τω πανταϲ τουϲ κακωϲ εχονταϲ ποι} & 17 &  &  \\
[0.2em]
\cline{4-4}
\end{tabular}
\end{center}
\end{table}
}
\clearpage
\newpage
 {
 \setlength\arrayrulewidth{1pt}
\begin{table}
\begin{center}
\begin{tabular}{ccc|l|ccc}
\cline{4-4} \\ [-1em]
\multicolumn{7}{c}{\foreignlanguage{greek}{ευαγγελιον κατα μαθθαιον} \textbf{(\nospace{4:24})} } \\ \\ [-1em] % Si on veut ajouter les bordures latérales, remplacer {7}{c} par {7}{|c|}
\cline{4-4} \\
\cline{4-4}
&  &  & &  &  & \\ [-0.9em]
&  & 17 & \foreignlanguage{greek}{κειλαιϲ νοϲοιϲ και βαϲανοιϲ ϲυνεχο} & 21 &  &  \\
&  & 21 & \foreignlanguage{greek}{μενουϲ και δαιμονιζομενουϲ και} & 24 &  &  \\
&  & 25 & \foreignlanguage{greek}{ϲεληνιαζομενουϲ και παραλυτικουϲ} & 27 &  &  \\
&  & 28 & \foreignlanguage{greek}{και εθεραπευϲεν αυτουϲ} & 30 &  &  \\
& \textbf{25} &  & \foreignlanguage{greek}{και ηκολουθηϲαν αυτω οχλοι πολλοι} & 5 &  &  \\
&  & 6 & \foreignlanguage{greek}{απο τηϲ γαλιλαιαϲ και δεκαπολεωϲ} & 10 &  &  \\
&  & 11 & \foreignlanguage{greek}{και ιεροϲολυμων και ιουδαιαϲ και} & 15 &  &  \\
&  & 16 & \foreignlanguage{greek}{περαν του ιορδανου} & 18 &  &  \\
& \mygospelchapter &  & \foreignlanguage{greek}{ιδων δε τουϲ οχλουϲ ανεβη ειϲ το οροϲ} & 8 &  &  \\
&  & 9 & \foreignlanguage{greek}{και καθειϲαντοϲ αυτου προϲηλθον} & 12 &  &  \\
&  & 13 & \foreignlanguage{greek}{αυτω οι μαθηται αυτου και ανοιξαϲ} & 2 & \textbf{2} &  \\
&  & 3 & \foreignlanguage{greek}{το ϲτομα αυτου εδιδαϲκεν αυτουϲ λεγω̅} & 8 &  &  \\
& \textbf{3} &  & \foreignlanguage{greek}{μακαριοι οι πτωχοι τω \textoverline{πνι} οτι αυτω̅} & 7 &  &  \\
&  & 8 & \foreignlanguage{greek}{εϲτιν η βαϲιλεια των ουρανων} & 12 &  &  \\
& \textbf{4} &  & \foreignlanguage{greek}{μακαριοι οι πενθουντεϲ οτι αυτοι} & 5 &  &  \\
&  & 6 & \foreignlanguage{greek}{παρακληθηϲονται μακαριοι} & 1 & \textbf{5} &  \\
&  & 2 & \foreignlanguage{greek}{οι πραειϲ οτι αυτοι κληρονομηϲουϲι̅} & 6 &  &  \\
&  & 7 & \foreignlanguage{greek}{την γην μακαριοι οι πινωντεϲ} & 3 & \textbf{6} &  \\
&  & 4 & \foreignlanguage{greek}{και διψωντεϲ την δικαιωϲυνην} & 7 &  &  \\
&  & 9 & \foreignlanguage{greek}{οτι αυτοι χορταϲθηϲονται} & 11 &  &  \\
& \textbf{7} &  & \foreignlanguage{greek}{μακαριοι οι ελεημονεϲ οτι αυτοι ελε} & 6 &  &  \\
&  & 6 & \foreignlanguage{greek}{ηθηϲονται μακαριοι οι καθα} & 3 & \textbf{8} &  \\
&  & 3 & \foreignlanguage{greek}{ροι τη καρδια οτι αυτοι τον \textoverline{θν} οψονται} & 10 &  &  \\
& \textbf{9} &  & \foreignlanguage{greek}{μακαριοι οι ειρηνοποιοι οτι αυτοι υιοι} & 6 &  &  \\
&  & 7 & \foreignlanguage{greek}{\textoverline{θυ} κληθηϲονται μακαριοι οι δε} & 3 & \textbf{10} &  \\
&  & 3 & \foreignlanguage{greek}{διωγμενοι ενεκεν δικαιωϲυνηϲ οτι} & 6 &  &  \\
&  & 7 & \foreignlanguage{greek}{αυτων εϲτιν η βαϲιλεια των ουρανων} & 12 &  &  \\
& \textbf{11} &  & \foreignlanguage{greek}{μακαριοι εϲται οταν ονιδιϲωϲιν υμαϲ} & 5 &  &  \\
&  & 6 & \foreignlanguage{greek}{και διωξουϲιν και ειπωϲιν παν πο} & 11 &  &  \\
&  & 11 & \foreignlanguage{greek}{νηρον ρημα καθ υμων ψευδομενοι} & 15 &  &  \\
[0.2em]
\cline{4-4}
\end{tabular}
\end{center}
\end{table}
}
\clearpage
\newpage
 {
 \setlength\arrayrulewidth{1pt}
\begin{table}
\begin{center}
\begin{tabular}{ccc|l|ccc}
\cline{4-4} \\ [-1em]
\multicolumn{7}{c}{\foreignlanguage{greek}{ευαγγελιον κατα μαθθαιον} \textbf{(\nospace{5:11})} } \\ \\ [-1em] % Si on veut ajouter les bordures latérales, remplacer {7}{c} par {7}{|c|}
\cline{4-4} \\
\cline{4-4}
&  &  & &  &  & \\ [-0.9em]
&  & 16 & \foreignlanguage{greek}{ενεκεν εμου χαιρεται και αγαλλιαϲθαι} & 3 & \textbf{12} &  \\
&  & 4 & \foreignlanguage{greek}{οτι ο μιϲθοϲ υμων πολυϲ εν τοιϲ ουρανοιϲ} & 11 &  &  \\
&  & 12 & \foreignlanguage{greek}{ουτωϲ γαρ εδιωξαν τουϲ προφηταϲ} & 16 &  &  \\
&  & 17 & \foreignlanguage{greek}{τουϲ προ υμων} & 19 &  &  \\
& \textbf{13} &  & \foreignlanguage{greek}{υμειϲ εϲται το αλα τηϲ γηϲ εαν δε το} & 9 &  &  \\
&  & 10 & \foreignlanguage{greek}{αλα μωρανθη εν τινι αλιϲθηϲεται} & 14 &  &  \\
&  & 15 & \foreignlanguage{greek}{ειϲ ουδεν ιϲχυει ει μη βληθηναι εξω} & 21 &  &  \\
&  & 22 & \foreignlanguage{greek}{και καταπατιϲθαι υπο των \textoverline{ανων}} & 26 &  &  \\
& \textbf{14} &  & \foreignlanguage{greek}{υμειϲ εϲται το φωϲ του κοϲμου ου δυ} & 8 &  &  \\
&  & 8 & \foreignlanguage{greek}{ναται πολιϲ κρυβηναι επανω ορουϲ} & 12 &  &  \\
&  & 13 & \foreignlanguage{greek}{κειμενη ουδε καιουϲιν λυχνον και} & 4 & \textbf{15} &  \\
&  & 5 & \foreignlanguage{greek}{τιθεαϲιν αυτον υπο τον μοδιον αλλ} & 10 &  &  \\
&  & 11 & \foreignlanguage{greek}{επι την λυχνιαν και λαμπει παϲιν} & 16 &  &  \\
&  & 17 & \foreignlanguage{greek}{τοιϲ εν τη οικεια ουτωϲ λαμψατω} & 2 & \textbf{16} &  \\
&  & 3 & \foreignlanguage{greek}{το φωϲ υμων εμπροϲθεν των \textoverline{ανων}} & 8 &  &  \\
&  & 9 & \foreignlanguage{greek}{οπωϲ ιδωϲιν υμων τα καλα εργα και} & 15 &  &  \\
&  & 16 & \foreignlanguage{greek}{δοξαϲωϲιν τον \textoverline{πρα} υμων τον εν τοιϲ} & 22 &  &  \\
&  & 23 & \foreignlanguage{greek}{ουρανοιϲ μη νομιϲηται οτι ηλ} & 4 & \textbf{17} &  \\
&  & 4 & \foreignlanguage{greek}{θον καταλυϲαι τον νομον η τουϲ} & 9 &  &  \\
&  & 10 & \foreignlanguage{greek}{προφηταϲ ουκ ηλθον καταλυϲαι} & 13 &  &  \\
&  & 14 & \foreignlanguage{greek}{αλλα πληρωϲαι} & 15 &  &  \\
& \textbf{18} &  & \foreignlanguage{greek}{αμην γαρ λεγω υμιν εωϲ αν παρελθη} & 7 &  &  \\
&  & 8 & \foreignlanguage{greek}{ο ουρανοϲ και η γη ιωτα εν η μια κεραια} & 17 &  &  \\
&  & 18 & \foreignlanguage{greek}{ου μη παρελθη απο του νομου εωϲ α̅} & 25 &  &  \\
&  & 26 & \foreignlanguage{greek}{παντα γενηται} & 27 &  &  \\
& \textbf{19} &  & \foreignlanguage{greek}{οϲ εαν ουν λυϲη μιαν των εντολων} & 7 &  &  \\
&  & 8 & \foreignlanguage{greek}{τουτων των ελαχιϲτων και διδα} & 12 &  &  \\
&  & 12 & \foreignlanguage{greek}{ξη ουτωϲ τουϲ ανθρωπουϲ ελαχιϲτοϲ} & 16 &  &  \\
&  & 17 & \foreignlanguage{greek}{κληθηϲεται εν τη βαϲιλεια των ουρανων} & 22 &  &  \\
& \textbf{20} &  & \foreignlanguage{greek}{λεγω γαρ υμιν οτι εαν μη περιϲϲευϲη} & 7 &  &  \\
[0.2em]
\cline{4-4}
\end{tabular}
\end{center}
\end{table}
}
\clearpage
\newpage
 {
 \setlength\arrayrulewidth{1pt}
\begin{table}
\begin{center}
\begin{tabular}{ccc|l|ccc}
\cline{4-4} \\ [-1em]
\multicolumn{7}{c}{\foreignlanguage{greek}{ευαγγελιον κατα μαθθαιον} \textbf{(\nospace{5:20})} } \\ \\ [-1em] % Si on veut ajouter les bordures latérales, remplacer {7}{c} par {7}{|c|}
\cline{4-4} \\
\cline{4-4}
&  &  & &  &  & \\ [-0.9em]
&  & 8 & \foreignlanguage{greek}{υμων η δικαιοϲυνη πλεον των γραμ} & 13 &  &  \\
&  & 13 & \foreignlanguage{greek}{ματεων και φαριϲαιων ου μη ειϲελθη} & 18 &  &  \\
&  & 18 & \foreignlanguage{greek}{ται ειϲ την βαϲιλειαν των ουρανων} & 23 &  &  \\
& \textbf{21} &  & \foreignlanguage{greek}{ηκουϲατε οτι ερρεθη τοιϲ αρχαιοιϲ ου} & 6 &  &  \\
&  & 7 & \foreignlanguage{greek}{φονευϲηϲ οϲ δ αν φονευϲη ενοχοϲ ε} & 13 &  &  \\
&  & 13 & \foreignlanguage{greek}{ϲται τη κριϲει} & 15 &  &  \\
& \textbf{22} &  & \foreignlanguage{greek}{εγω δε λεγω υμιν οτι παϲ ο οργιζομε} & 8 &  &  \\
&  & 8 & \foreignlanguage{greek}{νοϲ τω αδελφω αυτου εικη ενοχοϲ ε} & 14 &  &  \\
&  & 14 & \foreignlanguage{greek}{ϲται τη κριϲει οϲ δ αν ειπη τω αδελ} & 22 &  &  \\
&  & 22 & \foreignlanguage{greek}{φω αυτου ραχα ενοχοϲ εϲται τω ϲυν} & 28 &  &  \\
&  & 28 & \foreignlanguage{greek}{εδριω οϲ δ α ειπη μωρε ενοχοϲ εϲται} & 35 &  &  \\
&  & 36 & \foreignlanguage{greek}{ειϲ την γεενναν του πυροϲ} & 40 &  &  \\
& \textbf{23} &  & \foreignlanguage{greek}{εαν ουν προϲφερηϲ το δωρον ϲου επι} & 7 &  &  \\
&  & 8 & \foreignlanguage{greek}{το θυϲιαϲτηριον κακει μνηϲθηϲ οτι} & 12 &  &  \\
&  & 13 & \foreignlanguage{greek}{ο αδελφοϲ ϲου εχει τι κατα ϲου αφεϲ ε} & 2 & \textbf{24} &  \\
&  & 2 & \foreignlanguage{greek}{κει το δωρον ϲου εμπροϲθεν του θυϲι} & 8 &  &  \\
&  & 8 & \foreignlanguage{greek}{αϲτηριου και υπαγε πρωτον διαλλα} & 12 &  &  \\
&  & 12 & \foreignlanguage{greek}{γηθει τω αδελφω ϲου και τοτε ελθω̅} & 18 &  &  \\
&  & 19 & \foreignlanguage{greek}{προϲφερε το δωρον ϲου ιϲθι ευνοω̅} & 2 & \textbf{25} &  \\
&  & 3 & \foreignlanguage{greek}{τω αντιδικω ϲου ταχυ εωϲ οτου ει} & 9 &  &  \\
&  & 10 & \foreignlanguage{greek}{μετ αυτου εν τη οδω μηποτε ϲε πα} & 17 &  &  \\
&  & 17 & \foreignlanguage{greek}{ραδω ο αντιδικοϲ τω κριτη και ο κρι} & 24 &  &  \\
&  & 24 & \foreignlanguage{greek}{τηϲ ϲε παραδω τω υπηρετη και ειϲ} & 30 &  &  \\
&  & 31 & \foreignlanguage{greek}{φυλακην βληθηϲη αμην λεγω ϲοι} & 3 & \textbf{26} &  \\
&  & 4 & \foreignlanguage{greek}{ου μη εξελθηϲ εκειθεν εωϲ ου απο} & 10 &  &  \\
&  & 10 & \foreignlanguage{greek}{δωϲ τον εϲχατον κοδραντην} & 13 &  &  \\
& \textbf{27} &  & \foreignlanguage{greek}{ηκουϲατε οτι ερρεθη ου μοιχευϲειϲ} & 5 &  &  \\
& \textbf{28} &  & \foreignlanguage{greek}{εγω δε λεγω υμιν οτι παϲ ο βλεπων γυ} & 9 &  &  \\
&  & 9 & \foreignlanguage{greek}{ναικα προϲ το επιθυμηϲαι αυτην η} & 14 &  &  \\
&  & 14 & \foreignlanguage{greek}{δη εμοιχευϲεν αυτην εν τη καρδια αυτου} & 20 &  &  \\
[0.2em]
\cline{4-4}
\end{tabular}
\end{center}
\end{table}
}
\clearpage
\newpage
 {
 \setlength\arrayrulewidth{1pt}
\begin{table}
\begin{center}
\begin{tabular}{ccc|l|ccc}
\cline{4-4} \\ [-1em]
\multicolumn{7}{c}{\foreignlanguage{greek}{ευαγγελιον κατα μαθθαιον} \textbf{(\nospace{5:29})} } \\ \\ [-1em] % Si on veut ajouter les bordures latérales, remplacer {7}{c} par {7}{|c|}
\cline{4-4} \\
\cline{4-4}
&  &  & &  &  & \\ [-0.9em]
& \textbf{29} &  & \foreignlanguage{greek}{ει δε ο οφθαλμοϲ ϲου ο δεξιοϲ ϲκανδα} & 8 &  &  \\
&  & 8 & \foreignlanguage{greek}{λιζει ϲε εξελε αυτον και βαλε απο ϲου} & 15 &  &  \\
&  & 16 & \foreignlanguage{greek}{ϲυμφερει γαρ ϲοι ινα αποληται εν τω̅} & 22 &  &  \\
&  & 23 & \foreignlanguage{greek}{μελων ϲου και μη ολον το ϲωμα ϲου} & 30 &  &  \\
&  & 31 & \foreignlanguage{greek}{βληθη ειϲ την γεενναν} & 34 &  &  \\
& \textbf{30} &  & \foreignlanguage{greek}{και ει η δεξια ϲου χειρ ϲκανδαλιζει ϲε} & 8 &  &  \\
&  & 9 & \foreignlanguage{greek}{κοψον αυτην και βαλε απο ϲου} & 14 &  &  \\
&  & 15 & \foreignlanguage{greek}{ϲυμφερει γαρ ϲοι ινα αποληται εν τω̅} & 21 &  &  \\
&  & 22 & \foreignlanguage{greek}{μελων ϲου και μη ολον το ϲωμα ϲου} & 29 &  &  \\
&  & 30 & \foreignlanguage{greek}{βληθη ειϲ γεενναν} & 32 &  &  \\
& \textbf{31} &  & \foreignlanguage{greek}{ερρεθη δε οτι οϲ εαν απολυϲη την γυ} & 8 &  &  \\
&  & 8 & \foreignlanguage{greek}{ναικα αυτου δοτω αυτη αποϲταϲιον} & 12 &  &  \\
& \textbf{32} &  & \foreignlanguage{greek}{εγω δε λεγω υμιν οτι παϲ ο απολυων} & 8 &  &  \\
&  & 9 & \foreignlanguage{greek}{την γυναικα αυτου παρεκτοϲ λογου} & 13 &  &  \\
&  & 14 & \foreignlanguage{greek}{πορνιαϲ ποιει αυτην μοιχευθηναι} & 17 &  &  \\
&  & 18 & \foreignlanguage{greek}{και οϲ εαν απολελυμενην γαμηϲη μοι} & 23 &  &  \\
&  & 23 & \foreignlanguage{greek}{χατε παλιν ηκουϲατε οτι ερρε} & 4 & \textbf{33} &  \\
&  & 4 & \foreignlanguage{greek}{θη τοιϲ αρχαιοιϲ ουκ επιορκηϲειϲ απο} & 9 &  &  \\
&  & 9 & \foreignlanguage{greek}{δωϲηϲ τω \textoverline{κω} τουϲ ορκουϲ ϲου} & 14 &  &  \\
& \textbf{34} &  & \foreignlanguage{greek}{εγω δε λεγω υμιν μη ομοϲαι ολωϲ μη} & 8 &  &  \\
&  & 8 & \foreignlanguage{greek}{τε εν τω ουρανω οτι θρονοϲ εϲτιν} & 14 &  &  \\
&  & 15 & \foreignlanguage{greek}{του \textoverline{θυ} μητε εν τη γη οτι υποποδιον} & 6 & \textbf{35} &  \\
&  & 7 & \foreignlanguage{greek}{εϲτιν των ποδων αυτου μητε ειϲ} & 12 &  &  \\
&  & 13 & \foreignlanguage{greek}{ιεροϲολυμα οτι πολιϲ εϲτιν του μεγα} & 18 &  &  \\
&  & 18 & \foreignlanguage{greek}{λου βαϲιλεωϲ μητε εν τη κεφαλη} & 4 & \textbf{36} &  \\
&  & 5 & \foreignlanguage{greek}{ϲου ομοϲηϲ οτι ου δυναϲαι μιαν τρι} & 11 &  &  \\
&  & 11 & \foreignlanguage{greek}{χαν λευκην ποιηϲαι η μελαναν} & 15 &  &  \\
& \textbf{37} &  & \foreignlanguage{greek}{εϲτω δε ο λογοϲ υμων ναι ναι ου ου} & 9 &  &  \\
&  & 10 & \foreignlanguage{greek}{το δε περιϲϲον τουτων εκ του πο} & 16 &  &  \\
&  & 16 & \foreignlanguage{greek}{νηρου εϲτιν} & 17 &  &  \\
[0.2em]
\cline{4-4}
\end{tabular}
\end{center}
\end{table}
}
\clearpage
\newpage
 {
 \setlength\arrayrulewidth{1pt}
\begin{table}
\begin{center}
\begin{tabular}{ccc|l|ccc}
\cline{4-4} \\ [-1em]
\multicolumn{7}{c}{\foreignlanguage{greek}{ευαγγελιον κατα μαθθαιον} \textbf{(\nospace{5:38})} } \\ \\ [-1em] % Si on veut ajouter les bordures latérales, remplacer {7}{c} par {7}{|c|}
\cline{4-4} \\
\cline{4-4}
&  &  & &  &  & \\ [-0.9em]
& \textbf{38} &  & \foreignlanguage{greek}{ηκουϲατε οτι ερρεθη οφθαλμον αντι του} & 6 &  &  \\
&  & 7 & \foreignlanguage{greek}{οφθαλμου και οδοντα αντι οδοντοϲ} & 11 &  &  \\
& \textbf{39} &  & \foreignlanguage{greek}{εγω δε λεγω υμιν μη αντιϲτηναι τω πο} & 8 &  &  \\
&  & 8 & \foreignlanguage{greek}{νηρω αλλ οϲτιϲ ϲε ραπιζει ειϲ την δε} & 15 &  &  \\
&  & 15 & \foreignlanguage{greek}{ξιαν ϲιαγονα ϲτρεψον αυτω ϗ την αλλη̅} & 21 &  &  \\
& \textbf{40} &  & \foreignlanguage{greek}{και τω θελοντι ϲοι κριθηναι και τον χει} & 8 &  &  \\
&  & 8 & \foreignlanguage{greek}{τωνα ϲου λαβειν αφεϲ αυτω και το ι} & 15 &  &  \\
&  & 15 & \foreignlanguage{greek}{ματιον και οϲτιϲ ϲε ανγαρευϲη μιλιο̅} & 5 & \textbf{41} &  \\
&  & 6 & \foreignlanguage{greek}{εν υπαγε μετ αυτου δυο τω αιτουν} & 2 & \textbf{42} &  \\
&  & 2 & \foreignlanguage{greek}{τι ϲε δοϲ και τον θελοντα απο ϲου δα} & 10 &  &  \\
&  & 10 & \foreignlanguage{greek}{νιϲαϲθαι μη αποϲτραφηϲ} & 12 &  &  \\
& \textbf{43} &  & \foreignlanguage{greek}{ηκουϲατε οτι ερρεθη αγαπηϲιϲ τον} & 5 &  &  \\
&  & 6 & \foreignlanguage{greek}{πληϲιον ϲου και μιϲηϲηϲ τον εχθρον ϲου} & 12 &  &  \\
& \textbf{44} &  & \foreignlanguage{greek}{εγω δε λεγω υμιν αγαπατε του εχθρουϲ} & 7 &  &  \\
&  & 8 & \foreignlanguage{greek}{υμων ευλογειται τουϲ καταρωμε} & 11 &  &  \\
&  & 11 & \foreignlanguage{greek}{νουϲ υμαϲ καλωϲ ποιειται τοιϲ μι} & 16 &  &  \\
&  & 16 & \foreignlanguage{greek}{ϲουϲιν υμαϲ και προϲευχεϲθαι υπερ} & 20 &  &  \\
&  & 21 & \foreignlanguage{greek}{των επηρεαζοντων υμαϲ και διω} & 25 &  &  \\
&  & 25 & \foreignlanguage{greek}{κοντων υμαϲ οπωϲ γενηϲθαι υιοι} & 3 & \textbf{45} &  \\
&  & 4 & \foreignlanguage{greek}{του \textoverline{πρϲ} υμων του εν ουρανοιϲ οτι το̅} & 11 &  &  \\
&  & 12 & \foreignlanguage{greek}{ηλιον αυτου ανατελλει επι πονηρουϲ} & 16 &  &  \\
&  & 17 & \foreignlanguage{greek}{και αγαθουϲ και βρεχει επι δικαιουϲ} & 22 &  &  \\
&  & 23 & \foreignlanguage{greek}{και αδικουϲ} & 24 &  &  \\
& \textbf{46} &  & \foreignlanguage{greek}{εαν γαρ αγαπηϲηται τουϲ αγαπωνταϲ} & 5 &  &  \\
&  & 6 & \foreignlanguage{greek}{υμαϲ τινα μιϲθον εχεται ουχι και} & 11 &  &  \\
&  & 12 & \foreignlanguage{greek}{οι τελωναι το αυτο ποιουϲιν} & 16 &  &  \\
& \textbf{47} &  & \foreignlanguage{greek}{και εαν αϲπαϲηϲθαι τουϲ φιλουϲ υμω̅} & 6 &  &  \\
&  & 7 & \foreignlanguage{greek}{μονον τι περιϲϲον ποιειται ουχι ϗ} & 12 &  &  \\
&  & 13 & \foreignlanguage{greek}{οι τελωναι το αυτο ποιουϲιν} & 17 &  &  \\
& \textbf{48} &  & \foreignlanguage{greek}{εϲεϲθαι ουν υμειϲ τελιοι ωϲπερ ο \textoverline{πηρ}} & 7 &  &  \\
[0.2em]
\cline{4-4}
\end{tabular}
\end{center}
\end{table}
}
\clearpage
\newpage
 {
 \setlength\arrayrulewidth{1pt}
\begin{table}
\begin{center}
\begin{tabular}{ccc|l|ccc}
\cline{4-4} \\ [-1em]
\multicolumn{7}{c}{\foreignlanguage{greek}{ευαγγελιον κατα μαθθαιον} \textbf{(\nospace{5:48})} } \\ \\ [-1em] % Si on veut ajouter les bordures latérales, remplacer {7}{c} par {7}{|c|}
\cline{4-4} \\
\cline{4-4}
&  &  & &  &  & \\ [-0.9em]
&  & 8 & \foreignlanguage{greek}{υμων ο ουρανιοιϲ τελιοϲ εϲτιν} & 12 &  &  \\
& \mygospelchapter &  & \foreignlanguage{greek}{προϲεχετε την ελεημοϲυνην υμων} & 4 &  &  \\
&  & 5 & \foreignlanguage{greek}{μη ποιειν εμπροϲθεν των ανθρωπω̅} & 9 &  &  \\
&  & 10 & \foreignlanguage{greek}{προϲ το θεαθηναι αυτοιϲ ει δε μη γε} & 17 &  &  \\
&  & 18 & \foreignlanguage{greek}{μιϲθον ουκ εχεται παρα τω \textoverline{πρι} υμων} & 24 &  &  \\
&  & 25 & \foreignlanguage{greek}{τω εν τοιϲ ουρανοιϲ} & 28 &  &  \\
& \textbf{2} &  & \foreignlanguage{greek}{οταν ουν ποιηϲ ελεημοϲυνην μη} & 5 &  &  \\
&  & 6 & \foreignlanguage{greek}{ϲαλπιϲηϲ εμπροϲθεν ϲου ωϲπερ οι υ} & 11 &  &  \\
&  & 11 & \foreignlanguage{greek}{ποκριτε ποιουϲιν εν ταιϲ ϲυναγωγαιϲ} & 15 &  &  \\
&  & 16 & \foreignlanguage{greek}{και εν ταιϲ ρυμαιϲ οπωϲ δοξαϲθωϲι̅} & 21 &  &  \\
&  & 22 & \foreignlanguage{greek}{υπο των ανθρωπων αμην λεγω υμι̅} & 27 &  &  \\
&  & 28 & \foreignlanguage{greek}{απεχουϲιν τον μιϲθον αυτων} & 31 &  &  \\
& \textbf{3} &  & \foreignlanguage{greek}{ϲου δε ποιουντοϲ ελεημοϲυνην μη} & 5 &  &  \\
&  & 6 & \foreignlanguage{greek}{γνωτω η αριϲτερα ϲου τι ποιει η δε} & 13 &  &  \\
&  & 13 & \foreignlanguage{greek}{ξια ϲου οπωϲ η ϲου η ελεημοϲυνη} & 5 & \textbf{4} &  \\
&  & 6 & \foreignlanguage{greek}{εν τω κρυπτω και ο \textoverline{πηρ} ϲου ο βλεπω̅} & 14 &  &  \\
&  & 15 & \foreignlanguage{greek}{εν τω κρυπτω αυτοϲ αποδωϲι ϲοι ε̅} & 21 &  &  \\
&  & 22 & \foreignlanguage{greek}{τω φανερω και οταν προϲευχη ου} & 4 & \textbf{5} &  \\
&  & 4 & \foreignlanguage{greek}{κ εϲη ωϲπερ οι υποκριται οτι φιλου} & 10 &  &  \\
&  & 10 & \foreignlanguage{greek}{ϲιν εν ταιϲ ϲυναγωγαιϲ και εν ταιϲ} & 16 &  &  \\
&  & 17 & \foreignlanguage{greek}{γωνιαιϲ των πλατιων εϲτωτεϲ} & 20 &  &  \\
&  & 21 & \foreignlanguage{greek}{προϲευχεϲθαι οπωϲ αν φανωϲιν} & 24 &  &  \\
&  & 25 & \foreignlanguage{greek}{τοιϲ ανθρωποιϲ αμην λεγω υμιν} & 29 &  &  \\
&  & 30 & \foreignlanguage{greek}{οτι απεχουϲιν τον μιϲθον αυτων} & 34 &  &  \\
& \textbf{6} &  & \foreignlanguage{greek}{ϲυ δε οταν προϲευχη ειϲελθε ειϲ το} & 7 &  &  \\
&  & 8 & \foreignlanguage{greek}{ταμιον ϲου και κλιϲαϲ την θυραν} & 13 &  &  \\
&  & 14 & \foreignlanguage{greek}{ϲου προϲευξε τω \textoverline{πρι} ϲου τω εν τω} & 21 &  &  \\
&  & 22 & \foreignlanguage{greek}{κρυπτω και ο \textoverline{πηρ} ϲου ο βλεπων ε̅} & 29 &  &  \\
&  & 30 & \foreignlanguage{greek}{τω κρυπτω αποδωϲη ϲοι εν τω} & 35 &  &  \\
&  & 36 & \foreignlanguage{greek}{φανερω προϲευχομενοι δε} & 2 & \textbf{7} &  \\
[0.2em]
\cline{4-4}
\end{tabular}
\end{center}
\end{table}
}
\clearpage
\newpage
 {
 \setlength\arrayrulewidth{1pt}
\begin{table}
\begin{center}
\begin{tabular}{ccc|l|ccc}
\cline{4-4} \\ [-1em]
\multicolumn{7}{c}{\foreignlanguage{greek}{ευαγγελιον κατα μαθθαιον} \textbf{(\nospace{6:7})} } \\ \\ [-1em] % Si on veut ajouter les bordures latérales, remplacer {7}{c} par {7}{|c|}
\cline{4-4} \\
\cline{4-4}
&  &  & &  &  & \\ [-0.9em]
&  & 3 & \foreignlanguage{greek}{μη βατταλογειται ωϲπερ οι εθνικοι} & 7 &  &  \\
&  & 8 & \foreignlanguage{greek}{δοκουϲιν γαρ οτι εν τη πολυλογια αυτων} & 14 &  &  \\
&  & 15 & \foreignlanguage{greek}{ειϲακουϲθηϲονται μη ουν ομοιω} & 3 & \textbf{8} &  \\
&  & 3 & \foreignlanguage{greek}{θηται αυτοιϲ οιδεν γαρ ο \textoverline{πηρ} υμων} & 9 &  &  \\
&  & 10 & \foreignlanguage{greek}{ων χρειαν εχεται προ του υμαϲ αιτη} & 16 &  &  \\
&  & 16 & \foreignlanguage{greek}{ϲαι αυτον ουτωϲ ουν προϲευχεϲθαι} & 3 & \textbf{9} &  \\
&  & 4 & \foreignlanguage{greek}{υμειϲ πατερ ημων ο εν τοιϲ ουρανοιϲ} & 10 &  &  \\
&  & 11 & \foreignlanguage{greek}{αγιαϲθητω το ονομα ϲου ελθατω η} & 2 & \textbf{10} &  \\
&  & 3 & \foreignlanguage{greek}{βαϲιλεια ϲου γενηθητω το θελημα ϲου} & 8 &  &  \\
&  & 9 & \foreignlanguage{greek}{ωϲ εν ουρανω και επι γηϲ τον αρτον} & 2 & \textbf{11} &  \\
&  & 3 & \foreignlanguage{greek}{ημων τον επιουϲιον δοϲ ημιν ϲημερο̅} & 8 &  &  \\
& \textbf{12} &  & \foreignlanguage{greek}{και αφεϲ ημιν τα οφιληματα ημων} & 6 &  &  \\
&  & 7 & \foreignlanguage{greek}{ωϲ και ημειϲ αφιομεν τοιϲ οφιλεταιϲ} & 12 &  &  \\
&  & 13 & \foreignlanguage{greek}{ημων και μη ειϲενεγκηϲ ημαϲ ειϲ} & 5 & \textbf{13} &  \\
&  & 6 & \foreignlanguage{greek}{πειραϲμον αλλα ρυϲαι ημαϲ απο του} & 11 &  &  \\
&  & 12 & \foreignlanguage{greek}{πονηρου οτι ϲου εϲτιν η βαϲιλεια} & 17 &  &  \\
&  & 18 & \foreignlanguage{greek}{και η δυναμειϲ και η δοξα ειϲ τουϲ} & 25 &  &  \\
&  & 26 & \foreignlanguage{greek}{αιωναϲ αμην} & 27 &  &  \\
& \textbf{14} &  & \foreignlanguage{greek}{εαν γαρ αφηται τοιϲ ανθρωποιϲ τα πα} & 7 &  &  \\
&  & 7 & \foreignlanguage{greek}{ραπτωματα αυτων αφηϲει και υ} & 12 &  &  \\
&  & 12 & \foreignlanguage{greek}{μιν ο \textoverline{πηρ} υμων ο ουρανιοϲ εαν δε} & 2 & \textbf{15} &  \\
&  & 3 & \foreignlanguage{greek}{μη αφηται τοιϲ ανθρωποιϲ τα παρα} & 8 &  &  \\
&  & 8 & \foreignlanguage{greek}{πτωματα αυτων ουδε ο \textoverline{πηρ} υμων} & 13 &  &  \\
&  & 14 & \foreignlanguage{greek}{αφηϲει τα παραπτωματα υμων} & 17 &  &  \\
& \textbf{16} &  & \foreignlanguage{greek}{οταν δε νηϲτευηται μη γινεϲθαι ωϲ} & 6 &  &  \\
&  & 6 & \foreignlanguage{greek}{περ οι υποκριται ϲκυθρωποι αφανι} & 10 &  &  \\
&  & 10 & \foreignlanguage{greek}{ζουϲιν γαρ τα προϲωπα αυτων οπωϲ} & 15 &  &  \\
&  & 16 & \foreignlanguage{greek}{φανωϲιν τοιϲ ανθρωποιϲ νηϲτευοντεϲ} & 19 &  &  \\
&  & 20 & \foreignlanguage{greek}{αμην λεγω υμιν οτι απεχουϲιν τον} & 25 &  &  \\
&  & 26 & \foreignlanguage{greek}{μιϲθον αυτων ϲυ δε νηϲτευων} & 3 & \textbf{17} &  \\
[0.2em]
\cline{4-4}
\end{tabular}
\end{center}
\end{table}
}
\clearpage
\newpage
 {
 \setlength\arrayrulewidth{1pt}
\begin{table}
\begin{center}
\begin{tabular}{ccc|l|ccc}
\cline{4-4} \\ [-1em]
\multicolumn{7}{c}{\foreignlanguage{greek}{ευαγγελιον κατα μαθθαιον} \textbf{(\nospace{6:17})} } \\ \\ [-1em] % Si on veut ajouter les bordures latérales, remplacer {7}{c} par {7}{|c|}
\cline{4-4} \\
\cline{4-4}
&  &  & &  &  & \\ [-0.9em]
&  & 4 & \foreignlanguage{greek}{αλιψε ϲου την κεφαλην και το προϲω} & 10 &  &  \\
&  & 10 & \foreignlanguage{greek}{πον ϲου νιψε οπωϲ μη φανηϲ τοιϲ αν} & 5 & \textbf{18} &  \\
&  & 5 & \foreignlanguage{greek}{θρωποιϲ νηϲτευων αλλα τω \textoverline{πρι} ϲου} & 10 &  &  \\
&  & 11 & \foreignlanguage{greek}{τω εν τω κρυπτω και ο \textoverline{πηρ} ϲου ο βλε} & 20 &  &  \\
&  & 20 & \foreignlanguage{greek}{πων εν τω κρυπτω αυτοϲ αποδωϲι ϲοι} & 26 &  &  \\
& \textbf{19} &  & \foreignlanguage{greek}{μη θηϲαυριζεται υμιν θηϲαυρουϲ επι} & 5 &  &  \\
&  & 6 & \foreignlanguage{greek}{τηϲ γηϲ οπου ϲηϲ και βρωϲιϲ αφανιζει} & 12 &  &  \\
&  & 13 & \foreignlanguage{greek}{και οπου κλεπται διορυϲϲουϲιν και} & 17 &  &  \\
&  & 18 & \foreignlanguage{greek}{κλεπτουϲιν θηϲαυριζεται δε υμι̅} & 3 & \textbf{20} &  \\
&  & 4 & \foreignlanguage{greek}{θηϲαυρουϲ εν ουρανω οπου ουτε} & 8 &  &  \\
&  & 10 & \foreignlanguage{greek}{ϲηϲ ουτε βρωϲιϲ αφανιζει και οπου} & 15 &  &  \\
&  & 16 & \foreignlanguage{greek}{κλεπται ου διορυϲϲουϲιν οπου γαρ} & 2 & \textbf{21} &  \\
&  & 3 & \foreignlanguage{greek}{εϲτιν ο θηϲαυροϲ υμων εκει εϲται} & 8 &  &  \\
&  & 9 & \foreignlanguage{greek}{και η καρδια υμων} & 12 &  &  \\
& \textbf{22} &  & \foreignlanguage{greek}{ο λυχνοϲ του ϲωματοϲ εϲτιν ο οφθαλ} & 7 &  &  \\
&  & 7 & \foreignlanguage{greek}{μοϲ εαν ουν η ο οφθαλμοϲ ϲου απλουϲ} & 14 &  &  \\
&  & 15 & \foreignlanguage{greek}{ολον το ϲωμα ϲου φωτινον εϲται} & 20 &  &  \\
& \textbf{23} &  & \foreignlanguage{greek}{εαν δε η ο οφθαλμοϲ ϲου πονηροϲ} & 7 &  &  \\
&  & 8 & \foreignlanguage{greek}{ολον το ϲωμα ϲου ϲκοτινον εϲται} & 13 &  &  \\
&  & 14 & \foreignlanguage{greek}{ει ουν το φωϲ το εν ϲοι εϲτιν ϲκοτοϲ} & 22 &  &  \\
&  & 23 & \foreignlanguage{greek}{το ϲκοτοϲ ποϲον} & 25 &  &  \\
& \textbf{24} &  & \foreignlanguage{greek}{ουδειϲ δυναται δυϲιν κυριοιϲ δου} & 5 &  &  \\
&  & 5 & \foreignlanguage{greek}{λευειν η γαρ τον ενα μειϲηϲει και} & 11 &  &  \\
&  & 12 & \foreignlanguage{greek}{τον ετερον αγαπηϲει η ενοϲ αν} & 17 &  &  \\
&  & 17 & \foreignlanguage{greek}{θεξεται και του ετερου καταφρο} & 21 &  &  \\
&  & 21 & \foreignlanguage{greek}{νηϲει ου δυναϲθαι \textoverline{θω} δου} & 25 &  &  \\
&  & 25 & \foreignlanguage{greek}{λευειν και μαμωνα δια τουτο} & 2 & \textbf{25} &  \\
&  & 3 & \foreignlanguage{greek}{λεγω υμιν μη μεριμναται τη} & 7 &  &  \\
&  & 8 & \foreignlanguage{greek}{ψυχη υμων τι φαγηται η τι πιηται} & 14 &  &  \\
&  & 15 & \foreignlanguage{greek}{μηδε τω ϲωματι υμων τι ενδυ} & 20 &  &  \\
[0.2em]
\cline{4-4}
\end{tabular}
\end{center}
\end{table}
}
\clearpage
\newpage
 {
 \setlength\arrayrulewidth{1pt}
\begin{table}
\begin{center}
\begin{tabular}{ccc|l|ccc}
\cline{4-4} \\ [-1em]
\multicolumn{7}{c}{\foreignlanguage{greek}{ευαγγελιον κατα μαθθαιον} \textbf{(\nospace{6:25})} } \\ \\ [-1em] % Si on veut ajouter les bordures latérales, remplacer {7}{c} par {7}{|c|}
\cline{4-4} \\
\cline{4-4}
&  &  & &  &  & \\ [-0.9em]
&  & 20 & \foreignlanguage{greek}{ϲηϲθαι ουχι η ψυχη πλειον εϲτιν τηϲ} & 26 &  &  \\
&  & 27 & \foreignlanguage{greek}{τροφηϲ και το ϲωμα του ενδυματοϲ} & 32 &  &  \\
& \textbf{26} &  & \foreignlanguage{greek}{εμβλεψατε ειϲ τα πετινα του ουρανου} & 6 &  &  \\
&  & 7 & \foreignlanguage{greek}{οτι ου ϲπιρουϲιν ουδε θεριζουϲιν ου} & 12 &  &  \\
&  & 12 & \foreignlanguage{greek}{δε ϲυναγουϲιν ειϲ αποθηκαϲ και ο \textoverline{πηρ}} & 18 &  &  \\
&  & 19 & \foreignlanguage{greek}{υμων ο ουρανιοϲ τρεφει αυτα ουχει} & 24 &  &  \\
&  & 25 & \foreignlanguage{greek}{υμειϲ μαλλον διαφερεται αυτων} & 28 &  &  \\
& \textbf{27} &  & \foreignlanguage{greek}{τιϲ δε εξ υμων μεριμνων δυναται} & 6 &  &  \\
&  & 7 & \foreignlanguage{greek}{προϲθειναι επι την ηλικειαν αυτου} & 11 &  &  \\
&  & 12 & \foreignlanguage{greek}{πηχυν ενα και περι ενδυματοϲ τι} & 4 & \textbf{28} &  \\
&  & 5 & \foreignlanguage{greek}{μεριμναται καταμαθεται τα κρι} & 8 &  &  \\
&  & 8 & \foreignlanguage{greek}{να του αγρου πωϲ αυξανει ου κοπια} & 14 &  &  \\
&  & 15 & \foreignlanguage{greek}{ουδε νηθει λεγω δε υμιν ουδε ϲολο} & 5 & \textbf{29} &  \\
&  & 5 & \foreignlanguage{greek}{μων εν παϲη τη δοξη αυτου περιε} & 11 &  &  \\
&  & 11 & \foreignlanguage{greek}{βαλετο ωϲ εν τουτων} & 14 &  &  \\
& \textbf{30} &  & \foreignlanguage{greek}{ει δε τον χορτον του αγρου ϲημερον} & 7 &  &  \\
&  & 8 & \foreignlanguage{greek}{εν αγρω οντα και αυριον ειϲ κλειβα} & 14 &  &  \\
&  & 14 & \foreignlanguage{greek}{νον βαλλομενον ο \textoverline{θϲ} ουτωϲ αμφι} & 19 &  &  \\
&  & 19 & \foreignlanguage{greek}{εννυϲιν ου πολλω μαλλον υμαϲ ολι} & 24 &  &  \\
&  & 24 & \foreignlanguage{greek}{γοπιϲτοι μη ουν μεριμνηϲηται} & 3 & \textbf{31} &  \\
&  & 4 & \foreignlanguage{greek}{λεγοντεϲ τι φαγωμεν η τι πιωμε̅} & 9 &  &  \\
&  & 10 & \foreignlanguage{greek}{η τι περιβαλωμεθα παντα γαρ ταυ} & 3 & \textbf{32} &  \\
&  & 3 & \foreignlanguage{greek}{τα τα εθνη επιζητει οιδεν γαρ ο} & 9 &  &  \\
&  & 10 & \foreignlanguage{greek}{\textoverline{πηρ} υμων ο ουρανιοϲ οτι χρηζεται} & 15 &  &  \\
&  & 16 & \foreignlanguage{greek}{τουτων απαντων} & 17 &  &  \\
& \textbf{33} &  & \foreignlanguage{greek}{ζητειται δε πρωτον την βαϲιλειαν} & 5 &  &  \\
&  & 6 & \foreignlanguage{greek}{του \textoverline{θυ} και την δικαιωϲυνην αυτου} & 11 &  &  \\
&  & 12 & \foreignlanguage{greek}{και ταυτα παντα προϲτεθηϲεται υμιν} & 16 &  &  \\
& \textbf{34} &  & \foreignlanguage{greek}{μη ουν μεριμνηϲηται ειϲ την αυριον} & 6 &  &  \\
&  & 7 & \foreignlanguage{greek}{η γαρ αυριον μεριμνηϲει εαυτηϲ} & 11 &  &  \\
[0.2em]
\cline{4-4}
\end{tabular}
\end{center}
\end{table}
}
\clearpage
\newpage
 {
 \setlength\arrayrulewidth{1pt}
\begin{table}
\begin{center}
\begin{tabular}{ccc|l|ccc}
\cline{4-4} \\ [-1em]
\multicolumn{7}{c}{\foreignlanguage{greek}{ευαγγελιον κατα μαθθαιον} \textbf{(\nospace{6:34})} } \\ \\ [-1em] % Si on veut ajouter les bordures latérales, remplacer {7}{c} par {7}{|c|}
\cline{4-4} \\
\cline{4-4}
&  &  & &  &  & \\ [-0.9em]
&  & 12 & \foreignlanguage{greek}{αρκετον τη ημερα η κακεια αυτηϲ} & 17 &  &  \\
& \mygospelchapter &  & \foreignlanguage{greek}{μη κρινεται ινα μη κριθηται εν ω γαρ} & 3 &  &  \\
&  & 4 & \foreignlanguage{greek}{κριματι κρινεται κριθηϲεϲθαι και} & 7 &  &  \\
&  & 8 & \foreignlanguage{greek}{εν ω μετρω μετριται μετρηθηϲεται} & 12 &  &  \\
&  & 13 & \foreignlanguage{greek}{υμιν} & 13 &  &  \\
& \textbf{3} &  & \foreignlanguage{greek}{τι δε βλεπειϲ το καρφοϲ το εν τω οφθαλ} & 9 &  &  \\
&  & 9 & \foreignlanguage{greek}{μω του αδελφου ϲου την δε εν τω} & 16 &  &  \\
&  & 17 & \foreignlanguage{greek}{ϲω οφθαλμω δοκον ου κατανοειϲ} & 21 &  &  \\
& \textbf{4} &  & \foreignlanguage{greek}{η πωϲ ερειϲ τω αδελφω ϲου αφεϲ εκ} & 8 &  &  \\
&  & 8 & \foreignlanguage{greek}{βαλω το καρφοϲ απο του οφθαλμου ϲου} & 14 &  &  \\
&  & 15 & \foreignlanguage{greek}{και ιδου η δοκοϲ εν τω οφθαλμω ϲου} & 22 &  &  \\
& \textbf{5} &  & \foreignlanguage{greek}{υποκριτα εκβαλε πρωτον την δο} & 5 &  &  \\
&  & 5 & \foreignlanguage{greek}{κον εκ του οφθαλμου ϲου και τοτε} & 11 &  &  \\
&  & 12 & \foreignlanguage{greek}{διαβλεψειϲ εκβαλειν το καρφοϲ εκ} & 16 &  &  \\
&  & 17 & \foreignlanguage{greek}{του οφθαλμου του αδελφου ϲου} & 21 &  &  \\
& \textbf{6} &  & \foreignlanguage{greek}{μη δωτε το αγιον τοιϲ κυϲιν μηδε} & 7 &  &  \\
&  & 8 & \foreignlanguage{greek}{βαληται τουϲ μαργαριταϲ υμων εμ} & 12 &  &  \\
&  & 12 & \foreignlanguage{greek}{προϲθεν των χοιρων μηποτε κα} & 16 &  &  \\
&  & 16 & \foreignlanguage{greek}{ταπατηϲουϲιν αυτουϲ εν τοιϲ ποϲι̅} & 20 &  &  \\
&  & 21 & \foreignlanguage{greek}{αυτων και ϲτραφεντεϲ ρηξωϲιν} & 24 &  &  \\
&  & 25 & \foreignlanguage{greek}{υμαϲ αιτιτε και δοθηϲεται υ} & 4 & \textbf{7} &  \\
&  & 4 & \foreignlanguage{greek}{μιν ζητειτε και ευρηϲεται κρου} & 8 &  &  \\
&  & 8 & \foreignlanguage{greek}{εται και ανυγηϲεται υμιν παϲ γαρ} & 2 & \textbf{8} &  \\
&  & 3 & \foreignlanguage{greek}{ο αιτων λαμβανει και ο ζητων ευρι} & 10 &  &  \\
&  & 10 & \foreignlanguage{greek}{ϲκει και τω κρουοντι ανοιγηϲεται} & 14 &  &  \\
& \textbf{9} &  & \foreignlanguage{greek}{η τιϲ εϲτιν εξ υμων \textoverline{ανοϲ} ον εαν αι} & 9 &  &  \\
&  & 9 & \foreignlanguage{greek}{τηϲη ο υιοϲ αυτου αρτον μη λιθον ε} & 16 &  &  \\
&  & 16 & \foreignlanguage{greek}{πιδωϲει αυτω και εαν ιχθυν αιτη} & 4 & \textbf{10} &  \\
&  & 4 & \foreignlanguage{greek}{ϲει μη οφιν επιδωϲει αυτω ει ουν} & 2 & \textbf{11} &  \\
&  & 3 & \foreignlanguage{greek}{υμειϲ πονηροι οντεϲ οιδατε δοματα} & 7 &  &  \\
[0.2em]
\cline{4-4}
\end{tabular}
\end{center}
\end{table}
}
\clearpage
\newpage
 {
 \setlength\arrayrulewidth{1pt}
\begin{table}
\begin{center}
\begin{tabular}{ccc|l|ccc}
\cline{4-4} \\ [-1em]
\multicolumn{7}{c}{\foreignlanguage{greek}{ευαγγελιον κατα μαθθαιον} \textbf{(\nospace{7:11})} } \\ \\ [-1em] % Si on veut ajouter les bordures latérales, remplacer {7}{c} par {7}{|c|}
\cline{4-4} \\
\cline{4-4}
&  &  & &  &  & \\ [-0.9em]
&  & 8 & \foreignlanguage{greek}{αγαθα διδοναι τοιϲ τεκνοιϲ υμων} & 12 &  &  \\
&  & 13 & \foreignlanguage{greek}{ποϲω μαλλον ο \textoverline{πηρ} υμων ο εν τοιϲ ου} & 21 &  &  \\
&  & 21 & \foreignlanguage{greek}{ρανοιϲ δωϲει αγαθα τοιϲ αιτουϲιν αυτο̅} & 26 &  &  \\
& \textbf{12} &  & \foreignlanguage{greek}{παντα ουν οϲα εαν θεληται ινα ποιω} & 7 &  &  \\
&  & 7 & \foreignlanguage{greek}{ϲιν υμιν οι \textoverline{ανοι} ουτωϲ και υμειϲ ποι} & 14 &  &  \\
&  & 14 & \foreignlanguage{greek}{ειται αυτοιϲ ουτοϲ γαρ εϲτιν ο νομοϲ} & 20 &  &  \\
&  & 21 & \foreignlanguage{greek}{και οι προφηται} & 23 &  &  \\
& \textbf{13} &  & \foreignlanguage{greek}{ειϲελθατε δια τηϲ ϲτενηϲ πυληϲ οτι} & 6 &  &  \\
&  & 7 & \foreignlanguage{greek}{πλατια η πυλη και ευρυχωροϲ η οδοϲ} & 13 &  &  \\
&  & 14 & \foreignlanguage{greek}{η απαγουϲα ειϲ την απωλειαν και πολ} & 20 &  &  \\
&  & 20 & \foreignlanguage{greek}{λοι ειϲιν οι ειϲερχομενοι δι αυτηϲ} & 25 &  &  \\
& \textbf{14} &  & \foreignlanguage{greek}{τι ϲτενη η πυλη και τεθλιμμενη η ο} & 8 &  &  \\
&  & 8 & \foreignlanguage{greek}{δοϲ η απαγουϲα ειϲ την ζωην και ολει} & 15 &  &  \\
&  & 15 & \foreignlanguage{greek}{γοι ειϲιν οι ευριϲκοντεϲ αυτην} & 19 &  &  \\
& \textbf{15} &  & \foreignlanguage{greek}{προϲεχεται δε απο των ψευδοπροφη} & 5 &  &  \\
&  & 5 & \foreignlanguage{greek}{των οιτινεϲ ερχονται προϲ υμαϲ εν} & 10 &  &  \\
&  & 11 & \foreignlanguage{greek}{ενδυμαϲιν προβατων εϲωθεν δε ει} & 15 &  &  \\
&  & 15 & \foreignlanguage{greek}{ϲιν λυκοι αρπαγεϲ απο των καρπων} & 3 & \textbf{16} &  \\
&  & 4 & \foreignlanguage{greek}{αυτων επιγνωϲεϲθαι αυτουϲ} & 6 &  &  \\
&  & 7 & \foreignlanguage{greek}{μητι ϲυλλεγουϲιν απο ακανθων ϲτα} & 11 &  &  \\
&  & 11 & \foreignlanguage{greek}{φυλην η απο τριβολων ϲυκα} & 15 &  &  \\
& \textbf{17} &  & \foreignlanguage{greek}{ουτωϲ παν δενδρον αγαθον καρπουϲ καλουϲ} & 6 &  &  \\
&  & 7 & \foreignlanguage{greek}{ποιει το δε ϲαπρον δενδρον καρ} & 12 &  &  \\
&  & 12 & \foreignlanguage{greek}{πουϲ πονηρουϲ ποιει} & 14 &  &  \\
& \textbf{18} &  & \foreignlanguage{greek}{ου δυναται δενδρον αγαθον καρπουϲ} & 5 &  &  \\
&  & 6 & \foreignlanguage{greek}{πονηρουϲ ποιειν ουδε δενδρον ϲα} & 10 &  &  \\
&  & 10 & \foreignlanguage{greek}{προν καρπουϲ καλουϲ ποιειν} & 13 &  &  \\
& \textbf{19} &  & \foreignlanguage{greek}{παν δενδρον μη ποιουν καρπον κα} & 6 &  &  \\
&  & 6 & \foreignlanguage{greek}{λον εκκοπτεται και ειϲ πυρ βαλλεται} & 11 &  &  \\
& \textbf{20} &  & \foreignlanguage{greek}{αρα γε απο των καρπων αυτων επιγνωϲεϲθε} & 7 &  &  \\
&  & 8 & \foreignlanguage{greek}{αυτουϲ} & 8 &  &  \\
[0.2em]
\cline{4-4}
\end{tabular}
\end{center}
\end{table}
}
\clearpage
\newpage
 {
 \setlength\arrayrulewidth{1pt}
\begin{table}
\begin{center}
\begin{tabular}{ccc|l|ccc}
\cline{4-4} \\ [-1em]
\multicolumn{7}{c}{\foreignlanguage{greek}{ευαγγελιον κατα μαθθαιον} \textbf{(\nospace{7:21})} } \\ \\ [-1em] % Si on veut ajouter les bordures latérales, remplacer {7}{c} par {7}{|c|}
\cline{4-4} \\
\cline{4-4}
&  &  & &  &  & \\ [-0.9em]
& \textbf{21} &  & \foreignlanguage{greek}{ου παϲ ο λεγων μοι \textoverline{κε} \textoverline{κε} ειϲελευϲεται} & 8 &  &  \\
&  & 9 & \foreignlanguage{greek}{ειϲ την βαϲιλειαν των ουρανων} & 13 &  &  \\
&  & 14 & \foreignlanguage{greek}{αλλ ο ποιων το θελημα του \textoverline{πρϲ} μου} & 21 &  &  \\
&  & 22 & \foreignlanguage{greek}{του εν ουρανοιϲ αυτοϲ ειϲελευϲεται} & 26 &  &  \\
&  & 27 & \foreignlanguage{greek}{ειϲ την βαϲιλειαν των ουρανων} & 31 &  &  \\
& \textbf{22} &  & \foreignlanguage{greek}{πολλοι ερουϲιν μοι εν εκεινη τη ημε} & 7 &  &  \\
&  & 7 & \foreignlanguage{greek}{ρα \textoverline{κε} \textoverline{κε} ου τω ϲω ονοματι επροφη} & 14 &  &  \\
&  & 14 & \foreignlanguage{greek}{τευϲαμεν και τω ϲω ονοματι δαι} & 19 &  &  \\
&  & 19 & \foreignlanguage{greek}{μονια εξεβαλομεν και τω ϲω ονο} & 24 &  &  \\
&  & 24 & \foreignlanguage{greek}{ματι δυναμειϲ πολλαϲ εποιηϲαμεν} & 27 &  &  \\
& \textbf{23} &  & \foreignlanguage{greek}{και τοτε ομολογηϲω αυτοιϲ οτι ου} & 6 &  &  \\
&  & 6 & \foreignlanguage{greek}{δεποτε εγνων υμαϲ αποχωριται} & 9 &  &  \\
&  & 10 & \foreignlanguage{greek}{απ εμου οι εργαζομενοι την ανομια̅} & 15 &  &  \\
& \textbf{24} &  & \foreignlanguage{greek}{παϲ ουν οϲτιϲ ακουει μου τουϲ λογουϲ} & 7 &  &  \\
&  & 8 & \foreignlanguage{greek}{τουτουϲ και ποιει αυτουϲ ομοιωϲω} & 12 &  &  \\
&  & 13 & \foreignlanguage{greek}{αυτον ανδρι φρονιμω οϲτιϲ ωκο} & 17 &  &  \\
&  & 17 & \foreignlanguage{greek}{δομηϲεν αυτου την οικιαν επι την} & 22 &  &  \\
&  & 23 & \foreignlanguage{greek}{πετραν και κατεβη η βροχη και ηλ} & 6 & \textbf{25} &  \\
&  & 6 & \foreignlanguage{greek}{θον οι ποταμοι και επνευϲαν οι ανε} & 12 &  &  \\
&  & 12 & \foreignlanguage{greek}{μοι και προϲεκρουϲαν τη οικεια εκει} & 17 &  &  \\
&  & 17 & \foreignlanguage{greek}{νη και ουκ επεϲεν τεθεμελιωτο γαρ} & 22 &  &  \\
&  & 23 & \foreignlanguage{greek}{επι την πετραν} & 25 &  &  \\
& \textbf{26} &  & \foreignlanguage{greek}{και παϲ ο ακουων μου τουϲ λογουϲ του} & 8 &  &  \\
&  & 8 & \foreignlanguage{greek}{τουϲ και μη ποιων αυτουϲ ομοιω} & 13 &  &  \\
&  & 13 & \foreignlanguage{greek}{θηϲεται ανδρι μωρω οϲτιϲ ωκοδο} & 17 &  &  \\
&  & 17 & \foreignlanguage{greek}{μηϲεν αυτου την οικειαν επι την} & 22 &  &  \\
&  & 23 & \foreignlanguage{greek}{αμμον και κατεβη η βροχη και ηλ} & 6 & \textbf{27} &  \\
&  & 6 & \foreignlanguage{greek}{θον οι ποταμοι και επνευϲαν οι ανε} & 12 &  &  \\
&  & 12 & \foreignlanguage{greek}{μοι και προϲεκοψαν τη οικεια εκεινη} & 17 &  &  \\
&  & 18 & \foreignlanguage{greek}{και επεϲεν και ην η πτωϲιϲ αυτηϲ με} & 25 &  &  \\
&  & 25 & \foreignlanguage{greek}{γαλη} & 25 &  &  \\
[0.2em]
\cline{4-4}
\end{tabular}
\end{center}
\end{table}
}
\clearpage
\newpage
 {
 \setlength\arrayrulewidth{1pt}
\begin{table}
\begin{center}
\begin{tabular}{ccc|l|ccc}
\cline{4-4} \\ [-1em]
\multicolumn{7}{c}{\foreignlanguage{greek}{ευαγγελιον κατα μαθθαιον} \textbf{(\nospace{7:28})} } \\ \\ [-1em] % Si on veut ajouter les bordures latérales, remplacer {7}{c} par {7}{|c|}
\cline{4-4} \\
\cline{4-4}
&  &  & &  &  & \\ [-0.9em]
& \textbf{28} &  & \foreignlanguage{greek}{και εγενετο οτε ετελεϲεν ο \textoverline{ιϲ} τουϲ λο} & 8 &  &  \\
&  & 8 & \foreignlanguage{greek}{γουϲ τουτουϲ εξεπληϲϲοντο οι οχλοι} & 12 &  &  \\
&  & 13 & \foreignlanguage{greek}{επι τη διδαχη αυτου ην γαρ διδαϲκω̅} & 3 & \textbf{29} &  \\
&  & 4 & \foreignlanguage{greek}{αυτουϲ ωϲ εξουϲιαν εχων και ουχ ωϲ} & 10 &  &  \\
&  & 11 & \foreignlanguage{greek}{οι γραμματειϲ αυτων ϗ οι φαριϲαιοι} & 16 &  &  \\
& \mygospelchapter &  & \foreignlanguage{greek}{καταβαντοϲ δε αυτου απο του ορουϲ η} & 7 &  &  \\
&  & 7 & \foreignlanguage{greek}{κολουθηϲαν αυτω οχλοι πολλοι} & 10 &  &  \\
& \textbf{2} &  & \foreignlanguage{greek}{και ιδου λεπροϲ ελθων προϲεκυνει αυ} & 6 &  &  \\
&  & 6 & \foreignlanguage{greek}{τω λεγων \textoverline{κε} εαν θεληϲ δυναϲαι με} & 12 &  &  \\
&  & 13 & \foreignlanguage{greek}{καθαριϲαι και εκτιναϲ την χειρα} & 4 & \textbf{3} &  \\
&  & 5 & \foreignlanguage{greek}{ηψατο αυτου ο \textoverline{ιϲ} λεγων θελω καθαρι} & 11 &  &  \\
&  & 11 & \foreignlanguage{greek}{ϲθητι και ευθεωϲ εκαθαριϲθη αυτου η λεπρα} & 17 &  &  \\
& \textbf{4} &  & \foreignlanguage{greek}{και λεγει αυτω ο \textoverline{ιϲ} ορα μηδενι ειπηϲ αλ} & 9 &  &  \\
&  & 9 & \foreignlanguage{greek}{λα υπαγε ϲεαυτον διξον τω ιερει και} & 15 &  &  \\
&  & 16 & \foreignlanguage{greek}{προϲενεγκε το δωρον ο προϲεταξεν} & 20 &  &  \\
&  & 21 & \foreignlanguage{greek}{μωυϲηϲ ειϲ μαρτυριον αυτοιϲ} & 24 &  &  \\
& \textbf{5} &  & \foreignlanguage{greek}{ειϲελθοντι δε αυτω ειϲ καπερναουμ} & 5 &  &  \\
&  & 6 & \foreignlanguage{greek}{προϲηλθεν αυτω εκατονταρχηϲ πα} & 9 &  &  \\
&  & 9 & \foreignlanguage{greek}{ρακαλων αυτον και λεγων \textoverline{κε} ο παιϲ} & 5 & \textbf{6} &  \\
&  & 6 & \foreignlanguage{greek}{μου βεβληται εν τη οικεια παραλυτικοϲ} & 11 &  &  \\
&  & 12 & \foreignlanguage{greek}{δινωϲ βαϲανιζομενοϲ} & 13 &  &  \\
& \textbf{7} &  & \foreignlanguage{greek}{και λεγει αυτω ο \textoverline{ιϲ} εγω ελθων θεραπευ} & 8 &  &  \\
&  & 8 & \foreignlanguage{greek}{ϲω αυτον και αποκριθειϲ ο εκατον} & 4 & \textbf{8} &  \\
&  & 4 & \foreignlanguage{greek}{ταρχοϲ εφη \textoverline{κε} ουκ ιμει ικανοϲ ινα} & 10 &  &  \\
&  & 11 & \foreignlanguage{greek}{μου υπο την ϲτεγην ειϲελθηϲ αλλα μο} & 17 &  &  \\
&  & 17 & \foreignlanguage{greek}{νον ειπε λογω και ιαθηϲεται ο παιϲ μου} & 24 &  &  \\
& \textbf{9} &  & \foreignlanguage{greek}{και γαρ εγω \textoverline{ανοϲ} ειμει υπο εξουϲιαν ε} & 8 &  &  \\
&  & 8 & \foreignlanguage{greek}{χων υπ εμαυτον ϲτρατιωταϲ και λε} & 13 &  &  \\
&  & 13 & \foreignlanguage{greek}{γω τουτω πορευθητι και πορευεται} & 17 &  &  \\
&  & 18 & \foreignlanguage{greek}{και αλλω ερχου και ερχεται και τω δου} & 25 &  &  \\
[0.2em]
\cline{4-4}
\end{tabular}
\end{center}
\end{table}
}
\clearpage
\newpage
 {
 \setlength\arrayrulewidth{1pt}
\begin{table}
\begin{center}
\begin{tabular}{ccc|l|ccc}
\cline{4-4} \\ [-1em]
\multicolumn{7}{c}{\foreignlanguage{greek}{ευαγγελιον κατα μαθθαιον} \textbf{(\nospace{8:9})} } \\ \\ [-1em] % Si on veut ajouter les bordures latérales, remplacer {7}{c} par {7}{|c|}
\cline{4-4} \\
\cline{4-4}
&  &  & &  &  & \\ [-0.9em]
&  & 25 & \foreignlanguage{greek}{λω μου ποιηϲον τουτο και ποιει} & 30 &  &  \\
& \textbf{10} &  & \foreignlanguage{greek}{ακουϲαϲ δε ο \textoverline{ιϲ} εθαυμαϲεν και ειπεν τοιϲ} & 8 &  &  \\
&  & 9 & \foreignlanguage{greek}{ακολουθουϲιν αμην λεγω υμιν παρ ου} & 14 &  &  \\
&  & 14 & \foreignlanguage{greek}{δενι τοϲαυτην πιϲτιν εν τω ιϲραηλ ευρο̅} & 20 &  &  \\
& \textbf{11} &  & \foreignlanguage{greek}{λεγω δε υμιν οτι πολλοι απο ανατολων} & 7 &  &  \\
&  & 8 & \foreignlanguage{greek}{και δυϲμων ηξουϲιν και ανακλειθη} & 12 &  &  \\
&  & 12 & \foreignlanguage{greek}{ϲονται μετα αβρααμ και ιϲαακ και ι} & 18 &  &  \\
&  & 18 & \foreignlanguage{greek}{ακωβ εν τη βαϲιλεια των ουρανων} & 23 &  &  \\
& \textbf{12} &  & \foreignlanguage{greek}{οι δε υιοι τηϲ βαϲιλειαϲ εκβληθηϲονται} & 6 &  &  \\
&  & 7 & \foreignlanguage{greek}{ειϲ το ϲκοτοϲ το εξωτερον εκει εϲται} & 13 &  &  \\
&  & 14 & \foreignlanguage{greek}{ο κλαυθμοϲ και ο βρυγμοϲ των οδοντω̅} & 20 &  &  \\
& \textbf{13} &  & \foreignlanguage{greek}{και ειπεν ο \textoverline{ιϲ} τω εκατονταρχη υπαγε} & 7 &  &  \\
&  & 8 & \foreignlanguage{greek}{ωϲ επιϲτευϲαϲ γενηθητω ϲοι και ιαθη} & 13 &  &  \\
&  & 14 & \foreignlanguage{greek}{ο παιϲ αυτου εν τη ημερα εκεινη} & 20 &  &  \\
& \textbf{14} &  & \foreignlanguage{greek}{και ελθων ο \textoverline{ιϲ} ειϲ την οικειαν πετρου} & 8 &  &  \\
&  & 9 & \foreignlanguage{greek}{ειδεν την πενθεραν αυτου βεβλη} & 13 &  &  \\
&  & 13 & \foreignlanguage{greek}{μενην και πυρεϲϲουϲαν και ηψατο} & 2 & \textbf{15} &  \\
&  & 3 & \foreignlanguage{greek}{τηϲ χειροϲ αυτηϲ και αφηκεν αυτην} & 8 &  &  \\
&  & 9 & \foreignlanguage{greek}{ο πυρετοϲ και ηγερθη και διηκονι αυτω} & 15 &  &  \\
& \textbf{16} &  & \foreignlanguage{greek}{οψειαϲ δε γονομενηϲ προϲηνεγκαν} & 4 &  &  \\
&  & 5 & \foreignlanguage{greek}{αυτω δαιμονιζομενουϲ πολλουϲ} & 7 &  &  \\
&  & 8 & \foreignlanguage{greek}{και εξεβαλεν τα \textoverline{πντα} λογω και παν} & 15 &  &  \\
&  & 15 & \foreignlanguage{greek}{ταϲ τουϲ κακωϲ εχονταϲ εθεραπευϲε̅} & 19 &  &  \\
& \textbf{17} &  & \foreignlanguage{greek}{οπωϲ πληρωθη το ρηθεν δια ηϲαιου} & 6 &  &  \\
&  & 7 & \foreignlanguage{greek}{του προφητου λεγοντοϲ οτι αυτοϲ} & 11 &  &  \\
&  & 12 & \foreignlanguage{greek}{ταϲ αϲθενιαϲ ημων ελαβεν και ταϲ} & 17 &  &  \\
&  & 18 & \foreignlanguage{greek}{νοϲουϲ εβαϲταϲεν} & 19 &  &  \\
& \textbf{18} &  & \foreignlanguage{greek}{ιδων δε ο \textoverline{ιϲ} οχλον πολυν περι αυτον} & 8 &  &  \\
&  & 9 & \foreignlanguage{greek}{εκελευϲεν απελθειν ειϲ το περαν} & 13 &  &  \\
& \textbf{19} &  & \foreignlanguage{greek}{και προϲελθων ειϲ γραμματευϲ ειπε̅} & 5 &  &  \\
[0.2em]
\cline{4-4}
\end{tabular}
\end{center}
\end{table}
}
\clearpage
\newpage
 {
 \setlength\arrayrulewidth{1pt}
\begin{table}
\begin{center}
\begin{tabular}{ccc|l|ccc}
\cline{4-4} \\ [-1em]
\multicolumn{7}{c}{\foreignlanguage{greek}{ευαγγελιον κατα μαθθαιον} \textbf{(\nospace{8:19})} } \\ \\ [-1em] % Si on veut ajouter les bordures latérales, remplacer {7}{c} par {7}{|c|}
\cline{4-4} \\
\cline{4-4}
&  &  & &  &  & \\ [-0.9em]
&  & 6 & \foreignlanguage{greek}{αυτω διδαϲκαλε ακολουθηϲω ϲοι οπου} & 10 &  &  \\
&  & 11 & \foreignlanguage{greek}{αν απερχη} & 12 &  &  \\
& \textbf{20} &  & \foreignlanguage{greek}{και λεγει αυτω ο \textoverline{ιϲ} αι αλωπεκεϲ φωλαι} & 8 &  &  \\
&  & 8 & \foreignlanguage{greek}{ουϲ εχουϲιν και τα πετινα του ουρανου} & 14 &  &  \\
&  & 15 & \foreignlanguage{greek}{καταϲκηνωϲειϲ ο δε υιοϲ του ανθρω} & 20 &  &  \\
&  & 20 & \foreignlanguage{greek}{που ουκ εχει που την κεφαλην κλεινη} & 26 &  &  \\
& \textbf{21} &  & \foreignlanguage{greek}{ετεροϲ δε των μαθητων αυτου ειπεν} & 6 &  &  \\
&  & 7 & \foreignlanguage{greek}{αυτω \textoverline{κε} επιτρεψον μοι πρωτον α} & 12 &  &  \\
&  & 12 & \foreignlanguage{greek}{πελθειν και θαψαι τον \textoverline{πρα} μου} & 17 &  &  \\
& \textbf{22} &  & \foreignlanguage{greek}{ο δε \textoverline{ιϲ} ειπεν αυτω ακολουθει μοι και α} & 9 &  &  \\
&  & 9 & \foreignlanguage{greek}{φεϲ τουϲ νεκρουϲ θαψαι τουϲ εαυτων} & 14 &  &  \\
&  & 15 & \foreignlanguage{greek}{νεκρουϲ και ενβαντι αυτω ειϲ το} & 5 & \textbf{23} &  \\
&  & 6 & \foreignlanguage{greek}{πλοιον ηκολουθηϲαν αυτω οι μαθη} & 10 &  &  \\
&  & 10 & \foreignlanguage{greek}{ται αυτου και ιδου ϲιϲμοϲ μεγαϲ} & 4 & \textbf{24} &  \\
&  & 5 & \foreignlanguage{greek}{εγενετο εν τη θαλαϲϲη ωϲτε το πλοιο̅} & 11 &  &  \\
&  & 12 & \foreignlanguage{greek}{καλυπτεϲθαι υπο των κυματων} & 15 &  &  \\
&  & 16 & \foreignlanguage{greek}{αυτοϲ δε εκαθευδεν και προϲελθον} & 2 & \textbf{25} &  \\
&  & 2 & \foreignlanguage{greek}{τεϲ οι μαθηται αυτου ηγειραν αυτον} & 7 &  &  \\
&  & 8 & \foreignlanguage{greek}{λεγοντεϲ \textoverline{κε} ϲωϲον ημαϲ απολλυμεθα} & 12 &  &  \\
& \textbf{26} &  & \foreignlanguage{greek}{και λεγει αυτοιϲ τι δειλοι εϲται ολιγοπιϲτοι} & 7 &  &  \\
&  & 8 & \foreignlanguage{greek}{τοτε εγερθειϲ επετιμηϲεν τοιϲ ανε} & 12 &  &  \\
&  & 12 & \foreignlanguage{greek}{μοιϲ και τη θαλαϲϲη και εγενετο γα} & 18 &  &  \\
&  & 18 & \foreignlanguage{greek}{ληνη μεγαλη} & 19 &  &  \\
& \textbf{27} &  & \foreignlanguage{greek}{οι δε ανθρωποι εθαυμαϲαν λεγοντεϲ} & 5 &  &  \\
&  & 6 & \foreignlanguage{greek}{ποταποϲ εϲτιν ουτοϲ ο \textoverline{ανοϲ} οτι και οι} & 13 &  &  \\
&  & 14 & \foreignlanguage{greek}{ανεμοι και η θαλαϲϲα αυτω υπακουουϲι̅} & 19 &  &  \\
& \textbf{28} &  & \foreignlanguage{greek}{και ελθοντι αυτω ειϲ το περαν των γερ} & 8 &  &  \\
&  & 8 & \foreignlanguage{greek}{γεϲηνων υπηντηϲαν αυτω δυο δαι} & 12 &  &  \\
&  & 12 & \foreignlanguage{greek}{μονιζομενοι εκ των μνημιων εξερ} & 16 &  &  \\
&  & 16 & \foreignlanguage{greek}{χομενοι χαλεποι λιαν ωϲτε μη ιϲχυει̅} & 21 &  &  \\
[0.2em]
\cline{4-4}
\end{tabular}
\end{center}
\end{table}
}
\clearpage
\newpage
 {
 \setlength\arrayrulewidth{1pt}
\begin{table}
\begin{center}
\begin{tabular}{ccc|l|ccc}
\cline{4-4} \\ [-1em]
\multicolumn{7}{c}{\foreignlanguage{greek}{ευαγγελιον κατα μαθθαιον} \textbf{(\nospace{8:28})} } \\ \\ [-1em] % Si on veut ajouter les bordures latérales, remplacer {7}{c} par {7}{|c|}
\cline{4-4} \\
\cline{4-4}
&  &  & &  &  & \\ [-0.9em]
&  & 22 & \foreignlanguage{greek}{τινα παρελθειν δια τηϲ οδου εκεινηϲ} & 27 &  &  \\
& \textbf{29} &  & \foreignlanguage{greek}{και ιδου εκραζον λεγοντεϲ τι ημιν και} & 7 &  &  \\
&  & 8 & \foreignlanguage{greek}{ϲοι \textoverline{ιυ} υιε του \textoverline{θυ} ηλθεϲ ωδε απολεϲαι} & 15 &  &  \\
&  & 16 & \foreignlanguage{greek}{ημαϲ και προ καιρου βαϲανιϲαι} & 20 &  &  \\
& \textbf{30} &  & \foreignlanguage{greek}{ην δε μακραν απ αυτων αγελη χοιρω̅} & 7 &  &  \\
&  & 8 & \foreignlanguage{greek}{πολλων βοϲκομενων οι δε δαιμο} & 3 & \textbf{31} &  \\
&  & 3 & \foreignlanguage{greek}{νεϲ παρεκαλουν αυτον λεγοντεϲ} & 6 &  &  \\
&  & 7 & \foreignlanguage{greek}{ει εκβαλλειϲ ημαϲ επιτρεψον ημι̅} & 11 &  &  \\
&  & 12 & \foreignlanguage{greek}{απελθειν ειϲ την αγελην των χοιρω̅} & 17 &  &  \\
& \textbf{32} &  & \foreignlanguage{greek}{και ειπεν αυτοιϲ υπαγεται} & 4 &  &  \\
&  & 5 & \foreignlanguage{greek}{οι δε εξελθοντεϲ απηλθον ειϲ την} & 10 &  &  \\
&  & 11 & \foreignlanguage{greek}{αγελην των χοιρων και ιδου ωρ} & 16 &  &  \\
&  & 16 & \foreignlanguage{greek}{μηϲεν παϲα η αγελη κατα του κρη} & 22 &  &  \\
&  & 22 & \foreignlanguage{greek}{μνου ειϲ την θαλαϲϲαν και απεθα} & 27 &  &  \\
&  & 27 & \foreignlanguage{greek}{νον εν τοιϲ υδαϲιν} & 30 &  &  \\
& \textbf{33} &  & \foreignlanguage{greek}{οι δε βοϲκοντεϲ εφυγον και απελ} & 6 &  &  \\
&  & 6 & \foreignlanguage{greek}{θοντεϲ ειϲ την πολιν απηγγειλο̅} & 10 &  &  \\
&  & 11 & \foreignlanguage{greek}{παντα και τα των δαιμονιζομενω̅} & 15 &  &  \\
& \textbf{34} &  & \foreignlanguage{greek}{και ιδου παϲα η πολιϲ εξηλθεν ειϲ} & 7 &  &  \\
&  & 8 & \foreignlanguage{greek}{ϲυναντηϲιν τω \textoverline{ιυ} και ιδοντεϲ αυ} & 13 &  &  \\
&  & 13 & \foreignlanguage{greek}{τον παρεκαλεϲαν ινα μεταβη α} & 17 &  &  \\
&  & 17 & \foreignlanguage{greek}{πο των οριων αυτων} & 20 &  &  \\
& \mygospelchapter &  & \foreignlanguage{greek}{και ενβαϲ ειϲ το πλοιον διεπεραϲεν} & 6 &  &  \\
&  & 7 & \foreignlanguage{greek}{και ηλθεν ειϲ την ιουδαιαν πολιν} & 12 &  &  \\
& \textbf{2} &  & \foreignlanguage{greek}{και ιδου προϲεφερον αυτω παραλυ} & 5 &  &  \\
&  & 5 & \foreignlanguage{greek}{τικον επι κλεινηϲ βεβλημενον} & 8 &  &  \\
&  & 9 & \foreignlanguage{greek}{και ιδων ο \textoverline{ιϲ} την πιϲτιν αυτων ειπε̅} & 16 &  &  \\
&  & 17 & \foreignlanguage{greek}{τω παραλυτικω θαρϲει τεκνον} & 20 &  &  \\
&  & 21 & \foreignlanguage{greek}{αφεωνται ϲου αι αμαρτιαι} & 24 &  &  \\
& \textbf{3} &  & \foreignlanguage{greek}{και ιδου τινεϲ των γραμματεων ει} & 6 &  &  \\
[0.2em]
\cline{4-4}
\end{tabular}
\end{center}
\end{table}
}
\clearpage
\newpage
 {
 \setlength\arrayrulewidth{1pt}
\begin{table}
\begin{center}
\begin{tabular}{ccc|l|ccc}
\cline{4-4} \\ [-1em]
\multicolumn{7}{c}{\foreignlanguage{greek}{ευαγγελιον κατα μαθθαιον} \textbf{(\nospace{9:3})} } \\ \\ [-1em] % Si on veut ajouter les bordures latérales, remplacer {7}{c} par {7}{|c|}
\cline{4-4} \\
\cline{4-4}
&  &  & &  &  & \\ [-0.9em]
&  & 6 & \foreignlanguage{greek}{πον εν εαυτοιϲ ουτοϲ βλαϲφημει} & 10 &  &  \\
& \textbf{4} &  & \foreignlanguage{greek}{και ιδων ο \textoverline{ιϲ} ταϲ ενθυμηϲειϲ αυτων} & 7 &  &  \\
&  & 8 & \foreignlanguage{greek}{ειπεν ινα τι υμειϲ ενθυμιϲθαι πο} & 13 &  &  \\
&  & 13 & \foreignlanguage{greek}{νηρα εν ταιϲ καρδιαιϲ υμων τι γαρ ε} & 3 & \textbf{5} &  \\
&  & 3 & \foreignlanguage{greek}{ϲτιν ευκοπωτερον ειπειν αφαιων} & 6 &  &  \\
&  & 6 & \foreignlanguage{greek}{ται ϲου αι αμαρτιαι η ειπειν εγειρε} & 12 &  &  \\
&  & 13 & \foreignlanguage{greek}{και περιπατει ινα δε ειδηται οτι} & 4 & \textbf{6} &  \\
&  & 5 & \foreignlanguage{greek}{εξουϲιαν εχει ο υιοϲ του \textoverline{ανου} αφιεναι} & 11 &  &  \\
&  & 12 & \foreignlanguage{greek}{επι τηϲ γηϲ αμαρτιαϲ τοτε λεγει} & 17 &  &  \\
&  & 18 & \foreignlanguage{greek}{τω παραλυτικω εγερθειϲ αρον ϲου} & 22 &  &  \\
&  & 23 & \foreignlanguage{greek}{την κλεινην και υπαγε ειϲ τον οι} & 29 &  &  \\
&  & 29 & \foreignlanguage{greek}{κον ϲου και εγερθειϲ απηλθεν ειϲ} & 4 & \textbf{7} &  \\
&  & 5 & \foreignlanguage{greek}{τον οικον αυτου} & 7 &  &  \\
& \textbf{8} &  & \foreignlanguage{greek}{ιδοντεϲ δε οι οχλοι εφοβηθηϲαν και} & 6 &  &  \\
&  & 7 & \foreignlanguage{greek}{εδοξαϲαν τον \textoverline{θν} τον δοντα εξου} & 12 &  &  \\
&  & 12 & \foreignlanguage{greek}{ϲιαν τοιαυτην τοιϲ \textoverline{ανοιϲ}} & 15 &  &  \\
& \textbf{9} &  & \foreignlanguage{greek}{και παραγων ο \textoverline{ιϲ} εκειθεν ειδεν \textoverline{ανον}} & 7 &  &  \\
&  & 8 & \foreignlanguage{greek}{καθημενον επι το τελωνιον μαθ} & 12 &  &  \\
&  & 12 & \foreignlanguage{greek}{θεον καλουμενον και λεγει αυτω} & 16 &  &  \\
&  & 17 & \foreignlanguage{greek}{ακολουθει μοι και αναϲταϲ ηκο} & 21 &  &  \\
&  & 21 & \foreignlanguage{greek}{λουθηϲεν αυτω} & 22 &  &  \\
& \textbf{10} &  & \foreignlanguage{greek}{και εγενετο αυτου ανακειμενου εν} & 5 &  &  \\
&  & 6 & \foreignlanguage{greek}{τη οικεια και ιδου τελωναι πολλοι} & 11 &  &  \\
&  & 12 & \foreignlanguage{greek}{και αμαρτωλοι ελθοντεϲ ϲυνανε} & 15 &  &  \\
&  & 15 & \foreignlanguage{greek}{κιντο τω \textoverline{ιυ} και τοιϲ μαθηταιϲ αυτου} & 21 &  &  \\
& \textbf{11} &  & \foreignlanguage{greek}{και ιδοντεϲ οι φαριϲαιοι ελεγον τοιϲ} & 6 &  &  \\
&  & 7 & \foreignlanguage{greek}{μαθηταιϲ αυτου δια τι μετα των} & 12 &  &  \\
&  & 13 & \foreignlanguage{greek}{τελωνων και αμαρτωλων εϲθιει ο} & 17 &  &  \\
&  & 18 & \foreignlanguage{greek}{διδαϲκαλοϲ υμων} & 19 &  &  \\
& \textbf{12} &  & \foreignlanguage{greek}{ο δε \textoverline{ιϲ} ακουϲαϲ ειπεν αυτοιϲ ου χρεια̅} & 8 &  &  \\
[0.2em]
\cline{4-4}
\end{tabular}
\end{center}
\end{table}
}
\clearpage
\newpage
 {
 \setlength\arrayrulewidth{1pt}
\begin{table}
\begin{center}
\begin{tabular}{ccc|l|ccc}
\cline{4-4} \\ [-1em]
\multicolumn{7}{c}{\foreignlanguage{greek}{ευαγγελιον κατα μαθθαιον} \textbf{(\nospace{9:12})} } \\ \\ [-1em] % Si on veut ajouter les bordures latérales, remplacer {7}{c} par {7}{|c|}
\cline{4-4} \\
\cline{4-4}
&  &  & &  &  & \\ [-0.9em]
&  & 9 & \foreignlanguage{greek}{εχουϲιν οι ιϲχυοντεϲ ιατρου αλλα οι κα} & 15 &  &  \\
&  & 15 & \foreignlanguage{greek}{κωϲ εχοντεϲ πορευθεντεϲ δε μαθε} & 3 & \textbf{13} &  \\
&  & 3 & \foreignlanguage{greek}{ται τι εϲτιν ελεον θελω και ου θυϲιαν} & 10 &  &  \\
&  & 11 & \foreignlanguage{greek}{ου γαρ ηλθον δικαιουϲ καλεϲαι αλλα} & 16 &  &  \\
&  & 17 & \foreignlanguage{greek}{αμαρτωλουϲ} & 17 &  &  \\
& \textbf{14} &  & \foreignlanguage{greek}{τοτε προϲερχονται αυτω οι μαθηται} & 5 &  &  \\
&  & 6 & \foreignlanguage{greek}{ιωαννου λεγοντεϲ δια τι ημειϲ και} & 11 &  &  \\
&  & 12 & \foreignlanguage{greek}{οι φαριϲαιοι νηϲτευομεν πολλα} & 15 &  &  \\
&  & 16 & \foreignlanguage{greek}{οι δε μαθηται ϲου ου νηϲτευουϲιν} & 21 &  &  \\
& \textbf{15} &  & \foreignlanguage{greek}{και ειπεν αυτοιϲ ο \textoverline{ιϲ} μη δυνανται οι} & 8 &  &  \\
&  & 9 & \foreignlanguage{greek}{υιοι του νυμφωνοϲ νηϲτευειν εφ ο} & 14 &  &  \\
&  & 14 & \foreignlanguage{greek}{ϲον μετ αυτων εϲτιν ο νυμφιοϲ} & 19 &  &  \\
&  & 20 & \foreignlanguage{greek}{ελευϲονται δε ημεραι οταν αφαρε} & 24 &  &  \\
&  & 24 & \foreignlanguage{greek}{θη απ αυτων ο νυμφιοϲ και τοτε} & 30 &  &  \\
&  & 31 & \foreignlanguage{greek}{νηϲτευϲουϲιν} & 31 &  &  \\
& \textbf{16} &  & \foreignlanguage{greek}{ουδειϲ δε επιβαλλει επιβλημα ρακουϲ} & 5 &  &  \\
&  & 6 & \foreignlanguage{greek}{αγναφουϲ αγναφου επι ιματιω παλαιω ερει γαρ} & 12 &  &  \\
&  & 13 & \foreignlanguage{greek}{το πληρωμα αυτου απο του ιματιου} & 18 &  &  \\
&  & 19 & \foreignlanguage{greek}{και χειρον ϲχιϲμα γεινεται} & 22 &  &  \\
& \textbf{17} &  & \foreignlanguage{greek}{ουδε βαλλουϲιν οινον νεον ειϲ αϲκουϲ} & 6 &  &  \\
&  & 7 & \foreignlanguage{greek}{παλαιουϲ ει δε μη γε ρηγνυνται οι α} & 14 &  &  \\
&  & 14 & \foreignlanguage{greek}{ϲκοι και ο οινοϲ εκχειται και οι αϲκοι} & 21 &  &  \\
&  & 22 & \foreignlanguage{greek}{απολουνται αλλα βαλλουϲιν οινον} & 25 &  &  \\
&  & 26 & \foreignlanguage{greek}{νεον ειϲ αϲκουϲ καινουϲ και αμφο} & 31 &  &  \\
&  & 31 & \foreignlanguage{greek}{τεροι ϲυντηρουνται} & 32 &  &  \\
& \textbf{18} &  & \foreignlanguage{greek}{ταυτα αυτου λαλουντοϲ αυτοιϲ ιδου} & 5 &  &  \\
&  & 6 & \foreignlanguage{greek}{αρχων ειϲ ειϲελθων προϲεκυνει αυτω} & 10 &  &  \\
&  & 11 & \foreignlanguage{greek}{λεγων οτι η θυγατηρ μου αρτι ετε} & 17 &  &  \\
&  & 17 & \foreignlanguage{greek}{λευτηϲεν αλλα ελθων επιθεϲ την} & 21 &  &  \\
&  & 22 & \foreignlanguage{greek}{χειρα ϲου επ αυτην και ζηϲεται} & 27 &  &  \\
[0.2em]
\cline{4-4}
\end{tabular}
\end{center}
\end{table}
}
\clearpage
\newpage
 {
 \setlength\arrayrulewidth{1pt}
\begin{table}
\begin{center}
\begin{tabular}{ccc|l|ccc}
\cline{4-4} \\ [-1em]
\multicolumn{7}{c}{\foreignlanguage{greek}{ευαγγελιον κατα μαθθαιον} \textbf{(\nospace{9:19})} } \\ \\ [-1em] % Si on veut ajouter les bordures latérales, remplacer {7}{c} par {7}{|c|}
\cline{4-4} \\
\cline{4-4}
&  &  & &  &  & \\ [-0.9em]
& \textbf{19} &  & \foreignlanguage{greek}{και εγερθειϲ ο \textoverline{ιϲ} ηκολουθηϲεν αυτω και} & 7 &  &  \\
&  & 8 & \foreignlanguage{greek}{οι μαθηται αυτου} & 10 &  &  \\
& \textbf{20} &  & \foreignlanguage{greek}{και ιδου γυνη αιμοροουϲα δωδεκα ετη} & 6 &  &  \\
&  & 7 & \foreignlanguage{greek}{προϲελθουϲα οπιϲθεν ηψατο του κρα} & 11 &  &  \\
&  & 11 & \foreignlanguage{greek}{ϲπεδου του ιματιου αυτου ελεγεν} & 1 & \textbf{21} &  \\
&  & 2 & \foreignlanguage{greek}{γαρ εν εαυτη εαν μονον αψωμαι του} & 8 &  &  \\
&  & 9 & \foreignlanguage{greek}{ιματιου αυτου ϲωθηϲομαι} & 11 &  &  \\
& \textbf{22} &  & \foreignlanguage{greek}{ο δε \textoverline{ιϲ} επιϲτραφειϲ και ιδων αυτην ειπε̅} & 8 &  &  \\
&  & 9 & \foreignlanguage{greek}{θαρϲει θυγατηρ η πιϲτιϲ ϲου ϲεϲωκεν ϲε} & 15 &  &  \\
&  & 16 & \foreignlanguage{greek}{και εϲωθη η γυνη απο τηϲ ωραϲ εκεινηϲ} & 23 &  &  \\
& \textbf{23} &  & \foreignlanguage{greek}{και ελθων ο \textoverline{ιϲ} ειϲ την οικειαν του αρχον} & 9 &  &  \\
&  & 9 & \foreignlanguage{greek}{τοϲ και ιδων τουϲ αυληταϲ και τον ο} & 16 &  &  \\
&  & 16 & \foreignlanguage{greek}{χλον θορυβουμενον λεγει αυτοιϲ} & 2 & \textbf{24} &  \\
&  & 3 & \foreignlanguage{greek}{αναχωρειτε ου γαρ απεθανεν το κο} & 8 &  &  \\
&  & 8 & \foreignlanguage{greek}{ραϲιον αλλα καθευδει και κατεγε} & 12 &  &  \\
&  & 12 & \foreignlanguage{greek}{λουν αυτου} & 13 &  &  \\
& \textbf{25} &  & \foreignlanguage{greek}{οτε δε εξεβληθη ο οχλοϲ ειϲελθων ε} & 7 &  &  \\
&  & 7 & \foreignlanguage{greek}{κρατηϲεν τηϲ χειροϲ αυτηϲ και ηγερ} & 12 &  &  \\
&  & 12 & \foreignlanguage{greek}{θη το κοραϲιον και εξηλθεν η φη} & 4 & \textbf{26} &  \\
&  & 4 & \foreignlanguage{greek}{μη αυτη ειϲ ολην την γην εκεινην} & 10 &  &  \\
& \textbf{27} &  & \foreignlanguage{greek}{και παραγοντι τω \textoverline{ιυ} εκειθεν ηκολου} & 6 &  &  \\
&  & 6 & \foreignlanguage{greek}{θηϲαν αυτω δυο τυφλοι κραζοντεϲ} & 10 &  &  \\
&  & 11 & \foreignlanguage{greek}{και λεγοντεϲ ελεηϲον ημαϲ υιοϲ δαυ} & 16 &  &  \\
&  & 16 & \foreignlanguage{greek}{ειδ ελθοντι δε ειϲ την οικειαν} & 5 & \textbf{28} &  \\
&  & 6 & \foreignlanguage{greek}{προϲηλθον αυτω οι τυφλοι} & 9 &  &  \\
&  & 10 & \foreignlanguage{greek}{και λεγει αυτοιϲ ο \textoverline{ιϲ} πιϲτευεται οτι} & 16 &  &  \\
&  & 17 & \foreignlanguage{greek}{δυναμαι τουτο ποιηϲαι λεγουϲιν} & 20 &  &  \\
&  & 21 & \foreignlanguage{greek}{αυτω ναι \textoverline{κε} τοτε ηψατο των οφθαλ} & 4 & \textbf{29} &  \\
&  & 4 & \foreignlanguage{greek}{μων αυτων λεγων κατα την πιϲτιν} & 9 &  &  \\
&  & 10 & \foreignlanguage{greek}{υμων γενηθητω υμιν και ανεω} & 2 & \textbf{30} &  \\
[0.2em]
\cline{4-4}
\end{tabular}
\end{center}
\end{table}
}
\clearpage
\newpage
 {
 \setlength\arrayrulewidth{1pt}
\begin{table}
\begin{center}
\begin{tabular}{ccc|l|ccc}
\cline{4-4} \\ [-1em]
\multicolumn{7}{c}{\foreignlanguage{greek}{ευαγγελιον κατα μαθθαιον} \textbf{(\nospace{9:30})} } \\ \\ [-1em] % Si on veut ajouter les bordures latérales, remplacer {7}{c} par {7}{|c|}
\cline{4-4} \\
\cline{4-4}
&  &  & &  &  & \\ [-0.9em]
&  & 2 & \foreignlanguage{greek}{χθηϲαν αυτων οι οφθαλμοι} & 5 &  &  \\
&  & 6 & \foreignlanguage{greek}{και ενεβριμηϲατο αυτοιϲ ο \textoverline{ιϲ} λεγων} & 11 &  &  \\
&  & 12 & \foreignlanguage{greek}{ορατε μηδειϲ γινωϲκετω οι δε εξελθο̅} & 3 & \textbf{31} &  \\
&  & 3 & \foreignlanguage{greek}{τεϲ διεφημιϲαν αυτον εν ολη τη γη ε} & 10 &  &  \\
&  & 10 & \foreignlanguage{greek}{κεινη} & 10 &  &  \\
& \textbf{32} &  & \foreignlanguage{greek}{αυτων δε εξερχομενων ιδου προϲη} & 5 &  &  \\
&  & 5 & \foreignlanguage{greek}{νεγκαν αυτω \textoverline{ανον} κωφον δαιμονι} & 9 &  &  \\
&  & 9 & \foreignlanguage{greek}{ζομενον και εκβληθεντοϲ του δαι} & 4 & \textbf{33} &  \\
&  & 4 & \foreignlanguage{greek}{μονιου ελαληϲεν ο κωφοϲ} & 7 &  &  \\
&  & 8 & \foreignlanguage{greek}{και εθαυμαϲαν οι οχλοι λεγοντεϲ} & 12 &  &  \\
&  & 13 & \foreignlanguage{greek}{ουδεποτε εφανη ουτωϲ εν τω ιϲραηλ} & 18 &  &  \\
& \textbf{34} &  & \foreignlanguage{greek}{οι δε φαριϲαιοι ελεγον τω αρχοντι τω̅} & 7 &  &  \\
&  & 8 & \foreignlanguage{greek}{δαιμονιων εκβαλλει τα δαιμονια} & 11 &  &  \\
& \textbf{35} &  & \foreignlanguage{greek}{και περιηγεν ο \textoverline{ιϲ} ταϲ πολειϲ παϲαϲ και} & 8 &  &  \\
&  & 9 & \foreignlanguage{greek}{ταϲ κωμαϲ διδαϲκων εν ταιϲ ϲυν} & 14 &  &  \\
&  & 14 & \foreignlanguage{greek}{αγωγαιϲ αυτων και κηρυϲϲων το} & 18 &  &  \\
&  & 19 & \foreignlanguage{greek}{ευαγγελιον τηϲ βαϲιλειαϲ και θερα} & 23 &  &  \\
&  & 23 & \foreignlanguage{greek}{πευων παϲαν νοϲον και παϲαν μα} & 28 &  &  \\
&  & 28 & \foreignlanguage{greek}{λακιαν ιδων δε τουϲ οχλουϲ εϲπλαγ} & 5 & \textbf{36} &  \\
&  & 5 & \foreignlanguage{greek}{χνιϲθη περι αυτων οτι ηϲαν εϲκυλ} & 10 &  &  \\
&  & 10 & \foreignlanguage{greek}{μενοι και ερριμμενοι ωϲει προβα} & 14 &  &  \\
&  & 14 & \foreignlanguage{greek}{τα μη εχοντα ποιμενα} & 17 &  &  \\
& \textbf{37} &  & \foreignlanguage{greek}{τοτε λεγει τοιϲ μαθηταιϲ αυτου ο με̅} & 7 &  &  \\
&  & 8 & \foreignlanguage{greek}{θεριϲμοϲ πολυϲ οι δε εργατε ολειγοι} & 13 &  &  \\
& \textbf{38} &  & \foreignlanguage{greek}{δεηθηται ουν του \textoverline{κυ} του θεριϲμου} & 6 &  &  \\
&  & 7 & \foreignlanguage{greek}{οπωϲ εκβαλη εργαταϲ ειϲ τον θεριϲμο̅} & 12 &  &  \\
&  & 13 & \foreignlanguage{greek}{αυτου και προϲκαλεϲαμενοϲ} & 2 & \mygospelchapter &  \\
&  & 3 & \foreignlanguage{greek}{τουϲ δωδεκα μαθηταϲ αυτου εδω} & 7 &  &  \\
&  & 7 & \foreignlanguage{greek}{κεν αυτοιϲ εξουϲιαν πνευματων} & 10 &  &  \\
&  & 11 & \foreignlanguage{greek}{ακαθαρτων ωϲτε εκβαλλιν αυτα} & 14 &  &  \\
[0.2em]
\cline{4-4}
\end{tabular}
\end{center}
\end{table}
}
\clearpage
\newpage
 {
 \setlength\arrayrulewidth{1pt}
\begin{table}
\begin{center}
\begin{tabular}{ccc|l|ccc}
\cline{4-4} \\ [-1em]
\multicolumn{7}{c}{\foreignlanguage{greek}{ευαγγελιον κατα μαθθαιον} \textbf{(\nospace{10:1})} } \\ \\ [-1em] % Si on veut ajouter les bordures latérales, remplacer {7}{c} par {7}{|c|}
\cline{4-4} \\
\cline{4-4}
&  &  & &  &  & \\ [-0.9em]
&  & 15 & \foreignlanguage{greek}{και θεραπευειν παϲαν νοϲον και παϲαν} & 20 &  &  \\
&  & 21 & \foreignlanguage{greek}{μαλακειαν των δε δωδεκα αποϲτο} & 4 & \textbf{2} &  \\
&  & 4 & \foreignlanguage{greek}{λων τα ονοματα εϲτιν ταυτα} & 8 &  &  \\
&  & 9 & \foreignlanguage{greek}{πρωτοϲ ϲιμων ο λεγομενοϲ πετροϲ} & 13 &  &  \\
&  & 14 & \foreignlanguage{greek}{και ανδρεαϲ ο αδελφοϲ αυτου} & 18 &  &  \\
&  & 19 & \foreignlanguage{greek}{ιακωβοϲ ο του ζεβαιδεου και ιωαν} & 24 &  &  \\
&  & 24 & \foreignlanguage{greek}{νηϲ ο αδελφοϲ αυτου φιλιπποϲ και} & 2 & \textbf{3} &  \\
&  & 3 & \foreignlanguage{greek}{βαρθολομαιοϲ θωμαϲ και ματθαιοϲ} & 6 &  &  \\
&  & 7 & \foreignlanguage{greek}{ο τελωνηϲ ιακωβοϲ ο του αλφαιου} & 12 &  &  \\
&  & 13 & \foreignlanguage{greek}{και λεββαιοϲ ο επικληθειϲ θαδδαιοϲ} & 17 &  &  \\
& \textbf{4} &  & \foreignlanguage{greek}{ϲιμων ο κανανιτηϲ και ιουδαϲ ιϲκα} & 6 &  &  \\
&  & 6 & \foreignlanguage{greek}{ριωτηϲ ο και παραδουϲ αυτον} & 10 &  &  \\
& \textbf{5} &  & \foreignlanguage{greek}{τουτουϲ τουϲ δωδεκα εξαπεϲτιλεν} & 4 &  &  \\
&  & 5 & \foreignlanguage{greek}{ο \textoverline{ιϲ} παραγγειλαϲ αυτοιϲ λεγων} & 9 &  &  \\
&  & 10 & \foreignlanguage{greek}{ειϲ οδον εθνων μη απελθηται και} & 15 &  &  \\
&  & 16 & \foreignlanguage{greek}{ειϲ πολιν ϲαμαριτων μη ειϲελθηται} & 20 &  &  \\
& \textbf{6} &  & \foreignlanguage{greek}{πορευεϲθαι δε μαλλον προϲ τα προβα} & 6 &  &  \\
&  & 6 & \foreignlanguage{greek}{τα τα απολωλοτα οικου ιϲραηλ} & 10 &  &  \\
& \textbf{7} &  & \foreignlanguage{greek}{πορευομενοι δε κηρυϲϲεται λεγοντεϲ} & 4 &  &  \\
&  & 5 & \foreignlanguage{greek}{οτι ηγγεικεν η βαϲιλεια των ουρανω̅} & 10 &  &  \\
& \textbf{8} &  & \foreignlanguage{greek}{αϲθενουνταϲ θεραπευεται λεπρουϲ} & 3 &  &  \\
&  & 4 & \foreignlanguage{greek}{καθαριζεται δαιμονια εκβαλλε} & 6 &  &  \\
&  & 6 & \foreignlanguage{greek}{ται νεκρουϲ εγειρεται δωραιαν} & 9 &  &  \\
&  & 10 & \foreignlanguage{greek}{ελαβεται δωραιαν δοται} & 12 &  &  \\
& \textbf{9} &  & \foreignlanguage{greek}{μη κτηϲηϲθαι χρυϲον μηδε αργυρο̅} & 5 &  &  \\
&  & 6 & \foreignlanguage{greek}{μηδε χαλκον ειϲ ταϲ ζωναϲ υμων} & 11 &  &  \\
& \textbf{10} &  & \foreignlanguage{greek}{μη πηραν ειϲ οδον μηδε δυο χιτω} & 7 &  &  \\
&  & 7 & \foreignlanguage{greek}{ναϲ μηδε υποδηματα μηδε ραβδουϲ} & 11 &  &  \\
&  & 12 & \foreignlanguage{greek}{αξιοϲ γαρ ο εργατηϲ τηϲ τροφηϲ αυτου} & 18 &  &  \\
&  & 19 & \foreignlanguage{greek}{εϲτιν ειϲ ην δ αν πολιν η κωμη̅} & 7 & \textbf{11} &  \\
[0.2em]
\cline{4-4}
\end{tabular}
\end{center}
\end{table}
}
\clearpage
\newpage
 {
 \setlength\arrayrulewidth{1pt}
\begin{table}
\begin{center}
\begin{tabular}{ccc|l|ccc}
\cline{4-4} \\ [-1em]
\multicolumn{7}{c}{\foreignlanguage{greek}{ευαγγελιον κατα μαθθαιον} \textbf{(\nospace{10:11})} } \\ \\ [-1em] % Si on veut ajouter les bordures latérales, remplacer {7}{c} par {7}{|c|}
\cline{4-4} \\
\cline{4-4}
&  &  & &  &  & \\ [-0.9em]
&  & 8 & \foreignlanguage{greek}{ειϲελθηται εξεταϲατε τιϲ εν αυτη α} & 13 &  &  \\
&  & 13 & \foreignlanguage{greek}{ξιοϲ εϲτιν κακει μειναται εωϲ αν εξελ} & 19 &  &  \\
&  & 19 & \foreignlanguage{greek}{θηται ειϲερχομενοι δε ειϲ την οικει} & 5 & \textbf{12} &  \\
&  & 5 & \foreignlanguage{greek}{αν αϲπαϲαϲθαι αυτην λεγοντεϲ} & 8 &  &  \\
&  & 9 & \foreignlanguage{greek}{ειρηνη τω οικω τουτω και εαν με̅} & 3 & \textbf{13} &  \\
&  & 4 & \foreignlanguage{greek}{η η οικεια αξια ελθατω η ειρηνη υμω̅} & 11 &  &  \\
&  & 12 & \foreignlanguage{greek}{επ αυτην εαν δε μη η αξια η ειρηνη} & 20 &  &  \\
&  & 21 & \foreignlanguage{greek}{υμων εφ υμαϲ επιϲτραφητω} & 24 &  &  \\
& \textbf{14} &  & \foreignlanguage{greek}{και οϲ αν μη δεξηται υμαϲ μηδε ακου} & 8 &  &  \\
&  & 8 & \foreignlanguage{greek}{ϲη τουϲ λογουϲ υμων εξερχομενοι} & 14 &  &  \\
&  & 15 & \foreignlanguage{greek}{τηϲ οικειαϲ η τηϲ πολεωϲ εκεινηϲ} & 20 &  &  \\
&  & 21 & \foreignlanguage{greek}{εκτιναξαται τον κονιορτον των πο} & 25 &  &  \\
&  & 25 & \foreignlanguage{greek}{δων υμων} & 26 &  &  \\
& \textbf{15} &  & \foreignlanguage{greek}{αμην λεγω υμιν ανεκτοτερον εϲται} & 5 &  &  \\
&  & 6 & \foreignlanguage{greek}{γη ϲοδομων και γομορων εν ημερα} & 11 &  &  \\
&  & 12 & \foreignlanguage{greek}{κριϲεωϲ η τη πολει εκεινη} & 16 &  &  \\
& \textbf{16} &  & \foreignlanguage{greek}{ιδου εγω αποϲτελλω υμαϲ ωϲ προβα} & 6 &  &  \\
&  & 6 & \foreignlanguage{greek}{τα εν μεϲω λυκων γινεϲθαι ουν} & 11 &  &  \\
&  & 12 & \foreignlanguage{greek}{φρονιμοι ωϲ οι οφειϲ και ακεραιοι ωϲ} & 18 &  &  \\
&  & 19 & \foreignlanguage{greek}{αι περιϲτεραι προϲεχεται δε απο} & 3 & \textbf{17} &  \\
&  & 4 & \foreignlanguage{greek}{των \textoverline{ανων} παραδωϲωϲιν γαρ υμαϲ} & 8 &  &  \\
&  & 9 & \foreignlanguage{greek}{ειϲ ϲυνεδρια και εν ταιϲ ϲυναγωγαιϲ} & 14 &  &  \\
&  & 15 & \foreignlanguage{greek}{μαϲτιγωϲουϲιν υμαϲ και επι η} & 3 & \textbf{18} &  \\
&  & 3 & \foreignlanguage{greek}{γεμοναϲ δε και βαϲιλειϲ αχθηϲεϲθε} & 7 &  &  \\
&  & 8 & \foreignlanguage{greek}{ενεκεν εμου ειϲ μαρτυριον αυτοιϲ} & 12 &  &  \\
&  & 13 & \foreignlanguage{greek}{και τοιϲ εθνεϲιν} & 15 &  &  \\
& \textbf{19} &  & \foreignlanguage{greek}{οταν δε παραδωϲουϲιν υμαϲ μη με} & 6 &  &  \\
&  & 6 & \foreignlanguage{greek}{ριμνηϲηται πωϲ η τι λαληϲηται} & 10 &  &  \\
&  & 11 & \foreignlanguage{greek}{δοθηϲεται γαρ υμιν εν εκεινη τη} & 16 &  &  \\
&  & 17 & \foreignlanguage{greek}{ωρα τι λαληϲεται ου γαρ υμειϲ εϲται} & 4 & \textbf{20} &  \\
[0.2em]
\cline{4-4}
\end{tabular}
\end{center}
\end{table}
}
\clearpage
\newpage
 {
 \setlength\arrayrulewidth{1pt}
\begin{table}
\begin{center}
\begin{tabular}{ccc|l|ccc}
\cline{4-4} \\ [-1em]
\multicolumn{7}{c}{\foreignlanguage{greek}{ευαγγελιον κατα μαθθαιον} \textbf{(\nospace{10:20})} } \\ \\ [-1em] % Si on veut ajouter les bordures latérales, remplacer {7}{c} par {7}{|c|}
\cline{4-4} \\
\cline{4-4}
&  &  & &  &  & \\ [-0.9em]
&  & 5 & \foreignlanguage{greek}{οι λαλουντεϲ αλλα το \textoverline{πνα} του \textoverline{πρϲ} υ} & 12 &  &  \\
&  & 12 & \foreignlanguage{greek}{μων το λαλουν εν υμιν} & 16 &  &  \\
& \textbf{21} &  & \foreignlanguage{greek}{παραδωϲει δε αδελφοϲ αδελφον ειϲ} & 5 &  &  \\
&  & 6 & \foreignlanguage{greek}{θανατον και \textoverline{πηρ} τεκνα και επανα} & 11 &  &  \\
&  & 11 & \foreignlanguage{greek}{ϲτηϲονται τεκνα επι γονειϲ και θανα} & 16 &  &  \\
&  & 16 & \foreignlanguage{greek}{τωϲουϲιν αυτουϲ και εϲεϲθαι μιϲου} & 3 & \textbf{22} &  \\
&  & 3 & \foreignlanguage{greek}{μενοι υπο παντων δια το ονομα μου} & 9 &  &  \\
&  & 10 & \foreignlanguage{greek}{ο δε υπομειναϲ ειϲ τελοϲ ϲωθηϲεται} & 15 &  &  \\
& \textbf{23} &  & \foreignlanguage{greek}{οταν δε διωκωϲιν υμαϲ εν τη πολει} & 7 &  &  \\
&  & 8 & \foreignlanguage{greek}{ταυτη φευγεται ειϲ την ετεραν} & 12 &  &  \\
&  & 13 & \foreignlanguage{greek}{αμην γαρ λεγω υμιν ου μη τελεϲηται} & 19 &  &  \\
&  & 20 & \foreignlanguage{greek}{ταϲ πολειϲ του ιϲραηλ εωϲ αν ελθη ο υ} & 28 &  &  \\
&  & 28 & \foreignlanguage{greek}{ιοϲ του \textoverline{ανου}} & 30 &  &  \\
& \textbf{24} &  & \foreignlanguage{greek}{ουκ εϲτιν μαθητηϲ υπερ τον διδαϲκα} & 6 &  &  \\
&  & 6 & \foreignlanguage{greek}{λον αυτου ουδε δουλοϲ υπερ τον \textoverline{κν}} & 12 &  &  \\
&  & 13 & \foreignlanguage{greek}{αυτου αρκετον τω μαθητη ινα γενη} & 5 & \textbf{25} &  \\
&  & 5 & \foreignlanguage{greek}{ται ωϲ ο διδαϲκαλοϲ αυτου και ο δου} & 12 &  &  \\
&  & 12 & \foreignlanguage{greek}{λοϲ ωϲ ο κυριοϲ αυτου} & 16 &  &  \\
&  & 17 & \foreignlanguage{greek}{ει τον οικοδεϲποτην βεελζεβουλ επε} & 21 &  &  \\
&  & 21 & \foreignlanguage{greek}{καλεϲαν ποϲω μαλλον τουϲ οικεια} & 25 &  &  \\
&  & 25 & \foreignlanguage{greek}{κουϲ αυτου μη ουν φοβηθηται αυτουϲ} & 4 & \textbf{26} &  \\
&  & 5 & \foreignlanguage{greek}{ουδεν γαρ εϲτιν κεκαλυμμενον ο ου} & 10 &  &  \\
&  & 10 & \foreignlanguage{greek}{κ αποκαλυφθηϲεται και κρυπτον} & 13 &  &  \\
&  & 14 & \foreignlanguage{greek}{ο ου γνωϲθηϲεται ο λεγω υμιν εν τη} & 5 & \textbf{27} &  \\
&  & 6 & \foreignlanguage{greek}{ϲκοτεια ειπατε εν τω φωτι και ο ειϲ} & 13 &  &  \\
&  & 14 & \foreignlanguage{greek}{το ουϲ ακουεται κηρυξαται επι των} & 19 &  &  \\
&  & 20 & \foreignlanguage{greek}{δωματων και μη φοβηθηται απο} & 4 & \textbf{28} &  \\
&  & 5 & \foreignlanguage{greek}{των αποκτεννοντων το ϲωμα την} & 9 &  &  \\
&  & 10 & \foreignlanguage{greek}{δε ψυχην μη δυναμενων αποκτιναι} & 14 &  &  \\
&  & 15 & \foreignlanguage{greek}{φοβειϲθαι δε μαλλον τον δυναμενο̅} & 19 &  &  \\
[0.2em]
\cline{4-4}
\end{tabular}
\end{center}
\end{table}
}
\clearpage
\newpage
 {
 \setlength\arrayrulewidth{1pt}
\begin{table}
\begin{center}
\begin{tabular}{ccc|l|ccc}
\cline{4-4} \\ [-1em]
\multicolumn{7}{c}{\foreignlanguage{greek}{ευαγγελιον κατα μαθθαιον} \textbf{(\nospace{10:28})} } \\ \\ [-1em] % Si on veut ajouter les bordures latérales, remplacer {7}{c} par {7}{|c|}
\cline{4-4} \\
\cline{4-4}
&  &  & &  &  & \\ [-0.9em]
&  & 20 & \foreignlanguage{greek}{και την ψυχην και το ϲωμα απολεϲαι} & 26 &  &  \\
&  & 27 & \foreignlanguage{greek}{εν γεεννη ουχι δυο ϲτρουθια αϲϲα} & 4 & \textbf{29} &  \\
&  & 4 & \foreignlanguage{greek}{ριου πωλειται και εν εξ αυτων ου} & 10 &  &  \\
&  & 11 & \foreignlanguage{greek}{πεϲειται επι την γην ανευ του \textoverline{προϲ}} & 17 &  &  \\
&  & 18 & \foreignlanguage{greek}{υμων υμων δε και αι τριχεϲ τηϲ} & 6 & \textbf{30} &  \\
&  & 7 & \foreignlanguage{greek}{κεφαληϲ παϲαι ηριθμημεναι ειϲι̅} & 10 &  &  \\
& \textbf{31} &  & \foreignlanguage{greek}{μη ουν φοβειϲθαι αυτουϲ πολλων} & 5 &  &  \\
&  & 6 & \foreignlanguage{greek}{ϲτρουθιων διαφερεται υμειϲ} & 8 &  &  \\
& \textbf{32} &  & \foreignlanguage{greek}{παϲ ουν οϲτιϲ ομολογηϲει εν εμοι} & 6 &  &  \\
&  & 7 & \foreignlanguage{greek}{εμπροϲθεν των \textoverline{ανων} ομολογηϲω} & 10 &  &  \\
&  & 11 & \foreignlanguage{greek}{καγω εν αυτω εμπροϲθεν του \textoverline{προϲ}} & 16 &  &  \\
&  & 17 & \foreignlanguage{greek}{μου του εν ουρανοιϲ και οϲτιϲ αρ} & 3 & \textbf{33} &  \\
&  & 3 & \foreignlanguage{greek}{νηϲηται με εμπροϲθεν των \textoverline{ανων}} & 7 &  &  \\
&  & 8 & \foreignlanguage{greek}{αρνηϲομαι καγω αυτον εμπροϲθεν} & 11 &  &  \\
&  & 12 & \foreignlanguage{greek}{του \textoverline{πρϲ} μου του εν ουρανοιϲ} & 17 &  &  \\
& \textbf{34} &  & \foreignlanguage{greek}{μη νομειϲηται οτι ηλθον βαλιν ειρη} & 6 &  &  \\
&  & 6 & \foreignlanguage{greek}{νην επι την γην ουκ ηλθον βαλιν} & 12 &  &  \\
&  & 13 & \foreignlanguage{greek}{ειρηνην αλλα μαχαιραν} & 15 &  &  \\
& \textbf{35} &  & \foreignlanguage{greek}{ηλθον γαρ διχαϲαι ανθρωπον κατα} & 5 &  &  \\
&  & 6 & \foreignlanguage{greek}{του \textoverline{πρϲ} αυτου και θυγατερα κατα τηϲ} & 12 &  &  \\
&  & 13 & \foreignlanguage{greek}{μητροϲ αυτηϲ και νυμφην κατα} & 17 &  &  \\
&  & 18 & \foreignlanguage{greek}{τηϲ πενθεραϲ αυτηϲ και εχθροι} & 2 & \textbf{36} &  \\
&  & 3 & \foreignlanguage{greek}{του \textoverline{ανου} οι οικειακοι αυτου} & 7 &  &  \\
& \textbf{37} &  & \foreignlanguage{greek}{ο φιλων \textoverline{πρα} η μητερα υπερ εμε ουκ ε} & 9 &  &  \\
&  & 9 & \foreignlanguage{greek}{ϲτιν μου αξιοϲ και ο φιλων υιον η} & 16 &  &  \\
&  & 17 & \foreignlanguage{greek}{θυγατερα υπερ εμε ουκ εϲτιν μου} & 22 &  &  \\
&  & 23 & \foreignlanguage{greek}{αξιοϲ και οϲ ου λαμβανει τον ϲταυ} & 6 & \textbf{38} &  \\
&  & 6 & \foreignlanguage{greek}{ρον αυτου και ακολουθει οπιϲω μου} & 11 &  &  \\
&  & 12 & \foreignlanguage{greek}{ουκ εϲτιν μου αξιοϲ} & 15 &  &  \\
& \textbf{39} &  & \foreignlanguage{greek}{ο ευρων την ψυχην αυτου απολεϲει} & 6 &  &  \\
[0.2em]
\cline{4-4}
\end{tabular}
\end{center}
\end{table}
}
\clearpage
\newpage
 {
 \setlength\arrayrulewidth{1pt}
\begin{table}
\begin{center}
\begin{tabular}{ccc|l|ccc}
\cline{4-4} \\ [-1em]
\multicolumn{7}{c}{\foreignlanguage{greek}{ευαγγελιον κατα μαθθαιον} \textbf{(\nospace{10:39})} } \\ \\ [-1em] % Si on veut ajouter les bordures latérales, remplacer {7}{c} par {7}{|c|}
\cline{4-4} \\
\cline{4-4}
&  &  & &  &  & \\ [-0.9em]
&  & 7 & \foreignlanguage{greek}{αυτην και ο απολεϲαϲ την ψυχην αυ} & 13 &  &  \\
&  & 13 & \foreignlanguage{greek}{του ενεκεν εμου ευρηϲει αυτην} & 17 &  &  \\
& \textbf{40} &  & \foreignlanguage{greek}{ο δεχομενοϲ υμαϲ εμε δεχεται και ο} & 7 &  &  \\
&  & 8 & \foreignlanguage{greek}{εμε δεχομενοϲ δεχεται τον αποϲτι} & 12 &  &  \\
&  & 12 & \foreignlanguage{greek}{λοντα με ο δεχομενοϲ προφη} & 3 & \textbf{41} &  \\
&  & 3 & \foreignlanguage{greek}{την ειϲ ονομα προφητου μιϲθον προ} & 8 &  &  \\
&  & 8 & \foreignlanguage{greek}{φητου λημψεται και ο δεχομενοϲ} & 12 &  &  \\
&  & 13 & \foreignlanguage{greek}{δικαιον ειϲ ονομα δικαιου μιϲθον} & 17 &  &  \\
&  & 18 & \foreignlanguage{greek}{δικαιου λημψεται και οϲ εαν πο} & 4 & \textbf{42} &  \\
&  & 4 & \foreignlanguage{greek}{τιϲη ενα των μικρων τουτων ποτη} & 9 &  &  \\
&  & 9 & \foreignlanguage{greek}{ριον ψυχρου μονον ειϲ ονομα μαθη} & 14 &  &  \\
&  & 14 & \foreignlanguage{greek}{του αμην λεγω υμιν ου μη απολε} & 20 &  &  \\
&  & 20 & \foreignlanguage{greek}{ϲη τον μιϲθον αυτου} & 23 &  &  \\
& \mygospelchapter &  & \foreignlanguage{greek}{και εγενετο οτε ετελεϲεν ο \textoverline{ιϲ} διαταϲ} & 7 &  &  \\
&  & 7 & \foreignlanguage{greek}{ϲων τοιϲ δωδεκα μαθηταιϲ αυτου} & 11 &  &  \\
&  & 12 & \foreignlanguage{greek}{μετεβη εκειθεν του διδαϲκειν και} & 16 &  &  \\
&  & 17 & \foreignlanguage{greek}{κηρυϲϲιν εν ταιϲ πολεϲιν αυτων} & 21 &  &  \\
& \textbf{2} &  & \foreignlanguage{greek}{ο δε ιωαννηϲ ακουϲαϲ εν τω δεϲμω} & 7 &  &  \\
&  & 7 & \foreignlanguage{greek}{τηριω τα εργα του \textoverline{χυ} πεμψαϲ δια} & 13 &  &  \\
&  & 14 & \foreignlanguage{greek}{των μαθητων αυτου ειπεν αυτω} & 2 & \textbf{3} &  \\
&  & 3 & \foreignlanguage{greek}{ϲυ ει ο ερχομενοϲ η ετερον προϲδο} & 9 &  &  \\
&  & 9 & \foreignlanguage{greek}{κωμεν και αποκριθειϲ ο \textoverline{ιϲ} ει} & 5 & \textbf{4} &  \\
&  & 5 & \foreignlanguage{greek}{πεν αυτοιϲ πορευθεντεϲ απαγγει} & 8 &  &  \\
&  & 8 & \foreignlanguage{greek}{λατε ιωαννει α ακουεται και βλε} & 13 &  &  \\
&  & 13 & \foreignlanguage{greek}{πεται τυφλοι αναβλεπουϲιν} & 2 & \textbf{5} &  \\
&  & 3 & \foreignlanguage{greek}{και χωλοι περιπατουϲιν λεπροι κα} & 7 &  &  \\
&  & 7 & \foreignlanguage{greek}{θαριζονται και κωφοι ακουουϲιν} & 10 &  &  \\
&  & 11 & \foreignlanguage{greek}{και νεκροι εγειρονται και πτωχοι} & 15 &  &  \\
&  & 16 & \foreignlanguage{greek}{ευαγγελιζονται και μακαριοϲ ε} & 3 & \textbf{6} &  \\
&  & 3 & \foreignlanguage{greek}{ϲτιν οϲ εαν μη ϲκανδαλιϲθη εν εμοι} & 9 &  &  \\
[0.2em]
\cline{4-4}
\end{tabular}
\end{center}
\end{table}
}
\clearpage
\newpage
 {
 \setlength\arrayrulewidth{1pt}
\begin{table}
\begin{center}
\begin{tabular}{ccc|l|ccc}
\cline{4-4} \\ [-1em]
\multicolumn{7}{c}{\foreignlanguage{greek}{ευαγγελιον κατα μαθθαιον} \textbf{(\nospace{11:7})} } \\ \\ [-1em] % Si on veut ajouter les bordures latérales, remplacer {7}{c} par {7}{|c|}
\cline{4-4} \\
\cline{4-4}
&  &  & &  &  & \\ [-0.9em]
& \textbf{7} &  & \foreignlanguage{greek}{τουτων δε πορευομενων ηρξατο ο \textoverline{ιϲ} λεγει̅} & 7 &  &  \\
&  & 8 & \foreignlanguage{greek}{τοιϲ οχλοιϲ περι ιωαννου τι εξηλθα} & 13 &  &  \\
&  & 13 & \foreignlanguage{greek}{τε ειϲ την ερημον θεαϲαϲθαι καλα} & 18 &  &  \\
&  & 18 & \foreignlanguage{greek}{μον υπο ανεμου ϲαλευομενον} & 21 &  &  \\
& \textbf{8} &  & \foreignlanguage{greek}{αλλα τι εξηλθατε ειδειν \textoverline{ανον} εν} & 6 &  &  \\
&  & 7 & \foreignlanguage{greek}{μαλακοιϲ ιματιοιϲ ημφιεϲμενον} & 9 &  &  \\
&  & 10 & \foreignlanguage{greek}{ιδου οι τα μαλακα φορουντεϲ εν τοιϲ} & 16 &  &  \\
&  & 17 & \foreignlanguage{greek}{οικοιϲ των βαϲιλεων ειϲιν αλλα} & 1 & \textbf{9} &  \\
&  & 2 & \foreignlanguage{greek}{τι εξεληλυθατε προφητην ιδειν} & 5 &  &  \\
&  & 6 & \foreignlanguage{greek}{ναι λεγω υμιν και περιϲϲοτερον προ} & 11 &  &  \\
&  & 11 & \foreignlanguage{greek}{φητου ουτοϲ γαρ εϲτιν περι ου γε} & 6 & \textbf{10} &  \\
&  & 6 & \foreignlanguage{greek}{γραπται ιδου εγω αποϲτελλω τον} & 10 &  &  \\
&  & 11 & \foreignlanguage{greek}{αγγελον μου προ προϲωπου ϲου οϲ} & 16 &  &  \\
&  & 17 & \foreignlanguage{greek}{καταϲκευαϲει την οδον ϲου εμ} & 21 &  &  \\
&  & 21 & \foreignlanguage{greek}{προϲθεν ϲου} & 22 &  &  \\
& \textbf{11} &  & \foreignlanguage{greek}{αμην λεγω υμιν ουκ εγηγερται εν γε} & 7 &  &  \\
&  & 7 & \foreignlanguage{greek}{νητοιϲ γυναικων μιζων ιωαννου} & 10 &  &  \\
&  & 11 & \foreignlanguage{greek}{του βαπτιϲτου ο δε μεικροτεροϲ} & 15 &  &  \\
&  & 16 & \foreignlanguage{greek}{εν τη βαϲιλεια των ουρανων μιζω̅} & 21 &  &  \\
&  & 22 & \foreignlanguage{greek}{εϲτιν αυτου} & 23 &  &  \\
& \textbf{12} &  & \foreignlanguage{greek}{απο δε των ημερων ιωαννου του βα} & 7 &  &  \\
&  & 7 & \foreignlanguage{greek}{πτιϲτου εωϲ αρτι η βαϲιλεια των ου} & 13 &  &  \\
&  & 13 & \foreignlanguage{greek}{ρανων βιαζεται και βιαϲται αρπα} & 17 &  &  \\
&  & 17 & \foreignlanguage{greek}{ζουϲιν αυτην παντεϲ γαρ οι προ} & 4 & \textbf{13} &  \\
&  & 4 & \foreignlanguage{greek}{φηται και ο νομοϲ εωϲ ιωαννου προ} & 10 &  &  \\
&  & 10 & \foreignlanguage{greek}{εφητευϲαν και ει θελεται δεξα} & 4 & \textbf{14} &  \\
&  & 4 & \foreignlanguage{greek}{ϲθαι αυτοϲ εϲτιν ηλιαϲ ο μελλων ερ} & 10 &  &  \\
&  & 10 & \foreignlanguage{greek}{χεϲθαι ο εχων ωτα ακουειν ακου} & 5 & \textbf{15} &  \\
&  & 5 & \foreignlanguage{greek}{ετω} & 5 &  &  \\
& \textbf{16} &  & \foreignlanguage{greek}{τινι δε ομοιωϲω την γενεαν ταυτην} & 6 &  &  \\
[0.2em]
\cline{4-4}
\end{tabular}
\end{center}
\end{table}
}
\clearpage
\newpage
 {
 \setlength\arrayrulewidth{1pt}
\begin{table}
\begin{center}
\begin{tabular}{ccc|l|ccc}
\cline{4-4} \\ [-1em]
\multicolumn{7}{c}{\foreignlanguage{greek}{ευαγγελιον κατα μαθθαιον} \textbf{(\nospace{11:16})} } \\ \\ [-1em] % Si on veut ajouter les bordures latérales, remplacer {7}{c} par {7}{|c|}
\cline{4-4} \\
\cline{4-4}
&  &  & &  &  & \\ [-0.9em]
&  & 7 & \foreignlanguage{greek}{ομοια εϲτιν παιδιοιϲ καθημενοιϲ εν} & 11 &  &  \\
&  & 12 & \foreignlanguage{greek}{αγοραιϲ και προϲφωνουϲιν τοιϲ ετε} & 16 &  &  \\
&  & 16 & \foreignlanguage{greek}{ροιϲ αυτων και λεγουϲιν ηυληϲαμε̅} & 3 & \textbf{17} &  \\
&  & 4 & \foreignlanguage{greek}{υμιν και ουκ ωρχηϲαϲθαι εθρηνηϲα} & 8 &  &  \\
&  & 8 & \foreignlanguage{greek}{μεν υμιν και ουκ εκλαυϲαϲθαι} & 12 &  &  \\
& \textbf{18} &  & \foreignlanguage{greek}{ηλθεν γαρ ιωαννηϲ μητε εϲθιων μη} & 6 &  &  \\
&  & 6 & \foreignlanguage{greek}{τε πινων και λεγουϲιν δαιμονιον εχει} & 11 &  &  \\
& \textbf{19} &  & \foreignlanguage{greek}{ηλθεν ο υιοϲ του \textoverline{ανου} εϲθιων και πινω̅} & 8 &  &  \\
&  & 9 & \foreignlanguage{greek}{και λεγουϲιν ιδου \textoverline{ανοϲ} φαγοϲ και οι} & 15 &  &  \\
&  & 15 & \foreignlanguage{greek}{νοποτηϲ τελωνων φιλοϲ και αμαρ} & 19 &  &  \\
&  & 19 & \foreignlanguage{greek}{τωλων και εδικαιωθη η ϲοφια απο} & 24 &  &  \\
&  & 25 & \foreignlanguage{greek}{των εργων αυτηϲ} & 27 &  &  \\
& \textbf{20} &  & \foreignlanguage{greek}{τοτε ηρξατο ο \textoverline{ιϲ} ονιδιζειν ταϲ πολειϲ} & 7 &  &  \\
&  & 8 & \foreignlanguage{greek}{εν αιϲ εγενοντο αι πλειϲται δυναμειϲ} & 13 &  &  \\
&  & 14 & \foreignlanguage{greek}{αυτου οτι ου μετενοηϲαν} & 17 &  &  \\
& \textbf{21} &  & \foreignlanguage{greek}{ουαι ϲοι χοραζειν ουαι ϲοι βηθϲαιδα̅} & 6 &  &  \\
&  & 7 & \foreignlanguage{greek}{οτι ει εν τυρω και ϲιδονι εγενοντο αι} & 14 &  &  \\
&  & 15 & \foreignlanguage{greek}{δυναμειϲ αι γενομεναι εν υμιν} & 19 &  &  \\
&  & 20 & \foreignlanguage{greek}{παλαι αν εν ϲακκω και ϲποδω μετε} & 26 &  &  \\
&  & 26 & \foreignlanguage{greek}{νοηϲαν πλην λεγω υμιν τυρω και} & 5 & \textbf{22} &  \\
&  & 6 & \foreignlanguage{greek}{ϲιδονει ανεκτοτερον εϲται εν η} & 10 &  &  \\
&  & 10 & \foreignlanguage{greek}{μερα κριϲεωϲ η υμιν} & 13 &  &  \\
& \textbf{23} &  & \foreignlanguage{greek}{και ϲυ καπερναουμ μη εωϲ ουρανου} & 6 &  &  \\
&  & 7 & \foreignlanguage{greek}{υψωθηϲη εωϲ αδου καταβηϲη} & 10 &  &  \\
&  & 11 & \foreignlanguage{greek}{οτι ει εν ϲοδομοιϲ εγενοντο αι δυνα} & 17 &  &  \\
&  & 17 & \foreignlanguage{greek}{μειϲ αι γενομεναι εν ϲοι εμεινον α̅} & 23 &  &  \\
&  & 24 & \foreignlanguage{greek}{μεχρι τηϲ ϲημερον πλην λεγω υ} & 3 & \textbf{24} &  \\
&  & 3 & \foreignlanguage{greek}{μιν οτι γη ϲοδομων ανεκτοτερον} & 7 &  &  \\
&  & 8 & \foreignlanguage{greek}{εϲται εν ημερα κριϲεωϲ η ϲοι} & 13 &  &  \\
& \textbf{25} &  & \foreignlanguage{greek}{εν εκεινω τω καιρω αποκριθειϲ ο \textoverline{ιϲ} ειπε̅} & 8 &  &  \\
[0.2em]
\cline{4-4}
\end{tabular}
\end{center}
\end{table}
}
\clearpage
\newpage
 {
 \setlength\arrayrulewidth{1pt}
\begin{table}
\begin{center}
\begin{tabular}{ccc|l|ccc}
\cline{4-4} \\ [-1em]
\multicolumn{7}{c}{\foreignlanguage{greek}{ευαγγελιον κατα μαθθαιον} \textbf{(\nospace{11:25})} } \\ \\ [-1em] % Si on veut ajouter les bordures latérales, remplacer {7}{c} par {7}{|c|}
\cline{4-4} \\
\cline{4-4}
&  &  & &  &  & \\ [-0.9em]
&  & 9 & \foreignlanguage{greek}{εξομολογουμαι ϲοι πατερ \textoverline{κε} του ουρα} & 14 &  &  \\
&  & 14 & \foreignlanguage{greek}{νου και τηϲ γηϲ οτι απεκρυψαϲ ταυ} & 20 &  &  \\
&  & 20 & \foreignlanguage{greek}{τα απο ϲοφων και ϲυνετων και απε} & 26 &  &  \\
&  & 26 & \foreignlanguage{greek}{καλυψαϲ αυτα νηπιοιϲ ναι ο \textoverline{πηρ}} & 3 & \textbf{26} &  \\
&  & 4 & \foreignlanguage{greek}{οτι ουτωϲ ευδοκεια εγενετο εμπρο} & 8 &  &  \\
&  & 8 & \foreignlanguage{greek}{ϲθεν ϲου παντα μοι παρεδοθη υ} & 4 & \textbf{27} &  \\
&  & 4 & \foreignlanguage{greek}{πο του \textoverline{πρϲ} μου και ουδειϲ επιγι} & 10 &  &  \\
&  & 10 & \foreignlanguage{greek}{γνωϲκει τον υιον ει μη ο \textoverline{πηρ} ουδε} & 17 &  &  \\
&  & 18 & \foreignlanguage{greek}{τον \textoverline{πρα} τιϲ επιγιγνωϲκει ει μη ο υιοϲ} & 25 &  &  \\
&  & 26 & \foreignlanguage{greek}{και ω εαν βουλεται ο υιοϲ αποκαλυψαι} & 32 &  &  \\
& \textbf{28} &  & \foreignlanguage{greek}{δευτε προϲ με παντεϲ οι κοπιωντεϲ} & 6 &  &  \\
&  & 7 & \foreignlanguage{greek}{και πεφορτιϲμενοι καγω αναπαυ} & 10 &  &  \\
&  & 10 & \foreignlanguage{greek}{ϲω υμαϲ αρατε τον ζυγον μου εφ υ} & 6 & \textbf{29} &  \\
&  & 6 & \foreignlanguage{greek}{μαϲ και μαθεται απ εμου οτι πραοϲ} & 12 &  &  \\
&  & 13 & \foreignlanguage{greek}{ειμει και ταπινοϲ τη καρδια και ευρη} & 19 &  &  \\
&  & 19 & \foreignlanguage{greek}{ϲεται αναπαυϲιν ταιϲ ψυχαιϲ υμων} & 23 &  &  \\
& \textbf{30} &  & \foreignlanguage{greek}{ο γαρ ζυγοϲ μου χρηϲτοϲ και το φορ} & 8 &  &  \\
&  & 8 & \foreignlanguage{greek}{τιον μου ελαφρον εϲτιν} & 11 &  &  \\
& \mygospelchapter &  & \foreignlanguage{greek}{εν εκεινω τω καιρω επορευθη ο \textoverline{ιϲ} ε̅} & 8 &  &  \\
&  & 9 & \foreignlanguage{greek}{τοιϲ ϲαββαϲιν δια των ϲποριμων} & 13 &  &  \\
&  & 14 & \foreignlanguage{greek}{οι δε μαθηται αυτου επιναϲαν και ηρ} & 20 &  &  \\
&  & 20 & \foreignlanguage{greek}{ξαντο τιλλιν τουϲ ϲταχυαϲ και ε} & 25 &  &  \\
&  & 25 & \foreignlanguage{greek}{ϲθιειν οι δε φαριϲαιοι ιδοντεϲ ειπο̅} & 5 & \textbf{2} &  \\
&  & 6 & \foreignlanguage{greek}{αυτω ιδου οι μαθηται ϲου ποιουϲιν} & 11 &  &  \\
&  & 12 & \foreignlanguage{greek}{ο ουκ εξεϲτιν ποιειν εν ϲαββατω} & 17 &  &  \\
& \textbf{3} &  & \foreignlanguage{greek}{ο δε ειπεν αυτοιϲ ουκ ανεγνωται τι} & 7 &  &  \\
&  & 8 & \foreignlanguage{greek}{εποιηϲεν δαυειδ οτε επιναϲεν και} & 12 &  &  \\
&  & 13 & \foreignlanguage{greek}{οι μετ αυτου ωϲ ειϲηλθεν ειϲ τον οι} & 5 & \textbf{4} &  \\
&  & 5 & \foreignlanguage{greek}{κον του \textoverline{θυ} και τουϲ αρτουϲ τηϲ προ} & 12 &  &  \\
&  & 12 & \foreignlanguage{greek}{θεϲεωϲ εφαγεν ο ουκ εξον ην αυτω} & 18 &  &  \\
[0.2em]
\cline{4-4}
\end{tabular}
\end{center}
\end{table}
}
\clearpage
\newpage
 {
 \setlength\arrayrulewidth{1pt}
\begin{table}
\begin{center}
\begin{tabular}{ccc|l|ccc}
\cline{4-4} \\ [-1em]
\multicolumn{7}{c}{\foreignlanguage{greek}{ευαγγελιον κατα μαθθαιον} \textbf{(\nospace{12:4})} } \\ \\ [-1em] % Si on veut ajouter les bordures latérales, remplacer {7}{c} par {7}{|c|}
\cline{4-4} \\
\cline{4-4}
&  &  & &  &  & \\ [-0.9em]
&  & 19 & \foreignlanguage{greek}{φαγειν ουδε τοιϲ μετ αυτου ει μη τοιϲ} & 26 &  &  \\
&  & 27 & \foreignlanguage{greek}{ιερευϲιν μονοιϲ} & 28 &  &  \\
& \textbf{5} &  & \foreignlanguage{greek}{η ουκ ανεγνωται εν τω νομω οτι εν} & 8 &  &  \\
&  & 9 & \foreignlanguage{greek}{τοιϲ ϲαββαϲιν οι ιερειϲ εν τω ιερω το} & 16 &  &  \\
&  & 17 & \foreignlanguage{greek}{ϲαββατον βεβηλουϲιν και αναιτιοι} & 20 &  &  \\
&  & 21 & \foreignlanguage{greek}{ειϲιν λεγω δε υμιν οτι του ιερου} & 6 & \textbf{6} &  \\
&  & 7 & \foreignlanguage{greek}{μιζων εϲτιν ωδε} & 9 &  &  \\
& \textbf{7} &  & \foreignlanguage{greek}{ει δε εγνωκειτε τι εϲτιν ελεον θελω} & 7 &  &  \\
&  & 8 & \foreignlanguage{greek}{και ου θυϲιαν ουκ αν κατεδικαϲατε} & 13 &  &  \\
&  & 14 & \foreignlanguage{greek}{τουϲ αναιτιουϲ \textoverline{κϲ} γαρ εϲτιν του ϲαβ} & 5 & \textbf{8} &  \\
&  & 5 & \foreignlanguage{greek}{βατου ο υιοϲ του \textoverline{ανου}} & 9 &  &  \\
& \textbf{9} &  & \foreignlanguage{greek}{και μεταβαϲ εκειθεν ηλθεν ειϲ την} & 6 &  &  \\
&  & 7 & \foreignlanguage{greek}{ϲυναγωγην αυτων και ιδου \textoverline{ανοϲ}} & 3 & \textbf{10} &  \\
&  & 4 & \foreignlanguage{greek}{χειραν εχων ξηραν και επηρωτηϲα̅} & 8 &  &  \\
&  & 9 & \foreignlanguage{greek}{αυτον λεγοντεϲ ει εξεϲτιν τοιϲ ϲαβ} & 14 &  &  \\
&  & 14 & \foreignlanguage{greek}{βαϲιν θεραπευϲαι ινα κατηγορηϲουϲιν} & 17 &  &  \\
&  & 18 & \foreignlanguage{greek}{αυτου ο δε ειπεν αυτοιϲ} & 4 & \textbf{11} &  \\
&  & 5 & \foreignlanguage{greek}{τιϲ εϲται εξ υμων ανθρωποϲ οϲ εξει} & 11 &  &  \\
&  & 12 & \foreignlanguage{greek}{προβατον εν και εαν ενπεϲη του} & 17 &  &  \\
&  & 17 & \foreignlanguage{greek}{το τοιϲ ϲαββαϲιν ειϲ βοθυνον ουχι} & 22 &  &  \\
&  & 23 & \foreignlanguage{greek}{κρατηϲει αυτο και εγερει} & 26 &  &  \\
& \textbf{12} &  & \foreignlanguage{greek}{ποϲω ου διαφερει \textoverline{ανοϲ} προβατου} & 5 &  &  \\
&  & 6 & \foreignlanguage{greek}{ωϲτε εξεϲτιν τοιϲ ϲαββαϲιν καλωϲ} & 10 &  &  \\
&  & 11 & \foreignlanguage{greek}{ποιειν τοτε λεγει τω \textoverline{ανω}} & 4 & \textbf{13} &  \\
&  & 5 & \foreignlanguage{greek}{εκτινον την χειρα ϲου και εξετινε̅} & 10 &  &  \\
&  & 11 & \foreignlanguage{greek}{και απεκατεϲταθη υγιηϲ ωϲ η αλλη} & 16 &  &  \\
& \textbf{14} &  & \foreignlanguage{greek}{οι δε φαριϲαιοι ϲυμβουλιον ελαβον} & 5 &  &  \\
&  & 6 & \foreignlanguage{greek}{κατ αυτου οπωϲ αυτον απολεϲωϲιν} & 10 &  &  \\
& \textbf{15} &  & \foreignlanguage{greek}{ο δε \textoverline{ιϲ} γνουϲ ανεχωρηϲεν εκειθεν} & 6 &  &  \\
&  & 7 & \foreignlanguage{greek}{και ηκολουθηϲαν αυτω οχλοι πολλοι} & 11 &  &  \\
[0.2em]
\cline{4-4}
\end{tabular}
\end{center}
\end{table}
}
\clearpage
\newpage
 {
 \setlength\arrayrulewidth{1pt}
\begin{table}
\begin{center}
\begin{tabular}{ccc|l|ccc}
\cline{4-4} \\ [-1em]
\multicolumn{7}{c}{\foreignlanguage{greek}{ευαγγελιον κατα μαθθαιον} \textbf{(\nospace{12:15})} } \\ \\ [-1em] % Si on veut ajouter les bordures latérales, remplacer {7}{c} par {7}{|c|}
\cline{4-4} \\
\cline{4-4}
&  &  & &  &  & \\ [-0.9em]
&  & 12 & \foreignlanguage{greek}{και εθεραπευϲεν αυτουϲ πανταϲ δε} & 16 &  &  \\
&  & 17 & \foreignlanguage{greek}{ουϲ εθεραπευϲεν επεπληξεν αυτοιϲ} & 20 &  &  \\
& \textbf{16} &  & \foreignlanguage{greek}{και επετιμηϲεν αυτοιϲ ινα μη φα} & 6 &  &  \\
&  & 6 & \foreignlanguage{greek}{νερον αυτον ποιηϲωϲιν οπωϲ πλη} & 2 & \textbf{17} &  \\
&  & 2 & \foreignlanguage{greek}{ρωθη το ρηθεν δια ηϲαιου του προφη} & 8 &  &  \\
&  & 8 & \foreignlanguage{greek}{του λεγοντοϲ ιδου ο παιϲ μου ον ηρεν} & 6 & \textbf{18} &  \\
&  & 6 & \foreignlanguage{greek}{τιϲα ο αγαπητοϲ μου ειϲ ον ηυδοκη} & 12 &  &  \\
&  & 12 & \foreignlanguage{greek}{ϲεν η ψυχη μου θηϲω το \textoverline{πνα} μου} & 19 &  &  \\
&  & 20 & \foreignlanguage{greek}{επ αυτον και κριϲιν τοιϲ εθνεϲιν α} & 26 &  &  \\
&  & 26 & \foreignlanguage{greek}{παγγελει ουκ εριϲει ουδε κραυγαϲει} & 4 & \textbf{19} &  \\
&  & 5 & \foreignlanguage{greek}{ουδε ακουϲει τιϲ εν ταιϲ πλατιαιϲ τη̅} & 11 &  &  \\
&  & 12 & \foreignlanguage{greek}{φωνην αυτου καλαμον ϲυντετριμ} & 2 & \textbf{20} &  \\
&  & 2 & \foreignlanguage{greek}{μενον ου μη κατεαξει και λινον τυ} & 8 &  &  \\
&  & 8 & \foreignlanguage{greek}{φομενον ου ϲβεϲει εωϲ αν εκβαλη} & 13 &  &  \\
&  & 14 & \foreignlanguage{greek}{ειϲ νικοϲ την κριϲιν και επι τω ονο} & 4 & \textbf{21} &  \\
&  & 4 & \foreignlanguage{greek}{ματι αυτου εθνη ελπιουϲιν} & 7 &  &  \\
& \textbf{22} &  & \foreignlanguage{greek}{τοτε προϲηνεχθη αυτω δαιμονιζομε} & 4 &  &  \\
&  & 4 & \foreignlanguage{greek}{νοϲ τυφλοϲ και κωφοϲ και εθεραπευ} & 9 &  &  \\
&  & 9 & \foreignlanguage{greek}{ϲεν αυτον ωϲτε τον κωφον και τυ} & 15 &  &  \\
&  & 15 & \foreignlanguage{greek}{φλον λαλιν και βλεπειν} & 18 &  &  \\
& \textbf{23} &  & \foreignlanguage{greek}{και εξιϲταντο παντεϲ οι οχλοι και ε} & 7 &  &  \\
&  & 7 & \foreignlanguage{greek}{λεγον μητι ουτοϲ εϲτιν ο υιοϲ \textoverline{δαδ}} & 13 &  &  \\
& \textbf{24} &  & \foreignlanguage{greek}{οι δε φαριϲαιοι ακουϲαντεϲ ειπον ουτοϲ} & 6 &  &  \\
&  & 7 & \foreignlanguage{greek}{ουκ εκβαλλει τα δαιμονια ει μη εν} & 13 &  &  \\
&  & 14 & \foreignlanguage{greek}{τω βεελζεβουλ αρχοντι των δαιμονιω̅} & 18 &  &  \\
& \textbf{25} &  & \foreignlanguage{greek}{ιδωϲ δε ο \textoverline{ιϲ} ταϲ ενθυμηϲειϲ αυτων} & 7 &  &  \\
&  & 8 & \foreignlanguage{greek}{ειπεν αυτοιϲ παϲα βαϲιλεια μερι} & 12 &  &  \\
&  & 12 & \foreignlanguage{greek}{ϲθειϲα καθ εαυτηϲ ερημουται και} & 16 &  &  \\
&  & 17 & \foreignlanguage{greek}{παϲα πολιϲ η οικεια μεριϲθειϲα κα} & 22 &  &  \\
&  & 22 & \foreignlanguage{greek}{θ εαυτηϲ ου ϲταθηϲεται και ει ο ϲα} & 4 & \textbf{26} &  \\
[0.2em]
\cline{4-4}
\end{tabular}
\end{center}
\end{table}
}
\clearpage
\newpage
 {
 \setlength\arrayrulewidth{1pt}
\begin{table}
\begin{center}
\begin{tabular}{ccc|l|ccc}
\cline{4-4} \\ [-1em]
\multicolumn{7}{c}{\foreignlanguage{greek}{ευαγγελιον κατα μαθθαιον} \textbf{(\nospace{12:26})} } \\ \\ [-1em] % Si on veut ajouter les bordures latérales, remplacer {7}{c} par {7}{|c|}
\cline{4-4} \\
\cline{4-4}
&  &  & &  &  & \\ [-0.9em]
&  & 4 & \foreignlanguage{greek}{ταναϲ τον ϲαταναν εκβαλλει εφ εαυτο̅} & 9 &  &  \\
&  & 10 & \foreignlanguage{greek}{εμεριϲθη πωϲ ουν ϲταθηϲεται η βαϲι} & 15 &  &  \\
&  & 15 & \foreignlanguage{greek}{λεια αυτου και ει εγω εν βεελζεβουλ} & 5 & \textbf{27} &  \\
&  & 6 & \foreignlanguage{greek}{εκβαλλω τα δαιμονια οι υιοι υμων} & 11 &  &  \\
&  & 12 & \foreignlanguage{greek}{εν τινι εκβαλλουϲιν δια τουτο κριται} & 17 &  &  \\
&  & 18 & \foreignlanguage{greek}{εϲονται αυτοι υμων} & 20 &  &  \\
& \textbf{28} &  & \foreignlanguage{greek}{ει δε εν \textoverline{πνι} \textoverline{θυ} εγω εκβαλλω τα δαιμο} & 9 &  &  \\
&  & 9 & \foreignlanguage{greek}{νια αρα εφθαϲεν εφ υμαϲ η βαϲιλεια} & 15 &  &  \\
&  & 16 & \foreignlanguage{greek}{του \textoverline{θυ} η πωϲ δυναται τιϲ ειϲελθει̅} & 5 & \textbf{29} &  \\
&  & 6 & \foreignlanguage{greek}{ειϲ την οικειαν του ιϲχυρου και τα ϲκευ} & 13 &  &  \\
&  & 13 & \foreignlanguage{greek}{η αυτου αρπαϲαι εαν μη πρωτον δη} & 19 &  &  \\
&  & 19 & \foreignlanguage{greek}{ϲη τον ιϲχυρον και τοτε την οικιαν} & 25 &  &  \\
&  & 26 & \foreignlanguage{greek}{αυτου διαρπαϲη ο μη ων μετ εμου} & 5 & \textbf{30} &  \\
&  & 6 & \foreignlanguage{greek}{κατ εμου εϲτιν και ο μη ϲυναγων} & 12 &  &  \\
&  & 13 & \foreignlanguage{greek}{μετ εμου ϲκορπιζει} & 15 &  &  \\
& \textbf{31} &  & \foreignlanguage{greek}{δια τουτο λεγω υμιν παϲα αμαρτια} & 6 &  &  \\
&  & 7 & \foreignlanguage{greek}{και βλαϲφημια αφεθηϲεται τοιϲ \textoverline{ανοιϲ}} & 11 &  &  \\
&  & 12 & \foreignlanguage{greek}{η δε του \textoverline{πνϲ} βλαϲφημια ουκ αφε} & 18 &  &  \\
&  & 18 & \foreignlanguage{greek}{θηϲεται τοιϲ ανθρωποιϲ} & 20 &  &  \\
& \textbf{32} &  & \foreignlanguage{greek}{και οϲ εαν ειπη λογον κατα του υιου} & 8 &  &  \\
&  & 9 & \foreignlanguage{greek}{του \textoverline{ανου} αφεθηϲεται αυτω} & 12 &  &  \\
&  & 13 & \foreignlanguage{greek}{οϲ δ αν ειπη κατα του \textoverline{πνϲ} του αγιου} & 21 &  &  \\
&  & 22 & \foreignlanguage{greek}{ουκ αφεθηϲεται αυτω ουτε εν τουτω} & 27 &  &  \\
&  & 28 & \foreignlanguage{greek}{τω αιωνι ουτε εν τω μελλοντι} & 33 &  &  \\
& \textbf{33} &  & \foreignlanguage{greek}{η ποιηϲηται το δενδρον καλον και} & 6 &  &  \\
&  & 7 & \foreignlanguage{greek}{τον καρπον αυτου καλον η ποιηϲα} & 12 &  &  \\
&  & 12 & \foreignlanguage{greek}{τε το δενδρον ϲαπρον και τον καρπο̅} & 18 &  &  \\
&  & 19 & \foreignlanguage{greek}{αυτου ϲαπρον εκ γαρ του καρπου} & 24 &  &  \\
&  & 25 & \foreignlanguage{greek}{το δενδρον γινωϲκεται} & 27 &  &  \\
& \textbf{34} &  & \foreignlanguage{greek}{γεννηματα αιχιδνων πωϲ δυναϲθε} & 4 &  &  \\
[0.2em]
\cline{4-4}
\end{tabular}
\end{center}
\end{table}
}
\clearpage
\newpage
 {
 \setlength\arrayrulewidth{1pt}
\begin{table}
\begin{center}
\begin{tabular}{ccc|l|ccc}
\cline{4-4} \\ [-1em]
\multicolumn{7}{c}{\foreignlanguage{greek}{ευαγγελιον κατα μαθθαιον} \textbf{(\nospace{12:34})} } \\ \\ [-1em] % Si on veut ajouter les bordures latérales, remplacer {7}{c} par {7}{|c|}
\cline{4-4} \\
\cline{4-4}
&  &  & &  &  & \\ [-0.9em]
&  & 5 & \foreignlanguage{greek}{αγαθα λαλιν πονηροι οντεϲ εκ γαρ} & 10 &  &  \\
&  & 11 & \foreignlanguage{greek}{του περιϲευματοϲ τηϲ καρδιαϲ το} & 15 &  &  \\
&  & 16 & \foreignlanguage{greek}{ϲτομα λαλει} & 17 &  &  \\
& \textbf{35} &  & \foreignlanguage{greek}{ο αγαθοϲ \textoverline{ανοϲ} εκ του αγαθου θηϲαυρου} & 7 &  &  \\
&  & 8 & \foreignlanguage{greek}{εκβαλλει αγαθα και ο πονηροϲ \textoverline{ανοϲ}} & 13 &  &  \\
&  & 14 & \foreignlanguage{greek}{εκ του πονηρου θηϲαυρου εκβαλλει} & 18 &  &  \\
&  & 19 & \foreignlanguage{greek}{πονηρα} & 19 &  &  \\
& \textbf{36} &  & \foreignlanguage{greek}{λεγω δε υμιν οτι παν ρημα αργον ο ε} & 9 &  &  \\
&  & 9 & \foreignlanguage{greek}{αν λαληϲωϲιν οι ανθρωποι αποδω} & 13 &  &  \\
&  & 13 & \foreignlanguage{greek}{ϲωϲιν περι αυτου λογον εν ημερα κρι} & 19 &  &  \\
&  & 19 & \foreignlanguage{greek}{ϲεωϲ εκ γαρ των λογων ϲου δικαιω} & 6 & \textbf{37} &  \\
&  & 6 & \foreignlanguage{greek}{θηϲη και εκ των λογων ϲου καταδι} & 12 &  &  \\
&  & 12 & \foreignlanguage{greek}{καϲθηϲη τοτε απεκριθηϲαν} & 2 & \textbf{38} &  \\
&  & 3 & \foreignlanguage{greek}{τινεϲ των γραμματεων και φαρι} & 7 &  &  \\
&  & 7 & \foreignlanguage{greek}{ϲεων λεγοντεϲ διδαϲκαλε θελο} & 10 &  &  \\
&  & 10 & \foreignlanguage{greek}{μεν απο ϲου ϲημιον ιδειν} & 14 &  &  \\
& \textbf{39} &  & \foreignlanguage{greek}{ο δε αποκριθειϲ ειπεν αυτοιϲ γενεα} & 6 &  &  \\
&  & 7 & \foreignlanguage{greek}{πονηρα και μοιχαλιϲ ϲημιον επι} & 11 &  &  \\
&  & 11 & \foreignlanguage{greek}{ζητει και ϲημιον ου δοθηϲεται αυτη} & 16 &  &  \\
&  & 17 & \foreignlanguage{greek}{ει μη το ϲημιον ιωνα του προφητου} & 23 &  &  \\
& \textbf{40} &  & \foreignlanguage{greek}{ωϲπερ γαρ ην ιωναϲ εν τη κοιλια} & 7 &  &  \\
&  & 8 & \foreignlanguage{greek}{του κητουϲ τριϲ ημεραϲ και τριϲ} & 13 &  &  \\
&  & 14 & \foreignlanguage{greek}{νυκταϲ ουτωϲ εϲται και ο υιοϲ του} & 20 &  &  \\
&  & 21 & \foreignlanguage{greek}{ανθρωπου εν τη καρδια τηϲ γηϲ} & 26 &  &  \\
&  & 27 & \foreignlanguage{greek}{τριϲ ημεραϲ και τριϲ νυκταϲ} & 31 &  &  \\
& \textbf{41} &  & \foreignlanguage{greek}{ανδρεϲ νινευειται αναϲτηϲονται} & 3 &  &  \\
&  & 4 & \foreignlanguage{greek}{εν τη κριϲει μετα τηϲ γενεαϲ ταυτηϲ} & 10 &  &  \\
&  & 11 & \foreignlanguage{greek}{και κατακρινουϲιν αυτην οτι με} & 15 &  &  \\
&  & 15 & \foreignlanguage{greek}{τενοηϲαν ειϲ το κηρυγμα ιωνα και} & 20 &  &  \\
&  & 21 & \foreignlanguage{greek}{ιδου πλιον ιωνα ωδε} & 24 &  &  \\
[0.2em]
\cline{4-4}
\end{tabular}
\end{center}
\end{table}
}
\clearpage
\newpage
 {
 \setlength\arrayrulewidth{1pt}
\begin{table}
\begin{center}
\begin{tabular}{ccc|l|ccc}
\cline{4-4} \\ [-1em]
\multicolumn{7}{c}{\foreignlanguage{greek}{ευαγγελιον κατα μαθθαιον} \textbf{(\nospace{12:42})} } \\ \\ [-1em] % Si on veut ajouter les bordures latérales, remplacer {7}{c} par {7}{|c|}
\cline{4-4} \\
\cline{4-4}
&  &  & &  &  & \\ [-0.9em]
& \textbf{42} &  & \foreignlanguage{greek}{βαϲιλιϲϲα νοτου εγερθηϲεται εν τη κρι} & 6 &  &  \\
&  & 6 & \foreignlanguage{greek}{ϲει μετα τηϲ γενεαϲ ταυτηϲ και κατα} & 12 &  &  \\
&  & 12 & \foreignlanguage{greek}{κρινει αυτην οτι ηλθεν εκ των περα} & 18 &  &  \\
&  & 18 & \foreignlanguage{greek}{των τηϲ γηϲ ακουϲαι την ϲοφιαν ϲο} & 24 &  &  \\
&  & 24 & \foreignlanguage{greek}{λομωνοϲ και ιδου πλιον ϲολομωνοϲ} & 28 &  &  \\
&  & 29 & \foreignlanguage{greek}{ωδε οταν δε το ακαθαρτον} & 4 & \textbf{43} &  \\
&  & 5 & \foreignlanguage{greek}{\textoverline{πνα} εξελθη απο του ανθρωπου διερ} & 10 &  &  \\
&  & 10 & \foreignlanguage{greek}{χεται δι ανυδρων τοπων ζητουν} & 14 &  &  \\
&  & 15 & \foreignlanguage{greek}{αναπαυϲιν και ουχ ευριϲκει τοτε} & 1 & \textbf{44} &  \\
&  & 2 & \foreignlanguage{greek}{λεγει επιϲτρεψω ειϲ τον οικον μου} & 7 &  &  \\
&  & 8 & \foreignlanguage{greek}{οθεν εξηλθον και ελθον ευριϲκει} & 12 &  &  \\
&  & 13 & \foreignlanguage{greek}{ϲχολαζοντα ϲεϲαρωμενον και κε} & 16 &  &  \\
&  & 16 & \foreignlanguage{greek}{κοϲμημενον τοτε πορευεται και} & 3 & \textbf{45} &  \\
&  & 4 & \foreignlanguage{greek}{παραλαμβανει μεθ εαυτου επτα} & 7 &  &  \\
&  & 8 & \foreignlanguage{greek}{ετερα \textoverline{πντα} πονηροτερα εαυτου} & 11 &  &  \\
&  & 12 & \foreignlanguage{greek}{και ειϲελθοντα κατοικει εκει} & 15 &  &  \\
&  & 16 & \foreignlanguage{greek}{και γεινεται τα εϲχατα του \textoverline{ανου} ε} & 22 &  &  \\
&  & 22 & \foreignlanguage{greek}{κεινου χειρονα των πρωτων} & 25 &  &  \\
&  & 26 & \foreignlanguage{greek}{ουτωϲ εϲται και τη γενεα ταυτη} & 31 &  &  \\
&  & 32 & \foreignlanguage{greek}{τη πονηρα} & 33 &  &  \\
& \textbf{46} &  & \foreignlanguage{greek}{ετι δε αυτου λαλουντοϲ τοιϲ οχλοιϲ} & 6 &  &  \\
&  & 7 & \foreignlanguage{greek}{ιδου η \textoverline{μηρ} και οι αδελφοι αυτου ι} & 14 &  &  \\
&  & 14 & \foreignlanguage{greek}{ϲτηκειϲαν εξω ζητουντεϲ αυτω λα} & 18 &  &  \\
&  & 18 & \foreignlanguage{greek}{ληϲαι ειπεν δε τιϲ αυτω ιδου η} & 6 & \textbf{47} &  \\
&  & 7 & \foreignlanguage{greek}{\textoverline{μηρ} ϲου και οι αδελφοι ϲου εξω εϲτη} & 14 &  &  \\
&  & 14 & \foreignlanguage{greek}{καϲιν ζητουντεϲ ϲοι λαληϲαι} & 17 &  &  \\
& \textbf{48} &  & \foreignlanguage{greek}{ο δε αποκριθειϲ ειπεν τιϲ εϲτιν η} & 7 &  &  \\
&  & 8 & \foreignlanguage{greek}{\textoverline{μηρ} μου η τινεϲ οι αδελφοι μου} & 14 &  &  \\
& \textbf{49} &  & \foreignlanguage{greek}{και εκτιναϲ την χειρα αυτου επι} & 6 &  &  \\
&  & 7 & \foreignlanguage{greek}{τουϲ μαθηταϲ αυτου ειπεν ιδου} & 11 &  &  \\
[0.2em]
\cline{4-4}
\end{tabular}
\end{center}
\end{table}
}
\clearpage
\newpage
 {
 \setlength\arrayrulewidth{1pt}
\begin{table}
\begin{center}
\begin{tabular}{ccc|l|ccc}
\cline{4-4} \\ [-1em]
\multicolumn{7}{c}{\foreignlanguage{greek}{ευαγγελιον κατα μαθθαιον} \textbf{(\nospace{12:49})} } \\ \\ [-1em] % Si on veut ajouter les bordures latérales, remplacer {7}{c} par {7}{|c|}
\cline{4-4} \\
\cline{4-4}
&  &  & &  &  & \\ [-0.9em]
&  & 12 & \foreignlanguage{greek}{η \textoverline{μηρ} μου και οι αδελφοι μου} & 18 &  &  \\
& \textbf{50} &  & \foreignlanguage{greek}{οϲτιϲ γαρ αν ποιηϲη το θελημα του} & 7 &  &  \\
&  & 8 & \foreignlanguage{greek}{\textoverline{πρϲ} μου του εν ουρανοιϲ αυτοϲ μου} & 14 &  &  \\
&  & 15 & \foreignlanguage{greek}{αδελφοϲ και αδελφη κα \textoverline{μηρ} εϲτι̅} & 20 &  &  \\
& \mygospelchapter &  & \foreignlanguage{greek}{εν δε τη ημερα εκεινη εξελθων ο \textoverline{ιϲ}} & 8 &  &  \\
&  & 9 & \foreignlanguage{greek}{απο τηϲ οικειαϲ εκαθητο παρα την} & 14 &  &  \\
&  & 15 & \foreignlanguage{greek}{θαλαϲϲαν και ϲυνηχθηϲαν προϲ αυ} & 4 & \textbf{2} &  \\
&  & 4 & \foreignlanguage{greek}{τον οχλοι πολλοι ωϲτε αυτον ειϲ} & 9 &  &  \\
&  & 10 & \foreignlanguage{greek}{πλοιον ενβαντα καθηϲθαι και παϲ} & 14 &  &  \\
&  & 15 & \foreignlanguage{greek}{ο οχλοϲ επι τον εγειαλον ιϲτηκει} & 20 &  &  \\
& \textbf{3} &  & \foreignlanguage{greek}{και ελαληϲεν αυτοιϲ πολλα εν πα} & 6 &  &  \\
&  & 6 & \foreignlanguage{greek}{ραβολαιϲ λεγων} & 7 &  &  \\
&  & 8 & \foreignlanguage{greek}{ιδου εξηλθεν ο ϲπειρων του ϲπειραι} & 13 &  &  \\
& \textbf{4} &  & \foreignlanguage{greek}{και εν τω ϲπιρειν αυτον α μεν επε} & 8 &  &  \\
&  & 8 & \foreignlanguage{greek}{ϲεν παρα την οδον και ηλθεν τα πε} & 15 &  &  \\
&  & 15 & \foreignlanguage{greek}{τεινα και κατεφαγεν αυτα αλλα} & 1 & \textbf{5} &  \\
&  & 2 & \foreignlanguage{greek}{δε επεϲεν επι τα πετρωδη οπου ου} & 8 &  &  \\
&  & 8 & \foreignlanguage{greek}{κ ειχεν γην πολλην και ευθεωϲ εξα} & 14 &  &  \\
&  & 14 & \foreignlanguage{greek}{νετιλεν δια το μη εχειν βαθοϲ γηϲ} & 20 &  &  \\
& \textbf{6} &  & \foreignlanguage{greek}{ηλιου δε ανατιλαντοϲ εκαυματι} & 4 &  &  \\
&  & 4 & \foreignlanguage{greek}{ϲθη και δια το μη εχειν ριζαν εξη} & 11 &  &  \\
&  & 11 & \foreignlanguage{greek}{ρανθη αλλα δε επεϲεν επι ταϲ α} & 6 & \textbf{7} &  \\
&  & 6 & \foreignlanguage{greek}{κανθαϲ και ανεβηϲαν αι ακανθαι} & 10 &  &  \\
&  & 11 & \foreignlanguage{greek}{και απεπνιξαν αυτα} & 13 &  &  \\
& \textbf{8} &  & \foreignlanguage{greek}{αλλα δε επεϲαν επι την γην την} & 7 &  &  \\
&  & 8 & \foreignlanguage{greek}{καλην και εδιδου καρπον ο μεν ε} & 14 &  &  \\
&  & 14 & \foreignlanguage{greek}{κατον ο δε εξηκοντα ο δε τριακοντα} & 20 &  &  \\
& \textbf{9} &  & \foreignlanguage{greek}{ο εχων ωτα ακουειν ακουετω} & 5 &  &  \\
& \textbf{10} &  & \foreignlanguage{greek}{και προϲελθοντεϲ οι μαθηται ειπον} & 5 &  &  \\
&  & 6 & \foreignlanguage{greek}{αυτω δια τι εν παραβολαιϲ λαλειϲ αυτοιϲ} & 12 &  &  \\
[0.2em]
\cline{4-4}
\end{tabular}
\end{center}
\end{table}
}
\clearpage
\newpage
 {
 \setlength\arrayrulewidth{1pt}
\begin{table}
\begin{center}
\begin{tabular}{ccc|l|ccc}
\cline{4-4} \\ [-1em]
\multicolumn{7}{c}{\foreignlanguage{greek}{ευαγγελιον κατα μαθθαιον} \textbf{(\nospace{13:11})} } \\ \\ [-1em] % Si on veut ajouter les bordures latérales, remplacer {7}{c} par {7}{|c|}
\cline{4-4} \\
\cline{4-4}
&  &  & &  &  & \\ [-0.9em]
& \textbf{11} &  & \foreignlanguage{greek}{ο δε αποκριθειϲ ειπεν αυτοιϲ οτι υμιν} & 7 &  &  \\
&  & 8 & \foreignlanguage{greek}{δεδοται γνωναι τα μυϲτηρια τηϲ βα} & 13 &  &  \\
&  & 13 & \foreignlanguage{greek}{ϲιλειαϲ των ουρανων εκεινοιϲ δε} & 17 &  &  \\
&  & 18 & \foreignlanguage{greek}{ου δεδοται οϲτιϲ γαρ εχει δοθηϲεται} & 4 & \textbf{12} &  \\
&  & 5 & \foreignlanguage{greek}{αυτω και περιϲϲευθηϲεται οϲτιϲ δε} & 9 &  &  \\
&  & 10 & \foreignlanguage{greek}{ουκ εχει και ο εχει αρθηϲεται απ αυτου} & 17 &  &  \\
& \textbf{13} &  & \foreignlanguage{greek}{δια τουτο εν παραβολαιϲ αυτοιϲ λαλω} & 6 &  &  \\
&  & 7 & \foreignlanguage{greek}{οτι βλεποντεϲ ου βλεπουϲιν και α} & 12 &  &  \\
&  & 12 & \foreignlanguage{greek}{κουοντεϲ ουκ ακουουϲιν ουδε ϲυνι} & 16 &  &  \\
&  & 16 & \foreignlanguage{greek}{ουϲιν και αναπληρουται επ αυτοιϲ} & 4 & \textbf{14} &  \\
&  & 5 & \foreignlanguage{greek}{η προφητια ηϲαιου η λεγουϲα} & 9 &  &  \\
&  & 10 & \foreignlanguage{greek}{ακοη ακουϲητε και ου μη ϲυνητε} & 15 &  &  \\
&  & 16 & \foreignlanguage{greek}{και βλεποντεϲ βλεψηται και ου μη} & 21 &  &  \\
&  & 22 & \foreignlanguage{greek}{ιδητε επαχυνθη γαρ η καρδια του} & 5 & \textbf{15} &  \\
&  & 6 & \foreignlanguage{greek}{λαου τουτου και τοιϲ ωϲιν βαρεωϲ} & 11 &  &  \\
&  & 12 & \foreignlanguage{greek}{ηκουϲαν και τουϲ οφθαλμουϲ αυ} & 16 &  &  \\
&  & 16 & \foreignlanguage{greek}{των εκαμμυϲαν μηποτε ιδωϲιν} & 19 &  &  \\
&  & 20 & \foreignlanguage{greek}{τοιϲ οφθαλμοιϲ και τοιϲ ωϲιν ακου} & 25 &  &  \\
&  & 25 & \foreignlanguage{greek}{ϲωϲιν και τη καρδια ϲυνωϲιν και} & 30 &  &  \\
&  & 31 & \foreignlanguage{greek}{επιϲτρεψουϲιν και ιαϲομαι αυτουϲ} & 34 &  &  \\
& \textbf{16} &  & \foreignlanguage{greek}{υμων δε μακαριοι οι οφθαλμοι οτι} & 6 &  &  \\
&  & 7 & \foreignlanguage{greek}{βλεπουϲιν και τα ωτα υμων οτι α} & 13 &  &  \\
&  & 13 & \foreignlanguage{greek}{κουει αμην γαρ λεγω υμιν οτι} & 5 & \textbf{17} &  \\
&  & 6 & \foreignlanguage{greek}{πολλοι προφηται και δικαιοι επε} & 10 &  &  \\
&  & 10 & \foreignlanguage{greek}{θυμηϲαν ειδειν α βλεπεται και ου} & 15 &  &  \\
&  & 15 & \foreignlanguage{greek}{κ ειδον και ακουϲαι α ακουεται} & 20 &  &  \\
&  & 21 & \foreignlanguage{greek}{και ουκ ηκουϲαν} & 23 &  &  \\
& \textbf{18} &  & \foreignlanguage{greek}{υμειϲ ουν ακουϲαται την παραβολη̅} & 5 &  &  \\
&  & 6 & \foreignlanguage{greek}{του ϲπειροντοϲ παντοϲ ακουοντοϲ} & 2 & \textbf{19} &  \\
&  & 3 & \foreignlanguage{greek}{τον λογον τηϲ βαϲιλειαϲ και μη ϲυνιεντοϲ} & 9 &  &  \\
[0.2em]
\cline{4-4}
\end{tabular}
\end{center}
\end{table}
}
\clearpage
\newpage
 {
 \setlength\arrayrulewidth{1pt}
\begin{table}
\begin{center}
\begin{tabular}{ccc|l|ccc}
\cline{4-4} \\ [-1em]
\multicolumn{7}{c}{\foreignlanguage{greek}{ευαγγελιον κατα μαθθαιον} \textbf{(\nospace{13:19})} } \\ \\ [-1em] % Si on veut ajouter les bordures latérales, remplacer {7}{c} par {7}{|c|}
\cline{4-4} \\
\cline{4-4}
&  &  & &  &  & \\ [-0.9em]
&  & 10 & \foreignlanguage{greek}{ερχεται ο πονηροϲ και αρπαζει το ϲπει} & 16 &  &  \\
&  & 16 & \foreignlanguage{greek}{ρομενον εν τη καρδια αυτου} & 20 &  &  \\
&  & 21 & \foreignlanguage{greek}{ουτοϲ εϲτιν ο παρα την οδον ϲπαριϲ} & 27 &  &  \\
& \textbf{20} &  & \foreignlanguage{greek}{ο δε επι τα πετρωδη ϲπαρειϲ ουτοϲ} & 7 &  &  \\
&  & 8 & \foreignlanguage{greek}{εϲτιν ο τον λογον μου ακουων και} & 14 &  &  \\
&  & 15 & \foreignlanguage{greek}{ευθυϲ και μετα χαραϲ λαμβανων} & 19 &  &  \\
&  & 20 & \foreignlanguage{greek}{αυτον ουκ εχει δε ριζαν εν εαυτω} & 6 & \textbf{21} &  \\
&  & 7 & \foreignlanguage{greek}{αλλα προϲκαιροϲ εϲτιν} & 9 &  &  \\
&  & 10 & \foreignlanguage{greek}{γενομενηϲ δε θλιψεωϲ η διωγμου} & 14 &  &  \\
&  & 15 & \foreignlanguage{greek}{δια τον λογον ευθυϲ ϲκανδαλιζεται} & 19 &  &  \\
& \textbf{22} &  & \foreignlanguage{greek}{ο δε ειϲ ταϲ ακανθαϲ ϲπαρειϲ ουτοϲ} & 7 &  &  \\
&  & 8 & \foreignlanguage{greek}{εϲτιν ο τον λογον μου ακουων} & 13 &  &  \\
&  & 14 & \foreignlanguage{greek}{και η μεριμνα του αιωνοϲ τουτου} & 19 &  &  \\
&  & 20 & \foreignlanguage{greek}{και η απατη του πλουτου ϲυνπνι} & 25 &  &  \\
&  & 25 & \foreignlanguage{greek}{γει τον λογον και ακαρποϲ γεινεται} & 30 &  &  \\
& \textbf{23} &  & \foreignlanguage{greek}{ο δε επι την γην την καλην ϲπαρειϲ} & 8 &  &  \\
&  & 9 & \foreignlanguage{greek}{ουτοϲ εϲτιν ο τον λογον μου ακουω̅} & 15 &  &  \\
&  & 16 & \foreignlanguage{greek}{και ϲυνιων οϲ δη καρποφορι και} & 21 &  &  \\
&  & 22 & \foreignlanguage{greek}{ποιει ο μεν εκατον ο δε εξηκοντα} & 28 &  &  \\
&  & 29 & \foreignlanguage{greek}{ο δε τριακοντα} & 31 &  &  \\
& \textbf{24} &  & \foreignlanguage{greek}{αλλην παραβολην παρεθηκεν αυ} & 4 &  &  \\
&  & 4 & \foreignlanguage{greek}{τοιϲ λεγων ομοιωθη η βαϲιλεια} & 8 &  &  \\
&  & 9 & \foreignlanguage{greek}{των ουρανων ανθρωπω ϲπειραν} & 12 &  &  \\
&  & 12 & \foreignlanguage{greek}{τι καλον ϲπερμα εν τω αγρω αυτου} & 18 &  &  \\
& \textbf{25} &  & \foreignlanguage{greek}{εν δε τω καθευδειν τουϲ \textoverline{ανουϲ}} & 6 &  &  \\
&  & 7 & \foreignlanguage{greek}{ηλθεν αυτου ο εχθροϲ και εϲπειρε̅} & 12 &  &  \\
&  & 13 & \foreignlanguage{greek}{ζιζανια ανα μεϲον του ϲιτου και} & 18 &  &  \\
&  & 19 & \foreignlanguage{greek}{απηλθεν οτε δε εβλαϲτηϲεν ο} & 4 & \textbf{26} &  \\
&  & 5 & \foreignlanguage{greek}{χορτοϲ και καρπον εποιηϲεν τοτε} & 9 &  &  \\
&  & 10 & \foreignlanguage{greek}{εφανη τα ζιζανια} & 12 &  &  \\
[0.2em]
\cline{4-4}
\end{tabular}
\end{center}
\end{table}
}
\clearpage
\newpage
 {
 \setlength\arrayrulewidth{1pt}
\begin{table}
\begin{center}
\begin{tabular}{ccc|l|ccc}
\cline{4-4} \\ [-1em]
\multicolumn{7}{c}{\foreignlanguage{greek}{ευαγγελιον κατα μαθθαιον} \textbf{(\nospace{13:27})} } \\ \\ [-1em] % Si on veut ajouter les bordures latérales, remplacer {7}{c} par {7}{|c|}
\cline{4-4} \\
\cline{4-4}
&  &  & &  &  & \\ [-0.9em]
& \textbf{27} &  & \foreignlanguage{greek}{προϲελθοντεϲ δε οι δουλοι του οικο} & 6 &  &  \\
&  & 6 & \foreignlanguage{greek}{δεϲποτου ειπον αυτω \textoverline{κε} ουχι κα} & 11 &  &  \\
&  & 11 & \foreignlanguage{greek}{λον ϲπερμα εϲπειρεϲ εν τω ϲω αγρω} & 17 &  &  \\
&  & 18 & \foreignlanguage{greek}{ποθεν ουν εχει ζιζανια ο δε εφη αυ} & 4 & \textbf{28} &  \\
&  & 4 & \foreignlanguage{greek}{τοιϲ εχθροϲ \textoverline{ανοϲ} τουτο εποιηϲεν} & 8 &  &  \\
&  & 9 & \foreignlanguage{greek}{οι δε δουλοι ειπον αυτω θελειϲ ουν} & 15 &  &  \\
&  & 16 & \foreignlanguage{greek}{απελθοντεϲ ϲυνλεξωμεν αυτα} & 18 &  &  \\
& \textbf{29} &  & \foreignlanguage{greek}{ο δε εφη ου μηποτε ϲυλλεγοντεϲ} & 6 &  &  \\
&  & 7 & \foreignlanguage{greek}{τα ζιζανια εκριζωϲηται αμα αυτοιϲ} & 11 &  &  \\
&  & 12 & \foreignlanguage{greek}{τον ϲιτον αφετε ϲυναυξανεϲθαι} & 2 & \textbf{30} &  \\
&  & 3 & \foreignlanguage{greek}{αμφοτερα μεχριϲ του θεριϲμου} & 6 &  &  \\
&  & 7 & \foreignlanguage{greek}{και εν καιρω του θεριϲμου ερω τοιϲ} & 13 &  &  \\
&  & 14 & \foreignlanguage{greek}{θεριϲταιϲ ϲυλλεξατε πρωτον τα ζι} & 18 &  &  \\
&  & 18 & \foreignlanguage{greek}{ζανια και δηϲατε αυτα ειϲ δεϲμαϲ} & 23 &  &  \\
&  & 24 & \foreignlanguage{greek}{προϲ το κατακαυϲαι αυτα} & 27 &  &  \\
&  & 28 & \foreignlanguage{greek}{τον δε ϲιτον ϲυναγαγεται ειϲ την} & 33 &  &  \\
&  & 34 & \foreignlanguage{greek}{αποθηκην μου} & 35 &  &  \\
& \textbf{31} &  & \foreignlanguage{greek}{αλλην παραβολην παρεθηκεν αυτοιϲ} & 4 &  &  \\
&  & 5 & \foreignlanguage{greek}{λεγων ομοια εϲτιν η βαϲιλεια των} & 10 &  &  \\
&  & 11 & \foreignlanguage{greek}{ουρανων κοκκω ϲιναπεωϲ ον λα} & 15 &  &  \\
&  & 15 & \foreignlanguage{greek}{βων ανθρωποϲ εϲπειρεν εν τω αγρω} & 20 &  &  \\
&  & 21 & \foreignlanguage{greek}{αυτου ο μεικροτερον μεν εϲτιν} & 4 & \textbf{32} &  \\
&  & 5 & \foreignlanguage{greek}{παντων των ϲπερματων οταν δε} & 9 &  &  \\
&  & 10 & \foreignlanguage{greek}{αυξηθη μιζον των λαχανων εϲτι̅} & 14 &  &  \\
&  & 15 & \foreignlanguage{greek}{και γεινεται δενδρον ωϲτε ελθει̅} & 19 &  &  \\
&  & 20 & \foreignlanguage{greek}{τα πετινα του ουρανου και καταϲκη} & 25 &  &  \\
&  & 25 & \foreignlanguage{greek}{νουν εν τοιϲ κλαδοιϲ αυτου} & 29 &  &  \\
& \textbf{33} &  & \foreignlanguage{greek}{αλλην παραβολην ελαληϲεν αυτοιϲ} & 4 &  &  \\
&  & 5 & \foreignlanguage{greek}{ομοια εϲτιν η βαϲιλεια των ουρανω̅} & 10 &  &  \\
&  & 11 & \foreignlanguage{greek}{ζυμη ην λαβουϲα γυνη ενεκρυψεν} & 15 &  &  \\
[0.2em]
\cline{4-4}
\end{tabular}
\end{center}
\end{table}
}
\clearpage
\newpage
 {
 \setlength\arrayrulewidth{1pt}
\begin{table}
\begin{center}
\begin{tabular}{ccc|l|ccc}
\cline{4-4} \\ [-1em]
\multicolumn{7}{c}{\foreignlanguage{greek}{ευαγγελιον κατα μαθθαιον} \textbf{(\nospace{13:33})} } \\ \\ [-1em] % Si on veut ajouter les bordures latérales, remplacer {7}{c} par {7}{|c|}
\cline{4-4} \\
\cline{4-4}
&  &  & &  &  & \\ [-0.9em]
&  & 16 & \foreignlanguage{greek}{ειϲ αλευρου ϲατα τρια εωϲ ου εζυμωθη} & 22 &  &  \\
&  & 23 & \foreignlanguage{greek}{ολον ταυτα παντα ελαληϲεν} & 3 & \textbf{34} &  \\
&  & 4 & \foreignlanguage{greek}{ο \textoverline{ιϲ} εν παραβολαιϲ τοιϲ οχλοιϲ και χω} & 11 &  &  \\
&  & 11 & \foreignlanguage{greek}{ριϲ παραβοληϲ ουδεν ελαλι αυτοιϲ} & 15 &  &  \\
& \textbf{35} &  & \foreignlanguage{greek}{οπωϲ πληρωθη το ρηθεν δια του προ} & 7 &  &  \\
&  & 7 & \foreignlanguage{greek}{φητου λεγοντοϲ ανοιξω εν παραβο} & 11 &  &  \\
&  & 11 & \foreignlanguage{greek}{λαιϲ το ϲτομα μου ερευξομαι κεκρυμ} & 16 &  &  \\
&  & 16 & \foreignlanguage{greek}{μενα απο καταβοληϲ κοϲμου} & 19 &  &  \\
& \textbf{36} &  & \foreignlanguage{greek}{τοτε αφειϲ τουϲ οχλουϲ ηλθεν ειϲ την} & 7 &  &  \\
&  & 8 & \foreignlanguage{greek}{οικειαν ο \textoverline{ιϲ} και προϲηλθον αυτω οι} & 14 &  &  \\
&  & 15 & \foreignlanguage{greek}{μαθηται αυτου λεγοντεϲ φραϲον} & 18 &  &  \\
&  & 19 & \foreignlanguage{greek}{ημιν την παραβολην των ζιζανι} & 23 &  &  \\
&  & 23 & \foreignlanguage{greek}{ων του αγρου} & 25 &  &  \\
& \textbf{37} &  & \foreignlanguage{greek}{ο δε αποκριθειϲ ειπεν αυτοιϲ ο ϲπιρω̅} & 7 &  &  \\
&  & 8 & \foreignlanguage{greek}{το καλον ϲπερμα εϲτιν ο υιοϲ του αν} & 15 &  &  \\
&  & 15 & \foreignlanguage{greek}{θρωπου ο δε αγροϲ εϲτιν ο κοϲμοϲ} & 6 & \textbf{38} &  \\
&  & 7 & \foreignlanguage{greek}{το δε καλον ϲπερμα ουτοι ειϲιν οι υ} & 14 &  &  \\
&  & 14 & \foreignlanguage{greek}{ιοι τηϲ βαϲιλειαϲ τα δε ζιζανια ειϲι̅} & 20 &  &  \\
&  & 22 & \foreignlanguage{greek}{οι υιοι του πονηρου ο δε εχθροϲ ο ϲπει} & 5 & \textbf{39} &  \\
&  & 5 & \foreignlanguage{greek}{ραϲ αυτα εϲτιν ο διαβολοϲ ο δε θε} & 12 &  &  \\
&  & 12 & \foreignlanguage{greek}{ριϲμοϲ ϲυντελεια του αιωνοϲ εϲτιν} & 16 &  &  \\
&  & 17 & \foreignlanguage{greek}{οι δε θεριϲται αγγελοι ειϲιν ωϲπερ} & 1 & \textbf{40} &  \\
&  & 2 & \foreignlanguage{greek}{ουν ϲυλλεγεται τα ζιζανια και πυ} & 7 &  &  \\
&  & 7 & \foreignlanguage{greek}{ρι καιεται ουτωϲ εϲται εν τη ϲυν} & 13 &  &  \\
&  & 13 & \foreignlanguage{greek}{τελεια του αιωνοϲ τουτου} & 16 &  &  \\
& \textbf{41} &  & \foreignlanguage{greek}{και αποϲτελει ο υιοϲ του ανθρωπου} & 6 &  &  \\
&  & 7 & \foreignlanguage{greek}{τουϲ αγγελουϲ αυτου και ϲυλλεξουϲι̅} & 11 &  &  \\
&  & 12 & \foreignlanguage{greek}{εκ τηϲ βαϲιλειαϲ αυτου παντα τα} & 17 &  &  \\
&  & 18 & \foreignlanguage{greek}{ϲκανδαλα και τουϲ ποιουνταϲ την} & 22 &  &  \\
&  & 23 & \foreignlanguage{greek}{ανομιαν και βαλουϲιν αυτουϲ ειϲ τη̅} & 5 & \textbf{42} &  \\
[0.2em]
\cline{4-4}
\end{tabular}
\end{center}
\end{table}
}
\clearpage
\newpage
 {
 \setlength\arrayrulewidth{1pt}
\begin{table}
\begin{center}
\begin{tabular}{ccc|l|ccc}
\cline{4-4} \\ [-1em]
\multicolumn{7}{c}{\foreignlanguage{greek}{ευαγγελιον κατα μαθθαιον} \textbf{(\nospace{13:42})} } \\ \\ [-1em] % Si on veut ajouter les bordures latérales, remplacer {7}{c} par {7}{|c|}
\cline{4-4} \\
\cline{4-4}
&  &  & &  &  & \\ [-0.9em]
&  & 6 & \foreignlanguage{greek}{καμινον του πυροϲ εκει εϲται ο κλα} & 12 &  &  \\
&  & 12 & \foreignlanguage{greek}{θμοϲ και ο βρυγμοϲ των οδοντων} & 17 &  &  \\
& \textbf{43} &  & \foreignlanguage{greek}{τοτε οι δικαιοι εκλαμψουϲιν ωϲ ο ηλιοϲ} & 7 &  &  \\
&  & 8 & \foreignlanguage{greek}{εν τη βαϲιλεια του \textoverline{πρϲ} αυτων ο εχων} & 15 &  &  \\
&  & 16 & \foreignlanguage{greek}{ωτα ακουειν ακουετω} & 18 &  &  \\
& \textbf{44} &  & \foreignlanguage{greek}{παλιν ομοια εϲτιν η βαϲιλεια των ουρα} & 7 &  &  \\
&  & 7 & \foreignlanguage{greek}{νων θηϲαυρω κεκρυμμενω εν τω α} & 12 &  &  \\
&  & 12 & \foreignlanguage{greek}{γρω ον ευρων ανθρωποϲ εκρυψεν} & 16 &  &  \\
&  & 17 & \foreignlanguage{greek}{και απο τηϲ χαραϲ αυτου υπαγει και} & 23 &  &  \\
&  & 24 & \foreignlanguage{greek}{παντα οϲα εχει πωλει και αγοραζει το̅} & 30 &  &  \\
&  & 31 & \foreignlanguage{greek}{αγρον εκεινον} & 32 &  &  \\
& \textbf{45} &  & \foreignlanguage{greek}{παλιν ομοια εϲτιν η βαϲιλεια των ου} & 7 &  &  \\
&  & 7 & \foreignlanguage{greek}{ρανων \textoverline{ανω} εμπορω ζητουντι κα} & 11 &  &  \\
&  & 11 & \foreignlanguage{greek}{λουϲ μαργαρειταϲ οϲ ευρων ενα πο} & 4 & \textbf{46} &  \\
&  & 4 & \foreignlanguage{greek}{λυτιμιον μαργαριτην απελθων πε} & 7 &  &  \\
&  & 7 & \foreignlanguage{greek}{πρακεν παντα οϲα ειχεν και ηγορα} & 12 &  &  \\
&  & 12 & \foreignlanguage{greek}{ϲεν αυτον παλιν ομοια εϲτιν} & 3 & \textbf{47} &  \\
&  & 4 & \foreignlanguage{greek}{η βαϲιλεια των ουρανων ϲαγηνη} & 8 &  &  \\
&  & 9 & \foreignlanguage{greek}{βληθειϲη ειϲ την θαλαϲϲαν και εκ} & 14 &  &  \\
&  & 15 & \foreignlanguage{greek}{παντοϲ γενουϲ ϲυναγαγουϲη ην οτε} & 2 & \textbf{48} &  \\
&  & 3 & \foreignlanguage{greek}{επληρωθη αναβιβαϲαντεϲ επι τον} & 6 &  &  \\
&  & 7 & \foreignlanguage{greek}{εγιαλον και καθειϲαντεϲ ϲυνελε} & 10 &  &  \\
&  & 10 & \foreignlanguage{greek}{ξαν τα καλα ειϲ αγγια τα δε ϲαπρα} & 17 &  &  \\
&  & 18 & \foreignlanguage{greek}{εξω εβαλον ουτωϲ εϲται εν τη} & 4 & \textbf{49} &  \\
&  & 5 & \foreignlanguage{greek}{ϲυντελια του αιωνοϲ εξελευϲονται} & 8 &  &  \\
&  & 9 & \foreignlanguage{greek}{οι αγγελοι και αφοριουϲιν τουϲ πονη} & 14 &  &  \\
&  & 14 & \foreignlanguage{greek}{ρουϲ εκ μεϲου των δικαιων και βα} & 2 & \textbf{50} &  \\
&  & 2 & \foreignlanguage{greek}{λουϲιν αυτουϲ ειϲ την καμινον του} & 7 &  &  \\
&  & 8 & \foreignlanguage{greek}{πυροϲ εκει εϲται ο κλαθμοϲ και ο} & 14 &  &  \\
&  & 15 & \foreignlanguage{greek}{βρυγμοϲ των οδοντων} & 17 &  &  \\
[0.2em]
\cline{4-4}
\end{tabular}
\end{center}
\end{table}
}
\clearpage
\newpage
 {
 \setlength\arrayrulewidth{1pt}
\begin{table}
\begin{center}
\begin{tabular}{ccc|l|ccc}
\cline{4-4} \\ [-1em]
\multicolumn{7}{c}{\foreignlanguage{greek}{ευαγγελιον κατα μαθθαιον} \textbf{(\nospace{13:51})} } \\ \\ [-1em] % Si on veut ajouter les bordures latérales, remplacer {7}{c} par {7}{|c|}
\cline{4-4} \\
\cline{4-4}
&  &  & &  &  & \\ [-0.9em]
& \textbf{51} &  & \foreignlanguage{greek}{λεγει αυτοιϲ ο \textoverline{ιϲ} ϲυνηκατε ταυτα παντα} & 7 &  &  \\
&  & 8 & \foreignlanguage{greek}{λεγουϲιν αυτω ναι \textoverline{κε} ο δε ειπεν αυτοιϲ} & 4 & \textbf{52} &  \\
&  & 5 & \foreignlanguage{greek}{δια τουτο παϲ γραμματευϲ μαθητευ} & 9 &  &  \\
&  & 9 & \foreignlanguage{greek}{θειϲ τη βαϲιλεια των ουρανων ομοι} & 14 &  &  \\
&  & 14 & \foreignlanguage{greek}{οϲ εϲτιν ανθρωπω οικοδεϲποτη} & 17 &  &  \\
&  & 18 & \foreignlanguage{greek}{οϲτιϲ εκβαλει εκ του θηϲαυρου αυ} & 23 &  &  \\
&  & 23 & \foreignlanguage{greek}{του καινα και παλαια} & 26 &  &  \\
& \textbf{53} &  & \foreignlanguage{greek}{και εγενετο οτε ετελεϲεν ο \textoverline{ιϲ} ταϲ} & 7 &  &  \\
&  & 8 & \foreignlanguage{greek}{παραβολαϲ ταυταϲ μετηρεν εκειθεν} & 11 &  &  \\
& \textbf{54} &  & \foreignlanguage{greek}{και ελθων ειϲ την πατριδα αυτου} & 6 &  &  \\
&  & 7 & \foreignlanguage{greek}{εδιδαϲκεν αυτουϲ εν τη ϲυναγω} & 11 &  &  \\
&  & 11 & \foreignlanguage{greek}{γη αυτων ωϲτε εκπληϲϲεϲθαι} & 14 &  &  \\
&  & 15 & \foreignlanguage{greek}{αυτουϲ και λεγειν ποθεν τουτω} & 19 &  &  \\
&  & 20 & \foreignlanguage{greek}{ταυτα και τιϲ η ϲοφια αυτη και αι} & 27 &  &  \\
&  & 28 & \foreignlanguage{greek}{δυναμειϲ ουχ ουτοϲ εϲτιν ο του} & 5 & \textbf{55} &  \\
&  & 6 & \foreignlanguage{greek}{τεκτονοϲ υιοϲ ουχ η \textoverline{μηρ} αυτου} & 11 &  &  \\
&  & 12 & \foreignlanguage{greek}{λεγεται μαριαμ και οι αδελφοι αυ} & 17 &  &  \\
&  & 17 & \foreignlanguage{greek}{του ιακωβοϲ και ιωϲηϲ και ϲιμων} & 22 &  &  \\
&  & 23 & \foreignlanguage{greek}{και ιουδαϲ και αι αδελφαι αυτου} & 4 & \textbf{56} &  \\
&  & 5 & \foreignlanguage{greek}{ουχι παϲαι προϲ ημαϲ ειϲιν} & 9 &  &  \\
&  & 10 & \foreignlanguage{greek}{ποθεν ουν τουτω παντα ταυτα και} & 1 & \textbf{57} &  \\
&  & 2 & \foreignlanguage{greek}{εϲκανδαλιζοντο επ αυτω} & 4 &  &  \\
&  & 5 & \foreignlanguage{greek}{ο δε \textoverline{ιϲ} ειπεν αυτοιϲ ουκ εϲτιν προ} & 12 &  &  \\
&  & 12 & \foreignlanguage{greek}{φητηϲ ατιμοϲ ει μη εν τη πατριδι} & 18 &  &  \\
&  & 19 & \foreignlanguage{greek}{αυτου και εν τη οικεια αυτου} & 24 &  &  \\
& \textbf{58} &  & \foreignlanguage{greek}{και ουκ εποιηϲεν εκει δυναμειϲ πολ} & 6 &  &  \\
&  & 6 & \foreignlanguage{greek}{λαϲ δια την απιϲτιαν αυτων} & 10 &  &  \\
& \mygospelchapter &  & \foreignlanguage{greek}{εν εκεινω τω καιρω ηκουϲεν ηρωδηϲ} & 6 &  &  \\
&  & 7 & \foreignlanguage{greek}{ο τετραρχηϲ την ακοην \textoverline{ιυ} και ειπε̅} & 2 & \textbf{2} &  \\
&  & 3 & \foreignlanguage{greek}{τοιϲ παιϲιν αυτου ουτοϲ εϲτιν ιωαννηϲ} & 8 &  &  \\
[0.2em]
\cline{4-4}
\end{tabular}
\end{center}
\end{table}
}
\clearpage
\newpage
 {
 \setlength\arrayrulewidth{1pt}
\begin{table}
\begin{center}
\begin{tabular}{ccc|l|ccc}
\cline{4-4} \\ [-1em]
\multicolumn{7}{c}{\foreignlanguage{greek}{ευαγγελιον κατα μαθθαιον} \textbf{(\nospace{14:2})} } \\ \\ [-1em] % Si on veut ajouter les bordures latérales, remplacer {7}{c} par {7}{|c|}
\cline{4-4} \\
\cline{4-4}
&  &  & &  &  & \\ [-0.9em]
&  & 9 & \foreignlanguage{greek}{ο βαπτιϲτηϲ αυτοϲ ηγερθη απο των νε} & 15 &  &  \\
&  & 15 & \foreignlanguage{greek}{κρων και δια τουτο αι δυναμειϲ ε} & 21 &  &  \\
&  & 21 & \foreignlanguage{greek}{νεργουϲιν εν αυτω} & 23 &  &  \\
& \textbf{3} &  & \foreignlanguage{greek}{ο γαρ ηρωδηϲ κρατηϲαϲ τον ιωαννην} & 6 &  &  \\
&  & 7 & \foreignlanguage{greek}{εδηϲεν αυτον και εθετο εν φυλακη} & 12 &  &  \\
&  & 13 & \foreignlanguage{greek}{δια ηρωιαδα την γυναικα φιλιππου} & 17 &  &  \\
&  & 18 & \foreignlanguage{greek}{του αδελφου αυτου} & 20 &  &  \\
& \textbf{4} &  & \foreignlanguage{greek}{ελεγεν γαρ αυτω ο ιωαννηϲ ουκ εξε} & 7 &  &  \\
&  & 7 & \foreignlanguage{greek}{ϲτιν ϲοι εχειν αυτην και θελων αυτο̅} & 3 & \textbf{5} &  \\
&  & 4 & \foreignlanguage{greek}{αποκτειναι εφοβηθη τον οχλον οτι} & 8 &  &  \\
&  & 9 & \foreignlanguage{greek}{ωϲ προφητην αυτον ειχον} & 12 &  &  \\
& \textbf{6} &  & \foreignlanguage{greek}{γενεϲιων δε αγομενων του ηρωδου} & 5 &  &  \\
&  & 6 & \foreignlanguage{greek}{ωρχηϲατο η θυγατηρ ηρωδιαδοϲ εν} & 10 &  &  \\
&  & 11 & \foreignlanguage{greek}{τω μεϲω και ηρεϲεν τω ηρωδη} & 16 &  &  \\
& \textbf{7} &  & \foreignlanguage{greek}{οθεν μεθ ορκου ωμολογηϲεν δου} & 5 &  &  \\
&  & 5 & \foreignlanguage{greek}{ναι αυτη ο εαν αιτηϲηται} & 9 &  &  \\
& \textbf{8} &  & \foreignlanguage{greek}{η δε προβιβαϲθειϲα υπο τηϲ μητροϲ} & 6 &  &  \\
&  & 7 & \foreignlanguage{greek}{αυτηϲ ειπεν δοϲ μοι φηϲιν ωδε} & 12 &  &  \\
&  & 13 & \foreignlanguage{greek}{επι πινακει την κεφαλην ιωαννου} & 17 &  &  \\
&  & 18 & \foreignlanguage{greek}{του βαπτιϲτου και ελυπηθη ο βαϲι} & 4 & \textbf{9} &  \\
&  & 4 & \foreignlanguage{greek}{λευϲ δια δε τουϲ ορκουϲ και τουϲ} & 10 &  &  \\
&  & 11 & \foreignlanguage{greek}{ϲυνανακειμενουϲ εκελευϲεν δο} & 13 &  &  \\
&  & 13 & \foreignlanguage{greek}{θηναι και πεμψαϲ απεκεφαλι} & 3 & \textbf{10} &  \\
&  & 3 & \foreignlanguage{greek}{ϲεν τον ιωαννην εν τη φυλακη} & 8 &  &  \\
& \textbf{11} &  & \foreignlanguage{greek}{και ηνεχθη η κεφαλη αυτου επι πι} & 7 &  &  \\
&  & 7 & \foreignlanguage{greek}{νακει και εδοθη τω κοραϲιω και η} & 13 &  &  \\
&  & 13 & \foreignlanguage{greek}{νεγκεν τη μητρι αυτηϲ} & 16 &  &  \\
& \textbf{12} &  & \foreignlanguage{greek}{και προϲελθοντεϲ οι μαθηται αυτου} & 5 &  &  \\
&  & 6 & \foreignlanguage{greek}{ηραν το ϲωμα και εθαψαν αυτο και} & 12 &  &  \\
&  & 13 & \foreignlanguage{greek}{ελθοντεϲ απηγγειλαν τω \textoverline{ιυ}} & 16 &  &  \\
[0.2em]
\cline{4-4}
\end{tabular}
\end{center}
\end{table}
}
\clearpage
\newpage
 {
 \setlength\arrayrulewidth{1pt}
\begin{table}
\begin{center}
\begin{tabular}{ccc|l|ccc}
\cline{4-4} \\ [-1em]
\multicolumn{7}{c}{\foreignlanguage{greek}{ευαγγελιον κατα μαθθαιον} \textbf{(\nospace{14:13})} } \\ \\ [-1em] % Si on veut ajouter les bordures latérales, remplacer {7}{c} par {7}{|c|}
\cline{4-4} \\
\cline{4-4}
&  &  & &  &  & \\ [-0.9em]
& \textbf{13} &  & \foreignlanguage{greek}{και ακουϲαϲ ο \textoverline{ιϲ} ανεχωρηϲεν εκειθεν} & 6 &  &  \\
&  & 7 & \foreignlanguage{greek}{εν πλοιω ειϲ ερημον τοπον κατ ιδιαν} & 13 &  &  \\
&  & 14 & \foreignlanguage{greek}{και ακουϲαντεϲ οι οχλοι ηκολουθηϲα̅} & 18 &  &  \\
&  & 19 & \foreignlanguage{greek}{αυτω πεζη απο των πολεων} & 23 &  &  \\
& \textbf{14} &  & \foreignlanguage{greek}{και εξελθων ο \textoverline{ιϲ} ιδεν πολυν οχλον και} & 8 &  &  \\
&  & 9 & \foreignlanguage{greek}{εϲπλαγχνιϲθη επ αυτοιϲ και εθεραπευ} & 13 &  &  \\
&  & 13 & \foreignlanguage{greek}{ϲεν τουϲ αρρωϲτουϲ αυτων} & 16 &  &  \\
& \textbf{15} &  & \foreignlanguage{greek}{οψιαϲ δε γενομενηϲ προϲηλθον αυτω} & 5 &  &  \\
&  & 6 & \foreignlanguage{greek}{οι μαθηται αυτου λεγοντεϲ ερημοϲ} & 10 &  &  \\
&  & 11 & \foreignlanguage{greek}{εϲτιν ο τοποϲ και η ωρα ηδη παρηλθε̅} & 18 &  &  \\
&  & 19 & \foreignlanguage{greek}{απολυϲον τουϲ οχλουϲ ινα απελθον} & 23 &  &  \\
&  & 23 & \foreignlanguage{greek}{τεϲ ειϲ ταϲ κωμαϲ αγοραϲωϲιν εαυτοιϲ} & 28 &  &  \\
&  & 29 & \foreignlanguage{greek}{βρωματα} & 29 &  &  \\
& \textbf{16} &  & \foreignlanguage{greek}{ο δε \textoverline{ιϲ} ειπεν αυτοιϲ ου χρειαν εχουϲι̅} & 8 &  &  \\
&  & 9 & \foreignlanguage{greek}{απελθειν δοτε αυτοιϲ υμειϲ φαγειν} & 13 &  &  \\
& \textbf{17} &  & \foreignlanguage{greek}{οι δε λεγουϲιν αυτω ουκ εχομεν ωδε} & 7 &  &  \\
&  & 8 & \foreignlanguage{greek}{ει μη πεντε αρτουϲ και δυο ιχθυαϲ} & 14 &  &  \\
& \textbf{18} &  & \foreignlanguage{greek}{ο δε ειπεν φερεται μοι αυτουϲ ωδε} & 7 &  &  \\
& \textbf{19} &  & \foreignlanguage{greek}{και κελευϲαϲ τουϲ οχλουϲ ανακλιθη} & 5 &  &  \\
&  & 5 & \foreignlanguage{greek}{ναι επι του χορτου και λαβων τουϲ} & 11 &  &  \\
&  & 12 & \foreignlanguage{greek}{πεντε αρτουϲ και τουϲ δυο ιχθυαϲ} & 17 &  &  \\
&  & 18 & \foreignlanguage{greek}{αναβλεψαϲ ειϲ τον ουρανον ηυλογη} & 22 &  &  \\
&  & 22 & \foreignlanguage{greek}{ϲεν και κλαϲαϲ εδωκεν τοιϲ μαθη} & 27 &  &  \\
&  & 27 & \foreignlanguage{greek}{ταιϲ τουϲ αρτουϲ οι δε μαθηται τοιϲ} & 33 &  &  \\
&  & 34 & \foreignlanguage{greek}{οχλοιϲ και εφαγον παντεϲ και εχορ} & 5 & \textbf{20} &  \\
&  & 5 & \foreignlanguage{greek}{ταϲθηϲαν και ηραν το περιϲϲευον} & 9 &  &  \\
&  & 10 & \foreignlanguage{greek}{των κλαϲματων δωδεκα κοφινουϲ} & 13 &  &  \\
&  & 14 & \foreignlanguage{greek}{πληρειϲ οι δε εϲθιοντεϲ ηϲαν ανδρεϲ} & 5 & \textbf{21} &  \\
&  & 6 & \foreignlanguage{greek}{πεντακιϲχειλιοι χωριϲ γυναικων} & 8 &  &  \\
&  & 9 & \foreignlanguage{greek}{και παιδιων} & 10 &  &  \\
[0.2em]
\cline{4-4}
\end{tabular}
\end{center}
\end{table}
}
\clearpage
\newpage
 {
 \setlength\arrayrulewidth{1pt}
\begin{table}
\begin{center}
\begin{tabular}{ccc|l|ccc}
\cline{4-4} \\ [-1em]
\multicolumn{7}{c}{\foreignlanguage{greek}{ευαγγελιον κατα μαθθαιον} \textbf{(\nospace{14:22})} } \\ \\ [-1em] % Si on veut ajouter les bordures latérales, remplacer {7}{c} par {7}{|c|}
\cline{4-4} \\
\cline{4-4}
&  &  & &  &  & \\ [-0.9em]
& \textbf{22} &  & \foreignlanguage{greek}{και ευθεωϲ ηναγκαϲεν τουϲ μαθηταϲ} & 5 &  &  \\
&  & 6 & \foreignlanguage{greek}{ενβηναι ειϲ το πλοιον και προαγειν} & 11 &  &  \\
&  & 12 & \foreignlanguage{greek}{αυτον ειϲ το περαν εωϲ ου απολυϲη} & 18 &  &  \\
&  & 19 & \foreignlanguage{greek}{τουϲ οχλουϲ και απολυϲαϲ τουϲ ο} & 4 & \textbf{23} &  \\
&  & 4 & \foreignlanguage{greek}{χλουϲ ανεβη ειϲ το οροϲ κατ ειδιαν} & 10 &  &  \\
&  & 11 & \foreignlanguage{greek}{προϲευξαϲθαι οψειαϲ δε γενομενηϲ} & 14 &  &  \\
&  & 15 & \foreignlanguage{greek}{μονοϲ ην εκει το δε πλοιον ηδη με} & 5 & \textbf{24} &  \\
&  & 5 & \foreignlanguage{greek}{ϲον τηϲ θαλαϲϲηϲ ην βαϲανιζομε} & 9 &  &  \\
&  & 9 & \foreignlanguage{greek}{νον υπο των κυματων ην γαρ εναν} & 16 &  &  \\
&  & 16 & \foreignlanguage{greek}{τιοϲ ο ανεμοϲ} & 18 &  &  \\
& \textbf{25} &  & \foreignlanguage{greek}{τεταρτη ουν φυλακη τηϲ νυκτοϲ} & 5 &  &  \\
&  & 6 & \foreignlanguage{greek}{απηλθεν προϲ αυτουϲ περιπατων ε} & 10 &  &  \\
&  & 10 & \foreignlanguage{greek}{πι την θαλαϲϲαν και ιδοντεϲ αυτο̅} & 3 & \textbf{26} &  \\
&  & 4 & \foreignlanguage{greek}{οι μαθηται επι την θαλαϲϲαν περιπα} & 9 &  &  \\
&  & 9 & \foreignlanguage{greek}{τουντα εταραχθηϲαν λεγοντεϲ οτι} & 12 &  &  \\
&  & 13 & \foreignlanguage{greek}{φανταϲμα εϲτιν και απο του φο} & 18 &  &  \\
&  & 18 & \foreignlanguage{greek}{βου εκραξαν ευθεωϲ δε ελαλη} & 3 & \textbf{27} &  \\
&  & 3 & \foreignlanguage{greek}{ϲεν αυτοιϲ ο \textoverline{ιϲ} λεγων θαρϲειται εγω} & 9 &  &  \\
&  & 10 & \foreignlanguage{greek}{ειμει μη φοβιϲθαι} & 12 &  &  \\
& \textbf{28} &  & \foreignlanguage{greek}{αποκριθειϲ δε αυτω ο πετροϲ ειπεν} & 6 &  &  \\
&  & 7 & \foreignlanguage{greek}{\textoverline{κε} ει ϲυ ει κελευϲον με ελθειν προϲ ϲε} & 15 &  &  \\
&  & 16 & \foreignlanguage{greek}{επι τα υδατα ο δε ειπεν ελθε} & 4 & \textbf{29} &  \\
&  & 5 & \foreignlanguage{greek}{και καταβαϲ απο του πλοιου ο πετροϲ} & 11 &  &  \\
&  & 12 & \foreignlanguage{greek}{περιεπατηϲεν επι τα υδατα ελθειν} & 16 &  &  \\
&  & 17 & \foreignlanguage{greek}{προϲ τον \textoverline{ιν} βλεπων δε τον ανε} & 4 & \textbf{30} &  \\
&  & 4 & \foreignlanguage{greek}{μον ιϲχυρον ϲφοδρα εφοβηθη ελθει̅} & 8 &  &  \\
&  & 9 & \foreignlanguage{greek}{και αρξαμενοϲ καταποντιζεϲθαι} & 11 &  &  \\
&  & 12 & \foreignlanguage{greek}{εκραξεν λεγων \textoverline{κε} ϲωϲον με} & 16 &  &  \\
& \textbf{31} &  & \foreignlanguage{greek}{ευθεωϲ δε ο \textoverline{ιϲ} εκτιναϲ την χειρα ε} & 8 &  &  \\
&  & 8 & \foreignlanguage{greek}{πελαβετο αυτου και λεγει αυτω} & 12 &  &  \\
[0.2em]
\cline{4-4}
\end{tabular}
\end{center}
\end{table}
}
\clearpage
\newpage
 {
 \setlength\arrayrulewidth{1pt}
\begin{table}
\begin{center}
\begin{tabular}{ccc|l|ccc}
\cline{4-4} \\ [-1em]
\multicolumn{7}{c}{\foreignlanguage{greek}{ευαγγελιον κατα μαθθαιον} \textbf{(\nospace{14:31})} } \\ \\ [-1em] % Si on veut ajouter les bordures latérales, remplacer {7}{c} par {7}{|c|}
\cline{4-4} \\
\cline{4-4}
&  &  & &  &  & \\ [-0.9em]
&  & 13 & \foreignlanguage{greek}{ολιγοπιϲτε ειϲ τι εδιϲταϲαϲ και ενβαν} & 2 & \textbf{32} &  \\
&  & 2 & \foreignlanguage{greek}{των αυτων ειϲ το πλοιον εκοπαϲεν ο} & 8 &  &  \\
&  & 9 & \foreignlanguage{greek}{ανεμοϲ οι δε εν τω πλοιω ελθοντεϲ} & 6 & \textbf{33} &  \\
&  & 7 & \foreignlanguage{greek}{προϲεκυνηϲαν αυτω λεγοντεϲ αλη} & 10 &  &  \\
&  & 10 & \foreignlanguage{greek}{θωϲ \textoverline{θυ} υιοϲ ει και διαπεραϲαντεϲ} & 2 & \textbf{34} &  \\
&  & 3 & \foreignlanguage{greek}{ηλθον επι την γην ειϲ γεννηϲαρετ} & 8 &  &  \\
& \textbf{35} &  & \foreignlanguage{greek}{και επιγνοντεϲ αυτον οι ανδρεϲ} & 5 &  &  \\
&  & 6 & \foreignlanguage{greek}{του τοπου εκεινου απεϲτιλαν} & 9 &  &  \\
&  & 10 & \foreignlanguage{greek}{ειϲ ολην την περιχωρον εκεινην} & 14 &  &  \\
&  & 15 & \foreignlanguage{greek}{και προϲηνεγκαν αυτω πανταϲ τουϲ} & 19 &  &  \\
&  & 20 & \foreignlanguage{greek}{κακωϲ εχονταϲ και παρεκαλουν} & 2 & \textbf{36} &  \\
&  & 3 & \foreignlanguage{greek}{αυτον ινα μονον αψωνται του} & 7 &  &  \\
&  & 8 & \foreignlanguage{greek}{κραϲπεδου του ιματιου αυτου και ο} & 13 &  &  \\
&  & 13 & \foreignlanguage{greek}{ϲοι ηψαντο διελωθηϲαν} & 15 &  &  \\
& \mygospelchapter &  & \foreignlanguage{greek}{τοτε προϲερχονται τω \textoverline{ιυ} οι απο ιερο} & 7 &  &  \\
&  & 7 & \foreignlanguage{greek}{ϲολυμων γραμματιϲ και φαριϲαιοι} & 10 &  &  \\
&  & 11 & \foreignlanguage{greek}{λεγοντεϲ δια τι οι μαθηται ϲου} & 5 & \textbf{2} &  \\
&  & 6 & \foreignlanguage{greek}{παραβαινουϲιν την παραδοϲιν τω̅} & 9 &  &  \\
&  & 10 & \foreignlanguage{greek}{πρεϲβυτερων ου γαρ νιπτονται} & 13 &  &  \\
&  & 14 & \foreignlanguage{greek}{ταϲ χειραϲ αυτων οταν αρτον εϲ} & 19 &  &  \\
&  & 19 & \foreignlanguage{greek}{θιουϲιν} & 19 &  &  \\
& \textbf{3} &  & \foreignlanguage{greek}{ο δε αποκριθειϲ ειπεν αυτοιϲ δια} & 6 &  &  \\
&  & 7 & \foreignlanguage{greek}{τι και υμειϲ παραβαινεται την εν} & 12 &  &  \\
&  & 12 & \foreignlanguage{greek}{τολην του \textoverline{θυ} δια την παραδοϲιν} & 17 &  &  \\
&  & 18 & \foreignlanguage{greek}{υμων ο γαρ \textoverline{θϲ} ενετιλατο λεγων} & 5 & \textbf{4} &  \\
&  & 6 & \foreignlanguage{greek}{τιμα τον \textoverline{πρα} ϲου και την μητερα ϲου} & 13 &  &  \\
&  & 14 & \foreignlanguage{greek}{και ο κακολογων \textoverline{πρα} η \textoverline{μρα} θανατω} & 20 &  &  \\
&  & 21 & \foreignlanguage{greek}{τελευτατω υμειϲ δε λεγεται} & 3 & \textbf{5} &  \\
&  & 4 & \foreignlanguage{greek}{οϲ εαν ειπη τω \textoverline{πρι} η τη μητρι δωρο̅} & 12 &  &  \\
&  & 13 & \foreignlanguage{greek}{ο αν εξ εμου ωφεληθηϲ και ου μη} & 3 & \textbf{6} &  \\
[0.2em]
\cline{4-4}
\end{tabular}
\end{center}
\end{table}
}
\clearpage
\newpage
 {
 \setlength\arrayrulewidth{1pt}
\begin{table}
\begin{center}
\begin{tabular}{ccc|l|ccc}
\cline{4-4} \\ [-1em]
\multicolumn{7}{c}{\foreignlanguage{greek}{ευαγγελιον κατα μαθθαιον} \textbf{(\nospace{15:6})} } \\ \\ [-1em] % Si on veut ajouter les bordures latérales, remplacer {7}{c} par {7}{|c|}
\cline{4-4} \\
\cline{4-4}
&  &  & &  &  & \\ [-0.9em]
&  & 4 & \foreignlanguage{greek}{τιμηϲει τον \textoverline{πρα} αυτου η την μητερα} & 10 &  &  \\
&  & 11 & \foreignlanguage{greek}{αυτου και ηκυρωϲατε την εντολη̅} & 15 &  &  \\
&  & 16 & \foreignlanguage{greek}{του \textoverline{θυ} δια την παραδοϲιν υμων} & 21 &  &  \\
& \textbf{7} &  & \foreignlanguage{greek}{υποκριται καλωϲ προεφητευϲεν} & 3 &  &  \\
&  & 4 & \foreignlanguage{greek}{περι υμων ηϲαιαϲ λεγων εγγιζει} & 1 & \textbf{8} &  \\
&  & 2 & \foreignlanguage{greek}{μοι ο λαοϲ ουτοϲ τω ϲτοματι αυτων και} & 9 &  &  \\
&  & 10 & \foreignlanguage{greek}{τοιϲ χειλεϲιν με τιμα η δε καρδια αυτω̅} & 17 &  &  \\
&  & 18 & \foreignlanguage{greek}{πορρω απεχει απ εμου ματην δε ϲε} & 3 & \textbf{9} &  \\
&  & 3 & \foreignlanguage{greek}{βονται με διδαϲκοντεϲ διδαϲκαλιαϲ} & 6 &  &  \\
&  & 7 & \foreignlanguage{greek}{ενταλματα ανθρωπων} & 8 &  &  \\
& \textbf{10} &  & \foreignlanguage{greek}{και προϲκαλεϲαμενοϲ τον οχλον ειπεν} & 5 &  &  \\
&  & 6 & \foreignlanguage{greek}{αυτοιϲ ακουεται και ϲυνιεται ου το} & 2 & \textbf{11} &  \\
&  & 3 & \foreignlanguage{greek}{ειϲερχομενον ειϲ το ϲτομα κοινοι το̅} & 8 &  &  \\
&  & 9 & \foreignlanguage{greek}{\textoverline{ανον} αλλα το εκπορευομενον εκ του} & 14 &  &  \\
&  & 15 & \foreignlanguage{greek}{ϲτοματοϲ τουτο κοινοι τον \textoverline{ανον}} & 19 &  &  \\
& \textbf{12} &  & \foreignlanguage{greek}{τοτε προϲελθοντεϲ οι μαθηται αυτου} & 5 &  &  \\
&  & 6 & \foreignlanguage{greek}{ειπον αυτω οιδαϲ οτι οι φαριϲαιοι} & 11 &  &  \\
&  & 12 & \foreignlanguage{greek}{ακουϲαντεϲ τον λογον εϲκανδαλι} & 15 &  &  \\
&  & 15 & \foreignlanguage{greek}{ϲθηϲαν ο δε αποκριθειϲ ειπεν} & 4 & \textbf{13} &  \\
&  & 5 & \foreignlanguage{greek}{παϲα φυτια ην ουκ εφυτευϲεν ο \textoverline{πηρ}} & 11 &  &  \\
&  & 12 & \foreignlanguage{greek}{μου ο ουρανιοϲ εκριζωθηϲεται} & 15 &  &  \\
& \textbf{14} &  & \foreignlanguage{greek}{αφεται αυτουϲ οδηγοι ειϲιν τυφλοι} & 5 &  &  \\
&  & 6 & \foreignlanguage{greek}{τυφλων τυφλοϲ δε τυφλον εαν} & 10 &  &  \\
&  & 11 & \foreignlanguage{greek}{οδηγη αμφοτεροι ειϲ βοθυνον εμ} & 15 &  &  \\
&  & 15 & \foreignlanguage{greek}{πεϲουνται} & 15 &  &  \\
& \textbf{15} &  & \foreignlanguage{greek}{αποκριθειϲ δε ο πετροϲ ειπεν αυτω} & 6 &  &  \\
&  & 7 & \foreignlanguage{greek}{φραϲον ημιν την παραβολην ταυτη̅} & 11 &  &  \\
& \textbf{16} &  & \foreignlanguage{greek}{ο δε \textoverline{ιϲ} ειπεν ακμην και υμειϲ αϲυνε} & 8 &  &  \\
&  & 8 & \foreignlanguage{greek}{τοι εϲται ουπω νοειται οτι παν το ειϲ} & 6 & \textbf{17} &  \\
&  & 6 & \foreignlanguage{greek}{πορευομενον ειϲ το ϲτομα ειϲ την κοι} & 12 &  &  \\
[0.2em]
\cline{4-4}
\end{tabular}
\end{center}
\end{table}
}
\clearpage
\newpage
 {
 \setlength\arrayrulewidth{1pt}
\begin{table}
\begin{center}
\begin{tabular}{ccc|l|ccc}
\cline{4-4} \\ [-1em]
\multicolumn{7}{c}{\foreignlanguage{greek}{ευαγγελιον κατα μαθθαιον} \textbf{(\nospace{15:17})} } \\ \\ [-1em] % Si on veut ajouter les bordures latérales, remplacer {7}{c} par {7}{|c|}
\cline{4-4} \\
\cline{4-4}
&  &  & &  &  & \\ [-0.9em]
&  & 12 & \foreignlanguage{greek}{λιαν χωρει και ειϲ αφεδρωνα εκβαλλε} & 17 &  &  \\
&  & 17 & \foreignlanguage{greek}{ται τα δε εκπορευομενα εκ του ϲτο} & 6 & \textbf{18} &  \\
&  & 6 & \foreignlanguage{greek}{ματοϲ εκ τηϲ καρδιαϲ εξερχονται δια} & 6 & \textbf{19} &  \\
&  & 6 & \foreignlanguage{greek}{λογιϲμοι πονηροι πορνιαι μοιχι} & 9 &  &  \\
&  & 9 & \foreignlanguage{greek}{αι φονοι κλοπαι ψευδομαρτυριαι} & 12 &  &  \\
&  & 13 & \foreignlanguage{greek}{βλαϲφημιαι ταυτα εϲτιν τα κοινου̅} & 4 & \textbf{20} &  \\
&  & 4 & \foreignlanguage{greek}{τα τον ανθρωπον το δε ανιπτοιϲ} & 9 &  &  \\
&  & 10 & \foreignlanguage{greek}{χερϲιν φαγειν ου κοινοι τον \textoverline{ανον}} & 15 &  &  \\
& \textbf{21} &  & \foreignlanguage{greek}{και εξελθων εκειθεν ο \textoverline{ιϲ} ανεχωρηϲε̅} & 6 &  &  \\
&  & 7 & \foreignlanguage{greek}{ειϲ τα μερη τυρου και ϲιδωνοϲ} & 12 &  &  \\
& \textbf{22} &  & \foreignlanguage{greek}{και ιδου γυνη χαναναια απο των οριω̅} & 7 &  &  \\
&  & 8 & \foreignlanguage{greek}{εκεινων εξελθουϲα εκραυγαϲεν} & 10 &  &  \\
&  & 11 & \foreignlanguage{greek}{αυτω λεγουϲα ελεηϲον με \textoverline{κε} υιοϲ} & 16 &  &  \\
&  & 17 & \foreignlanguage{greek}{δαυειδ η θυγατηρ μου κακωϲ δαι} & 22 &  &  \\
&  & 22 & \foreignlanguage{greek}{μονιζεται ο δε ουκ απεκριθη αυτη} & 5 & \textbf{23} &  \\
&  & 6 & \foreignlanguage{greek}{λογον και προϲελθοντεϲ οι μαθη} & 10 &  &  \\
&  & 10 & \foreignlanguage{greek}{ται αυτου ηρωτων αυτον λεγοντεϲ} & 14 &  &  \\
&  & 15 & \foreignlanguage{greek}{απολυϲον αυτην οτι κραζει εμ} & 19 &  &  \\
&  & 19 & \foreignlanguage{greek}{προϲθεν ημων} & 20 &  &  \\
& \textbf{24} &  & \foreignlanguage{greek}{ο δε αποκριθειϲ ειπεν ουκ απεϲτα} & 6 &  &  \\
&  & 6 & \foreignlanguage{greek}{λην ει μη ειϲ τα προβατα τα απολω} & 13 &  &  \\
&  & 13 & \foreignlanguage{greek}{λοτα οικου ιϲραηλ} & 15 &  &  \\
& \textbf{25} &  & \foreignlanguage{greek}{η δε ελθουϲα προϲεκυνηϲεν αυτω} & 5 &  &  \\
&  & 6 & \foreignlanguage{greek}{λεγουϲα \textoverline{κε} βοηθει μοι} & 9 &  &  \\
& \textbf{26} &  & \foreignlanguage{greek}{ο δε αποκριθειϲ ειπεν ουκ εϲτιν κα} & 7 &  &  \\
&  & 7 & \foreignlanguage{greek}{λον λαβειν τον αρτον των τεκνω̅} & 12 &  &  \\
&  & 13 & \foreignlanguage{greek}{και βαλειν τοιϲ κυναριοιϲ} & 16 &  &  \\
& \textbf{27} &  & \foreignlanguage{greek}{η δε ειπεν ναι \textoverline{κε} και γαρ τα κυναρι} & 9 &  &  \\
&  & 9 & \foreignlanguage{greek}{α εϲθιει απο των ψιχιων των πιπτο̅} & 15 &  &  \\
&  & 15 & \foreignlanguage{greek}{των απο τηϲ τραπεζηϲ των κυριων αυτων} & 21 &  &  \\
[0.2em]
\cline{4-4}
\end{tabular}
\end{center}
\end{table}
}
\clearpage
\newpage
 {
 \setlength\arrayrulewidth{1pt}
\begin{table}
\begin{center}
\begin{tabular}{ccc|l|ccc}
\cline{4-4} \\ [-1em]
\multicolumn{7}{c}{\foreignlanguage{greek}{ευαγγελιον κατα μαθθαιον} \textbf{(\nospace{15:28})} } \\ \\ [-1em] % Si on veut ajouter les bordures latérales, remplacer {7}{c} par {7}{|c|}
\cline{4-4} \\
\cline{4-4}
&  &  & &  &  & \\ [-0.9em]
& \textbf{28} &  & \foreignlanguage{greek}{τοτε αποκριθειϲ ο \textoverline{ιϲ} ειπεν αυτη ω γυναι} & 8 &  &  \\
&  & 9 & \foreignlanguage{greek}{μεγαλη ϲου η πιϲτιϲ γενηθητω ϲοι ωϲ} & 15 &  &  \\
&  & 16 & \foreignlanguage{greek}{θελειϲ και ιαθη η θυγατηρ αυτηϲ απο} & 22 &  &  \\
&  & 23 & \foreignlanguage{greek}{τηϲ ωραϲ εκεινηϲ} & 25 &  &  \\
& \textbf{29} &  & \foreignlanguage{greek}{και μεταβαϲ εκειθεν ο \textoverline{ιϲ} ηλθεν παρα} & 7 &  &  \\
&  & 8 & \foreignlanguage{greek}{την θαλαϲϲαν τηϲ γαλιλαιαϲ και α} & 13 &  &  \\
&  & 13 & \foreignlanguage{greek}{ναβαϲ ειϲ το οροϲ εκαθητο εκει και} & 1 & \textbf{30} &  \\
&  & 2 & \foreignlanguage{greek}{προϲηλθον αυτω οχλοι πολλοι εχον} & 6 &  &  \\
&  & 6 & \foreignlanguage{greek}{τεϲ μεθ εαυτων κωφουϲ χωλουϲ} & 10 &  &  \\
&  & 11 & \foreignlanguage{greek}{τυφλουϲ κυλλουϲ και ετερουϲ πολ} & 15 &  &  \\
&  & 15 & \foreignlanguage{greek}{λουϲ και ερριψαν αυτουϲ παρα τουϲ} & 20 &  &  \\
&  & 21 & \foreignlanguage{greek}{ποδαϲ του \textoverline{ιυ} και εθεραπευϲεν αυτουϲ} & 26 &  &  \\
& \textbf{31} &  & \foreignlanguage{greek}{ωϲτε τουϲ οχλουϲ θαυμαϲαι βλεπο̅} & 5 &  &  \\
&  & 5 & \foreignlanguage{greek}{ταϲ κωφουϲ λαλουνταϲ κυλλουϲ} & 8 &  &  \\
&  & 9 & \foreignlanguage{greek}{υγιειϲ και χωλουϲ περιπατουνταϲ} & 12 &  &  \\
&  & 13 & \foreignlanguage{greek}{και τυφλουϲ βλεπονταϲ και εδο} & 17 &  &  \\
&  & 17 & \foreignlanguage{greek}{ξαϲαν τον \textoverline{θν} ιϲραηλ} & 20 &  &  \\
& \textbf{32} &  & \foreignlanguage{greek}{ο δε \textoverline{ιϲ} προϲκαλεϲαμενοϲ τουϲ μαθη} & 6 &  &  \\
&  & 6 & \foreignlanguage{greek}{ταϲ ειπεν ϲπλαγχνιζομε επι τον ο} & 11 &  &  \\
&  & 11 & \foreignlanguage{greek}{χλον οτι ηδη ημεραι τριϲ προϲμενου} & 16 &  &  \\
&  & 16 & \foreignlanguage{greek}{ϲιν μοι και ουκ εχουϲιν τι φαγειν} & 22 &  &  \\
&  & 23 & \foreignlanguage{greek}{και απολυϲαι αυτουϲ νηϲτιϲ ου θελω} & 28 &  &  \\
&  & 29 & \foreignlanguage{greek}{μηποτε εκλυθωϲιν εν τη οδω} & 33 &  &  \\
& \textbf{33} &  & \foreignlanguage{greek}{και λεγουϲιν αυτω οι μαθηται αυτου} & 6 &  &  \\
&  & 7 & \foreignlanguage{greek}{ποθεν ημιν εν ερημια αρτοι τοϲου} & 12 &  &  \\
&  & 12 & \foreignlanguage{greek}{τοι ωϲτε χορταϲαι οχλον τοϲουτον} & 16 &  &  \\
& \textbf{34} &  & \foreignlanguage{greek}{και λεγει αυτοιϲ ο \textoverline{ιϲ} ποϲουϲ αρτουϲ εχετε} & 8 &  &  \\
&  & 9 & \foreignlanguage{greek}{οι δε ειπον επτα και ολιγα ιχθυδια} & 15 &  &  \\
& \textbf{35} &  & \foreignlanguage{greek}{και εκελευϲεν τοιϲ οχλοιϲ αναπεϲι̅} & 5 &  &  \\
&  & 6 & \foreignlanguage{greek}{επι την γην ϗ λαβων τουϲ επτα} & 4 & \textbf{36} &  \\
[0.2em]
\cline{4-4}
\end{tabular}
\end{center}
\end{table}
}
\clearpage
\newpage
 {
 \setlength\arrayrulewidth{1pt}
\begin{table}
\begin{center}
\begin{tabular}{ccc|l|ccc}
\cline{4-4} \\ [-1em]
\multicolumn{7}{c}{\foreignlanguage{greek}{ευαγγελιον κατα μαθθαιον} \textbf{(\nospace{15:36})} } \\ \\ [-1em] % Si on veut ajouter les bordures latérales, remplacer {7}{c} par {7}{|c|}
\cline{4-4} \\
\cline{4-4}
&  &  & &  &  & \\ [-0.9em]
&  & 5 & \foreignlanguage{greek}{αρτουϲ και τουϲ ιχθυαϲ ευχαριϲτηϲαϲ ε} & 10 &  &  \\
&  & 10 & \foreignlanguage{greek}{κλαϲεν και εδωκεν τοιϲ μαθηταιϲ αυτου} & 15 &  &  \\
&  & 16 & \foreignlanguage{greek}{οι δε μαθηται τω οχλω και εφαγον πα̅} & 3 & \textbf{37} &  \\
&  & 3 & \foreignlanguage{greek}{τεϲ και εχορταϲθηϲαν και ηραν το} & 8 &  &  \\
&  & 9 & \foreignlanguage{greek}{περιϲϲευον των κλαϲματων επτα} & 12 &  &  \\
&  & 13 & \foreignlanguage{greek}{ϲπυριδαϲ πληρειϲ οι δε εϲθιοντεϲ} & 3 & \textbf{38} &  \\
&  & 4 & \foreignlanguage{greek}{ηϲαν τετρακιϲχειλιοι ανδρεϲ χωριϲ} & 7 &  &  \\
&  & 8 & \foreignlanguage{greek}{γυναικων και παιδιων} & 10 &  &  \\
& \textbf{39} &  & \foreignlanguage{greek}{και απολυϲαϲ τουϲ οχλουϲ ανεβη ειϲ} & 6 &  &  \\
&  & 7 & \foreignlanguage{greek}{το πλοιον και ηλθεν ειϲ τα ορια μα} & 14 &  &  \\
&  & 14 & \foreignlanguage{greek}{γδαλαν και προϲελθοντεϲ οι φα} & 4 & \mygospelchapter &  \\
&  & 4 & \foreignlanguage{greek}{ριϲαιοι και ϲαδδουκεοι πειραζοντεϲ} & 7 &  &  \\
&  & 8 & \foreignlanguage{greek}{επηρωτηϲαν αυτον ϲημιον εκ του} & 12 &  &  \\
&  & 13 & \foreignlanguage{greek}{ουρανου επιδειξαι αυτοιϲ} & 15 &  &  \\
& \textbf{2} &  & \foreignlanguage{greek}{ο δε αποκριθειϲ ειπεν αυτοιϲ οψιαϲ} & 6 &  &  \\
&  & 7 & \foreignlanguage{greek}{γενομενηϲ λεγεται ευδια πυρα} & 1 & \textbf{3} &  \\
&  & 1 & \foreignlanguage{greek}{ζει γαρ ϲτυγναζων ο ουρανοϲ το με̅} & 7 &  &  \\
&  & 8 & \foreignlanguage{greek}{προϲωπον του ουρανου γιγνωϲκε} & 11 &  &  \\
&  & 11 & \foreignlanguage{greek}{ται διακρινειν τα δε ϲημια των} & 16 &  &  \\
&  & 17 & \foreignlanguage{greek}{καιρων ου δυναϲθαι δοκιμαϲαι} & 20 &  &  \\
& \textbf{4} &  & \foreignlanguage{greek}{γενεα πονηρα και μοιχαλιϲ ϲημιο̅} & 5 &  &  \\
&  & 6 & \foreignlanguage{greek}{επιζητει και ϲημιον ου δοθηϲεται} & 10 &  &  \\
&  & 11 & \foreignlanguage{greek}{αυτη ει μη το ϲημιον ιωνα του προ} & 18 &  &  \\
&  & 18 & \foreignlanguage{greek}{φητου και καταλιπων αυτουϲ} & 21 &  &  \\
&  & 22 & \foreignlanguage{greek}{απηλθεν και ελθοντεϲ οι μαθηται} & 4 & \textbf{5} &  \\
&  & 5 & \foreignlanguage{greek}{αυτου ειϲ το περαν επελαθοντο} & 9 &  &  \\
&  & 10 & \foreignlanguage{greek}{αρτουϲ λαβειν} & 11 &  &  \\
& \textbf{6} &  & \foreignlanguage{greek}{ο δε \textoverline{ιϲ} ειπεν αυτοιϲ ορατε και προϲεχε} & 8 &  &  \\
&  & 8 & \foreignlanguage{greek}{τε απο τηϲ ζυμηϲ των φαριϲαιων} & 13 &  &  \\
&  & 14 & \foreignlanguage{greek}{και ϲαδδουκεων οι δε διελογιζον} & 3 & \textbf{7} &  \\
[0.2em]
\cline{4-4}
\end{tabular}
\end{center}
\end{table}
}
\clearpage
\newpage
 {
 \setlength\arrayrulewidth{1pt}
\begin{table}
\begin{center}
\begin{tabular}{ccc|l|ccc}
\cline{4-4} \\ [-1em]
\multicolumn{7}{c}{\foreignlanguage{greek}{ευαγγελιον κατα μαθθαιον} \textbf{(\nospace{16:7})} } \\ \\ [-1em] % Si on veut ajouter les bordures latérales, remplacer {7}{c} par {7}{|c|}
\cline{4-4} \\
\cline{4-4}
&  &  & &  &  & \\ [-0.9em]
&  & 3 & \foreignlanguage{greek}{το εν εαυτοιϲ λεγοντεϲ οτι αρτουϲ} & 8 &  &  \\
&  & 9 & \foreignlanguage{greek}{ουκ ελαβομεν γνουϲ δε ο \textoverline{ιϲ} ειπεν} & 5 & \textbf{8} &  \\
&  & 6 & \foreignlanguage{greek}{τι διαλογιζεϲθαι εν εαυτοιϲ ολιγοπι} & 10 &  &  \\
&  & 10 & \foreignlanguage{greek}{ϲτοι οτι αρτουϲ ουκ ελαβεται ουπω} & 1 & \textbf{9} &  \\
&  & 2 & \foreignlanguage{greek}{νοειτε ουτε μνημονευεται τουϲ πε̅} & 6 &  &  \\
&  & 6 & \foreignlanguage{greek}{τε αρτουϲ των πεντακιϲχειλιων} & 9 &  &  \\
&  & 10 & \foreignlanguage{greek}{και ποϲουϲ κοφινουϲ ελαβεται ου} & 1 & \textbf{10} &  \\
&  & 1 & \foreignlanguage{greek}{δε τουϲ επτα αρτουϲ των τετρακιϲ} & 6 &  &  \\
&  & 6 & \foreignlanguage{greek}{χειλιων και ποϲαϲ ϲπυριδαϲ ελαβεται} & 10 &  &  \\
& \textbf{11} &  & \foreignlanguage{greek}{πωϲ ου νοειται οτι ου περι αρτου ειπο̅} & 8 &  &  \\
&  & 9 & \foreignlanguage{greek}{υμιν προϲεχειν απο τηϲ ζυμηϲ τω̅} & 14 &  &  \\
&  & 15 & \foreignlanguage{greek}{φαριϲαιων και ϲαδδουκεων} & 17 &  &  \\
& \textbf{12} &  & \foreignlanguage{greek}{τοτε ϲυνηκαν οτι ουκ ειπεν προϲε} & 6 &  &  \\
&  & 6 & \foreignlanguage{greek}{χειν απο τηϲ ζυμηϲ του αρτου αλλα} & 12 &  &  \\
&  & 13 & \foreignlanguage{greek}{απο τηϲ διδαχηϲ των φαριϲαιων και} & 18 &  &  \\
&  & 19 & \foreignlanguage{greek}{ϲαδδουκεων} & 19 &  &  \\
& \textbf{13} &  & \foreignlanguage{greek}{εξελθων δε ο \textoverline{ιϲ} ειϲ τα μερη κεϲαριαϲ} & 8 &  &  \\
&  & 9 & \foreignlanguage{greek}{τηϲ φιλιππου ηρωτα τουϲ μαθηταϲ} & 13 &  &  \\
&  & 14 & \foreignlanguage{greek}{αυτου λεγων τινα λεγουϲιν με οι α̅} & 20 &  &  \\
&  & 20 & \foreignlanguage{greek}{θρωποι ειναι τον υιον του \textoverline{ανου}} & 25 &  &  \\
& \textbf{14} &  & \foreignlanguage{greek}{οι δε ειπον ιωαννην τον βαπτιϲτη̅} & 6 &  &  \\
&  & 7 & \foreignlanguage{greek}{αλλοι δε ηλιαν ετεροι δε ιερεμιαν} & 12 &  &  \\
&  & 13 & \foreignlanguage{greek}{η ενα των προφητων λεγει αυτοιϲ} & 2 & \textbf{15} &  \\
&  & 3 & \foreignlanguage{greek}{υμειϲ δε τινα με λεγεται ειναι} & 8 &  &  \\
& \textbf{16} &  & \foreignlanguage{greek}{αποκριθειϲ δε ϲιμων πετροϲ ειπεν} & 5 &  &  \\
&  & 6 & \foreignlanguage{greek}{ϲυ ει ο \textoverline{χϲ} ο υιοϲ του \textoverline{θυ} του ζωντοϲ} & 15 &  &  \\
& \textbf{17} &  & \foreignlanguage{greek}{και αποκριθειϲ ο \textoverline{ιϲ} ειπεν αυτω μακα} & 7 &  &  \\
&  & 7 & \foreignlanguage{greek}{ριοϲ ει ϲιμων βαριωνα οτι ϲαρξ και} & 13 &  &  \\
&  & 14 & \foreignlanguage{greek}{αιμα ουκ απεκαλυψεν ϲοι αλλ ο \textoverline{πηρ}} & 20 &  &  \\
&  & 21 & \foreignlanguage{greek}{μου ο εν τοιϲ ουρανοιϲ καγω δε ϲοι λεγω} & 4 & \textbf{18} &  \\
[0.2em]
\cline{4-4}
\end{tabular}
\end{center}
\end{table}
}
\clearpage
\newpage
 {
 \setlength\arrayrulewidth{1pt}
\begin{table}
\begin{center}
\begin{tabular}{ccc|l|ccc}
\cline{4-4} \\ [-1em]
\multicolumn{7}{c}{\foreignlanguage{greek}{ευαγγελιον κατα μαθθαιον} \textbf{(\nospace{16:18})} } \\ \\ [-1em] % Si on veut ajouter les bordures latérales, remplacer {7}{c} par {7}{|c|}
\cline{4-4} \\
\cline{4-4}
&  &  & &  &  & \\ [-0.9em]
&  & 5 & \foreignlanguage{greek}{οτι ϲυ ει πετροϲ και επι ταυτη τη πετρα} & 13 &  &  \\
&  & 14 & \foreignlanguage{greek}{οικοδομηϲω μου την εκκληϲιαν} & 17 &  &  \\
&  & 18 & \foreignlanguage{greek}{και πυλαι αδου ου κατιϲχυϲουϲιν αυ} & 23 &  &  \\
&  & 23 & \foreignlanguage{greek}{τηϲ και δωϲω ϲοι ταϲ κλειδαϲ τηϲ βα} & 7 & \textbf{19} &  \\
&  & 7 & \foreignlanguage{greek}{ϲιλειαϲ των ουρανων και ο αν δηϲηϲ} & 13 &  &  \\
&  & 14 & \foreignlanguage{greek}{επι τηϲ γηϲ εϲται δεδεμενον εν τοιϲ} & 20 &  &  \\
&  & 21 & \foreignlanguage{greek}{ουρανοιϲ και ο εαν λυϲηϲ επι τηϲ γηϲ} & 28 &  &  \\
&  & 29 & \foreignlanguage{greek}{εϲται λελυμενον εν τοιϲ ουρανοιϲ} & 33 &  &  \\
& \textbf{20} &  & \foreignlanguage{greek}{τοτε διεϲτιλατο τοιϲ μαθηταιϲ αυτου} & 5 &  &  \\
&  & 6 & \foreignlanguage{greek}{ινα μηδενι ειπωϲιν οτι αυτοϲ εϲτιν} & 11 &  &  \\
&  & 12 & \foreignlanguage{greek}{\textoverline{ιϲ} ο \textoverline{χϲ} απο τοτε ηρξατο ο \textoverline{ιϲ} δικνυ} & 6 & \textbf{21} &  \\
&  & 6 & \foreignlanguage{greek}{ειν τοιϲ μαθηταιϲ αυτου οτι δι αυτο̅} & 12 &  &  \\
&  & 13 & \foreignlanguage{greek}{απελθειν ειϲ ιεροϲολυμα και πολλα} & 17 &  &  \\
&  & 18 & \foreignlanguage{greek}{παθειν απο των πρεϲβυτερων και} & 22 &  &  \\
&  & 23 & \foreignlanguage{greek}{αρχιερεων και γραμματεων και α} & 27 &  &  \\
&  & 27 & \foreignlanguage{greek}{ποκτανθηναι και τη τριτη ημερα} & 31 &  &  \\
&  & 32 & \foreignlanguage{greek}{εγερθηναι} & 32 &  &  \\
& \textbf{22} &  & \foreignlanguage{greek}{και προϲλαβομενοϲ αυτον ο πετροϲ} & 5 &  &  \\
&  & 6 & \foreignlanguage{greek}{ηρξατο επιτιμαν αυτω λεγων ειλε} & 10 &  &  \\
&  & 10 & \foreignlanguage{greek}{ωϲ ϲοι \textoverline{κε} ου μη εϲται ϲοι τουτο} & 17 &  &  \\
& \textbf{23} &  & \foreignlanguage{greek}{ο δε ϲτραφειϲ ειπεν τω πετρω υπαγε} & 7 &  &  \\
&  & 8 & \foreignlanguage{greek}{οπιϲω μου ϲατανα ϲκανδαλον μου ει} & 13 &  &  \\
&  & 14 & \foreignlanguage{greek}{οτι ου φρονιϲ τα του \textoverline{θυ} αλλα τα των} & 22 &  &  \\
&  & 23 & \foreignlanguage{greek}{ανθρωπων} & 23 &  &  \\
& \textbf{24} &  & \foreignlanguage{greek}{τοτε ο \textoverline{ιϲ} ειπεν τοιϲ μαθηταιϲ αυτου} & 7 &  &  \\
&  & 8 & \foreignlanguage{greek}{ει τιϲ θελει οπιϲω μου ελθειν απαρ} & 14 &  &  \\
&  & 14 & \foreignlanguage{greek}{νηϲαϲθω εαυτον και αρατω τον ϲταυ} & 20 &  &  \\
&  & 20 & \foreignlanguage{greek}{ρον αυτου και ακολουθιτω μοι} & 24 &  &  \\
& \textbf{25} &  & \foreignlanguage{greek}{οϲ γαρ αν θελη την ψυχην αυτου ϲωϲαι} & 8 &  &  \\
&  & 9 & \foreignlanguage{greek}{απολεϲει αυτην οϲ δ αν απολεϲει τη̅} & 15 &  &  \\
[0.2em]
\cline{4-4}
\end{tabular}
\end{center}
\end{table}
}
\clearpage
\newpage
 {
 \setlength\arrayrulewidth{1pt}
\begin{table}
\begin{center}
\begin{tabular}{ccc|l|ccc}
\cline{4-4} \\ [-1em]
\multicolumn{7}{c}{\foreignlanguage{greek}{ευαγγελιον κατα μαθθαιον} \textbf{(\nospace{16:25})} } \\ \\ [-1em] % Si on veut ajouter les bordures latérales, remplacer {7}{c} par {7}{|c|}
\cline{4-4} \\
\cline{4-4}
&  &  & &  &  & \\ [-0.9em]
&  & 16 & \foreignlanguage{greek}{ψυχην αυτου ενεκεν εμου ευρηϲει αυτη̅} & 21 &  &  \\
& \textbf{26} &  & \foreignlanguage{greek}{τι γαρ ωφελειται \textoverline{ανοϲ} εαν τον κοϲμο̅} & 7 &  &  \\
&  & 8 & \foreignlanguage{greek}{ολον κερδηϲη την δε ψυχην αυτου} & 13 &  &  \\
&  & 14 & \foreignlanguage{greek}{ζημιωθη η τι δωϲει ανθρωποϲ αν} & 19 &  &  \\
&  & 19 & \foreignlanguage{greek}{ταλλαγμα τηϲ ψυχηϲ αυτου} & 22 &  &  \\
& \textbf{27} &  & \foreignlanguage{greek}{μελλει γαρ ο υιοϲ του \textoverline{ανου} ερχεϲθαι εν} & 8 &  &  \\
&  & 9 & \foreignlanguage{greek}{τη δοξη του \textoverline{πρϲ} αυτου μετα των αγ} & 16 &  &  \\
&  & 16 & \foreignlanguage{greek}{γελων αυτου και τοτε αποδωϲη ε} & 21 &  &  \\
&  & 21 & \foreignlanguage{greek}{καϲτω κατα την πραξιν αυτου} & 25 &  &  \\
& \textbf{28} &  & \foreignlanguage{greek}{αμην λεγω υμιν ειϲιν τινεϲ ωδε εϲτω} & 7 &  &  \\
&  & 7 & \foreignlanguage{greek}{τεϲ οιτινεϲ ου μη γευϲωνται θανατου} & 12 &  &  \\
&  & 13 & \foreignlanguage{greek}{εωϲ αν ιδωϲιν τον υιον του \textoverline{ανου} ερ} & 20 &  &  \\
&  & 20 & \foreignlanguage{greek}{χομενον εν τη βαϲιλεια αυτου} & 24 &  &  \\
& \mygospelchapter &  & \foreignlanguage{greek}{και μεθ ημεραϲ εξ παραλαμβανει ο \textoverline{ιϲ}} & 7 &  &  \\
&  & 8 & \foreignlanguage{greek}{τον πετρον και ιακωβον και ιωαν} & 13 &  &  \\
&  & 13 & \foreignlanguage{greek}{νην τον αδελφον αυτου και ανα} & 18 &  &  \\
&  & 18 & \foreignlanguage{greek}{φερει αυτουϲ ειϲ οροϲ υψηλον κατ ιδι} & 24 &  &  \\
&  & 24 & \foreignlanguage{greek}{αν και μετεμορφωθη εμπροϲθεν αυ} & 4 & \textbf{2} &  \\
&  & 4 & \foreignlanguage{greek}{των και ελαμψεν το προϲωπον αυ} & 9 &  &  \\
&  & 9 & \foreignlanguage{greek}{του ωϲ ο ηλιοϲ τα δε ιματια αυτου} & 16 &  &  \\
&  & 17 & \foreignlanguage{greek}{εγενετο λευκα ωϲ το φωϲ} & 21 &  &  \\
& \textbf{3} &  & \foreignlanguage{greek}{και ιδου ωφθηϲαν αυτοιϲ μωυϲηϲ} & 5 &  &  \\
&  & 6 & \foreignlanguage{greek}{και ηλιαϲ ϲυνλαλουντεϲ μετ αυτου} & 10 &  &  \\
& \textbf{4} &  & \foreignlanguage{greek}{αποκριθειϲ δε πετροϲ ειπεν τω \textoverline{ιυ}} & 6 &  &  \\
&  & 7 & \foreignlanguage{greek}{\textoverline{κε} καλον εϲτιν ημαϲ ωδε ειναι} & 12 &  &  \\
&  & 13 & \foreignlanguage{greek}{θελειϲ ποιηϲωμεν ωδε τριϲ ϲκηναϲ} & 17 &  &  \\
&  & 18 & \foreignlanguage{greek}{ϲοι μιαν και ηλια μιαν και μωυϲι μια̅} & 25 &  &  \\
& \textbf{5} &  & \foreignlanguage{greek}{ετι αυτου λαλουντοϲ ιδου νεφελη φω} & 6 &  &  \\
&  & 6 & \foreignlanguage{greek}{τινη επεϲκιαϲεν αυτουϲ και ιδου} & 10 &  &  \\
&  & 11 & \foreignlanguage{greek}{φωνη εκ τηϲ νεφεληϲ λεγουϲα} & 15 &  &  \\
[0.2em]
\cline{4-4}
\end{tabular}
\end{center}
\end{table}
}
\clearpage
\newpage
 {
 \setlength\arrayrulewidth{1pt}
\begin{table}
\begin{center}
\begin{tabular}{ccc|l|ccc}
\cline{4-4} \\ [-1em]
\multicolumn{7}{c}{\foreignlanguage{greek}{ευαγγελιον κατα μαθθαιον} \textbf{(\nospace{17:5})} } \\ \\ [-1em] % Si on veut ajouter les bordures latérales, remplacer {7}{c} par {7}{|c|}
\cline{4-4} \\
\cline{4-4}
&  &  & &  &  & \\ [-0.9em]
&  & 16 & \foreignlanguage{greek}{ουτοϲ εϲτιν ο υιοϲ μου ο αγαπητοϲ εν ω} & 24 &  &  \\
&  & 25 & \foreignlanguage{greek}{ηυδοκηϲα αυτου ακουεται} & 27 &  &  \\
& \textbf{6} &  & \foreignlanguage{greek}{και ακουϲαντεϲ οι μαθηται επεϲαν} & 5 &  &  \\
&  & 6 & \foreignlanguage{greek}{επι προϲωπον αυτων και εφοβηθη} & 10 &  &  \\
&  & 10 & \foreignlanguage{greek}{ϲαν ϲφοδρα και προϲελθων ο \textoverline{ιϲ} ηψα} & 5 & \textbf{7} &  \\
&  & 5 & \foreignlanguage{greek}{το αυτων και ειπεν εγερθηται και} & 10 &  &  \\
&  & 11 & \foreignlanguage{greek}{μη φοβειϲθαι επαραντεϲ δε τουϲ} & 3 & \textbf{8} &  \\
&  & 4 & \foreignlanguage{greek}{οφθαλμουϲ ουδενα ειδον ει μη \textoverline{ιν}} & 9 &  &  \\
&  & 10 & \foreignlanguage{greek}{μονον και καταβενοντων εκ} & 3 & \textbf{9} &  \\
&  & 4 & \foreignlanguage{greek}{του ορουϲ ενετιλατο αυτοιϲ ο \textoverline{ιϲ} λεγω̅} & 10 &  &  \\
&  & 11 & \foreignlanguage{greek}{μηδενι ειπηται το οραμα εωϲ ου ο} & 17 &  &  \\
&  & 18 & \foreignlanguage{greek}{υιοϲ του \textoverline{ανου} αναϲτη εκ νεκρων} & 23 &  &  \\
& \textbf{10} &  & \foreignlanguage{greek}{και επηρωτηϲαν αυτον οι μαθηται} & 5 &  &  \\
&  & 6 & \foreignlanguage{greek}{λεγοντεϲ τι ουν οι γραμματιϲ} & 10 &  &  \\
&  & 11 & \foreignlanguage{greek}{λεγουϲιν οτι ηλιαν δει ελθειν πρω} & 16 &  &  \\
&  & 16 & \foreignlanguage{greek}{τον ο δε αποκριθειϲ ειπεν} & 4 & \textbf{11} &  \\
&  & 5 & \foreignlanguage{greek}{ηλιαϲ μεν ερχεται και αποκατα} & 9 &  &  \\
&  & 9 & \foreignlanguage{greek}{ϲτηϲει παντα λεγω δε υμιν οτι} & 4 & \textbf{12} &  \\
&  & 5 & \foreignlanguage{greek}{ηλιαϲ ηδη ηλθεν και ουκ επεγνω} & 10 &  &  \\
&  & 10 & \foreignlanguage{greek}{ϲαν αυτον αλλα εποιηϲαν αυτω ο} & 15 &  &  \\
&  & 15 & \foreignlanguage{greek}{ϲα ηθεληϲαν ουτωϲ και ο υιοϲ του} & 21 &  &  \\
&  & 22 & \foreignlanguage{greek}{\textoverline{ανου} μελλει παϲχειν υπ αυτων} & 26 &  &  \\
& \textbf{13} &  & \foreignlanguage{greek}{τοτε ϲυνηκαν οι μαθηται οτι περι} & 6 &  &  \\
&  & 7 & \foreignlanguage{greek}{ιωαννου του βαπτιϲτου ειπεν αυτοιϲ} & 11 &  &  \\
& \textbf{14} &  & \foreignlanguage{greek}{και ελθοντων αυτων προϲ τον οχλο̅} & 6 &  &  \\
&  & 7 & \foreignlanguage{greek}{προϲηλθεν αυτω \textoverline{ανοϲ} γονυπετω̅} & 10 &  &  \\
&  & 11 & \foreignlanguage{greek}{αυτον και λεγων \textoverline{κε} ελεηϲον μου} & 5 & \textbf{15} &  \\
&  & 6 & \foreignlanguage{greek}{τον υιον οτι ϲεληνιαζεται και κα} & 11 &  &  \\
&  & 11 & \foreignlanguage{greek}{κωϲ παϲχει πολλακιϲ γαρ πιπτει} & 15 &  &  \\
&  & 16 & \foreignlanguage{greek}{ειϲ το πυρ και ειϲ το υδωρ} & 22 &  &  \\
[0.2em]
\cline{4-4}
\end{tabular}
\end{center}
\end{table}
}
\clearpage
\newpage
 {
 \setlength\arrayrulewidth{1pt}
\begin{table}
\begin{center}
\begin{tabular}{ccc|l|ccc}
\cline{4-4} \\ [-1em]
\multicolumn{7}{c}{\foreignlanguage{greek}{ευαγγελιον κατα μαθθαιον} \textbf{(\nospace{17:16})} } \\ \\ [-1em] % Si on veut ajouter les bordures latérales, remplacer {7}{c} par {7}{|c|}
\cline{4-4} \\
\cline{4-4}
&  &  & &  &  & \\ [-0.9em]
& \textbf{16} &  & \foreignlanguage{greek}{και προϲηνεγκα αυτον τοιϲ μαθη} & 5 &  &  \\
&  & 5 & \foreignlanguage{greek}{ταιϲ ϲου και ουκ ηδυνηθηϲαν αυτον} & 10 &  &  \\
&  & 11 & \foreignlanguage{greek}{θεραπευϲαι} & 11 &  &  \\
& \textbf{17} &  & \foreignlanguage{greek}{αποκριθειϲ δε ο \textoverline{ιϲ} ειπεν ω γενεα} & 7 &  &  \\
&  & 8 & \foreignlanguage{greek}{απιϲτοϲ και διεϲτραμμενη εωϲ πο} & 12 &  &  \\
&  & 12 & \foreignlanguage{greek}{τε εϲομαι μεθ υμων εωϲ ποτε ανε} & 18 &  &  \\
&  & 18 & \foreignlanguage{greek}{ξομαι υμων φερεται μοι αυτον ωδε} & 23 &  &  \\
& \textbf{18} &  & \foreignlanguage{greek}{και επετιμηϲεν αυτω ο \textoverline{ιϲ} και εξηλθε̅} & 7 &  &  \\
&  & 8 & \foreignlanguage{greek}{απ αυτου το δαιμονιον και εθεραπευ} & 13 &  &  \\
&  & 13 & \foreignlanguage{greek}{θη ο παιϲ απο τηϲ ωραϲ εκεινηϲ} & 19 &  &  \\
& \textbf{19} &  & \foreignlanguage{greek}{τοτε προϲελθοντεϲ οι μαθηται τω \textoverline{ιυ}} & 6 &  &  \\
&  & 7 & \foreignlanguage{greek}{κατ ιδιαν ειπον δια τι ημειϲ ουκ η} & 15 &  &  \\
&  & 15 & \foreignlanguage{greek}{δυνηθημεν εκβαλειν αυτο} & 17 &  &  \\
& \textbf{20} &  & \foreignlanguage{greek}{ο δε \textoverline{ιϲ} ειπεν αυτοιϲ δια την απιϲτι} & 8 &  &  \\
&  & 8 & \foreignlanguage{greek}{αν υμων αμην γαρ λεγω υμιν εα̅} & 14 &  &  \\
&  & 15 & \foreignlanguage{greek}{εχηται πιϲτιν ωϲ κοκκον ϲιναπε} & 19 &  &  \\
&  & 19 & \foreignlanguage{greek}{ωϲ ερειται τω ορι τουτω μεταβηθει} & 24 &  &  \\
&  & 25 & \foreignlanguage{greek}{εντευθεν εκει και μεταβηϲεται} & 28 &  &  \\
&  & 29 & \foreignlanguage{greek}{και ουδεν αδυνατηϲει υμιν} & 32 &  &  \\
& \textbf{21} &  & \foreignlanguage{greek}{τουτο δε το γενοϲ ουκ εκπορευεται} & 6 &  &  \\
&  & 7 & \foreignlanguage{greek}{ει μη εν προϲευχη και νηϲτια} & 12 &  &  \\
& \textbf{22} &  & \foreignlanguage{greek}{αναϲτρεφομενων δε αυτων εν τη} & 5 &  &  \\
&  & 6 & \foreignlanguage{greek}{γαλιλαια ειπεν αυτοιϲ ο \textoverline{ιϲ} μελλει} & 11 &  &  \\
&  & 12 & \foreignlanguage{greek}{ο υιοϲ του \textoverline{ανου} παραδιδοϲθαι ειϲ χει} & 18 &  &  \\
&  & 18 & \foreignlanguage{greek}{ραϲ \textoverline{ανων} και αποκτενουϲιν αυτο̅} & 3 & \textbf{23} &  \\
&  & 4 & \foreignlanguage{greek}{και τη τριτη ημερα εγερθηϲεται} & 8 &  &  \\
&  & 9 & \foreignlanguage{greek}{και ελυπηθηϲαν ϲφοδρα} & 11 &  &  \\
& \textbf{24} &  & \foreignlanguage{greek}{ελθοντων δε αυτων ειϲ καφαρναουμ} & 5 &  &  \\
&  & 6 & \foreignlanguage{greek}{προϲηλθον οι το διδραγμα λαμβανο̅} & 10 &  &  \\
&  & 10 & \foreignlanguage{greek}{τεϲ τω πετρω και ειπον ο διδαϲκα} & 16 &  &  \\
[0.2em]
\cline{4-4}
\end{tabular}
\end{center}
\end{table}
}
\clearpage
\newpage
 {
 \setlength\arrayrulewidth{1pt}
\begin{table}
\begin{center}
\begin{tabular}{ccc|l|ccc}
\cline{4-4} \\ [-1em]
\multicolumn{7}{c}{\foreignlanguage{greek}{ευαγγελιον κατα μαθθαιον} \textbf{(\nospace{17:24})} } \\ \\ [-1em] % Si on veut ajouter les bordures latérales, remplacer {7}{c} par {7}{|c|}
\cline{4-4} \\
\cline{4-4}
&  &  & &  &  & \\ [-0.9em]
&  & 16 & \foreignlanguage{greek}{λοϲ υμων ουτε το διδραγμα λεγει ναι} & 2 & \textbf{25} &  \\
&  & 3 & \foreignlanguage{greek}{και οτε ειϲηλθεν ειϲ την οικειαν προ} & 11 &  &  \\
&  & 11 & \foreignlanguage{greek}{εφθαϲεν αυτον ο \textoverline{ιϲ} λεγων τι ϲοι δοκει} & 18 &  &  \\
&  & 19 & \foreignlanguage{greek}{ϲιμων οι βαϲιλειϲ τηϲ γηϲ απο τινων} & 25 &  &  \\
&  & 26 & \foreignlanguage{greek}{λαμβανουϲιν τελη η κηνϲον απο τω̅} & 31 &  &  \\
&  & 32 & \foreignlanguage{greek}{υιων αυτων η απο των αλλοτριων} & 37 &  &  \\
& \textbf{26} &  & \foreignlanguage{greek}{λεγει αυτω ο πετροϲ απο των αλλοτριω̅} & 7 &  &  \\
&  & 8 & \foreignlanguage{greek}{εφη αυτω ο \textoverline{ιϲ} αρα γε ελευθεροι ειϲιν οι} & 16 &  &  \\
&  & 17 & \foreignlanguage{greek}{υιοι ινα δε μη ϲκανδαλιϲωμεν αυ} & 5 & \textbf{27} &  \\
&  & 5 & \foreignlanguage{greek}{τουϲ πορευθειϲ ειϲ θαλαϲϲαν βαλε αγ} & 10 &  &  \\
&  & 10 & \foreignlanguage{greek}{κιϲτρον και τον αναβαιναντα πρω} & 14 &  &  \\
&  & 14 & \foreignlanguage{greek}{τον ιχθυν αρον και ανοιξαϲ το ϲτο} & 20 &  &  \\
&  & 20 & \foreignlanguage{greek}{μα αυτου ευρηϲειϲ ϲτατηρα εκεινο̅} & 24 &  &  \\
&  & 25 & \foreignlanguage{greek}{λαβων δοϲ αυτοιϲ αντι εμου και ϲου} & 31 &  &  \\
& \mygospelchapter &  & \foreignlanguage{greek}{εν εκεινη τη ωρα προϲηλθον οι μαθη} & 7 &  &  \\
&  & 7 & \foreignlanguage{greek}{ται τω \textoverline{ιυ} λεγοντεϲ τιϲ αρα μιζων ε} & 14 &  &  \\
&  & 14 & \foreignlanguage{greek}{ϲτιν εν τη βαϲιλεια των ουρανων} & 19 &  &  \\
& \textbf{2} &  & \foreignlanguage{greek}{και προϲκαλεϲαμενοϲ ο \textoverline{ιϲ} παιδιον} & 5 &  &  \\
&  & 6 & \foreignlanguage{greek}{εϲτηϲεν αυτο εν μεϲω αυτων και ει} & 2 & \textbf{3} &  \\
&  & 2 & \foreignlanguage{greek}{πεν αμην λεγω υμιν εαν μη ϲτρα} & 8 &  &  \\
&  & 8 & \foreignlanguage{greek}{φηται και γενηϲθαι ωϲ τα παιδια ου} & 14 &  &  \\
&  & 15 & \foreignlanguage{greek}{μη ειϲελθηται ειϲ την βαϲιλειαν} & 19 &  &  \\
&  & 20 & \foreignlanguage{greek}{των ουρανων οϲτιϲ γαρ ταπινω} & 3 & \textbf{4} &  \\
&  & 3 & \foreignlanguage{greek}{ϲει εαυτον ωϲ το παιδιον τουτο ου} & 9 &  &  \\
&  & 9 & \foreignlanguage{greek}{τοϲ εϲτιν ο μιζων εν τη βαϲιλεια} & 15 &  &  \\
&  & 18 & \foreignlanguage{greek}{των ουρανων και οϲ εαν δεξη} & 4 & \textbf{5} &  \\
&  & 4 & \foreignlanguage{greek}{ται παιδιον τοιουτο εν επι τω ονο} & 10 &  &  \\
&  & 10 & \foreignlanguage{greek}{ματι μου εμε δεχεται} & 13 &  &  \\
& \textbf{6} &  & \foreignlanguage{greek}{οϲ δ αν ϲκανδαλιϲη ενα των μικρω̅} & 7 &  &  \\
&  & 8 & \foreignlanguage{greek}{τουτων των πιϲτευοντων ειϲ εμε} & 12 &  &  \\
[0.2em]
\cline{4-4}
\end{tabular}
\end{center}
\end{table}
}
\clearpage
\newpage
 {
 \setlength\arrayrulewidth{1pt}
\begin{table}
\begin{center}
\begin{tabular}{ccc|l|ccc}
\cline{4-4} \\ [-1em]
\multicolumn{7}{c}{\foreignlanguage{greek}{ευαγγελιον κατα μαθθαιον} \textbf{(\nospace{18:6})} } \\ \\ [-1em] % Si on veut ajouter les bordures latérales, remplacer {7}{c} par {7}{|c|}
\cline{4-4} \\
\cline{4-4}
&  &  & &  &  & \\ [-0.9em]
&  & 13 & \foreignlanguage{greek}{ϲυμφερει αυτω ινα κρεμαϲθη μυλοϲ} & 17 &  &  \\
&  & 18 & \foreignlanguage{greek}{ονικοϲ ειϲ τον τραχηλον αυτου και} & 23 &  &  \\
&  & 24 & \foreignlanguage{greek}{καταποντιϲθη εν τω πελαγει τηϲ θα} & 29 &  &  \\
&  & 29 & \foreignlanguage{greek}{λαϲϲηϲ ουαι τω κοϲμω απο των ϲκαν} & 6 & \textbf{7} &  \\
&  & 6 & \foreignlanguage{greek}{δαλων αναγκη γαρ εϲτιν ελθειν τα} & 11 &  &  \\
&  & 12 & \foreignlanguage{greek}{ϲκανδαλα πλην εκεινω ουαι τω \textoverline{ανω}} & 17 &  &  \\
&  & 18 & \foreignlanguage{greek}{δι ου το ϲκανδαλον ερχεται} & 22 &  &  \\
& \textbf{8} &  & \foreignlanguage{greek}{ει δε η χειρ ϲου η ο πουϲ ϲου ϲκανδαλιζει} & 10 &  &  \\
&  & 11 & \foreignlanguage{greek}{ϲε εκκοψον αυτα και βαλε απο ϲου} & 17 &  &  \\
&  & 18 & \foreignlanguage{greek}{καλον ϲοι εϲτιν ειϲ την ζωην χωλον} & 24 &  &  \\
&  & 25 & \foreignlanguage{greek}{η κυλλον η δυο χειραϲ η δυο ποδαϲ} & 32 &  &  \\
&  & 33 & \foreignlanguage{greek}{εχοντα βληθηναι ειϲ το πυρ το αιωνιο̅} & 39 &  &  \\
& \textbf{9} &  & \foreignlanguage{greek}{και ει ο οφθαλμοϲ ϲου ϲκανδαλιζει ϲε} & 7 &  &  \\
&  & 8 & \foreignlanguage{greek}{εξελε αυτον και βαλε απο ϲου καλο̅} & 14 &  &  \\
&  & 15 & \foreignlanguage{greek}{ϲοι εϲτιν μονοφθαλμον ειϲ την ζω} & 20 &  &  \\
&  & 20 & \foreignlanguage{greek}{ην ειϲελθειν η δυο οφθαλμουϲ εχο̅} & 25 &  &  \\
&  & 25 & \foreignlanguage{greek}{τα βληθηναι ειϲ την γεενναν του πυροϲ} & 31 &  &  \\
& \textbf{10} &  & \foreignlanguage{greek}{οραται μη καταφρονηϲηται ενοϲ των} & 5 &  &  \\
&  & 6 & \foreignlanguage{greek}{μικρων τουτων λεγω γαρ υμιν οτι} & 11 &  &  \\
&  & 12 & \foreignlanguage{greek}{οι αγγελοι αυτων εν ουρανοιϲ δια του πα̅} & 19 &  &  \\
&  & 19 & \foreignlanguage{greek}{τοϲ βλεπουϲιν το προϲωπον του \textoverline{πρϲ}} & 24 &  &  \\
&  & 25 & \foreignlanguage{greek}{μου του εν ουρανοιϲ} & 28 &  &  \\
& \textbf{11} &  & \foreignlanguage{greek}{ηλθεν γαρ ο υιοϲ του ανθρωπου ϲωϲαι} & 7 &  &  \\
&  & 8 & \foreignlanguage{greek}{το απολωλοϲ τι υμιν δοκει} & 3 & \textbf{12} &  \\
&  & 4 & \foreignlanguage{greek}{εαν γενηται τινι \textoverline{ανω} εκατον προβα} & 9 &  &  \\
&  & 9 & \foreignlanguage{greek}{τα και πλανηθη εν εξ αυτων ουχει} & 15 &  &  \\
&  & 16 & \foreignlanguage{greek}{αφειϲ τα ενενηκοντα εννεα επι τα} & 21 &  &  \\
&  & 22 & \foreignlanguage{greek}{ορη πορευθειϲ ζητει το πλανωμενο̅} & 26 &  &  \\
& \textbf{13} &  & \foreignlanguage{greek}{και εαν γενηται ευρειν αυτο αμην} & 6 &  &  \\
&  & 7 & \foreignlanguage{greek}{λεγω υμιν οτι χαιρει επ αυτω μαλλο̅} & 13 &  &  \\
[0.2em]
\cline{4-4}
\end{tabular}
\end{center}
\end{table}
}
\clearpage
\newpage
 {
 \setlength\arrayrulewidth{1pt}
\begin{table}
\begin{center}
\begin{tabular}{ccc|l|ccc}
\cline{4-4} \\ [-1em]
\multicolumn{7}{c}{\foreignlanguage{greek}{ευαγγελιον κατα μαθθαιον} \textbf{(\nospace{18:13})} } \\ \\ [-1em] % Si on veut ajouter les bordures latérales, remplacer {7}{c} par {7}{|c|}
\cline{4-4} \\
\cline{4-4}
&  &  & &  &  & \\ [-0.9em]
&  & 14 & \foreignlanguage{greek}{η επι τοιϲ ενενηκοντα εννεα τοιϲ μη} & 20 &  &  \\
&  & 21 & \foreignlanguage{greek}{πεπλανημενοιϲ ουτωϲ ουκ εϲτιν θε} & 4 & \textbf{14} &  \\
&  & 4 & \foreignlanguage{greek}{λημα εμπροϲθεν του \textoverline{πρϲ} υμων του εν} & 10 &  &  \\
&  & 11 & \foreignlanguage{greek}{ουρανοιϲ ινα αποληται ειϲ των μικρω̅} & 16 &  &  \\
&  & 17 & \foreignlanguage{greek}{τουτων} & 17 &  &  \\
& \textbf{15} &  & \foreignlanguage{greek}{εαν δε αμαρτη ειϲ ϲε ο αδελφοϲ ϲου υπα} & 9 &  &  \\
&  & 9 & \foreignlanguage{greek}{γε και ελεγξε αυτον μεταξυ ϲου και αυ} & 16 &  &  \\
&  & 16 & \foreignlanguage{greek}{του μονου εαν ϲου ακουϲη εκερδη} & 21 &  &  \\
&  & 21 & \foreignlanguage{greek}{ϲαϲ τον αδελφον ϲου εαν δε μη α} & 4 & \textbf{16} &  \\
&  & 4 & \foreignlanguage{greek}{κουϲη παραλαβε μετα ϲου ετι ενα η δυο} & 11 &  &  \\
&  & 12 & \foreignlanguage{greek}{ινα επι ϲτοματοϲ δυο μαρτυρων η τρι} & 18 &  &  \\
&  & 18 & \foreignlanguage{greek}{ων ϲταθη παν ρημα εαν δε παρακου} & 3 & \textbf{17} &  \\
&  & 3 & \foreignlanguage{greek}{ϲη αυτων ειπε τη εκκληϲια εαν δε} & 9 &  &  \\
&  & 10 & \foreignlanguage{greek}{και τηϲ εκκληϲιαϲ παρακουϲη εϲτω ϲοι} & 15 &  &  \\
&  & 16 & \foreignlanguage{greek}{ωϲπερ εθνικοϲ και ο τελωνηϲ} & 20 &  &  \\
& \textbf{18} &  & \foreignlanguage{greek}{αμην λεγω υμιν οϲα εαν δηϲηται επι} & 7 &  &  \\
&  & 8 & \foreignlanguage{greek}{τηϲ γηϲ εϲται δεδεμενα εν τω ουρανω} & 14 &  &  \\
&  & 15 & \foreignlanguage{greek}{και οϲα εαν λυϲηται επι τηϲ γηϲ εϲται} & 22 &  &  \\
&  & 23 & \foreignlanguage{greek}{λελυμενα εν τω ουρανω} & 26 &  &  \\
& \textbf{19} &  & \foreignlanguage{greek}{παλιν δε υμιν λεγω οτι εαν δυο υμω̅} & 8 &  &  \\
&  & 9 & \foreignlanguage{greek}{ϲυμφωνηϲωϲιν επι τηϲ γηϲ περι του παν} & 15 &  &  \\
&  & 15 & \foreignlanguage{greek}{τοϲ πραγματοϲ ο εαν αιτηϲωνται γε} & 20 &  &  \\
&  & 20 & \foreignlanguage{greek}{νηϲεται αυτοιϲ παρα του \textoverline{πρϲ} μου} & 25 &  &  \\
&  & 26 & \foreignlanguage{greek}{του εν ουρανοιϲ} & 28 &  &  \\
& \textbf{20} &  & \foreignlanguage{greek}{ου γαρ ειϲιν δυο η τριϲ ϲυνηγμενοι} & 7 &  &  \\
&  & 8 & \foreignlanguage{greek}{ειϲ το εμον ονομα εκει ειμει εν μεϲω} & 15 &  &  \\
&  & 16 & \foreignlanguage{greek}{αυτων} & 16 &  &  \\
& \textbf{21} &  & \foreignlanguage{greek}{τοτε προϲελθων αυτω ο πετροϲ ειπε̅} & 6 &  &  \\
&  & 7 & \foreignlanguage{greek}{\textoverline{κε} ποϲακειϲ αμαρτηϲη ειϲ εμε ο α} & 13 &  &  \\
&  & 13 & \foreignlanguage{greek}{δελφοϲ μου και αφηϲω αυτω εωϲ ε} & 19 &  &  \\
&  & 19 & \foreignlanguage{greek}{πτακειϲ} & 19 &  &  \\
[0.2em]
\cline{4-4}
\end{tabular}
\end{center}
\end{table}
}
\clearpage
\newpage
 {
 \setlength\arrayrulewidth{1pt}
\begin{table}
\begin{center}
\begin{tabular}{ccc|l|ccc}
\cline{4-4} \\ [-1em]
\multicolumn{7}{c}{\foreignlanguage{greek}{ευαγγελιον κατα μαθθαιον} \textbf{(\nospace{18:22})} } \\ \\ [-1em] % Si on veut ajouter les bordures latérales, remplacer {7}{c} par {7}{|c|}
\cline{4-4} \\
\cline{4-4}
&  &  & &  &  & \\ [-0.9em]
& \textbf{22} &  & \foreignlanguage{greek}{λεγει αυτω ο \textoverline{ιϲ} ου λεγω ϲοι εωϲ επτακιϲ} & 9 &  &  \\
&  & 10 & \foreignlanguage{greek}{αλλ εωϲ εβδομηκοντακιϲ επτα} & 13 &  &  \\
& \textbf{23} &  & \foreignlanguage{greek}{δια τουτο ωμοιωθη η βαϲιλεια των ου} & 7 &  &  \\
&  & 7 & \foreignlanguage{greek}{ρανων ανθρωπω βαϲιλει οϲ ηθελη} & 11 &  &  \\
&  & 11 & \foreignlanguage{greek}{ϲεν ϲυναρε λογον μετα των δουλω̅} & 16 &  &  \\
&  & 17 & \foreignlanguage{greek}{αυτου αρξαμενου δε αυτου ϲυνε} & 4 & \textbf{24} &  \\
&  & 4 & \foreignlanguage{greek}{ρειν προϲηνεχθη αυτω ειϲ οφιλτηϲ} & 8 &  &  \\
&  & 9 & \foreignlanguage{greek}{μυριων ταλαντων μη εχοντοϲ δε} & 3 & \textbf{25} &  \\
&  & 4 & \foreignlanguage{greek}{αυτου αποδουναι εκελευϲεν αυτο̅} & 7 &  &  \\
&  & 8 & \foreignlanguage{greek}{ο \textoverline{κϲ} αυτου πραθηναι και την γυναι} & 14 &  &  \\
&  & 14 & \foreignlanguage{greek}{κα αυτου και τα τεκνα και παντα} & 20 &  &  \\
&  & 21 & \foreignlanguage{greek}{οϲα ειχεν και αποδοθηναι} & 24 &  &  \\
& \textbf{26} &  & \foreignlanguage{greek}{πεϲων ουν ο δουλοϲ προϲεκυνει αυ} & 6 &  &  \\
&  & 6 & \foreignlanguage{greek}{τω λεγων \textoverline{κε} μακροθυμηϲον ε} & 10 &  &  \\
&  & 10 & \foreignlanguage{greek}{π εμοι και παντα ϲοι αποδωϲω} & 15 &  &  \\
& \textbf{27} &  & \foreignlanguage{greek}{ϲπλαγχνιϲθειϲ δε ο \textoverline{κϲ} του δουλου ε} & 7 &  &  \\
&  & 7 & \foreignlanguage{greek}{κεινου απελυϲεν αυτον και το να} & 12 &  &  \\
&  & 12 & \foreignlanguage{greek}{νιον αφηκεν αυτω} & 14 &  &  \\
& \textbf{28} &  & \foreignlanguage{greek}{εξελθων δε ο δουλοϲ εκεινοϲ ευρεν} & 6 &  &  \\
&  & 7 & \foreignlanguage{greek}{ενα των ϲυνδουλων αυτου οϲ ωφει} & 12 &  &  \\
&  & 12 & \foreignlanguage{greek}{λεν αυτω εκατον δηναρια και κρα} & 17 &  &  \\
&  & 17 & \foreignlanguage{greek}{τηϲαϲ αυτον επνιγεν λεγων αποδοϲ} & 21 &  &  \\
&  & 22 & \foreignlanguage{greek}{ει τι οφιλειϲ πεϲων ουν ο ϲυνδου} & 4 & \textbf{29} &  \\
&  & 4 & \foreignlanguage{greek}{λοϲ αυτου ειϲ τουϲ ποδαϲ αυτου πα} & 10 &  &  \\
&  & 10 & \foreignlanguage{greek}{ρεκαλει αυτον λεγων μακροθυ} & 13 &  &  \\
&  & 13 & \foreignlanguage{greek}{μηϲον επ εμοι και παντα αποδωϲω ϲοι} & 19 &  &  \\
& \textbf{30} &  & \foreignlanguage{greek}{ο δε ουκ ηθελεν αλλα απελθων ε} & 7 &  &  \\
&  & 7 & \foreignlanguage{greek}{βαλεν αυτον ειϲ φυλακην εωϲ ου} & 12 &  &  \\
&  & 13 & \foreignlanguage{greek}{αποδω το οφιλομενον} & 15 &  &  \\
& \textbf{31} &  & \foreignlanguage{greek}{ιδοντεϲ δε οι ϲυνδουλοι αυτου τα γε} & 7 &  &  \\
[0.2em]
\cline{4-4}
\end{tabular}
\end{center}
\end{table}
}
\clearpage
\newpage
 {
 \setlength\arrayrulewidth{1pt}
\begin{table}
\begin{center}
\begin{tabular}{ccc|l|ccc}
\cline{4-4} \\ [-1em]
\multicolumn{7}{c}{\foreignlanguage{greek}{ευαγγελιον κατα μαθθαιον} \textbf{(\nospace{18:31})} } \\ \\ [-1em] % Si on veut ajouter les bordures latérales, remplacer {7}{c} par {7}{|c|}
\cline{4-4} \\
\cline{4-4}
&  &  & &  &  & \\ [-0.9em]
&  & 7 & \foreignlanguage{greek}{νομενα ελυπηθηϲαν ϲφοδρα και ελθο̅} & 11 &  &  \\
&  & 11 & \foreignlanguage{greek}{τεϲ διεϲαφηϲαν τω \textoverline{κω} εαυτων παν} & 16 &  &  \\
&  & 16 & \foreignlanguage{greek}{τα τα γενομενα} & 18 &  &  \\
& \textbf{32} &  & \foreignlanguage{greek}{τοτε προϲκαλεϲαμενοϲ αυτον ο κυρι} & 5 &  &  \\
&  & 5 & \foreignlanguage{greek}{οϲ αυτου λεγει αυτω δουλε πονηρε} & 10 &  &  \\
&  & 11 & \foreignlanguage{greek}{παϲαν την οφιλην εκεινην αφηκα} & 15 &  &  \\
&  & 16 & \foreignlanguage{greek}{ϲοι επι παρεκαλεϲαϲ με ουκ εδει και} & 3 & \textbf{33} &  \\
&  & 4 & \foreignlanguage{greek}{ϲε ελεηϲαι τον ϲυνδουλον ϲου ωϲ και} & 10 &  &  \\
&  & 11 & \foreignlanguage{greek}{εγω ϲε ηλεηϲα και οργειϲθειϲ ο \textoverline{κϲ}} & 4 & \textbf{34} &  \\
&  & 5 & \foreignlanguage{greek}{αυτου παρεδωκεν αυτον τοιϲ μαϲα} & 9 &  &  \\
&  & 9 & \foreignlanguage{greek}{νιϲταιϲ εωϲ ου αποδω παν το οφιλο} & 15 &  &  \\
&  & 15 & \foreignlanguage{greek}{μενον αυτω ουτωϲ και ο \textoverline{πηρ} μου} & 5 & \textbf{35} &  \\
&  & 6 & \foreignlanguage{greek}{ο επουρανιοϲ ποιηϲει υμιν εαν μη} & 11 &  &  \\
&  & 12 & \foreignlanguage{greek}{αφηται εκαϲτοϲ τω αδελφω αυτου} & 16 &  &  \\
&  & 17 & \foreignlanguage{greek}{απο των καρδιων υμων τα παρα} & 22 &  &  \\
&  & 22 & \foreignlanguage{greek}{πτωματα αυτων} & 23 &  &  \\
& \mygospelchapter &  & \foreignlanguage{greek}{και εγενετο οτε ετελεϲεν ο \textoverline{ιϲ} τουϲ λο} & 8 &  &  \\
&  & 8 & \foreignlanguage{greek}{γουϲ τουτουϲ μετηρεν απο τηϲ γαλι} & 13 &  &  \\
&  & 13 & \foreignlanguage{greek}{λαιαϲ και ηλθεν ειϲ τα ορια τηϲ ιου} & 20 &  &  \\
&  & 20 & \foreignlanguage{greek}{δαιαϲ περαν του ιορδανου και ηκο} & 2 & \textbf{2} &  \\
&  & 2 & \foreignlanguage{greek}{λουθηϲαν αυτω οχλοι πολλοι και εθε} & 7 &  &  \\
&  & 7 & \foreignlanguage{greek}{ραπευϲεν αυτουϲ εκει} & 9 &  &  \\
& \textbf{3} &  & \foreignlanguage{greek}{και προϲηλθον αυτω φαριϲαιοι πει} & 5 &  &  \\
&  & 5 & \foreignlanguage{greek}{ραζοντεϲ αυτον και λεγοντεϲ αυτω} & 9 &  &  \\
&  & 10 & \foreignlanguage{greek}{ει εξεϲτιν \textoverline{ανω} απολυϲαι την γυναι} & 15 &  &  \\
&  & 15 & \foreignlanguage{greek}{κα αυτου κατα παϲαν αιτιαν} & 19 &  &  \\
& \textbf{4} &  & \foreignlanguage{greek}{ο δε αποκριθειϲ ειπεν αυτοιϲ ουκ α} & 7 &  &  \\
&  & 7 & \foreignlanguage{greek}{νεγνωται οτι ο ποιηϲαϲ απ αρχηϲ αρ} & 13 &  &  \\
&  & 13 & \foreignlanguage{greek}{ϲεν και θηλυ εποιηϲεν αυτουϲ και} & 1 & \textbf{5} &  \\
&  & 2 & \foreignlanguage{greek}{ειπεν ενεκεν τουτου καταλιψει} & 5 &  &  \\
[0.2em]
\cline{4-4}
\end{tabular}
\end{center}
\end{table}
}
\clearpage
\newpage
 {
 \setlength\arrayrulewidth{1pt}
\begin{table}
\begin{center}
\begin{tabular}{ccc|l|ccc}
\cline{4-4} \\ [-1em]
\multicolumn{7}{c}{\foreignlanguage{greek}{ευαγγελιον κατα μαθθαιον} \textbf{(\nospace{19:5})} } \\ \\ [-1em] % Si on veut ajouter les bordures latérales, remplacer {7}{c} par {7}{|c|}
\cline{4-4} \\
\cline{4-4}
&  &  & &  &  & \\ [-0.9em]
&  & 6 & \foreignlanguage{greek}{\textoverline{ανοϲ} τον \textoverline{πρα} και την \textoverline{μρα} αυτου και κολ} & 14 &  &  \\
&  & 14 & \foreignlanguage{greek}{ληθηϲεται τη γυναικει αυτου και εϲο̅} & 19 &  &  \\
&  & 19 & \foreignlanguage{greek}{ται οι δυο ειϲ ϲαρκα μιαν ωϲτε ουκετι} & 2 & \textbf{6} &  \\
&  & 3 & \foreignlanguage{greek}{ειϲιν δυο αλλα ϲαρξ μια ο ουν ο \textoverline{θϲ} ϲυν} & 12 &  &  \\
&  & 12 & \foreignlanguage{greek}{εζευξεν \textoverline{ανοϲ} μη χωριζετω} & 15 &  &  \\
& \textbf{7} &  & \foreignlanguage{greek}{λεγουϲιν αυτω τι ουν μωυϲηϲ ενετι} & 6 &  &  \\
&  & 6 & \foreignlanguage{greek}{λατο δουναι βιβλιον αποϲταϲιου και α} & 12 &  &  \\
&  & 12 & \foreignlanguage{greek}{πολυϲαι αυτην λεγει αυτοιϲ οτι μω} & 4 & \textbf{8} &  \\
&  & 4 & \foreignlanguage{greek}{υϲηϲ επετρεψεν υμιν προϲ την ϲκληροκαρ} & 9 &  &  \\
&  & 9 & \foreignlanguage{greek}{διαν υμων απολυϲαι ταϲ γυναικαϲ υ} & 14 &  &  \\
&  & 14 & \foreignlanguage{greek}{μων απ αρχηϲ δε ου γεγονεν ουτωϲ} & 20 &  &  \\
& \textbf{9} &  & \foreignlanguage{greek}{λεγω δε υμιν οτι οϲ αν απολυϲη την} & 8 &  &  \\
&  & 9 & \foreignlanguage{greek}{γυναικα αυτου μη επι πορνια γαμηϲη} & 15 &  &  \\
&  & 16 & \foreignlanguage{greek}{αλλην μοιχατε και ο απολελυμενη̅} & 20 &  &  \\
&  & 21 & \foreignlanguage{greek}{γαμων μοιχατε} & 22 &  &  \\
& \textbf{10} &  & \foreignlanguage{greek}{λεγουϲιν αυτω οι μαθηται αυτου ει ου} & 7 &  &  \\
&  & 7 & \foreignlanguage{greek}{τωϲ εϲτιν η αιτια του \textoverline{ανου} μετα τηϲ} & 14 &  &  \\
&  & 15 & \foreignlanguage{greek}{γυναικοϲ ου ϲυμφερει γαμηϲαι} & 18 &  &  \\
& \textbf{11} &  & \foreignlanguage{greek}{ο δε ειπεν αυτοιϲ ου παντεϲ χωρουϲιν} & 7 &  &  \\
&  & 8 & \foreignlanguage{greek}{τον λογον τουτον αλλ οιϲ δεδοται} & 13 &  &  \\
& \textbf{12} &  & \foreignlanguage{greek}{ειϲιν γαρ ευνουχοι οιτινεϲ εκ κοιλι} & 6 &  &  \\
&  & 6 & \foreignlanguage{greek}{αϲ μητροϲ εγεννηθηϲαν ουτωϲ} & 9 &  &  \\
&  & 10 & \foreignlanguage{greek}{και ειϲιν ευνουχοι οιτινεϲ ευνουχι} & 14 &  &  \\
&  & 14 & \foreignlanguage{greek}{ϲθηϲαν υπο των \textoverline{ανων} και ειϲιν ευ} & 20 &  &  \\
&  & 20 & \foreignlanguage{greek}{νουχοι οιτινεϲ ευνουχιϲαν εαυτουϲ} & 23 &  &  \\
&  & 24 & \foreignlanguage{greek}{δια την βαϲιλειαν των ουρανων} & 28 &  &  \\
&  & 29 & \foreignlanguage{greek}{ο δυναμενοϲ χωριν χωρειτω} & 32 &  &  \\
& \textbf{13} &  & \foreignlanguage{greek}{τοτε προϲηνεχθη αυτω παιδια ινα ταϲ} & 6 &  &  \\
&  & 7 & \foreignlanguage{greek}{χειραϲ επιθη αυτοιϲ και προϲευξηται} & 11 &  &  \\
&  & 12 & \foreignlanguage{greek}{οι δε μαθηται επετιμηϲαν αυτοιϲ} & 16 &  &  \\
[0.2em]
\cline{4-4}
\end{tabular}
\end{center}
\end{table}
}
\clearpage
\newpage
 {
 \setlength\arrayrulewidth{1pt}
\begin{table}
\begin{center}
\begin{tabular}{ccc|l|ccc}
\cline{4-4} \\ [-1em]
\multicolumn{7}{c}{\foreignlanguage{greek}{ευαγγελιον κατα μαθθαιον} \textbf{(\nospace{19:14})} } \\ \\ [-1em] % Si on veut ajouter les bordures latérales, remplacer {7}{c} par {7}{|c|}
\cline{4-4} \\
\cline{4-4}
&  &  & &  &  & \\ [-0.9em]
& \textbf{14} &  & \foreignlanguage{greek}{ο δε \textoverline{ιϲ} ειπεν αυτοιϲ αφεται τα παιδια και} & 9 &  &  \\
&  & 10 & \foreignlanguage{greek}{μη κωλυεται αυτα ελθειν προϲ με των} & 16 &  &  \\
&  & 17 & \foreignlanguage{greek}{γαρ τοιουτων εϲτιν η βαϲιλεια των ου} & 23 &  &  \\
&  & 23 & \foreignlanguage{greek}{ρανων και επιθειϲ αυτοιϲ ταϲ χειραϲ} & 5 & \textbf{15} &  \\
&  & 6 & \foreignlanguage{greek}{επορευθη εκειθεν} & 7 &  &  \\
& \textbf{16} &  & \foreignlanguage{greek}{και ιδου ειϲ προϲελθων ειπεν αυτω} & 6 &  &  \\
&  & 7 & \foreignlanguage{greek}{διδαϲκαλε αγαθε τι αγαθον ποιηϲω} & 11 &  &  \\
&  & 12 & \foreignlanguage{greek}{ινα ζωην εχω αιωνιον ο δε ειπεν} & 3 & \textbf{17} &  \\
&  & 4 & \foreignlanguage{greek}{αυτω τι με λεγειϲ αγαθον ουδειϲ α} & 10 &  &  \\
&  & 10 & \foreignlanguage{greek}{γαθοϲ ει μη ειϲ ο \textoverline{θϲ} ει δε θελειϲ ειϲ} & 19 &  &  \\
&  & 19 & \foreignlanguage{greek}{ελθειν ειϲ την ζωην τηρηϲον ταϲ εν} & 25 &  &  \\
&  & 25 & \foreignlanguage{greek}{τολαϲ λεγει αυτω ποιαϲ} & 3 & \textbf{18} &  \\
&  & 4 & \foreignlanguage{greek}{ο δε \textoverline{ιϲ} ειπεν ου φονευϲιϲ ου μοιχευ} & 11 &  &  \\
&  & 11 & \foreignlanguage{greek}{ϲιϲ ου κλεψειϲ ου ψευδομαρτυρηϲιϲ} & 15 &  &  \\
& \textbf{19} &  & \foreignlanguage{greek}{τιμα τον \textoverline{πρα} ϲου και την \textoverline{μρα} και αγα} & 9 &  &  \\
&  & 9 & \foreignlanguage{greek}{πηϲιϲ τον πληϲιον ϲου ωϲ ϲεαυτον} & 14 &  &  \\
& \textbf{20} &  & \foreignlanguage{greek}{λεγει αυτω ο νεανιϲκοϲ παντα ταυτα} & 6 &  &  \\
&  & 7 & \foreignlanguage{greek}{εφυλαξαμην εκ νεοτητοϲ μου τι} & 11 &  &  \\
&  & 12 & \foreignlanguage{greek}{ετι υϲτερω εφη αυτω ο \textoverline{ιϲ}} & 4 & \textbf{21} &  \\
&  & 5 & \foreignlanguage{greek}{ει θελειϲ τελιοϲ ειναι υπαγε πωλη} & 10 &  &  \\
&  & 10 & \foreignlanguage{greek}{ϲον ϲου τα υπαρχοντα και δοϲ πτω} & 16 &  &  \\
&  & 16 & \foreignlanguage{greek}{χοιϲ και εξειϲ θηϲαυρον εν ουρανω} & 21 &  &  \\
&  & 22 & \foreignlanguage{greek}{και δευρο ακολουθει μοι} & 25 &  &  \\
& \textbf{22} &  & \foreignlanguage{greek}{ακουϲαϲ δε ο νεανιϲκοϲ τον λογον} & 6 &  &  \\
&  & 7 & \foreignlanguage{greek}{απηλθεν λυπουμενοϲ ην γαρ εχω̅} & 11 &  &  \\
&  & 12 & \foreignlanguage{greek}{κτηματα πολλα} & 13 &  &  \\
& \textbf{23} &  & \foreignlanguage{greek}{ο δε \textoverline{ιϲ} ειπεν τοιϲ μαθηταιϲ αυτου} & 7 &  &  \\
&  & 8 & \foreignlanguage{greek}{αμην λεγω υμιν οτι δυϲκολωϲ πλου} & 13 &  &  \\
&  & 13 & \foreignlanguage{greek}{ϲιοϲ ειϲελευϲεται ειϲ την βαϲιλεια̅} & 17 &  &  \\
&  & 18 & \foreignlanguage{greek}{των ουρανων παλιν δε λεγω υμι̅} & 4 & \textbf{24} &  \\
[0.2em]
\cline{4-4}
\end{tabular}
\end{center}
\end{table}
}
\clearpage
\newpage
 {
 \setlength\arrayrulewidth{1pt}
\begin{table}
\begin{center}
\begin{tabular}{ccc|l|ccc}
\cline{4-4} \\ [-1em]
\multicolumn{7}{c}{\foreignlanguage{greek}{ευαγγελιον κατα μαθθαιον} \textbf{(\nospace{19:24})} } \\ \\ [-1em] % Si on veut ajouter les bordures latérales, remplacer {7}{c} par {7}{|c|}
\cline{4-4} \\
\cline{4-4}
&  &  & &  &  & \\ [-0.9em]
&  & 5 & \foreignlanguage{greek}{ευκοπωτερον εϲτιν καμηλον ελθει̅} & 8 &  &  \\
&  & 9 & \foreignlanguage{greek}{δια τρυπηματοϲ ραφιδοϲ η πλουϲιον} & 13 &  &  \\
&  & 14 & \foreignlanguage{greek}{ειϲ την βαϲιλειαν του \textoverline{θυ} ειϲελθειν} & 19 &  &  \\
& \textbf{25} &  & \foreignlanguage{greek}{ακουϲαντεϲ δε οι μαθηται αυτου εξε} & 6 &  &  \\
&  & 6 & \foreignlanguage{greek}{πληϲϲοντο ϲφοδρα λεγοντεϲ τιϲ αρα} & 10 &  &  \\
&  & 11 & \foreignlanguage{greek}{δυναται ϲωθηναι} & 12 &  &  \\
& \textbf{26} &  & \foreignlanguage{greek}{εμβλεψαϲ δε ο \textoverline{ιϲ} ειπεν αυτοιϲ παρα} & 7 &  &  \\
&  & 8 & \foreignlanguage{greek}{\textoverline{ανοιϲ} τουτο αδυνατον εϲτιν παρα} & 12 &  &  \\
&  & 13 & \foreignlanguage{greek}{δε \textoverline{θω} παντα δυνατα} & 16 &  &  \\
& \textbf{27} &  & \foreignlanguage{greek}{τοτε αποκριθειϲ ο πετροϲ ειπεν αυτω} & 6 &  &  \\
&  & 7 & \foreignlanguage{greek}{ιδου ημειϲ αφηκαμεν παντα και η} & 12 &  &  \\
&  & 12 & \foreignlanguage{greek}{κολουθηϲαμεν ϲοι τι αρα εϲται ημιν} & 17 &  &  \\
& \textbf{28} &  & \foreignlanguage{greek}{ο δε \textoverline{ιϲ} ειπεν αυτοιϲ αμην λεγω υμιν} & 8 &  &  \\
&  & 9 & \foreignlanguage{greek}{οτι υμειϲ οι ακολουθηϲαντεϲ μοι εν} & 14 &  &  \\
&  & 15 & \foreignlanguage{greek}{τη παλινγενεϲια οταν καθειϲη ο} & 19 &  &  \\
&  & 20 & \foreignlanguage{greek}{υιοϲ του \textoverline{ανου} επι θρονου δοξηϲ αυτου} & 26 &  &  \\
&  & 27 & \foreignlanguage{greek}{καθηϲεϲθαι και υμειϲ επι δωδεκα} & 31 &  &  \\
&  & 32 & \foreignlanguage{greek}{θρονουϲ κρινοντεϲ ταϲ δωδεκα φυ} & 36 &  &  \\
&  & 36 & \foreignlanguage{greek}{λαϲ του ιϲτραηλ και παϲ οϲτιϲ αφηκε̅} & 4 & \textbf{29} &  \\
&  & 5 & \foreignlanguage{greek}{οικειαϲ η αδελφουϲ η αδελφαϲ η πα} & 11 &  &  \\
&  & 11 & \foreignlanguage{greek}{τερα η \textoverline{μρα} η γυναικα η τεκνα η α} & 19 &  &  \\
&  & 19 & \foreignlanguage{greek}{γρουϲ ενεκεν του ονοματοϲ μου} & 23 &  &  \\
&  & 24 & \foreignlanguage{greek}{εκατονταπλαϲιονα λημψεται και} & 26 &  &  \\
&  & 27 & \foreignlanguage{greek}{ζωην αιωνιον κληρονομηϲει} & 29 &  &  \\
& \textbf{30} &  & \foreignlanguage{greek}{πολλοι δε εϲονται πρωτοι εϲχατοι} & 5 &  &  \\
&  & 6 & \foreignlanguage{greek}{και εϲχατοι εϲονται πρωτοι} & 9 &  &  \\
& \mygospelchapter &  & \foreignlanguage{greek}{οιμοια γαρ εϲτιν η βαϲιλεια των ουρα} & 7 &  &  \\
&  & 7 & \foreignlanguage{greek}{νων \textoverline{ανω} οικοδεϲποτη οϲτιϲ εξηλθε̅} & 11 &  &  \\
&  & 12 & \foreignlanguage{greek}{αμα πρωει μιϲθωϲαϲθαι εργαταϲ ειϲ το̅} & 17 &  &  \\
&  & 18 & \foreignlanguage{greek}{αμπελωνα αυτου ϲυμφωνηϲαϲ δε} & 2 & \textbf{2} &  \\
[0.2em]
\cline{4-4}
\end{tabular}
\end{center}
\end{table}
}
\clearpage
\newpage
 {
 \setlength\arrayrulewidth{1pt}
\begin{table}
\begin{center}
\begin{tabular}{ccc|l|ccc}
\cline{4-4} \\ [-1em]
\multicolumn{7}{c}{\foreignlanguage{greek}{ευαγγελιον κατα μαθθαιον} \textbf{(\nospace{20:2})} } \\ \\ [-1em] % Si on veut ajouter les bordures latérales, remplacer {7}{c} par {7}{|c|}
\cline{4-4} \\
\cline{4-4}
&  &  & &  &  & \\ [-0.9em]
&  & 3 & \foreignlanguage{greek}{μετα των εργατων εκ δηναριου την} & 8 &  &  \\
&  & 9 & \foreignlanguage{greek}{ημεραν απεϲτιλεν αυτουϲ ειϲ τον αμ} & 14 &  &  \\
&  & 14 & \foreignlanguage{greek}{πελωνα αυτου και εξελθων περι} & 3 & \textbf{3} &  \\
&  & 4 & \foreignlanguage{greek}{τριτην ωραν ειδεν αλλουϲ εϲτωταϲ} & 8 &  &  \\
&  & 9 & \foreignlanguage{greek}{εν τη αγορα αργουϲ και εκεινοϲ} & 2 & \textbf{4} &  \\
&  & 3 & \foreignlanguage{greek}{ειπεν υπαγεται και υμειϲ ειϲ τον} & 8 &  &  \\
&  & 9 & \foreignlanguage{greek}{αμπελωνα και ο εαν η δικαιον δω} & 15 &  &  \\
&  & 15 & \foreignlanguage{greek}{ϲω υμιν οι δε απηλθον παλιν} & 4 & \textbf{5} &  \\
&  & 5 & \foreignlanguage{greek}{εξελθων περι εκτην και ενατην} & 9 &  &  \\
&  & 10 & \foreignlanguage{greek}{ωραν εποιηϲεν ωϲαυτωϲ} & 12 &  &  \\
& \textbf{6} &  & \foreignlanguage{greek}{περι δε την ενδεκατην ωραν εξελ} & 6 &  &  \\
&  & 6 & \foreignlanguage{greek}{θων ευρεν αλλουϲ εϲτωταϲ αργουϲ} & 10 &  &  \\
&  & 11 & \foreignlanguage{greek}{και λεγει αυτοιϲ τι ωδε εϲτηκατε} & 16 &  &  \\
&  & 17 & \foreignlanguage{greek}{ολην την ημεραν αργοι λεγουϲιν} & 1 & \textbf{7} &  \\
&  & 2 & \foreignlanguage{greek}{αυτω οτι ουδειϲ ημαϲ εμιϲθωϲατο} & 6 &  &  \\
&  & 7 & \foreignlanguage{greek}{λεγει αυτοιϲ υπαγεται και υμειϲ} & 11 &  &  \\
&  & 12 & \foreignlanguage{greek}{ειϲ τον αμπελωνα και ο εαν η δι} & 19 &  &  \\
&  & 19 & \foreignlanguage{greek}{καιον ληψεϲθαι} & 20 &  &  \\
& \textbf{8} &  & \foreignlanguage{greek}{οψειαϲ δε γενομενηϲ λεγει ο \textoverline{κϲ} του} & 7 &  &  \\
&  & 8 & \foreignlanguage{greek}{αμπελωνοϲ τω επιτροπω αυτου} & 11 &  &  \\
&  & 12 & \foreignlanguage{greek}{καλεϲον τουϲ εργαταϲ και αποδοϲ} & 16 &  &  \\
&  & 17 & \foreignlanguage{greek}{αυτοιϲ τον μιϲθον αρξαμενοϲ α} & 21 &  &  \\
&  & 21 & \foreignlanguage{greek}{πο των εϲχατων εωϲ των πρωτων} & 26 &  &  \\
& \textbf{9} &  & \foreignlanguage{greek}{και ελθοντεϲ οι περι την ενδεκατη̅} & 6 &  &  \\
&  & 7 & \foreignlanguage{greek}{ωραν ελαβον ανα δηναριον ελθο̅} & 1 & \textbf{10} &  \\
&  & 1 & \foreignlanguage{greek}{τεϲ δε οι πρωτοι ενομιϲαν οτι πλιο̅} & 7 &  &  \\
&  & 8 & \foreignlanguage{greek}{λημψονται και ελαβον και αυτοι} & 12 &  &  \\
&  & 13 & \foreignlanguage{greek}{ανα δηναριον λαβοντεϲ δε εγογ} & 3 & \textbf{11} &  \\
&  & 3 & \foreignlanguage{greek}{γυζον κατα του οικοδεϲποτου λεγο̅} & 1 & \textbf{12} &  \\
&  & 1 & \foreignlanguage{greek}{τεϲ οτι ουτοι οι εϲχατοι μιαν ωραν} & 7 &  &  \\
[0.2em]
\cline{4-4}
\end{tabular}
\end{center}
\end{table}
}
\clearpage
\newpage
 {
 \setlength\arrayrulewidth{1pt}
\begin{table}
\begin{center}
\begin{tabular}{ccc|l|ccc}
\cline{4-4} \\ [-1em]
\multicolumn{7}{c}{\foreignlanguage{greek}{ευαγγελιον κατα μαθθαιον} \textbf{(\nospace{20:12})} } \\ \\ [-1em] % Si on veut ajouter les bordures latérales, remplacer {7}{c} par {7}{|c|}
\cline{4-4} \\
\cline{4-4}
&  &  & &  &  & \\ [-0.9em]
&  & 8 & \foreignlanguage{greek}{εποιηϲαν και ιϲουϲ ημιν αυτουϲ εποι} & 14 &  &  \\
&  & 14 & \foreignlanguage{greek}{ηϲαϲ τοιϲ βαϲταϲαϲιν το βαροϲ τηϲ} & 19 &  &  \\
&  & 20 & \foreignlanguage{greek}{ημεραϲ και τον καυϲωνα ο δε απο} & 3 & \textbf{13} &  \\
&  & 3 & \foreignlanguage{greek}{κριθειϲ ειπεν ενι αυτων ετερε ουκ α} & 9 &  &  \\
&  & 9 & \foreignlanguage{greek}{δικω ϲε ουχι δηναριου ϲυνεφωνη} & 13 &  &  \\
&  & 13 & \foreignlanguage{greek}{ϲαϲ μοι αρον το ϲον και υπαγε θελω} & 6 & \textbf{14} &  \\
&  & 7 & \foreignlanguage{greek}{δε τουτω τω εϲχατω δουναι ωϲ και ϲοι} & 14 &  &  \\
& \textbf{15} &  & \foreignlanguage{greek}{η ουκ εξεϲτιν μοι ποιηϲαι ωϲ θελω εν} & 8 &  &  \\
&  & 9 & \foreignlanguage{greek}{τοιϲ εμοιϲ η ο οφθαλμοϲ ϲου πονηροϲ} & 15 &  &  \\
&  & 16 & \foreignlanguage{greek}{εϲτιν οτι εγω αγαθοϲ ειμει ουτωϲ} & 1 & \textbf{16} &  \\
&  & 2 & \foreignlanguage{greek}{εϲονται οι εϲχατοι πρωτοι και οι πρωτοι} & 8 &  &  \\
&  & 9 & \foreignlanguage{greek}{εϲχατοι πολλοι γαρ ειϲιν κλητοι ολι} & 14 &  &  \\
&  & 14 & \foreignlanguage{greek}{γοι δε εκλεκτοι} & 16 &  &  \\
& \textbf{17} &  & \foreignlanguage{greek}{και αναβαινων ο \textoverline{ιϲ} ειϲ ιεροϲολυμα πα} & 7 &  &  \\
&  & 7 & \foreignlanguage{greek}{ρελαβεν τουϲ δωδεκα μαθηταϲ κατ} & 11 &  &  \\
&  & 12 & \foreignlanguage{greek}{ιδιαν εν τη οδω και ειπεν αυτοιϲ} & 18 &  &  \\
& \textbf{18} &  & \foreignlanguage{greek}{ιδου αναβαινομεν ειϲ ιεροϲολυμα και} & 5 &  &  \\
&  & 6 & \foreignlanguage{greek}{ο υιοϲ του \textoverline{ανου} παραδοθηϲεται τοιϲ αρ} & 12 &  &  \\
&  & 12 & \foreignlanguage{greek}{χιερευϲιν και γραμματευϲιν και κα} & 16 &  &  \\
&  & 16 & \foreignlanguage{greek}{τακρινουϲιν αυτον θανατω και παρα} & 2 & \textbf{19} &  \\
&  & 2 & \foreignlanguage{greek}{δωϲουϲιν αυτον τοιϲ εθνεϲιν ειϲ το} & 7 &  &  \\
&  & 8 & \foreignlanguage{greek}{ενπεξαι και μαϲτιγωϲαι και ϲταυρωϲαι} & 12 &  &  \\
&  & 13 & \foreignlanguage{greek}{και τη τριτη ημερα αναϲτηϲεται} & 17 &  &  \\
& \textbf{20} &  & \foreignlanguage{greek}{τοτε προϲηλθεν αυτω η \textoverline{μηρ} των υιων} & 7 &  &  \\
&  & 8 & \foreignlanguage{greek}{ζεβεδαιου μετα των υιων αυτηϲ προϲ} & 13 &  &  \\
&  & 13 & \foreignlanguage{greek}{κυνουϲα και αιτουϲα τι παρ αυτου} & 18 &  &  \\
& \textbf{21} &  & \foreignlanguage{greek}{ο δε ειπεν αυτη τι θελειϲ λεγει αυτω} & 8 &  &  \\
&  & 9 & \foreignlanguage{greek}{ειπε ινα καθιϲωϲιν ουτοι οι δυο υιοι μου} & 16 &  &  \\
&  & 17 & \foreignlanguage{greek}{ειϲ εκ δεξιων ϲου και ειϲ εξ ευωνυμω̅} & 24 &  &  \\
&  & 25 & \foreignlanguage{greek}{ϲου εν τη βαϲιλεια ϲου} & 29 &  &  \\
[0.2em]
\cline{4-4}
\end{tabular}
\end{center}
\end{table}
}
\clearpage
\newpage
 {
 \setlength\arrayrulewidth{1pt}
\begin{table}
\begin{center}
\begin{tabular}{ccc|l|ccc}
\cline{4-4} \\ [-1em]
\multicolumn{7}{c}{\foreignlanguage{greek}{ευαγγελιον κατα μαθθαιον} \textbf{(\nospace{20:22})} } \\ \\ [-1em] % Si on veut ajouter les bordures latérales, remplacer {7}{c} par {7}{|c|}
\cline{4-4} \\
\cline{4-4}
&  &  & &  &  & \\ [-0.9em]
& \textbf{22} &  & \foreignlanguage{greek}{αποκριθειϲ δε ο \textoverline{ιϲ} ειπεν ουκ οιδατε} & 7 &  &  \\
&  & 8 & \foreignlanguage{greek}{τι αιτιϲθαι δυναϲθαι πιν το ποτηρι} & 13 &  &  \\
&  & 13 & \foreignlanguage{greek}{ον ο εγω μελλω πινειν η το βαπτιϲ} & 20 &  &  \\
&  & 20 & \foreignlanguage{greek}{μα ο εγω βαπτιζομαι βαπτιϲθηναι} & 24 &  &  \\
&  & 25 & \foreignlanguage{greek}{λεγουϲιν αυτω δυναμεθα και λεγει} & 2 & \textbf{23} &  \\
&  & 3 & \foreignlanguage{greek}{αυτοιϲ το μεν ποτηριον μου πιεϲθε} & 8 &  &  \\
&  & 9 & \foreignlanguage{greek}{και το βαπτιϲμα ο εγω βαπτιζομαι} & 14 &  &  \\
&  & 15 & \foreignlanguage{greek}{βαπτιϲθηϲεϲθαι το δε καθειϲαι εκ} & 19 &  &  \\
&  & 20 & \foreignlanguage{greek}{δεξιων μου και εξ ευωνυμων μου} & 25 &  &  \\
&  & 26 & \foreignlanguage{greek}{ουκ εϲτιν εμον τουτο δουναι αλλ οιϲ} & 32 &  &  \\
&  & 33 & \foreignlanguage{greek}{ητοιμαϲται υπο του \textoverline{πρϲ} μου} & 37 &  &  \\
& \textbf{24} &  & \foreignlanguage{greek}{και ακουϲαντεϲ οι δεκα ηγανακτη} & 5 &  &  \\
&  & 5 & \foreignlanguage{greek}{ϲαν περι των δυο αδελφων} & 9 &  &  \\
& \textbf{25} &  & \foreignlanguage{greek}{ο δε \textoverline{ιϲ} προϲκαλεϲαμενοϲ αυτουϲ ει} & 6 &  &  \\
&  & 6 & \foreignlanguage{greek}{πεν αυτοιϲ οιδατε οτι οι αρχοντεϲ} & 11 &  &  \\
&  & 12 & \foreignlanguage{greek}{των εθνων κατακυριευουϲιν αυτω̅} & 15 &  &  \\
&  & 16 & \foreignlanguage{greek}{και οι μεγαλοι κατεξουϲιαζουϲιν αυ} & 20 &  &  \\
&  & 20 & \foreignlanguage{greek}{των ουχ ουτωϲ εϲται εν υμιν} & 5 & \textbf{26} &  \\
&  & 6 & \foreignlanguage{greek}{αλλ οϲ εαν θελη εν υμιν μεγαϲ γενε} & 13 &  &  \\
&  & 13 & \foreignlanguage{greek}{ϲθαι εϲται υμων διακονοϲ και οϲ αν} & 3 & \textbf{27} &  \\
&  & 4 & \foreignlanguage{greek}{θελη εν υμιν πρωτοϲ ειναι εϲται υ} & 10 &  &  \\
&  & 10 & \foreignlanguage{greek}{μων δουλοϲ ωϲπερ ο υιοϲ του} & 4 & \textbf{28} &  \\
&  & 5 & \foreignlanguage{greek}{ανθρωπου ουκ ηλθεν διακονηθηναι} & 8 &  &  \\
&  & 9 & \foreignlanguage{greek}{αλλα διακονηϲαι και δουναι την ψυ} & 14 &  &  \\
&  & 14 & \foreignlanguage{greek}{χην αυτου λυτρον αντι πολλων} & 18 &  &  \\
& \textbf{29} &  & \foreignlanguage{greek}{και εκπορευομενων αυτων απο ιε} & 5 &  &  \\
&  & 5 & \foreignlanguage{greek}{ριχω ηκολουθηϲεν αυτω οχλοϲ πολυϲ} & 9 &  &  \\
& \textbf{30} &  & \foreignlanguage{greek}{και ιδου δυο τυφλοι καθημενοι παρα} & 6 &  &  \\
&  & 7 & \foreignlanguage{greek}{την οδον ακουϲαντεϲ οτι \textoverline{ιϲ} παραγει} & 12 &  &  \\
&  & 13 & \foreignlanguage{greek}{εκραξαν λεγοντεϲ ελεηϲον ημαϲ} & 16 &  &  \\
[0.2em]
\cline{4-4}
\end{tabular}
\end{center}
\end{table}
}
\clearpage
\newpage
 {
 \setlength\arrayrulewidth{1pt}
\begin{table}
\begin{center}
\begin{tabular}{ccc|l|ccc}
\cline{4-4} \\ [-1em]
\multicolumn{7}{c}{\foreignlanguage{greek}{ευαγγελιον κατα μαθθαιον} \textbf{(\nospace{20:30})} } \\ \\ [-1em] % Si on veut ajouter les bordures latérales, remplacer {7}{c} par {7}{|c|}
\cline{4-4} \\
\cline{4-4}
&  &  & &  &  & \\ [-0.9em]
&  & 17 & \foreignlanguage{greek}{\textoverline{κε} υιοϲ δαυειδ} & 19 &  &  \\
& \textbf{31} &  & \foreignlanguage{greek}{ο δε οχλοϲ επετιμηϲεν αυτοιϲ ινα ϲιω} & 7 &  &  \\
&  & 7 & \foreignlanguage{greek}{πηϲωϲιν οι δε μιζον εκραζον λεγοντεϲ} & 12 &  &  \\
&  & 13 & \foreignlanguage{greek}{ελεηϲον ημαϲ \textoverline{κε} υιοϲ δαυειδ} & 17 &  &  \\
& \textbf{32} &  & \foreignlanguage{greek}{και ϲταϲ ο \textoverline{ιϲ} εφωνηϲεν αυτουϲ και ει} & 8 &  &  \\
&  & 8 & \foreignlanguage{greek}{πεν τι θελεται ποιηϲω υμιν} & 12 &  &  \\
& \textbf{33} &  & \foreignlanguage{greek}{λεγουϲιν αυτω \textoverline{κε} ινα ανεωχθωϲιν ημω̅} & 6 &  &  \\
&  & 7 & \foreignlanguage{greek}{οι οφθαλμοι ϲπλανχνιϲθειϲ δε ο \textoverline{ιϲ}} & 4 & \textbf{34} &  \\
&  & 5 & \foreignlanguage{greek}{ηψατο των οφθαλμων αυτων και ευθε} & 10 &  &  \\
&  & 10 & \foreignlanguage{greek}{ωϲ ανεβλεψαν αυτων οι οφθαλμοι και η} & 16 &  &  \\
&  & 16 & \foreignlanguage{greek}{κολουθηϲαν αυτω} & 17 &  &  \\
& \mygospelchapter &  & \foreignlanguage{greek}{και οτε ηγγιϲαν ειϲ ιεροϲολυμα και ηλθε̅} & 7 &  &  \\
&  & 8 & \foreignlanguage{greek}{ειϲ βηθϲφαγη προϲ το οροϲ των ελεων} & 14 &  &  \\
&  & 15 & \foreignlanguage{greek}{τοτε ο \textoverline{ιϲ} απεϲτιλεν δυο μαθηταϲ λεγω̅} & 1 & \textbf{2} &  \\
&  & 2 & \foreignlanguage{greek}{αυτοιϲ πορευθηται ειϲ την κωμην} & 6 &  &  \\
&  & 7 & \foreignlanguage{greek}{την απεναντι υμων και ευθεωϲ ευ} & 12 &  &  \\
&  & 12 & \foreignlanguage{greek}{ρηϲεται ονον δεδεμενην και πωλον} & 16 &  &  \\
&  & 17 & \foreignlanguage{greek}{μετ αυτηϲ λυϲαντεϲ αγαγεται μοι} & 21 &  &  \\
& \textbf{3} &  & \foreignlanguage{greek}{και εαν τιϲ υμιν ειπη τι ερειται οτι ο} & 9 &  &  \\
&  & 10 & \foreignlanguage{greek}{\textoverline{κϲ} αυτων χρειαν εχει ευθεωϲ δε απο} & 16 &  &  \\
&  & 16 & \foreignlanguage{greek}{ϲτελλει αυτουϲ} & 17 &  &  \\
& \textbf{4} &  & \foreignlanguage{greek}{τουτο δε ολον γεγονεν ινα πληρωθη} & 6 &  &  \\
&  & 7 & \foreignlanguage{greek}{το ρηθεν δια του προφητου λεγοντοϲ} & 12 &  &  \\
& \textbf{5} &  & \foreignlanguage{greek}{ειπατε τη θυγατρι ϲιων ιδου α βα} & 7 &  &  \\
&  & 7 & \foreignlanguage{greek}{ϲιλευϲ ϲου ερχεται ϲοι πραυϲ και ε} & 13 &  &  \\
&  & 13 & \foreignlanguage{greek}{πιβεβηκωϲ επι ονον και πωλον} & 17 &  &  \\
&  & 18 & \foreignlanguage{greek}{υιον υποζυγιου} & 19 &  &  \\
& \textbf{6} &  & \foreignlanguage{greek}{πορευθεντεϲ δε οι μαθηται και ποι} & 6 &  &  \\
&  & 6 & \foreignlanguage{greek}{ηϲαντεϲ καθωϲ προϲε} & 8 &  &  \\
&  & 8 & \foreignlanguage{greek}{ταξεν αυτοιϲ ο \textoverline{ιϲ} ηγαγον} & 1 & \textbf{7} &  \\
[0.2em]
\cline{4-4}
\end{tabular}
\end{center}
\end{table}
}
\clearpage
\newpage
 {
 \setlength\arrayrulewidth{1pt}
\begin{table}
\begin{center}
\begin{tabular}{ccc|l|ccc}
\cline{4-4} \\ [-1em]
\multicolumn{7}{c}{\foreignlanguage{greek}{ευαγγελιον κατα μαθθαιον} \textbf{(\nospace{21:7})} } \\ \\ [-1em] % Si on veut ajouter les bordures latérales, remplacer {7}{c} par {7}{|c|}
\cline{4-4} \\
\cline{4-4}
&  &  & &  &  & \\ [-0.9em]
&  & 2 & \foreignlanguage{greek}{την ονον και τον πωλον και επεθηκα̅} & 8 &  &  \\
&  & 9 & \foreignlanguage{greek}{επανω αυτων τα ιματια αυτων και ε} & 15 &  &  \\
&  & 15 & \foreignlanguage{greek}{καθειϲεν επανω αυτων} & 17 &  &  \\
& \textbf{8} &  & \foreignlanguage{greek}{ο δε πλειϲτοϲ οχλοϲ εϲτρωϲαν αυτω̅} & 6 &  &  \\
&  & 7 & \foreignlanguage{greek}{τα ιματια εν τη οδω αλλοι δε εκο} & 14 &  &  \\
&  & 14 & \foreignlanguage{greek}{πτον κλαδουϲ και εϲτρωννυον ε̅} & 18 &  &  \\
&  & 19 & \foreignlanguage{greek}{τη οδω οι δε οχλοι οι προαγοντεϲ ϗ} & 6 & \textbf{9} &  \\
&  & 7 & \foreignlanguage{greek}{ακολουθουντεϲ εκραξαν λεγοντεϲ} & 9 &  &  \\
&  & 10 & \foreignlanguage{greek}{ωϲαννα τω υιω δαυειδ ευλογημενοϲ} & 14 &  &  \\
&  & 15 & \foreignlanguage{greek}{ο ερχομενοϲ εν ονοματι \textoverline{κυ} ωϲαννα} & 20 &  &  \\
&  & 21 & \foreignlanguage{greek}{εν τοιϲ υψιϲτοιϲ} & 23 &  &  \\
& \textbf{10} &  & \foreignlanguage{greek}{και ειϲελθοντοϲ αυτου ειϲ ιεροϲολυμα} & 5 &  &  \\
&  & 6 & \foreignlanguage{greek}{εϲειϲθη παϲα η πολειϲ λεγουϲα τιϲ ε} & 12 &  &  \\
&  & 12 & \foreignlanguage{greek}{ϲτιν ουτοϲ οι δε οχλοι ελεγον} & 4 & \textbf{11} &  \\
&  & 5 & \foreignlanguage{greek}{ουτοϲ εϲτιν \textoverline{ιϲ} ο προφητηϲ ο απο να} & 12 &  &  \\
&  & 12 & \foreignlanguage{greek}{ζαρετ τηϲ γαλιλαιαϲ} & 14 &  &  \\
& \textbf{12} &  & \foreignlanguage{greek}{και ειϲηλθεν \textoverline{ιϲ} ειϲ το ιερον του \textoverline{θυ} και} & 9 &  &  \\
&  & 10 & \foreignlanguage{greek}{εξεβαλεν πανταϲ τουϲ πωλουνταϲ ϗ} & 14 &  &  \\
&  & 15 & \foreignlanguage{greek}{αγοραζονταϲ εν τω ιερω και ταϲ τρα} & 21 &  &  \\
&  & 21 & \foreignlanguage{greek}{πεζαϲ των κολλυβιϲτων κατεϲτρε} & 24 &  &  \\
&  & 24 & \foreignlanguage{greek}{ψεν και ταϲ καθεδραϲ των πωλουντω̅} & 29 &  &  \\
&  & 30 & \foreignlanguage{greek}{ταϲ περιϲτεραϲ και λεγει αυτοιϲ} & 3 & \textbf{13} &  \\
&  & 4 & \foreignlanguage{greek}{γεγραπται ο οικοϲ μου οικοϲ προϲευ} & 9 &  &  \\
&  & 9 & \foreignlanguage{greek}{χηϲ κληθηϲεται υμειϲ δε αυτον εποι} & 14 &  &  \\
&  & 14 & \foreignlanguage{greek}{ηϲατε ϲπηλεον ληϲτων} & 16 &  &  \\
& \textbf{14} &  & \foreignlanguage{greek}{και προϲηλθον αυτω χωλοι και τυφλοι} & 6 &  &  \\
&  & 7 & \foreignlanguage{greek}{εν τω ιερω και εθεραπευϲεν αυτουϲ} & 12 &  &  \\
& \textbf{15} &  & \foreignlanguage{greek}{ειδοντεϲ δε οι αρχιερειϲ και οι γραμμα} & 7 &  &  \\
&  & 7 & \foreignlanguage{greek}{τειϲ τα θαυμαϲια α εποιηϲεν} & 11 &  &  \\
&  & 12 & \foreignlanguage{greek}{και τουϲ παιδαϲ κραζονταϲ ε̅} & 16 &  &  \\
[0.2em]
\cline{4-4}
\end{tabular}
\end{center}
\end{table}
}
\clearpage
\newpage
 {
 \setlength\arrayrulewidth{1pt}
\begin{table}
\begin{center}
\begin{tabular}{ccc|l|ccc}
\cline{4-4} \\ [-1em]
\multicolumn{7}{c}{\foreignlanguage{greek}{ευαγγελιον κατα μαθθαιον} \textbf{(\nospace{21:15})} } \\ \\ [-1em] % Si on veut ajouter les bordures latérales, remplacer {7}{c} par {7}{|c|}
\cline{4-4} \\
\cline{4-4}
&  &  & &  &  & \\ [-0.9em]
&  & 17 & \foreignlanguage{greek}{τω ιερω και λεγονταϲ ωϲαννα τω υιω} & 23 &  &  \\
&  & 24 & \foreignlanguage{greek}{δαυειδ ηγανακτηϲαν και ειπον αυ} & 3 & \textbf{16} &  \\
&  & 3 & \foreignlanguage{greek}{τω ακουειϲ τι ουτοι λεγουϲιν} & 7 &  &  \\
&  & 8 & \foreignlanguage{greek}{ο δε \textoverline{ιϲ} λεγει αυτοιϲ ναι ουδεποτε ανε} & 15 &  &  \\
&  & 15 & \foreignlanguage{greek}{γνωτε οτι εκ ϲτοματοϲ νηπιων και} & 20 &  &  \\
&  & 21 & \foreignlanguage{greek}{θηλαζοντων κατηρτιϲω αινον} & 23 &  &  \\
& \textbf{17} &  & \foreignlanguage{greek}{και καταλιπων αυτουϲ εξηλθεν εξω} & 5 &  &  \\
&  & 6 & \foreignlanguage{greek}{τηϲ πολεωϲ ειϲ βηθανιαν και ηυλιϲθη} & 11 &  &  \\
&  & 12 & \foreignlanguage{greek}{εκει πρωιαϲ δε υπαγων ειϲ την πολιν} & 6 & \textbf{18} &  \\
&  & 7 & \foreignlanguage{greek}{επειναϲεν και ιδων ϲυκην μιαν επι τηϲ} & 6 & \textbf{19} &  \\
&  & 7 & \foreignlanguage{greek}{οδου ηλθεν επ αυτηϲ και ουδεν ευρεν} & 13 &  &  \\
&  & 14 & \foreignlanguage{greek}{εν επ αυτη ει μη φυλλα μονον} & 20 &  &  \\
&  & 21 & \foreignlanguage{greek}{και λεγει αυτη μηκετι εκ ϲου καρποϲ γε} & 28 &  &  \\
&  & 28 & \foreignlanguage{greek}{νηται ειϲ τον αιωνα και εξηρανθη πα} & 34 &  &  \\
&  & 34 & \foreignlanguage{greek}{ραχρημα η ϲυκη και ιδοντεϲ οι μα} & 4 & \textbf{20} &  \\
&  & 4 & \foreignlanguage{greek}{θηται εθαυμαϲαν λεγοντεϲ πωϲ παρα} & 8 &  &  \\
&  & 8 & \foreignlanguage{greek}{χρημα εξηρανθη η ϲυκη} & 11 &  &  \\
& \textbf{21} &  & \foreignlanguage{greek}{αποκριθειϲ δε ο \textoverline{ιϲ} ειπεν αυτοιϲ αμην λε} & 8 &  &  \\
&  & 8 & \foreignlanguage{greek}{γω υμιν εαν εχηται πιϲτιν και μη δια} & 15 &  &  \\
&  & 15 & \foreignlanguage{greek}{κριθηται ου μονον το τηϲ ϲυκηϲ ποιη} & 21 &  &  \\
&  & 21 & \foreignlanguage{greek}{ϲεται αλλα και τω ορι τουτω ειπηται} & 27 &  &  \\
&  & 28 & \foreignlanguage{greek}{αρθητι και βληθητι ειϲ την θαλαϲϲαν} & 33 &  &  \\
&  & 34 & \foreignlanguage{greek}{γενηϲεται και παντα οϲα εαν αιτη} & 5 & \textbf{22} &  \\
&  & 5 & \foreignlanguage{greek}{ϲηται εν τη προϲευχη πιϲτευοντεϲ} & 9 &  &  \\
&  & 10 & \foreignlanguage{greek}{λημψεϲθαι} & 10 &  &  \\
& \textbf{23} &  & \foreignlanguage{greek}{και ελθοντι αυτω ειϲ το ιερον προϲηλθε̅} & 7 &  &  \\
&  & 8 & \foreignlanguage{greek}{αυτω διδαϲκοντι οι αρχιερειϲ και οι πρε} & 14 &  &  \\
&  & 14 & \foreignlanguage{greek}{ϲβυτεροι του λαου λεγοντεϲ εν ποια} & 19 &  &  \\
&  & 20 & \foreignlanguage{greek}{εξουϲια ταυτα ποιειϲ και τιϲ ϲοι εδωκε̅} & 26 &  &  \\
&  & 27 & \foreignlanguage{greek}{την εξουϲιαν ταυτην} & 29 &  &  \\
[0.2em]
\cline{4-4}
\end{tabular}
\end{center}
\end{table}
}
\clearpage
\newpage
 {
 \setlength\arrayrulewidth{1pt}
\begin{table}
\begin{center}
\begin{tabular}{ccc|l|ccc}
\cline{4-4} \\ [-1em]
\multicolumn{7}{c}{\foreignlanguage{greek}{ευαγγελιον κατα μαθθαιον} \textbf{(\nospace{21:24})} } \\ \\ [-1em] % Si on veut ajouter les bordures latérales, remplacer {7}{c} par {7}{|c|}
\cline{4-4} \\
\cline{4-4}
&  &  & &  &  & \\ [-0.9em]
& \textbf{24} &  & \foreignlanguage{greek}{αποκριθειϲ δε ο \textoverline{ιϲ} ειπεν αυτοιϲ ερωτη} & 7 &  &  \\
&  & 7 & \foreignlanguage{greek}{ϲω υμαϲ καγω λογον ενα ον εαν ειπη} & 14 &  &  \\
&  & 14 & \foreignlanguage{greek}{ται μοι καγω υμιν ερω εν ποια εξουϲια} & 21 &  &  \\
&  & 22 & \foreignlanguage{greek}{ταυτα ποιω το βαπτιϲμα ιωαννου} & 3 & \textbf{25} &  \\
&  & 4 & \foreignlanguage{greek}{ποθεν ην εξ ουρανου η εξ \textoverline{ανων}} & 10 &  &  \\
&  & 11 & \foreignlanguage{greek}{οι δε διελογιζοντο παρ εαυτοιϲ λεγοντεϲ} & 16 &  &  \\
&  & 17 & \foreignlanguage{greek}{εαν ειπωμεν εξ ουρανου ερι ημιν δια} & 23 &  &  \\
&  & 24 & \foreignlanguage{greek}{τι ουν ουκ επιϲτευϲαται αυτω εαν δε} & 2 & \textbf{26} &  \\
&  & 3 & \foreignlanguage{greek}{ειπωμεν εξ ανθρωπου φοβουμεθα το̅} & 7 &  &  \\
&  & 8 & \foreignlanguage{greek}{οχλον παντεϲ γαρ εχουϲιν τον ιωαννη̅} & 13 &  &  \\
&  & 14 & \foreignlanguage{greek}{ωϲ προφητην και αποκριθεν} & 2 & \textbf{27} &  \\
&  & 2 & \foreignlanguage{greek}{τεϲ τω \textoverline{ιυ} ειπον ουκ οιδαμεν} & 7 &  &  \\
&  & 8 & \foreignlanguage{greek}{εφη αυτοιϲ και αυτοϲ ουδε εγω υμιν} & 14 &  &  \\
&  & 15 & \foreignlanguage{greek}{λεγω εν ποια εξουϲια ταυτα ποιω} & 20 &  &  \\
& \textbf{28} &  & \foreignlanguage{greek}{τι δε υμιν δοκει \textoverline{ανοϲ} ειχεν τεκνα} & 7 &  &  \\
&  & 8 & \foreignlanguage{greek}{δυο και προϲελθων τω πρωτω ειπεν} & 13 &  &  \\
&  & 14 & \foreignlanguage{greek}{τεκνον υπαγε ϲημερον εργαζου εν τω} & 19 &  &  \\
&  & 20 & \foreignlanguage{greek}{αμπελωνι μου ο δε αποκριθειϲ ειπεν} & 4 & \textbf{29} &  \\
&  & 5 & \foreignlanguage{greek}{ου θελω υϲτερον δε μεταμεληθειϲ α} & 10 &  &  \\
&  & 10 & \foreignlanguage{greek}{πηλθεν και προϲελθων τω ετερω} & 4 & \textbf{30} &  \\
&  & 5 & \foreignlanguage{greek}{ειπεν ωϲαυτωϲ ο δε απεκριθειϲ ειπεν} & 11 &  &  \\
&  & 12 & \foreignlanguage{greek}{εγω \textoverline{κε} και ουκ απηλθεν τιϲ εκ των} & 3 & \textbf{31} &  \\
&  & 4 & \foreignlanguage{greek}{δυο εποιηϲεν το θελημα του \textoverline{πρϲ} λεγ} & 10 &  &  \\
&  & 10 & \foreignlanguage{greek}{ουϲιν αυτω ο πρωτοϲ} & 13 &  &  \\
&  & 14 & \foreignlanguage{greek}{λεγει αυτοιϲ ο \textoverline{ιϲ} αμην λεγω υμιν οτι οι τε} & 23 &  &  \\
&  & 23 & \foreignlanguage{greek}{λωναι και αι πορναι προαγουϲιν υμαϲ} & 28 &  &  \\
&  & 29 & \foreignlanguage{greek}{ειϲ την βαϲιλειαν του \textoverline{θυ} ηλθεν γαρ} & 2 & \textbf{32} &  \\
&  & 3 & \foreignlanguage{greek}{προϲ υμαϲ ιωαννηϲ εν οδω δικαιοϲυ} & 8 &  &  \\
&  & 8 & \foreignlanguage{greek}{νηϲ και ουκ επιϲτευϲατε αυτω οι δε τε} & 15 &  &  \\
&  & 15 & \foreignlanguage{greek}{λωναι και αι πορναι επιϲτευϲαν αυτω} & 20 &  &  \\
[0.2em]
\cline{4-4}
\end{tabular}
\end{center}
\end{table}
}
\clearpage
\newpage
 {
 \setlength\arrayrulewidth{1pt}
\begin{table}
\begin{center}
\begin{tabular}{ccc|l|ccc}
\cline{4-4} \\ [-1em]
\multicolumn{7}{c}{\foreignlanguage{greek}{ευαγγελιον κατα μαθθαιον} \textbf{(\nospace{21:32})} } \\ \\ [-1em] % Si on veut ajouter les bordures latérales, remplacer {7}{c} par {7}{|c|}
\cline{4-4} \\
\cline{4-4}
&  &  & &  &  & \\ [-0.9em]
&  & 21 & \foreignlanguage{greek}{υμειϲ δε ιδοντεϲ ου μετεμεληθηται} & 25 &  &  \\
&  & 26 & \foreignlanguage{greek}{υϲτερον τω πιϲτευϲαι αυτω} & 29 &  &  \\
& \textbf{33} &  & \foreignlanguage{greek}{αλλην παραβολην ακουϲατε \textoverline{ανοϲ} η̅} & 5 &  &  \\
&  & 6 & \foreignlanguage{greek}{οικοδεϲποτηϲ οϲτιϲ εφυτευϲεν αμπε} & 9 &  &  \\
&  & 9 & \foreignlanguage{greek}{λωνα και φραγμον αυτω περιεθηκεν} & 13 &  &  \\
&  & 14 & \foreignlanguage{greek}{και ωρυξεν εν αυτω ληνον και ωκο} & 20 &  &  \\
&  & 20 & \foreignlanguage{greek}{δομηϲεν πυργον και εξεδοτο αυτον} & 24 &  &  \\
&  & 25 & \foreignlanguage{greek}{γεωργοιϲ και απεδημηϲεν} & 27 &  &  \\
& \textbf{34} &  & \foreignlanguage{greek}{οτε δε ηγγειϲεν ο καιροϲ των καρπω̅} & 7 &  &  \\
&  & 8 & \foreignlanguage{greek}{απεϲτιλεν τουϲ δουλουϲ αυτου προϲ} & 12 &  &  \\
&  & 13 & \foreignlanguage{greek}{τουϲ γεωργουϲ λαβειν τουϲ καρπουϲ} & 17 &  &  \\
&  & 18 & \foreignlanguage{greek}{αυτου και λαβοντεϲ οι γεωργοι} & 4 & \textbf{35} &  \\
&  & 5 & \foreignlanguage{greek}{τουϲ δουλουϲ αυτου ον μεν εδιρα̅} & 10 &  &  \\
&  & 11 & \foreignlanguage{greek}{ον δε απεκτιναν ον δε ελιθοβολη} & 16 &  &  \\
&  & 16 & \foreignlanguage{greek}{ϲαν παλιν απεϲτιλεν αλλουϲ} & 3 & \textbf{36} &  \\
&  & 4 & \foreignlanguage{greek}{δουλουϲ πλιοναϲ των πρωτων και} & 8 &  &  \\
&  & 9 & \foreignlanguage{greek}{εποιηϲαν αυτοιϲ ωϲαυτωϲ} & 11 &  &  \\
& \textbf{37} &  & \foreignlanguage{greek}{υϲτερον δε απεϲτιλεν προϲ αυτουϲ} & 5 &  &  \\
&  & 6 & \foreignlanguage{greek}{τον υιον αυτου λεγων εντραπηϲο̅} & 10 &  &  \\
&  & 10 & \foreignlanguage{greek}{ται τον υιον μου οι δε γεωργοι} & 3 & \textbf{38} &  \\
&  & 4 & \foreignlanguage{greek}{ιδοντεϲ τον υιον ειπον εν εαυτοιϲ} & 9 &  &  \\
&  & 10 & \foreignlanguage{greek}{ουτοϲ εϲτιν ο κληρονομοϲ δευτε} & 14 &  &  \\
&  & 15 & \foreignlanguage{greek}{αποκτινωμεν αυτον και καταϲχω} & 18 &  &  \\
&  & 18 & \foreignlanguage{greek}{μεν την κληρονομιαν αυτου} & 21 &  &  \\
& \textbf{39} &  & \foreignlanguage{greek}{και λαβοντεϲ αυτον εξεβαλον εξω} & 5 &  &  \\
&  & 6 & \foreignlanguage{greek}{του αμπελωνοϲ και απεκτιναν} & 9 &  &  \\
& \textbf{40} &  & \foreignlanguage{greek}{οταν ουν ελθη ο \textoverline{κϲ} του αμπελωνοϲ} & 7 &  &  \\
&  & 8 & \foreignlanguage{greek}{τι ποιηϲει τοιϲ γεωργοιϲ εκεινοιϲ} & 12 &  &  \\
& \textbf{41} &  & \foreignlanguage{greek}{λεγουϲιν αυτω κακουϲ κακωϲ απολει} & 5 &  &  \\
&  & 6 & \foreignlanguage{greek}{αυτουϲ και τον αμπελωνα εκδωϲεται} & 10 &  &  \\
[0.2em]
\cline{4-4}
\end{tabular}
\end{center}
\end{table}
}
\clearpage
\newpage
 {
 \setlength\arrayrulewidth{1pt}
\begin{table}
\begin{center}
\begin{tabular}{ccc|l|ccc}
\cline{4-4} \\ [-1em]
\multicolumn{7}{c}{\foreignlanguage{greek}{ευαγγελιον κατα μαθθαιον} \textbf{(\nospace{21:41})} } \\ \\ [-1em] % Si on veut ajouter les bordures latérales, remplacer {7}{c} par {7}{|c|}
\cline{4-4} \\
\cline{4-4}
&  &  & &  &  & \\ [-0.9em]
&  & 11 & \foreignlanguage{greek}{αλλοιϲ γεωργοιϲ οιτινεϲ αποδωϲωϲιν} & 14 &  &  \\
&  & 15 & \foreignlanguage{greek}{αυτω τουϲ καρπουϲ εν τοιϲ καιροιϲ αυτω̅} & 21 &  &  \\
& \textbf{42} &  & \foreignlanguage{greek}{λεγει αυτοιϲ ο \textoverline{ιϲ} ουδεποτε ανεγνωται} & 6 &  &  \\
&  & 7 & \foreignlanguage{greek}{εν ταιϲ γραφαιϲ λιθον ον απεδοκι} & 12 &  &  \\
&  & 12 & \foreignlanguage{greek}{μαϲαν οι οικοδομουντεϲ ουτοϲ εγε} & 16 &  &  \\
&  & 16 & \foreignlanguage{greek}{νηθη ειϲ κεφαλην γωνιαϲ παρα \textoverline{κυ}} & 21 &  &  \\
&  & 22 & \foreignlanguage{greek}{εγενετο αυτη και εϲτιν θαυμαϲτη} & 26 &  &  \\
&  & 27 & \foreignlanguage{greek}{εν οφθαλμοιϲ ημων} & 29 &  &  \\
& \textbf{43} &  & \foreignlanguage{greek}{δια τουτο λεγω υμιν οτι αρθηϲεται} & 6 &  &  \\
&  & 7 & \foreignlanguage{greek}{αφ υμων η βαϲιλεια του \textoverline{θυ} και δοθη} & 14 &  &  \\
&  & 14 & \foreignlanguage{greek}{ϲεται εθνι ποιουντι τουϲ καρπουϲ} & 18 &  &  \\
&  & 19 & \foreignlanguage{greek}{αυτηϲ και ο πεϲων επι τον λιθο̅} & 6 & \textbf{44} &  \\
&  & 7 & \foreignlanguage{greek}{τουτον ϲυνθλαϲθηϲεται εφ ον δ αν} & 12 &  &  \\
&  & 13 & \foreignlanguage{greek}{πεϲη λικμηϲει αυτον} & 15 &  &  \\
& \textbf{45} &  & \foreignlanguage{greek}{και ακουϲαντεϲ οι αρχιερειϲ και οι φα} & 7 &  &  \\
&  & 7 & \foreignlanguage{greek}{ριϲαιοι ταϲ παραβολαϲ αυτου εγνωϲα̅} & 11 &  &  \\
&  & 12 & \foreignlanguage{greek}{οτι περι αυτων λεγει και ζητουν} & 2 & \textbf{46} &  \\
&  & 2 & \foreignlanguage{greek}{τεϲ αυτον κρατηϲαι εφοβηθηϲαν τουϲ} & 6 &  &  \\
&  & 7 & \foreignlanguage{greek}{οχλουϲ επειδη ωϲ προφητην αυτον ειχο̅} & 12 &  &  \\
& \mygospelchapter &  & \foreignlanguage{greek}{και αποκριθειϲ ο \textoverline{ιϲ} ειπεν αυτοιϲ εν πα} & 8 &  &  \\
&  & 8 & \foreignlanguage{greek}{ραβολαιϲ λεγων ομοιωθη η βαϲιλεια} & 3 & \textbf{2} &  \\
&  & 4 & \foreignlanguage{greek}{των ουρανων \textoverline{ανω} βαϲιλει οϲτιϲ ε} & 9 &  &  \\
&  & 9 & \foreignlanguage{greek}{ποιηϲεν γαμουϲ τω υιω αυτου και απε} & 2 & \textbf{3} &  \\
&  & 2 & \foreignlanguage{greek}{ϲτιλεν τουϲ δουλουϲ αυτου καλεϲαι τουϲ} & 7 &  &  \\
&  & 8 & \foreignlanguage{greek}{κεκλημενουϲ ειϲ τουϲ γαμουϲ και ου} & 13 &  &  \\
&  & 13 & \foreignlanguage{greek}{κ ηθελον ελθειν παλιν απεϲτιλεν} & 2 & \textbf{4} &  \\
&  & 3 & \foreignlanguage{greek}{αλλουϲ δουλουϲ λεγων ειπατε τοιϲ} & 7 &  &  \\
&  & 8 & \foreignlanguage{greek}{κεκλημενοιϲ ιδου το αριϲτον μου} & 12 &  &  \\
&  & 13 & \foreignlanguage{greek}{ητοιμαϲα οι ταυροι μου και τα ϲιτι} & 19 &  &  \\
&  & 19 & \foreignlanguage{greek}{ϲτα μου τεθυμενα και παντα ετοιμα} & 24 &  &  \\
[0.2em]
\cline{4-4}
\end{tabular}
\end{center}
\end{table}
}
\clearpage
\newpage
 {
 \setlength\arrayrulewidth{1pt}
\begin{table}
\begin{center}
\begin{tabular}{ccc|l|ccc}
\cline{4-4} \\ [-1em]
\multicolumn{7}{c}{\foreignlanguage{greek}{ευαγγελιον κατα μαθθαιον} \textbf{(\nospace{22:4})} } \\ \\ [-1em] % Si on veut ajouter les bordures latérales, remplacer {7}{c} par {7}{|c|}
\cline{4-4} \\
\cline{4-4}
&  &  & &  &  & \\ [-0.9em]
&  & 25 & \foreignlanguage{greek}{δευτε ειϲ τουϲ γαμουϲ οι δε αμεληϲα̅} & 3 & \textbf{5} &  \\
&  & 3 & \foreignlanguage{greek}{τεϲ απηλθον οϲ μεν ειϲ τον ιδιον αγρον} & 10 &  &  \\
&  & 11 & \foreignlanguage{greek}{οϲ δε ειϲ την εμποριαν αυτου οι δε λοι} & 3 & \textbf{6} &  \\
&  & 3 & \foreignlanguage{greek}{ποι κρατηϲαντεϲ τουϲ δουλουϲ υβριϲαν} & 7 &  &  \\
&  & 8 & \foreignlanguage{greek}{και απεκτιναν και ακουϲαϲ ο βαϲιλευϲ} & 4 & \textbf{7} &  \\
&  & 5 & \foreignlanguage{greek}{εκεινοϲ ωργιϲθη και πεμψαϲ τα ϲτρατευ} & 11 &  &  \\
&  & 11 & \foreignlanguage{greek}{ματα αυτου απωλεϲεν τουϲ φονειϲ ε} & 16 &  &  \\
&  & 16 & \foreignlanguage{greek}{κεινουϲ και την πολιν αυτων ενεπρηϲε̅} & 21 &  &  \\
& \textbf{8} &  & \foreignlanguage{greek}{τοτε λεγει τοιϲ δουλοιϲ αυτου ο μεν γα} & 8 &  &  \\
&  & 8 & \foreignlanguage{greek}{μοϲ ετοιμοϲ εϲτιν οι δε κεκλημενοι} & 13 &  &  \\
&  & 14 & \foreignlanguage{greek}{ουκ ηϲαν αξιοι πορευεϲθαι ουν ε} & 3 & \textbf{9} &  \\
&  & 3 & \foreignlanguage{greek}{πι ταϲ διεξοδουϲ των οδων και οϲουϲ} & 9 &  &  \\
&  & 10 & \foreignlanguage{greek}{αν ευρηται καλεϲατε ειϲ τουϲ γαμουϲ} & 15 &  &  \\
& \textbf{10} &  & \foreignlanguage{greek}{και εξελθοντεϲ οι δουλοι εκεινοι ειϲ ταϲ} & 7 &  &  \\
&  & 8 & \foreignlanguage{greek}{οδουϲ ϲυνηγαγον πανταϲ οϲουϲ ευρον} & 12 &  &  \\
&  & 13 & \foreignlanguage{greek}{πονηρουϲ τε και αγαθουϲ και επληϲθη} & 18 &  &  \\
&  & 19 & \foreignlanguage{greek}{ο γαμοϲ ανακειμενων} & 21 &  &  \\
& \textbf{11} &  & \foreignlanguage{greek}{ειϲελθων δε ο βαϲιλευϲ θεαϲαϲθαι τουϲ} & 6 &  &  \\
&  & 7 & \foreignlanguage{greek}{ανακειμενουϲ ειδεν εκει \textoverline{ανον} ουκ ε̅} & 12 &  &  \\
&  & 12 & \foreignlanguage{greek}{δεδυμενον ενδυμα γαμου και λεγει} & 2 & \textbf{12} &  \\
&  & 3 & \foreignlanguage{greek}{αυτω ετερε πωϲ ειϲηλθεϲ ωδε μη ε} & 9 &  &  \\
&  & 9 & \foreignlanguage{greek}{χων ενδυμα γαμου ο δε εφιμωθη} & 14 &  &  \\
& \textbf{13} &  & \foreignlanguage{greek}{τοτε ειπεν ο βαϲιλευϲ τοιϲ διακονοιϲ} & 6 &  &  \\
&  & 7 & \foreignlanguage{greek}{δηϲαντεϲ αυτου ποδαϲ και χειραϲ α} & 12 &  &  \\
&  & 12 & \foreignlanguage{greek}{ρατε αυτον και εκβαλεται ειϲ το ϲκοτοϲ} & 18 &  &  \\
&  & 19 & \foreignlanguage{greek}{το εξωτερον εκει εϲται ο κλαυθμοϲ} & 24 &  &  \\
&  & 26 & \foreignlanguage{greek}{και ο βρυγμοϲ των οδοντων πολλοι} & 1 & \textbf{14} &  \\
&  & 2 & \foreignlanguage{greek}{γαρ ειϲιν κλητοι ολειγοι δε εκλεκτοι} & 7 &  &  \\
& \textbf{15} &  & \foreignlanguage{greek}{τοτε πορευθεντεϲ οι φαριϲαιοι ϲυμβου} & 5 &  &  \\
&  & 5 & \foreignlanguage{greek}{λιον ελαβον οπωϲ αυτον παγιδευϲωϲι̅} & 9 &  &  \\
[0.2em]
\cline{4-4}
\end{tabular}
\end{center}
\end{table}
}
\clearpage
\newpage
 {
 \setlength\arrayrulewidth{1pt}
\begin{table}
\begin{center}
\begin{tabular}{ccc|l|ccc}
\cline{4-4} \\ [-1em]
\multicolumn{7}{c}{\foreignlanguage{greek}{ευαγγελιον κατα μαθθαιον} \textbf{(\nospace{22:15})} } \\ \\ [-1em] % Si on veut ajouter les bordures latérales, remplacer {7}{c} par {7}{|c|}
\cline{4-4} \\
\cline{4-4}
&  &  & &  &  & \\ [-0.9em]
&  & 10 & \foreignlanguage{greek}{εν λογω και αποϲτελλουϲιν αυτω τουϲ} & 4 & \textbf{16} &  \\
&  & 5 & \foreignlanguage{greek}{μαθηταϲ αυτων μετα των ηρωδιανω̅} & 9 &  &  \\
&  & 10 & \foreignlanguage{greek}{λεγοντεϲ διδαϲκαλε οιδαμεν οτι α} & 14 &  &  \\
&  & 14 & \foreignlanguage{greek}{ληθηϲ ει και την οδον του \textoverline{θυ} εν αληθει} & 22 &  &  \\
&  & 22 & \foreignlanguage{greek}{α διδαϲκειϲ και ου μελει ϲοι περι ουδε} & 29 &  &  \\
&  & 29 & \foreignlanguage{greek}{νοϲ ου γαρ βλεπειϲ ειϲ προϲωπον αν} & 35 &  &  \\
&  & 35 & \foreignlanguage{greek}{θρωπων ειπε ουν ημιν τι ϲοι δοκει εξε} & 7 & \textbf{17} &  \\
&  & 7 & \foreignlanguage{greek}{ϲτι κηνϲον δουναι καιϲαρι η ου} & 12 &  &  \\
& \textbf{18} &  & \foreignlanguage{greek}{γνουϲ δε ο \textoverline{ιϲ} ταϲ πονηριαϲ αυτων ειπε̅} & 8 &  &  \\
&  & 9 & \foreignlanguage{greek}{τι με πειραζεται υποκριται επιδιξατε} & 1 & \textbf{19} &  \\
&  & 2 & \foreignlanguage{greek}{μοι το νομιϲμα του κηνϲου οι δε προϲ} & 9 &  &  \\
&  & 9 & \foreignlanguage{greek}{ηνεγκαν αυτω δηναριον} & 11 &  &  \\
& \textbf{20} &  & \foreignlanguage{greek}{και λεγει αυτοιϲ τινοϲ η ικων αυτη και} & 8 &  &  \\
&  & 9 & \foreignlanguage{greek}{η επιγραφη λεγουϲιν αυτω καιϲαροϲ} & 3 & \textbf{21} &  \\
&  & 4 & \foreignlanguage{greek}{τοτε λεγει αυτοιϲ αποδοτε ουν τα κεϲα} & 10 &  &  \\
&  & 10 & \foreignlanguage{greek}{ροϲ κεϲαρι και τα του \textoverline{θυ} τω \textoverline{θω}} & 17 &  &  \\
& \textbf{22} &  & \foreignlanguage{greek}{και ακουϲαντεϲ εθαυμαϲαν και αφεν} & 5 &  &  \\
&  & 5 & \foreignlanguage{greek}{τεϲ αυτον απηλθον} & 7 &  &  \\
& \textbf{23} &  & \foreignlanguage{greek}{εν εκεινη τη ημερα προϲηλθον αυτω} & 6 &  &  \\
&  & 7 & \foreignlanguage{greek}{ϲαδδουκαιοι οι λεγοντεϲ μη ειναι ανα} & 12 &  &  \\
&  & 12 & \foreignlanguage{greek}{ϲταϲιν και επηρωτηϲαν αυτον λεγο̅} & 1 & \textbf{24} &  \\
&  & 1 & \foreignlanguage{greek}{τεϲ διδαϲκαλε μωυϲηϲ ειπεν} & 4 &  &  \\
&  & 5 & \foreignlanguage{greek}{εαν τιϲ αποθανη μη εχων τεκνα επι} & 11 &  &  \\
&  & 11 & \foreignlanguage{greek}{γαμβρευϲη ο αδελφοϲ αυτου την γυ} & 16 &  &  \\
&  & 16 & \foreignlanguage{greek}{ναικα αυτου και αναϲτηϲει ϲπερμα} & 20 &  &  \\
&  & 21 & \foreignlanguage{greek}{τω αδελφω αυτου} & 23 &  &  \\
& \textbf{25} &  & \foreignlanguage{greek}{ηϲαν δε παρ ημιν επτα αδελφοι και} & 7 &  &  \\
&  & 8 & \foreignlanguage{greek}{ο πρωτοϲ γαμηϲαϲ ετελευτηϲεν και} & 12 &  &  \\
&  & 13 & \foreignlanguage{greek}{μη εχων ϲπερμα αφηκεν την γυναι} & 18 &  &  \\
&  & 18 & \foreignlanguage{greek}{κα αυτου τω αδελφω αυτου ομοιωϲ} & 1 & \textbf{26} &  \\
[0.2em]
\cline{4-4}
\end{tabular}
\end{center}
\end{table}
}
\clearpage
\newpage
 {
 \setlength\arrayrulewidth{1pt}
\begin{table}
\begin{center}
\begin{tabular}{ccc|l|ccc}
\cline{4-4} \\ [-1em]
\multicolumn{7}{c}{\foreignlanguage{greek}{ευαγγελιον κατα μαθθαιον} \textbf{(\nospace{22:26})} } \\ \\ [-1em] % Si on veut ajouter les bordures latérales, remplacer {7}{c} par {7}{|c|}
\cline{4-4} \\
\cline{4-4}
&  &  & &  &  & \\ [-0.9em]
&  & 2 & \foreignlanguage{greek}{και ο δευτεροϲ και ο τριτοϲ εωϲ των επτα} & 10 &  &  \\
& \textbf{27} &  & \foreignlanguage{greek}{υϲτερον δε παντων απεθανεν η γυνη} & 6 &  &  \\
& \textbf{28} &  & \foreignlanguage{greek}{εν τη ουν αναϲταϲι τινοϲ των επτα εϲται} & 8 &  &  \\
&  & 9 & \foreignlanguage{greek}{γυνη παντεϲ γαρ εϲχον αυτην} & 13 &  &  \\
& \textbf{29} &  & \foreignlanguage{greek}{αποκριθειϲ δε ο \textoverline{ιϲ} ειπεν αυτοιϲ πλανα} & 7 &  &  \\
&  & 7 & \foreignlanguage{greek}{ϲθαι μη ειδοτεϲ ταϲ γραφαϲ μηδε την δυ} & 14 &  &  \\
&  & 14 & \foreignlanguage{greek}{ναμιν του \textoverline{θυ} εν γαρ τη αναϲταϲει ουτε} & 5 & \textbf{30} &  \\
&  & 6 & \foreignlanguage{greek}{γαμουϲιν ουτε γαμιϲκονται αλλ ωϲ αγ} & 11 &  &  \\
&  & 11 & \foreignlanguage{greek}{γελοι του \textoverline{θυ} εν ουρανω ειϲιν} & 16 &  &  \\
& \textbf{31} &  & \foreignlanguage{greek}{περι δε τηϲ αναϲταϲεωϲ των νεκρων} & 6 &  &  \\
&  & 7 & \foreignlanguage{greek}{ουκ ανεγνωτε το ρηθεν υμιν υπο του} & 13 &  &  \\
&  & 14 & \foreignlanguage{greek}{\textoverline{θυ} λεγοντοϲ εγω ειμει ο \textoverline{θϲ} αβρααμ} & 5 & \textbf{32} &  \\
&  & 6 & \foreignlanguage{greek}{και ο \textoverline{θϲ} ιϲαακ και ο \textoverline{θϲ} ιακωβ ουκ εϲτι̅} & 15 &  &  \\
&  & 16 & \foreignlanguage{greek}{\textoverline{θϲ} νεκρων αλλα ζωντων και ακου} & 2 & \textbf{33} &  \\
&  & 2 & \foreignlanguage{greek}{ϲαντεϲ οι οχλοι εξεπληϲϲοντο επι τη} & 7 &  &  \\
&  & 8 & \foreignlanguage{greek}{διδαχη αυτου} & 9 &  &  \\
& \textbf{34} &  & \foreignlanguage{greek}{οι δε φαριϲαιοι ακουϲαντεϲ οτι εφιμω} & 6 &  &  \\
&  & 6 & \foreignlanguage{greek}{ϲεν τουϲ ϲαδδουκαιουϲ ϲυνηχθηϲαν} & 9 &  &  \\
&  & 10 & \foreignlanguage{greek}{επι το αυτο και επηρωτηϲεν ειϲ εξ} & 4 & \textbf{35} &  \\
&  & 5 & \foreignlanguage{greek}{αυτων νομικοϲ πειραζων αυτον και} & 9 &  &  \\
&  & 10 & \foreignlanguage{greek}{λεγων διδαϲκαλε ποια εντολη μεγαλη} & 4 & \textbf{36} &  \\
&  & 5 & \foreignlanguage{greek}{εν τω νομω ο δε \textoverline{ιϲ} ειπεν αυτω} & 5 & \textbf{37} &  \\
&  & 6 & \foreignlanguage{greek}{αγαπηϲειϲ \textoverline{κν} τον \textoverline{θν} ϲου εν ολη καρδι} & 13 &  &  \\
&  & 13 & \foreignlanguage{greek}{α ϲου και εν ολη ψυχη ϲου και εν ολη τη} & 23 &  &  \\
&  & 24 & \foreignlanguage{greek}{διανοια ϲου αυτη εϲτιν η πρωτη και} & 5 & \textbf{38} &  \\
&  & 6 & \foreignlanguage{greek}{η μεγαλη εντολη} & 8 &  &  \\
& \textbf{39} &  & \foreignlanguage{greek}{δευτερα δε ομοια αυτη αγαπηϲιϲ τον πλη} & 7 &  &  \\
&  & 7 & \foreignlanguage{greek}{ϲιον ϲου ωϲ ϲεαυτον εν ταυταιϲ ταιϲ} & 3 & \textbf{40} &  \\
&  & 4 & \foreignlanguage{greek}{δυϲιν εντολαιϲ ολοϲ ο νομοϲ και οι προ} & 11 &  &  \\
&  & 11 & \foreignlanguage{greek}{φηται κρεμανται} & 12 &  &  \\
[0.2em]
\cline{4-4}
\end{tabular}
\end{center}
\end{table}
}
\clearpage
\newpage
 {
 \setlength\arrayrulewidth{1pt}
\begin{table}
\begin{center}
\begin{tabular}{ccc|l|ccc}
\cline{4-4} \\ [-1em]
\multicolumn{7}{c}{\foreignlanguage{greek}{ευαγγελιον κατα μαθθαιον} \textbf{(\nospace{22:41})} } \\ \\ [-1em] % Si on veut ajouter les bordures latérales, remplacer {7}{c} par {7}{|c|}
\cline{4-4} \\
\cline{4-4}
&  &  & &  &  & \\ [-0.9em]
& \textbf{41} &  & \foreignlanguage{greek}{ϲυνηγμενων δε των φαριϲαιων επηρωτη} & 5 &  &  \\
&  & 5 & \foreignlanguage{greek}{ϲεν αυτουϲ ο \textoverline{ιϲ} λεγων τι υμιν δοκει περι} & 5 & \textbf{42} &  \\
&  & 6 & \foreignlanguage{greek}{του \textoverline{χυ} τινοϲ υιοϲ εϲτιν λεγουϲιν αυ} & 12 &  &  \\
&  & 12 & \foreignlanguage{greek}{τω του δαυειδ} & 14 &  &  \\
& \textbf{43} &  & \foreignlanguage{greek}{λεγει αυτοιϲ πωϲ ουν δαυειδ εν \textoverline{πνι} \textoverline{κν}} & 8 &  &  \\
&  & 9 & \foreignlanguage{greek}{αυτον καλει λεγων ειπεν ο \textoverline{κϲ} τω \textoverline{κω}} & 5 & \textbf{44} &  \\
&  & 6 & \foreignlanguage{greek}{μου καθου εκ δεξιων μου εωϲ αν θω τουϲ} & 14 &  &  \\
&  & 15 & \foreignlanguage{greek}{εχθρουϲ ϲου υποποδιον των ποδων ϲου} & 20 &  &  \\
& \textbf{45} &  & \foreignlanguage{greek}{ει ουν δαυειδ καλει αυτον \textoverline{κν} πωϲ υι} & 8 &  &  \\
&  & 8 & \foreignlanguage{greek}{οϲ αυτου εϲτιν και ουδειϲ εδυνατο} & 3 & \textbf{46} &  \\
&  & 4 & \foreignlanguage{greek}{αυτω αποκριθηναι λογον ουδε ετολ} & 8 &  &  \\
&  & 8 & \foreignlanguage{greek}{μηϲεν τιϲ απ εκεινηϲ τηϲ ωραϲ επερω} & 14 &  &  \\
&  & 14 & \foreignlanguage{greek}{τηϲαι αυτον ουκετι} & 16 &  &  \\
& \mygospelchapter &  & \foreignlanguage{greek}{τοτε \textoverline{ιϲ} ελαληϲεν τοιϲ οχλοιϲ και τοιϲ} & 7 &  &  \\
&  & 8 & \foreignlanguage{greek}{μαθηταιϲ αυτου λεγων επι τηϲ μω} & 4 & \textbf{2} &  \\
&  & 4 & \foreignlanguage{greek}{υϲεωϲ καθεδραϲ εκαθειϲαν οι γραμμα} & 8 &  &  \\
&  & 8 & \foreignlanguage{greek}{τιϲ και οι φαριϲαιοι παντα ουν οϲα εαν} & 4 & \textbf{3} &  \\
&  & 5 & \foreignlanguage{greek}{ειπωϲιν υμιν τηρειν τηρειται και ποι} & 10 &  &  \\
&  & 10 & \foreignlanguage{greek}{ειται κατα δε τα εργα αυτων μη ποιει} & 17 &  &  \\
&  & 17 & \foreignlanguage{greek}{ται λεγουϲιν γαρ και ου ποιουϲιν} & 22 &  &  \\
& \textbf{4} &  & \foreignlanguage{greek}{δεϲμευουϲιν δε φορτια βαρεα και δυϲ} & 6 &  &  \\
&  & 6 & \foreignlanguage{greek}{βαϲτακτα και επιτιθεαϲιν επι τουϲ} & 10 &  &  \\
&  & 11 & \foreignlanguage{greek}{ωμουϲ των \textoverline{ανων} τω δε δακτυλω} & 16 &  &  \\
&  & 17 & \foreignlanguage{greek}{αυτων ου θελουϲιν κεινηϲαι αυτα} & 21 &  &  \\
& \textbf{5} &  & \foreignlanguage{greek}{παντα δε τα εργα αυτων ποιουϲιν προϲ} & 7 &  &  \\
&  & 8 & \foreignlanguage{greek}{το θεαθηναι τοιϲ ανθρωποιϲ} & 11 &  &  \\
&  & 12 & \foreignlanguage{greek}{πλατυνουϲιν δε τα φυλακτηρια αυ} & 16 &  &  \\
&  & 16 & \foreignlanguage{greek}{των και μεγαλυνουϲιν τα κραϲπεδα} & 20 &  &  \\
&  & 21 & \foreignlanguage{greek}{των ιματιων αυτων φιλουϲιν τε} & 2 & \textbf{6} &  \\
&  & 3 & \foreignlanguage{greek}{την πρωτοκλιϲιαν εν τοιϲ διπνοιϲ} & 7 &  &  \\
[0.2em]
\cline{4-4}
\end{tabular}
\end{center}
\end{table}
}
\clearpage
\newpage
 {
 \setlength\arrayrulewidth{1pt}
\begin{table}
\begin{center}
\begin{tabular}{ccc|l|ccc}
\cline{4-4} \\ [-1em]
\multicolumn{7}{c}{\foreignlanguage{greek}{ευαγγελιον κατα μαθθαιον} \textbf{(\nospace{23:6})} } \\ \\ [-1em] % Si on veut ajouter les bordures latérales, remplacer {7}{c} par {7}{|c|}
\cline{4-4} \\
\cline{4-4}
&  &  & &  &  & \\ [-0.9em]
&  & 8 & \foreignlanguage{greek}{και ταϲ πρωτοκαθεδριαϲ εν ταιϲ ϲυναγω} & 13 &  &  \\
&  & 13 & \foreignlanguage{greek}{γαιϲ και τουϲ αϲπαϲμουϲ εν ταιϲ αγοραιϲ} & 6 & \textbf{7} &  \\
&  & 7 & \foreignlanguage{greek}{και καλειϲθαι υπο των ανθρωπων ραβ} & 12 &  &  \\
&  & 12 & \foreignlanguage{greek}{βει ραββει υμειϲ δε μη κληθηται ραβ} & 5 & \textbf{8} &  \\
&  & 5 & \foreignlanguage{greek}{βει ειϲ γαρ εϲτιν ο καθηγητηϲ υμων} & 11 &  &  \\
&  & 12 & \foreignlanguage{greek}{παντεϲ δε υμειϲ αδελφοι εϲται και \textoverline{πρα}} & 2 & \textbf{9} &  \\
&  & 3 & \foreignlanguage{greek}{μη καλεϲηται υμων επι τηϲ γηϲ ειϲ γαρ} & 10 &  &  \\
&  & 11 & \foreignlanguage{greek}{εϲτιν ο \textoverline{πηρ} υμων ο εν ουρανοιϲ μηδε κλη} & 2 & \textbf{10} &  \\
&  & 2 & \foreignlanguage{greek}{θηται καθηγηται ειϲ γαρ εϲτιν ο καθηγη} & 8 &  &  \\
&  & 8 & \foreignlanguage{greek}{τηϲ ο \textoverline{χϲ} ο δε μιζων υμων εϲται υμων} & 6 & \textbf{11} &  \\
&  & 7 & \foreignlanguage{greek}{διακονοϲ οϲτιϲ δε υψωϲει εαυ} & 4 & \textbf{12} &  \\
&  & 4 & \foreignlanguage{greek}{τον ταπινωθηϲεται και οϲτιϲ ταπινω} & 8 &  &  \\
&  & 8 & \foreignlanguage{greek}{ϲει εαυτον υψωθηϲεται} & 10 &  &  \\
& \textbf{14} &  & \foreignlanguage{greek}{ουαι υμιν γραμματειϲ και φαριϲαιοι} & 5 &  &  \\
&  & 6 & \foreignlanguage{greek}{υποκριται οτι καταιϲθιεται ταϲ οι} & 10 &  &  \\
&  & 10 & \foreignlanguage{greek}{κειαϲ των χηρων και προφαει μακρα} & 15 &  &  \\
&  & 16 & \foreignlanguage{greek}{προϲευχομενοι δια τουτο λημψε} & 19 &  &  \\
&  & 19 & \foreignlanguage{greek}{ϲθαι περιϲϲοτερον κριμα} & 21 &  &  \\
& \textbf{13} &  & \foreignlanguage{greek}{ουαι υμιν γραμματιϲ και φαριϲαιοι} & 5 &  &  \\
&  & 6 & \foreignlanguage{greek}{υποκριται οτι κλιεται την βαϲιλει} & 10 &  &  \\
&  & 10 & \foreignlanguage{greek}{αν των ουρανων εμπροϲθεν των} & 14 &  &  \\
&  & 15 & \foreignlanguage{greek}{\textoverline{ανων} υμειϲ γαρ ουκ ειϲερχεϲθαι ου} & 20 &  &  \\
&  & 20 & \foreignlanguage{greek}{δε τουϲ ειϲερχομενουϲ αφιεται ειϲελθει̅} & 24 &  &  \\
& \textbf{15} &  & \foreignlanguage{greek}{ουαι υμιν γραμματειϲ και φαριϲαιοι} & 5 &  &  \\
&  & 6 & \foreignlanguage{greek}{υποκρειται οτι περιαγεται την θα} & 10 &  &  \\
&  & 10 & \foreignlanguage{greek}{λαϲϲαν και την ξηραν ποιηϲαι ενα} & 15 &  &  \\
&  & 16 & \foreignlanguage{greek}{προϲηλυτον και οταν γενηται ποι} & 20 &  &  \\
&  & 20 & \foreignlanguage{greek}{ειται αυτον υιον γεεννηϲ διπλο} & 24 &  &  \\
&  & 24 & \foreignlanguage{greek}{τερον υμων} & 25 &  &  \\
& \textbf{16} &  & \foreignlanguage{greek}{ουαι υμιν οδηγοι τυφλοι οι λεγοντεϲ} & 6 &  &  \\
[0.2em]
\cline{4-4}
\end{tabular}
\end{center}
\end{table}
}
\clearpage
\newpage
 {
 \setlength\arrayrulewidth{1pt}
\begin{table}
\begin{center}
\begin{tabular}{ccc|l|ccc}
\cline{4-4} \\ [-1em]
\multicolumn{7}{c}{\foreignlanguage{greek}{ευαγγελιον κατα μαθθαιον} \textbf{(\nospace{23:16})} } \\ \\ [-1em] % Si on veut ajouter les bordures latérales, remplacer {7}{c} par {7}{|c|}
\cline{4-4} \\
\cline{4-4}
&  &  & &  &  & \\ [-0.9em]
&  & 7 & \foreignlanguage{greek}{οϲ αν ομοϲη εν τω ναω ουδεν εϲτιν} & 14 &  &  \\
&  & 15 & \foreignlanguage{greek}{οϲ δ αν ομοϲη εν τω χρυϲω του ναου} & 23 &  &  \\
&  & 24 & \foreignlanguage{greek}{οφιλει μωροι και τυφλοι τι γαρ} & 5 & \textbf{17} &  \\
&  & 6 & \foreignlanguage{greek}{μιζων εϲτιν ο χρυϲοϲ η ο ναοϲ ο αγι} & 14 &  &  \\
&  & 14 & \foreignlanguage{greek}{αζων τον χρυϲον και οϲ εαν ομο} & 4 & \textbf{18} &  \\
&  & 4 & \foreignlanguage{greek}{ϲη εν τω θυϲιαϲτηριω ουδεν εϲτιν} & 9 &  &  \\
&  & 10 & \foreignlanguage{greek}{οϲ δ αν ομοϲη εν τω δωρω τω επανω} & 18 &  &  \\
&  & 19 & \foreignlanguage{greek}{αυτου οφιλει μωροι και τυφλοι} & 3 & \textbf{19} &  \\
&  & 4 & \foreignlanguage{greek}{τι γαρ μιζον το δωρον η το θυϲιαϲτη} & 11 &  &  \\
&  & 11 & \foreignlanguage{greek}{ριον το αγιαζον το δωρον} & 15 &  &  \\
& \textbf{20} &  & \foreignlanguage{greek}{ο ουν ομοϲαϲ εν τω θυϲιαϲτηριω ο} & 7 &  &  \\
&  & 7 & \foreignlanguage{greek}{μνυει εν αυτω και εν παϲιν τοιϲ επα} & 14 &  &  \\
&  & 14 & \foreignlanguage{greek}{νω αυτου και ο ομοϲαϲ εν τω ναω} & 6 & \textbf{21} &  \\
&  & 7 & \foreignlanguage{greek}{ομνυει εν αυτω και εν τω κατοικη} & 13 &  &  \\
&  & 13 & \foreignlanguage{greek}{ϲαντι αυτον και ο ομοϲαϲ εν τω} & 5 & \textbf{22} &  \\
&  & 6 & \foreignlanguage{greek}{ουρανω ομνυει εν τω θρονω του \textoverline{θυ}} & 12 &  &  \\
&  & 13 & \foreignlanguage{greek}{και εν τω καθημενω επανω αυτου} & 18 &  &  \\
& \textbf{23} &  & \foreignlanguage{greek}{ουαι υμιν γραμματιϲ και φαριϲαιοι} & 5 &  &  \\
&  & 6 & \foreignlanguage{greek}{υποκριται οτι αποδεκατουται το η} & 10 &  &  \\
&  & 10 & \foreignlanguage{greek}{δυοϲμον και το ανηθον και το κυ} & 16 &  &  \\
&  & 16 & \foreignlanguage{greek}{μινον και αφηκατε τα βαρυτερα} & 20 &  &  \\
&  & 21 & \foreignlanguage{greek}{του νομου την κριϲιν και τον ελεον} & 27 &  &  \\
&  & 28 & \foreignlanguage{greek}{και την πιϲτιν ταυτα δε εδει ποι} & 34 &  &  \\
&  & 34 & \foreignlanguage{greek}{ηϲαι κακεινα μη αφιεναι} & 37 &  &  \\
& \textbf{24} &  & \foreignlanguage{greek}{οδηγοι τυφλοι οι διυλιζοντεϲ τον} & 5 &  &  \\
&  & 6 & \foreignlanguage{greek}{κωνωπα την δε καμηλον καταπι} & 10 &  &  \\
&  & 10 & \foreignlanguage{greek}{νοντεϲ ουαι υμιν γραμματιϲ} & 3 & \textbf{25} &  \\
&  & 4 & \foreignlanguage{greek}{και φαριϲαιοι υποκριται οτι καθα} & 8 &  &  \\
&  & 8 & \foreignlanguage{greek}{ριζεται το εξωθεν του ποτηριου και} & 13 &  &  \\
&  & 14 & \foreignlanguage{greek}{τηϲ παροψιδοϲ εϲωθεν δε γεμουϲιν} & 18 &  &  \\
[0.2em]
\cline{4-4}
\end{tabular}
\end{center}
\end{table}
}
\clearpage
\newpage
 {
 \setlength\arrayrulewidth{1pt}
\begin{table}
\begin{center}
\begin{tabular}{ccc|l|ccc}
\cline{4-4} \\ [-1em]
\multicolumn{7}{c}{\foreignlanguage{greek}{ευαγγελιον κατα μαθθαιον} \textbf{(\nospace{23:25})} } \\ \\ [-1em] % Si on veut ajouter les bordures latérales, remplacer {7}{c} par {7}{|c|}
\cline{4-4} \\
\cline{4-4}
&  &  & &  &  & \\ [-0.9em]
&  & 19 & \foreignlanguage{greek}{εξ αρπαγηϲ και ακραϲιαϲ αδικειαϲ} & 23 &  &  \\
& \textbf{26} &  & \foreignlanguage{greek}{φαριϲαιε τυφλε καθαριϲον πρωτον} & 4 &  &  \\
&  & 5 & \foreignlanguage{greek}{το εντοϲ του ποτηριου και τηϲ παροψι} & 11 &  &  \\
&  & 11 & \foreignlanguage{greek}{δοϲ ινα γενηται και το εκτοϲ αυτων} & 17 &  &  \\
&  & 18 & \foreignlanguage{greek}{καθαρον ουαι υμιν γραμματιϲ και} & 4 & \textbf{27} &  \\
&  & 5 & \foreignlanguage{greek}{φαριϲαιοι υποκριται οτι παρομοιαζεται} & 8 &  &  \\
&  & 9 & \foreignlanguage{greek}{ταφοιϲ κεκονιαϲμενοιϲ οιτινεϲ εξω} & 12 &  &  \\
&  & 12 & \foreignlanguage{greek}{θεν μεν φαινονται ωρεοι εϲωθεν δε γε} & 18 &  &  \\
&  & 18 & \foreignlanguage{greek}{μουϲιν οϲτεων νεκρων και παϲηϲ α} & 23 &  &  \\
&  & 23 & \foreignlanguage{greek}{καθαρϲιαϲ ουτωϲ και υμειϲ εξωθε̅} & 4 & \textbf{28} &  \\
&  & 5 & \foreignlanguage{greek}{μεν φαινεϲθαι τοιϲ ανθρωποιϲ δικαιοι} & 9 &  &  \\
&  & 10 & \foreignlanguage{greek}{εϲωθεν δε μεϲτοι εϲται υποκριϲεωϲ} & 14 &  &  \\
&  & 15 & \foreignlanguage{greek}{και ανομιαϲ ουαι υμιν γραμμα} & 3 & \textbf{29} &  \\
&  & 3 & \foreignlanguage{greek}{τιϲ και φαριϲαιοι υποκριται οτι οικο} & 8 &  &  \\
&  & 8 & \foreignlanguage{greek}{δομειται τουϲ ταφουϲ των προφητω̅} & 12 &  &  \\
&  & 13 & \foreignlanguage{greek}{και κοϲμειται τα μνημια των δικαιω̅} & 18 &  &  \\
& \textbf{30} &  & \foreignlanguage{greek}{και λεγεται ει ημεν εν ταιϲ ημεραιϲ τω̅} & 8 &  &  \\
&  & 9 & \foreignlanguage{greek}{πατερων ημων ουκ αν ημεν κοινω} & 14 &  &  \\
&  & 14 & \foreignlanguage{greek}{νοι αυτων εν τω αιματι των προφητω̅} & 20 &  &  \\
& \textbf{31} &  & \foreignlanguage{greek}{ωϲτε μαρτυριται εαυτοιϲ οτι υιοι εϲται} & 6 &  &  \\
&  & 7 & \foreignlanguage{greek}{των φονευϲαντων τουϲ προφηταϲ} & 10 &  &  \\
& \textbf{32} &  & \foreignlanguage{greek}{και υμειϲ πληρωϲατε το μετρον των} & 6 &  &  \\
&  & 7 & \foreignlanguage{greek}{πατερων υμων οφειϲ γεννηματα} & 2 & \textbf{33} &  \\
&  & 3 & \foreignlanguage{greek}{εχιδνων πωϲ φυγηται απο τηϲ κρι} & 8 &  &  \\
&  & 8 & \foreignlanguage{greek}{ϲεωϲ τηϲ γεεννηϲ} & 10 &  &  \\
& \textbf{34} &  & \foreignlanguage{greek}{δια τουτο ιδου εγω αποϲτελλω προϲ υ} & 7 &  &  \\
&  & 7 & \foreignlanguage{greek}{μαϲ προφηταϲ και ϲοφουϲ και γραμ} & 12 &  &  \\
&  & 12 & \foreignlanguage{greek}{ματειϲ εξ αυτων αποκτενιται και} & 16 &  &  \\
&  & 17 & \foreignlanguage{greek}{ϲταυρωϲεται και εξ αυτων μαϲτιγωϲεται} & 21 &  &  \\
&  & 22 & \foreignlanguage{greek}{εν ταιϲ ϲυναγωγαιϲ υμων} & 25 &  &  \\
[0.2em]
\cline{4-4}
\end{tabular}
\end{center}
\end{table}
}
\clearpage
\newpage
 {
 \setlength\arrayrulewidth{1pt}
\begin{table}
\begin{center}
\begin{tabular}{ccc|l|ccc}
\cline{4-4} \\ [-1em]
\multicolumn{7}{c}{\foreignlanguage{greek}{ευαγγελιον κατα μαθθαιον} \textbf{(\nospace{23:34})} } \\ \\ [-1em] % Si on veut ajouter les bordures latérales, remplacer {7}{c} par {7}{|c|}
\cline{4-4} \\
\cline{4-4}
&  &  & &  &  & \\ [-0.9em]
&  & 26 & \foreignlanguage{greek}{και διωξεται απο πολεωϲ ειϲ πολιν οπωϲ} & 1 & \textbf{35} &  \\
&  & 2 & \foreignlanguage{greek}{ελθη εφ υμαϲ παν αιμα δικαιον εκχυννο} & 8 &  &  \\
&  & 8 & \foreignlanguage{greek}{μενον επι τηϲ γηϲ απο του αιματοϲ αβελ} & 15 &  &  \\
&  & 16 & \foreignlanguage{greek}{του δικαιου εωϲ του αιματοϲ ζαχαριου} & 21 &  &  \\
&  & 22 & \foreignlanguage{greek}{υιου βαραχιου ον εφονευϲατε μεταξυ} & 26 &  &  \\
&  & 27 & \foreignlanguage{greek}{του ναου και του θυϲιαϲτηριου} & 31 &  &  \\
& \textbf{36} &  & \foreignlanguage{greek}{αμην λεγω υμιν οτι ηξει παντα ταυτα} & 7 &  &  \\
&  & 8 & \foreignlanguage{greek}{επι την γενεαν ταυτην} & 11 &  &  \\
& \textbf{37} &  & \foreignlanguage{greek}{ιερουϲαλημ ιερουϲαλημ η αποκτινου} & 4 &  &  \\
&  & 4 & \foreignlanguage{greek}{ϲα τουϲ προφηταϲ και λιθοβοληϲαϲα τουϲ} & 10 &  &  \\
&  & 11 & \foreignlanguage{greek}{απεϲταλμενουϲ προϲ αυτην} & 13 &  &  \\
&  & 14 & \foreignlanguage{greek}{ποϲακειϲ ηθεληϲα επιϲυναγαγειν τα} & 17 &  &  \\
&  & 18 & \foreignlanguage{greek}{τεκνα ϲου ον τροπον επιϲυναγει ορνιϲ} & 23 &  &  \\
&  & 24 & \foreignlanguage{greek}{τα νοϲϲια αυτηϲ υπο ταϲ πτερυγαϲ και ου} & 31 &  &  \\
&  & 31 & \foreignlanguage{greek}{κ ηθεληϲατε ιδου αφιεται υμιν ο} & 4 & \textbf{38} &  \\
&  & 5 & \foreignlanguage{greek}{οικοϲ υμων ερημοϲ λεγω γαρ υμιν} & 3 & \textbf{39} &  \\
&  & 4 & \foreignlanguage{greek}{ου μη με ιδηται απ αρτι εωϲ αν ειπηται} & 12 &  &  \\
&  & 13 & \foreignlanguage{greek}{ευλογημενοϲ ο ερχομενοϲ εν ονοματι \textoverline{κυ}} & 18 &  &  \\
& \mygospelchapter &  & \foreignlanguage{greek}{και εξελθων ο \textoverline{ιϲ} επορευετο απο του ιερου} & 8 &  &  \\
&  & 9 & \foreignlanguage{greek}{και προϲηλθον οι μαθηται αυτου επι} & 14 &  &  \\
&  & 14 & \foreignlanguage{greek}{δειξαι αυτω ταϲ οικοδομαϲ του ιερου} & 19 &  &  \\
& \textbf{2} &  & \foreignlanguage{greek}{ο δε \textoverline{ιϲ} ειπεν αυτοιϲ ου βλεπεται παν} & 8 &  &  \\
&  & 8 & \foreignlanguage{greek}{τα ταυτα αμην λεγω υμιν ου μη α} & 15 &  &  \\
&  & 15 & \foreignlanguage{greek}{φεθη ωδε λιθοϲ επι λιθον οϲ ου καταλυ} & 22 &  &  \\
&  & 22 & \foreignlanguage{greek}{θηϲεται καθημενου δε αυτου} & 3 & \textbf{3} &  \\
&  & 4 & \foreignlanguage{greek}{επι του ορουϲ των ελαιων προϲηλθο̅} & 9 &  &  \\
&  & 10 & \foreignlanguage{greek}{αυτω οι μαθηται κατ ιδιαν λεγο̅} & 15 &  &  \\
&  & 15 & \foreignlanguage{greek}{τεϲ ειπε ημιν ποτε ταυτα εϲται και τι} & 22 &  &  \\
&  & 23 & \foreignlanguage{greek}{το ϲημιον τηϲ ϲηϲ παρουϲιαϲ και τηϲ ϲυ̅} & 30 &  &  \\
&  & 30 & \foreignlanguage{greek}{τελειαϲ του αιωνοϲ} & 32 &  &  \\
[0.2em]
\cline{4-4}
\end{tabular}
\end{center}
\end{table}
}
\clearpage
\newpage
 {
 \setlength\arrayrulewidth{1pt}
\begin{table}
\begin{center}
\begin{tabular}{ccc|l|ccc}
\cline{4-4} \\ [-1em]
\multicolumn{7}{c}{\foreignlanguage{greek}{ευαγγελιον κατα μαθθαιον} \textbf{(\nospace{24:4})} } \\ \\ [-1em] % Si on veut ajouter les bordures latérales, remplacer {7}{c} par {7}{|c|}
\cline{4-4} \\
\cline{4-4}
&  &  & &  &  & \\ [-0.9em]
& \textbf{4} &  & \foreignlanguage{greek}{και αποκριθειϲ ο \textoverline{ιϲ} ειπεν αυτοιϲ βλεπεται} & 7 &  &  \\
&  & 8 & \foreignlanguage{greek}{μη τιϲ υμαϲ πλανηϲη πολλοι γαρ ελευϲον} & 3 & \textbf{5} &  \\
&  & 3 & \foreignlanguage{greek}{ται επι τω ονοματι μου λεγοντεϲ εγω ει} & 10 &  &  \\
&  & 10 & \foreignlanguage{greek}{μει ο \textoverline{χϲ} και πολλουϲ πλανηϲουϲιν} & 15 &  &  \\
& \textbf{6} &  & \foreignlanguage{greek}{μελληϲεται δε ακουειν πολεμουϲ και} & 5 &  &  \\
&  & 6 & \foreignlanguage{greek}{ακοαϲ πολεμων ορατε μη θροειϲθαι δει} & 11 &  &  \\
&  & 12 & \foreignlanguage{greek}{γαρ παντα γενεϲθαι αλλ ουπω εϲτιν} & 17 &  &  \\
&  & 18 & \foreignlanguage{greek}{το τελοϲ} & 19 &  &  \\
& \textbf{7} &  & \foreignlanguage{greek}{εγερθηϲεται γαρ εθνοϲ επ εθνοϲ και βα} & 7 &  &  \\
&  & 7 & \foreignlanguage{greek}{ϲιλεια επι βαϲιλειαν και εϲονται λοι} & 12 &  &  \\
&  & 12 & \foreignlanguage{greek}{μοι και λιμοι και ϲιϲμοι κατα τοπουϲ} & 18 &  &  \\
& \textbf{8} &  & \foreignlanguage{greek}{ταυτα δε παντα αρχη ωδινων} & 5 &  &  \\
& \textbf{9} &  & \foreignlanguage{greek}{τοτε παραδωϲωϲιν υμαϲ ειϲ θλιψιν και} & 6 &  &  \\
&  & 7 & \foreignlanguage{greek}{αποκτενουϲιν υμαϲ και εϲεϲθαι μιϲου} & 11 &  &  \\
&  & 11 & \foreignlanguage{greek}{μενοι υπο παντων των εθνων δια το} & 17 &  &  \\
&  & 18 & \foreignlanguage{greek}{ονομα μου και τοτε ϲκανδαλιϲθη} & 3 & \textbf{10} &  \\
&  & 3 & \foreignlanguage{greek}{ϲονται πολλοι και αλληλουϲ παραδωϲου} & 7 &  &  \\
&  & 7 & \foreignlanguage{greek}{ϲιν και μιϲηϲουϲιν αλληλουϲ και πολ} & 2 & \textbf{11} &  \\
&  & 2 & \foreignlanguage{greek}{λοι ψευδοπροφηται αναϲτηϲονται και} & 5 &  &  \\
&  & 6 & \foreignlanguage{greek}{πλανηϲουϲιν υμαϲ και δια το πλη} & 4 & \textbf{12} &  \\
&  & 4 & \foreignlanguage{greek}{θυνθηναι την ανομιαν ψυγηϲεται} & 7 &  &  \\
&  & 8 & \foreignlanguage{greek}{η αγαπη των πολλων ο δε υπομειναϲ} & 3 & \textbf{13} &  \\
&  & 4 & \foreignlanguage{greek}{ειϲ τελοϲ ϲωθηϲεται} & 6 &  &  \\
& \textbf{14} &  & \foreignlanguage{greek}{και κηρυχθηϲεται τουτο το ευαγγελιο̅} & 5 &  &  \\
&  & 6 & \foreignlanguage{greek}{τηϲ βαϲιλειαϲ εν ολη τη οικουμενη ειϲ} & 12 &  &  \\
&  & 13 & \foreignlanguage{greek}{μαρτυριον τοιϲ εθνεϲιν και τοτε} & 17 &  &  \\
&  & 18 & \foreignlanguage{greek}{ηξει το τελοϲ} & 20 &  &  \\
& \textbf{15} &  & \foreignlanguage{greek}{οταν ουν ιδηται το βδελυγμα τηϲ ερη} & 7 &  &  \\
&  & 7 & \foreignlanguage{greek}{μωϲεωϲ το ρηθεν δια δανιηλ του προ} & 13 &  &  \\
&  & 13 & \foreignlanguage{greek}{φητου εϲτοϲ εν τοπω αγιω ο αναγιγνωϲκω̅} & 19 &  &  \\
&  & 20 & \foreignlanguage{greek}{νοειτω} & 20 &  &  \\
[0.2em]
\cline{4-4}
\end{tabular}
\end{center}
\end{table}
}
\clearpage
\newpage
 {
 \setlength\arrayrulewidth{1pt}
\begin{table}
\begin{center}
\begin{tabular}{ccc|l|ccc}
\cline{4-4} \\ [-1em]
\multicolumn{7}{c}{\foreignlanguage{greek}{ευαγγελιον κατα μαθθαιον} \textbf{(\nospace{24:16})} } \\ \\ [-1em] % Si on veut ajouter les bordures latérales, remplacer {7}{c} par {7}{|c|}
\cline{4-4} \\
\cline{4-4}
&  &  & &  &  & \\ [-0.9em]
& \textbf{16} &  & \foreignlanguage{greek}{τοτε οι εν τη ιουδαια φευγετωϲαν επι τα} & 8 &  &  \\
&  & 9 & \foreignlanguage{greek}{ορη ο επι του δωματοϲ μη καταβαινετω} & 6 & \textbf{17} &  \\
&  & 7 & \foreignlanguage{greek}{αραι τα εκ τηϲ οικειαϲ αυτου κα ο εν τω} & 4 & \textbf{18} &  \\
&  & 5 & \foreignlanguage{greek}{αγρω μη επιϲτρεψατω οπιϲω αραι τα ιμα} & 11 &  &  \\
&  & 11 & \foreignlanguage{greek}{τια αυτου ουαι δε ταιϲ εν γαϲτρι ε} & 6 & \textbf{19} &  \\
&  & 6 & \foreignlanguage{greek}{χουϲαιϲ και ταιϲ θηλαζουϲαιϲ εν εκει} & 11 &  &  \\
&  & 11 & \foreignlanguage{greek}{ναιϲ ταιϲ ημεραιϲ} & 13 &  &  \\
& \textbf{20} &  & \foreignlanguage{greek}{προϲευχεϲθαι δε ινα μη γενηται υμω̅} & 6 &  &  \\
&  & 7 & \foreignlanguage{greek}{η φυγη χειμωνοϲ μηδε ϲαββατω} & 11 &  &  \\
& \textbf{21} &  & \foreignlanguage{greek}{εϲται γαρ τοτε θλιψειϲ μεγαλη οια ου} & 7 &  &  \\
&  & 8 & \foreignlanguage{greek}{γεγονεν απ αρχηϲ κοϲμου εωϲ του νυ̅} & 14 &  &  \\
&  & 15 & \foreignlanguage{greek}{ουδε μη γενηται και ει μη εκο} & 4 & \textbf{22} &  \\
&  & 4 & \foreignlanguage{greek}{λοβωθηϲαν αι ημεραι εκειναι ουκ α̅} & 9 &  &  \\
&  & 10 & \foreignlanguage{greek}{εϲωθη παϲα ϲαρξ δια δε τουϲ εκλε} & 16 &  &  \\
&  & 16 & \foreignlanguage{greek}{κτουϲ κολοβωθηϲονται αι ημεραι} & 19 &  &  \\
&  & 20 & \foreignlanguage{greek}{εκειναι τοτε εαν τιϲ υμιν ειπη} & 5 & \textbf{23} &  \\
&  & 6 & \foreignlanguage{greek}{ιδου ωδε ο \textoverline{χϲ} η ωδε μη πιϲτευϲηται} & 13 &  &  \\
& \textbf{24} &  & \foreignlanguage{greek}{εγερθηϲονται γαρ ψευδοχριϲτοι και} & 4 &  &  \\
&  & 5 & \foreignlanguage{greek}{ψευδοπροφηται και δωϲουϲιν ϲη} & 8 &  &  \\
&  & 8 & \foreignlanguage{greek}{μια μεγαλα και τερατα ωϲτε πλανηϲαι ει δυ} & 15 &  &  \\
&  & 15 & \foreignlanguage{greek}{νατον και τουϲ εκλεκτουϲ ιδου} & 1 & \textbf{25} &  \\
&  & 2 & \foreignlanguage{greek}{προειρηκα υμιν} & 3 &  &  \\
& \textbf{26} &  & \foreignlanguage{greek}{εαν ουν ειπωϲιν υμιν ιδου εν τη ερη} & 8 &  &  \\
&  & 8 & \foreignlanguage{greek}{μω εϲτιν μη εξελθητε ιδου εν τοιϲ} & 14 &  &  \\
&  & 15 & \foreignlanguage{greek}{ταμιοιϲ μη πιϲτευϲηται} & 17 &  &  \\
& \textbf{27} &  & \foreignlanguage{greek}{ωϲπερ γαρ η αϲτραπη εξερχεται απο} & 6 &  &  \\
&  & 7 & \foreignlanguage{greek}{ανατολων και φαινεται εωϲ δυϲμω̅} & 11 &  &  \\
&  & 12 & \foreignlanguage{greek}{ουτωϲ εϲται και η παρουϲια του υι} & 18 &  &  \\
&  & 18 & \foreignlanguage{greek}{ου του ανθρωπου οπου γαρ εαν η το} & 5 & \textbf{28} &  \\
&  & 6 & \foreignlanguage{greek}{πτωμα εκει ϲυναχθηϲονται οι αετοι} & 10 &  &  \\
[0.2em]
\cline{4-4}
\end{tabular}
\end{center}
\end{table}
}
\clearpage
\newpage
 {
 \setlength\arrayrulewidth{1pt}
\begin{table}
\begin{center}
\begin{tabular}{ccc|l|ccc}
\cline{4-4} \\ [-1em]
\multicolumn{7}{c}{\foreignlanguage{greek}{ευαγγελιον κατα μαθθαιον} \textbf{(\nospace{24:29})} } \\ \\ [-1em] % Si on veut ajouter les bordures latérales, remplacer {7}{c} par {7}{|c|}
\cline{4-4} \\
\cline{4-4}
&  &  & &  &  & \\ [-0.9em]
& \textbf{29} &  & \foreignlanguage{greek}{ευθεωϲ δε μετα την θλιψιν των ημερω̅} & 7 &  &  \\
&  & 8 & \foreignlanguage{greek}{εκεινων ο ηλιοϲ ϲκοτιϲθηϲεται και η} & 13 &  &  \\
&  & 14 & \foreignlanguage{greek}{ϲεληνη ου δωϲει το φεγγοϲ αυτηϲ και οι} & 21 &  &  \\
&  & 22 & \foreignlanguage{greek}{αϲτερεϲ πεϲουνται απο του ουρανου} & 26 &  &  \\
&  & 27 & \foreignlanguage{greek}{και αι δυναμειϲ των ουρανων ϲαλευ} & 32 &  &  \\
&  & 32 & \foreignlanguage{greek}{θηϲονται και τοτε φανηϲεται το ϲη} & 5 & \textbf{30} &  \\
&  & 5 & \foreignlanguage{greek}{μιον του υιου του \textoverline{ανου} εν τω ουρανω} & 12 &  &  \\
&  & 13 & \foreignlanguage{greek}{και τοτε κοψονται παϲαι αι φυλαι τηϲ γηϲ} & 20 &  &  \\
&  & 21 & \foreignlanguage{greek}{και οψονται τον υιον του \textoverline{ανου} ερχομενο̅} & 27 &  &  \\
&  & 28 & \foreignlanguage{greek}{επι των νεφελων του ουρανου μετα} & 33 &  &  \\
&  & 34 & \foreignlanguage{greek}{δυναμεωϲ και δοξηϲ πολληϲ και τοτε} & 2 & \textbf{31} &  \\
&  & 3 & \foreignlanguage{greek}{αποϲτελει τουϲ αγγελουϲ αυτου μετα} & 7 &  &  \\
&  & 8 & \foreignlanguage{greek}{ϲαλπιγγοϲ μεγαληϲ και επιϲυναξουϲι̅} & 11 &  &  \\
&  & 12 & \foreignlanguage{greek}{τουϲ εκλεκτουϲ αυτου εκ των τεϲϲαρω̅} & 17 &  &  \\
&  & 18 & \foreignlanguage{greek}{ανεμων απ ακρων ουρανων εωϲ α} & 23 &  &  \\
&  & 23 & \foreignlanguage{greek}{κρων αυτων} & 24 &  &  \\
& \textbf{32} &  & \foreignlanguage{greek}{απο δε τηϲ ϲυκηϲ μαθεται την παραβολη̅} & 7 &  &  \\
&  & 8 & \foreignlanguage{greek}{οταν ηδη το κλαδοϲ αυτηϲ γενηται απα} & 15 &  &  \\
&  & 15 & \foreignlanguage{greek}{λοϲ και τα φυλλα εκφυει γιγνωϲκεται} & 20 &  &  \\
&  & 21 & \foreignlanguage{greek}{οτι εγγυϲ το θεροϲ ουτωϲ και υμειϲ} & 3 & \textbf{33} &  \\
&  & 4 & \foreignlanguage{greek}{οταν ειδηται ταυτα παντα γινωϲκε} & 8 &  &  \\
&  & 8 & \foreignlanguage{greek}{ται οτι εγγυϲ εϲτιν επι θυραιϲ} & 13 &  &  \\
& \textbf{34} &  & \foreignlanguage{greek}{αμην λεγω υμιν ου μη παρελθη η γενε} & 8 &  &  \\
&  & 8 & \foreignlanguage{greek}{α αυτη εωϲ αν παντα ταυτα γενηται} & 14 &  &  \\
& \textbf{35} &  & \foreignlanguage{greek}{ο ουρανοϲ και η γη παρελευϲεται οι δε} & 8 &  &  \\
&  & 9 & \foreignlanguage{greek}{λογοι μου ου μη παρελθωϲιν} & 13 &  &  \\
& \textbf{36} &  & \foreignlanguage{greek}{περι δε τηϲ ημεραϲ εκεινηϲ και ωραϲ} & 7 &  &  \\
&  & 8 & \foreignlanguage{greek}{ουδειϲ οιδεν ουδε οι αγγελοι των ουρα} & 14 &  &  \\
&  & 14 & \foreignlanguage{greek}{νων ει μη ο \textoverline{πηρ} μου μονοϲ} & 20 &  &  \\
& \textbf{37} &  & \foreignlanguage{greek}{ωϲπερ δε αι ημεραι του νωε ουτωϲ εϲται} & 8 &  &  \\
[0.2em]
\cline{4-4}
\end{tabular}
\end{center}
\end{table}
}
\clearpage
\newpage
 {
 \setlength\arrayrulewidth{1pt}
\begin{table}
\begin{center}
\begin{tabular}{ccc|l|ccc}
\cline{4-4} \\ [-1em]
\multicolumn{7}{c}{\foreignlanguage{greek}{ευαγγελιον κατα μαθθαιον} \textbf{(\nospace{24:37})} } \\ \\ [-1em] % Si on veut ajouter les bordures latérales, remplacer {7}{c} par {7}{|c|}
\cline{4-4} \\
\cline{4-4}
&  &  & &  &  & \\ [-0.9em]
&  & 9 & \foreignlanguage{greek}{και η παρουϲια του υιου του \textoverline{ανου}} & 15 &  &  \\
& \textbf{38} &  & \foreignlanguage{greek}{ωϲπερ γαρ ηϲαν εν ταιϲ ημεραιϲ ταιϲ} & 7 &  &  \\
&  & 8 & \foreignlanguage{greek}{προ του κατακλυϲμου τρωγοντεϲ και} & 12 &  &  \\
&  & 13 & \foreignlanguage{greek}{πινοντεϲ γαμουντεϲ και εκγαμιϲκο̅} & 16 &  &  \\
&  & 16 & \foreignlanguage{greek}{τεϲ αχρι ηϲ ημεραϲ ειϲηλθεν νωε ειϲ} & 22 &  &  \\
&  & 23 & \foreignlanguage{greek}{την κιβωτον και ουκ εγνωϲαν εωϲ} & 4 & \textbf{39} &  \\
&  & 5 & \foreignlanguage{greek}{αν ηλθεν ο κατακλυϲμοϲ και ηρεν α} & 11 &  &  \\
&  & 11 & \foreignlanguage{greek}{πανταϲ ουτωϲ εϲται και η παρουϲια} & 16 &  &  \\
&  & 17 & \foreignlanguage{greek}{του υιου του ανθρωπου} & 20 &  &  \\
& \textbf{40} &  & \foreignlanguage{greek}{τοτε δυο εϲονται εν τω αγρω ο ειϲ παρα} & 9 &  &  \\
&  & 9 & \foreignlanguage{greek}{λαμβανεται και ο ειϲ αφιεται δυο α} & 2 & \textbf{41} &  \\
&  & 2 & \foreignlanguage{greek}{ληθουϲαι εν τω μυλω μια παραλαμβα} & 7 &  &  \\
&  & 7 & \foreignlanguage{greek}{νεται και μια αφιεται γρηγοριται} & 1 & \textbf{42} &  \\
&  & 2 & \foreignlanguage{greek}{ουν οτι ουκ οιδατε ποια ημερα ο \textoverline{κϲ} υ} & 10 &  &  \\
&  & 10 & \foreignlanguage{greek}{μων ερχεται} & 11 &  &  \\
& \textbf{43} &  & \foreignlanguage{greek}{εκεινο δε γινωϲκεται οτι ει ηδει ο οικο} & 8 &  &  \\
&  & 8 & \foreignlanguage{greek}{δεϲποτηϲ ποια φυλακη ο κλεπτηϲ ερ} & 13 &  &  \\
&  & 13 & \foreignlanguage{greek}{χεται εγρηγορηϲεν αν και ουκ αν ηα} & 19 &  &  \\
&  & 19 & \foreignlanguage{greek}{ϲεν διορυγηναι τον οικον αυτου} & 23 &  &  \\
& \textbf{44} &  & \foreignlanguage{greek}{δια τουτο και υμειϲ γινεϲθαι ετοιμοι} & 6 &  &  \\
&  & 7 & \foreignlanguage{greek}{οτι η ωρα ου δοκειται ο υιοϲ του \textoverline{ανου} ερ} & 16 &  &  \\
&  & 16 & \foreignlanguage{greek}{χεται τιϲ αρα εϲτιν ο πιϲτοϲ δουλοϲ} & 6 & \textbf{45} &  \\
&  & 7 & \foreignlanguage{greek}{και φρονιμοϲ ον κατεϲτηϲεν ο \textoverline{κϲ} αυτου} & 13 &  &  \\
&  & 14 & \foreignlanguage{greek}{επι τηϲ οικετιαϲ αυτου του διδοναι} & 19 &  &  \\
&  & 20 & \foreignlanguage{greek}{την τροφην εν καιρω} & 23 &  &  \\
& \textbf{46} &  & \foreignlanguage{greek}{μακαριοϲ ο δουλοϲ εκεινοϲ ον ελθων} & 6 &  &  \\
&  & 7 & \foreignlanguage{greek}{ο \textoverline{κϲ} αυτου ευρηϲει ποιουντα ουτωϲ} & 12 &  &  \\
& \textbf{47} &  & \foreignlanguage{greek}{αμην λεγω υμιν οτι επι παϲιν τοιϲ υπαρ} & 8 &  &  \\
&  & 8 & \foreignlanguage{greek}{χουϲιν αυτου καταϲτηϲει αυτον} & 11 &  &  \\
& \textbf{48} &  & \foreignlanguage{greek}{εαν δε ειπη ο κακοϲ δουλοϲ εκεινοϲ εν} & 8 &  &  \\
[0.2em]
\cline{4-4}
\end{tabular}
\end{center}
\end{table}
}
\clearpage
\newpage
 {
 \setlength\arrayrulewidth{1pt}
\begin{table}
\begin{center}
\begin{tabular}{ccc|l|ccc}
\cline{4-4} \\ [-1em]
\multicolumn{7}{c}{\foreignlanguage{greek}{ευαγγελιον κατα μαθθαιον} \textbf{(\nospace{24:48})} } \\ \\ [-1em] % Si on veut ajouter les bordures latérales, remplacer {7}{c} par {7}{|c|}
\cline{4-4} \\
\cline{4-4}
&  &  & &  &  & \\ [-0.9em]
&  & 9 & \foreignlanguage{greek}{τη καρδια αυτου χρονιζει ο \textoverline{κϲ} μου ελθει̅} & 16 &  &  \\
& \textbf{49} &  & \foreignlanguage{greek}{και αρξηται τυπτειν τουϲ ϲυνδουλουϲ} & 5 &  &  \\
&  & 6 & \foreignlanguage{greek}{εϲθιειν τε και πινειν μετα των μεθυϲτω̅} & 12 &  &  \\
& \textbf{50} &  & \foreignlanguage{greek}{ηξει ο \textoverline{κϲ} του δουλου εκεινου εν ημερα η} & 9 &  &  \\
&  & 10 & \foreignlanguage{greek}{ου προϲδοκα και εν ωρα η ου γινωϲκει} & 17 &  &  \\
& \textbf{51} &  & \foreignlanguage{greek}{και διχοτομηϲει αυτον και το μεροϲ αυ} & 7 &  &  \\
&  & 7 & \foreignlanguage{greek}{του μετα των υποκριτων θηϲει εκει εϲτε} & 13 &  &  \\
&  & 14 & \foreignlanguage{greek}{ο κλαυθμοϲ και ο βρυγμοϲ των οδοντων} & 20 &  &  \\
& \mygospelchapter &  & \foreignlanguage{greek}{τοτε ωμοιωθη η βαϲιλεια των ουρανων} & 6 &  &  \\
&  & 7 & \foreignlanguage{greek}{δεκα παρθενοιϲ αιτινεϲ λαβουϲαι ταϲ} & 11 &  &  \\
&  & 12 & \foreignlanguage{greek}{λαμπαδαϲ αυτων εξηλθον ειϲ απαντη} & 16 &  &  \\
&  & 16 & \foreignlanguage{greek}{ϲιν του νυμφιου πεντε δε ηϲαν εξ αυ} & 5 & \textbf{2} &  \\
&  & 5 & \foreignlanguage{greek}{των φρονιμοι και πεντε μωραι αιτι} & 1 & \textbf{3} &  \\
&  & 1 & \foreignlanguage{greek}{νεϲ μωραι λαβουϲαι ταϲ λαμπαδαϲ αυτω̅} & 6 &  &  \\
&  & 7 & \foreignlanguage{greek}{ουκ ελαβον μεθ εαυτων ελαιον} & 11 &  &  \\
& \textbf{4} &  & \foreignlanguage{greek}{αι δε φρονιμοι ελαβον ελαιον εν τοιϲ αγ} & 8 &  &  \\
&  & 8 & \foreignlanguage{greek}{γιοιϲ αυτων μετα των λαμπαδων αυτω̅} & 13 &  &  \\
& \textbf{5} &  & \foreignlanguage{greek}{χρονιζοντοϲ δε του νυμφιου ενυϲτα} & 5 &  &  \\
&  & 5 & \foreignlanguage{greek}{ξαν παϲαι και εκαθευδον} & 8 &  &  \\
& \textbf{6} &  & \foreignlanguage{greek}{μεϲηϲ δε νυκτοϲ κραυγη γεγονεν ιδου} & 6 &  &  \\
&  & 7 & \foreignlanguage{greek}{ο νυμφιοϲ ερχεται εξερχεϲθαι ειϲ απα̅} & 12 &  &  \\
&  & 12 & \foreignlanguage{greek}{τηϲιν αυτου τοτε ηγερθηϲαν πα} & 3 & \textbf{7} &  \\
&  & 3 & \foreignlanguage{greek}{ϲαι αι παρθενοι εκειναι και εκοϲμη} & 8 &  &  \\
&  & 8 & \foreignlanguage{greek}{ϲαν ταϲ λαμπαδαϲ αυτων} & 11 &  &  \\
& \textbf{8} &  & \foreignlanguage{greek}{αι δε μωραι ταιϲ φρονιμοιϲ ειπον δοτε} & 7 &  &  \\
&  & 8 & \foreignlanguage{greek}{ημιν εκ του ελαιου υμων οτι αι λαμ} & 15 &  &  \\
&  & 15 & \foreignlanguage{greek}{παδεϲ ημων ϲβεννυνται} & 17 &  &  \\
& \textbf{9} &  & \foreignlanguage{greek}{απεκριθηϲαν δε αι φρονιμαοι λεγουϲαι} & 6 &  &  \\
&  & 7 & \foreignlanguage{greek}{μηποτε ου μη αρκεϲη ημιν και υμιν} & 13 &  &  \\
&  & 14 & \foreignlanguage{greek}{πορευεϲθαι δε μαλλον προϲ τουϲ πωλουνταϲ} & 19 &  &  \\
[0.2em]
\cline{4-4}
\end{tabular}
\end{center}
\end{table}
}
\clearpage
\newpage
 {
 \setlength\arrayrulewidth{1pt}
\begin{table}
\begin{center}
\begin{tabular}{ccc|l|ccc}
\cline{4-4} \\ [-1em]
\multicolumn{7}{c}{\foreignlanguage{greek}{ευαγγελιον κατα μαθθαιον} \textbf{(\nospace{25:9})} } \\ \\ [-1em] % Si on veut ajouter les bordures latérales, remplacer {7}{c} par {7}{|c|}
\cline{4-4} \\
\cline{4-4}
&  &  & &  &  & \\ [-0.9em]
&  & 20 & \foreignlanguage{greek}{και αγοραϲαται εαυταιϲ απερχομε} & 1 & \textbf{10} &  \\
&  & 1 & \foreignlanguage{greek}{νων δε αυτων αγοραϲε ηλθεν ο νυμφι} & 7 &  &  \\
&  & 7 & \foreignlanguage{greek}{οϲ και αι ετοιμοι ειϲηλθον μετ αυτου} & 13 &  &  \\
&  & 14 & \foreignlanguage{greek}{ειϲ τουϲ γαμουϲ και εκλιϲθη η θυρα} & 20 &  &  \\
& \textbf{11} &  & \foreignlanguage{greek}{υϲτερον δε ηλθον και αι λοιπαι παρθε} & 7 &  &  \\
&  & 7 & \foreignlanguage{greek}{νοι λεγουϲαι \textoverline{κε} \textoverline{κε} ανοιξον ημιν} & 12 &  &  \\
& \textbf{12} &  & \foreignlanguage{greek}{ο δε αποκριθειϲ ειπεν αμην λεγω υμι̅} & 7 &  &  \\
&  & 8 & \foreignlanguage{greek}{ουκ οιδα υμαϲ γρηγορειται ουν οτι ου} & 4 & \textbf{13} &  \\
&  & 4 & \foreignlanguage{greek}{κ οιδατε την ημεραν ουδε την ωραν} & 10 &  &  \\
& \textbf{14} &  & \foreignlanguage{greek}{ωϲπερ ανθρωποϲ αποδημων εκαλεϲε̅} & 4 &  &  \\
&  & 5 & \foreignlanguage{greek}{τουϲ ιδιουϲ δουλουϲ και παρεδωκεν} & 9 &  &  \\
&  & 10 & \foreignlanguage{greek}{αυτοιϲ τα υπαρχοντα αυτου και ω με̅} & 3 & \textbf{15} &  \\
&  & 4 & \foreignlanguage{greek}{εδωκεν πεντε ταλαντα ω δε δυο} & 9 &  &  \\
&  & 10 & \foreignlanguage{greek}{ω δε εν εκαϲτω κατα την ιδιαν δυνα} & 17 &  &  \\
&  & 17 & \foreignlanguage{greek}{μιν και απεδημηϲεν ευθεωϲ} & 20 &  &  \\
& \textbf{16} &  & \foreignlanguage{greek}{πορευθειϲ δε ο τα πεντε ταλαντα λαβω̅} & 7 &  &  \\
&  & 8 & \foreignlanguage{greek}{ηργαϲατο εν αυτοιϲ και εποιηϲεν αλλα} & 13 &  &  \\
&  & 14 & \foreignlanguage{greek}{πεντε ταλαντα ωϲαυτωϲ και ο τα δυο} & 5 & \textbf{17} &  \\
&  & 6 & \foreignlanguage{greek}{εκερδηϲεν και αυτοϲ αλλα δυο} & 10 &  &  \\
& \textbf{18} &  & \foreignlanguage{greek}{ο δε το εν λαβων απελθων ωρυξεν εν} & 8 &  &  \\
&  & 9 & \foreignlanguage{greek}{τη γη και απεκρυψεν το αργυριον του} & 15 &  &  \\
&  & 16 & \foreignlanguage{greek}{\textoverline{κυ} αυτου μετα δε χρονον τινα ερ} & 5 & \textbf{19} &  \\
&  & 5 & \foreignlanguage{greek}{χεται ο \textoverline{κϲ} των δουλων εκεινων και} & 11 &  &  \\
&  & 12 & \foreignlanguage{greek}{ϲυνερει μετ αυτων λογον} & 15 &  &  \\
& \textbf{20} &  & \foreignlanguage{greek}{και προϲελθων ο τα πεντε ταλαντα} & 6 &  &  \\
&  & 7 & \foreignlanguage{greek}{λαβων προϲηνεγκεν αλλα πεντε λεγω̅} & 11 &  &  \\
&  & 12 & \foreignlanguage{greek}{\textoverline{κε} πεντε ταλαντα μοι παρεδωκαϲ ει} & 17 &  &  \\
&  & 17 & \foreignlanguage{greek}{δε αλλα πεντε ταλαντα εκερδηϲα επ αυ} & 23 &  &  \\
&  & 23 & \foreignlanguage{greek}{τοιϲ εφη δε αυτω ο \textoverline{κϲ} αυτου ευ} & 7 & \textbf{21} &  \\
&  & 8 & \foreignlanguage{greek}{δουλε αγαθε και πιϲτε επι ολιγα ηϲ πιϲτοϲ} & 15 &  &  \\
[0.2em]
\cline{4-4}
\end{tabular}
\end{center}
\end{table}
}
\clearpage
\newpage
 {
 \setlength\arrayrulewidth{1pt}
\begin{table}
\begin{center}
\begin{tabular}{ccc|l|ccc}
\cline{4-4} \\ [-1em]
\multicolumn{7}{c}{\foreignlanguage{greek}{ευαγγελιον κατα μαθθαιον} \textbf{(\nospace{25:21})} } \\ \\ [-1em] % Si on veut ajouter les bordures latérales, remplacer {7}{c} par {7}{|c|}
\cline{4-4} \\
\cline{4-4}
&  &  & &  &  & \\ [-0.9em]
&  & 16 & \foreignlanguage{greek}{επι πολλων ϲε καταϲτηϲω ειϲελθε ειϲ} & 21 &  &  \\
&  & 22 & \foreignlanguage{greek}{την χαραν του \textoverline{κυ} ϲου} & 26 &  &  \\
& \textbf{22} &  & \foreignlanguage{greek}{προϲελθων δε και ο τα δυο ταλαντα ειπε̅} & 8 &  &  \\
&  & 9 & \foreignlanguage{greek}{\textoverline{κε} δυο ταλαντα μοι παρεδωκαϲ ειδε} & 14 &  &  \\
&  & 15 & \foreignlanguage{greek}{αλλα δυο ταλαντα εκερδηϲα επ αυτοιϲ} & 20 &  &  \\
& \textbf{23} &  & \foreignlanguage{greek}{εφη αυτω ο \textoverline{κϲ} αυτου ευ δουλε αγαθε} & 8 &  &  \\
&  & 9 & \foreignlanguage{greek}{και πιϲτε επι ολειγα ηϲ πιϲτοϲ επι πολ} & 16 &  &  \\
&  & 16 & \foreignlanguage{greek}{λων ϲε καταϲτηϲω ειϲελθε ειϲ την} & 21 &  &  \\
&  & 22 & \foreignlanguage{greek}{χαραν του \textoverline{κυ} ϲου προϲελθων δε} & 2 & \textbf{24} &  \\
&  & 3 & \foreignlanguage{greek}{και ο το εν ταλαντον ειληφωϲ ειπεν} & 9 &  &  \\
&  & 10 & \foreignlanguage{greek}{\textoverline{κε} εγνων ϲε οτι ϲκληροϲ ει ανθρωποϲ} & 16 &  &  \\
&  & 17 & \foreignlanguage{greek}{θεριζων οπου ουκ εϲπειραϲ και ϲυνα} & 22 &  &  \\
&  & 22 & \foreignlanguage{greek}{γων οπου ουκ εϲκορπιϲαϲ και φοβη} & 2 & \textbf{25} &  \\
&  & 2 & \foreignlanguage{greek}{θειϲ απελθων εκρυψα το ταλαντον ϲου} & 7 &  &  \\
&  & 8 & \foreignlanguage{greek}{εν τη γη ειδε εχειϲ το ϲον} & 14 &  &  \\
& \textbf{26} &  & \foreignlanguage{greek}{αποκριθειϲ δε ο \textoverline{κϲ} αυτου ειπεν αυτω} & 7 &  &  \\
&  & 8 & \foreignlanguage{greek}{πονηρε δουλε και οκνηρε ηδιϲ οτι} & 13 &  &  \\
&  & 14 & \foreignlanguage{greek}{εγω \textoverline{ανοϲ} αυϲτηροϲ ειμει θεριζω οπου} & 19 &  &  \\
&  & 20 & \foreignlanguage{greek}{ουκ εϲπειρα και ϲυναγω οθεν ου διε} & 26 &  &  \\
&  & 26 & \foreignlanguage{greek}{ϲκορπιϲα εδει ουν ϲε βαλιν τα αργυ} & 6 & \textbf{27} &  \\
&  & 6 & \foreignlanguage{greek}{ρια μου τοιϲ τραπεζιταιϲ και ελθων} & 11 &  &  \\
&  & 12 & \foreignlanguage{greek}{εγω εκομιϲαμην αν το εμον ϲυν τω} & 18 &  &  \\
&  & 19 & \foreignlanguage{greek}{τοκω αρατε ουν απ αυτου το ταλαν} & 6 & \textbf{28} &  \\
&  & 6 & \foreignlanguage{greek}{τον και δοτε τω εχοντι τα δεκα ταλα̅} & 13 &  &  \\
&  & 13 & \foreignlanguage{greek}{τα τω γαρ εχοντι δοθηϲεται και πε} & 6 & \textbf{29} &  \\
&  & 6 & \foreignlanguage{greek}{ριϲευθηϲεται απο δε του μη εχοντοϲ} & 11 &  &  \\
&  & 12 & \foreignlanguage{greek}{και ο εχει αρθηϲεται απ αυτου και το̅} & 2 & \textbf{30} &  \\
&  & 3 & \foreignlanguage{greek}{αχριον δουλον εκβαλετε ειϲ το ϲκοτοϲ} & 8 &  &  \\
&  & 9 & \foreignlanguage{greek}{το εξωτερον εκει εϲται ο κλαθμοϲ} & 14 &  &  \\
&  & 15 & \foreignlanguage{greek}{και ο βρυγμοϲ των οδοντων} & 19 &  &  \\
[0.2em]
\cline{4-4}
\end{tabular}
\end{center}
\end{table}
}
\clearpage
\newpage
 {
 \setlength\arrayrulewidth{1pt}
\begin{table}
\begin{center}
\begin{tabular}{ccc|l|ccc}
\cline{4-4} \\ [-1em]
\multicolumn{7}{c}{\foreignlanguage{greek}{ευαγγελιον κατα μαθθαιον} \textbf{(\nospace{25:31})} } \\ \\ [-1em] % Si on veut ajouter les bordures latérales, remplacer {7}{c} par {7}{|c|}
\cline{4-4} \\
\cline{4-4}
&  &  & &  &  & \\ [-0.9em]
& \textbf{31} &  & \foreignlanguage{greek}{οταν δε ελθη ο υιοϲ του ανθρωπου εν τη} & 9 &  &  \\
&  & 10 & \foreignlanguage{greek}{δοξη αυτου και παντεϲ οι αγιοι αγγελοι} & 16 &  &  \\
&  & 17 & \foreignlanguage{greek}{μετ αυτου τοτε καθιϲει επι θρονου δο} & 23 &  &  \\
&  & 23 & \foreignlanguage{greek}{ξηϲ αυτου και ϲυναχθηϲονται παντα} & 3 & \textbf{32} &  \\
&  & 4 & \foreignlanguage{greek}{τα εθνη εμπροϲθεν αυτου και αφοριϲει} & 9 &  &  \\
&  & 10 & \foreignlanguage{greek}{αυτουϲ απ αλληλων ωϲπερ ο ποιμην} & 15 &  &  \\
&  & 16 & \foreignlanguage{greek}{αφοριζει τα προβατα απο των εριφων} & 21 &  &  \\
& \textbf{33} &  & \foreignlanguage{greek}{και ϲτηϲει τα μεν προβατα εκ δεξιων αυ} & 8 &  &  \\
&  & 8 & \foreignlanguage{greek}{του τα δε εριφια εξ ευωνυμων} & 13 &  &  \\
& \textbf{34} &  & \foreignlanguage{greek}{τοτε ερει ο βαϲιλευϲ τοιϲ εκ δεξιων αυ} & 8 &  &  \\
&  & 8 & \foreignlanguage{greek}{του δευτε οι ευλογημενοι του \textoverline{πρϲ} μου} & 14 &  &  \\
&  & 15 & \foreignlanguage{greek}{κληρονομηϲατε κληρονομηϲητε την ητοιμαϲμενη̅} & 18 &  &  \\
&  & 19 & \foreignlanguage{greek}{υμιν βαϲιλειαν απο καταβοληϲ κοϲμου} & 23 &  &  \\
& \textbf{35} &  & \foreignlanguage{greek}{επιναϲα γαρ και εδωκατε μοι φαγειν} & 6 &  &  \\
&  & 7 & \foreignlanguage{greek}{και εδιψηϲα και εποτιϲατε με ξενοϲ η} & 13 &  &  \\
&  & 13 & \foreignlanguage{greek}{μην και ϲυνηγαγετε με γυμνοϲ και} & 2 & \textbf{36} &  \\
&  & 3 & \foreignlanguage{greek}{περιεβαλεται με ηϲθενηϲα και επε} & 7 &  &  \\
&  & 7 & \foreignlanguage{greek}{ϲκεψαϲθαι με εν φυλακη ημην και} & 12 &  &  \\
&  & 13 & \foreignlanguage{greek}{ηλθατε προϲ με τοτε αποκριθηϲο̅} & 2 & \textbf{37} &  \\
&  & 2 & \foreignlanguage{greek}{ται αυτω οι δικαιοι λεγοντεϲ \textoverline{κε} ποτε} & 8 &  &  \\
&  & 9 & \foreignlanguage{greek}{ϲε ιδομεν πινωντα και εθρεψαμεν} & 13 &  &  \\
&  & 14 & \foreignlanguage{greek}{η διψωντα και εποτιϲαμεν ποτε δε} & 2 & \textbf{38} &  \\
&  & 3 & \foreignlanguage{greek}{ϲε ιδομεν ξενον και ϲυνηγαγομεν} & 7 &  &  \\
&  & 8 & \foreignlanguage{greek}{η γυμνον και περιεβαλομεν ποτε δε} & 2 & \textbf{39} &  \\
&  & 3 & \foreignlanguage{greek}{ϲε ιδομεν αϲθενη η εν φυλακη και ηλ} & 10 &  &  \\
&  & 10 & \foreignlanguage{greek}{θομεν προϲ ϲε} & 12 &  &  \\
& \textbf{40} &  & \foreignlanguage{greek}{και αποκριθειϲ ο βαϲιλευϲ ερει αυτοιϲ} & 6 &  &  \\
&  & 7 & \foreignlanguage{greek}{αμην λεγω υμιν εφ οϲον εποιηϲατε} & 12 &  &  \\
&  & 13 & \foreignlanguage{greek}{ενι τουτων των αδελφων μου των} & 18 &  &  \\
&  & 19 & \foreignlanguage{greek}{ελαχιϲτων εμοι εποιηϲατε} & 21 &  &  \\
[0.2em]
\cline{4-4}
\end{tabular}
\end{center}
\end{table}
}
\clearpage
\newpage
 {
 \setlength\arrayrulewidth{1pt}
\begin{table}
\begin{center}
\begin{tabular}{ccc|l|ccc}
\cline{4-4} \\ [-1em]
\multicolumn{7}{c}{\foreignlanguage{greek}{ευαγγελιον κατα μαθθαιον} \textbf{(\nospace{25:41})} } \\ \\ [-1em] % Si on veut ajouter les bordures latérales, remplacer {7}{c} par {7}{|c|}
\cline{4-4} \\
\cline{4-4}
&  &  & &  &  & \\ [-0.9em]
& \textbf{41} &  & \foreignlanguage{greek}{τοτε ερει και τοιϲ εξ ευωνυμοιϲ πορευ} & 7 &  &  \\
&  & 7 & \foreignlanguage{greek}{εϲθαι απ εμου οι κατηραμενοι ειϲ το πυρ} & 14 &  &  \\
&  & 15 & \foreignlanguage{greek}{το αιωνιον το ητοιμαϲμενον τω διαβολω} & 20 &  &  \\
&  & 21 & \foreignlanguage{greek}{και τοιϲ αγγελοιϲ αυτου επιναϲα γαρ} & 2 & \textbf{42} &  \\
&  & 3 & \foreignlanguage{greek}{και ουκ εδωκατε μοι φαγειν εδιψηϲα} & 8 &  &  \\
&  & 9 & \foreignlanguage{greek}{και ουκ εποτιϲαται με ξενοϲ ημην και} & 3 & \textbf{43} &  \\
&  & 4 & \foreignlanguage{greek}{ου ϲυνηγαγεται με γυμνοϲ και ου περιε} & 10 &  &  \\
&  & 10 & \foreignlanguage{greek}{βαλεται με αϲθενηϲ και εν φυλακη και} & 16 &  &  \\
&  & 17 & \foreignlanguage{greek}{ουκ επεϲκεψαϲθαι με} & 19 &  &  \\
& \textbf{44} &  & \foreignlanguage{greek}{τοτε αποκριθηϲονται και αυτοι λεγοντεϲ} & 5 &  &  \\
&  & 6 & \foreignlanguage{greek}{\textoverline{κε} ποτε ϲε ειδομεν πινωντα η διψων} & 12 &  &  \\
&  & 12 & \foreignlanguage{greek}{τα η ξενον η γυμνον η αϲθενη η εν} & 20 &  &  \\
&  & 21 & \foreignlanguage{greek}{φυλακη και ου διηκονηϲαμεν ϲοι} & 25 &  &  \\
& \textbf{45} &  & \foreignlanguage{greek}{τοτε αποκριθηϲεται αυτοιϲ λεγων αμη̅} & 5 &  &  \\
&  & 6 & \foreignlanguage{greek}{λεγω υμιν εφ οϲον ουκ εποιηϲατε ενι του} & 13 &  &  \\
&  & 13 & \foreignlanguage{greek}{των των ελαχιϲτων ουδε εμοι εποιηϲατε} & 18 &  &  \\
& \textbf{46} &  & \foreignlanguage{greek}{και απελευϲονται ουτοι ει κολαϲιν αι} & 6 &  &  \\
&  & 6 & \foreignlanguage{greek}{ωνιον οι δε δικαιοι ειϲ ζωην αιωνιον} & 12 &  &  \\
& \mygospelchapter &  & \foreignlanguage{greek}{και εγενετο οτε ετελεϲεν ο \textoverline{ιϲ} πανταϲ} & 7 &  &  \\
&  & 8 & \foreignlanguage{greek}{τουϲ λογουϲ τουϲ ειπεν τοιϲ μαθηταιϲ} & 13 &  &  \\
&  & 14 & \foreignlanguage{greek}{αυτου οιδατε οτι μετα ημεραϲ δυο} & 5 & \textbf{2} &  \\
&  & 6 & \foreignlanguage{greek}{το παϲχα γεινεται και ο υιοϲ του \textoverline{ανου}} & 13 &  &  \\
&  & 14 & \foreignlanguage{greek}{παραδιδοτε ειϲ το ϲταυρωθηναι} & 17 &  &  \\
& \textbf{3} &  & \foreignlanguage{greek}{τοτε ϲυνηχθηϲαν οι αρχιερειϲ και οι φα} & 7 &  &  \\
&  & 7 & \foreignlanguage{greek}{ριϲαιοι και οι πρεϲβυτεροι του λαου} & 12 &  &  \\
&  & 13 & \foreignlanguage{greek}{ειϲ την αυλην του αρχιερεωϲ του λε} & 19 &  &  \\
&  & 19 & \foreignlanguage{greek}{γομενου καιαφα και ϲυνεβουλευ} & 2 & \textbf{4} &  \\
&  & 2 & \foreignlanguage{greek}{ϲαντο ινα τον \textoverline{ιν} δολω κρατηϲωϲιν} & 7 &  &  \\
&  & 8 & \foreignlanguage{greek}{και αποκτινωϲιν} & 9 &  &  \\
& \textbf{5} &  & \foreignlanguage{greek}{ελεγον δε μη εν τη εορτη ινα μη θορυ} & 9 &  &  \\
[0.2em]
\cline{4-4}
\end{tabular}
\end{center}
\end{table}
}
\clearpage
\newpage
 {
 \setlength\arrayrulewidth{1pt}
\begin{table}
\begin{center}
\begin{tabular}{ccc|l|ccc}
\cline{4-4} \\ [-1em]
\multicolumn{7}{c}{\foreignlanguage{greek}{ευαγγελιον κατα μαθθαιον} \textbf{(\nospace{26:5})} } \\ \\ [-1em] % Si on veut ajouter les bordures latérales, remplacer {7}{c} par {7}{|c|}
\cline{4-4} \\
\cline{4-4}
&  &  & &  &  & \\ [-0.9em]
&  & 9 & \foreignlanguage{greek}{βοϲ γενηται εν τω λαω} & 13 &  &  \\
& \textbf{6} &  & \foreignlanguage{greek}{του δε \textoverline{ιυ} γενομενου εν βηθανια εν οι} & 8 &  &  \\
&  & 8 & \foreignlanguage{greek}{κεια ϲιμωνοϲ του λεπρου προϲηλθε̅} & 1 & \textbf{7} &  \\
&  & 2 & \foreignlanguage{greek}{αυτω γυνη αλαβαϲτρον μυρου εχουϲα} & 6 &  &  \\
&  & 7 & \foreignlanguage{greek}{βαρυτιμου και κατεχεεν επι την κε} & 12 &  &  \\
&  & 12 & \foreignlanguage{greek}{φαλην αυτου ανακειμενου} & 14 &  &  \\
& \textbf{8} &  & \foreignlanguage{greek}{ιδοντεϲ δε οι μαθηται αυτου ηγανα} & 6 &  &  \\
&  & 6 & \foreignlanguage{greek}{κτηϲαν λεγοντεϲ ειϲ τι η απωλια αυτη} & 12 &  &  \\
& \textbf{9} &  & \foreignlanguage{greek}{εδυνατο γαρ τουτο πραθηναι πολλου} & 5 &  &  \\
&  & 6 & \foreignlanguage{greek}{και δοθηναι πτωχοιϲ} & 8 &  &  \\
& \textbf{10} &  & \foreignlanguage{greek}{γνουϲ δε ο \textoverline{ιϲ} ειπεν αυτοιϲ τι κοπουϲ} & 8 &  &  \\
&  & 9 & \foreignlanguage{greek}{παρεχεται τη γυναικει εργον γαρ κα} & 14 &  &  \\
&  & 14 & \foreignlanguage{greek}{λον ηγαϲατο ειϲ εμε παντοτε γαρ} & 2 & \textbf{11} &  \\
&  & 3 & \foreignlanguage{greek}{τουϲ πτωχουϲ εχεται μεθ εαυτων ε} & 8 &  &  \\
&  & 8 & \foreignlanguage{greek}{με δε ου παντοτε εχεται} & 12 &  &  \\
& \textbf{12} &  & \foreignlanguage{greek}{βαλουϲα γαρ αυτη το μυρον τουτο ε} & 7 &  &  \\
&  & 7 & \foreignlanguage{greek}{πι του ϲωματοϲ μου προϲ το εντα} & 13 &  &  \\
&  & 13 & \foreignlanguage{greek}{φιαϲαι με εποιηϲεν αμην λεγω υμι̅} & 3 & \textbf{13} &  \\
&  & 4 & \foreignlanguage{greek}{οπου εαν κηρυχθη το ευαγγελιον του} & 9 &  &  \\
&  & 9 & \foreignlanguage{greek}{το εν ολω τω κοϲμω λαληθηϲεται} & 14 &  &  \\
&  & 15 & \foreignlanguage{greek}{και ο εποιηϲεν αυτη ειϲ μνημοϲυνο̅} & 20 &  &  \\
&  & 21 & \foreignlanguage{greek}{αυτηϲ τοτε πορευθειϲ ειϲ των} & 4 & \textbf{14} &  \\
&  & 5 & \foreignlanguage{greek}{δεκαδυο ο λεγομενοϲ ιουδαϲ ιϲκα} & 9 &  &  \\
&  & 9 & \foreignlanguage{greek}{ριωτηϲ προϲ τουϲ αρχιερειϲ ειπεν τι} & 2 & \textbf{15} &  \\
&  & 3 & \foreignlanguage{greek}{θελεται μοι δουναι και εγω υμιν παρα} & 9 &  &  \\
&  & 9 & \foreignlanguage{greek}{δω αυτον οι δε εϲτηϲαν αυτω τρι} & 15 &  &  \\
&  & 15 & \foreignlanguage{greek}{ακοντα αργυρια και απο τοτε εζητι} & 4 & \textbf{16} &  \\
&  & 5 & \foreignlanguage{greek}{ευκαιριαν ινα αυτον παραδω} & 8 &  &  \\
& \textbf{17} &  & \foreignlanguage{greek}{τη δε πρωτη των αζυμων προϲηλθον} & 6 &  &  \\
&  & 7 & \foreignlanguage{greek}{οι μαθηται λεγοντεϲ τω \textoverline{ιυ} που θελιϲ} & 13 &  &  \\
[0.2em]
\cline{4-4}
\end{tabular}
\end{center}
\end{table}
}
\clearpage
\newpage
 {
 \setlength\arrayrulewidth{1pt}
\begin{table}
\begin{center}
\begin{tabular}{ccc|l|ccc}
\cline{4-4} \\ [-1em]
\multicolumn{7}{c}{\foreignlanguage{greek}{ευαγγελιον κατα μαθθαιον} \textbf{(\nospace{26:17})} } \\ \\ [-1em] % Si on veut ajouter les bordures latérales, remplacer {7}{c} par {7}{|c|}
\cline{4-4} \\
\cline{4-4}
&  &  & &  &  & \\ [-0.9em]
&  & 14 & \foreignlanguage{greek}{απελθοντεϲ ετοιμαϲωμεν ϲοι φαγειν} & 17 &  &  \\
&  & 18 & \foreignlanguage{greek}{το παϲχα ο δε ειπεν υπαγεται ειϲ τη̅} & 6 & \textbf{18} &  \\
&  & 7 & \foreignlanguage{greek}{πολιν προϲ τον δινα και ειπατε αυτω} & 13 &  &  \\
&  & 14 & \foreignlanguage{greek}{ο διδαϲκαλοϲ λεγει ο καιροϲ μου εγγυϲ} & 20 &  &  \\
&  & 21 & \foreignlanguage{greek}{εϲτιν προϲ ϲε ποιω τα παϲχα μετα τω̅} & 28 &  &  \\
&  & 29 & \foreignlanguage{greek}{μαθητων μου εποιηϲαν ουν οι μα} & 4 & \textbf{19} &  \\
&  & 4 & \foreignlanguage{greek}{θηται ωϲ ϲυνεταξεν αυτοιϲ ο \textoverline{ιϲ} και η} & 11 &  &  \\
&  & 11 & \foreignlanguage{greek}{τοιμαϲαν το παϲχα} & 13 &  &  \\
& \textbf{20} &  & \foreignlanguage{greek}{οψειαϲ δε γενομενηϲ ανεκειτο μετα} & 5 &  &  \\
&  & 6 & \foreignlanguage{greek}{των δωδεκα μαθητων και εϲθιοντω̅} & 2 & \textbf{21} &  \\
&  & 3 & \foreignlanguage{greek}{αυτων ειπεν αμην λεγω υμιν οτι ειϲ} & 9 &  &  \\
&  & 10 & \foreignlanguage{greek}{εξ υμων παραδωϲει με} & 13 &  &  \\
& \textbf{22} &  & \foreignlanguage{greek}{και λυπουμενοι ϲφοδρα ηρξαντο λεγει̅} & 5 &  &  \\
&  & 6 & \foreignlanguage{greek}{αυτω εκαϲτοϲ αυτων μητι εγω ειμει \textoverline{κε}} & 12 &  &  \\
& \textbf{23} &  & \foreignlanguage{greek}{ο δε αποκριθειϲ ειπεν ο εμβαψαϲ με} & 7 &  &  \\
&  & 7 & \foreignlanguage{greek}{τ εμου εν τω τρυβλιω την χειρα εκει} & 14 &  &  \\
&  & 14 & \foreignlanguage{greek}{νοϲ με παραδωϲει ο μεν υιοϲ του} & 4 & \textbf{24} &  \\
&  & 5 & \foreignlanguage{greek}{\textoverline{ανου} υπαγει καθωϲ γεγραπται περι} & 9 &  &  \\
&  & 10 & \foreignlanguage{greek}{αυτου ουαι δε τω \textoverline{ανω} εκεινω δι ου} & 17 &  &  \\
&  & 18 & \foreignlanguage{greek}{ο υιοϲ του \textoverline{ανου} παραδιδοτε} & 22 &  &  \\
&  & 23 & \foreignlanguage{greek}{καλον ην αυτω ει ουκ εγεννηθη ο αν} & 30 &  &  \\
&  & 30 & \foreignlanguage{greek}{θρωποϲ εκεινοϲ} & 31 &  &  \\
& \textbf{25} &  & \foreignlanguage{greek}{αποκριθειϲ δε ιουδαϲ ο παραδιδουϲ αυτο̅} & 6 &  &  \\
&  & 7 & \foreignlanguage{greek}{ειπεν μητι εγω ειμει ραββει} & 11 &  &  \\
&  & 12 & \foreignlanguage{greek}{λεγει αυτω ϲυ ειπαϲ εϲθιοντων δε} & 2 & \textbf{26} &  \\
&  & 3 & \foreignlanguage{greek}{αυτων λαβων ο \textoverline{ιϲ} τον αρτον ευχαρι} & 9 &  &  \\
&  & 9 & \foreignlanguage{greek}{ϲτηϲαϲ εκλαϲεν και εδιδου τοιϲ μαθηταιϲ} & 14 &  &  \\
&  & 15 & \foreignlanguage{greek}{και ειπεν λαβετε φαγετε τουτο εϲτι̅} & 20 &  &  \\
&  & 21 & \foreignlanguage{greek}{το ϲωμα μου και λαβων ποτηριον και} & 4 & \textbf{27} &  \\
&  & 5 & \foreignlanguage{greek}{ευχαριϲτηϲαϲ εδωκεν αυτοιϲ λεγων} & 8 &  &  \\
[0.2em]
\cline{4-4}
\end{tabular}
\end{center}
\end{table}
}
\clearpage
\newpage
 {
 \setlength\arrayrulewidth{1pt}
\begin{table}
\begin{center}
\begin{tabular}{ccc|l|ccc}
\cline{4-4} \\ [-1em]
\multicolumn{7}{c}{\foreignlanguage{greek}{ευαγγελιον κατα μαθθαιον} \textbf{(\nospace{26:27})} } \\ \\ [-1em] % Si on veut ajouter les bordures latérales, remplacer {7}{c} par {7}{|c|}
\cline{4-4} \\
\cline{4-4}
&  &  & &  &  & \\ [-0.9em]
&  & 9 & \foreignlanguage{greek}{πιεται εξ αυτου παντεϲ τουτο γαρ εϲτι̅} & 3 & \textbf{28} &  \\
&  & 4 & \foreignlanguage{greek}{το αιμα μου το τηϲ καινηϲ διαθηκηϲ} & 10 &  &  \\
&  & 11 & \foreignlanguage{greek}{το περι πολλων εκχυνομενον ειϲ αφε} & 16 &  &  \\
&  & 16 & \foreignlanguage{greek}{ϲιν αμαρτιων λεγω δε υμιν} & 3 & \textbf{29} &  \\
&  & 4 & \foreignlanguage{greek}{οτι ου μη πιω απ αρτι εκ τουτου του γε} & 13 &  &  \\
&  & 13 & \foreignlanguage{greek}{νηματοϲ τηϲ αμπελου εωϲ τηϲ ημε} & 18 &  &  \\
&  & 18 & \foreignlanguage{greek}{ραϲ εκεινηϲ οταν αυτο πινω μεθ υ} & 24 &  &  \\
&  & 24 & \foreignlanguage{greek}{μων καινον εν τη βαϲιλεια του \textoverline{πρϲ} μου} & 31 &  &  \\
& \textbf{30} &  & \foreignlanguage{greek}{και υμνηϲαντεϲ εξηλθον ειϲ το οροϲ} & 6 &  &  \\
&  & 7 & \foreignlanguage{greek}{των ελεων} & 8 &  &  \\
& \textbf{31} &  & \foreignlanguage{greek}{τοτε λεγει αυτοιϲ ο \textoverline{ιϲ} παντεϲ υμειϲ ϲκα̅} & 8 &  &  \\
&  & 8 & \foreignlanguage{greek}{δαλιϲθηϲεϲθαι εν εμοι εν τη νυκτι} & 13 &  &  \\
&  & 14 & \foreignlanguage{greek}{ταυτη γεγραπται γαρ παταξω τον} & 18 &  &  \\
&  & 19 & \foreignlanguage{greek}{ποιμενα και διαϲκορπιϲθηϲονται τα} & 22 &  &  \\
&  & 23 & \foreignlanguage{greek}{προβατα τηϲ ποιμνηϲ} & 25 &  &  \\
& \textbf{32} &  & \foreignlanguage{greek}{μετα δε το εγερθηναι με προαξω υμαϲ} & 7 &  &  \\
&  & 8 & \foreignlanguage{greek}{ειϲ την γαλιλαιαν αποκριθειϲ δε} & 2 & \textbf{33} &  \\
&  & 3 & \foreignlanguage{greek}{ο πετροϲ ειπεν αυτω ει και παντεϲ} & 9 &  &  \\
&  & 10 & \foreignlanguage{greek}{ϲκανδαλιϲθηϲονται εν ϲοι εγω ουδε} & 14 &  &  \\
&  & 14 & \foreignlanguage{greek}{ποτε ϲκανδαλιϲθηϲομαι} & 15 &  &  \\
& \textbf{34} &  & \foreignlanguage{greek}{εφη αυτω ο \textoverline{ιϲ} αμην λεγω ϲοι οτι εν} & 9 &  &  \\
&  & 10 & \foreignlanguage{greek}{ταυτη τη νυκτι πριν αλεκτορα φωνη} & 15 &  &  \\
&  & 15 & \foreignlanguage{greek}{ϲαι τριϲ απαρνηϲη με} & 18 &  &  \\
& \textbf{35} &  & \foreignlanguage{greek}{λεγει αυτω ο πετροϲ καν δεη με ϲυ̅} & 8 &  &  \\
&  & 9 & \foreignlanguage{greek}{ϲοι αποθανειν ου μη ϲε απαρνηϲομε} & 14 &  &  \\
&  & 15 & \foreignlanguage{greek}{ομοιωϲ δε και παντεϲ οι μαθηται ειπο̅} & 21 &  &  \\
& \textbf{36} &  & \foreignlanguage{greek}{τοτε ερχεται ο \textoverline{ιϲ} μετ αυτων ειϲ χωρι} & 8 &  &  \\
&  & 8 & \foreignlanguage{greek}{ον λεγομενον γεδϲημανι} & 10 &  &  \\
&  & 11 & \foreignlanguage{greek}{και λεγει τοιϲ μαθηταιϲ αυτου καθει} & 16 &  &  \\
&  & 16 & \foreignlanguage{greek}{ϲατε αυτου εωϲ αν απελθων προϲευ} & 21 &  &  \\
[0.2em]
\cline{4-4}
\end{tabular}
\end{center}
\end{table}
}
\clearpage
\newpage
 {
 \setlength\arrayrulewidth{1pt}
\begin{table}
\begin{center}
\begin{tabular}{ccc|l|ccc}
\cline{4-4} \\ [-1em]
\multicolumn{7}{c}{\foreignlanguage{greek}{ευαγγελιον κατα μαθθαιον} \textbf{(\nospace{26:36})} } \\ \\ [-1em] % Si on veut ajouter les bordures latérales, remplacer {7}{c} par {7}{|c|}
\cline{4-4} \\
\cline{4-4}
&  &  & &  &  & \\ [-0.9em]
&  & 21 & \foreignlanguage{greek}{ξωμαι εκει και παραλαβων τον πετρον} & 4 & \textbf{37} &  \\
&  & 5 & \foreignlanguage{greek}{και τουϲ δυο υιουϲ ζεβαιδεου ηρξατο λυπι} & 11 &  &  \\
&  & 11 & \foreignlanguage{greek}{ϲθαι και αδημονειν τοτε λεγει αυτοιϲ} & 3 & \textbf{38} &  \\
&  & 4 & \foreignlanguage{greek}{περιλυποϲ εϲτιν η ψυχη μου εωϲ θανατου} & 10 &  &  \\
&  & 11 & \foreignlanguage{greek}{μεινατε ωδε και γρηγορειτε μετ εμου} & 16 &  &  \\
& \textbf{39} &  & \foreignlanguage{greek}{και προϲελθων μικρον επεϲεν επι προϲωπον} & 6 &  &  \\
&  & 7 & \foreignlanguage{greek}{αυτου προϲευχομενοϲ και λεγων} & 10 &  &  \\
&  & 11 & \foreignlanguage{greek}{πατερ μου ει δυνατον εϲτιν παρελθατω} & 16 &  &  \\
&  & 17 & \foreignlanguage{greek}{απ εμου το ποτηριον τουτο πλην ουχ} & 23 &  &  \\
&  & 24 & \foreignlanguage{greek}{ωϲ εγω θελω αλλ ωϲ ϲυ και ερχετε προϲ} & 3 & \textbf{40} &  \\
&  & 4 & \foreignlanguage{greek}{τουϲ μαθηταϲ και ευριϲκει αυτουϲ καθευ} & 9 &  &  \\
&  & 9 & \foreignlanguage{greek}{δονταϲ και λεγει τω πετρω ουτωϲ ουκ} & 15 &  &  \\
&  & 16 & \foreignlanguage{greek}{καθευδονταϲ μιαν ωραν γρηγορηϲαι μετ εμου} & 21 &  &  \\
& \textbf{41} &  & \foreignlanguage{greek}{γρηγοριτε και προϲευχεϲθαι ινα μη ειϲελ} & 6 &  &  \\
&  & 6 & \foreignlanguage{greek}{θητε ειϲ πειραϲμον το μεν \textoverline{πνα} προ} & 12 &  &  \\
&  & 12 & \foreignlanguage{greek}{θυμον η δε ϲαρξ αϲθενηϲ} & 16 &  &  \\
& \textbf{42} &  & \foreignlanguage{greek}{παλιν εκ δευτερου απελθων προϲηυξατο} & 5 &  &  \\
&  & 6 & \foreignlanguage{greek}{λεγων πατερ μου ει ου δυναται τουτο} & 12 &  &  \\
&  & 13 & \foreignlanguage{greek}{παρελθειν απ εμου εαν μη αυτο πιω} & 19 &  &  \\
&  & 20 & \foreignlanguage{greek}{γενηθητω το θελημα ϲου και ελθων} & 2 & \textbf{43} &  \\
&  & 3 & \foreignlanguage{greek}{ευρεν αυτουϲ παλιν καθευδονταϲ ηϲα̅} & 7 &  &  \\
&  & 8 & \foreignlanguage{greek}{γαρ αυτων οι οφθαλμοι βεβαρημενοι} & 12 &  &  \\
& \textbf{44} &  & \foreignlanguage{greek}{και αφειϲ αυτουϲ απελθων προϲηυξα} & 5 &  &  \\
&  & 5 & \foreignlanguage{greek}{το παλιν εκ τριτου τον αυτον λογον ειπω̅} & 12 &  &  \\
& \textbf{45} &  & \foreignlanguage{greek}{τοτε ερχεται προϲ τουϲ μαθηταϲ αυτου} & 6 &  &  \\
&  & 7 & \foreignlanguage{greek}{και λεγει αυτοιϲ καθευδεται λοιπον και} & 12 &  &  \\
&  & 13 & \foreignlanguage{greek}{αναπαυεϲθαι ιδου ηγγικεν η ωρα και} & 18 &  &  \\
&  & 19 & \foreignlanguage{greek}{ο υιοϲ του \textoverline{ανου} παραδιδοτε ειϲ χειραϲ α} & 26 &  &  \\
&  & 26 & \foreignlanguage{greek}{μαρτωλων εγειρεϲθαι αγωμεν ιδου} & 3 & \textbf{46} &  \\
&  & 4 & \foreignlanguage{greek}{ηγγεικεν ο παραδιδουϲ με} & 7 &  &  \\
[0.2em]
\cline{4-4}
\end{tabular}
\end{center}
\end{table}
}
\clearpage
\newpage
 {
 \setlength\arrayrulewidth{1pt}
\begin{table}
\begin{center}
\begin{tabular}{ccc|l|ccc}
\cline{4-4} \\ [-1em]
\multicolumn{7}{c}{\foreignlanguage{greek}{ευαγγελιον κατα μαθθαιον} \textbf{(\nospace{26:47})} } \\ \\ [-1em] % Si on veut ajouter les bordures latérales, remplacer {7}{c} par {7}{|c|}
\cline{4-4} \\
\cline{4-4}
&  &  & &  &  & \\ [-0.9em]
& \textbf{47} &  & \foreignlanguage{greek}{και ετι αυτου λαλουντοϲ ιδου ιουδαϲ ειϲ} & 7 &  &  \\
&  & 8 & \foreignlanguage{greek}{των δωδεκα ηλθεν και μετ αυτου οχλοϲ} & 14 &  &  \\
&  & 15 & \foreignlanguage{greek}{πολυϲ μετα μαχερων και ξυλων απο των} & 21 &  &  \\
&  & 22 & \foreignlanguage{greek}{αρχιερεων και πρεϲβυτερων του λαου} & 26 &  &  \\
& \textbf{48} &  & \foreignlanguage{greek}{ο δε παραδιδουϲ αυτον εδωκεν αυτοιϲ ϲη} & 7 &  &  \\
&  & 7 & \foreignlanguage{greek}{μιον λεγων ον εαν φιληϲω αυτοϲ εϲτιν} & 13 &  &  \\
&  & 14 & \foreignlanguage{greek}{κρατηϲατε αυτον και ευθεωϲ προϲηλ} & 3 & \textbf{49} &  \\
&  & 3 & \foreignlanguage{greek}{θεν τω \textoverline{ιυ} και ειπεν χαιρε ραββει και κα} & 11 &  &  \\
&  & 11 & \foreignlanguage{greek}{τεφιληϲεν αυτον ο δε \textoverline{ιϲ} ειπεν αυτω} & 5 & \textbf{50} &  \\
&  & 6 & \foreignlanguage{greek}{ετερε εφ ο παρει} & 9 &  &  \\
&  & 10 & \foreignlanguage{greek}{τοτε προϲελθοντεϲ επεβαλον ταϲ χειραϲ} & 14 &  &  \\
&  & 15 & \foreignlanguage{greek}{επι τον \textoverline{ιν} και εκρατηϲαν αυτον} & 20 &  &  \\
& \textbf{51} &  & \foreignlanguage{greek}{και ιδου ειϲ των μετα \textoverline{ιυ} εκτειναϲ την} & 8 &  &  \\
&  & 9 & \foreignlanguage{greek}{χειρα απεϲπαϲεν την μαχαιραν αυτου ϗ} & 14 &  &  \\
&  & 15 & \foreignlanguage{greek}{παταξαϲ τον δουλον του αρχιερεωϲ αφι} & 20 &  &  \\
&  & 20 & \foreignlanguage{greek}{λεν αυτου το ωτιον} & 23 &  &  \\
& \textbf{52} &  & \foreignlanguage{greek}{τοτε λεγει αυτω ο \textoverline{ιϲ} αποϲτρεψον ϲου την μα} & 9 &  &  \\
&  & 9 & \foreignlanguage{greek}{χαιραν ειϲ τον τοπον αυτηϲ παντεϲ γαρ} & 15 &  &  \\
&  & 16 & \foreignlanguage{greek}{οι λαβοντεϲ μαχαιραν εν μαχαιρα αποθα} & 21 &  &  \\
&  & 21 & \foreignlanguage{greek}{νουνται η δοκειϲ οτι ου δυναμαι αρτι} & 6 & \textbf{53} &  \\
&  & 7 & \foreignlanguage{greek}{παρακαλεϲαι τον \textoverline{πρα} μου και παραϲτηϲι μοι} & 13 &  &  \\
&  & 14 & \foreignlanguage{greek}{πλιουϲ η δωδεκα λεγεωναϲ αγγελων} & 19 &  &  \\
& \textbf{54} &  & \foreignlanguage{greek}{πωϲ ουν πληρωθωϲιν αι γραφαι οτι ουτωϲ} & 7 &  &  \\
&  & 8 & \foreignlanguage{greek}{δει γενεϲθαι εν εκεινη τη ωρα ει} & 5 & \textbf{55} &  \\
&  & 5 & \foreignlanguage{greek}{πεν ο \textoverline{ιϲ} τοιϲ οχλοιϲ ωϲ επι ληϲτην εξηλ} & 13 &  &  \\
&  & 13 & \foreignlanguage{greek}{θατε μετα μαχαιρων και ξυλων ϲυλλαβει̅} & 18 &  &  \\
&  & 19 & \foreignlanguage{greek}{με καθ ημεραν προϲ υμαϲ εκαθεζομη̅} & 24 &  &  \\
&  & 25 & \foreignlanguage{greek}{διδαϲκων εν τω ιερω και ουκ εκρατηϲα} & 31 &  &  \\
&  & 31 & \foreignlanguage{greek}{τε με τουτο δε ολον γεγονεν ινα πλη} & 6 & \textbf{56} &  \\
&  & 6 & \foreignlanguage{greek}{ρωθωϲιν αι γραφαι των προφητων} & 10 &  &  \\
[0.2em]
\cline{4-4}
\end{tabular}
\end{center}
\end{table}
}
\clearpage
\newpage
 {
 \setlength\arrayrulewidth{1pt}
\begin{table}
\begin{center}
\begin{tabular}{ccc|l|ccc}
\cline{4-4} \\ [-1em]
\multicolumn{7}{c}{\foreignlanguage{greek}{ευαγγελιον κατα μαθθαιον} \textbf{(\nospace{26:56})} } \\ \\ [-1em] % Si on veut ajouter les bordures latérales, remplacer {7}{c} par {7}{|c|}
\cline{4-4} \\
\cline{4-4}
&  &  & &  &  & \\ [-0.9em]
&  & 11 & \foreignlanguage{greek}{τοτε οι μαθητε παντεϲ αφεντεϲ αυτον ε} & 17 &  &  \\
&  & 17 & \foreignlanguage{greek}{φυγον οι δε κρατηϲαντεϲ τον \textoverline{ιν} απηγαγο̅} & 6 & \textbf{57} &  \\
&  & 7 & \foreignlanguage{greek}{προϲ καιαφαν τον αρχιερεα οπου οι γραμ} & 13 &  &  \\
&  & 13 & \foreignlanguage{greek}{ματειϲ και οι πρεϲβυτεροι ϲυνηχθηϲαν} & 17 &  &  \\
& \textbf{58} &  & \foreignlanguage{greek}{ο δε πετροϲ ηκολουθει αυτω απο μακροθε̅} & 7 &  &  \\
&  & 8 & \foreignlanguage{greek}{εωϲ τηϲ αυληϲ του αρχιερεωϲ και ειϲελ} & 14 &  &  \\
&  & 14 & \foreignlanguage{greek}{θων εϲω εκαθητο μετα των υπηρετων} & 19 &  &  \\
&  & 20 & \foreignlanguage{greek}{ιδειν το τελοϲ οι δε αρχιερειϲ και} & 4 & \textbf{59} &  \\
&  & 5 & \foreignlanguage{greek}{οι πρεϲβυτεροι και το ϲυνεδριον ολον εζη} & 11 &  &  \\
&  & 11 & \foreignlanguage{greek}{τουν ψευδομαρτυριαν κατα του \textoverline{ιυ} οπωϲ} & 16 &  &  \\
&  & 17 & \foreignlanguage{greek}{θανατωϲουϲιν αυτον και ουχ ευρον} & 3 & \textbf{60} &  \\
&  & 4 & \foreignlanguage{greek}{και πολλων ψευδομαρτυρων προϲελθοντω̅} & 7 &  &  \\
&  & 8 & \foreignlanguage{greek}{ουχ ευρον υϲτερον δε προϲελθοντεϲ} & 12 &  &  \\
&  & 13 & \foreignlanguage{greek}{δυο τινεϲ ψευδομαρτυρεϲ ειπον} & 1 & \textbf{61} &  \\
&  & 2 & \foreignlanguage{greek}{ουτοϲ εφη δυναμαι καταλυϲαι τον ναο̅} & 7 &  &  \\
&  & 8 & \foreignlanguage{greek}{του \textoverline{θυ} και δια τριων ημερων οικοδομηϲαι} & 14 &  &  \\
&  & 15 & \foreignlanguage{greek}{αυτον και αναϲταϲ ο αρχιερευϲ ειπεν} & 5 & \textbf{62} &  \\
&  & 6 & \foreignlanguage{greek}{αυτω ουδεν αποκρινη τι ουτοι ϲου} & 11 &  &  \\
&  & 12 & \foreignlanguage{greek}{καταμαρτυρουϲιν ο δε \textoverline{ιϲ} εϲιωπα} & 4 & \textbf{63} &  \\
&  & 5 & \foreignlanguage{greek}{και αποκριθειϲ ο αρχιερευϲ ειπεν αυτω} & 10 &  &  \\
&  & 11 & \foreignlanguage{greek}{εξορκιζω ϲε κατα του \textoverline{θυ} του ζωντοϲ} & 17 &  &  \\
&  & 18 & \foreignlanguage{greek}{ινα ημιν ειπηϲ ει ϲυ ει ο \textoverline{χϲ} ο υιοϲ του \textoverline{θυ}} & 29 &  &  \\
&  & 30 & \foreignlanguage{greek}{του ζωντοϲ λεγει αυτω ο \textoverline{ιϲ} ϲυ ειπαϲ} & 6 & \textbf{64} &  \\
&  & 7 & \foreignlanguage{greek}{πλην λεγω υμιν απ αρτι οψεϲθαι τον υ} & 14 &  &  \\
&  & 14 & \foreignlanguage{greek}{ιον του \textoverline{ανου} καθημενον εκ δεξιων τηϲ} & 20 &  &  \\
&  & 21 & \foreignlanguage{greek}{δυναμεωϲ και ερχομενον επι των νε} & 26 &  &  \\
&  & 26 & \foreignlanguage{greek}{φελων του ουρανου τοτε ο αρχιερευϲ} & 3 & \textbf{65} &  \\
&  & 4 & \foreignlanguage{greek}{διερηξεν τα ιματια αυτου λεγων οτι ε} & 10 &  &  \\
&  & 10 & \foreignlanguage{greek}{βλαϲφημηϲεν τι ετι χριαν εχομεν} & 14 &  &  \\
&  & 15 & \foreignlanguage{greek}{μαρτυρων ειδε νυν ηκουϲατε την βλα} & 20 &  &  \\
[0.2em]
\cline{4-4}
\end{tabular}
\end{center}
\end{table}
}
\clearpage
\newpage
 {
 \setlength\arrayrulewidth{1pt}
\begin{table}
\begin{center}
\begin{tabular}{ccc|l|ccc}
\cline{4-4} \\ [-1em]
\multicolumn{7}{c}{\foreignlanguage{greek}{ευαγγελιον κατα μαθθαιον} \textbf{(\nospace{26:65})} } \\ \\ [-1em] % Si on veut ajouter les bordures latérales, remplacer {7}{c} par {7}{|c|}
\cline{4-4} \\
\cline{4-4}
&  &  & &  &  & \\ [-0.9em]
&  & 20 & \foreignlanguage{greek}{ϲφημιαν αυτου τι υμιν δοκει οι δε απο} & 6 & \textbf{66} &  \\
&  & 6 & \foreignlanguage{greek}{κριθεντεϲ ειπον ενοχοϲ θανατου εϲτιν} & 10 &  &  \\
& \textbf{67} &  & \foreignlanguage{greek}{τοτε ενεπτυϲαν ειϲ το προϲωπον αυτου} & 6 &  &  \\
&  & 7 & \foreignlanguage{greek}{και εκολαφιϲαν αυτον οι δε εριπιϲαν λε} & 1 & \textbf{68} &  \\
&  & 1 & \foreignlanguage{greek}{γοντεϲ προφητευϲον ημιν \textoverline{χε} τιϲ εϲτιν} & 6 &  &  \\
&  & 7 & \foreignlanguage{greek}{ο πεϲαϲ ϲε ο δε πετροϲ εξω εκαθητο ε̅} & 6 & \textbf{69} &  \\
&  & 7 & \foreignlanguage{greek}{τη αυλη και προϲηλθεν αυτω μια παι} & 13 &  &  \\
&  & 13 & \foreignlanguage{greek}{διϲκη λεγουϲα και ϲυ ηϲθα μετα \textoverline{ιυ} του} & 20 &  &  \\
&  & 21 & \foreignlanguage{greek}{γαλιλαιου ο δε ηρνηϲατο εμπροϲθεν αυ} & 5 & \textbf{70} &  \\
&  & 5 & \foreignlanguage{greek}{των παντων λεγων ουκ οιδα τι λεγειϲ} & 11 &  &  \\
& \textbf{71} &  & \foreignlanguage{greek}{εξελθοντα δε αυτον ειϲ τον πυλωνα} & 6 &  &  \\
&  & 7 & \foreignlanguage{greek}{ιδεν αυτον αλλη και λεγει τοιϲ εκει και} & 14 &  &  \\
&  & 15 & \foreignlanguage{greek}{ουτοϲ ην μετα \textoverline{ιυ} του ναζωραιου και πα} & 2 & \textbf{72} &  \\
&  & 2 & \foreignlanguage{greek}{λιν ηρνηϲατο μετα ρ ορκου οτι ουκ οιδα} & 9 &  &  \\
&  & 10 & \foreignlanguage{greek}{τον ανθρωπον} & 11 &  &  \\
& \textbf{73} &  & \foreignlanguage{greek}{μετα μικρον δε προϲελθοντεϲ οι εϲτω} & 6 &  &  \\
&  & 6 & \foreignlanguage{greek}{τεϲ ειπον τω πετρω αληθωϲ και ϲυ εξ} & 13 &  &  \\
&  & 14 & \foreignlanguage{greek}{αυτων ει και γαρ η λαλια ϲου δηλον ϲε} & 22 &  &  \\
&  & 23 & \foreignlanguage{greek}{ποιει τοτε ηρξατο καταθεματιζει̅} & 3 & \textbf{74} &  \\
&  & 4 & \foreignlanguage{greek}{και ομνυειν οτι ουκ οιδα τον \textoverline{ανον} και} & 11 &  &  \\
&  & 12 & \foreignlanguage{greek}{ευθεωϲ αλεκτωρ εφωνηϲεν} & 14 &  &  \\
& \textbf{75} &  & \foreignlanguage{greek}{και εμνηϲθη ο πετροϲ του ρηματοϲ του \textoverline{ιυ}} & 8 &  &  \\
&  & 9 & \foreignlanguage{greek}{ειρηκοτοϲ αυτω οτι πριν αλεκτορα φωνη} & 14 &  &  \\
&  & 14 & \foreignlanguage{greek}{ϲε τριϲ απαρνηϲη με και εξελθων εξω} & 20 &  &  \\
&  & 21 & \foreignlanguage{greek}{εκλαυϲεν πικρωϲ} & 22 &  &  \\
& \mygospelchapter &  & \foreignlanguage{greek}{πρωειαϲ δε γενομενηϲ ϲυμβουλιον ελα} & 5 &  &  \\
&  & 5 & \foreignlanguage{greek}{βον παντεϲ οι αρχιερειϲ και οι πρεϲβυτε} & 11 &  &  \\
&  & 11 & \foreignlanguage{greek}{ροι του λαου κατα του \textoverline{ιυ} ωϲτε θανατωϲαι} & 18 &  &  \\
&  & 19 & \foreignlanguage{greek}{αυτον και δηϲαντεϲ αυτον απηγαγον} & 4 & \textbf{2} &  \\
&  & 5 & \foreignlanguage{greek}{και παρεδωκαν αυτον ποντιω πιλατω τω η} & 11 &  &  \\
&  & 11 & \foreignlanguage{greek}{γεμονει} & 11 &  &  \\
[0.2em]
\cline{4-4}
\end{tabular}
\end{center}
\end{table}
}
\clearpage
\newpage
 {
 \setlength\arrayrulewidth{1pt}
\begin{table}
\begin{center}
\begin{tabular}{ccc|l|ccc}
\cline{4-4} \\ [-1em]
\multicolumn{7}{c}{\foreignlanguage{greek}{ευαγγελιον κατα μαθθαιον} \textbf{(\nospace{27:3})} } \\ \\ [-1em] % Si on veut ajouter les bordures latérales, remplacer {7}{c} par {7}{|c|}
\cline{4-4} \\
\cline{4-4}
&  &  & &  &  & \\ [-0.9em]
& \textbf{3} &  & \foreignlanguage{greek}{τοτε ιδων ιουδαϲ ο παραδιδουϲ αυτον οτι} & 7 &  &  \\
&  & 8 & \foreignlanguage{greek}{κατεκριθη μεταμεληθειϲ απεϲτρεψεν} & 10 &  &  \\
&  & 11 & \foreignlanguage{greek}{τα τριακοντα αργυρια τοιϲ αρχιερευϲιν ϗ} & 16 &  &  \\
&  & 17 & \foreignlanguage{greek}{τοιϲ πρεϲβυτεροιϲ λεγων ημαρτον παρα} & 3 & \textbf{4} &  \\
&  & 3 & \foreignlanguage{greek}{δουϲ αιμα αθωον οι δε ειπον τι προϲ ημαϲ} & 12 &  &  \\
&  & 13 & \foreignlanguage{greek}{ϲυ οψη και ριψαϲ τα αργυρια εν τω ναω} & 7 & \textbf{5} &  \\
&  & 8 & \foreignlanguage{greek}{ανεχωρηϲεν και απελθων απηγξατο} & 11 &  &  \\
& \textbf{6} &  & \foreignlanguage{greek}{οι δε αρχιερειϲ λαβοντεϲ τα αργυρια ειπο̅} & 7 &  &  \\
&  & 8 & \foreignlanguage{greek}{ουκ εϲτιν βαλιν αυτα ειϲ τον κορβαναν} & 14 &  &  \\
&  & 15 & \foreignlanguage{greek}{επι τιμη αιματοϲ εϲτιν} & 18 &  &  \\
& \textbf{7} &  & \foreignlanguage{greek}{ϲυμβουλιον δε λαβοντεϲ ηγοραϲαν εξ} & 5 &  &  \\
&  & 6 & \foreignlanguage{greek}{αυτων τον αγρον του κεραμεωϲ ειϲ τα} & 12 &  &  \\
&  & 12 & \foreignlanguage{greek}{φην τοιϲ ξενοιϲ διο εκληθη ο αγροϲ ε} & 5 & \textbf{8} &  \\
&  & 5 & \foreignlanguage{greek}{κεινοϲ αγροϲ αιματοϲ εωϲ τηϲ ϲημερον} & 10 &  &  \\
& \textbf{9} &  & \foreignlanguage{greek}{τοτε επληρωθη το ρηθεν δια ιηρεμιου του} & 7 &  &  \\
&  & 8 & \foreignlanguage{greek}{προφητου λεγοντοϲ και ελαβον τα τρι} & 13 &  &  \\
&  & 13 & \foreignlanguage{greek}{ακοντα αργυρια την τιμην του τετιμη} & 18 &  &  \\
&  & 18 & \foreignlanguage{greek}{μενου ον ετιμηϲαντο απο υιων ιϲραηλ} & 23 &  &  \\
& \textbf{10} &  & \foreignlanguage{greek}{και εδωκα αυτα ειϲ τον αγρον του κερα} & 8 &  &  \\
&  & 8 & \foreignlanguage{greek}{μεωϲ καθα ϲυνεταξεν μοι \textoverline{κϲ}} & 12 &  &  \\
& \textbf{11} &  & \foreignlanguage{greek}{ο δε \textoverline{ιϲ} εϲτη εμπροϲθεν του ηγεμονοϲ και} & 8 &  &  \\
&  & 9 & \foreignlanguage{greek}{επηρωτηϲεν αυτον λεγων ϲυ ει ο βαϲι} & 15 &  &  \\
&  & 15 & \foreignlanguage{greek}{λευϲ των ιουδαιων ο δε \textoverline{ιϲ} εφη αυτω} & 22 &  &  \\
&  & 23 & \foreignlanguage{greek}{ϲυ λεγειϲ και εν τω κατηγοριϲθαι αυτο̅} & 5 & \textbf{12} &  \\
&  & 6 & \foreignlanguage{greek}{υπο των αρχιερεων και των πρεϲβυτε} & 11 &  &  \\
&  & 11 & \foreignlanguage{greek}{ρων ουδεν απεκρινατο} & 13 &  &  \\
& \textbf{13} &  & \foreignlanguage{greek}{τοτε λεγει αυτω ο πιλατοϲ ουκ ακουειϲ} & 7 &  &  \\
&  & 8 & \foreignlanguage{greek}{ποϲα ϲου καταμαρτυρουϲιν και ουκ α} & 3 & \textbf{14} &  \\
&  & 3 & \foreignlanguage{greek}{πεκριθη αυτω προϲ ουδε εν ρημα ωϲτε} & 9 &  &  \\
&  & 10 & \foreignlanguage{greek}{θαυμαζειν τον ηγεμονα λιαν} & 13 &  &  \\
[0.2em]
\cline{4-4}
\end{tabular}
\end{center}
\end{table}
}
\clearpage
\newpage
 {
 \setlength\arrayrulewidth{1pt}
\begin{table}
\begin{center}
\begin{tabular}{ccc|l|ccc}
\cline{4-4} \\ [-1em]
\multicolumn{7}{c}{\foreignlanguage{greek}{ευαγγελιον κατα μαθθαιον} \textbf{(\nospace{27:15})} } \\ \\ [-1em] % Si on veut ajouter les bordures latérales, remplacer {7}{c} par {7}{|c|}
\cline{4-4} \\
\cline{4-4}
&  &  & &  &  & \\ [-0.9em]
& \textbf{15} &  & \foreignlanguage{greek}{κατα δε εορτην ειωθει ο ηγεμων απολυ} & 7 &  &  \\
&  & 7 & \foreignlanguage{greek}{ειν ενα τω οχλω δεϲμιον ον ηθελον} & 13 &  &  \\
& \textbf{16} &  & \foreignlanguage{greek}{ειχον δε τοτε δεϲμιον επιϲημον λεγο} & 6 &  &  \\
&  & 6 & \foreignlanguage{greek}{μενον βαραββαν ϲυνηγμενων ουν} & 2 & \textbf{17} &  \\
&  & 3 & \foreignlanguage{greek}{αυτων ειπεν αυτοιϲ ο πιλατοϲ τινα θε} & 9 &  &  \\
&  & 9 & \foreignlanguage{greek}{λεται απολυϲω υμιν βαραββαν η \textoverline{ιν} το̅} & 15 &  &  \\
&  & 16 & \foreignlanguage{greek}{λεγομενον \textoverline{χν} ηδει γαρ οτι δια φθονον} & 5 & \textbf{18} &  \\
&  & 6 & \foreignlanguage{greek}{παρεδωκαν αυτον} & 7 &  &  \\
& \textbf{19} &  & \foreignlanguage{greek}{καθημενου δε αυτου επι του βηματοϲ απε} & 7 &  &  \\
&  & 7 & \foreignlanguage{greek}{ϲτιλεν προϲ αυτον η γυνη αυτου λεγουϲα} & 13 &  &  \\
&  & 14 & \foreignlanguage{greek}{μηδεν ϲοι και τω δικαιω εκεινω πολλα} & 20 &  &  \\
&  & 21 & \foreignlanguage{greek}{γαρ επαθον ϲημερον κατ οναρ δι αυτον} & 27 &  &  \\
& \textbf{20} &  & \foreignlanguage{greek}{οι δε αρχιερειϲ και οι πρεϲβυτεροι επιϲα̅} & 7 &  &  \\
&  & 8 & \foreignlanguage{greek}{τουϲ οχλουϲ ινα ετηϲωνται τον βαραβ} & 13 &  &  \\
&  & 13 & \foreignlanguage{greek}{βαν τον δε \textoverline{ιν} απολεϲωϲιν} & 17 &  &  \\
& \textbf{21} &  & \foreignlanguage{greek}{αποκριθειϲ δε ο ηγεμων ειπεν αυτοιϲ} & 6 &  &  \\
&  & 7 & \foreignlanguage{greek}{τινα θελεται απο των δυο απολυϲω υμι̅} & 13 &  &  \\
&  & 14 & \foreignlanguage{greek}{οι δε ειπον βαραββαν λεγει αυτοιϲ ο} & 3 & \textbf{22} &  \\
&  & 4 & \foreignlanguage{greek}{πιλατοϲ τι ουν ποιηϲω \textoverline{ιν} τον λεγομενον \textoverline{χν}} & 11 &  &  \\
&  & 12 & \foreignlanguage{greek}{λεγουϲιν παντεϲ ϲταυρωθητω ο δε η} & 3 & \textbf{23} &  \\
&  & 3 & \foreignlanguage{greek}{γεμων εφη τι γαρ κακον εποιηϲεν} & 8 &  &  \\
&  & 9 & \foreignlanguage{greek}{οι δε περιϲϲωϲ εκραζον λεγοντεϲ ϲταυ} & 14 &  &  \\
&  & 14 & \foreignlanguage{greek}{ρωθητω ιδων δε ο πιλατοϲ οτι ουδε̅} & 6 & \textbf{24} &  \\
&  & 7 & \foreignlanguage{greek}{ωφελει αλλα μαλλον θορυβοϲ γινεται} & 11 &  &  \\
&  & 12 & \foreignlanguage{greek}{λαβων υδωρ απενιψατο ταϲ χειραϲ απε} & 17 &  &  \\
&  & 17 & \foreignlanguage{greek}{ναντι του οχλου λεγων αθωοϲ ειμει} & 22 &  &  \\
&  & 23 & \foreignlanguage{greek}{απο του αιματοϲ του δικαιου τουτου} & 28 &  &  \\
&  & 29 & \foreignlanguage{greek}{υμειϲ οψεϲθαι και αποκριθειϲ παϲ ο} & 4 & \textbf{25} &  \\
&  & 5 & \foreignlanguage{greek}{λαοϲ ειπεν το αιμα αυτου εφ ημαϲ και} & 12 &  &  \\
&  & 13 & \foreignlanguage{greek}{επι τα τεκνα ημων τοτε απελυϲε̅} & 2 & \textbf{26} &  \\
[0.2em]
\cline{4-4}
\end{tabular}
\end{center}
\end{table}
}
\clearpage
\newpage
 {
 \setlength\arrayrulewidth{1pt}
\begin{table}
\begin{center}
\begin{tabular}{ccc|l|ccc}
\cline{4-4} \\ [-1em]
\multicolumn{7}{c}{\foreignlanguage{greek}{ευαγγελιον κατα μαθθαιον} \textbf{(\nospace{27:26})} } \\ \\ [-1em] % Si on veut ajouter les bordures latérales, remplacer {7}{c} par {7}{|c|}
\cline{4-4} \\
\cline{4-4}
&  &  & &  &  & \\ [-0.9em]
&  & 3 & \foreignlanguage{greek}{αυτοιϲ τον βαραββαν τον δε \textoverline{ιν} φραγελ} & 9 &  &  \\
&  & 9 & \foreignlanguage{greek}{λωϲαϲ παρεδωκεν ινα ϲταυρωθη} & 12 &  &  \\
& \textbf{27} &  & \foreignlanguage{greek}{τοτε οι ϲτρατιωτε του ηγεμονοϲ παραλαβο̅} & 6 &  &  \\
&  & 6 & \foreignlanguage{greek}{τεϲ τον \textoverline{ιν} ειϲ το πρετωριον ϲυνηγαγον επ αυ} & 14 &  &  \\
&  & 14 & \foreignlanguage{greek}{τον ολην την ϲπιραν και εκδυϲαντεϲ} & 2 & \textbf{28} &  \\
&  & 3 & \foreignlanguage{greek}{αυτον περιεθηκαν αυτω χλαμυδα κοκκινη̅} & 7 &  &  \\
& \textbf{29} &  & \foreignlanguage{greek}{και πλεξαντεϲ ϲτεφανον εξ ακανθων} & 5 &  &  \\
&  & 6 & \foreignlanguage{greek}{εθηκαν επι την κεφαλην αυτου και κα} & 12 &  &  \\
&  & 12 & \foreignlanguage{greek}{λαμον επι την δεξιαν αυτου και γονυ} & 18 &  &  \\
&  & 18 & \foreignlanguage{greek}{πετηϲαντεϲ εμπροϲθεν αυτου ενεπεζον} & 21 &  &  \\
&  & 22 & \foreignlanguage{greek}{αυτω λεγοντεϲ χαιρε ο βαϲιλευϲ των ιουδαιω̅} & 28 &  &  \\
& \textbf{30} &  & \foreignlanguage{greek}{και εμπτυϲαντεϲ ειϲ αυτον ελαβον τον} & 6 &  &  \\
&  & 7 & \foreignlanguage{greek}{καλαμον και ετυπτον ειϲ την κεφαλη̅} & 12 &  &  \\
&  & 13 & \foreignlanguage{greek}{αυτου και οτε ενεπεξαν αυτω εξε} & 5 & \textbf{31} &  \\
&  & 5 & \foreignlanguage{greek}{δυϲαν αυτον την χλαμυδα και ενεδυ} & 10 &  &  \\
&  & 10 & \foreignlanguage{greek}{ϲαν αυτον τα ιματια αυτου και απηγαγο̅} & 16 &  &  \\
&  & 17 & \foreignlanguage{greek}{αυτον ειϲ το ϲταυρωϲαι} & 20 &  &  \\
& \textbf{32} &  & \foreignlanguage{greek}{εξερχομενοι δε ευρον \textoverline{ανον} κυρηναιον} & 5 &  &  \\
&  & 6 & \foreignlanguage{greek}{ονοματι ϲιμωνα τουτον ηνγαρευϲα̅} & 9 &  &  \\
&  & 10 & \foreignlanguage{greek}{ινα αρη τον ϲταυρον αυτου και ελθο̅} & 2 & \textbf{33} &  \\
&  & 2 & \foreignlanguage{greek}{τεϲ ειϲ τοπον λεγομενον γολγοθα ο εϲτι̅} & 8 &  &  \\
&  & 9 & \foreignlanguage{greek}{λεγομενον κρανιου τοποϲ εδωκαν} & 1 & \textbf{34} &  \\
&  & 2 & \foreignlanguage{greek}{αυτω πιειν οξοϲ μετα χοληϲ μεμιγμενο̅} & 7 &  &  \\
&  & 8 & \foreignlanguage{greek}{και γευϲαμενοϲ ουκ ηθελεν πιειν} & 12 &  &  \\
& \textbf{35} &  & \foreignlanguage{greek}{ϲταυρωϲαντεϲ δε αυτον διεμεριϲαν} & 4 &  &  \\
&  & 4 & \foreignlanguage{greek}{το τα ιματια αυτου βαλλοντεϲ κληρον ϗ} & 1 & \textbf{36} &  \\
&  & 2 & \foreignlanguage{greek}{καθημενοι ετηρουν αυτον εκει} & 5 &  &  \\
& \textbf{37} &  & \foreignlanguage{greek}{και επεθηκαν επανω τηϲ κεφαληϲ αυ} & 6 &  &  \\
&  & 6 & \foreignlanguage{greek}{του την αιτιαν αυτου γεγραμμενην} & 10 &  &  \\
&  & 11 & \foreignlanguage{greek}{ουτοϲ εϲτιν \textoverline{ιϲ} ο βαϲιλευϲ των ιουδαιων} & 17 &  &  \\
[0.2em]
\cline{4-4}
\end{tabular}
\end{center}
\end{table}
}
\clearpage
\newpage
 {
 \setlength\arrayrulewidth{1pt}
\begin{table}
\begin{center}
\begin{tabular}{ccc|l|ccc}
\cline{4-4} \\ [-1em]
\multicolumn{7}{c}{\foreignlanguage{greek}{ευαγγελιον κατα μαθθαιον} \textbf{(\nospace{27:38})} } \\ \\ [-1em] % Si on veut ajouter les bordures latérales, remplacer {7}{c} par {7}{|c|}
\cline{4-4} \\
\cline{4-4}
&  &  & &  &  & \\ [-0.9em]
& \textbf{38} &  & \foreignlanguage{greek}{τοτε ϲταυρουνται ϲυν αυτω δυο ληϲται ειϲ} & 7 &  &  \\
&  & 8 & \foreignlanguage{greek}{εκ δεξιων και ειϲ εξ ευωνυμων οι δε πα} & 3 & \textbf{39} &  \\
&  & 3 & \foreignlanguage{greek}{ραπορευομενοι εβλαϲφημουν αυτον κει} & 6 &  &  \\
&  & 6 & \foreignlanguage{greek}{νουντεϲ αυτων ταϲ κεφαλαϲ και λεγοντεϲ} & 2 & \textbf{40} &  \\
&  & 3 & \foreignlanguage{greek}{ο καταλυων τον ναον και εν τριϲιν ημε} & 10 &  &  \\
&  & 10 & \foreignlanguage{greek}{ραιϲ οικοδομων ϲωϲον ϲεαυτον ει υιοϲ ει} & 16 &  &  \\
&  & 17 & \foreignlanguage{greek}{του \textoverline{θυ} καταβηθει απο του ϲταυρου} & 22 &  &  \\
& \textbf{41} &  & \foreignlanguage{greek}{ομοιωϲ οι αρχιερειϲ εμπεζοντεϲ μετα} & 5 &  &  \\
&  & 6 & \foreignlanguage{greek}{των γραμματεων και φαριϲαιω ελεγον} & 10 &  &  \\
& \textbf{42} &  & \foreignlanguage{greek}{αλλουϲ εϲωϲεν εαυτον ου δυνατε ϲωϲε} & 6 &  &  \\
&  & 7 & \foreignlanguage{greek}{ει βαϲιλευϲ \textoverline{ιϲρλ} εϲτιν καταβατω νυν απο} & 13 &  &  \\
&  & 14 & \foreignlanguage{greek}{του ϲταυρου και πιϲτευϲωμεν επ αυτω} & 19 &  &  \\
& \textbf{43} &  & \foreignlanguage{greek}{πεποιθεν επι τον \textoverline{θν} ρυϲαϲθω νυν αυτο̅} & 7 &  &  \\
&  & 8 & \foreignlanguage{greek}{ει θελει αυτον ειπεν γαρ οτι του \textoverline{θυ} ει} & 16 &  &  \\
&  & 16 & \foreignlanguage{greek}{μι υιοϲ το δ αυτο και οι ληϲται οι ϲυ̅} & 8 & \textbf{44} &  \\
&  & 8 & \foreignlanguage{greek}{ϲταυρωθεντεϲ αυτω ωνιδιζον αυτο̅} & 11 &  &  \\
& \textbf{45} &  & \foreignlanguage{greek}{απο δε εκτηϲ ωραϲ εγενετο ϲκοτοϲ επι} & 7 &  &  \\
&  & 8 & \foreignlanguage{greek}{παϲαν την γην εωϲ ωραϲ ενατηϲ} & 13 &  &  \\
& \textbf{46} &  & \foreignlanguage{greek}{περι δε την ενατην ωραν εβοηϲεν ο \textoverline{ιϲ}} & 8 &  &  \\
&  & 9 & \foreignlanguage{greek}{φωνη μεγαλη λεγων ηλι ηλι μα ϲαβα} & 15 &  &  \\
&  & 15 & \foreignlanguage{greek}{χθανει τουτ εϲτιν θεε μου θεε μου} & 22 &  &  \\
&  & 23 & \foreignlanguage{greek}{ινα τι με ενκατελειπεϲ} & 26 &  &  \\
& \textbf{47} &  & \foreignlanguage{greek}{τινεϲ δε των εκει ϲτηκωτων ακουϲαν} & 6 &  &  \\
&  & 6 & \foreignlanguage{greek}{τεϲ ελεγον οτι ηλιαν φωνει ουτοϲ} & 11 &  &  \\
& \textbf{48} &  & \foreignlanguage{greek}{και ευθεωϲ δραμων ειϲ εξ αυτων και λα} & 8 &  &  \\
&  & 8 & \foreignlanguage{greek}{βων ϲπογγον πληϲαϲ τε οξουϲ και περι} & 14 &  &  \\
&  & 14 & \foreignlanguage{greek}{θειϲ καλαμω εποτιζεν αυτον} & 17 &  &  \\
& \textbf{49} &  & \foreignlanguage{greek}{οι δε λοιποι ελεγον αφεϲ ειδωμεν ει} & 7 &  &  \\
&  & 8 & \foreignlanguage{greek}{ερχεται ηλιαϲ ϲωζων αυτον ο δε \textoverline{ιϲ}} & 3 & \textbf{50} &  \\
&  & 4 & \foreignlanguage{greek}{κραξαϲ παλιν φωνη μεγαλη αφηκεν το} & 9 &  &  \\
&  & 10 & \foreignlanguage{greek}{\textoverline{πνα}} & 10 &  &  \\
[0.2em]
\cline{4-4}
\end{tabular}
\end{center}
\end{table}
}
\clearpage
\newpage
 {
 \setlength\arrayrulewidth{1pt}
\begin{table}
\begin{center}
\begin{tabular}{ccc|l|ccc}
\cline{4-4} \\ [-1em]
\multicolumn{7}{c}{\foreignlanguage{greek}{ευαγγελιον κατα μαθθαιον} \textbf{(\nospace{27:51})} } \\ \\ [-1em] % Si on veut ajouter les bordures latérales, remplacer {7}{c} par {7}{|c|}
\cline{4-4} \\
\cline{4-4}
&  &  & &  &  & \\ [-0.9em]
& \textbf{51} &  & \foreignlanguage{greek}{και ιδου το καταπεταϲμα του ναου εϲχι} & 7 &  &  \\
&  & 7 & \foreignlanguage{greek}{ϲθη ειϲ δυο απ ανωθεν εωϲ κατω} & 13 &  &  \\
&  & 14 & \foreignlanguage{greek}{και η γη εϲχιϲθη και αι πετραι εϲχιϲθηϲα̅} & 21 &  &  \\
& \textbf{52} &  & \foreignlanguage{greek}{και τα μνημια ανεωχθη και πολλα ϲω} & 7 &  &  \\
&  & 7 & \foreignlanguage{greek}{ματα των κεκοιμημενων αγιων ηγερθη} & 11 &  &  \\
& \textbf{53} &  & \foreignlanguage{greek}{και εξελθοντεϲ εκ των μνημιων μετα} & 6 &  &  \\
&  & 7 & \foreignlanguage{greek}{την εγερϲιν αυτου ειϲηλθον ειϲ την αγι} & 13 &  &  \\
&  & 13 & \foreignlanguage{greek}{αν πολιν και ενεφανιϲθηϲαν πολλοιϲ} & 17 &  &  \\
& \textbf{54} &  & \foreignlanguage{greek}{ο δε εκατονταρχοϲ και οι μετ αυτου τη} & 8 &  &  \\
&  & 8 & \foreignlanguage{greek}{ρουντεϲ τον \textoverline{ιν} ιδοντεϲ τον ϲιϲμον και} & 14 &  &  \\
&  & 15 & \foreignlanguage{greek}{τα γενομενα εφοβηθηϲαν ϲφοδρα λε} & 19 &  &  \\
&  & 19 & \foreignlanguage{greek}{γοντεϲ αληθωϲ \textoverline{θυ} υιοϲ ην ουτοϲ} & 24 &  &  \\
& \textbf{55} &  & \foreignlanguage{greek}{ηϲαν δε εκει γυναικεϲ πολλαι μακροθε̅} & 6 &  &  \\
&  & 7 & \foreignlanguage{greek}{θεωρουϲαι αιτινεϲ ηκολουθηϲαν τω \textoverline{ιυ}} & 11 &  &  \\
&  & 12 & \foreignlanguage{greek}{απο τηϲ γαλιλαιαϲ διακονουϲαι αυτω} & 17 &  &  \\
& \textbf{56} &  & \foreignlanguage{greek}{εν αιϲ ην μαρια η μαγδαληνη και μαρια} & 8 &  &  \\
&  & 9 & \foreignlanguage{greek}{η του ιακωβου και ιωϲηφ \textoverline{μηρ} και η \textoverline{μηρ}} & 17 &  &  \\
&  & 18 & \foreignlanguage{greek}{των υιων ζεβαιδεου} & 20 &  &  \\
& \textbf{57} &  & \foreignlanguage{greek}{οψειαϲ δε γενομενηϲ ηλθεν \textoverline{ανοϲ} πλου} & 6 &  &  \\
&  & 6 & \foreignlanguage{greek}{ϲιοϲ απο αριμαθεαϲ τουνομα ιωϲηφ} & 10 &  &  \\
&  & 11 & \foreignlanguage{greek}{οϲ και αυτοϲ εμαθητευϲεν τω \textoverline{ιυ} ουτοϲ} & 1 & \textbf{58} &  \\
&  & 2 & \foreignlanguage{greek}{προϲελθω τω πιλατω ητηϲατο το ϲωμα του \textoverline{ιυ}} & 9 &  &  \\
&  & 10 & \foreignlanguage{greek}{τοτε ο πιλατοϲ εκελευϲεν αποδοθηναι το} & 15 &  &  \\
&  & 16 & \foreignlanguage{greek}{ϲωμα και λαβων το ϲωμα ο ιωϲηφ} & 6 & \textbf{59} &  \\
&  & 7 & \foreignlanguage{greek}{ενετυλιξεν αυτο ϲινδονι καθαρα και} & 1 & \textbf{60} &  \\
&  & 2 & \foreignlanguage{greek}{εθηκεν αυτο εν τω καινω αυτου μνημιω} & 8 &  &  \\
&  & 9 & \foreignlanguage{greek}{ω ελατομηϲεν εν τη πετρα και προϲ} & 15 &  &  \\
&  & 15 & \foreignlanguage{greek}{κυλιϲαϲ λιθον μεγα εν τη θυρα του μνη} & 22 &  &  \\
&  & 22 & \foreignlanguage{greek}{μιου απηλθεν ην δε εκει μαρια η μα} & 6 & \textbf{61} &  \\
&  & 6 & \foreignlanguage{greek}{γδαληνη και η αλλη μαρια καθημεναι} & 11 &  &  \\
[0.2em]
\cline{4-4}
\end{tabular}
\end{center}
\end{table}
}
\clearpage
\newpage
 {
 \setlength\arrayrulewidth{1pt}
\begin{table}
\begin{center}
\begin{tabular}{ccc|l|ccc}
\cline{4-4} \\ [-1em]
\multicolumn{7}{c}{\foreignlanguage{greek}{ευαγγελιον κατα μαθθαιον} \textbf{(\nospace{27:61})} } \\ \\ [-1em] % Si on veut ajouter les bordures latérales, remplacer {7}{c} par {7}{|c|}
\cline{4-4} \\
\cline{4-4}
&  &  & &  &  & \\ [-0.9em]
&  & 12 & \foreignlanguage{greek}{επι του ταφου τη δε επαυριον ητιϲ ε} & 5 & \textbf{62} &  \\
&  & 5 & \foreignlanguage{greek}{ϲτιν μετα την παραϲκευην ϲυνηχθη} & 9 &  &  \\
&  & 9 & \foreignlanguage{greek}{ϲαν οι αρχιερειϲ και οι φαριϲαιοι προϲ πι} & 16 &  &  \\
&  & 16 & \foreignlanguage{greek}{λατον λεγοντεϲ \textoverline{κε} εμνηϲθημεν} & 3 & \textbf{63} &  \\
&  & 4 & \foreignlanguage{greek}{οτι εκεινοϲ ο πλανοϲ ειπεν ετι ζων} & 10 &  &  \\
&  & 11 & \foreignlanguage{greek}{μετα τριϲ ημεραϲ εγειρομαι κελευ} & 1 & \textbf{64} &  \\
&  & 1 & \foreignlanguage{greek}{ϲον ουν αϲφαλιϲθηναι τον ταφον ε} & 6 &  &  \\
&  & 6 & \foreignlanguage{greek}{ωϲ τηϲ τριτηϲ ημεραϲ μηποτε ελθο̅} & 11 &  &  \\
&  & 11 & \foreignlanguage{greek}{τεϲ οι μαθηται αυτου κλεψωϲιν αυτο̅} & 16 &  &  \\
&  & 17 & \foreignlanguage{greek}{και ειπωϲιν τω λαω ηγερθη απο των} & 23 &  &  \\
&  & 24 & \foreignlanguage{greek}{νεκρων και εϲται η εϲχατη πλανη} & 29 &  &  \\
&  & 30 & \foreignlanguage{greek}{χειρων τηϲ πρωτηϲ} & 32 &  &  \\
& \textbf{65} &  & \foreignlanguage{greek}{εφη δε αυτοιϲ ο πειλατοϲ εχεται κου} & 7 &  &  \\
&  & 7 & \foreignlanguage{greek}{ϲτωδιαν υπαγεται αϲφαλιϲαϲθαι ωϲ} & 10 &  &  \\
&  & 11 & \foreignlanguage{greek}{οιδατε οι δε πορευθεντεϲ ηϲφαλι} & 4 & \textbf{66} &  \\
&  & 4 & \foreignlanguage{greek}{ϲαντο τον ταφον ϲφραγιϲαντεϲ τον} & 8 &  &  \\
&  & 9 & \foreignlanguage{greek}{λιθον μετα τηϲ κουϲτωδιαϲ} & 12 &  &  \\
& \mygospelchapter &  & \foreignlanguage{greek}{οψε δε ϲαββατων τη επιφωϲκουϲη ειϲ} & 6 &  &  \\
&  & 7 & \foreignlanguage{greek}{μιαν ϲαββατων ηλθεν μαρια η μαγδα} & 12 &  &  \\
&  & 12 & \foreignlanguage{greek}{ληνη και η αλλη μαρια θεωρουϲαι τον} & 18 &  &  \\
&  & 19 & \foreignlanguage{greek}{ταφον και ιδου ϲιϲμοϲ εγενετο μεγαϲ} & 5 & \textbf{2} &  \\
&  & 6 & \foreignlanguage{greek}{αγγελοϲ γαρ \textoverline{κυ} κατεβη εξ ουρανου} & 11 &  &  \\
&  & 12 & \foreignlanguage{greek}{και προϲελθων απεκυλιϲεν τον λιθον} & 16 &  &  \\
&  & 17 & \foreignlanguage{greek}{απο τηϲ θυραϲ και εκαθητο επανω} & 22 &  &  \\
&  & 23 & \foreignlanguage{greek}{αυτου ην δε η ιδεα αυτου ωϲ αϲτρα} & 7 & \textbf{3} &  \\
&  & 7 & \foreignlanguage{greek}{πη και το ενδυμα αυτου λευκον ωϲει} & 13 &  &  \\
&  & 14 & \foreignlanguage{greek}{χιων απο δε του φοβου αυτου ε} & 6 & \textbf{4} &  \\
&  & 6 & \foreignlanguage{greek}{ϲειϲθηϲαν οι τηρουντεϲ και εγενοντο} & 10 &  &  \\
&  & 11 & \foreignlanguage{greek}{ωϲει νεκροι αποκριθειϲ ο αγγελοϲ} & 3 & \textbf{5} &  \\
&  & 4 & \foreignlanguage{greek}{ειπεν ταιϲ γυναιξιν μη φοβειϲθε υμειϲ} & 9 &  &  \\
[0.2em]
\cline{4-4}
\end{tabular}
\end{center}
\end{table}
}
\clearpage
\newpage
 {
 \setlength\arrayrulewidth{1pt}
\begin{table}
\begin{center}
\begin{tabular}{ccc|l|ccc}
\cline{4-4} \\ [-1em]
\multicolumn{7}{c}{\foreignlanguage{greek}{ευαγγελιον κατα μαθθαιον} \textbf{(\nospace{28:5})} } \\ \\ [-1em] % Si on veut ajouter les bordures latérales, remplacer {7}{c} par {7}{|c|}
\cline{4-4} \\
\cline{4-4}
&  &  & &  &  & \\ [-0.9em]
&  & 10 & \foreignlanguage{greek}{οιδα γαρ οτι \textoverline{ιν} τον εϲταυρωμενον ζητιται} & 16 &  &  \\
& \textbf{6} &  & \foreignlanguage{greek}{ουκ εϲτιν ωδε ηγερθη γαρ καθωϲ ειπεν} & 7 &  &  \\
&  & 8 & \foreignlanguage{greek}{δευτε ειδεται τον τοπον οπου εκειτο} & 13 &  &  \\
&  & 14 & \foreignlanguage{greek}{ο \textoverline{κϲ} και ταχυ πορευθειϲαι ειπατε τοιϲ} & 5 & \textbf{7} &  \\
&  & 6 & \foreignlanguage{greek}{μαθηταιϲ αυτου οτι ηγερθη απο τω̅} & 11 &  &  \\
&  & 12 & \foreignlanguage{greek}{νεκρων και ιδου προαγει υμαϲ ειϲ τη̅} & 18 &  &  \\
&  & 19 & \foreignlanguage{greek}{γαλιλαιαν εκει αυτον οψεϲθαι ιδου} & 23 &  &  \\
&  & 24 & \foreignlanguage{greek}{ειπον υμιν και εξελθουϲαι ταχυ} & 3 & \textbf{8} &  \\
&  & 4 & \foreignlanguage{greek}{απο του μνημιου μετα φοβου και χαραϲ} & 10 &  &  \\
&  & 11 & \foreignlanguage{greek}{μεγαληϲ εδραμον απαγγειλαι τοιϲ μα} & 15 &  &  \\
&  & 15 & \foreignlanguage{greek}{θηταιϲ αυτου} & 16 &  &  \\
& \textbf{9} &  & \foreignlanguage{greek}{και ιδου ο \textoverline{ιϲ} απηντηϲεν αυταιϲ λεγω̅} & 7 &  &  \\
&  & 8 & \foreignlanguage{greek}{χαιρεται αι δε προϲελθουϲαι εκρα} & 12 &  &  \\
&  & 12 & \foreignlanguage{greek}{τηϲαν αυτου τουϲ ποδαϲ και προϲε} & 17 &  &  \\
&  & 17 & \foreignlanguage{greek}{κυνηϲαν αυτω τοτε λεγει αυταιϲ} & 3 & \textbf{10} &  \\
&  & 4 & \foreignlanguage{greek}{ο \textoverline{ιϲ} μη φοβειϲθαι υπαγεται απαγγειλα} & 9 &  &  \\
&  & 9 & \foreignlanguage{greek}{τε τοιϲ αδελφοιϲ μου ινα απελθωϲιν} & 14 &  &  \\
&  & 15 & \foreignlanguage{greek}{ειϲ την γαλιλαιαν ϗ εκει με οψονται} & 21 &  &  \\
& \textbf{11} &  & \foreignlanguage{greek}{πορευομενων δε αυτων ιδου τινεϲ τηϲ} & 6 &  &  \\
&  & 7 & \foreignlanguage{greek}{κουϲτωδιαϲ ελθοντεϲ ειϲ την πολιν} & 11 &  &  \\
&  & 12 & \foreignlanguage{greek}{απηγγειλον τοιϲ αρχιερευϲιν απαντα} & 15 &  &  \\
&  & 16 & \foreignlanguage{greek}{τα γενομενα και ϲυναχθεντεϲ με} & 3 & \textbf{12} &  \\
&  & 3 & \foreignlanguage{greek}{τα των πρεϲβυτερων ϲυνβουλιον τε} & 7 &  &  \\
&  & 8 & \foreignlanguage{greek}{λαβοντεϲ αργυρια ικανα εδωκαν τοιϲ} & 12 &  &  \\
&  & 13 & \foreignlanguage{greek}{ϲτρατιωταιϲ λεγοντεϲ ειπατε οτι} & 3 & \textbf{13} &  \\
&  & 4 & \foreignlanguage{greek}{οι μαθηται αυτου νυκτοϲ ελθοντεϲ} & 8 &  &  \\
&  & 9 & \foreignlanguage{greek}{εκλεψαν αυτον ημων κοιμωμενων} & 12 &  &  \\
& \textbf{14} &  & \foreignlanguage{greek}{και εαν ακουϲθη τουτο επι του ηγεμο} & 7 &  &  \\
&  & 7 & \foreignlanguage{greek}{νοϲ ημειϲ πιϲομεν αυτον και υμαϲ α} & 13 &  &  \\
&  & 13 & \foreignlanguage{greek}{μεριμνουϲ ποιηϲωμεν} & 14 &  &  \\
[0.2em]
\cline{4-4}
\end{tabular}
\end{center}
\end{table}
}
\clearpage
\newpage
 {
 \setlength\arrayrulewidth{1pt}
\begin{table}
\begin{center}
\begin{tabular}{ccc|l|ccc}
\cline{4-4} \\ [-1em]
\multicolumn{7}{c}{\foreignlanguage{greek}{ευαγγελιον κατα μαθθαιον} \textbf{(\nospace{28:15})} } \\ \\ [-1em] % Si on veut ajouter les bordures latérales, remplacer {7}{c} par {7}{|c|}
\cline{4-4} \\
\cline{4-4}
&  &  & &  &  & \\ [-0.9em]
& \textbf{15} &  & \foreignlanguage{greek}{οι δε λαβοντεϲ αργυρια εποιηϲαν ωϲ εδιδα} & 7 &  &  \\
&  & 7 & \foreignlanguage{greek}{χθηϲαν και διεφημιϲθη ο λογοϲ ουτοϲ} & 12 &  &  \\
&  & 13 & \foreignlanguage{greek}{παρα ιουδαιοιϲ μεχριϲ τηϲ ϲημερον} & 17 &  &  \\
& \textbf{16} &  & \foreignlanguage{greek}{οι δε ενδεκα μαθηται επορευθηϲαν ειϲ} & 6 &  &  \\
&  & 7 & \foreignlanguage{greek}{την γαλιλαιαν ειϲ το οροϲ ου εταξατο} & 13 &  &  \\
&  & 14 & \foreignlanguage{greek}{αυτοιϲ ο \textoverline{ιϲ} και ιδοντεϲ αυτον προϲε} & 4 & \textbf{17} &  \\
&  & 4 & \foreignlanguage{greek}{κυνηϲαν αυτω οι δε εδιϲταϲαν} & 8 &  &  \\
& \textbf{18} &  & \foreignlanguage{greek}{και προϲελθων ο \textoverline{ιϲ} ελαληϲεν αυτοιϲ λεγω̅} & 7 &  &  \\
&  & 8 & \foreignlanguage{greek}{εδοθη μοι παϲα εξουϲια εν ουρανω και} & 14 &  &  \\
&  & 15 & \foreignlanguage{greek}{επι γηϲ πορευθεντεϲ ουν μαθητευ} & 3 & \textbf{19} &  \\
&  & 3 & \foreignlanguage{greek}{ϲατε παντα τα εθνη βαπτιζοντεϲ αυ} & 8 &  &  \\
&  & 8 & \foreignlanguage{greek}{τουϲ ειϲ το ονομα του \textoverline{πρϲ} και του υιου} & 16 &  &  \\
&  & 17 & \foreignlanguage{greek}{και του αγιου \textoverline{πνϲ} διδαϲκοντεϲ αυ} & 2 & \textbf{20} &  \\
&  & 2 & \foreignlanguage{greek}{τουϲ τηρειν παντα οϲα ενετειλαμην} & 6 &  &  \\
&  & 7 & \foreignlanguage{greek}{υμιν και ιδου εγω μεθ υμων ειμει} & 13 &  &  \\
&  & 14 & \foreignlanguage{greek}{παϲαϲ ταϲ ημεραϲ εωϲ τηϲ ϲυντελει} & 19 &  &  \\
&  & 19 & \foreignlanguage{greek}{αϲ του αιωνοϲ} & 21 &  &  \\
[0.2em]
\cline{4-4}
\end{tabular}
\end{center}
\end{table}
}
\clearpage
\newpage
 {
 \setlength\arrayrulewidth{1pt}
\begin{table}
\begin{center}
\begin{tabular}{ccc|l|ccc}
\cline{4-4} \\ [-1em]
\multicolumn{7}{c}{\agospelbook{\foreignlanguage{greek}{ευαγγελιον κατα ιωαννην}} \textbf{(\nospace{1:1})} } \\ \\ [-1em] % Si on veut ajouter les bordures latérales, remplacer {7}{c} par {7}{|c|}
\cline{4-4} \\
\cline{4-4}
&  &  & &  &  & \\ [-0.9em]
& \mygospelchapter &  & \foreignlanguage{greek}{εν αρχη ην ο λογοϲ και ο λογοϲ ην προϲ τον \textoverline{θν}} & 12 &  &  \\
&  & 13 & \foreignlanguage{greek}{και ο \textoverline{θϲ} ην ο λογοϲ ουτοϲ ην εν αρχη προϲ} & 5 & \textbf{2} &  \\
&  & 6 & \foreignlanguage{greek}{τον \textoverline{θν} παντα δι αυτου εγενετο και χω} & 6 & \textbf{3} &  \\
&  & 6 & \foreignlanguage{greek}{ριϲ αυτου εγενετο ουδε εν ο γεγονεν εν} & 1 & \textbf{4} &  \\
&  & 2 & \foreignlanguage{greek}{αυτω ζωη και η ζωη ην το φωϲ των} & 10 &  &  \\
&  & 11 & \foreignlanguage{greek}{\textoverline{ανων} και το φωϲ εν τη ϲκοτια φενει} & 7 & \textbf{5} &  \\
&  & 8 & \foreignlanguage{greek}{και η ϲκοτια αυτο ου κατελαβεν} & 13 &  &  \\
& \textbf{6} &  & \foreignlanguage{greek}{εγενετο \textoverline{ανοϲ} απεϲταλμενοϲ απο \textoverline{θυ} ην ο} & 7 &  &  \\
&  & 7 & \foreignlanguage{greek}{νομα αυτω ιωαννηϲ ουτοϲ ηλθεν ειϲ} & 3 & \textbf{7} &  \\
&  & 4 & \foreignlanguage{greek}{μαρτυριαν ινα μαρτυρηϲη περι του} & 8 &  &  \\
&  & 9 & \foreignlanguage{greek}{φωτοϲ ινα παντεϲ πιϲτευϲωϲιν δι αυ} & 14 &  &  \\
&  & 14 & \foreignlanguage{greek}{του ουκ ην εκινοϲ το φωϲ αλλ ινα} & 7 & \textbf{8} &  \\
&  & 8 & \foreignlanguage{greek}{μαρτυρηϲη περι του φωτοϲ} & 11 &  &  \\
& \textbf{9} &  & \foreignlanguage{greek}{ην το φωϲ το αληθινον ο φωτιζι παντα} & 8 &  &  \\
&  & 9 & \foreignlanguage{greek}{\textoverline{ανον} ερχομενον ειϲ τον κοϲμον} & 13 &  &  \\
& \textbf{10} &  & \foreignlanguage{greek}{εν τω κοϲμω ην και ο κοϲμοϲ δι αυτου} & 9 &  &  \\
&  & 10 & \foreignlanguage{greek}{εγενετο και ο κοϲμοϲ αυτον ουκ εγνω} & 16 &  &  \\
& \textbf{11} &  & \foreignlanguage{greek}{ειϲ τα ιδια ηλθεν και οι ειδιοι αυτον ου} & 9 &  &  \\
&  & 10 & \foreignlanguage{greek}{παρελαβον οϲοι δε ελαβον αυτον} & 4 & \textbf{12} &  \\
&  & 5 & \foreignlanguage{greek}{εδωκεν αυτοιϲ εξουϲιαν τεκνα \textoverline{θυ}} & 9 &  &  \\
&  & 10 & \foreignlanguage{greek}{γενεϲθε τοιϲ πιϲτευουϲιν ειϲ το ονομα} & 15 &  &  \\
&  & 16 & \foreignlanguage{greek}{αυτου οι ουκ εξ εματων ουδε εκ} & 6 & \textbf{13} &  \\
&  & 7 & \foreignlanguage{greek}{θεληματοϲ ϲαρκοϲ ουδε εκ θελημα} & 11 &  &  \\
&  & 11 & \foreignlanguage{greek}{τοϲ ανδροϲ αλλα εκ \textoverline{θυ} εγεννηθηϲαν} & 16 &  &  \\
& \textbf{14} &  & \foreignlanguage{greek}{και ο λογοϲ ϲαρξ εγενετο και εϲκηνω} & 7 &  &  \\
&  & 7 & \foreignlanguage{greek}{ϲεν εν ημιν και εθεαϲαμεθα την} & 12 &  &  \\
&  & 13 & \foreignlanguage{greek}{δοξαν αυτου δοξαν ωϲ μονογενουϲ} & 17 &  &  \\
&  & 18 & \foreignlanguage{greek}{παρα \textoverline{πρϲ} πληριϲ χαριτοϲ και αληθιαϲ} & 23 &  &  \\
& \textbf{15} &  & \foreignlanguage{greek}{ιωαννηϲ μαρτυρι περι αυτου και κε} & 6 &  &  \\
&  & 6 & \foreignlanguage{greek}{κραγεν λεγων ουτοϲ ην ον ειπον} & 11 &  &  \\
[0.2em]
\cline{4-4}
\end{tabular}
\end{center}
\end{table}
}
\clearpage
\newpage
 {
 \setlength\arrayrulewidth{1pt}
\begin{table}
\begin{center}
\begin{tabular}{ccc|l|ccc}
\cline{4-4} \\ [-1em]
\multicolumn{7}{c}{\foreignlanguage{greek}{ευαγγελιον κατα ιωαννην} \textbf{(\nospace{1:15})} } \\ \\ [-1em] % Si on veut ajouter les bordures latérales, remplacer {7}{c} par {7}{|c|}
\cline{4-4} \\
\cline{4-4}
&  &  & &  &  & \\ [-0.9em]
&  & 12 & \foreignlanguage{greek}{υμιν ο οπιϲω μου ερχομενοϲ οϲ εμ} & 18 &  &  \\
&  & 18 & \foreignlanguage{greek}{προϲθεν μου γεγονεν οτι πρωτοϲ μου} & 23 &  &  \\
&  & 24 & \foreignlanguage{greek}{ην και εκ του πληρωματοϲ αυτου} & 5 & \textbf{16} &  \\
&  & 6 & \foreignlanguage{greek}{ημιϲ παντεϲ ζωην ελαβομεν και χα} & 11 &  &  \\
&  & 11 & \foreignlanguage{greek}{ριν αντι χαριτοϲ οτι ο νομοϲ δια μω} & 5 & \textbf{17} &  \\
&  & 5 & \foreignlanguage{greek}{υϲεωϲ εδοθη η δε χαριϲ και η αληθια} & 12 &  &  \\
&  & 13 & \foreignlanguage{greek}{δια \textoverline{ιυ} \textoverline{χυ} εγενετο} & 16 &  &  \\
& \textbf{18} &  & \foreignlanguage{greek}{\textoverline{θν} ουδιϲ εωρακεν πωποτε ει μη ο μονο} & 8 &  &  \\
&  & 8 & \foreignlanguage{greek}{γενηϲ \textoverline{υϲ} ο ων ειϲ τον κολπον του \textoverline{πρϲ}} & 16 &  &  \\
&  & 17 & \foreignlanguage{greek}{εκινοϲ εξηγηϲατο ημιν και αυτη} & 2 & \textbf{19} &  \\
&  & 3 & \foreignlanguage{greek}{εϲτιν η μαρτυρια του ιωαννου οτε α} & 9 &  &  \\
&  & 9 & \foreignlanguage{greek}{πεϲτιλαν οι ιουδεοι εξ ιεροϲολυμων} & 13 &  &  \\
&  & 14 & \foreignlanguage{greek}{ιεριϲ και λευειταϲ ινα ερωτηϲουϲιν} & 18 &  &  \\
&  & 19 & \foreignlanguage{greek}{αυτον ϲυ τιϲ ει και ωμολογηϲεν και} & 3 & \textbf{20} &  \\
&  & 4 & \foreignlanguage{greek}{ουκ ηρνηϲατο ωμολογηϲεν οτι} & 7 &  &  \\
&  & 8 & \foreignlanguage{greek}{εγω ουκ ιμι ο \textoverline{χϲ} και ηρωτηϲαν αυτον} & 3 & \textbf{21} &  \\
&  & 4 & \foreignlanguage{greek}{παλιν τι ουν ϲυ ει ηλιαϲ και λεγι ουκ ει} & 13 &  &  \\
&  & 13 & \foreignlanguage{greek}{μει τι ουν ο προφητηϲ ει ϲυ και απε} & 21 &  &  \\
&  & 21 & \foreignlanguage{greek}{κριθη ου ειπαν ουν αυτω τιϲ ει} & 5 & \textbf{22} &  \\
&  & 6 & \foreignlanguage{greek}{ινα αποκριϲιν δωμεν τοιϲ πεμ} & 10 &  &  \\
&  & 10 & \foreignlanguage{greek}{ψαϲιν ημαϲ τι λεγιϲ περι ϲεαυτου} & 15 &  &  \\
& \textbf{23} &  & \foreignlanguage{greek}{εφη εγω φωνη βοωντοϲ εν τη} & 6 &  &  \\
&  & 7 & \foreignlanguage{greek}{ερημω ευθυνατε την οδον \textoverline{κυ}} & 11 &  &  \\
&  & 12 & \foreignlanguage{greek}{ευθιαϲ ποιειτε ταϲ τριβουϲ αυτου} & 16 &  &  \\
&  & 17 & \foreignlanguage{greek}{καθωϲ ειπεν ηϲαιαϲ ο προφητηϲ} & 21 &  &  \\
& \textbf{24} &  & \foreignlanguage{greek}{και οι απεϲταλμενοι ηϲαν εκ των φαρι} & 7 &  &  \\
&  & 7 & \foreignlanguage{greek}{ϲεων και ηρωτηϲαν αυτον και} & 4 & \textbf{25} &  \\
&  & 5 & \foreignlanguage{greek}{ειπαν αυτω τι ουν βαπτιζιϲ ει ϲυ ου} & 12 &  &  \\
&  & 12 & \foreignlanguage{greek}{κ ι ο \textoverline{χϲ} ουδε ηλιαϲ ουδε ο προφητηϲ} & 20 &  &  \\
& \textbf{26} &  & \foreignlanguage{greek}{απεκριθη αυτοιϲ ο ιωαννηϲ λεγων} & 5 &  &  \\
[0.2em]
\cline{4-4}
\end{tabular}
\end{center}
\end{table}
}
\clearpage
\newpage
 {
 \setlength\arrayrulewidth{1pt}
\begin{table}
\begin{center}
\begin{tabular}{ccc|l|ccc}
\cline{4-4} \\ [-1em]
\multicolumn{7}{c}{\foreignlanguage{greek}{ευαγγελιον κατα ιωαννην} \textbf{(\nospace{1:26})} } \\ \\ [-1em] % Si on veut ajouter les bordures latérales, remplacer {7}{c} par {7}{|c|}
\cline{4-4} \\
\cline{4-4}
&  &  & &  &  & \\ [-0.9em]
&  & 6 & \foreignlanguage{greek}{εγω βαπτιζω εν υδατι μεϲοϲ δε} & 11 &  &  \\
&  & 12 & \foreignlanguage{greek}{υμων εϲτηκεν ον υμιϲ ουκ οιδατε} & 17 &  &  \\
& \textbf{27} &  & \foreignlanguage{greek}{ο οπιϲω μου ερχομενοϲ ου ουκ ιμι} & 7 &  &  \\
&  & 8 & \foreignlanguage{greek}{εγω αξιοϲ ινα λυϲω αυτου τον ι} & 14 &  &  \\
&  & 14 & \foreignlanguage{greek}{μαντα του υποδηματοϲ} & 16 &  &  \\
& \textbf{28} &  & \foreignlanguage{greek}{ταυτα εν βηθανια εγενετο περαν του} & 6 &  &  \\
&  & 7 & \foreignlanguage{greek}{ιορδανου οπου ην ο ιωαννηϲ βαπτι} & 12 &  &  \\
&  & 12 & \foreignlanguage{greek}{ζων τη επαυριον βλεπι τον \textoverline{ιν}} & 5 & \textbf{29} &  \\
&  & 6 & \foreignlanguage{greek}{ερχομενον και λεγι ιδε ο αμνοϲ} & 11 &  &  \\
&  & 12 & \foreignlanguage{greek}{του \textoverline{θυ} ο ερων ταϲ αμαρτιαϲ του} & 18 &  &  \\
&  & 19 & \foreignlanguage{greek}{κοϲμου} & 19 &  &  \\
& \textbf{30} &  & \foreignlanguage{greek}{ουτοϲ εϲτιν υπερ ου εγω ειπον υμιν} & 7 &  &  \\
&  & 8 & \foreignlanguage{greek}{οτι οπιϲω μου ερχεται ανηρ οϲ εμ} & 14 &  &  \\
&  & 14 & \foreignlanguage{greek}{προϲθεν μου γεγονεν οτι πρωτοϲ} & 18 &  &  \\
&  & 19 & \foreignlanguage{greek}{μου ην καγω ουκ ηδιν αυτον αλλ ι} & 6 & \textbf{31} &  \\
&  & 6 & \foreignlanguage{greek}{να φανερωθη τω \textoverline{ιηλ} δια του} & 11 &  &  \\
&  & 11 & \foreignlanguage{greek}{το ηλθον εγω εν υδατι βαπτιζιν} & 16 &  &  \\
& \textbf{32} &  & \foreignlanguage{greek}{και εμαρτυρηϲεν ιωαννηϲ λεγω̅} & 4 &  &  \\
&  & 5 & \foreignlanguage{greek}{οτι τεθεαμε το \textoverline{πνα} καταβενον ωϲ} & 10 &  &  \\
&  & 11 & \foreignlanguage{greek}{περιϲτεραν εξ \textoverline{ουρου} και μενον επ αυ} & 17 &  &  \\
&  & 17 & \foreignlanguage{greek}{τον καγω ουκ ηδιν αυτον αλλ ο πεμ} & 7 & \textbf{33} &  \\
&  & 7 & \foreignlanguage{greek}{ψαϲ με βαπτιζιν εν υδατι εκινοϲ μοι} & 13 &  &  \\
&  & 14 & \foreignlanguage{greek}{ειπεν εφ ον αν ειδηϲ το \textoverline{πνα} κα} & 21 &  &  \\
&  & 21 & \foreignlanguage{greek}{ταβενον και μενον επ αυτω ουτοϲ} & 26 &  &  \\
&  & 28 & \foreignlanguage{greek}{εϲτιν ο βαπτιζων εν \textoverline{πνι} αγιω} & 33 &  &  \\
& \textbf{34} &  & \foreignlanguage{greek}{καγω εορακα και μεμαρτυρηκα οτι} & 5 &  &  \\
&  & 6 & \foreignlanguage{greek}{ουτοϲ εϲτιν ο \textoverline{υϲ} του \textoverline{θυ} τη επαυριον} & 2 & \textbf{35} &  \\
&  & 3 & \foreignlanguage{greek}{παλι ιϲτηκι ο ιωαννηϲ και εκ των μα} & 10 &  &  \\
&  & 10 & \foreignlanguage{greek}{θητων αυτου δυο και εμβλεψαϲ} & 2 & \textbf{36} &  \\
&  & 3 & \foreignlanguage{greek}{τω \textoverline{ιυ} περιπατουντι λεγι ειδε ο αμνοϲ} & 9 &  &  \\
[0.2em]
\cline{4-4}
\end{tabular}
\end{center}
\end{table}
}
\clearpage
\newpage
 {
 \setlength\arrayrulewidth{1pt}
\begin{table}
\begin{center}
\begin{tabular}{ccc|l|ccc}
\cline{4-4} \\ [-1em]
\multicolumn{7}{c}{\foreignlanguage{greek}{ευαγγελιον κατα ιωαννην} \textbf{(\nospace{1:36})} } \\ \\ [-1em] % Si on veut ajouter les bordures latérales, remplacer {7}{c} par {7}{|c|}
\cline{4-4} \\
\cline{4-4}
&  &  & &  &  & \\ [-0.9em]
&  & 10 & \foreignlanguage{greek}{του \textoverline{θυ} ο ερων ταϲ αμαρτιαϲ του κοϲμου} & 17 &  &  \\
& \textbf{37} &  & \foreignlanguage{greek}{κηκουϲαν οι δυο αυτου μαθηται ια} & 6 &  &  \\
&  & 7 & \foreignlanguage{greek}{λαλουντοϲ και ηκολουθηϲαν τω \textoverline{ιυ}} & 11 &  &  \\
& \textbf{38} &  & \foreignlanguage{greek}{ϲτραφιϲ δε ο \textoverline{ιϲ} και θεαϲαμενοϲ αυτουϲ} & 7 &  &  \\
&  & 8 & \foreignlanguage{greek}{ακολουθουνταϲ λεγι αυτοιϲ τι ζη} & 12 &  &  \\
&  & 12 & \foreignlanguage{greek}{τιται οι δε ειπαν αυτω ραββει ο λε} & 19 &  &  \\
&  & 19 & \foreignlanguage{greek}{γετε μεθερμηνευομενον διδαϲκα} & 21 &  &  \\
&  & 21 & \foreignlanguage{greek}{λε που μενιϲ λεγι αυτοιϲ ερχεϲθαι} & 3 & \textbf{39} &  \\
&  & 4 & \foreignlanguage{greek}{και οψεϲθαι ηλθαν ουν και ειδαν} & 9 &  &  \\
&  & 10 & \foreignlanguage{greek}{που μενι και παρ αυτω εμιναν τη̅} & 16 &  &  \\
&  & 17 & \foreignlanguage{greek}{ημεραν εκινην ωρα ην ωϲ δεκατη} & 22 &  &  \\
& \textbf{40} &  & \foreignlanguage{greek}{ην δε ανδρεαϲ ο αδελφοϲ ϲειμωνοϲ} & 6 &  &  \\
&  & 7 & \foreignlanguage{greek}{πετρου ειϲ εκ των δυο των ακουϲαν} & 13 &  &  \\
&  & 13 & \foreignlanguage{greek}{των παρα ιωαννου και ακολουθη} & 17 &  &  \\
&  & 17 & \foreignlanguage{greek}{ϲαντων αυτω ευριϲκι ουτοϲ πρω} & 3 & \textbf{41} &  \\
&  & 3 & \foreignlanguage{greek}{τοϲ τον αδελφον τον ιδιον ϲιμωνα} & 8 &  &  \\
&  & 9 & \foreignlanguage{greek}{και λεγι αυτω ευρηκαμεν τον μεϲ} & 14 &  &  \\
&  & 14 & \foreignlanguage{greek}{ϲιαν ο εϲτιν μεθερμηνευομενον \textoverline{χϲ}} & 18 &  &  \\
& \textbf{42} &  & \foreignlanguage{greek}{και ηγαγεν αυτον προϲ τον \textoverline{ιν} και εμ} & 8 &  &  \\
&  & 8 & \foreignlanguage{greek}{βλεψαϲ αυτω ο \textoverline{ιϲ} ειπεν ϲυ ει ϲιμων} & 15 &  &  \\
&  & 16 & \foreignlanguage{greek}{ο \textoverline{υϲ} ιωαννου ϲυ κληθηϲη κηφαϲ} & 21 &  &  \\
&  & 22 & \foreignlanguage{greek}{ο ερμηνευετε πετροϲ τη επαυριο̅} & 2 & \textbf{43} &  \\
&  & 3 & \foreignlanguage{greek}{ηθεληϲεν εξελθιν ειϲ την γαλιδεαν} & 7 &  &  \\
&  & 8 & \foreignlanguage{greek}{και ευριϲκι φιλιππον και λεγι αυτω} & 13 &  &  \\
&  & 14 & \foreignlanguage{greek}{ο \textoverline{ιϲ} ακολουθι μοι ην δε φιλιπποϲ} & 3 & \textbf{44} &  \\
&  & 4 & \foreignlanguage{greek}{απο βηθϲαιδα εκ τηϲ πολεωϲ αν} & 9 &  &  \\
&  & 9 & \foreignlanguage{greek}{δρεου και πετρου ευριϲκι φιλιπποϲ} & 2 & \textbf{45} &  \\
&  & 3 & \foreignlanguage{greek}{τον ναθαναηλ και λεγι αυτω ον ε} & 9 &  &  \\
&  & 9 & \foreignlanguage{greek}{γραψεν μωυϲηϲ εν τω νομω και} & 14 &  &  \\
&  & 15 & \foreignlanguage{greek}{οι προφηται ευρηκαμεν \textoverline{ιν} τον τω} & 20 &  &  \\
[0.2em]
\cline{4-4}
\end{tabular}
\end{center}
\end{table}
}
\clearpage
\newpage
 {
 \setlength\arrayrulewidth{1pt}
\begin{table}
\begin{center}
\begin{tabular}{ccc|l|ccc}
\cline{4-4} \\ [-1em]
\multicolumn{7}{c}{\foreignlanguage{greek}{ευαγγελιον κατα ιωαννην} \textbf{(\nospace{1:45})} } \\ \\ [-1em] % Si on veut ajouter les bordures latérales, remplacer {7}{c} par {7}{|c|}
\cline{4-4} \\
\cline{4-4}
&  &  & &  &  & \\ [-0.9em]
&  & 21 & \foreignlanguage{greek}{ιωϲηφ τον απο ναζαρεθ και ειπεν} & 2 & \textbf{46} &  \\
&  & 3 & \foreignlanguage{greek}{αυτω ναθαναηλ εκ ναζαρετ δυνα} & 7 &  &  \\
&  & 7 & \foreignlanguage{greek}{τε τι αγαθον ειναι λεγι αυτω φιλιπ} & 13 &  &  \\
&  & 13 & \foreignlanguage{greek}{ποϲ ερχου και ιδε ειδεν δε ο \textoverline{ιϲ} τον} & 5 & \textbf{47} &  \\
&  & 6 & \foreignlanguage{greek}{ναθαναηλ ερχομενον προϲ αυτον} & 9 &  &  \\
&  & 10 & \foreignlanguage{greek}{και λεγι περι αυτου ειδε αληθωϲ} & 15 &  &  \\
&  & 16 & \foreignlanguage{greek}{ιϲραηλιτηϲ εν ω δολοϲ ουκ εϲτιν} & 21 &  &  \\
& \textbf{48} &  & \foreignlanguage{greek}{λεγι αυτω ναθαναηλ ποθεν με γι} & 6 &  &  \\
&  & 6 & \foreignlanguage{greek}{γνωϲκιϲ απεκριθη \textoverline{ιϲ} και ειπεν} & 10 &  &  \\
&  & 11 & \foreignlanguage{greek}{αυτω προ του ϲε φιλιππον φωνηϲε} & 16 &  &  \\
&  & 17 & \foreignlanguage{greek}{οντα υπο την ϲυκην ειδον ϲε απε} & 1 & \textbf{49} &  \\
&  & 1 & \foreignlanguage{greek}{κριθη αυτω ναθαναηλ ραββι ϲυ ει} & 6 &  &  \\
&  & 7 & \foreignlanguage{greek}{ο \textoverline{υϲ} του \textoverline{θυ} ϲυ \textoverline{βλευϲ} ει του \textoverline{ιηλ} απε} & 1 & \textbf{50} &  \\
&  & 1 & \foreignlanguage{greek}{κριθη \textoverline{ιϲ} και ειπεν αυτω οτι ειπον ϲοι} & 8 &  &  \\
&  & 9 & \foreignlanguage{greek}{οτι ειδον υποκατω τηϲ ϲυκηϲ πιϲτευ} & 14 &  &  \\
&  & 14 & \foreignlanguage{greek}{ειϲ τουτων μιζω οψη και λεγι αυ} & 3 & \textbf{51} &  \\
&  & 3 & \foreignlanguage{greek}{τω αμην αμην λεγω υμιν} & 7 &  &  \\
&  & 8 & \foreignlanguage{greek}{οψεϲθαι τον \textoverline{ουρον} ανεωγοτα και} & 12 &  &  \\
&  & 13 & \foreignlanguage{greek}{τουϲ αγγελουϲ του \textoverline{θυ} αναβενονταϲ} & 17 &  &  \\
&  & 18 & \foreignlanguage{greek}{και καταβενονταϲ επι τον \textoverline{υν} του} & 23 &  &  \\
&  & 24 & \foreignlanguage{greek}{\textoverline{ανου} και τη ημερα τη \textoverline{γ} γαμοϲ} & 6 & \mygospelchapter &  \\
&  & 7 & \foreignlanguage{greek}{εγινετο εν κανα τηϲ γαλιλεαϲ} & 11 &  &  \\
&  & 12 & \foreignlanguage{greek}{και ην η \textoverline{μηρ} του \textoverline{ιυ} εκι εκληθη δε} & 2 & \textbf{2} &  \\
&  & 3 & \foreignlanguage{greek}{και ο \textoverline{ιϲ} εκι και οι μαθηται αυτου ειϲ} & 11 &  &  \\
&  & 12 & \foreignlanguage{greek}{τον γαμον και υϲτερηϲαντοϲ οινου} & 3 & \textbf{3} &  \\
&  & 4 & \foreignlanguage{greek}{λεγι η \textoverline{μηρ} του \textoverline{ιυ} προϲ αυτον οινον} & 11 &  &  \\
&  & 12 & \foreignlanguage{greek}{ουκ εχουϲιν και λεγει αυτη ο \textoverline{ιϲ} τι ε} & 7 & \textbf{4} &  \\
&  & 7 & \foreignlanguage{greek}{μοι και ϲυ γυναι ουπω ηκι η ωρα μου} & 15 &  &  \\
& \textbf{5} &  & \foreignlanguage{greek}{λεγι η \textoverline{μηρ} αυτου τοιϲ διακονοιϲ ο τι} & 8 &  &  \\
&  & 9 & \foreignlanguage{greek}{εαν λεγη υμιν ποιηϲατε ηϲαν δε εκι} & 3 & \textbf{6} &  \\
[0.2em]
\cline{4-4}
\end{tabular}
\end{center}
\end{table}
}
\clearpage
\newpage
 {
 \setlength\arrayrulewidth{1pt}
\begin{table}
\begin{center}
\begin{tabular}{ccc|l|ccc}
\cline{4-4} \\ [-1em]
\multicolumn{7}{c}{\foreignlanguage{greek}{ευαγγελιον κατα ιωαννην} \textbf{(\nospace{2:6})} } \\ \\ [-1em] % Si on veut ajouter les bordures latérales, remplacer {7}{c} par {7}{|c|}
\cline{4-4} \\
\cline{4-4}
&  &  & &  &  & \\ [-0.9em]
&  & 4 & \foreignlanguage{greek}{υδριε λιθινε εξ κατα τον καθαριϲ} & 9 &  &  \\
&  & 9 & \foreignlanguage{greek}{μον των ιουδεων κιμεναι χω} & 13 &  &  \\
&  & 13 & \foreignlanguage{greek}{ρουϲαι ανα μετρηταϲ \textoverline{β} η τριϲ και λε} & 2 & \textbf{7} &  \\
&  & 2 & \foreignlanguage{greek}{γι αυτοιϲ ο \textoverline{ιϲ} γεμιϲατε ταϲ υδριαϲ υ} & 9 &  &  \\
&  & 9 & \foreignlanguage{greek}{δατοϲ και εγεμιϲαν αυταϲ εωϲ ανω} & 14 &  &  \\
& \textbf{8} &  & \foreignlanguage{greek}{και λεγι αυτοιϲ αντληϲατε νυν} & 5 &  &  \\
&  & 6 & \foreignlanguage{greek}{και φερετε τω αρχιτρικλινω οι δε} & 11 &  &  \\
&  & 12 & \foreignlanguage{greek}{ηνεγκαν ωϲ δε εγευϲατο ο αρχιτρι} & 5 & \textbf{9} &  \\
&  & 5 & \foreignlanguage{greek}{κλινοϲ το υδωρ οινον γεγενημενο̅} & 9 &  &  \\
&  & 10 & \foreignlanguage{greek}{και ουκ ηδι ποθεν εϲτιν οι δε διακο} & 17 &  &  \\
&  & 17 & \foreignlanguage{greek}{νοι ηδιϲαν οι ηντληκοτεϲ το υδωρ} & 22 &  &  \\
&  & 23 & \foreignlanguage{greek}{φωνι τον νυμφιον ο αρχιτρικλινοϲ} & 27 &  &  \\
& \textbf{10} &  & \foreignlanguage{greek}{και λεγι αυτω παϲ \textoverline{ανοϲ} πρωτον τον} & 7 &  &  \\
&  & 8 & \foreignlanguage{greek}{καλον οινον τιθηϲιν και οταν με} & 13 &  &  \\
&  & 13 & \foreignlanguage{greek}{θυϲθωϲιν τον ελαϲϲω ϲυ τετηρη} & 17 &  &  \\
&  & 17 & \foreignlanguage{greek}{καϲ τον καλον οινον εωϲ αρτι ταυ} & 1 & \textbf{11} &  \\
&  & 1 & \foreignlanguage{greek}{την εποιηϲεν την αρχην των ϲη} & 6 &  &  \\
&  & 6 & \foreignlanguage{greek}{μιων ο \textoverline{ιϲ} εν κανα τηϲ γαλιλεαϲ} & 12 &  &  \\
&  & 13 & \foreignlanguage{greek}{και εφανερωϲεν την δοξαν αυτου} & 17 &  &  \\
&  & 18 & \foreignlanguage{greek}{και επιϲτευϲαν ειϲ αυτον οι μαθητε} & 23 &  &  \\
&  & 24 & \foreignlanguage{greek}{αυτου μετα τουτο κατεβη αυτοϲ} & 4 & \textbf{12} &  \\
&  & 5 & \foreignlanguage{greek}{και οι μαθητε αυτου και η \textoverline{μηρ} και} & 12 &  &  \\
&  & 13 & \foreignlanguage{greek}{οι αδελφοι αυτου και εμιναν ου} & 18 &  &  \\
&  & 19 & \foreignlanguage{greek}{πολλαϲ ημεραϲ και εγγυϲ ην το παϲ} & 5 & \textbf{13} &  \\
&  & 5 & \foreignlanguage{greek}{χα των ιουδεων και ανεβη ειϲ ιε} & 11 &  &  \\
&  & 11 & \foreignlanguage{greek}{ροϲολυμα ο \textoverline{ιϲ} και ευρεν εν τω ιερω} & 5 & \textbf{14} &  \\
&  & 6 & \foreignlanguage{greek}{τουϲ πωλουνταϲ βοαϲ και προβατα} & 10 &  &  \\
&  & 11 & \foreignlanguage{greek}{κε περιϲτεραϲ και τουϲ κολλυβιϲταϲ} & 15 &  &  \\
&  & 16 & \foreignlanguage{greek}{καθημενουϲ και ποιηϲαϲ ωϲ φραγελ} & 4 & \textbf{15} &  \\
&  & 4 & \foreignlanguage{greek}{λιον εχ ϲχοινιων πανταϲ εξεβαλεν} & 8 &  &  \\
[0.2em]
\cline{4-4}
\end{tabular}
\end{center}
\end{table}
}
\clearpage
\newpage
 {
 \setlength\arrayrulewidth{1pt}
\begin{table}
\begin{center}
\begin{tabular}{ccc|l|ccc}
\cline{4-4} \\ [-1em]
\multicolumn{7}{c}{\foreignlanguage{greek}{ευαγγελιον κατα ιωαννην} \textbf{(\nospace{2:15})} } \\ \\ [-1em] % Si on veut ajouter les bordures latérales, remplacer {7}{c} par {7}{|c|}
\cline{4-4} \\
\cline{4-4}
&  &  & &  &  & \\ [-0.9em]
&  & 9 & \foreignlanguage{greek}{εκ του ιερου τα τε προβατα και τουϲ} & 16 &  &  \\
&  & 17 & \foreignlanguage{greek}{βοαϲ και των κολλυβιϲτων εξεχεε̅} & 21 &  &  \\
&  & 22 & \foreignlanguage{greek}{τα κερματα και ταϲ τραπεζαϲ ανε} & 27 &  &  \\
&  & 27 & \foreignlanguage{greek}{τρεψεν και τοιϲ πωλουϲιν ταϲ πε} & 5 & \textbf{16} &  \\
&  & 5 & \foreignlanguage{greek}{ριϲτεραϲ ειπεν αρατε ταυτα εντευ} & 9 &  &  \\
&  & 9 & \foreignlanguage{greek}{θεν και μοη ποιειτε τον οικον του} & 15 &  &  \\
&  & 16 & \foreignlanguage{greek}{\textoverline{πρϲ} μου οικον ενποριου και εμνηϲ} & 2 & \textbf{17} &  \\
&  & 2 & \foreignlanguage{greek}{θηϲαν οι μαθηται αυτου οτι γεγραμ} & 7 &  &  \\
&  & 7 & \foreignlanguage{greek}{μενον εϲτιν οτι ο ζηλοϲ του οικου} & 13 &  &  \\
&  & 14 & \foreignlanguage{greek}{ϲου κατεφαγετε μαι απεκριθηϲα̅} & 1 & \textbf{18} &  \\
&  & 2 & \foreignlanguage{greek}{ουν οι ιουδεοι και ειπαν αυτω τι ϲη} & 9 &  &  \\
&  & 9 & \foreignlanguage{greek}{μιον δικνυειϲ ημιν οτι ταυτα ποιειϲ} & 14 &  &  \\
& \textbf{19} &  & \foreignlanguage{greek}{απεκριθη \textoverline{ιϲ} και ειπεν αυτοιϲ λυϲαται} & 6 &  &  \\
&  & 7 & \foreignlanguage{greek}{τον ναον τουτον και εν τριϲιν ημερεϲ} & 13 &  &  \\
&  & 14 & \foreignlanguage{greek}{εγερω αυτον ειπαν ουν οι ιουδεοι} & 4 & \textbf{20} &  \\
&  & 5 & \foreignlanguage{greek}{\textoverline{μ} και \textoverline{ϛ} ετεϲιν ο ναοϲ ουτοϲ οικοδο} & 12 &  &  \\
&  & 12 & \foreignlanguage{greek}{μηθη και ϲυ εν τριϲιν ημερεϲ γιριϲ} & 18 &  &  \\
&  & 19 & \foreignlanguage{greek}{αυτον αυτοϲ δε ελεγεν περι του να} & 6 & \textbf{21} &  \\
&  & 6 & \foreignlanguage{greek}{ου του ϲωματοϲ αυτου οτε ουν η} & 3 & \textbf{22} &  \\
&  & 3 & \foreignlanguage{greek}{νεϲτη εκ νεκρων εμνηϲθηϲα̅} & 6 &  &  \\
&  & 7 & \foreignlanguage{greek}{αυτω οτι τουτο ελεγεν και επιϲ} & 12 &  &  \\
&  & 12 & \foreignlanguage{greek}{τευϲαν τη γραφη και τω λογω ω ειπεν} & 19 &  &  \\
&  & 20 & \foreignlanguage{greek}{ο \textoverline{ιϲ} ωϲ δε ην εν τοιϲ ιεροϲολυ} & 6 & \textbf{23} &  \\
&  & 6 & \foreignlanguage{greek}{μοιϲ εν τω παϲχα εν τη εορτη πολ} & 13 &  &  \\
&  & 13 & \foreignlanguage{greek}{λοι επιϲτευϲαν ειϲ το ονομα αυτου} & 18 &  &  \\
&  & 19 & \foreignlanguage{greek}{θεωρουντεϲ αυτου τα ϲημια α εποιει} & 24 &  &  \\
& \textbf{24} &  & \foreignlanguage{greek}{αυτοϲ δε ο \textoverline{ιϲ} ουκ επιϲτευεν εαυτο̅} & 7 &  &  \\
&  & 8 & \foreignlanguage{greek}{αυτοιϲ δια το αυτον γινωϲκιν παν} & 13 &  &  \\
&  & 13 & \foreignlanguage{greek}{ταϲ και οτι ου χριαν ειχεν ινα τιϲ} & 7 & \textbf{25} &  \\
&  & 8 & \foreignlanguage{greek}{μαρτυρηϲη περι του \textoverline{ανου} αυτοϲ γαρ} & 13 &  &  \\
[0.2em]
\cline{4-4}
\end{tabular}
\end{center}
\end{table}
}
\clearpage
\newpage
 {
 \setlength\arrayrulewidth{1pt}
\begin{table}
\begin{center}
\begin{tabular}{ccc|l|ccc}
\cline{4-4} \\ [-1em]
\multicolumn{7}{c}{\foreignlanguage{greek}{ευαγγελιον κατα ιωαννην} \textbf{(\nospace{2:25})} } \\ \\ [-1em] % Si on veut ajouter les bordures latérales, remplacer {7}{c} par {7}{|c|}
\cline{4-4} \\
\cline{4-4}
&  &  & &  &  & \\ [-0.9em]
&  & 14 & \foreignlanguage{greek}{εγιγνωϲκεν τι ην εν τω \textoverline{ανω}} & 19 &  &  \\
& \mygospelchapter &  & \foreignlanguage{greek}{ην δε \textoverline{ανοϲ} εκ των φαριϲεων νικοδη} & 7 &  &  \\
&  & 7 & \foreignlanguage{greek}{μοϲ ονομα αυτω αρχων των ιου} & 12 &  &  \\
&  & 12 & \foreignlanguage{greek}{δεων ουτοϲ ηλθεν προϲ αυτον νυκτοϲ} & 5 & \textbf{2} &  \\
&  & 6 & \foreignlanguage{greek}{και ειπεν αυτω ραββι οιδαμεν οτι} & 11 &  &  \\
&  & 12 & \foreignlanguage{greek}{απο \textoverline{θυ} ελοιλυθαϲ διδαϲκαλοϲ ουδιϲ} & 16 &  &  \\
&  & 17 & \foreignlanguage{greek}{γαρ δυνατε τα ϲημια ταυτα ποιειν α} & 23 &  &  \\
&  & 24 & \foreignlanguage{greek}{ϲυ ποιειϲ εαν μη η ο \textoverline{θϲ} μετ αυτου} & 32 &  &  \\
& \textbf{3} &  & \foreignlanguage{greek}{απεκριθη \textoverline{ιϲ} και ειπεν αυτω αμην αμη̅} & 7 &  &  \\
&  & 8 & \foreignlanguage{greek}{λεγω ϲοι εαν μη τιϲ γεννηθη ανωθε̅} & 14 &  &  \\
&  & 15 & \foreignlanguage{greek}{ου δυνατε ειδιν την \textoverline{βλειαν} του \textoverline{θυ} λεγι} & 1 & \textbf{4} &  \\
&  & 2 & \foreignlanguage{greek}{προϲ αυτον νικοδημοϲ πωϲ δυναται} & 6 &  &  \\
&  & 7 & \foreignlanguage{greek}{\textoverline{ανοϲ} γεννηθηνε γερων ων μη δυ} & 12 &  &  \\
&  & 12 & \foreignlanguage{greek}{νατε ειϲ την κοιλιαν τηϲ \textoverline{μρϲ} αυτου} & 18 &  &  \\
&  & 19 & \foreignlanguage{greek}{δευτερον ειϲελθιν και γεννηθηνε} & 22 &  &  \\
& \textbf{5} &  & \foreignlanguage{greek}{απεκριθη \textoverline{ιϲ} αμην αμην λεγω ϲοι εαν} & 7 &  &  \\
&  & 8 & \foreignlanguage{greek}{μη τιϲ γεννηθη εξ υδατοϲ και \textoverline{πνϲ}} & 14 &  &  \\
&  & 15 & \foreignlanguage{greek}{ου δυνατε ειϲελθιν ειϲ την βαϲιλιαν} & 20 &  &  \\
&  & 21 & \foreignlanguage{greek}{του \textoverline{θυ} το γεγεννημενον εκ τηϲ ϲαρ} & 5 & \textbf{6} &  \\
&  & 5 & \foreignlanguage{greek}{κοϲ ϲαρξ εϲτιν και το γεγεννημε} & 10 &  &  \\
&  & 10 & \foreignlanguage{greek}{νον εκ του \textoverline{πνϲ} \textoverline{πνα} εϲτιν μη θαυ} & 2 & \textbf{7} &  \\
&  & 2 & \foreignlanguage{greek}{μαϲηϲ οτι ειπον ϲοι δι υμαϲ γεννη} & 8 &  &  \\
&  & 8 & \foreignlanguage{greek}{θηνε ανωθεν το \textoverline{πνα} οπου θελι πνι} & 5 & \textbf{8} &  \\
&  & 6 & \foreignlanguage{greek}{και την φωνην αυτου ακουειϲ αλ} & 11 &  &  \\
&  & 11 & \foreignlanguage{greek}{λ ουκ οιδαϲ ποθεν ερχεται και που υ} & 18 &  &  \\
&  & 18 & \foreignlanguage{greek}{παγει ουτωϲ εϲτιν παϲ ο γεγεννημε} & 23 &  &  \\
&  & 23 & \foreignlanguage{greek}{νοϲ εκ του \textoverline{πνϲ} απεκριθη νικο} & 2 & \textbf{9} &  \\
&  & 2 & \foreignlanguage{greek}{δημοϲ και ειπεν αυτω πωϲ δυνατε} & 7 &  &  \\
&  & 8 & \foreignlanguage{greek}{ταυτα γενεϲθαι απεκριθη \textoverline{ιϲ} και ει} & 4 & \textbf{10} &  \\
&  & 4 & \foreignlanguage{greek}{πεν αυτω ϲυ ει ο διδαϲκαλοϲ του \textoverline{ιηλ}} & 11 &  &  \\
[0.2em]
\cline{4-4}
\end{tabular}
\end{center}
\end{table}
}
\clearpage
\newpage
 {
 \setlength\arrayrulewidth{1pt}
\begin{table}
\begin{center}
\begin{tabular}{ccc|l|ccc}
\cline{4-4} \\ [-1em]
\multicolumn{7}{c}{\foreignlanguage{greek}{ευαγγελιον κατα ιωαννην} \textbf{(\nospace{3:10})} } \\ \\ [-1em] % Si on veut ajouter les bordures latérales, remplacer {7}{c} par {7}{|c|}
\cline{4-4} \\
\cline{4-4}
&  &  & &  &  & \\ [-0.9em]
&  & 12 & \foreignlanguage{greek}{και ταυτα ου γιγνωϲκιϲ αμην αμην} & 2 & \textbf{11} &  \\
&  & 3 & \foreignlanguage{greek}{λεγω ϲοι οτι ο οιδαμεν λαλουμεν} & 8 &  &  \\
&  & 9 & \foreignlanguage{greek}{και ο εορακαμεν μαρτυρουμεν και} & 13 &  &  \\
&  & 14 & \foreignlanguage{greek}{την μαρτυριαν ημων ου λαμβανε} & 18 &  &  \\
&  & 18 & \foreignlanguage{greek}{τε ει τα επιγια ειπον υμιν και ου πιϲ} & 8 & \textbf{12} &  \\
&  & 8 & \foreignlanguage{greek}{τευεται πωϲ εαν ειπω υμιν τα επου} & 14 &  &  \\
&  & 14 & \foreignlanguage{greek}{ρανια πιϲτευϲηται και ουδιϲ εϲτι̅} & 3 & \textbf{13} &  \\
&  & 4 & \foreignlanguage{greek}{οϲ ανεβη ειϲ τον \textoverline{ουρον} ει μη ο εκ του} & 13 &  &  \\
&  & 14 & \foreignlanguage{greek}{\textoverline{ουρου} καταβαϲ ο \textoverline{υϲ} του \textoverline{ανου} και καθωϲ} & 2 & \textbf{14} &  \\
&  & 3 & \foreignlanguage{greek}{μωυϲηϲ υψωϲεν τον οφιν εν τη ε} & 9 &  &  \\
&  & 9 & \foreignlanguage{greek}{ρημω ουτω δι υψωθηνε τον \textoverline{υν} του} & 15 &  &  \\
&  & 16 & \foreignlanguage{greek}{\textoverline{ανου} ινα παϲ ο πιϲτευων εν αυτω} & 6 & \textbf{15} &  \\
&  & 7 & \foreignlanguage{greek}{εχη ζωην αιωνιον ουτωϲ γαρ ηγαπη} & 3 & \textbf{16} &  \\
&  & 3 & \foreignlanguage{greek}{ϲεν ο \textoverline{θϲ} τον κοϲμον ωϲτε τον \textoverline{υν} τον} & 11 &  &  \\
&  & 12 & \foreignlanguage{greek}{μονογενη εδωκεν ινα παϲ ο πιϲ} & 17 &  &  \\
&  & 17 & \foreignlanguage{greek}{τευων ειϲ αυτον μη αποληται} & 21 &  &  \\
&  & 22 & \foreignlanguage{greek}{αλλα εχη ζωην αιωνιον ου γαρ α} & 3 & \textbf{17} &  \\
&  & 3 & \foreignlanguage{greek}{πεϲτιλεν ο \textoverline{θϲ} τον \textoverline{υν} ειϲ τον κοϲμο̅} & 10 &  &  \\
&  & 11 & \foreignlanguage{greek}{ινα κρινη τον κοϲμον αλλ ινα ϲω} & 17 &  &  \\
&  & 17 & \foreignlanguage{greek}{θη ο κοϲμοϲ δι αυτου ο πιϲτευων} & 2 & \textbf{18} &  \\
&  & 3 & \foreignlanguage{greek}{ειϲ αυτον ου κρινεται ο μη πιϲτευ} & 9 &  &  \\
&  & 9 & \foreignlanguage{greek}{ων ηδη κεκριται οτι μη πεπιϲτευ} & 14 &  &  \\
&  & 14 & \foreignlanguage{greek}{κεν ειϲ το ονομα του μονογενουϲ \textoverline{υυ}} & 20 &  &  \\
&  & 21 & \foreignlanguage{greek}{του \textoverline{θυ} αυτη δε εϲτιν η κριϲιϲ οτι} & 6 & \textbf{19} &  \\
&  & 7 & \foreignlanguage{greek}{το φωϲ εληλυθεν ειϲ τον κοϲμον} & 12 &  &  \\
&  & 13 & \foreignlanguage{greek}{και ηγαπηϲαν οι \textoverline{ανοι} μαλλον το ϲκο} & 19 &  &  \\
&  & 19 & \foreignlanguage{greek}{τοϲ η το φωϲ ην γαρ αυτων πονη} & 26 &  &  \\
&  & 26 & \foreignlanguage{greek}{ρα τα εργα παϲ γαρ ο φαυλα πραϲϲω̅} & 5 & \textbf{20} &  \\
&  & 6 & \foreignlanguage{greek}{μιϲι το φωϲ και ουκ ερχετε προϲ το} & 13 &  &  \\
&  & 14 & \foreignlanguage{greek}{φωϲ ινα μη ελεγχθη αυτου τα εργα} & 20 &  &  \\
[0.2em]
\cline{4-4}
\end{tabular}
\end{center}
\end{table}
}
\clearpage
\newpage
 {
 \setlength\arrayrulewidth{1pt}
\begin{table}
\begin{center}
\begin{tabular}{ccc|l|ccc}
\cline{4-4} \\ [-1em]
\multicolumn{7}{c}{\foreignlanguage{greek}{ευαγγελιον κατα ιωαννην} \textbf{(\nospace{3:21})} } \\ \\ [-1em] % Si on veut ajouter les bordures latérales, remplacer {7}{c} par {7}{|c|}
\cline{4-4} \\
\cline{4-4}
&  &  & &  &  & \\ [-0.9em]
& \textbf{21} &  & \foreignlanguage{greek}{ο δε ποιων την αληθιαν ερχετε προϲ} & 7 &  &  \\
&  & 8 & \foreignlanguage{greek}{το φωϲ ινα φανερωθη αυτου τα} & 13 &  &  \\
&  & 14 & \foreignlanguage{greek}{εργα οτι εν \textoverline{θω} ειϲιν ιργαϲμενα} & 19 &  &  \\
& \textbf{22} &  & \foreignlanguage{greek}{μετα ταυτα ηλθεν ο \textoverline{ιϲ} και οι μαθηται} & 8 &  &  \\
&  & 9 & \foreignlanguage{greek}{αυτου ειϲ την ιουδεαν γην κακι} & 15 &  &  \\
&  & 16 & \foreignlanguage{greek}{διετριβεν μετ αυτων και εβαπτι} & 20 &  &  \\
&  & 20 & \foreignlanguage{greek}{ζεν ην δε και ο ιωαννηϲ βαπτιζω̅} & 6 & \textbf{23} &  \\
&  & 7 & \foreignlanguage{greek}{εν ενων ενγυϲ του ϲαλιμ οτι υδα} & 13 &  &  \\
&  & 13 & \foreignlanguage{greek}{τα πολλα ην εκει και παρεγινοντο} & 18 &  &  \\
&  & 19 & \foreignlanguage{greek}{και εβαπτιζοντο ουπω γαρ ην βεβλη} & 4 & \textbf{24} &  \\
&  & 4 & \foreignlanguage{greek}{μενοϲ ειϲ την φυλακην ο ιωαννηϲ} & 9 &  &  \\
& \textbf{25} &  & \foreignlanguage{greek}{εγενετο ουν ζητηϲιϲ εκ των μαθητων} & 6 &  &  \\
&  & 7 & \foreignlanguage{greek}{ιωαννου μετα ιουδεου περι καθα} & 11 &  &  \\
&  & 11 & \foreignlanguage{greek}{ριϲμου και ηλθαν προϲ τον ιωαννην} & 5 & \textbf{26} &  \\
&  & 6 & \foreignlanguage{greek}{και ειπαν αυτω ραββει οϲ ην μετα ϲου} & 13 &  &  \\
&  & 14 & \foreignlanguage{greek}{περαν του ιορδανου ω ϲυ μεμαρτυ} & 19 &  &  \\
&  & 19 & \foreignlanguage{greek}{ρηκαϲ ειδε ουτοϲ βαπτιζι και παν} & 24 &  &  \\
&  & 24 & \foreignlanguage{greek}{τεϲ ερχοντε προϲ αυτον απεκριθη} & 1 & \textbf{27} &  \\
&  & 2 & \foreignlanguage{greek}{ιωαννηϲ και ειπεν ου δυνατε \textoverline{ανοϲ}} & 7 &  &  \\
&  & 8 & \foreignlanguage{greek}{λαμβανιν ουδεν εαν μη η δεδο} & 13 &  &  \\
&  & 13 & \foreignlanguage{greek}{μενον αυτω εκ του \textoverline{ουρου} αυτοι} & 1 & \textbf{28} &  \\
&  & 2 & \foreignlanguage{greek}{υμιϲ μοι μαρτυρειται οτι ειπον ου} & 7 &  &  \\
&  & 7 & \foreignlanguage{greek}{κ ιμι ο \textoverline{χϲ} αλλ οτι απεϲταλμενοϲ ειμι} & 14 &  &  \\
&  & 15 & \foreignlanguage{greek}{εμπροϲθεν εκινου ο εχων την} & 3 & \textbf{29} &  \\
&  & 4 & \foreignlanguage{greek}{νυμφην νυμφιοϲ εϲτιν ο δε} & 8 &  &  \\
&  & 9 & \foreignlanguage{greek}{φιλοϲ του νυμφιου ο εϲτηκωϲ} & 13 &  &  \\
&  & 14 & \foreignlanguage{greek}{και ακουων αυτου χαρα χαιρι δια} & 19 &  &  \\
&  & 20 & \foreignlanguage{greek}{την φωνην του νυμφιου αυτη} & 24 &  &  \\
&  & 25 & \foreignlanguage{greek}{ουν η χαρα η εμη πεπληρωται} & 30 &  &  \\
& \textbf{30} &  & \foreignlanguage{greek}{εκινον διαυξανειν εμε δε ελατ} & 5 &  &  \\
[0.2em]
\cline{4-4}
\end{tabular}
\end{center}
\end{table}
}
\clearpage
\newpage
 {
 \setlength\arrayrulewidth{1pt}
\begin{table}
\begin{center}
\begin{tabular}{ccc|l|ccc}
\cline{4-4} \\ [-1em]
\multicolumn{7}{c}{\foreignlanguage{greek}{ευαγγελιον κατα ιωαννην} \textbf{(\nospace{3:30})} } \\ \\ [-1em] % Si on veut ajouter les bordures latérales, remplacer {7}{c} par {7}{|c|}
\cline{4-4} \\
\cline{4-4}
&  &  & &  &  & \\ [-0.9em]
&  & 5 & \foreignlanguage{greek}{τουϲθαι ο ανοθεν ερχομενοϲ επα} & 4 & \textbf{31} &  \\
&  & 4 & \foreignlanguage{greek}{νω παντων εϲτιν ο ων εκ τηϲ γηϲ} & 11 &  &  \\
&  & 12 & \foreignlanguage{greek}{εκ τηϲ γηϲ λαλει ο εκ του \textoverline{ουρου} ερ} & 20 &  &  \\
&  & 20 & \foreignlanguage{greek}{χομενοϲ επανω παντων εϲτιν ο ε} & 2 & \textbf{32} &  \\
&  & 2 & \foreignlanguage{greek}{ωρακεν και ηκουϲεν τουτο μαρτυρι} & 6 &  &  \\
&  & 7 & \foreignlanguage{greek}{και την μαρτυριαν αυτου ουδιϲ λαμ} & 12 &  &  \\
&  & 12 & \foreignlanguage{greek}{βανι ο λαβων αυτου την μαρτυρι} & 5 & \textbf{33} &  \\
&  & 5 & \foreignlanguage{greek}{αν εϲφραγιϲεν οτι ο \textoverline{θϲ} αληθηϲ εϲτι̅} & 11 &  &  \\
& \textbf{34} &  & \foreignlanguage{greek}{ον γαρ απεϲτιλεν ο \textoverline{θϲ} τα ρηματα του} & 8 &  &  \\
&  & 9 & \foreignlanguage{greek}{\textoverline{θυ} λαλει ου γαρ εκ μετρου διδωϲιν} & 15 &  &  \\
&  & 16 & \foreignlanguage{greek}{το \textoverline{πνα} ο \textoverline{πηρ} αγαπα τον \textoverline{υν} και παν} & 7 & \textbf{35} &  \\
&  & 7 & \foreignlanguage{greek}{τα δεδωκεν εν τη χιρι αυτου ο πιϲ} & 2 & \textbf{36} &  \\
&  & 2 & \foreignlanguage{greek}{τευων ειϲ τον \textoverline{υν} εχι ζωην αιωνιον} & 8 &  &  \\
&  & 9 & \foreignlanguage{greek}{ο δε απιθων τω \textoverline{υω} ουχ οψετε ζωην} & 16 &  &  \\
&  & 17 & \foreignlanguage{greek}{αλλ η οργη του \textoverline{θυ} μενι επ αυτον} & 24 &  &  \\
& \mygospelchapter &  & \foreignlanguage{greek}{ωϲ ουν εγνω ο \textoverline{κϲ} οτι ηκουϲαν οι φα} & 9 &  &  \\
&  & 9 & \foreignlanguage{greek}{ριϲεοι οτι \textoverline{ιϲ} πλιοναϲ μαθηταϲ ποιει} & 14 &  &  \\
&  & 15 & \foreignlanguage{greek}{και βαπτιζι ιωαννηϲ καιτοιγε} & 1 & \textbf{2} &  \\
&  & 2 & \foreignlanguage{greek}{\textoverline{ιϲ} αυτοϲ ουκ εβαπτιζεν αλλ οι} & 7 &  &  \\
&  & 8 & \foreignlanguage{greek}{μαθηται αυτου αφηκεν την} & 2 & \textbf{3} &  \\
&  & 3 & \foreignlanguage{greek}{ιουδεαν και απηλθεν παλιν ειϲ τη̅} & 8 &  &  \\
&  & 9 & \foreignlanguage{greek}{γαλιλεαν εδι δε αυτον διερχεϲ} & 4 & \textbf{4} &  \\
&  & 4 & \foreignlanguage{greek}{θαι δια τηϲ ϲαμαριαϲ ερχετε ουν} & 2 & \textbf{5} &  \\
&  & 3 & \foreignlanguage{greek}{ειϲ πολιν τηϲ ϲαμαριαϲ λεγομε} & 7 &  &  \\
&  & 7 & \foreignlanguage{greek}{νην ϲυχαρ πληϲιον του χωριου} & 11 &  &  \\
&  & 12 & \foreignlanguage{greek}{ου εδωκεν ιακωβ ιωϲηφ τω \textoverline{υω}} & 17 &  &  \\
&  & 18 & \foreignlanguage{greek}{αυτου ην δε εκι πηγη του ιακωβ} & 6 & \textbf{6} &  \\
&  & 7 & \foreignlanguage{greek}{ο ουν \textoverline{ιϲ} κεκοπιακωϲ εκ τηϲ οδη} & 13 &  &  \\
&  & 13 & \foreignlanguage{greek}{ποριαϲ εκαθεζετο ουτωϲ επι τη} & 17 &  &  \\
&  & 18 & \foreignlanguage{greek}{πηγη ωρα δε ην ωϲ εκτη και ερ} & 2 & \textbf{7} &  \\
[0.2em]
\cline{4-4}
\end{tabular}
\end{center}
\end{table}
}
\clearpage
\newpage
 {
 \setlength\arrayrulewidth{1pt}
\begin{table}
\begin{center}
\begin{tabular}{ccc|l|ccc}
\cline{4-4} \\ [-1em]
\multicolumn{7}{c}{\foreignlanguage{greek}{ευαγγελιον κατα ιωαννην} \textbf{(\nospace{4:7})} } \\ \\ [-1em] % Si on veut ajouter les bordures latérales, remplacer {7}{c} par {7}{|c|}
\cline{4-4} \\
\cline{4-4}
&  &  & &  &  & \\ [-0.9em]
&  & 2 & \foreignlanguage{greek}{χαιται γυνη εκ τηϲ ϲαμαριαϲ αντλη} & 7 &  &  \\
&  & 7 & \foreignlanguage{greek}{ϲε υδωρ λεγι αυτη ο \textoverline{ιϲ} δοϲ μοι πιειν} & 15 &  &  \\
& \textbf{8} &  & \foreignlanguage{greek}{οι γαρ μαθητε αυτου απεληλυθιϲαν} & 5 &  &  \\
&  & 6 & \foreignlanguage{greek}{ειϲ την πολιν ινα τροφαϲ αγοραϲωϲι̅} & 11 &  &  \\
& \textbf{9} &  & \foreignlanguage{greek}{λεγι ουν αυτω η γυνη η ϲαμαριτιϲ} & 7 &  &  \\
&  & 8 & \foreignlanguage{greek}{πωϲ ϲυ ειουδεοϲ ων παρ εμου πιν} & 14 &  &  \\
&  & 14 & \foreignlanguage{greek}{ετιϲ γυναικοϲ ϲαμαριτιδοϲ ουϲηϲ} & 17 &  &  \\
&  & 18 & \foreignlanguage{greek}{ου γαρ ϲυνχρωνται ιουδεοι ϲαμαρι} & 22 &  &  \\
&  & 22 & \foreignlanguage{greek}{ταιϲ απεκριθη \textoverline{ιϲ} και ειπεν αυτη} & 5 & \textbf{10} &  \\
&  & 6 & \foreignlanguage{greek}{ει ηδιϲ την δωρεαν του \textoverline{θυ} και τιϲ} & 13 &  &  \\
&  & 14 & \foreignlanguage{greek}{εϲτιν ο λεγων ϲοι δοϲ μοι πιν ϲυ αν} & 22 &  &  \\
&  & 23 & \foreignlanguage{greek}{ητηϲαϲ αυτον και εδωκεν αν ϲοι} & 29 &  &  \\
&  & 30 & \foreignlanguage{greek}{υδωρ ζων λεγι αυτω η γυνη} & 4 & \textbf{11} &  \\
&  & 5 & \foreignlanguage{greek}{\textoverline{κε} το φρεαρ εϲτιν βαθυ και ουτε} & 11 &  &  \\
&  & 12 & \foreignlanguage{greek}{αντλημα εχιϲ και ποθεν εϲτιν} & 16 &  &  \\
&  & 17 & \foreignlanguage{greek}{το υδωρ το ζων μη ϲυ μιζον ει του} & 5 & \textbf{12} &  \\
&  & 6 & \foreignlanguage{greek}{\textoverline{πρϲ} ημων ιακωβ οϲ εδωκεν ημιν} & 11 &  &  \\
&  & 12 & \foreignlanguage{greek}{το φρεαρ το ζων και αυτοϲ εξ αυτου} & 19 &  &  \\
&  & 20 & \foreignlanguage{greek}{επιεν και οι \textoverline{υι} αυτου και τα θρεμμα} & 27 &  &  \\
&  & 27 & \foreignlanguage{greek}{τα αυτου απεκριθη \textoverline{ιϲ} και ειπεν} & 4 & \textbf{13} &  \\
&  & 5 & \foreignlanguage{greek}{αυτη παϲ ο πινων εκ του υδατοϲ} & 11 &  &  \\
&  & 12 & \foreignlanguage{greek}{τουτου διψηϲι παλιν οϲ δ αν δε πιη} & 5 & \textbf{14} &  \\
&  & 6 & \foreignlanguage{greek}{εκ του υδατοϲ ου εγω δωϲω αυτω} & 12 &  &  \\
&  & 13 & \foreignlanguage{greek}{ου μη διψηϲει ειϲ τον αιωνα αλλα} & 19 &  &  \\
&  & 20 & \foreignlanguage{greek}{το υδωρ ο δωϲω αυτω γενηϲεται} & 25 &  &  \\
&  & 26 & \foreignlanguage{greek}{εν αυτω πηγη υδατοϲ αλλομενου} & 30 &  &  \\
&  & 31 & \foreignlanguage{greek}{ειϲ ζωην αιωνιον λεγι προϲ αυτον} & 3 & \textbf{15} &  \\
&  & 4 & \foreignlanguage{greek}{η γυνη \textoverline{κε} δοϲ μοι τουτο το υδωρ} & 11 &  &  \\
&  & 12 & \foreignlanguage{greek}{ινα μη διψω μηδε ερχωμε ενθα} & 17 &  &  \\
&  & 17 & \foreignlanguage{greek}{δε αντλιν λεγι αυτη ο \textoverline{ιϲ} υπαγε φω} & 6 & \textbf{16} &  \\
[0.2em]
\cline{4-4}
\end{tabular}
\end{center}
\end{table}
}
\clearpage
\newpage
 {
 \setlength\arrayrulewidth{1pt}
\begin{table}
\begin{center}
\begin{tabular}{ccc|l|ccc}
\cline{4-4} \\ [-1em]
\multicolumn{7}{c}{\foreignlanguage{greek}{ευαγγελιον κατα ιωαννην} \textbf{(\nospace{4:16})} } \\ \\ [-1em] % Si on veut ajouter les bordures latérales, remplacer {7}{c} par {7}{|c|}
\cline{4-4} \\
\cline{4-4}
&  &  & &  &  & \\ [-0.9em]
&  & 6 & \foreignlanguage{greek}{νηϲον τον ανδρα ϲου και ελθε ενθαδε} & 12 &  &  \\
& \textbf{17} &  & \foreignlanguage{greek}{απεκριθη η γυνη και ειπεν ουκ εχω ανδρα} & 8 &  &  \\
&  & 9 & \foreignlanguage{greek}{λεγι αυτη \textoverline{ιϲ} καλωϲ ειπαϲ οτι ανδρα ου} & 16 &  &  \\
&  & 16 & \foreignlanguage{greek}{κ εχω \textoverline{ε} γαρ ανδραϲ εϲχεϲ και νυν ον εχιϲ} & 8 & \textbf{18} &  \\
&  & 9 & \foreignlanguage{greek}{ουκ εϲτιν ϲου ανηρ τουτο αληθεϲ ειρηκαϲ} & 15 &  &  \\
& \textbf{19} &  & \foreignlanguage{greek}{λεγι αυτω η γυνη \textoverline{κε} θεωρω οτι προφητηϲ} & 8 &  &  \\
&  & 9 & \foreignlanguage{greek}{ει ϲυ οι \textoverline{πρεϲ} ημων εν τω ορι τουτω προϲεκυ} & 8 & \textbf{20} &  \\
&  & 8 & \foreignlanguage{greek}{νηϲαν και υμιϲ λεγετε οτι εν ιεροϲολυ} & 14 &  &  \\
&  & 14 & \foreignlanguage{greek}{μοιϲ εϲτιν ο τοποϲ οπου προϲκυνιν δει} & 20 &  &  \\
& \textbf{21} &  & \foreignlanguage{greek}{λεγι αυτη ο \textoverline{ιϲ} πιϲτευε μοι γυναι οτι ερχετε ωρα} & 10 &  &  \\
&  & 11 & \foreignlanguage{greek}{οτε ουτε εν τω ορι τουτω ουτε εν ιερο} & 19 &  &  \\
&  & 19 & \foreignlanguage{greek}{ϲολυμοιϲ προϲκυνηϲεται τω \textoverline{πρι} υμιϲ} & 1 & \textbf{22} &  \\
&  & 2 & \foreignlanguage{greek}{προϲκυνειται ο ουκ οιδαται ημιϲ προϲκυ} & 8 &  &  \\
&  & 8 & \foreignlanguage{greek}{νουμεν ο οιδαμεν οτι η ϲωτηρια εκ των} & 15 &  &  \\
&  & 16 & \foreignlanguage{greek}{ιουδεων εϲτιν αλλα ερχετε ωρα και} & 4 & \textbf{23} &  \\
&  & 5 & \foreignlanguage{greek}{νυν εϲτιν οτε οι αληθινοι προϲκυνη} & 10 &  &  \\
&  & 10 & \foreignlanguage{greek}{ται προϲκυνηϲουϲιν τω \textoverline{πρι} εν \textoverline{πνι}} & 15 &  &  \\
&  & 16 & \foreignlanguage{greek}{και αληθια και γαρ ο \textoverline{πηρ} τοιουτουϲ ζη} & 23 &  &  \\
&  & 23 & \foreignlanguage{greek}{τι τουϲ προϲκυνουταϲ αυτον εν \textoverline{πνι}} & 28 &  &  \\
& \textbf{24} &  & \foreignlanguage{greek}{\textoverline{πνα} οϲ και τουϲ προϲκυνουνταϲ αυτον} & 6 &  &  \\
&  & 7 & \foreignlanguage{greek}{εν \textoverline{πνι} και αληθια δι προϲκυνιν λεγι} & 1 & \textbf{25} &  \\
&  & 2 & \foreignlanguage{greek}{αυτω η γυνη οιδα οτι μεϲϲιαϲ ερχεται} & 8 &  &  \\
&  & 9 & \foreignlanguage{greek}{ο λεγομενοϲ \textoverline{χϲ} οταν ελθη εκινοϲ αναγ} & 16 &  &  \\
&  & 16 & \foreignlanguage{greek}{γελλι ημιν απαντα λεγι αυτη ο \textoverline{ιϲ}} & 4 & \textbf{26} &  \\
&  & 5 & \foreignlanguage{greek}{εγω ειμι ο λαλων ϲοι και επι τουτω} & 3 & \textbf{27} &  \\
&  & 4 & \foreignlanguage{greek}{ηλθον οι μαθηται αυτου και εθαυμαζο̅} & 9 &  &  \\
&  & 10 & \foreignlanguage{greek}{οτι μετα γυναικοϲ λαλει ουδιϲ μεντοι} & 15 &  &  \\
&  & 16 & \foreignlanguage{greek}{γε ειπεν τι ζητιϲ η τι λαλιϲ μετ αυτηϲ} & 24 &  &  \\
& \textbf{28} &  & \foreignlanguage{greek}{αφηκεν ουν την υδριαν αυτηϲ η γυνη} & 7 &  &  \\
&  & 8 & \foreignlanguage{greek}{και απηλθεν ειϲ την πολιν και λεγι τοιϲ} & 15 &  &  \\
&  & 16 & \foreignlanguage{greek}{\textoverline{ανοιϲ} δευτε ειδετε \textoverline{ανον} οϲ ειπεν} & 5 & \textbf{29} &  \\
[0.2em]
\cline{4-4}
\end{tabular}
\end{center}
\end{table}
}
\clearpage
\newpage
 {
 \setlength\arrayrulewidth{1pt}
\begin{table}
\begin{center}
\begin{tabular}{ccc|l|ccc}
\cline{4-4} \\ [-1em]
\multicolumn{7}{c}{\foreignlanguage{greek}{ευαγγελιον κατα ιωαννην} \textbf{(\nospace{4:29})} } \\ \\ [-1em] % Si on veut ajouter les bordures latérales, remplacer {7}{c} par {7}{|c|}
\cline{4-4} \\
\cline{4-4}
&  &  & &  &  & \\ [-0.9em]
&  & 6 & \foreignlanguage{greek}{παντα οϲα εποιηϲα μητι ουτοϲ εϲτιν} & 11 &  &  \\
&  & 12 & \foreignlanguage{greek}{ο \textoverline{χϲ} εξηλθον ουν εκ τηϲ πολεωϲ και} & 6 & \textbf{30} &  \\
&  & 7 & \foreignlanguage{greek}{ηρχοντο προϲ αυτον και εν τω με} & 4 & \textbf{31} &  \\
&  & 4 & \foreignlanguage{greek}{ταξυ ηρωτων αυτον οι μαθηται αυ} & 9 &  &  \\
&  & 9 & \foreignlanguage{greek}{του λεγοντεϲ ραββει φαγε ο δε ειπεν} & 3 & \textbf{32} &  \\
&  & 4 & \foreignlanguage{greek}{αυτοιϲ εγω βρωϲιν εχω φαγιν ην υ} & 10 &  &  \\
&  & 10 & \foreignlanguage{greek}{μιϲ ουκ οιδαται ελεγον ουν οι μαθη} & 4 & \textbf{33} &  \\
&  & 4 & \foreignlanguage{greek}{ται προϲ αλληλουϲ μη τιϲ ηνεγκεν} & 9 &  &  \\
&  & 10 & \foreignlanguage{greek}{αυτω φαγιν λεγι αυτοιϲ ο \textoverline{ιϲ} εμον βρω} & 6 & \textbf{34} &  \\
&  & 6 & \foreignlanguage{greek}{μα εϲτιν ινα ποιηϲω το θελημα του πεμ} & 13 &  &  \\
&  & 13 & \foreignlanguage{greek}{ψαντοϲ με και τελειωϲω αυτου το εργον} & 19 &  &  \\
& \textbf{35} &  & \foreignlanguage{greek}{ουχ υμιϲ λεγεται οτι ετι τετραμηνον εϲτιν} & 7 &  &  \\
&  & 8 & \foreignlanguage{greek}{και ο θεριϲμοϲ ερχεται ιδου λεγω υμιν} & 14 &  &  \\
&  & 15 & \foreignlanguage{greek}{επαρατε τουϲ οφθαλμουϲ υμων και θεαϲαϲθαι} & 20 &  &  \\
&  & 21 & \foreignlanguage{greek}{ταϲ χωραϲ οτι λευκαι ειϲιν προϲ θεριϲμο̅} & 27 &  &  \\
&  & 28 & \foreignlanguage{greek}{ηδη ο θεριζων μιϲθον λαμβανι και} & 5 & \textbf{36} &  \\
&  & 6 & \foreignlanguage{greek}{ϲυναγι καρπον ειϲ ζων αιωνιον ινα} & 11 &  &  \\
&  & 12 & \foreignlanguage{greek}{ο ϲπειρων ομου χερη και ο θεριζων εν} & 1 & \textbf{37} &  \\
&  & 2 & \foreignlanguage{greek}{γαρ τουτω ο λογοϲ εϲτιν αληθινοϲ οτι αλ} & 9 &  &  \\
&  & 9 & \foreignlanguage{greek}{λοϲ εϲτιν ο ϲπιρων και αλλοϲ ο θεριζων} & 16 &  &  \\
& \textbf{38} &  & \foreignlanguage{greek}{εγω απεϲτιλα υμαϲ θεριζιν ουχ υμιϲ} & 6 &  &  \\
&  & 7 & \foreignlanguage{greek}{κεοπιακαται αλλοι κεκοπιακαϲιν} & 9 &  &  \\
&  & 10 & \foreignlanguage{greek}{και υμιϲ ειϲ τον κοπον αυτων ειϲεληλυ} & 16 &  &  \\
&  & 16 & \foreignlanguage{greek}{θαται εκ δε τηϲ πολεωϲ εκινηϲ πολλοι ε} & 7 & \textbf{39} &  \\
&  & 7 & \foreignlanguage{greek}{πιϲτευϲαν ειϲ αυτον των ϲαμαριτων} & 11 &  &  \\
&  & 12 & \foreignlanguage{greek}{δια τον λογον τηϲ γυναικοϲ μαρτυρουϲηϲ} & 17 &  &  \\
&  & 18 & \foreignlanguage{greek}{οτι ειπεν μοι παντα οϲα εποιηϲα ωϲ ουν ηλ} & 3 & \textbf{40} &  \\
&  & 3 & \foreignlanguage{greek}{θον προϲ αυτον οι ϲαμαριται ηρωτουν αυ} & 9 &  &  \\
&  & 9 & \foreignlanguage{greek}{τον μινε παρ αυτοιϲ και εμινεν εκει \textoverline{β} ημεραϲ} & 17 &  &  \\
& \textbf{41} &  & \foreignlanguage{greek}{και πολλω πλιουϲ επιϲτευϲαν δια τον λογο̅} & 7 &  &  \\
&  & 8 & \foreignlanguage{greek}{αυτου τη τε γυναικι ελεγον ουκετι} & 5 & \textbf{42} &  \\
[0.2em]
\cline{4-4}
\end{tabular}
\end{center}
\end{table}
}
\clearpage
\newpage
 {
 \setlength\arrayrulewidth{1pt}
\begin{table}
\begin{center}
\begin{tabular}{ccc|l|ccc}
\cline{4-4} \\ [-1em]
\multicolumn{7}{c}{\foreignlanguage{greek}{ευαγγελιον κατα ιωαννην} \textbf{(\nospace{4:42})} } \\ \\ [-1em] % Si on veut ajouter les bordures latérales, remplacer {7}{c} par {7}{|c|}
\cline{4-4} \\
\cline{4-4}
&  &  & &  &  & \\ [-0.9em]
&  & 6 & \foreignlanguage{greek}{δια την ϲην λαλιαν πιϲτευομεν αυτοι} & 11 &  &  \\
&  & 12 & \foreignlanguage{greek}{γαρ ακηκοαμεν και οιδαμεν οτι ουτοϲ} & 17 &  &  \\
&  & 18 & \foreignlanguage{greek}{εϲτιν ο \textoverline{ϲηρ} του κοϲμου μετα δε ταϲ \textoverline{β} η} & 5 & \textbf{43} &  \\
&  & 5 & \foreignlanguage{greek}{μεραϲ εξηλθεν εκιθεν ειϲ την γαλιλεα̅} & 10 &  &  \\
& \textbf{44} &  & \foreignlanguage{greek}{αυτοϲ γαρ \textoverline{ιϲ} εμαρτυρηϲεν οτι προφητηϲ} & 6 &  &  \\
&  & 7 & \foreignlanguage{greek}{εν τη ιδια πατριδι τιμην ουκ εχι οτε ου̅} & 2 & \textbf{45} &  \\
&  & 3 & \foreignlanguage{greek}{ηλθεν ειϲ την γαλιλεαν εδεξαντο αυτο̅} & 8 &  &  \\
&  & 9 & \foreignlanguage{greek}{οι γαλιλεοι παντα εορακοτεϲ οϲα εποιηϲεν} & 14 &  &  \\
&  & 15 & \foreignlanguage{greek}{εν τοιϲ ιεροϲολυμοιϲ εν τη εορτη και} & 21 &  &  \\
&  & 22 & \foreignlanguage{greek}{γαρ ηλθον ειϲ την εορτην ηλθεν ουν} & 2 & \textbf{46} &  \\
&  & 3 & \foreignlanguage{greek}{παλιν ειϲ την κανα τηϲ γαλιλεαϲ οπου} & 9 &  &  \\
&  & 10 & \foreignlanguage{greek}{εποιηϲεν το υδωρ οινον και ην τιϲ βαϲι} & 17 &  &  \\
&  & 17 & \foreignlanguage{greek}{λικοϲ ου ο \textoverline{υϲ} ηϲθενι εν καφαρναουμ} & 23 &  &  \\
& \textbf{47} &  & \foreignlanguage{greek}{ουτοϲ ακουϲαϲ οτι \textoverline{ιϲ} ηκεν εκ τηϲ ιου} & 8 &  &  \\
&  & 8 & \foreignlanguage{greek}{δεαϲ ειϲ την γαλιλεαν απηλθεν} & 12 &  &  \\
&  & 13 & \foreignlanguage{greek}{προϲ αυτον και ηρωτα ινα καταβη} & 18 &  &  \\
&  & 19 & \foreignlanguage{greek}{και ιαϲητε αυτου τον \textoverline{υν} ημελλεν γαρ} & 25 &  &  \\
&  & 26 & \foreignlanguage{greek}{αποθνηϲκιν ειπεν ουν \textoverline{ιϲ} προϲ αυ} & 5 & \textbf{48} &  \\
&  & 5 & \foreignlanguage{greek}{τον εαν μη ϲημια και τερατα ειδη} & 11 &  &  \\
&  & 11 & \foreignlanguage{greek}{τε ου μη πιϲτευϲηται λεγι προϲ αυ} & 3 & \textbf{49} &  \\
&  & 3 & \foreignlanguage{greek}{τον ο βαϲιλικοϲ \textoverline{κε} καταβηθι πριν} & 8 &  &  \\
&  & 9 & \foreignlanguage{greek}{αποθανιν το παιδιον μου λεγι αυ} & 2 & \textbf{50} &  \\
&  & 2 & \foreignlanguage{greek}{τω ο \textoverline{ιϲ} πορευου ο \textoverline{υϲ} ϲου ζη επιϲτευ} & 10 &  &  \\
&  & 10 & \foreignlanguage{greek}{ϲεν ο \textoverline{ανοϲ} τω λογω ω ειπεν αυτω \textoverline{ιϲ}} & 18 &  &  \\
&  & 19 & \foreignlanguage{greek}{και επορευετο ηδη δε αυτου καταβε} & 4 & \textbf{51} &  \\
&  & 4 & \foreignlanguage{greek}{νοντοϲ υπηντηϲαν αυτω οι δουλοι αυ} & 9 &  &  \\
&  & 9 & \foreignlanguage{greek}{του και απηγγιλαν λεγοντεϲ οτι ο παιϲ} & 15 &  &  \\
&  & 16 & \foreignlanguage{greek}{αυτου ζη επυθετο ουν την ωραν πα} & 5 & \textbf{52} &  \\
&  & 5 & \foreignlanguage{greek}{ρ αυτων εν η κομψοτερον εϲχεν ειπον} & 11 &  &  \\
&  & 12 & \foreignlanguage{greek}{ουν οτι εχθεϲ ωραν \textoverline{ζ} αφηκεν αυτον} & 18 &  &  \\
&  & 19 & \foreignlanguage{greek}{ο πυρετοϲ εγνω ουν ο \textoverline{πηρ} οτι εν εκινη} & 7 & \textbf{53} &  \\
[0.2em]
\cline{4-4}
\end{tabular}
\end{center}
\end{table}
}
\clearpage
\newpage
 {
 \setlength\arrayrulewidth{1pt}
\begin{table}
\begin{center}
\begin{tabular}{ccc|l|ccc}
\cline{4-4} \\ [-1em]
\multicolumn{7}{c}{\foreignlanguage{greek}{ευαγγελιον κατα ιωαννην} \textbf{(\nospace{4:53})} } \\ \\ [-1em] % Si on veut ajouter les bordures latérales, remplacer {7}{c} par {7}{|c|}
\cline{4-4} \\
\cline{4-4}
&  &  & &  &  & \\ [-0.9em]
&  & 8 & \foreignlanguage{greek}{τη ωρα εν η ειπεν αυτω ο \textoverline{ιϲ} οτι ο \textoverline{υϲ} ϲου ζη} & 20 &  &  \\
&  & 21 & \foreignlanguage{greek}{και επιϲτευϲεν αυτοϲ και η οικια αυτου ολη} & 28 &  &  \\
& \textbf{54} &  & \foreignlanguage{greek}{τουτο δε παλιν \textoverline{β} εποιηϲεν ϲημιον ο \textoverline{ιϲ} ελθω̅} & 9 &  &  \\
&  & 10 & \foreignlanguage{greek}{εκ τηϲ ιουδεαϲ ειϲ την γαλιλεαν μετα} & 1 & \mygospelchapter &  \\
&  & 2 & \foreignlanguage{greek}{ταυτα ην εορτη των ιουδεων και ανεβη} & 8 &  &  \\
&  & 9 & \foreignlanguage{greek}{ο \textoverline{ιϲ} ειϲ ιεροϲολυμα εϲτιν δε εν τοιϲ ιεροϲολυ} & 5 & \textbf{2} &  \\
&  & 5 & \foreignlanguage{greek}{μοιϲ επι τη προβατικη κολυμβηθρα τη ε} & 11 &  &  \\
&  & 11 & \foreignlanguage{greek}{πιλεγομενη εβραιϲτι βηθϲαιδα \textoverline{ε} ϲτοαϲ εχουϲα} & 16 &  &  \\
& \textbf{3} &  & \foreignlanguage{greek}{εν ταυταιϲ κατεκιτο πληθοϲ των αϲθενουντω̅} & 6 &  &  \\
&  & 7 & \foreignlanguage{greek}{τυφλων χωλων ξηρων εκδεχομενοι την του} & 12 &  &  \\
&  & 13 & \foreignlanguage{greek}{υδατοϲ κινηϲιν} & 14 &  &  \\
& \textbf{5} &  & \foreignlanguage{greek}{εχων εν τη αϲθενια αυτου τουτον ειδω̅} & 2 &  &  \\
&  & 3 & \foreignlanguage{greek}{ο \textoverline{ιϲ} κατακιμενον και γνουϲ οτι πολυν ηδη} & 10 &  &  \\
&  & 11 & \foreignlanguage{greek}{χρονον εχι λεγι αυτω θελιϲ υγιηϲ γενεϲ} & 17 &  &  \\
&  & 17 & \foreignlanguage{greek}{θαι απεκριθη αυτω ο αϲθενω̅} & 4 & \textbf{7} &  \\
&  & 5 & \foreignlanguage{greek}{\textoverline{κε} \textoverline{ανον} ουκ εχω ινα οταν ταραχθη} & 11 &  &  \\
&  & 12 & \foreignlanguage{greek}{το υδωρ βαλη με ειϲ την κολυμβη} & 18 &  &  \\
&  & 18 & \foreignlanguage{greek}{θραν εν οϲω δε ερχομε εγω αλλοϲ} & 24 &  &  \\
&  & 25 & \foreignlanguage{greek}{προ εμου καταβενει} & 27 &  &  \\
& \textbf{8} &  & \foreignlanguage{greek}{λεγι αυτω ο \textoverline{ιϲ} εγιρε αρον τον} & 7 &  &  \\
&  & 8 & \foreignlanguage{greek}{κραβαττον ϲου και περιπατι} & 11 &  &  \\
& \textbf{9} &  & \foreignlanguage{greek}{και εγενετο υγιηϲ ο \textoverline{ανοϲ} και η} & 7 &  &  \\
&  & 7 & \foreignlanguage{greek}{ρεν τον κραβαττον αυτου και} & 11 &  &  \\
&  & 12 & \foreignlanguage{greek}{περιεπατι} & 12 &  &  \\
&  & 13 & \foreignlanguage{greek}{ην δε ϲαββατον εν εκινη τη} & 18 &  &  \\
&  & 19 & \foreignlanguage{greek}{ημερα} & 19 &  &  \\
& \textbf{10} &  & \foreignlanguage{greek}{ελεγον ουν οι ιουδεοι τω τεθεραπευ} & 6 &  &  \\
&  & 6 & \foreignlanguage{greek}{μενω ϲαββατον εϲτιν και} & 9 &  &  \\
&  & 10 & \foreignlanguage{greek}{ουκ εξεϲτιν ϲοι αριν τον κραβαττον ϲου} & 16 &  &  \\
& \textbf{11} &  & \foreignlanguage{greek}{ο δε απεκρινατο αυτοιϲ ο ποιϲαϲ} & 6 &  &  \\
&  & 7 & \foreignlanguage{greek}{με υγιην εκινοϲ μοι ειπεν αρον τον} & 13 &  &  \\
[0.2em]
\cline{4-4}
\end{tabular}
\end{center}
\end{table}
}
\clearpage
\newpage
 {
 \setlength\arrayrulewidth{1pt}
\begin{table}
\begin{center}
\begin{tabular}{ccc|l|ccc}
\cline{4-4} \\ [-1em]
\multicolumn{7}{c}{\foreignlanguage{greek}{ευαγγελιον κατα ιωαννην} \textbf{(\nospace{5:11})} } \\ \\ [-1em] % Si on veut ajouter les bordures latérales, remplacer {7}{c} par {7}{|c|}
\cline{4-4} \\
\cline{4-4}
&  &  & &  &  & \\ [-0.9em]
&  & 14 & \foreignlanguage{greek}{κραβαττον ϲου και περιπατει ει} & 3 & \textbf{13} &  \\
&  & 3 & \foreignlanguage{greek}{αθειϲ ουκ ηδει τιϲ εϲτιν ο γαρ \textoverline{ιϲ} εξε} & 11 &  &  \\
&  & 11 & \foreignlanguage{greek}{νευϲεν οχλου οντοϲ εν τω τοπω} & 16 &  &  \\
& \textbf{14} &  & \foreignlanguage{greek}{μετα ταυτα ευριϲκει αυτον ο \textoverline{ιϲ} εν τω} & 8 &  &  \\
&  & 9 & \foreignlanguage{greek}{ιερω και ειπεν αυτω ειδε υγιηϲ γεγο} & 15 &  &  \\
&  & 15 & \foreignlanguage{greek}{ναϲ μηκετι αμαρτανε ινα μη χειρον} & 20 &  &  \\
&  & 21 & \foreignlanguage{greek}{τι ϲοι γενηται απηλθεν δε ο \textoverline{ανοϲ} και} & 5 & \textbf{15} &  \\
&  & 6 & \foreignlanguage{greek}{ανηγγειλεν τοιϲ ιουδαιοιϲ και ειπεν} & 10 &  &  \\
&  & 11 & \foreignlanguage{greek}{αυτοιϲ οτι \textoverline{ιϲ} εϲτιν ο ποιηϲαϲ αυτον υ} & 18 &  &  \\
&  & 18 & \foreignlanguage{greek}{γειη και δια τουτο εδιωκον οι ιουδαι} & 6 & \textbf{16} &  \\
&  & 6 & \foreignlanguage{greek}{οι τον \textoverline{ιν} οτι ταυτα εποιει εν τω ϲαβ} & 14 &  &  \\
&  & 14 & \foreignlanguage{greek}{βατω ο δε απεκριθη αυτοιϲ ο \textoverline{πηρ}} & 6 & \textbf{17} &  \\
&  & 7 & \foreignlanguage{greek}{μου εωϲ αρτι εργαζεται καγω εργαζο} & 12 &  &  \\
&  & 12 & \foreignlanguage{greek}{μαι δια τουτο ουν μαλλον εζητου̅} & 5 & \textbf{18} &  \\
&  & 6 & \foreignlanguage{greek}{αυτον αποκτειναι οι ιουδαιοι οτι} & 10 &  &  \\
&  & 11 & \foreignlanguage{greek}{ου μονον ελυεν το ϲαββατον αλλα} & 16 &  &  \\
&  & 17 & \foreignlanguage{greek}{και \textoverline{πρα} ιδιον ελεγεν τον \textoverline{θν} ιϲον ε} & 24 &  &  \\
&  & 24 & \foreignlanguage{greek}{αυτον ποιων τω \textoverline{θω}} & 27 &  &  \\
& \textbf{19} &  & \foreignlanguage{greek}{απεκριθη ουν ο \textoverline{ιϲ} και ειπεν αυτοιϲ} & 7 &  &  \\
&  & 8 & \foreignlanguage{greek}{αμη αμην λεγω υμιν ου δυναται} & 13 &  &  \\
&  & 14 & \foreignlanguage{greek}{ο υιοϲ αφ εαυτου ποιειν ουδεν εαν} & 20 &  &  \\
&  & 21 & \foreignlanguage{greek}{μη βλεπη τον \textoverline{πρα} ποιουντα ο γαρ αν} & 28 &  &  \\
&  & 29 & \foreignlanguage{greek}{εκεινοϲ ποιη ταυτα και ο υιοϲ ομοι} & 35 &  &  \\
&  & 35 & \foreignlanguage{greek}{ωϲ ποιει ο γαρ \textoverline{πηρ} φιλει τον υιον} & 6 & \textbf{20} &  \\
&  & 7 & \foreignlanguage{greek}{και παντα δικνυϲιν αυτω α αυτοϲ} & 12 &  &  \\
&  & 13 & \foreignlanguage{greek}{ποιει και μειζονα τουτων δειξη αυ} & 18 &  &  \\
&  & 18 & \foreignlanguage{greek}{τω εργα ινα υμειϲ θαυμαζηται} & 22 &  &  \\
& \textbf{21} &  & \foreignlanguage{greek}{ωϲπερ γαρ τουϲ νεκρουϲ εγειρει ο \textoverline{πηρ}} & 7 &  &  \\
&  & 8 & \foreignlanguage{greek}{και ζωοποιει ουτωϲ και ο υιοϲ ουϲ} & 14 &  &  \\
&  & 15 & \foreignlanguage{greek}{θελει ζωοποιει ουδε γαρ ο \textoverline{πηρ} κρι} & 5 & \textbf{22} &  \\
[0.2em]
\cline{4-4}
\end{tabular}
\end{center}
\end{table}
}
\clearpage
\newpage
 {
 \setlength\arrayrulewidth{1pt}
\begin{table}
\begin{center}
\begin{tabular}{ccc|l|ccc}
\cline{4-4} \\ [-1em]
\multicolumn{7}{c}{\foreignlanguage{greek}{ευαγγελιον κατα ιωαννην} \textbf{(\nospace{5:22})} } \\ \\ [-1em] % Si on veut ajouter les bordures latérales, remplacer {7}{c} par {7}{|c|}
\cline{4-4} \\
\cline{4-4}
&  &  & &  &  & \\ [-0.9em]
&  & 5 & \foreignlanguage{greek}{νει ουδενα αλλα την κριϲιν παϲαν} & 10 &  &  \\
&  & 11 & \foreignlanguage{greek}{δεδωκεν τω υιω ινα παντεϲ τιμω} & 3 & \textbf{23} &  \\
&  & 3 & \foreignlanguage{greek}{ϲι τον υιον καθωϲ τιμωϲειν τον \textoverline{πρα}} & 9 &  &  \\
&  & 10 & \foreignlanguage{greek}{ο μη τιμων τον υιον ου τειμα τον} & 17 &  &  \\
&  & 18 & \foreignlanguage{greek}{\textoverline{πρα} τον πεμψαντα αυτον} & 21 &  &  \\
& \textbf{24} &  & \foreignlanguage{greek}{αμην αμην λεγω υμιν οτι ο τον λο} & 8 &  &  \\
&  & 8 & \foreignlanguage{greek}{γον μου ακουων και πιϲτευων τω} & 13 &  &  \\
&  & 14 & \foreignlanguage{greek}{πεμψαντι με εχει ζωην αιωνιον} & 18 &  &  \\
&  & 19 & \foreignlanguage{greek}{και ουκ ερχεται ειϲ κριϲιν αλλα} & 24 &  &  \\
&  & 25 & \foreignlanguage{greek}{μεταβεβηκεν εκ του θανατου ειϲ} & 29 &  &  \\
&  & 30 & \foreignlanguage{greek}{την ζωην αμην αμην λεγω} & 3 & \textbf{25} &  \\
&  & 4 & \foreignlanguage{greek}{υμιν οτι ερχεται ωρα και νυν εϲτι̅} & 10 &  &  \\
&  & 11 & \foreignlanguage{greek}{οτε οι νεκροι ακουϲωϲιν τηϲ φωνηϲ} & 16 &  &  \\
&  & 17 & \foreignlanguage{greek}{του υιου του \textoverline{θυ} και οι ακουϲαντεϲ} & 23 &  &  \\
&  & 24 & \foreignlanguage{greek}{ζηϲουϲιν ωϲ γαρ ο \textoverline{πηρ} εχει ζωη̅} & 6 & \textbf{26} &  \\
&  & 7 & \foreignlanguage{greek}{εν εαυτω ουτωϲ και τω υιω ζωην} & 13 &  &  \\
&  & 14 & \foreignlanguage{greek}{εδωκεν εχειν εν εαυτω και εξου} & 2 & \textbf{27} &  \\
&  & 2 & \foreignlanguage{greek}{ϲιαν εδωκεν αυτω κριϲιν ποιειν} & 6 &  &  \\
&  & 7 & \foreignlanguage{greek}{οτι υιοϲ \textoverline{ανου} εϲτιν} & 10 &  &  \\
& \textbf{28} &  & \foreignlanguage{greek}{μη θαυμαζεται τουτο οτι ερχεται} & 5 &  &  \\
&  & 6 & \foreignlanguage{greek}{ωρα εν η παντεϲ οι εν τοιϲ μνημι} & 13 &  &  \\
&  & 13 & \foreignlanguage{greek}{οιϲ ακουϲωϲιν τηϲ φωνηϲ αυτου} & 17 &  &  \\
& \textbf{29} &  & \foreignlanguage{greek}{και εξελευϲονται οι τα αγαθα ποι} & 6 &  &  \\
&  & 6 & \foreignlanguage{greek}{ηϲαντεϲ ειϲ αναϲταϲιν ζωηϲ} & 9 &  &  \\
&  & 10 & \foreignlanguage{greek}{και οι τα φαυλα πραξαντεϲ ειϲ ανα} & 16 &  &  \\
&  & 16 & \foreignlanguage{greek}{ϲταϲιν κριϲεωϲ} & 17 &  &  \\
& \textbf{30} &  & \foreignlanguage{greek}{ου δυναμαι εγω ποιειν απ εμαυτου} & 6 &  &  \\
&  & 7 & \foreignlanguage{greek}{ουδεν καθωϲ ακουω κρινω και} & 11 &  &  \\
&  & 12 & \foreignlanguage{greek}{η κριϲιϲ η εμη δικαια εϲτιν οτι ου} & 19 &  &  \\
&  & 20 & \foreignlanguage{greek}{ζητω το θελημα το εμον αλλα το θε} & 27 &  &  \\
[0.2em]
\cline{4-4}
\end{tabular}
\end{center}
\end{table}
}
\clearpage
\newpage
 {
 \setlength\arrayrulewidth{1pt}
\begin{table}
\begin{center}
\begin{tabular}{ccc|l|ccc}
\cline{4-4} \\ [-1em]
\multicolumn{7}{c}{\foreignlanguage{greek}{ευαγγελιον κατα ιωαννην} \textbf{(\nospace{5:30})} } \\ \\ [-1em] % Si on veut ajouter les bordures latérales, remplacer {7}{c} par {7}{|c|}
\cline{4-4} \\
\cline{4-4}
&  &  & &  &  & \\ [-0.9em]
&  & 27 & \foreignlanguage{greek}{λημα του πεμψαντοϲ με εαν εγω} & 2 & \textbf{31} &  \\
&  & 3 & \foreignlanguage{greek}{μαρτυρω περι εμαυτου η μαρτυρια μου} & 8 &  &  \\
&  & 9 & \foreignlanguage{greek}{ουκ εϲτιν αληθηϲ αλλοϲ εϲτιν ο μαρ} & 4 & \textbf{32} &  \\
&  & 4 & \foreignlanguage{greek}{τυρων περι εμου και οιδα οτι αληθηϲ} & 10 &  &  \\
&  & 11 & \foreignlanguage{greek}{εϲτιν η μαρτυρια ην μαρτυρι περι εμου} & 17 &  &  \\
& \textbf{33} &  & \foreignlanguage{greek}{υμειϲ απεϲταλκατε προϲ ιωαννην} & 4 &  &  \\
&  & 5 & \foreignlanguage{greek}{και μεμαρτυρηκεν τη αληθεια} & 8 &  &  \\
& \textbf{34} &  & \foreignlanguage{greek}{εγω δε ου παρα \textoverline{ανου} την μαρτυριαν} & 7 &  &  \\
&  & 8 & \foreignlanguage{greek}{λαμβανω αλλα ταυτα λεγω ινα υ} & 13 &  &  \\
&  & 13 & \foreignlanguage{greek}{μειϲ ϲωθηται εκεινοϲ ην ο λυ} & 4 & \textbf{35} &  \\
&  & 4 & \foreignlanguage{greek}{χνοϲ ο καιομενοϲ και φαινων} & 8 &  &  \\
&  & 9 & \foreignlanguage{greek}{υμειϲ δε ηθεληϲατε προϲ ωραν αγαλ} & 14 &  &  \\
&  & 14 & \foreignlanguage{greek}{λιαθηναι εν τω φωτι αυτου εγω δε} & 2 & \textbf{36} &  \\
&  & 3 & \foreignlanguage{greek}{εχω την μαρτυριαν μειζων του ιω} & 8 &  &  \\
&  & 8 & \foreignlanguage{greek}{αννου τα γαρ εργα α δεδωκεν μοι} & 14 &  &  \\
&  & 15 & \foreignlanguage{greek}{ο \textoverline{πηρ} ινα τελιωϲω αυτα αυτα τα ερ} & 22 &  &  \\
&  & 22 & \foreignlanguage{greek}{γα α ποιω μαρτυρουϲιν περι εμου} & 27 &  &  \\
&  & 28 & \foreignlanguage{greek}{οτι ο \textoverline{πηρ} με απεϲταλκεν και ο πεμ} & 3 & \textbf{37} &  \\
&  & 3 & \foreignlanguage{greek}{ψαϲ με \textoverline{πηρ} εκεινοϲ μεμαρτυρηκε̅} & 7 &  &  \\
&  & 8 & \foreignlanguage{greek}{περι εμου ουτε φωνην αυτου πω} & 13 &  &  \\
&  & 13 & \foreignlanguage{greek}{ποτε ακηκοατε ουτε ειδοϲ εωρακα} & 17 &  &  \\
&  & 17 & \foreignlanguage{greek}{τε και τον λογον αυτου ουκ εχεται} & 6 & \textbf{38} &  \\
&  & 7 & \foreignlanguage{greek}{εν υμιν μενοντα οτι ον απεϲτι} & 12 &  &  \\
&  & 12 & \foreignlanguage{greek}{λεν εκεινοϲ τουτω υμειϲ ου πιϲτευ} & 17 &  &  \\
&  & 17 & \foreignlanguage{greek}{εται ερευναται ταϲ γραφαϲ} & 3 & \textbf{39} &  \\
&  & 4 & \foreignlanguage{greek}{οτι υμειϲ δοκειται εν αυταιϲ ζωην} & 9 &  &  \\
&  & 10 & \foreignlanguage{greek}{αιωνιον εχειν και αυται ειϲιν αι} & 15 &  &  \\
&  & 16 & \foreignlanguage{greek}{μαρτυρουϲαι περι εμου και ου θε} & 3 & \textbf{40} &  \\
&  & 3 & \foreignlanguage{greek}{λεται ελθειν προϲ με ινα ζωην ε} & 9 &  &  \\
&  & 9 & \foreignlanguage{greek}{χηται δοξαν παρα \textoverline{ανων} ου λαμ} & 5 & \textbf{41} &  \\
[0.2em]
\cline{4-4}
\end{tabular}
\end{center}
\end{table}
}
\clearpage
\newpage
 {
 \setlength\arrayrulewidth{1pt}
\begin{table}
\begin{center}
\begin{tabular}{ccc|l|ccc}
\cline{4-4} \\ [-1em]
\multicolumn{7}{c}{\foreignlanguage{greek}{ευαγγελιον κατα ιωαννην} \textbf{(\nospace{5:41})} } \\ \\ [-1em] % Si on veut ajouter les bordures latérales, remplacer {7}{c} par {7}{|c|}
\cline{4-4} \\
\cline{4-4}
&  &  & &  &  & \\ [-0.9em]
&  & 5 & \foreignlanguage{greek}{βανω αλλα εγνωκα υμαϲ οτι την α} & 6 & \textbf{42} &  \\
&  & 6 & \foreignlanguage{greek}{γαπην του \textoverline{θυ} ουκ εχεται εν εαυτοιϲ} & 12 &  &  \\
& \textbf{43} &  & \foreignlanguage{greek}{εγω εληλυθα εν τω ονοματι του \textoverline{πρϲ}} & 7 &  &  \\
&  & 8 & \foreignlanguage{greek}{μου και ου λαμβανεται με εαν αλλοϲ} & 14 &  &  \\
&  & 15 & \foreignlanguage{greek}{ελθη εν τω ονοματι τω ιδιω εκεινο̅} & 21 &  &  \\
&  & 22 & \foreignlanguage{greek}{ληψεϲθαι πωϲ δυναϲθαι υμειϲ πι} & 4 & \textbf{44} &  \\
&  & 4 & \foreignlanguage{greek}{ϲτευϲαι δοξαν παρα αλληλων λαμβα} & 8 &  &  \\
&  & 8 & \foreignlanguage{greek}{νοντεϲ και την δοξαν την παρα του} & 14 &  &  \\
&  & 15 & \foreignlanguage{greek}{μονου ου ζητειται μη δοκειται} & 2 & \textbf{45} &  \\
&  & 3 & \foreignlanguage{greek}{οτι εγω κατηγορηϲω υμων προϲ τον} & 8 &  &  \\
&  & 9 & \foreignlanguage{greek}{\textoverline{πρα} εϲτιν ο κατηγορων υμων μωυ} & 14 &  &  \\
&  & 14 & \foreignlanguage{greek}{ϲηϲ ειϲ ον υμειϲ ηλπεικατε ει γαρ επι} & 3 & \textbf{46} &  \\
&  & 3 & \foreignlanguage{greek}{ϲτευεται μωυϲει επιϲτευεται αν} & 6 &  &  \\
&  & 7 & \foreignlanguage{greek}{εμοι περι γαρ εμου εκεινοϲ εγραψεν} & 12 &  &  \\
& \textbf{47} &  & \foreignlanguage{greek}{ει δε τοιϲ εκεινου γραμμαϲιν ου πι} & 7 &  &  \\
&  & 7 & \foreignlanguage{greek}{ϲτευεται πωϲ τοιϲ εμοιϲ ρημαϲιν} & 11 &  &  \\
&  & 12 & \foreignlanguage{greek}{πιϲτευϲηται} & 12 &  &  \\
& \mygospelchapter &  & \foreignlanguage{greek}{μετα ταυτα απηλθεν ο \textoverline{ιϲ} περαν τηϲ θα} & 8 &  &  \\
&  & 8 & \foreignlanguage{greek}{λαϲϲηϲ τηϲ γαλιλαιαϲ τηϲ τιβερια} & 12 &  &  \\
&  & 12 & \foreignlanguage{greek}{δοϲ ηκολουθει δε αυτω οχλοϲ πολυϲ} & 5 & \textbf{2} &  \\
&  & 6 & \foreignlanguage{greek}{θεωρουντεϲ τα ϲημια α εποιει επι τω̅} & 12 &  &  \\
&  & 13 & \foreignlanguage{greek}{αϲθενουντων} & 13 &  &  \\
& \textbf{3} &  & \foreignlanguage{greek}{ανηλθεν ουν ειϲ το οροϲ \textoverline{ιϲ} και εκει ε} & 9 &  &  \\
&  & 9 & \foreignlanguage{greek}{καθητο μετα των μαθητων αυτου} & 13 &  &  \\
& \textbf{4} &  & \foreignlanguage{greek}{ην δε εγγυϲ το παϲχα η εορτη των ι} & 9 &  &  \\
&  & 9 & \foreignlanguage{greek}{ουδαιων επαραϲ ουν τουϲ οφθαλ} & 4 & \textbf{5} &  \\
&  & 4 & \foreignlanguage{greek}{μουϲ ο \textoverline{ιϲ} και θεαϲαμενοϲ οτι πολυϲ} & 10 &  &  \\
&  & 11 & \foreignlanguage{greek}{οχλοϲ ερχεται προϲ αυτον λεγει προϲ} & 16 &  &  \\
&  & 17 & \foreignlanguage{greek}{φιλιππον ποθεν αγοραϲωμεν αρ} & 20 &  &  \\
&  & 20 & \foreignlanguage{greek}{τουϲ ινα φαγωϲιν ουτοι τουτο δε} & 2 & \textbf{6} &  \\
[0.2em]
\cline{4-4}
\end{tabular}
\end{center}
\end{table}
}
\clearpage
\newpage
 {
 \setlength\arrayrulewidth{1pt}
\begin{table}
\begin{center}
\begin{tabular}{ccc|l|ccc}
\cline{4-4} \\ [-1em]
\multicolumn{7}{c}{\foreignlanguage{greek}{ευαγγελιον κατα ιωαννην} \textbf{(\nospace{6:6})} } \\ \\ [-1em] % Si on veut ajouter les bordures latérales, remplacer {7}{c} par {7}{|c|}
\cline{4-4} \\
\cline{4-4}
&  &  & &  &  & \\ [-0.9em]
&  & 3 & \foreignlanguage{greek}{ελεγεν πειραζων αυτον αυτοϲ γαρ η} & 8 &  &  \\
&  & 8 & \foreignlanguage{greek}{δει τι εμελλεν ποιειν απεκριθη αυ} & 2 & \textbf{7} &  \\
&  & 2 & \foreignlanguage{greek}{τω ο φιλιπποϲ διακοϲιων δηναριων αρ} & 7 &  &  \\
&  & 7 & \foreignlanguage{greek}{τοι ουκ αρκουϲιν αυτοιϲ ινα εκαϲτοϲ} & 12 &  &  \\
&  & 13 & \foreignlanguage{greek}{βραχυ τι λαβη λεγει αυτω ειϲ εκ τω̅} & 5 & \textbf{8} &  \\
&  & 6 & \foreignlanguage{greek}{μαθητων αυτου ανδρεαϲ ο αδελφοϲ} & 10 &  &  \\
&  & 11 & \foreignlanguage{greek}{ϲιμωνοϲ πετρου εϲτιν παιδαριον} & 2 & \textbf{9} &  \\
&  & 3 & \foreignlanguage{greek}{ωδε οϲ εχει πεντε αρτουϲ κριθινουϲ} & 8 &  &  \\
&  & 9 & \foreignlanguage{greek}{και δυο οψαρια αλλα ταυτα τι εϲτιν} & 15 &  &  \\
&  & 16 & \foreignlanguage{greek}{ειϲ τοϲουτουϲ} & 17 &  &  \\
& \textbf{10} &  & \foreignlanguage{greek}{ειπεν δε ο \textoverline{ιϲ} ποιηϲαται τουϲ \textoverline{ανουϲ} α} & 8 &  &  \\
&  & 8 & \foreignlanguage{greek}{ναπεϲιν ην δε χορτοϲ πολυϲ εν τω} & 14 &  &  \\
&  & 15 & \foreignlanguage{greek}{τοπω ανεπεϲαν ουν ανδρεϲ τον} & 19 &  &  \\
&  & 20 & \foreignlanguage{greek}{αριθμον ωϲ πεντακιϲχειλιοι} & 22 &  &  \\
& \textbf{11} &  & \foreignlanguage{greek}{ελαβεν ουν τουϲ αρτουϲ ο \textoverline{ιϲ} και ευχα} & 8 &  &  \\
&  & 8 & \foreignlanguage{greek}{ριϲτηϲαϲ διεδωκεν τοιϲ ανακειμε} & 11 &  &  \\
&  & 11 & \foreignlanguage{greek}{νοιϲ ομοιωϲ και εκ των οψαριων οϲο̅} & 17 &  &  \\
&  & 18 & \foreignlanguage{greek}{ηθελον ωϲ δε ενεπληϲθηϲαν λεγει} & 4 & \textbf{12} &  \\
&  & 5 & \foreignlanguage{greek}{τοιϲ μαθηταιϲ αυτου ϲυναγαγεται τα} & 9 &  &  \\
&  & 10 & \foreignlanguage{greek}{περιϲευϲαντα κλαϲματα ινα μη τι α} & 15 &  &  \\
&  & 15 & \foreignlanguage{greek}{ποληται ϲυνηγαγον ουν και εγε} & 4 & \textbf{13} &  \\
&  & 4 & \foreignlanguage{greek}{μιϲαν δωδεκα κοφινουϲ κλαϲματω̅} & 7 &  &  \\
&  & 8 & \foreignlanguage{greek}{εκ των πεντε αρτων των κριθινων} & 13 &  &  \\
&  & 14 & \foreignlanguage{greek}{α επεριϲϲευϲαν τοιϲ βεβρωκοϲιν} & 17 &  &  \\
& \textbf{14} &  & \foreignlanguage{greek}{οι ουν \textoverline{ανοι} ειδοντεϲ ο εποιηϲεν ϲη} & 7 &  &  \\
&  & 7 & \foreignlanguage{greek}{μιον ελεγον ουτοϲ εϲτιν αληθωϲ} & 11 &  &  \\
&  & 12 & \foreignlanguage{greek}{ο προφητηϲ ο ερχομενοϲ ειϲ τον κοϲμο̅} & 18 &  &  \\
& \textbf{15} &  & \foreignlanguage{greek}{\textoverline{ιϲ} ουν γνουϲ οτι μελλουϲιν ερχεϲθαι} & 6 &  &  \\
&  & 7 & \foreignlanguage{greek}{και αρπαζειν αυτον ινα ποιηϲωϲιν} & 11 &  &  \\
&  & 12 & \foreignlanguage{greek}{βαϲιλεα ανεχωρηϲεν ειϲ το οροϲ αυ} & 17 &  &  \\
[0.2em]
\cline{4-4}
\end{tabular}
\end{center}
\end{table}
}
\clearpage
\newpage
 {
 \setlength\arrayrulewidth{1pt}
\begin{table}
\begin{center}
\begin{tabular}{ccc|l|ccc}
\cline{4-4} \\ [-1em]
\multicolumn{7}{c}{\foreignlanguage{greek}{ευαγγελιον κατα ιωαννην} \textbf{(\nospace{6:15})} } \\ \\ [-1em] % Si on veut ajouter les bordures latérales, remplacer {7}{c} par {7}{|c|}
\cline{4-4} \\
\cline{4-4}
&  &  & &  &  & \\ [-0.9em]
&  & 17 & \foreignlanguage{greek}{τοϲ μονοϲ ωϲ δε οψεια εγενετο κατε} & 5 & \textbf{16} &  \\
&  & 5 & \foreignlanguage{greek}{βηϲαν επι την θαλαϲϲαν και ενβα̅} & 2 & \textbf{17} &  \\
&  & 2 & \foreignlanguage{greek}{τεϲ ειϲ το πλοιον ηρχοντο περαν τηϲ} & 8 &  &  \\
&  & 9 & \foreignlanguage{greek}{θαλαϲϲηϲ ειϲ καφαρναουμ και ϲκοτι} & 13 &  &  \\
&  & 13 & \foreignlanguage{greek}{α ηδη εγεγονει και ουπω εληλυθει} & 18 &  &  \\
&  & 19 & \foreignlanguage{greek}{προϲ αυτουϲ ο \textoverline{ιϲ} η τε θαλαϲϲα ανε} & 4 & \textbf{18} &  \\
&  & 4 & \foreignlanguage{greek}{μου μεγαλου πνεοντοϲ διηγειριτο} & 7 &  &  \\
& \textbf{19} &  & \foreignlanguage{greek}{εληλακοτεϲ ουν ωϲ ϲταδιουϲ \textoverline{κε} η \textoverline{λ}} & 7 &  &  \\
&  & 8 & \foreignlanguage{greek}{θεωρουϲιν τον \textoverline{ιν} περιπατουντα επι} & 12 &  &  \\
&  & 13 & \foreignlanguage{greek}{τηϲ θαλαϲϲηϲ και εγγυϲ του πλοιου} & 18 &  &  \\
&  & 19 & \foreignlanguage{greek}{γεινομενον και εφοβηθηϲαν} & 21 &  &  \\
& \textbf{20} &  & \foreignlanguage{greek}{ο δε λεγει αυτοιϲ εγω ειμει μη φοβει} & 8 &  &  \\
&  & 8 & \foreignlanguage{greek}{ϲθαι ηθελον ουν αυτον βαλιν ειϲ το} & 6 & \textbf{21} &  \\
&  & 7 & \foreignlanguage{greek}{πλοιον και ευθεωϲ εγενετο το πλοι} & 12 &  &  \\
&  & 12 & \foreignlanguage{greek}{ον επι τηϲ γηϲ ειϲ ην υπηγον} & 18 &  &  \\
& \textbf{22} &  & \foreignlanguage{greek}{τη επαυριον ο οχλοϲ ο εϲτηκωϲ περαν} & 7 &  &  \\
&  & 8 & \foreignlanguage{greek}{τηϲ θαλαϲϲηϲ ιδον οτι πλοιαριον} & 12 &  &  \\
&  & 13 & \foreignlanguage{greek}{αλλο ουκ ην εκει ει μη εν και οτι ου} & 22 &  &  \\
&  & 23 & \foreignlanguage{greek}{ϲυνειϲηλθεν τοιϲ μαθηταιϲ αυτου} & 26 &  &  \\
&  & 27 & \foreignlanguage{greek}{ο \textoverline{ιϲ} ειϲ το πλοιον αλλα μονοι οι μαθη} & 35 &  &  \\
&  & 35 & \foreignlanguage{greek}{ται αυτου απηλθον αλλα δε ηλθεν} & 3 & \textbf{23} &  \\
&  & 4 & \foreignlanguage{greek}{πλοια εκ τηϲ τιβεριαδοϲ οπου εφαγο̅} & 9 &  &  \\
&  & 10 & \foreignlanguage{greek}{τον αρτον ευχαριϲτηϲαντοϲ του \textoverline{κυ}} & 14 &  &  \\
& \textbf{24} &  & \foreignlanguage{greek}{οτε ουν ειδεν ο οχλοϲ οτι \textoverline{ιϲ} ουκ εϲτι̅} & 9 &  &  \\
&  & 10 & \foreignlanguage{greek}{εκει ουδε οι μαθηται αυτου ενε} & 15 &  &  \\
&  & 15 & \foreignlanguage{greek}{βηϲαν αυτοι ειϲ τα πλοιαρια και ηλ} & 21 &  &  \\
&  & 21 & \foreignlanguage{greek}{θον ειϲ καφαρναουμ ζητουντεϲ τον \textoverline{ιν}} & 26 &  &  \\
& \textbf{25} &  & \foreignlanguage{greek}{και ευροντεϲ αυτον περαν τηϲ θαλαϲ} & 6 &  &  \\
&  & 6 & \foreignlanguage{greek}{ϲηϲ ειπαν αυτω ραββει ποτε ωδε γε} & 12 &  &  \\
&  & 12 & \foreignlanguage{greek}{γοναϲ απεκριθη αυτοιϲ ο \textoverline{ιϲ} και} & 5 & \textbf{26} &  \\
[0.2em]
\cline{4-4}
\end{tabular}
\end{center}
\end{table}
}
\clearpage
\newpage
 {
 \setlength\arrayrulewidth{1pt}
\begin{table}
\begin{center}
\begin{tabular}{ccc|l|ccc}
\cline{4-4} \\ [-1em]
\multicolumn{7}{c}{\foreignlanguage{greek}{ευαγγελιον κατα ιωαννην} \textbf{(\nospace{6:26})} } \\ \\ [-1em] % Si on veut ajouter les bordures latérales, remplacer {7}{c} par {7}{|c|}
\cline{4-4} \\
\cline{4-4}
&  &  & &  &  & \\ [-0.9em]
&  & 6 & \foreignlanguage{greek}{ειπεν αμην αμην λεγω υμιν ζητει} & 11 &  &  \\
&  & 11 & \foreignlanguage{greek}{τε με ουχ οτι ειδετε ϲημια αλλ οτι εφα} & 19 &  &  \\
&  & 19 & \foreignlanguage{greek}{γεται εκ των αρτων και εχορταϲθηται} & 24 &  &  \\
& \textbf{27} &  & \foreignlanguage{greek}{εργαζεϲθαι μη την βρωϲιν την απολλυ} & 6 &  &  \\
&  & 6 & \foreignlanguage{greek}{μενην αλλα την βρωϲιν την μενουϲα̅} & 11 &  &  \\
&  & 12 & \foreignlanguage{greek}{ειϲ ζωην αιωνιον ην ο υιοϲ του \textoverline{ανου}} & 19 &  &  \\
&  & 20 & \foreignlanguage{greek}{υμιν δωϲει τουτον γαρ ο \textoverline{πηρ} εϲφρα} & 26 &  &  \\
&  & 26 & \foreignlanguage{greek}{γειϲεν ο \textoverline{θϲ} ειπον ουν αυτω τι ποιη} & 5 & \textbf{28} &  \\
&  & 5 & \foreignlanguage{greek}{ϲωμεν ινα εργαζωμεθα τα εργα του \textoverline{θυ}} & 11 &  &  \\
& \textbf{29} &  & \foreignlanguage{greek}{απεκριθη \textoverline{ιϲ} και ειπεν αυτοιϲ τουτο εϲτι̅} & 7 &  &  \\
&  & 8 & \foreignlanguage{greek}{το εργον του \textoverline{θυ} ινα πιϲτευϲηται ειϲ ον} & 15 &  &  \\
&  & 16 & \foreignlanguage{greek}{απεϲτειλεν εκεινοϲ ειπον ουν αυ} & 3 & \textbf{30} &  \\
&  & 3 & \foreignlanguage{greek}{τω τι ουν ποιειϲ ϲημιον ινα ιδωμεν} & 9 &  &  \\
&  & 10 & \foreignlanguage{greek}{και πιϲτευϲωμεν ϲοι τι εργαζη} & 14 &  &  \\
& \textbf{31} &  & \foreignlanguage{greek}{οι πατερεϲ ημων το μαννα εφαγον εν} & 7 &  &  \\
&  & 8 & \foreignlanguage{greek}{τη ερημω καθωϲ εϲτιν γεγραμμενο̅} & 12 &  &  \\
&  & 13 & \foreignlanguage{greek}{αρτον εκ του ουρανου δεδωκεν αυτοιϲ} & 18 &  &  \\
&  & 19 & \foreignlanguage{greek}{φαγειν ειπεν ουν αυτοιϲ ο \textoverline{ιϲ}} & 5 & \textbf{32} &  \\
&  & 6 & \foreignlanguage{greek}{αμην αμην λεγω υμιν ου μωυϲηϲ ε} & 12 &  &  \\
&  & 12 & \foreignlanguage{greek}{δωκεν υμιν τον αρτον εκ του ουρανου} & 18 &  &  \\
&  & 19 & \foreignlanguage{greek}{αλλ ο \textoverline{πηρ} μου διδωϲιν υμιν τον αρτο̅} & 26 &  &  \\
&  & 27 & \foreignlanguage{greek}{εκ του ουρανου τον αληθεινον} & 31 &  &  \\
& \textbf{33} &  & \foreignlanguage{greek}{ο γαρ αρτοϲ του \textoverline{θυ} εϲτιν ο καταβαινω̅} & 8 &  &  \\
&  & 9 & \foreignlanguage{greek}{εκ του ουρανου και ζωην διδουϲ τω} & 15 &  &  \\
&  & 16 & \foreignlanguage{greek}{κοϲμω ειπον ουν προϲ αυτο̅} & 4 & \textbf{34} &  \\
&  & 5 & \foreignlanguage{greek}{\textoverline{κε} παντοτε δοϲ ημιν τον αρτον του} & 11 &  &  \\
&  & 11 & \foreignlanguage{greek}{τον ειπεν αυτοιϲ ο \textoverline{ιϲ} εγω ειμει} & 6 & \textbf{35} &  \\
&  & 7 & \foreignlanguage{greek}{ο αρτοϲ τηϲ ζωηϲ ο ερχομενοϲ προϲ} & 13 &  &  \\
&  & 14 & \foreignlanguage{greek}{με ου μη πιναϲη και ο πιϲτευων ειϲ} & 21 &  &  \\
&  & 22 & \foreignlanguage{greek}{εμε ου μη διψηϲει πωποτε αλλα ει} & 2 & \textbf{36} &  \\
[0.2em]
\cline{4-4}
\end{tabular}
\end{center}
\end{table}
}
\clearpage
\newpage
 {
 \setlength\arrayrulewidth{1pt}
\begin{table}
\begin{center}
\begin{tabular}{ccc|l|ccc}
\cline{4-4} \\ [-1em]
\multicolumn{7}{c}{\foreignlanguage{greek}{ευαγγελιον κατα ιωαννην} \textbf{(\nospace{6:36})} } \\ \\ [-1em] % Si on veut ajouter les bordures latérales, remplacer {7}{c} par {7}{|c|}
\cline{4-4} \\
\cline{4-4}
&  &  & &  &  & \\ [-0.9em]
&  & 2 & \foreignlanguage{greek}{πον υμιν οτι και εωρακαται με και ου} & 9 &  &  \\
&  & 10 & \foreignlanguage{greek}{πιϲτευεται μοι παν ο διδωϲιν μοι ο} & 5 & \textbf{37} &  \\
&  & 6 & \foreignlanguage{greek}{\textoverline{πηρ} προϲ εμε ηξει και τον ερχομενον} & 12 &  &  \\
&  & 13 & \foreignlanguage{greek}{προϲ με ου μη εκβαλω εξω οτι καταβε} & 2 & \textbf{38} &  \\
&  & 2 & \foreignlanguage{greek}{βηκα απο του ουρανου ουχ ινα ποιη} & 8 &  &  \\
&  & 8 & \foreignlanguage{greek}{ϲω το θελημα το εμον αλλα το θελημα} & 15 &  &  \\
&  & 16 & \foreignlanguage{greek}{του πεμψαντοϲ με τουτο δε εϲτιν} & 3 & \textbf{39} &  \\
&  & 4 & \foreignlanguage{greek}{το θελημα του πεμψαντοϲ με ινα πα̅} & 10 &  &  \\
&  & 11 & \foreignlanguage{greek}{ο δεδωκεν μοι μη απολεϲω εξ αυτου} & 17 &  &  \\
&  & 18 & \foreignlanguage{greek}{αλλα αναϲτηϲω αυτον τη εϲχατη η} & 23 &  &  \\
&  & 23 & \foreignlanguage{greek}{μερα τουτο γαρ εϲτιν το θελημα} & 5 & \textbf{40} &  \\
&  & 6 & \foreignlanguage{greek}{του \textoverline{πρϲ} μου ινα παϲ ο θεωρων τον} & 13 &  &  \\
&  & 14 & \foreignlanguage{greek}{υιον και πιϲτευων ειϲ αυτον εχη} & 19 &  &  \\
&  & 20 & \foreignlanguage{greek}{ζωην αιωνιον και αναϲτηϲω αυτο̅} & 24 &  &  \\
&  & 25 & \foreignlanguage{greek}{εγω τη εϲχατη ημερα} & 28 &  &  \\
& \textbf{41} &  & \foreignlanguage{greek}{εγογγυζον ουν οι ιουδαιοι περι αυτου} & 6 &  &  \\
&  & 7 & \foreignlanguage{greek}{οτι ειπεν εγω ειμει ο αρτοϲ ο καταβαϲ} & 14 &  &  \\
&  & 15 & \foreignlanguage{greek}{εκ του ουρανου και ελεγον ουχ ου} & 4 & \textbf{42} &  \\
&  & 4 & \foreignlanguage{greek}{τοϲ εϲτιν \textoverline{ιϲ} ο υιοϲ ιωϲηφ ου ημειϲ οι} & 12 &  &  \\
&  & 12 & \foreignlanguage{greek}{δαμεν τον \textoverline{πρα} πωϲ νυν λεγει} & 17 &  &  \\
&  & 18 & \foreignlanguage{greek}{οτι εκ του ουρανου καταβεβηκα} & 22 &  &  \\
& \textbf{43} &  & \foreignlanguage{greek}{απεκρειθη ουν ο \textoverline{ιϲ} και ειπεν αυτοιϲ} & 7 &  &  \\
&  & 8 & \foreignlanguage{greek}{μη γογγυζεται μετ αλληλων ουδιϲ} & 1 & \textbf{44} &  \\
&  & 2 & \foreignlanguage{greek}{δυναται ελθειν προϲ με εαν μη ο} & 8 &  &  \\
&  & 9 & \foreignlanguage{greek}{\textoverline{πηρ} ο πεμψαϲ με ελκυϲη αυτον προϲ} & 15 &  &  \\
&  & 16 & \foreignlanguage{greek}{με καγω αναϲτηϲω αυτον εν τη ε} & 22 &  &  \\
&  & 22 & \foreignlanguage{greek}{ϲχατη ημερα εϲτιν γεγραμμενο̅} & 2 & \textbf{45} &  \\
&  & 3 & \foreignlanguage{greek}{εν τοιϲ προφηταιϲ και εϲονται} & 7 &  &  \\
&  & 8 & \foreignlanguage{greek}{παντεϲ διδακτοι \textoverline{θυ} παϲ ο ακου} & 13 &  &  \\
&  & 13 & \foreignlanguage{greek}{ϲαϲ παρα του \textoverline{πρϲ} και μαθων ερχεται} & 19 &  &  \\
[0.2em]
\cline{4-4}
\end{tabular}
\end{center}
\end{table}
}
\clearpage
\newpage
 {
 \setlength\arrayrulewidth{1pt}
\begin{table}
\begin{center}
\begin{tabular}{ccc|l|ccc}
\cline{4-4} \\ [-1em]
\multicolumn{7}{c}{\foreignlanguage{greek}{ευαγγελιον κατα ιωαννην} \textbf{(\nospace{6:45})} } \\ \\ [-1em] % Si on veut ajouter les bordures latérales, remplacer {7}{c} par {7}{|c|}
\cline{4-4} \\
\cline{4-4}
&  &  & &  &  & \\ [-0.9em]
&  & 20 & \foreignlanguage{greek}{προϲ με ουχ οτι τον \textoverline{πρα} εορακε τιϲ ει} & 7 & \textbf{46} &  \\
&  & 8 & \foreignlanguage{greek}{μη ο ων παρα του \textoverline{θυ} αυτοϲ εορακεν τον \textoverline{πρα}} & 17 &  &  \\
& \textbf{47} &  & \foreignlanguage{greek}{αμην αμην λεγω υμιν ο πιϲτευων ε} & 7 &  &  \\
&  & 7 & \foreignlanguage{greek}{χει ζωην αιωνιον εγω ειμει ο αρτοϲ} & 4 & \textbf{48} &  \\
&  & 5 & \foreignlanguage{greek}{τηϲ ζωηϲ οι πατερεϲ υμων εφαγον} & 4 & \textbf{49} &  \\
&  & 5 & \foreignlanguage{greek}{εν τη ερημω το μαννα και απεθανο̅} & 11 &  &  \\
& \textbf{50} &  & \foreignlanguage{greek}{ουτοϲ εϲτιν ο αρτοϲ ο εκ του ουρανου} & 8 &  &  \\
&  & 9 & \foreignlanguage{greek}{καταβαινων ινα τιϲ εξ αυτου φαγη} & 14 &  &  \\
&  & 15 & \foreignlanguage{greek}{και μη αποθανη εγω ειμει ο αρτοϲ} & 4 & \textbf{51} &  \\
&  & 5 & \foreignlanguage{greek}{ο ζων ο εκ του ουρανου καταβαϲ} & 11 &  &  \\
&  & 12 & \foreignlanguage{greek}{εαν τιϲ φαγη εκ τουτου του αρτου ζη} & 19 &  &  \\
&  & 19 & \foreignlanguage{greek}{ϲει ειϲ τον αιωνα και ο αρτοϲ ον εγω δω} & 28 &  &  \\
&  & 28 & \foreignlanguage{greek}{ϲω η ϲαρξ μου εϲτιν υπερ τηϲ του κοϲ} & 36 &  &  \\
&  & 36 & \foreignlanguage{greek}{μου ζωηϲ εμαχοντο ουν προϲ} & 3 & \textbf{52} &  \\
&  & 4 & \foreignlanguage{greek}{αλληλουϲ οι ιουδαιοι λεγοντεϲ} & 7 &  &  \\
&  & 8 & \foreignlanguage{greek}{πωϲ δυναται ουτοϲ ημιν δουναι τη̅} & 13 &  &  \\
&  & 14 & \foreignlanguage{greek}{ϲαρκα φαγειν ειπεν ουν αυτοιϲ} & 3 & \textbf{53} &  \\
&  & 4 & \foreignlanguage{greek}{ο \textoverline{ιϲ} αμην αμην λεγω υμιν εαν φαγη} & 11 &  &  \\
&  & 11 & \foreignlanguage{greek}{ται την ϲαρκα του υιου του \textoverline{ανου} και} & 18 &  &  \\
&  & 19 & \foreignlanguage{greek}{πιηται αυτου το αιμα ουκ εχεται ζωη̅} & 25 &  &  \\
&  & 26 & \foreignlanguage{greek}{εν εαυτοιϲ ο τρωγων μου την ϲαρ} & 5 & \textbf{54} &  \\
&  & 5 & \foreignlanguage{greek}{κα και πινων μου το αιμα εχει ζωην} & 12 &  &  \\
&  & 13 & \foreignlanguage{greek}{αιωνιον καγω αναϲτηϲω αυτον τη} & 17 &  &  \\
&  & 18 & \foreignlanguage{greek}{εϲχατη ημερα η γαρ ϲαρξ μου αλη} & 5 & \textbf{55} &  \\
&  & 5 & \foreignlanguage{greek}{θηϲ εϲτιν βρωϲιϲ και το αιμα μου} & 11 &  &  \\
&  & 12 & \foreignlanguage{greek}{αληθηϲ εϲτιν ποϲειϲ ο τρωγων μου} & 3 & \textbf{56} &  \\
&  & 4 & \foreignlanguage{greek}{την ϲαρκα και πεινων μου το αιμα} & 10 &  &  \\
&  & 11 & \foreignlanguage{greek}{εχει ζωην αιωνιον καγω αναϲτη} & 15 &  &  \\
&  & 15 & \foreignlanguage{greek}{ϲω αυτον τη εϲχατη ημερα η γαρ ϲαξ} & 22 &  &  \\
&  & 23 & \foreignlanguage{greek}{μου αληθηϲ εϲτιν βρωϲιϲ και το αιμα μου} & 30 &  &  \\
[0.2em]
\cline{4-4}
\end{tabular}
\end{center}
\end{table}
}
\clearpage
\newpage
 {
 \setlength\arrayrulewidth{1pt}
\begin{table}
\begin{center}
\begin{tabular}{ccc|l|ccc}
\cline{4-4} \\ [-1em]
\multicolumn{7}{c}{\foreignlanguage{greek}{ευαγγελιον κατα ιωαννην} \textbf{(\nospace{6:56})} } \\ \\ [-1em] % Si on veut ajouter les bordures latérales, remplacer {7}{c} par {7}{|c|}
\cline{4-4} \\
\cline{4-4}
&  &  & &  &  & \\ [-0.9em]
&  & 31 & \foreignlanguage{greek}{αληθηϲ εϲτιν ποϲιϲ ο τρωγων μου} & 36 &  &  \\
&  & 37 & \foreignlanguage{greek}{την ϲαρκα και πινων μου το αιμα} & 43 &  &  \\
&  & 44 & \foreignlanguage{greek}{εν εμοι μενει καγω εν αυτω καθωϲ} & 1 & \textbf{57} &  \\
&  & 2 & \foreignlanguage{greek}{απεϲτιλεν με ο ζων πατηρ καγω ζω} & 8 &  &  \\
&  & 9 & \foreignlanguage{greek}{δια τον \textoverline{πρα} και ο τρωγων με κακει} & 16 &  &  \\
&  & 16 & \foreignlanguage{greek}{νοϲ ζηϲεται δι εμε ουτοϲ εϲτιν ο αρ} & 4 & \textbf{58} &  \\
&  & 4 & \foreignlanguage{greek}{τοϲ ο εκ του ουρανου καταβαϲ ου κα} & 11 &  &  \\
&  & 11 & \foreignlanguage{greek}{θωϲ εφαγον οι πατερεϲ και απεθανο̅} & 16 &  &  \\
&  & 17 & \foreignlanguage{greek}{ο τρωγων τον αρτον τουτον ζηϲη} & 22 &  &  \\
&  & 23 & \foreignlanguage{greek}{ειϲ τον αιωνα ταυτα ειπεν εν} & 3 & \textbf{59} &  \\
&  & 4 & \foreignlanguage{greek}{ϲυναγωγη διδαϲκων εν καφαρνα} & 7 &  &  \\
&  & 7 & \foreignlanguage{greek}{ουμ πολλοι ουν ακουϲαντεϲ εκ} & 4 & \textbf{60} &  \\
&  & 5 & \foreignlanguage{greek}{των μαθητων αυτου ειπον ϲκλη} & 9 &  &  \\
&  & 9 & \foreignlanguage{greek}{ροϲ ο λογοϲ ουτοϲ τιϲ δυναται αυτου} & 15 &  &  \\
&  & 16 & \foreignlanguage{greek}{ακουειν ιδωϲ δε ο \textoverline{ιϲ} εν εαυτω ο} & 7 & \textbf{61} &  \\
&  & 7 & \foreignlanguage{greek}{τι γογγυζουϲιν περι τουτου οι μαθη} & 12 &  &  \\
&  & 12 & \foreignlanguage{greek}{ται αυτου ειπεν αυτοιϲ τουτο υ} & 17 &  &  \\
&  & 17 & \foreignlanguage{greek}{μαϲ ϲκανδαλιζει εαν ουν ειδη} & 3 & \textbf{62} &  \\
&  & 3 & \foreignlanguage{greek}{ται τον υιον του \textoverline{ανου} αναβαινον} & 8 &  &  \\
&  & 8 & \foreignlanguage{greek}{τα οπου ην το προτερον} & 12 &  &  \\
& \textbf{63} &  & \foreignlanguage{greek}{το \textoverline{πνα} εϲτιν το ζωοποιουν η ϲαρξ} & 7 &  &  \\
&  & 8 & \foreignlanguage{greek}{ουκ ωφελει ουδεν τα ρηματα α ε} & 14 &  &  \\
&  & 14 & \foreignlanguage{greek}{γω λελαληκα υμιν \textoverline{πνα} εϲτιν και} & 19 &  &  \\
&  & 20 & \foreignlanguage{greek}{ζωη εϲτιν αλλα ειϲιν εξ υμων τι} & 5 & \textbf{64} &  \\
&  & 5 & \foreignlanguage{greek}{νεϲ οι ου πιϲτευουϲιν ηδει γαρ εξ} & 11 &  &  \\
&  & 12 & \foreignlanguage{greek}{αρχηϲ ο \textoverline{ιϲ} τινεϲ ειϲιν οι μη πιϲτευ} & 19 &  &  \\
&  & 19 & \foreignlanguage{greek}{οντεϲ και τιϲ εϲτιν ο παραδωϲων} & 24 &  &  \\
&  & 25 & \foreignlanguage{greek}{αυτον και ελεγεν δια τουτο ει} & 5 & \textbf{65} &  \\
&  & 5 & \foreignlanguage{greek}{ρηκα υμιν ουδειϲ δυναται ελθειν} & 9 &  &  \\
&  & 10 & \foreignlanguage{greek}{προϲ με εαν μη η δεδομενον αυτω} & 16 &  &  \\
[0.2em]
\cline{4-4}
\end{tabular}
\end{center}
\end{table}
}
\clearpage
\newpage
 {
 \setlength\arrayrulewidth{1pt}
\begin{table}
\begin{center}
\begin{tabular}{ccc|l|ccc}
\cline{4-4} \\ [-1em]
\multicolumn{7}{c}{\foreignlanguage{greek}{ευαγγελιον κατα ιωαννην} \textbf{(\nospace{6:65})} } \\ \\ [-1em] % Si on veut ajouter les bordures latérales, remplacer {7}{c} par {7}{|c|}
\cline{4-4} \\
\cline{4-4}
&  &  & &  &  & \\ [-0.9em]
&  & 17 & \foreignlanguage{greek}{εκ του \textoverline{πρϲ} εκ τουτου πολλοι των μα} & 5 & \textbf{66} &  \\
&  & 5 & \foreignlanguage{greek}{θητων αυτου απηλθον ειϲ τα οπιϲω} & 10 &  &  \\
&  & 11 & \foreignlanguage{greek}{και ουκετι μετ αυτου περιεπατουν} & 15 &  &  \\
& \textbf{67} &  & \foreignlanguage{greek}{ειπεν ουν ο \textoverline{ιϲ} τοιϲ δωδεκα μη και υ} & 9 &  &  \\
&  & 9 & \foreignlanguage{greek}{μειϲ θελεται υπαγειν} & 11 &  &  \\
& \textbf{68} &  & \foreignlanguage{greek}{απεκριθη αυτω ϲιμων πετροϲ \textoverline{κε} προϲ} & 6 &  &  \\
&  & 7 & \foreignlanguage{greek}{τινα απελευϲομεθα ρηματα ζωηϲ} & 10 &  &  \\
&  & 11 & \foreignlanguage{greek}{αιωνιου εχειϲ και ημειϲ πεπιϲτευ} & 3 & \textbf{69} &  \\
&  & 3 & \foreignlanguage{greek}{καμεν και εγνωκαμεν οτι ϲυ ει ο} & 9 &  &  \\
&  & 10 & \foreignlanguage{greek}{αγιοϲ του \textoverline{θυ}} & 12 &  &  \\
& \textbf{70} &  & \foreignlanguage{greek}{απεκριθη αυτοιϲ ο \textoverline{ιϲ} ουκ εγω υμαϲ} & 7 &  &  \\
&  & 8 & \foreignlanguage{greek}{τουϲ δωδεκα εξελεξαμην και εξ} & 12 &  &  \\
&  & 13 & \foreignlanguage{greek}{υμων ειϲ διαβολοϲ εϲτιν ελεγεν} & 1 & \textbf{71} &  \\
&  & 2 & \foreignlanguage{greek}{δε τον ιουδαν ϲιμωνοϲ ιϲκαριωτου} & 6 &  &  \\
&  & 7 & \foreignlanguage{greek}{ουτοϲ γαρ εμελλεν παραδιδοναι αυ} & 11 &  &  \\
&  & 11 & \foreignlanguage{greek}{τον ειϲ ων εκ των δωδεκα} & 16 &  &  \\
& \mygospelchapter &  & \foreignlanguage{greek}{και μετα ταυτα περιεπατει ο \textoverline{ιϲ} εν τη} & 8 &  &  \\
&  & 9 & \foreignlanguage{greek}{γαλιλαια ου γαρ ειχεν εξουϲιαν εν} & 14 &  &  \\
&  & 15 & \foreignlanguage{greek}{τη ιουδαια περιπατειν οτι εζητου̅} & 19 &  &  \\
&  & 20 & \foreignlanguage{greek}{αυτον οι ιουδαιοι αποκτιναι} & 23 &  &  \\
& \textbf{2} &  & \foreignlanguage{greek}{ην δε εγγυϲ η εορτη των ιουδαιων} & 7 &  &  \\
&  & 8 & \foreignlanguage{greek}{η ϲκηνοπηγια και ειπον προϲ αυ} & 4 & \textbf{3} &  \\
&  & 4 & \foreignlanguage{greek}{τον οι αδελφοι αυτου μεταβηθει} & 8 &  &  \\
&  & 9 & \foreignlanguage{greek}{εντευθεν και υπαγε ειϲ την ιου} & 14 &  &  \\
&  & 14 & \foreignlanguage{greek}{δαιαν ινα και οι μαθηται ϲου θεω} & 20 &  &  \\
&  & 20 & \foreignlanguage{greek}{ρηϲουϲιν τα εργα ϲου α ποιειϲ} & 25 &  &  \\
& \textbf{4} &  & \foreignlanguage{greek}{ουδειϲ γαρ εν κρυπτω τι ποιει και} & 7 &  &  \\
&  & 8 & \foreignlanguage{greek}{ζητει αυτο εν παρρηϲια ειναι} & 12 &  &  \\
&  & 13 & \foreignlanguage{greek}{ει ταυτα ποιειϲ φανερωϲον ϲεαυτον τω} & 18 &  &  \\
&  & 19 & \foreignlanguage{greek}{κοϲμω ουδε γαρ οι αδελφοι αυτου} & 5 & \textbf{5} &  \\
[0.2em]
\cline{4-4}
\end{tabular}
\end{center}
\end{table}
}
\clearpage
\newpage
 {
 \setlength\arrayrulewidth{1pt}
\begin{table}
\begin{center}
\begin{tabular}{ccc|l|ccc}
\cline{4-4} \\ [-1em]
\multicolumn{7}{c}{\foreignlanguage{greek}{ευαγγελιον κατα ιωαννην} \textbf{(\nospace{7:5})} } \\ \\ [-1em] % Si on veut ajouter les bordures latérales, remplacer {7}{c} par {7}{|c|}
\cline{4-4} \\
\cline{4-4}
&  &  & &  &  & \\ [-0.9em]
&  & 6 & \foreignlanguage{greek}{επιϲτευϲαν ειϲ αυτον} & 8 &  &  \\
& \textbf{6} &  & \foreignlanguage{greek}{λεγει αυτοιϲ ο \textoverline{ιϲ} ο καιροϲ ο εμοϲ ουδεπω} & 9 &  &  \\
&  & 10 & \foreignlanguage{greek}{παρεϲτιν ο δε καιροϲ ο υμετεροϲ πα̅} & 16 &  &  \\
&  & 16 & \foreignlanguage{greek}{τοτε εϲτιν ετοιμοϲ ου δυναται ο κοϲ} & 4 & \textbf{7} &  \\
&  & 4 & \foreignlanguage{greek}{μοϲ μιϲιν υμαϲ εμε δε μειϲει οτι ε} & 11 &  &  \\
&  & 11 & \foreignlanguage{greek}{γω μαρτυρω περι αυτου οτι τα εργα} & 17 &  &  \\
&  & 18 & \foreignlanguage{greek}{αυτου πονηρα εϲτιν υμειϲ ανα} & 2 & \textbf{8} &  \\
&  & 2 & \foreignlanguage{greek}{βηται ειϲ την εορτην εγω ουπω α} & 8 &  &  \\
&  & 8 & \foreignlanguage{greek}{ναβαινω ειϲ την εορτην ταυτην} & 12 &  &  \\
&  & 13 & \foreignlanguage{greek}{οτι ο εμοϲ καιροϲ ουπω πεπληρωται} & 18 &  &  \\
& \textbf{9} &  & \foreignlanguage{greek}{ταυτα δε ειπων αυτοϲ εμεινεν εν} & 6 &  &  \\
&  & 7 & \foreignlanguage{greek}{τη γαλιλαια ωϲ δε ανεβηϲαν οι α} & 5 & \textbf{10} &  \\
&  & 5 & \foreignlanguage{greek}{δελφοι αυτου ειϲ την εορτην τοτε} & 10 &  &  \\
&  & 11 & \foreignlanguage{greek}{και αυτοϲ ανεβη ου φανερωϲ αλλ ωϲ} & 17 &  &  \\
&  & 18 & \foreignlanguage{greek}{εν κρυπτω οι ουν ιουδαιοι εζη} & 4 & \textbf{11} &  \\
&  & 4 & \foreignlanguage{greek}{τουν αυτον εν τη εορτη και ελεγον} & 10 &  &  \\
&  & 11 & \foreignlanguage{greek}{που εϲτιν εκεινοϲ και γογγυϲμοϲ} & 2 & \textbf{12} &  \\
&  & 3 & \foreignlanguage{greek}{περι αυτου ην πολυϲ εν τοιϲ οχλοιϲ} & 9 &  &  \\
&  & 10 & \foreignlanguage{greek}{οι μεν ελεγον οτι αγαθοϲ εϲτιν} & 15 &  &  \\
&  & 16 & \foreignlanguage{greek}{αλλοι δε ελεγον ου αλλα πλανα τον} & 22 &  &  \\
&  & 23 & \foreignlanguage{greek}{οχλον ουδειϲ μεντοι παρρηϲια ελα} & 4 & \textbf{13} &  \\
&  & 4 & \foreignlanguage{greek}{λει περι αυτου δια τον φοβον των ιου} & 11 &  &  \\
&  & 11 & \foreignlanguage{greek}{δαιων ηδη δε τηϲ εορτηϲ μεϲηϲ} & 5 & \textbf{14} &  \\
&  & 5 & \foreignlanguage{greek}{ουϲηϲ ανεβη ο \textoverline{ιϲ} ειϲ το ιερον και εδι} & 13 &  &  \\
&  & 13 & \foreignlanguage{greek}{δαϲκεν εθαυμαζον ουν οι ιουδαι} & 4 & \textbf{15} &  \\
&  & 4 & \foreignlanguage{greek}{οι λεγοντεϲ πωϲ ουτοϲ γραμματα} & 8 &  &  \\
&  & 9 & \foreignlanguage{greek}{οιδεν μη μεμαθηκωϲ} & 11 &  &  \\
& \textbf{16} &  & \foreignlanguage{greek}{απεκριθη ουν αυτοιϲ ο \textoverline{ιϲ} και ειπεν} & 7 &  &  \\
&  & 8 & \foreignlanguage{greek}{η εμη διδαχη ουκ εϲτιν εμη αλλα} & 14 &  &  \\
&  & 15 & \foreignlanguage{greek}{του πεμψαντοϲ με εαν τιϲ το θε} & 4 & \textbf{17} &  \\
[0.2em]
\cline{4-4}
\end{tabular}
\end{center}
\end{table}
}
\clearpage
\newpage
 {
 \setlength\arrayrulewidth{1pt}
\begin{table}
\begin{center}
\begin{tabular}{ccc|l|ccc}
\cline{4-4} \\ [-1em]
\multicolumn{7}{c}{\foreignlanguage{greek}{ευαγγελιον κατα ιωαννην} \textbf{(\nospace{7:17})} } \\ \\ [-1em] % Si on veut ajouter les bordures latérales, remplacer {7}{c} par {7}{|c|}
\cline{4-4} \\
\cline{4-4}
&  &  & &  &  & \\ [-0.9em]
&  & 4 & \foreignlanguage{greek}{λημα αυτου ποιη γνωϲεται περι τηϲ} & 9 &  &  \\
&  & 10 & \foreignlanguage{greek}{διδαχηϲ ποτερον εκ του \textoverline{θυ} εϲτιν η ε} & 17 &  &  \\
&  & 17 & \foreignlanguage{greek}{γω απ εμαυτου λαλω} & 20 &  &  \\
& \textbf{18} &  & \foreignlanguage{greek}{ο αφ εαυτου λαλων την δοξαν τη̅} & 7 &  &  \\
&  & 8 & \foreignlanguage{greek}{ιδιαν ζητει ο δε ζητων την δοξα̅} & 14 &  &  \\
&  & 15 & \foreignlanguage{greek}{του πεμψαντοϲ αυτον ουτοϲ αλη} & 19 &  &  \\
&  & 19 & \foreignlanguage{greek}{θηϲ εϲτιν και αδικεια εν αυτω ουκ ε} & 26 &  &  \\
&  & 26 & \foreignlanguage{greek}{ϲτιν ου μωυϲηϲ δεδωκεν υ} & 4 & \textbf{19} &  \\
&  & 4 & \foreignlanguage{greek}{μιν τον νομον και ουδειϲ εξ υμων} & 10 &  &  \\
&  & 11 & \foreignlanguage{greek}{ποιει τον νομον τι με ζητειται α} & 17 &  &  \\
&  & 17 & \foreignlanguage{greek}{ποκτιναι απεκριθη ο οχλοϲ δαι} & 4 & \textbf{20} &  \\
&  & 4 & \foreignlanguage{greek}{μονιον εχειϲ τιϲ ϲε ζητι αποκτιναι} & 9 &  &  \\
& \textbf{21} &  & \foreignlanguage{greek}{απεκριθη ο \textoverline{ιϲ} και ειπεν αυτοιϲ εν ερ} & 8 &  &  \\
&  & 8 & \foreignlanguage{greek}{γον εποιηϲα και παντεϲ θαυμαζεται} & 12 &  &  \\
& \textbf{22} &  & \foreignlanguage{greek}{δια τουτο μωυϲηϲ δεδωκεν υμιν} & 5 &  &  \\
&  & 6 & \foreignlanguage{greek}{την περιτομην ουχ οτι εκ του μωυ} & 12 &  &  \\
&  & 12 & \foreignlanguage{greek}{ϲεωϲ εϲτιν αλλ εκ των πατερων} & 17 &  &  \\
&  & 18 & \foreignlanguage{greek}{και εν ϲαββατω περιτεμνεται αν} & 22 &  &  \\
&  & 22 & \foreignlanguage{greek}{θρωπον ει περιτομην λαμβανει} & 3 & \textbf{23} &  \\
&  & 4 & \foreignlanguage{greek}{\textoverline{ανοϲ} εν ϲαββατω ινα μη λυθη ο νομοϲ} & 11 &  &  \\
&  & 12 & \foreignlanguage{greek}{μωυϲεωϲ εμοι χολατε οτι ολον αν} & 17 &  &  \\
&  & 17 & \foreignlanguage{greek}{θρωπον υγιη εποιηϲα εν ϲαββατω} & 21 &  &  \\
& \textbf{24} &  & \foreignlanguage{greek}{μη κρινεται κατ οψιν αλλα την δι} & 7 &  &  \\
&  & 7 & \foreignlanguage{greek}{καιαν κριϲιν κρινεται} & 9 &  &  \\
& \textbf{25} &  & \foreignlanguage{greek}{ελεγον ουν τινεϲ εκ των ιεροϲολυ} & 6 &  &  \\
&  & 6 & \foreignlanguage{greek}{μιτων ουχ ουτοϲ εϲτιν ον ζητου} & 11 &  &  \\
&  & 11 & \foreignlanguage{greek}{ϲιν αποκτιναι και ειδε παρρηϲι} & 3 & \textbf{26} &  \\
&  & 3 & \foreignlanguage{greek}{α λαλει και ουδεν αυτω λεγουϲιν} & 8 &  &  \\
&  & 9 & \foreignlanguage{greek}{μηποτε αληθωϲ εγνωϲαν οι αρχον} & 13 &  &  \\
&  & 13 & \foreignlanguage{greek}{τεϲ οτι ουτοϲ εϲτιν ο \textoverline{χϲ} αλλα του} & 2 & \textbf{27} &  \\
[0.2em]
\cline{4-4}
\end{tabular}
\end{center}
\end{table}
}
\clearpage
\newpage
 {
 \setlength\arrayrulewidth{1pt}
\begin{table}
\begin{center}
\begin{tabular}{ccc|l|ccc}
\cline{4-4} \\ [-1em]
\multicolumn{7}{c}{\foreignlanguage{greek}{ευαγγελιον κατα ιωαννην} \textbf{(\nospace{7:27})} } \\ \\ [-1em] % Si on veut ajouter les bordures latérales, remplacer {7}{c} par {7}{|c|}
\cline{4-4} \\
\cline{4-4}
&  &  & &  &  & \\ [-0.9em]
&  & 2 & \foreignlanguage{greek}{τον οιδαμεν ποθεν εϲτιν ο δε \textoverline{χϲ} ο} & 9 &  &  \\
&  & 9 & \foreignlanguage{greek}{ταν ερχηται ουδειϲ γιγνωϲκει πο} & 13 &  &  \\
&  & 13 & \foreignlanguage{greek}{θεν εϲτιν} & 14 &  &  \\
& \textbf{28} &  & \foreignlanguage{greek}{εκραξεν ουν εν τω ιερω διδα} & 6 &  &  \\
&  & 6 & \foreignlanguage{greek}{ϲκων \textoverline{ιϲ} λεγων καμε οιδατε και} & 11 &  &  \\
&  & 12 & \foreignlanguage{greek}{οιδατε ποθεν ειμει και απ εμαυτου} & 17 &  &  \\
&  & 18 & \foreignlanguage{greek}{ουκ εληλυθα αλλ εϲτιν αληθει} & 22 &  &  \\
&  & 22 & \foreignlanguage{greek}{νοϲ ο πεμψαϲ με ον υμειϲ ουκ οιδα} & 29 &  &  \\
&  & 29 & \foreignlanguage{greek}{τε εγω οιδα αυτον οτι παρ αυτου ει} & 7 & \textbf{29} &  \\
&  & 7 & \foreignlanguage{greek}{μει κακεινοϲ με απεϲτιλεν} & 10 &  &  \\
& \textbf{30} &  & \foreignlanguage{greek}{εζητουν ουν αυτον πιαϲαι και ουδειϲ} & 6 &  &  \\
&  & 7 & \foreignlanguage{greek}{επεβαλεν επ αυτον ταϲ χειραϲ οτι ου} & 13 &  &  \\
&  & 13 & \foreignlanguage{greek}{πω εληλυθει η ωρα αυτου εκ του ου̅} & 3 & \textbf{31} &  \\
&  & 4 & \foreignlanguage{greek}{οχλου επιϲτευϲαν ειϲ αυτον} & 7 &  &  \\
&  & 8 & \foreignlanguage{greek}{και ελεγον ο \textoverline{χϲ} οταν ελθη μη πλειο} & 15 &  &  \\
&  & 15 & \foreignlanguage{greek}{να ϲημεια ποιηϲει ων ουτοϲ εποιηϲε} & 20 &  &  \\
& \textbf{32} &  & \foreignlanguage{greek}{ηκουϲαν οι φαριϲαιοι του οχλου γογ} & 6 &  &  \\
&  & 6 & \foreignlanguage{greek}{γυζοντοϲ περι αυτου ταυτα} & 9 &  &  \\
&  & 10 & \foreignlanguage{greek}{και απεϲτιλαν οι αρχιερειϲ και οι φα} & 16 &  &  \\
&  & 16 & \foreignlanguage{greek}{ριϲαιοι υπηρεταϲ ινα πιαϲωϲιν αυτο̅} & 20 &  &  \\
& \textbf{33} &  & \foreignlanguage{greek}{ειπεν ουν ο \textoverline{ιϲ} ετι χρονον μικρον μεθ} & 8 &  &  \\
&  & 9 & \foreignlanguage{greek}{υμων ειμει και υπαγω προϲ τον πεμ} & 15 &  &  \\
&  & 15 & \foreignlanguage{greek}{ψαντα με ζητηϲεται με και ουχ ευ} & 5 & \textbf{34} &  \\
&  & 5 & \foreignlanguage{greek}{ρηϲεται και οπου ειμει εγω υμειϲ ου} & 11 &  &  \\
&  & 12 & \foreignlanguage{greek}{δυναϲθαι ελθειν} & 13 &  &  \\
& \textbf{35} &  & \foreignlanguage{greek}{ειπον ουν οι ιουδαιοι προϲ εαυτουϲ} & 6 &  &  \\
&  & 7 & \foreignlanguage{greek}{που ουτοϲ μελλει πορευεϲθαι οτι} & 11 &  &  \\
&  & 12 & \foreignlanguage{greek}{ημειϲ ουχ ευρηϲομεν αυτον μη ειϲ} & 17 &  &  \\
&  & 18 & \foreignlanguage{greek}{την διαϲποραν των ελληνων μελ} & 22 &  &  \\
&  & 22 & \foreignlanguage{greek}{λει πορευεϲθαι και διδαϲκιν τουϲ} & 26 &  &  \\
[0.2em]
\cline{4-4}
\end{tabular}
\end{center}
\end{table}
}
\clearpage
\newpage
 {
 \setlength\arrayrulewidth{1pt}
\begin{table}
\begin{center}
\begin{tabular}{ccc|l|ccc}
\cline{4-4} \\ [-1em]
\multicolumn{7}{c}{\foreignlanguage{greek}{ευαγγελιον κατα ιωαννην} \textbf{(\nospace{7:35})} } \\ \\ [-1em] % Si on veut ajouter les bordures latérales, remplacer {7}{c} par {7}{|c|}
\cline{4-4} \\
\cline{4-4}
&  &  & &  &  & \\ [-0.9em]
&  & 27 & \foreignlanguage{greek}{ελληναϲ τιϲ εϲτιν ο λογοϲ ουτοϲ ον} & 6 & \textbf{36} &  \\
&  & 7 & \foreignlanguage{greek}{ειπεν ζητηϲεται με και ουχ ευρηϲεται} & 12 &  &  \\
&  & 13 & \foreignlanguage{greek}{και οπου ειμει εγω υμειϲ ου δυναϲθαι} & 19 &  &  \\
&  & 20 & \foreignlanguage{greek}{ελθειν} & 20 &  &  \\
& \textbf{37} &  & \foreignlanguage{greek}{εν δε τη εϲχατη ημερα τηϲ εορτηϲ ιϲτη} & 8 &  &  \\
&  & 8 & \foreignlanguage{greek}{κει ο \textoverline{ιϲ} και εκραξεν λεγων ει τιϲ διψα} & 16 &  &  \\
&  & 17 & \foreignlanguage{greek}{ερχεϲθω προϲ με και πεινετω ο πι} & 2 & \textbf{38} &  \\
&  & 2 & \foreignlanguage{greek}{ϲτευων ειϲ εμε καθωϲ ειπεν η γραφη} & 8 &  &  \\
&  & 9 & \foreignlanguage{greek}{ποταμοι εκ τηϲ κοιλιαϲ αυτου ρευϲου} & 14 &  &  \\
&  & 14 & \foreignlanguage{greek}{ϲιν υδατοϲ ζωντοϲ} & 16 &  &  \\
& \textbf{39} &  & \foreignlanguage{greek}{τουτο δε ειπεν περι του \textoverline{πνϲ} ου ελαμ} & 8 &  &  \\
&  & 8 & \foreignlanguage{greek}{βανον οι πιϲτευϲαντεϲ ειϲ αυτον} & 12 &  &  \\
&  & 13 & \foreignlanguage{greek}{ουπω γαρ ην \textoverline{πνα} αγιον οτι \textoverline{ιϲ} ουδεπω} & 20 &  &  \\
&  & 21 & \foreignlanguage{greek}{εδοξαϲθη} & 21 &  &  \\
& \textbf{40} &  & \foreignlanguage{greek}{εκ του οχλου ουν ακουϲαντεϲ των} & 6 &  &  \\
&  & 7 & \foreignlanguage{greek}{λογων αυτου ελεγον ουτοϲ εϲτιν αλη} & 12 &  &  \\
&  & 12 & \foreignlanguage{greek}{θωϲ ο προφητηϲ αλλοι ελεγον οτι} & 3 & \textbf{41} &  \\
&  & 4 & \foreignlanguage{greek}{ουτοϲ εϲτιν ο \textoverline{χϲ} οι δε ελεγον μη γαρ} & 12 &  &  \\
&  & 13 & \foreignlanguage{greek}{εκ τηϲ γαλιλαιαϲ ο \textoverline{χϲ} ερχεται} & 18 &  &  \\
& \textbf{42} &  & \foreignlanguage{greek}{ουχει η γραφη ειπεν οτι εκ του ϲπερμα} & 8 &  &  \\
&  & 8 & \foreignlanguage{greek}{τοϲ \textoverline{δαδ} και απο βηθλεεμ τηϲ κωμηϲ} & 14 &  &  \\
&  & 15 & \foreignlanguage{greek}{οπου ην \textoverline{δαδ} ερχεται ο \textoverline{χϲ}} & 20 &  &  \\
& \textbf{43} &  & \foreignlanguage{greek}{ϲχιϲμα ουν εγενετο εν τω οχλω δι αυ} & 8 &  &  \\
&  & 8 & \foreignlanguage{greek}{τον τινεϲ δε ηθελον εξ αυτων πι} & 6 & \textbf{44} &  \\
&  & 6 & \foreignlanguage{greek}{αϲαι αυτον αλλ ουδειϲ επεβαλεν ε} & 11 &  &  \\
&  & 11 & \foreignlanguage{greek}{π αυτον ταϲ χειραϲ} & 14 &  &  \\
& \textbf{45} &  & \foreignlanguage{greek}{ηλθον ουν οι υπηρεται προϲ τουϲ αρ} & 7 &  &  \\
&  & 7 & \foreignlanguage{greek}{χιερειϲ και φαριϲαιουϲ και ειπον αυ} & 12 &  &  \\
&  & 12 & \foreignlanguage{greek}{αυτοιϲ εκεινοι δια τι ουκ ηγαγεται} & 17 &  &  \\
&  & 18 & \foreignlanguage{greek}{αυτον απεκριθηϲαν αυτοιϲ οι υπη} & 4 & \textbf{46} &  \\
[0.2em]
\cline{4-4}
\end{tabular}
\end{center}
\end{table}
}
\clearpage
\newpage
 {
 \setlength\arrayrulewidth{1pt}
\begin{table}
\begin{center}
\begin{tabular}{ccc|l|ccc}
\cline{4-4} \\ [-1em]
\multicolumn{7}{c}{\foreignlanguage{greek}{ευαγγελιον κατα ιωαννην} \textbf{(\nospace{7:46})} } \\ \\ [-1em] % Si on veut ajouter les bordures latérales, remplacer {7}{c} par {7}{|c|}
\cline{4-4} \\
\cline{4-4}
&  &  & &  &  & \\ [-0.9em]
&  & 4 & \foreignlanguage{greek}{ρεται ουδεποτε ελαληϲεν ουτωϲ \textoverline{ανοϲ}} & 8 &  &  \\
& \textbf{47} &  & \foreignlanguage{greek}{απεκριθηϲαν ουν αυτοιϲ οι φαριϲαιοι} & 5 &  &  \\
&  & 6 & \foreignlanguage{greek}{μη και υμειϲ πεπλανηϲθαι μη τιϲ τω̅} & 3 & \textbf{48} &  \\
&  & 4 & \foreignlanguage{greek}{αρχοντων επιϲτευϲεν ειϲ αυτον η εκ} & 9 &  &  \\
&  & 10 & \foreignlanguage{greek}{των φαριϲαιων αλλα ο οχλοϲ ουτοϲ ο μη} & 6 & \textbf{49} &  \\
&  & 7 & \foreignlanguage{greek}{γινωϲκων τον νομον επαρατοι ειϲιν} & 11 &  &  \\
& \textbf{50} &  & \foreignlanguage{greek}{λεγει νικοδημοϲ προϲ αυτουϲ ο ελθων προϲ} & 7 &  &  \\
&  & 8 & \foreignlanguage{greek}{αυτον το προτερον ειϲ ων εξ αυτων} & 14 &  &  \\
& \textbf{51} &  & \foreignlanguage{greek}{μη ο νομοϲ ημων κρινει τον \textoverline{ανον} εαν} & 8 &  &  \\
&  & 9 & \foreignlanguage{greek}{μη ακουϲη πρωτον παρ αυτου και γνω} & 15 &  &  \\
&  & 16 & \foreignlanguage{greek}{τι ποιει απεκρειθηϲαν και ειπαν} & 3 & \textbf{52} &  \\
&  & 4 & \foreignlanguage{greek}{αυτω μη και ϲυ εκ τηϲ γαλιλαιαϲ ει} & 11 &  &  \\
&  & 12 & \foreignlanguage{greek}{ερευνηϲον ταϲ γραφαϲ και ειδε οτι προ} & 18 &  &  \\
&  & 18 & \foreignlanguage{greek}{φητηϲ εκ τηϲ γαλιλαιαϲ ουκ εγειρεται} & 23 &  &  \\
& \mygospelchapter &  & \foreignlanguage{greek}{παλιν ουν αυτοιϲ ελαληϲεν ο \textoverline{ιϲ} λεγων} & 7 &  &  \\
&  & 10 & \foreignlanguage{greek}{εγω ειμει το φωϲ του κοϲμου ο ακολου} & 17 &  &  \\
&  & 17 & \foreignlanguage{greek}{θων εμοι ου μη περιπατηϲη εν τη ϲκο} & 24 &  &  \\
&  & 24 & \foreignlanguage{greek}{τια αλλ εξει το φωϲ τηϲ ζωηϲ} & 30 &  &  \\
& \textbf{13} &  & \foreignlanguage{greek}{ειπον ουν αυτω οι φαριϲαιοι ϲυ περι ϲε} & 8 &  &  \\
&  & 8 & \foreignlanguage{greek}{αυτου μαρτυρειϲ η μαρτυρια ϲου ου} & 13 &  &  \\
&  & 13 & \foreignlanguage{greek}{κ εϲτιν αληθηϲ} & 15 &  &  \\
& \textbf{14} &  & \foreignlanguage{greek}{απεκριθη \textoverline{ιϲ} και ειπεν αυτοιϲ καν εγω} & 7 &  &  \\
&  & 8 & \foreignlanguage{greek}{μαρτυρω περι εμαυτου η μαρτυρια μου} & 13 &  &  \\
&  & 14 & \foreignlanguage{greek}{αληθηϲ εϲτιν οτι οιδα ποθεν ηλθον} & 19 &  &  \\
&  & 20 & \foreignlanguage{greek}{και που υπαγω υμειϲ δε ουκ οιδατε} & 26 &  &  \\
&  & 27 & \foreignlanguage{greek}{ποθεν ερχομαι και που υπαγω} & 31 &  &  \\
& \textbf{15} &  & \foreignlanguage{greek}{υμειϲ κατα την ϲαρκα κρεινεται εγω} & 6 &  &  \\
&  & 7 & \foreignlanguage{greek}{ου κρεινω ουδενα εαν κρινω δε εγω} & 4 & \textbf{16} &  \\
&  & 5 & \foreignlanguage{greek}{η κριϲειϲ η εμη αληθεινη εϲτιν οτι} & 11 &  &  \\
&  & 12 & \foreignlanguage{greek}{μονοϲ ουκ ειμει αλλ εγω και ο πεμψαϲ} & 19 &  &  \\
[0.2em]
\cline{4-4}
\end{tabular}
\end{center}
\end{table}
}
\clearpage
\newpage
 {
 \setlength\arrayrulewidth{1pt}
\begin{table}
\begin{center}
\begin{tabular}{ccc|l|ccc}
\cline{4-4} \\ [-1em]
\multicolumn{7}{c}{\foreignlanguage{greek}{ευαγγελιον κατα ιωαννην} \textbf{(\nospace{8:16})} } \\ \\ [-1em] % Si on veut ajouter les bordures latérales, remplacer {7}{c} par {7}{|c|}
\cline{4-4} \\
\cline{4-4}
&  &  & &  &  & \\ [-0.9em]
&  & 20 & \foreignlanguage{greek}{με \textoverline{πηρ} και εν τω νομω δε τω υμετερω} & 7 & \textbf{17} &  \\
&  & 8 & \foreignlanguage{greek}{γεγραπται οτι δυο \textoverline{ανων} η μαρτυρια αλη} & 14 &  &  \\
&  & 14 & \foreignlanguage{greek}{θηϲ εϲτιν εγω ειμει ο μαρτυρων περι} & 5 & \textbf{18} &  \\
&  & 6 & \foreignlanguage{greek}{εμαυτου και μαρτυρι περι εμου ο πεμ} & 12 &  &  \\
&  & 12 & \foreignlanguage{greek}{ψαϲ με \textoverline{πηρ} ελεγον ουν αυτω που εϲτι̅} & 5 & \textbf{19} &  \\
&  & 6 & \foreignlanguage{greek}{ο \textoverline{πηρ} ϲου απεκριθη ο \textoverline{ιϲ} ουτε εμε οι} & 14 &  &  \\
&  & 14 & \foreignlanguage{greek}{δατε ουτε τον \textoverline{πρα} μου ει εμε ηδειται} & 21 &  &  \\
&  & 22 & \foreignlanguage{greek}{και τον \textoverline{πρα} μου αν ηδειται ταυτα τα} & 2 & \textbf{20} &  \\
&  & 3 & \foreignlanguage{greek}{ρηματα ελαληϲεν εν τω γαζοφυλακιω} & 7 &  &  \\
&  & 8 & \foreignlanguage{greek}{διδαϲκων εν τω ιερω και ουδειϲ επι} & 14 &  &  \\
&  & 14 & \foreignlanguage{greek}{αϲεν αυτον οτι ουπω εληλυθει η ωρα} & 20 &  &  \\
&  & 21 & \foreignlanguage{greek}{αυτου ειπεν ουν παλιν αυτοιϲ} & 4 & \textbf{21} &  \\
&  & 5 & \foreignlanguage{greek}{εγω υπαγω και ζητηϲεται με και εν τη} & 12 &  &  \\
&  & 13 & \foreignlanguage{greek}{αμαρτια υμων αποθανειϲθαι} & 15 &  &  \\
&  & 16 & \foreignlanguage{greek}{οπου εγω υπαγω υμειϲ ου δυναϲθαι ελθει̅} & 22 &  &  \\
& \textbf{22} &  & \foreignlanguage{greek}{ελεγον ουν οι ιουδαιοι μητι αποκτενει} & 6 &  &  \\
&  & 7 & \foreignlanguage{greek}{εαυτον οτι λεγει οπου εγω υπαγω υμειϲ} & 13 &  &  \\
&  & 14 & \foreignlanguage{greek}{ου δυναϲθαι ελθειν και ελεγεν αυ} & 3 & \textbf{23} &  \\
&  & 3 & \foreignlanguage{greek}{τοιϲ υμειϲ εκ των κατω εϲται εγω εκ} & 10 &  &  \\
&  & 11 & \foreignlanguage{greek}{των ανω ειμει υμειϲ εκ τουτου του} & 17 &  &  \\
&  & 18 & \foreignlanguage{greek}{κοϲμου εϲται εγω ουκ ειμει εκ του} & 24 &  &  \\
&  & 24 & \foreignlanguage{greek}{του του κοϲμου ειπον ουν υμιν α} & 4 & \textbf{24} &  \\
&  & 4 & \foreignlanguage{greek}{ποθανειϲθαι εν ταιϲ αμαρτιαιϲ υμων} & 8 &  &  \\
&  & 9 & \foreignlanguage{greek}{εαν γαρ μη πιϲτευϲηται οτι εγω ειμει} & 15 &  &  \\
&  & 16 & \foreignlanguage{greek}{αποθανειϲθε εν ταιϲ αμαρτιαιϲ υμω̅} & 20 &  &  \\
& \textbf{25} &  & \foreignlanguage{greek}{ειπον ουν αυτω ϲυ τιϲ ει ειπεν αυτοιϲ} & 8 &  &  \\
&  & 9 & \foreignlanguage{greek}{ο \textoverline{ιϲ} την αρχην ο τι και λαλω υμιν} & 17 &  &  \\
& \textbf{26} &  & \foreignlanguage{greek}{πολλα εχω περι υμων ειπειν και κρι} & 7 &  &  \\
&  & 7 & \foreignlanguage{greek}{νειν αλλα ο πεμψαϲ με αληθηϲ εϲτι̅} & 13 &  &  \\
&  & 14 & \foreignlanguage{greek}{καγω α ηκουϲα παρ αυτου ταυτα λαλω} & 20 &  &  \\
[0.2em]
\cline{4-4}
\end{tabular}
\end{center}
\end{table}
}
\clearpage
\newpage
 {
 \setlength\arrayrulewidth{1pt}
\begin{table}
\begin{center}
\begin{tabular}{ccc|l|ccc}
\cline{4-4} \\ [-1em]
\multicolumn{7}{c}{\foreignlanguage{greek}{ευαγγελιον κατα ιωαννην} \textbf{(\nospace{8:26})} } \\ \\ [-1em] % Si on veut ajouter les bordures latérales, remplacer {7}{c} par {7}{|c|}
\cline{4-4} \\
\cline{4-4}
&  &  & &  &  & \\ [-0.9em]
&  & 21 & \foreignlanguage{greek}{ειϲ τον κοϲμον ουκ εγνωϲαν οτι τον} & 4 & \textbf{27} &  \\
&  & 5 & \foreignlanguage{greek}{\textoverline{πρα} αυτοιϲ ελεγεν ειπεν ουν ο \textoverline{ιϲ} οτα̅} & 5 & \textbf{28} &  \\
&  & 6 & \foreignlanguage{greek}{υψωϲηται τον υιον του \textoverline{ανου} τοτε γνω} & 12 &  &  \\
&  & 12 & \foreignlanguage{greek}{ϲεϲθαι οτι εγω ειμει και απ εμαυτου} & 18 &  &  \\
&  & 19 & \foreignlanguage{greek}{ποιω ουδεν αλλα καθωϲ εδειδαξεν} & 23 &  &  \\
&  & 24 & \foreignlanguage{greek}{με ταυτα λαλω και ο πεμψαϲ με με} & 5 & \textbf{29} &  \\
&  & 5 & \foreignlanguage{greek}{τ εμου εϲτιν ουκ αφηκεν με μονο̅} & 11 &  &  \\
&  & 12 & \foreignlanguage{greek}{οτι εγω τα αρεϲτα αυτω ποιω παντοτε} & 18 &  &  \\
& \textbf{30} &  & \foreignlanguage{greek}{ταυτα αυτου λαλουντοϲ πολλοι επιϲτευ} & 5 &  &  \\
&  & 5 & \foreignlanguage{greek}{ϲαν ειϲ αυτον ελεγεν ουν ο \textoverline{ιϲ}} & 4 & \textbf{31} &  \\
&  & 5 & \foreignlanguage{greek}{προϲ τουϲ πεπιϲτευκοταϲ αυτω ιου} & 9 &  &  \\
&  & 9 & \foreignlanguage{greek}{δαιουϲ εαν υμειϲ μενηται εν τω} & 14 &  &  \\
&  & 15 & \foreignlanguage{greek}{λογω τω εμω αληθωϲ μαθηται μου} & 20 &  &  \\
&  & 21 & \foreignlanguage{greek}{εϲται και γνωϲεϲθαι την αληθειαν} & 4 & \textbf{32} &  \\
&  & 5 & \foreignlanguage{greek}{και η αληθεια ελευθερωϲει υμαϲ} & 9 &  &  \\
& \textbf{33} &  & \foreignlanguage{greek}{απεκριθηϲαν προϲ αυτον ϲπερμα αβρα} & 5 &  &  \\
&  & 5 & \foreignlanguage{greek}{αμ εϲμεν και ουδενι δεδουλευκα} & 9 &  &  \\
&  & 9 & \foreignlanguage{greek}{μεν πωποτε πωϲ ϲυ λεγειϲ ελευθε} & 14 &  &  \\
&  & 14 & \foreignlanguage{greek}{ροι γενηϲεϲθαι} & 15 &  &  \\
& \textbf{34} &  & \foreignlanguage{greek}{απεκριθη αυτοιϲ ο \textoverline{ιϲ} αμην αμην λεγω} & 7 &  &  \\
&  & 8 & \foreignlanguage{greek}{υμιν οτι παϲ ο ποιων την αμαρτια̅} & 14 &  &  \\
&  & 15 & \foreignlanguage{greek}{δουλοϲ εϲτιν τηϲ αμαρτιαϲ ο δε δου} & 3 & \textbf{35} &  \\
&  & 3 & \foreignlanguage{greek}{λοϲ ου μενει εν τη οικεια ειϲ τον αι} & 11 &  &  \\
&  & 11 & \foreignlanguage{greek}{ωνα εαν ουν υμαϲ ελευθερωϲη} & 4 & \textbf{36} &  \\
&  & 5 & \foreignlanguage{greek}{οντωϲ ελευθεροι εϲεϲθαι} & 7 &  &  \\
& \textbf{37} &  & \foreignlanguage{greek}{οιδα οτι ϲπερμα αβρααμ εϲται αλ} & 6 &  &  \\
&  & 6 & \foreignlanguage{greek}{λα ζητειται με αποκτιναι οτι ο λο} & 12 &  &  \\
&  & 12 & \foreignlanguage{greek}{γοϲ ο εμοϲ ου χωρει εν υμιν} & 18 &  &  \\
& \textbf{38} &  & \foreignlanguage{greek}{α εγω εωρακα απο του \textoverline{πρϲ} ταυτα λαλω} & 8 &  &  \\
&  & 9 & \foreignlanguage{greek}{και υμειϲ α ηκουϲατε παρα του \textoverline{πρϲ}} & 15 &  &  \\
[0.2em]
\cline{4-4}
\end{tabular}
\end{center}
\end{table}
}
\clearpage
\newpage
 {
 \setlength\arrayrulewidth{1pt}
\begin{table}
\begin{center}
\begin{tabular}{ccc|l|ccc}
\cline{4-4} \\ [-1em]
\multicolumn{7}{c}{\foreignlanguage{greek}{ευαγγελιον κατα ιωαννην} \textbf{(\nospace{8:38})} } \\ \\ [-1em] % Si on veut ajouter les bordures latérales, remplacer {7}{c} par {7}{|c|}
\cline{4-4} \\
\cline{4-4}
&  &  & &  &  & \\ [-0.9em]
&  & 16 & \foreignlanguage{greek}{ποιειται απεκριθηϲαν και ειπον αυτω} & 4 & \textbf{39} &  \\
&  & 5 & \foreignlanguage{greek}{ο \textoverline{πηρ} ημων αβρααμ εϲτιν λεγει αυτοιϲ} & 11 &  &  \\
&  & 12 & \foreignlanguage{greek}{ο \textoverline{ιϲ} ει τεκνα του αβρααμ ητε τα εργα} & 20 &  &  \\
&  & 21 & \foreignlanguage{greek}{του αβρααμ εποιειτε νυν δε ζητει} & 3 & \textbf{40} &  \\
&  & 3 & \foreignlanguage{greek}{ται με αποκτιναι \textoverline{ανον} οϲ την αληθει} & 9 &  &  \\
&  & 9 & \foreignlanguage{greek}{αν υμιν λελαληκα ην ηκουϲα παρα} & 14 &  &  \\
&  & 15 & \foreignlanguage{greek}{του \textoverline{θυ} τουτο αβρααμ ουκ εποιηϲεν} & 20 &  &  \\
& \textbf{41} &  & \foreignlanguage{greek}{υμειϲ ποιειται τα εργα του \textoverline{πρϲ} υμων} & 7 &  &  \\
&  & 8 & \foreignlanguage{greek}{ειπαν αυτω ημειϲ εκ πορνιαϲ ου γεγε} & 14 &  &  \\
&  & 14 & \foreignlanguage{greek}{νημεθα ενα \textoverline{πρα} εχομεν τον \textoverline{θν}} & 19 &  &  \\
& \textbf{42} &  & \foreignlanguage{greek}{ειπεν αυτοιϲ ο \textoverline{ιϲ} ει ο \textoverline{θϲ} \textoverline{πηρ} υμων ην} & 10 &  &  \\
&  & 11 & \foreignlanguage{greek}{ηγαπατε αν εμε εγω γαρ εκ του \textoverline{θυ}} & 18 &  &  \\
&  & 19 & \foreignlanguage{greek}{εξηλθον και ηκω ουδε γαρ απ εμαυ} & 25 &  &  \\
&  & 25 & \foreignlanguage{greek}{του ουκ εληλυθα αλλα εκεινοϲ με α} & 31 &  &  \\
&  & 31 & \foreignlanguage{greek}{πεϲτιλεν δια τι την λαλιαν την ε} & 6 & \textbf{43} &  \\
&  & 6 & \foreignlanguage{greek}{μην ου γινωϲκεται οτι ου δυναϲθαι} & 11 &  &  \\
&  & 12 & \foreignlanguage{greek}{ακουειν τον λογον τον εμον υμειϲ} & 1 & \textbf{44} &  \\
&  & 2 & \foreignlanguage{greek}{εκ του \textoverline{πρϲ} του διαβολου εϲται και ταϲ} & 9 &  &  \\
&  & 10 & \foreignlanguage{greek}{επιθυμειαϲ του \textoverline{πρϲ} υμων θελεται} & 14 &  &  \\
&  & 15 & \foreignlanguage{greek}{ποιειν εκεινοϲ ανθρωποκτονοϲ η̅} & 18 &  &  \\
&  & 19 & \foreignlanguage{greek}{απ αρχηϲ και εν τη αληθεια ουκ εϲτη} & 26 &  &  \\
&  & 26 & \foreignlanguage{greek}{κεν οτι ουκ εϲτιν αληθεια εν αυτω} & 32 &  &  \\
&  & 33 & \foreignlanguage{greek}{οταν λαλη το ψευδοϲ εκ των ιδιων} & 39 &  &  \\
&  & 40 & \foreignlanguage{greek}{λαλει οτι ψευϲτηϲ εϲτιν και ο \textoverline{πηρ} αυτου} & 47 &  &  \\
& \textbf{45} &  & \foreignlanguage{greek}{εγω δε οτι την αληθειαν λεγω ου πι} & 8 &  &  \\
&  & 8 & \foreignlanguage{greek}{ϲτευεται μοι τιϲ εξ υμων ελεγχει} & 4 & \textbf{46} &  \\
&  & 5 & \foreignlanguage{greek}{με περι αμαρτιαϲ ει αληθειαν λεγω δι} & 11 &  &  \\
&  & 11 & \foreignlanguage{greek}{α τι ου πιϲτευεται μοι ο ων εκ του} & 4 & \textbf{47} &  \\
&  & 5 & \foreignlanguage{greek}{\textoverline{θυ} τα ρηματα του \textoverline{θυ} ακουει δια του} & 12 &  &  \\
&  & 12 & \foreignlanguage{greek}{το υμειϲ ουκ ακουεται οτι εκ του \textoverline{θυ}} & 19 &  &  \\
[0.2em]
\cline{4-4}
\end{tabular}
\end{center}
\end{table}
}
\clearpage
\newpage
 {
 \setlength\arrayrulewidth{1pt}
\begin{table}
\begin{center}
\begin{tabular}{ccc|l|ccc}
\cline{4-4} \\ [-1em]
\multicolumn{7}{c}{\foreignlanguage{greek}{ευαγγελιον κατα ιωαννην} \textbf{(\nospace{8:47})} } \\ \\ [-1em] % Si on veut ajouter les bordures latérales, remplacer {7}{c} par {7}{|c|}
\cline{4-4} \\
\cline{4-4}
&  &  & &  &  & \\ [-0.9em]
&  & 20 & \foreignlanguage{greek}{ουκ εϲται απεκριθηϲαν οι ιουδαιοι} & 3 & \textbf{48} &  \\
&  & 4 & \foreignlanguage{greek}{και ειπαν αυτω ου καλωϲ λεγομεν} & 9 &  &  \\
&  & 10 & \foreignlanguage{greek}{ημειϲ οτι ϲαμαριτηϲ ει ϲυ και δαιμο} & 16 &  &  \\
&  & 16 & \foreignlanguage{greek}{νιον εχειϲ απεκριθη \textoverline{ιϲ} εγω δαι} & 4 & \textbf{49} &  \\
&  & 4 & \foreignlanguage{greek}{μονιον ουκ εχω αλλα τιμω τον \textoverline{πρα}} & 10 &  &  \\
&  & 11 & \foreignlanguage{greek}{μου και υμειϲ ατιμαζεται με} & 15 &  &  \\
& \textbf{50} &  & \foreignlanguage{greek}{εγω δε ου ζητω την δοξαν μου εϲτι̅} & 8 &  &  \\
&  & 9 & \foreignlanguage{greek}{ο ζητων και κρινων αμην αμην} & 2 & \textbf{51} &  \\
&  & 3 & \foreignlanguage{greek}{λεγω υμιν εαν τιϲ τον εμον λογον τη} & 10 &  &  \\
&  & 10 & \foreignlanguage{greek}{ρηϲη θανατον ου μη θεωρηϲη ειϲ το̅} & 16 &  &  \\
&  & 17 & \foreignlanguage{greek}{αιωνα ειπον αυτω οι ιουδαιοι} & 4 & \textbf{52} &  \\
&  & 5 & \foreignlanguage{greek}{νυν εγνωκαμεν οτι δαιμονιον εχειϲ} & 9 &  &  \\
&  & 10 & \foreignlanguage{greek}{αβρααμ απεθανεν και οι προφηται} & 14 &  &  \\
&  & 15 & \foreignlanguage{greek}{και ϲυ λεγειϲ εαν τιϲ τον λογον μου} & 22 &  &  \\
&  & 23 & \foreignlanguage{greek}{τηρηϲη ου μη γευϲηται θανατου ειϲ} & 28 &  &  \\
&  & 29 & \foreignlanguage{greek}{τον αιωνα μη ϲυ μειζων ει του α} & 6 & \textbf{53} &  \\
&  & 6 & \foreignlanguage{greek}{βρααμ οϲτιϲ απεθανεν και οι προ} & 11 &  &  \\
&  & 11 & \foreignlanguage{greek}{φηται απεθανον τινα ϲεαυτον ποιειϲ} & 15 &  &  \\
& \textbf{54} &  & \foreignlanguage{greek}{απεκριθη \textoverline{ιϲ} εαν εγω δοξαϲω εμαυτο̅} & 6 &  &  \\
&  & 7 & \foreignlanguage{greek}{η δοξα μου ουδεν εϲτιν εϲτιν ο \textoverline{πηρ}} & 14 &  &  \\
&  & 15 & \foreignlanguage{greek}{ο δοξαζων με ον υμειϲ λεγεται ο} & 21 &  &  \\
&  & 21 & \foreignlanguage{greek}{τι \textoverline{θϲ} ημων εϲτιν και ουκ εγνωκα} & 3 & \textbf{55} &  \\
&  & 3 & \foreignlanguage{greek}{τε αυτον εγω δε οιδα αυτον καν} & 9 &  &  \\
&  & 10 & \foreignlanguage{greek}{ειπω οτι ουκ οιδα αυτον εϲομαι ομοι} & 16 &  &  \\
&  & 16 & \foreignlanguage{greek}{οϲ υμιν ψευϲτηϲ αλλα οιδα αυτο̅} & 21 &  &  \\
&  & 22 & \foreignlanguage{greek}{και τον λογον αυτου τηρω} & 26 &  &  \\
& \textbf{56} &  & \foreignlanguage{greek}{αβρααμ ο \textoverline{πηρ} υμων ηγαλλιαϲατο} & 5 &  &  \\
&  & 6 & \foreignlanguage{greek}{ινα ειδη την ημεραν την εμην} & 11 &  &  \\
&  & 12 & \foreignlanguage{greek}{και ειδεν και εχαρη} & 15 &  &  \\
& \textbf{57} &  & \foreignlanguage{greek}{ειπον ουν οι ιουδαιοι προϲ αυτον πε̅} & 7 &  &  \\
[0.2em]
\cline{4-4}
\end{tabular}
\end{center}
\end{table}
}
\clearpage
\newpage
 {
 \setlength\arrayrulewidth{1pt}
\begin{table}
\begin{center}
\begin{tabular}{ccc|l|ccc}
\cline{4-4} \\ [-1em]
\multicolumn{7}{c}{\foreignlanguage{greek}{ευαγγελιον κατα ιωαννην} \textbf{(\nospace{8:57})} } \\ \\ [-1em] % Si on veut ajouter les bordures latérales, remplacer {7}{c} par {7}{|c|}
\cline{4-4} \\
\cline{4-4}
&  &  & &  &  & \\ [-0.9em]
&  & 7 & \foreignlanguage{greek}{τηκοντα ετη ουπω εχειϲ και αβρααμ ε} & 13 &  &  \\
&  & 13 & \foreignlanguage{greek}{ωρακαϲ ειπεν αυτοιϲ ο \textoverline{ιϲ} αμην αμη̅} & 6 & \textbf{58} &  \\
&  & 7 & \foreignlanguage{greek}{λεγω υμιν πριν αβρααμ γενεϲθαι εγω} & 12 &  &  \\
&  & 13 & \foreignlanguage{greek}{ειμει ηραν ουν λιθουϲ ινα βαλωϲιν} & 5 & \textbf{59} &  \\
&  & 6 & \foreignlanguage{greek}{επ αυτον \textoverline{ιϲ} εκρυβη και εξηλθεν εκ του} & 13 &  &  \\
& \mygospelchapter &  & \foreignlanguage{greek}{ιερου και παραγων ειδεν \textoverline{ανον}} & 5 &  &  \\
&  & 6 & \foreignlanguage{greek}{τυφλον εκ γενετηϲ και ηρωτηϲαν αυ} & 3 & \textbf{2} &  \\
&  & 3 & \foreignlanguage{greek}{τον οι μαθηται αυτου λεγοντεϲ ραβ} & 8 &  &  \\
&  & 8 & \foreignlanguage{greek}{βει τιϲ ημαρτεν ουτοϲ η οι γονειϲ αυτου} & 15 &  &  \\
&  & 16 & \foreignlanguage{greek}{ινα τυφλοϲ γεννηθη απεκριθη \textoverline{ιϲ}} & 2 & \textbf{3} &  \\
&  & 3 & \foreignlanguage{greek}{ουτε ουτοϲ ημαρτεν ουτε οι γονειϲ αυ} & 9 &  &  \\
&  & 9 & \foreignlanguage{greek}{του αλλ ινα φανερωθη τα εργα του \textoverline{θυ}} & 16 &  &  \\
&  & 17 & \foreignlanguage{greek}{εν αυτω ημαϲ διεργαζεϲθαι τα εργα} & 4 & \textbf{4} &  \\
&  & 5 & \foreignlanguage{greek}{του πεμψαντοϲ ημαϲ ωϲ ημερα εϲτιν} & 10 &  &  \\
&  & 11 & \foreignlanguage{greek}{ερχεται νυξ οτε ουδειϲ δυναται εργαζε} & 16 &  &  \\
&  & 16 & \foreignlanguage{greek}{ϲθαι οταν εν τω κοϲμω ω φωϲ ειμει του} & 8 & \textbf{5} &  \\
&  & 9 & \foreignlanguage{greek}{κοϲμου ταυτα ειπων επτυϲεν χαμε} & 4 & \textbf{6} &  \\
&  & 5 & \foreignlanguage{greek}{και εποιηϲεν πηλον εκ του πτυϲματοϲ} & 10 &  &  \\
&  & 11 & \foreignlanguage{greek}{και επεχριϲεν τον πηλον επι τουϲ ο} & 17 &  &  \\
&  & 17 & \foreignlanguage{greek}{φθαλμουϲ τυ τυφλου και ειπεν αυτω} & 3 & \textbf{7} &  \\
&  & 4 & \foreignlanguage{greek}{υπαγε νειψε ειϲ την κολυμβηθραν} & 8 &  &  \\
&  & 9 & \foreignlanguage{greek}{του ϲιλωαμ ο ερμηνευεται απεϲταλ} & 13 &  &  \\
&  & 13 & \foreignlanguage{greek}{μενοϲ απηλθεν ουν και ενιψατο και} & 18 &  &  \\
&  & 19 & \foreignlanguage{greek}{ηλθεν βλεπων οι ουν γειτονεϲ} & 3 & \textbf{8} &  \\
&  & 4 & \foreignlanguage{greek}{και οι θεωρουντεϲ αυτον το προτερο̅} & 9 &  &  \\
&  & 10 & \foreignlanguage{greek}{οτι προϲετηϲ ην ελεγον ουχ ουτοϲ} & 15 &  &  \\
&  & 16 & \foreignlanguage{greek}{εϲτιν ο καθημενοϲ και προϲετων} & 20 &  &  \\
& \textbf{9} &  & \foreignlanguage{greek}{αλλοι ελεγον ουτοϲ εϲτιν αλλοι ελε} & 6 &  &  \\
&  & 6 & \foreignlanguage{greek}{γον ουχι αλλα ομοιοϲ αυτω εϲτιν} & 11 &  &  \\
&  & 12 & \foreignlanguage{greek}{εκεινοϲ ελεγεν οτι εγω ειμει} & 16 &  &  \\
[0.2em]
\cline{4-4}
\end{tabular}
\end{center}
\end{table}
}
\clearpage
\newpage
 {
 \setlength\arrayrulewidth{1pt}
\begin{table}
\begin{center}
\begin{tabular}{ccc|l|ccc}
\cline{4-4} \\ [-1em]
\multicolumn{7}{c}{\foreignlanguage{greek}{ευαγγελιον κατα ιωαννην} \textbf{(\nospace{9:10})} } \\ \\ [-1em] % Si on veut ajouter les bordures latérales, remplacer {7}{c} par {7}{|c|}
\cline{4-4} \\
\cline{4-4}
&  &  & &  &  & \\ [-0.9em]
& \textbf{10} &  & \foreignlanguage{greek}{ελεγον ουν αυτω πωϲ ηνεωχθηϲαν ϲου} & 6 &  &  \\
&  & 7 & \foreignlanguage{greek}{οι οφθαλμοι απεκριθη εκεινοϲ \textoverline{ανοϲ} λε} & 4 & \textbf{11} &  \\
&  & 4 & \foreignlanguage{greek}{γομενοϲ \textoverline{ιϲ} πηλον εποιηϲεν και επε} & 9 &  &  \\
&  & 9 & \foreignlanguage{greek}{χριϲεν μου τουϲ οφθαλμουϲ και ει} & 14 &  &  \\
&  & 14 & \foreignlanguage{greek}{πεν μοι υπαγε ειϲ τον ϲιλωαμ και νι} & 21 &  &  \\
&  & 21 & \foreignlanguage{greek}{ψαι απελθων ουν και νιψαμενοϲ ανε} & 26 &  &  \\
&  & 26 & \foreignlanguage{greek}{βλεψα και ειπαν αυτω που εϲτιν εκει} & 6 & \textbf{12} &  \\
&  & 6 & \foreignlanguage{greek}{νοϲ λεγει ουκ οιδα αγουϲιν αυτον προϲ} & 3 & \textbf{13} &  \\
&  & 4 & \foreignlanguage{greek}{τουϲ φαριϲαιουϲ τον ποτε τυφλον} & 8 &  &  \\
& \textbf{14} &  & \foreignlanguage{greek}{ην δε ϲαββατον εν η ημερα τον πηλον} & 8 &  &  \\
&  & 9 & \foreignlanguage{greek}{εποιηϲεν ο \textoverline{ιϲ} και ηνεωξεν αυτου τουϲ} & 15 &  &  \\
&  & 16 & \foreignlanguage{greek}{οφθαλμουϲ παλιν ουν ηρωτων αυ} & 4 & \textbf{15} &  \\
&  & 4 & \foreignlanguage{greek}{τον και οι φαριϲαιοι πωϲ ανεβλεψεν} & 9 &  &  \\
&  & 10 & \foreignlanguage{greek}{ο δε ειπεν αυτοιϲ πηλον επεθηκεν μου} & 16 &  &  \\
&  & 17 & \foreignlanguage{greek}{επι τουϲ οφθαλμουϲ και ενιψαμην και} & 22 &  &  \\
&  & 23 & \foreignlanguage{greek}{βλεπω ελεγον ουν εκ των φαριϲαι} & 5 & \textbf{16} &  \\
&  & 5 & \foreignlanguage{greek}{ων τινεϲ ουκ εϲτιν ουτοϲ παρα \textoverline{θυ} ο} & 12 &  &  \\
&  & 13 & \foreignlanguage{greek}{\textoverline{ανοϲ} οτι το ϲαββατον ου τηρει} & 18 &  &  \\
&  & 19 & \foreignlanguage{greek}{αλλοι δε ελεγον πωϲ δυναται \textoverline{ανοϲ}} & 24 &  &  \\
&  & 25 & \foreignlanguage{greek}{αμαρτωλοϲ ϲημια τοιαυτα ποιειν} & 28 &  &  \\
&  & 29 & \foreignlanguage{greek}{και ϲχιϲμα ην εν αυτοιϲ λεγουϲιν} & 1 & \textbf{17} &  \\
&  & 2 & \foreignlanguage{greek}{ουν τω τυφλω παλιν ϲυ τι λεγειϲ πε} & 9 &  &  \\
&  & 9 & \foreignlanguage{greek}{ρι αυτου οτι ηνεωξεν ϲου τουϲ οφθαλ} & 15 &  &  \\
&  & 15 & \foreignlanguage{greek}{μουϲ ο δε ειπεν οτι προφητηϲ εϲτι̅} & 21 &  &  \\
& \textbf{18} &  & \foreignlanguage{greek}{ουκ επιϲτευϲαν ουν οι ιουδαιοι πε} & 6 &  &  \\
&  & 6 & \foreignlanguage{greek}{ρι αυτου οτι ην τυφλοϲ και ανεβλεψε̅} & 12 &  &  \\
&  & 13 & \foreignlanguage{greek}{εωϲ οτου εφωνηϲαν τουϲ γονειϲ αυ} & 18 &  &  \\
&  & 18 & \foreignlanguage{greek}{του του αναβλεψαντοϲ και ηρωτη} & 2 & \textbf{19} &  \\
&  & 2 & \foreignlanguage{greek}{ϲαν αυτουϲ ουτοϲ εϲτιν ο υιοϲ υ} & 8 &  &  \\
&  & 8 & \foreignlanguage{greek}{μων ον υμειϲ λεγεται οτι τυφλοϲ ε} & 14 &  &  \\
[0.2em]
\cline{4-4}
\end{tabular}
\end{center}
\end{table}
}
\clearpage
\newpage
 {
 \setlength\arrayrulewidth{1pt}
\begin{table}
\begin{center}
\begin{tabular}{ccc|l|ccc}
\cline{4-4} \\ [-1em]
\multicolumn{7}{c}{\foreignlanguage{greek}{ευαγγελιον κατα ιωαννην} \textbf{(\nospace{9:19})} } \\ \\ [-1em] % Si on veut ajouter les bordures latérales, remplacer {7}{c} par {7}{|c|}
\cline{4-4} \\
\cline{4-4}
&  &  & &  &  & \\ [-0.9em]
&  & 14 & \foreignlanguage{greek}{γεννηθη πωϲ ουν βλεπει αρτι απεκρι} & 1 & \textbf{20} &  \\
&  & 1 & \foreignlanguage{greek}{θηϲαν οι γονειϲ αυτου και ειπαν οιδα} & 7 &  &  \\
&  & 7 & \foreignlanguage{greek}{μεν οτι ουτοϲ εϲτιν ο υιοϲ ημων και} & 14 &  &  \\
&  & 15 & \foreignlanguage{greek}{οτι τυφλοϲ εγεννηθη πωϲ δε νυν βλε} & 4 & \textbf{21} &  \\
&  & 4 & \foreignlanguage{greek}{πει ουκ οιδαμεν η τιϲ ηνεωξεν αυ} & 10 &  &  \\
&  & 10 & \foreignlanguage{greek}{του τουϲ οφθαλμουϲ ημειϲ ουκ οιδα} & 15 &  &  \\
&  & 15 & \foreignlanguage{greek}{μεν ηλικειαν εχει αυτοϲ περι εαυτου} & 20 &  &  \\
&  & 21 & \foreignlanguage{greek}{λαληϲει ταυτα ειπον οι γονειϲ αυτου} & 5 & \textbf{22} &  \\
&  & 6 & \foreignlanguage{greek}{οτι εφοβουντο τουϲ ιουδαιουϲ ηδη} & 10 &  &  \\
&  & 11 & \foreignlanguage{greek}{γαρ ϲυνεθεντο οι ιουδαιοι ινα αν τιϲ} & 17 &  &  \\
&  & 18 & \foreignlanguage{greek}{αυτον ομολογηϲη \textoverline{χν} αποϲυναγω} & 21 &  &  \\
&  & 21 & \foreignlanguage{greek}{γοϲ γενηται δια τουτο οι γονειϲ αυ} & 5 & \textbf{23} &  \\
&  & 5 & \foreignlanguage{greek}{του ειπον οτι ηλικειαν εχει αυτον ε} & 11 &  &  \\
&  & 11 & \foreignlanguage{greek}{περωτηϲατε εφωνηϲαν ουν τον} & 3 & \textbf{24} &  \\
&  & 4 & \foreignlanguage{greek}{ανθρωπον εκ δευτερου οϲ ην τυφλοϲ} & 9 &  &  \\
&  & 10 & \foreignlanguage{greek}{και ειπαν αυτω δοϲ δοξαν τω \textoverline{θω}} & 16 &  &  \\
&  & 17 & \foreignlanguage{greek}{ημειϲ οιδαμεν οτι ουτοϲ ο \textoverline{ανοϲ} αμαρ} & 23 &  &  \\
&  & 23 & \foreignlanguage{greek}{τωλοϲ εϲτιν απεκριθη ουν εκει} & 3 & \textbf{25} &  \\
&  & 3 & \foreignlanguage{greek}{νοϲ ει αμαρτωλοϲ εϲτιν ουκ οιδα εν} & 9 &  &  \\
&  & 10 & \foreignlanguage{greek}{οιδα οτι τυφλοϲ ων αρτι βλεπω} & 15 &  &  \\
& \textbf{26} &  & \foreignlanguage{greek}{ειπον ουν αυτω τι εποιηϲεν ϲοι πωϲ} & 7 &  &  \\
&  & 8 & \foreignlanguage{greek}{ηνεωξεν ϲου τουϲ οφθαλμουϲ} & 11 &  &  \\
& \textbf{27} &  & \foreignlanguage{greek}{απεκριθη αυτοιϲ ειπον υμιν ηδη και} & 6 &  &  \\
&  & 7 & \foreignlanguage{greek}{ουκ ηκουϲατε τι παλιν θελεται ακουει̅} & 12 &  &  \\
&  & 13 & \foreignlanguage{greek}{μη και υμειϲ θελεται αυτου μαθηται} & 18 &  &  \\
&  & 19 & \foreignlanguage{greek}{γενεϲθαι και ελοιδορηϲαν αυτον} & 3 & \textbf{28} &  \\
&  & 4 & \foreignlanguage{greek}{και ειπαν ϲυ μαθητηϲ ει εκεινου} & 9 &  &  \\
&  & 10 & \foreignlanguage{greek}{ημειϲ δε του μωυϲεωϲ εϲμεν μαθη} & 15 &  &  \\
&  & 15 & \foreignlanguage{greek}{ται ημειϲ οιδαμεν οτι μωυϲει λε} & 5 & \textbf{29} &  \\
&  & 5 & \foreignlanguage{greek}{λαληκεν ο \textoverline{θϲ} τουτον δε ουκ οιδαμε̅} & 11 &  &  \\
[0.2em]
\cline{4-4}
\end{tabular}
\end{center}
\end{table}
}
\clearpage
\newpage
 {
 \setlength\arrayrulewidth{1pt}
\begin{table}
\begin{center}
\begin{tabular}{ccc|l|ccc}
\cline{4-4} \\ [-1em]
\multicolumn{7}{c}{\foreignlanguage{greek}{ευαγγελιον κατα ιωαννην} \textbf{(\nospace{9:29})} } \\ \\ [-1em] % Si on veut ajouter les bordures latérales, remplacer {7}{c} par {7}{|c|}
\cline{4-4} \\
\cline{4-4}
&  &  & &  &  & \\ [-0.9em]
&  & 12 & \foreignlanguage{greek}{ποθεν εϲτιν απεκριθη ο \textoverline{ανοϲ} και ει} & 5 & \textbf{30} &  \\
&  & 5 & \foreignlanguage{greek}{πεν αυτοιϲ εν τουτω γαρ θαυμαϲτο̅} & 10 &  &  \\
&  & 11 & \foreignlanguage{greek}{εϲτιν οτι υμειϲ ουκ οιδατε ποθεν εϲτι̅} & 17 &  &  \\
&  & 18 & \foreignlanguage{greek}{και ηνεωξεν μου τουϲ οφθαλμουϲ} & 22 &  &  \\
& \textbf{31} &  & \foreignlanguage{greek}{οιδαμεν δε οτι αμαρτωλων ο \textoverline{θϲ} ου} & 7 &  &  \\
&  & 7 & \foreignlanguage{greek}{κ ακουει αλλα εαν τιϲ θεοϲεβηϲ η και} & 14 &  &  \\
&  & 15 & \foreignlanguage{greek}{το θελημα αυτου ποιη τουτου ακουει} & 20 &  &  \\
& \textbf{32} &  & \foreignlanguage{greek}{εκ του αιωνοϲ ουκ ηκουϲθη οτι ηνε} & 7 &  &  \\
&  & 7 & \foreignlanguage{greek}{ωξεν τιϲ οφθαλμουϲ τυφλου γεγε} & 11 &  &  \\
&  & 11 & \foreignlanguage{greek}{ννημενου ει μη ην ουτοϲ παρα \textoverline{θυ} ου} & 7 & \textbf{33} &  \\
&  & 7 & \foreignlanguage{greek}{κ ηδυνατο ποιειν ουδεν} & 10 &  &  \\
& \textbf{34} &  & \foreignlanguage{greek}{απεκριθηϲαν και ειπαν αυτω εν α} & 6 &  &  \\
&  & 6 & \foreignlanguage{greek}{μαρτιαιϲ ϲυ εγεννηθηϲ ολοϲ και ϲυ} & 11 &  &  \\
&  & 12 & \foreignlanguage{greek}{διδαϲκειϲ ημαϲ και εξεβαλαν αυτο̅} & 16 &  &  \\
&  & 17 & \foreignlanguage{greek}{εξω ηκουϲεν δε ο \textoverline{ιϲ} οτι εξεβα} & 6 & \textbf{35} &  \\
&  & 6 & \foreignlanguage{greek}{λον αυτον και ευρων αυτον ειπεν} & 11 &  &  \\
&  & 12 & \foreignlanguage{greek}{ϲυ πιϲτευειϲ ειϲ τον υιον του \textoverline{ανου}} & 18 &  &  \\
& \textbf{36} &  & \foreignlanguage{greek}{και τιϲ εϲτιν εφη \textoverline{κε} ινα πιϲτευϲω} & 7 &  &  \\
&  & 8 & \foreignlanguage{greek}{ειϲ αυτον ειπεν αυτω ο \textoverline{ιϲ} και ε} & 6 & \textbf{37} &  \\
&  & 6 & \foreignlanguage{greek}{ορακαϲ αυτον και ο λαλων μετα ϲου} & 12 &  &  \\
&  & 13 & \foreignlanguage{greek}{εκεινοϲ εϲτιν} & 14 &  &  \\
& \textbf{39} &  & \foreignlanguage{greek}{τον κοϲμον τουτον ηλθον ινα οι μη} & 11 &  &  \\
&  & 12 & \foreignlanguage{greek}{βλεποντεϲ βλεπωϲιν και οι βλεπο̅} & 16 &  &  \\
&  & 16 & \foreignlanguage{greek}{τεϲ τυφλοι γενωνται} & 18 &  &  \\
& \textbf{40} &  & \foreignlanguage{greek}{ηκουϲαν εκ των φαριϲαιων ταυτα} & 5 &  &  \\
&  & 6 & \foreignlanguage{greek}{οι μετ αυτου οντεϲ και ειπαν αυτω} & 12 &  &  \\
&  & 13 & \foreignlanguage{greek}{μη και ημειϲ τυφλοι εϲμεν} & 17 &  &  \\
& \textbf{41} &  & \foreignlanguage{greek}{ειπεν αυτοιϲ ο \textoverline{ιϲ} ει τυφλοι ητε ουκ α̅} & 9 &  &  \\
&  & 10 & \foreignlanguage{greek}{ειχεται αμαρτιαν νυν δε λεγεται} & 14 &  &  \\
&  & 15 & \foreignlanguage{greek}{οτι βλεπομεν αι αμαρτιαι υμων} & 19 &  &  \\
[0.2em]
\cline{4-4}
\end{tabular}
\end{center}
\end{table}
}
\clearpage
\newpage
 {
 \setlength\arrayrulewidth{1pt}
\begin{table}
\begin{center}
\begin{tabular}{ccc|l|ccc}
\cline{4-4} \\ [-1em]
\multicolumn{7}{c}{\foreignlanguage{greek}{ευαγγελιον κατα ιωαννην} \textbf{(\nospace{9:41})} } \\ \\ [-1em] % Si on veut ajouter les bordures latérales, remplacer {7}{c} par {7}{|c|}
\cline{4-4} \\
\cline{4-4}
&  &  & &  &  & \\ [-0.9em]
&  & 20 & \foreignlanguage{greek}{μενουϲιν αμην αμην λεγω υμιν} & 4 & \mygospelchapter &  \\
&  & 5 & \foreignlanguage{greek}{ο μη ειϲερχομενοϲ δια τηϲ θυραϲ ειϲ τη̅} & 12 &  &  \\
&  & 13 & \foreignlanguage{greek}{αυλην των προβατων αλλα αναβαι} & 17 &  &  \\
&  & 17 & \foreignlanguage{greek}{νων αλλαχοθεν εκεινοϲ κλεπτηϲ εϲτι̅} & 21 &  &  \\
&  & 22 & \foreignlanguage{greek}{και ληϲτηϲ ο δε ειϲερχομενοϲ δια} & 4 & \textbf{2} &  \\
&  & 5 & \foreignlanguage{greek}{τηϲ θυραϲ εκεινοϲ εϲτιν ο ποιμην τω̅} & 11 &  &  \\
&  & 12 & \foreignlanguage{greek}{προβατων τουτω ο θυρωροϲ ανοιγει} & 4 & \textbf{3} &  \\
&  & 5 & \foreignlanguage{greek}{και τα προβατα τηϲ φωνηϲ αυτου ακουει} & 11 &  &  \\
&  & 12 & \foreignlanguage{greek}{και τα ιδια προβατα φωνει κατ ονομα} & 18 &  &  \\
&  & 19 & \foreignlanguage{greek}{και εξαγει αυτα οταν τα ιδια παντα} & 4 & \textbf{4} &  \\
&  & 5 & \foreignlanguage{greek}{εκβαλη εμπροϲθεν αυτων πορευεται} & 8 &  &  \\
&  & 9 & \foreignlanguage{greek}{και τα προβατα αυτω ακολουθει οτι} & 14 &  &  \\
&  & 15 & \foreignlanguage{greek}{οιδαϲιν την φωνην αυτου αλλοτρι} & 1 & \textbf{5} &  \\
&  & 1 & \foreignlanguage{greek}{ω δε ου μη ακολουθηϲωϲιν αλλα φευ} & 7 &  &  \\
&  & 7 & \foreignlanguage{greek}{ξονται απ αυτου οτι ουκ οιδαϲιν τω̅} & 13 &  &  \\
&  & 14 & \foreignlanguage{greek}{αλλοτριων την φωνην} & 16 &  &  \\
& \textbf{6} &  & \foreignlanguage{greek}{ταυτην την παροιμιαν ειπεν αυτοιϲ} & 5 &  &  \\
&  & 6 & \foreignlanguage{greek}{ο \textoverline{ιϲ} εκεινοι δε ουκ εγνωϲαν τινα ην} & 13 &  &  \\
&  & 14 & \foreignlanguage{greek}{α ελαλει αυτοιϲ} & 16 &  &  \\
& \textbf{7} &  & \foreignlanguage{greek}{ειπεν ουν αυτοιϲ ο \textoverline{ιϲ} αμην αμην λε} & 8 &  &  \\
&  & 8 & \foreignlanguage{greek}{γω υμιν οτι εγω ειμει η θυρα των} & 15 &  &  \\
&  & 16 & \foreignlanguage{greek}{προβατων παντεϲ οϲοι ηλθον προ} & 4 & \textbf{8} &  \\
&  & 5 & \foreignlanguage{greek}{εμου κλεπται ειϲιν και ληϲται αλλ ου} & 11 &  &  \\
&  & 11 & \foreignlanguage{greek}{κ ηκουϲαν αυτων τα προβατα} & 15 &  &  \\
& \textbf{9} &  & \foreignlanguage{greek}{εγω ειμει η θυρα δι εμου αν τιϲ ειϲελθη} & 9 &  &  \\
&  & 10 & \foreignlanguage{greek}{ϲωθηϲεται και εξελευϲεται και νομη̅} & 14 &  &  \\
&  & 15 & \foreignlanguage{greek}{ευρηϲει ο κλεπτηϲ ουκ ερχεται} & 4 & \textbf{10} &  \\
&  & 5 & \foreignlanguage{greek}{ει μη ινα κλεψη και θυϲη και απολεϲη} & 12 &  &  \\
&  & 13 & \foreignlanguage{greek}{εγω ηλθον ινα ζωην εχωϲιν και περι} & 19 &  &  \\
&  & 19 & \foreignlanguage{greek}{ϲον εχωϲιν εγω ειμει ο ποιμην ο καλοϲ} & 6 & \textbf{11} &  \\
[0.2em]
\cline{4-4}
\end{tabular}
\end{center}
\end{table}
}
\clearpage
\newpage
 {
 \setlength\arrayrulewidth{1pt}
\begin{table}
\begin{center}
\begin{tabular}{ccc|l|ccc}
\cline{4-4} \\ [-1em]
\multicolumn{7}{c}{\foreignlanguage{greek}{ευαγγελιον κατα ιωαννην} \textbf{(\nospace{10:11})} } \\ \\ [-1em] % Si on veut ajouter les bordures latérales, remplacer {7}{c} par {7}{|c|}
\cline{4-4} \\
\cline{4-4}
&  &  & &  &  & \\ [-0.9em]
&  & 7 & \foreignlanguage{greek}{ο ποιμην ο καλοϲ την ψυχην αυτου τι} & 14 &  &  \\
&  & 14 & \foreignlanguage{greek}{θηϲιν υπερ των προβατων ο μιϲθω} & 2 & \textbf{12} &  \\
&  & 2 & \foreignlanguage{greek}{τοϲ και ουκ ων ποιμην ου ουκ εϲτιν} & 9 &  &  \\
&  & 10 & \foreignlanguage{greek}{τα προβατα ιδια θεωρει τον λυκον} & 15 &  &  \\
&  & 16 & \foreignlanguage{greek}{ερχομενον και αφιηϲιν τα προβατα} & 20 &  &  \\
&  & 21 & \foreignlanguage{greek}{και φευγει και ο λυκοϲ αρπαζει αυ} & 27 &  &  \\
&  & 27 & \foreignlanguage{greek}{τα και ϲκορπιζει και ου μελει αυτω} & 4 & \textbf{13} &  \\
&  & 5 & \foreignlanguage{greek}{περι των προβατων} & 7 &  &  \\
& \textbf{14} &  & \foreignlanguage{greek}{εγω ειμει ο ποιμην ο καλοϲ και γινω} & 8 &  &  \\
&  & 8 & \foreignlanguage{greek}{ϲκω τα εμα και γινωϲκουϲιν με τα} & 14 &  &  \\
&  & 15 & \foreignlanguage{greek}{εμα καθωϲ γεινωϲκει με ο \textoverline{πηρ}} & 5 & \textbf{15} &  \\
&  & 6 & \foreignlanguage{greek}{καγω γινωϲκω τον \textoverline{πρα} και την} & 11 &  &  \\
&  & 12 & \foreignlanguage{greek}{ψυχην μου διδωμι υπερ των προβα} & 17 &  &  \\
&  & 17 & \foreignlanguage{greek}{των και αλλα προβατα εχω α ου} & 6 & \textbf{16} &  \\
&  & 6 & \foreignlanguage{greek}{κ εϲτιν εκ τηϲ αυληϲ ταυτηϲ κακεινα} & 12 &  &  \\
&  & 13 & \foreignlanguage{greek}{δει με αγαγειν και τηϲ φωνηϲ μου} & 19 &  &  \\
&  & 20 & \foreignlanguage{greek}{ακουϲωϲιν και γενηϲονται μια} & 23 &  &  \\
&  & 24 & \foreignlanguage{greek}{ποιμνη ειϲ ποιμην} & 26 &  &  \\
& \textbf{17} &  & \foreignlanguage{greek}{δια τουτο ο \textoverline{πηρ} με αγαπα οτι εγω τι} & 9 &  &  \\
&  & 9 & \foreignlanguage{greek}{θημει την ψυχην μου ινα παλιν} & 14 &  &  \\
&  & 15 & \foreignlanguage{greek}{λαβω αυτην ουδειϲ ερει αυτην} & 3 & \textbf{18} &  \\
&  & 4 & \foreignlanguage{greek}{απ εμου αλλ εγω τιθημει αυτην} & 9 &  &  \\
&  & 10 & \foreignlanguage{greek}{απ εμαυτου και εξουϲιαν εχω θει} & 15 &  &  \\
&  & 15 & \foreignlanguage{greek}{ναι αυτην και εξουϲιαν εχω παλι̅} & 20 &  &  \\
&  & 21 & \foreignlanguage{greek}{λαβειν αυτην ταυτην την εν} & 25 &  &  \\
&  & 25 & \foreignlanguage{greek}{τολην ελαβον παρα του \textoverline{πρϲ} μου} & 31 &  &  \\
& \textbf{19} &  & \foreignlanguage{greek}{ϲχιϲμα παλιν εγενετο εν τοιϲ ιου} & 6 &  &  \\
&  & 6 & \foreignlanguage{greek}{δαιοιϲ δια τουϲ λογουϲ τουτουϲ} & 10 &  &  \\
& \textbf{20} &  & \foreignlanguage{greek}{ελεγον δε πολλοι εξ αυτων δαιμο} & 6 &  &  \\
&  & 6 & \foreignlanguage{greek}{νιον εχει και μαινεται τι αυτου} & 11 &  &  \\
[0.2em]
\cline{4-4}
\end{tabular}
\end{center}
\end{table}
}
\clearpage
\newpage
 {
 \setlength\arrayrulewidth{1pt}
\begin{table}
\begin{center}
\begin{tabular}{ccc|l|ccc}
\cline{4-4} \\ [-1em]
\multicolumn{7}{c}{\foreignlanguage{greek}{ευαγγελιον κατα ιωαννην} \textbf{(\nospace{10:20})} } \\ \\ [-1em] % Si on veut ajouter les bordures latérales, remplacer {7}{c} par {7}{|c|}
\cline{4-4} \\
\cline{4-4}
&  &  & &  &  & \\ [-0.9em]
&  & 12 & \foreignlanguage{greek}{ακουεται ελεγον δε αλλοι ταυτα τα} & 5 & \textbf{21} &  \\
&  & 6 & \foreignlanguage{greek}{ρηματα ουκ εϲτιν δαιμονιζομενου} & 9 &  &  \\
&  & 10 & \foreignlanguage{greek}{μη δαιμονιον δυναται τυφλων οφθαλ} & 14 &  &  \\
&  & 14 & \foreignlanguage{greek}{μουϲ ανοιξαι} & 15 &  &  \\
& \textbf{22} &  & \foreignlanguage{greek}{εγενετο τοτε τα ενκενια εν τοιϲ ιεροϲολυ} & 7 &  &  \\
&  & 7 & \foreignlanguage{greek}{μοιϲ χειμων ην και περιεπατει ο \textoverline{ιϲ} εν} & 5 & \textbf{23} &  \\
&  & 6 & \foreignlanguage{greek}{τω ιερω εν τη ϲτοα του ϲαλομωντοϲ} & 12 &  &  \\
& \textbf{24} &  & \foreignlanguage{greek}{εκυκλωϲαν ουν αυτον οι ιουδαιοι και} & 6 &  &  \\
&  & 7 & \foreignlanguage{greek}{ελεγον αυτω εωϲ ποτε την ψυχην η} & 13 &  &  \\
&  & 13 & \foreignlanguage{greek}{μων ερειϲ ει ϲυ ει ο \textoverline{χϲ} ειπε ημιν παρρη} & 22 &  &  \\
&  & 22 & \foreignlanguage{greek}{ϲια απεκριθη αυτοιϲ ο \textoverline{ιϲ} ειπον υ} & 6 & \textbf{25} &  \\
&  & 6 & \foreignlanguage{greek}{μιν και ου πιϲτευεται τα εργα α εγω} & 13 &  &  \\
&  & 14 & \foreignlanguage{greek}{ποιω εν ονοματι του \textoverline{πρϲ} μου αυτα} & 20 &  &  \\
&  & 21 & \foreignlanguage{greek}{ταυτα τα εργα μαρτυρηϲει περι εμου} & 26 &  &  \\
& \textbf{26} &  & \foreignlanguage{greek}{αλλα υμειϲ ου πιϲτευεται οτι ουκ γαρ εϲται} & 8 &  &  \\
&  & 9 & \foreignlanguage{greek}{εκ των προβατων των εμων} & 13 &  &  \\
& \textbf{27} &  & \foreignlanguage{greek}{τα προβατα τα εμα τηϲ φωνηϲ μου α} & 8 &  &  \\
&  & 8 & \foreignlanguage{greek}{κουουϲιν καγω γινωϲκω αυτα και α} & 13 &  &  \\
&  & 13 & \foreignlanguage{greek}{κολουθουϲιν μοι καγω διδωμει αυτοιϲ} & 3 & \textbf{28} &  \\
&  & 4 & \foreignlanguage{greek}{ζωην αιωνιον και ου μη απολωνται} & 9 &  &  \\
&  & 10 & \foreignlanguage{greek}{ειϲ τον αιωνα και ουχ αρπαϲει τιϲ αυ} & 17 &  &  \\
&  & 17 & \foreignlanguage{greek}{τα εκ τηϲ χειροϲ μου} & 21 &  &  \\
& \textbf{29} &  & \foreignlanguage{greek}{ο \textoverline{πηρ} μου ο δεδωκεν μοι παντων μει} & 8 &  &  \\
&  & 8 & \foreignlanguage{greek}{ζων εϲτιν και ουδειϲ δυναται αρπα} & 13 &  &  \\
&  & 13 & \foreignlanguage{greek}{ζειν εκ τηϲ χειροϲ του \textoverline{πρϲ} μου} & 19 &  &  \\
& \textbf{30} &  & \foreignlanguage{greek}{εγω και ο \textoverline{πηρ} εν εϲμεν εβαϲτα} & 1 &  &  \\
&  & 1 & \foreignlanguage{greek}{ϲαν παλιν λιθουϲ ινα λιθαϲωϲιν αυτο̅} & 6 &  &  \\
&  & 7 & \foreignlanguage{greek}{απεκριθη αυτοιϲ ο \textoverline{ιϲ} πολλα εργα εδειξα} & 6 & \textbf{32} &  \\
&  & 7 & \foreignlanguage{greek}{υμιν εκ του \textoverline{πρϲ} μου δια ποιον ουν ερ} & 15 &  &  \\
&  & 15 & \foreignlanguage{greek}{γον λιθαζεται με} & 17 &  &  \\
[0.2em]
\cline{4-4}
\end{tabular}
\end{center}
\end{table}
}
\clearpage
\newpage
 {
 \setlength\arrayrulewidth{1pt}
\begin{table}
\begin{center}
\begin{tabular}{ccc|l|ccc}
\cline{4-4} \\ [-1em]
\multicolumn{7}{c}{\foreignlanguage{greek}{ευαγγελιον κατα ιωαννην} \textbf{(\nospace{10:33})} } \\ \\ [-1em] % Si on veut ajouter les bordures latérales, remplacer {7}{c} par {7}{|c|}
\cline{4-4} \\
\cline{4-4}
&  &  & &  &  & \\ [-0.9em]
& \textbf{33} &  & \foreignlanguage{greek}{απεκριθηϲαν αυτω οι ιουδαιοι περι κα} & 6 &  &  \\
&  & 6 & \foreignlanguage{greek}{λου εργου ου λιθαζομεν ϲε αλλα περι βλα} & 13 &  &  \\
&  & 13 & \foreignlanguage{greek}{ϲφημιαϲ και οτι ϲυ \textoverline{ανοϲ} ων ποιειϲ ϲεαυ} & 20 &  &  \\
&  & 20 & \foreignlanguage{greek}{τον \textoverline{θν} απεκριθη αυτοιϲ \textoverline{ιϲ} ουκ ε} & 5 & \textbf{34} &  \\
&  & 5 & \foreignlanguage{greek}{ϲτιν γεγραμμενον εν τω νομω υμω̅} & 10 &  &  \\
&  & 11 & \foreignlanguage{greek}{οτι εγω ειπα θεοι εϲται ει εκεινουϲ ει} & 3 & \textbf{35} &  \\
&  & 3 & \foreignlanguage{greek}{πεν θεουϲ προϲ ουϲ ο λογοϲ του \textoverline{θυ} εγε} & 11 &  &  \\
&  & 11 & \foreignlanguage{greek}{νετο και ου δυναται λυθηναι η γραφη} & 17 &  &  \\
& \textbf{36} &  & \foreignlanguage{greek}{ον ο \textoverline{πηρ} ηγιαϲεν και απεϲτιλεν ειϲ} & 7 &  &  \\
&  & 8 & \foreignlanguage{greek}{τον κοϲμον υμειϲ λεγεται οτι βλα} & 13 &  &  \\
&  & 13 & \foreignlanguage{greek}{ϲφημειϲ οτι ειπον υιοϲ \textoverline{θυ} ειμει} & 18 &  &  \\
& \textbf{37} &  & \foreignlanguage{greek}{ει ου ποιω τα εργα του \textoverline{πρϲ} μου μη πιϲτευ} & 10 &  &  \\
&  & 10 & \foreignlanguage{greek}{εται μοι ει δε ποιω καν εμοι μη πιϲ} & 7 & \textbf{38} &  \\
&  & 7 & \foreignlanguage{greek}{τευεται τοιϲ εργοιϲ πιϲτευεται} & 10 &  &  \\
&  & 11 & \foreignlanguage{greek}{αναγνωτε και γινωϲκεται οτι εν εμοι} & 16 &  &  \\
&  & 17 & \foreignlanguage{greek}{ο \textoverline{πηρ} καγω εν τω \textoverline{πρι} εζητουν ουν αυ} & 3 & \textbf{39} &  \\
&  & 3 & \foreignlanguage{greek}{τον παλιν πιαϲαι και εξηλθεν εκ τηϲ} & 9 &  &  \\
&  & 10 & \foreignlanguage{greek}{χειροϲ αυτων και απηλθεν παλιν} & 3 & \textbf{40} &  \\
&  & 4 & \foreignlanguage{greek}{περαν του ιορδανου ειϲ τον τοπον ο} & 10 &  &  \\
&  & 10 & \foreignlanguage{greek}{που ην ιωαννηϲ το πρωτον βαπτιζω̅} & 15 &  &  \\
&  & 16 & \foreignlanguage{greek}{και εμειν εκει και πολλοι ηλθον προϲ} & 4 & \textbf{41} &  \\
&  & 5 & \foreignlanguage{greek}{αυτον και ελεγον οτι ιωαννηϲ μεν} & 10 &  &  \\
&  & 11 & \foreignlanguage{greek}{εποιηϲεν ϲημιον ουδε εν παντα δε} & 16 &  &  \\
&  & 17 & \foreignlanguage{greek}{οϲα ειπεν περι τουτου αληθη ην πολ} & 1 & \textbf{42} &  \\
&  & 1 & \foreignlanguage{greek}{λοι ουν επιϲτευϲαν ειϲ αυτον εκει} & 6 &  &  \\
& \mygospelchapter &  & \foreignlanguage{greek}{ην δε τιϲ αϲθενων λαζαροϲ απο βηθανιαϲ} & 7 &  &  \\
&  & 8 & \foreignlanguage{greek}{εκ τηϲ κωμηϲ μαριαϲ και μαρθαϲ τηϲ} & 14 &  &  \\
&  & 15 & \foreignlanguage{greek}{αδελφηϲ αυτηϲ ην δε μαρια η αλι} & 5 & \textbf{2} &  \\
&  & 5 & \foreignlanguage{greek}{ψαϲα τον \textoverline{κν} μυρω και εκμαξαϲα τουϲ} & 11 &  &  \\
&  & 12 & \foreignlanguage{greek}{ποδαϲ αυτου ταιϲ θριξιν αυτηϲ ηϲ ο} & 18 &  &  \\
[0.2em]
\cline{4-4}
\end{tabular}
\end{center}
\end{table}
}
\clearpage
\newpage
 {
 \setlength\arrayrulewidth{1pt}
\begin{table}
\begin{center}
\begin{tabular}{ccc|l|ccc}
\cline{4-4} \\ [-1em]
\multicolumn{7}{c}{\foreignlanguage{greek}{ευαγγελιον κατα ιωαννην} \textbf{(\nospace{11:2})} } \\ \\ [-1em] % Si on veut ajouter les bordures latérales, remplacer {7}{c} par {7}{|c|}
\cline{4-4} \\
\cline{4-4}
&  &  & &  &  & \\ [-0.9em]
&  & 19 & \foreignlanguage{greek}{αδελφοϲ λαζαροϲ ηϲθενει} & 21 &  &  \\
& \textbf{3} &  & \foreignlanguage{greek}{απεϲτιλαν ουν αι αδελφαι προϲ αυτο̅} & 6 &  &  \\
&  & 7 & \foreignlanguage{greek}{λεγουϲαι \textoverline{κε} ειδε ον φιλειϲ αϲθενει} & 12 &  &  \\
& \textbf{4} &  & \foreignlanguage{greek}{ακουϲαϲ δε ο \textoverline{ιϲ} ειπεν αυτη η αϲθενια} & 8 &  &  \\
&  & 9 & \foreignlanguage{greek}{ουκ εϲτιν προϲ θανατον αλλα υπερ} & 14 &  &  \\
&  & 15 & \foreignlanguage{greek}{τηϲ δοξηϲ του \textoverline{θυ} ινα δοξαϲθη ο υιοϲ} & 22 &  &  \\
&  & 23 & \foreignlanguage{greek}{του \textoverline{θυ} δι αυτηϲ ηγαπα δε ο \textoverline{ιϲ} την} & 5 & \textbf{5} &  \\
&  & 6 & \foreignlanguage{greek}{μαρθαν και την αδελφην αυτηϲ και} & 11 &  &  \\
&  & 12 & \foreignlanguage{greek}{τον λαζαρον ωϲ ουν ηκουϲεν οτι} & 4 & \textbf{6} &  \\
&  & 5 & \foreignlanguage{greek}{αϲθενει τοτε μεν εμεινεν εν ω ην} & 11 &  &  \\
&  & 12 & \foreignlanguage{greek}{τοπω δυο ημεραϲ επειτα μετα του} & 3 & \textbf{7} &  \\
&  & 3 & \foreignlanguage{greek}{το λεγει τοιϲ μαθηταιϲ γωμεν ειϲ τη} & 9 &  &  \\
&  & 10 & \foreignlanguage{greek}{ιουδαιαν παλιν λεγουϲιν αυτω} & 2 & \textbf{8} &  \\
&  & 3 & \foreignlanguage{greek}{οι μαθηται ραββει νυν εζητουν ϲε} & 8 &  &  \\
&  & 9 & \foreignlanguage{greek}{λιθαϲαι οι ιουδαιοι και παλιν υπαγειϲ} & 14 &  &  \\
&  & 15 & \foreignlanguage{greek}{εκει απεκριθη \textoverline{ιϲ} ουχι δωδεκα ω} & 5 & \textbf{9} &  \\
&  & 5 & \foreignlanguage{greek}{ραι ειϲιν τηϲ ημεραϲ εαν τιϲ περιπα} & 11 &  &  \\
&  & 11 & \foreignlanguage{greek}{τη εν τη ημερα ου προϲκοπτει οτι} & 17 &  &  \\
&  & 18 & \foreignlanguage{greek}{το φωϲ του κοϲμου τουτου βλεπει} & 23 &  &  \\
& \textbf{10} &  & \foreignlanguage{greek}{εαν δε τιϲ περιπατη εν τη νυκτι} & 7 &  &  \\
&  & 8 & \foreignlanguage{greek}{προϲκοπτι οτι φωϲ ουκ εϲτιν εν αυτω} & 14 &  &  \\
& \textbf{11} &  & \foreignlanguage{greek}{ταυτα ειπεν και μετα τουτο λεγει αυ} & 7 &  &  \\
&  & 7 & \foreignlanguage{greek}{τοιϲ λαζαροϲ ο φιλοϲ ημων κεκοι} & 12 &  &  \\
&  & 12 & \foreignlanguage{greek}{μηται αλλα πορευομαι ινα εξυπνι} & 16 &  &  \\
&  & 16 & \foreignlanguage{greek}{ϲω αυτον ειπον ουν αυτω οι μα} & 5 & \textbf{12} &  \\
&  & 5 & \foreignlanguage{greek}{θηται \textoverline{κε} ει κεκοιμηται ϲωθηϲεται} & 9 &  &  \\
& \textbf{13} &  & \foreignlanguage{greek}{ειρηκει δε ο \textoverline{ιϲ} περι του θανατου αυτου} & 8 &  &  \\
&  & 9 & \foreignlanguage{greek}{εκεινοι δε εδοξαν οτι περι τηϲ κοι} & 15 &  &  \\
&  & 15 & \foreignlanguage{greek}{μηϲεωϲ του υπνου λεγει} & 18 &  &  \\
& \textbf{14} &  & \foreignlanguage{greek}{τοτε λεγει αυτοιϲ ο \textoverline{ιϲ} παρηϲια λαζαροϲ} & 7 &  &  \\
[0.2em]
\cline{4-4}
\end{tabular}
\end{center}
\end{table}
}
\clearpage
\newpage
 {
 \setlength\arrayrulewidth{1pt}
\begin{table}
\begin{center}
\begin{tabular}{ccc|l|ccc}
\cline{4-4} \\ [-1em]
\multicolumn{7}{c}{\foreignlanguage{greek}{ευαγγελιον κατα ιωαννην} \textbf{(\nospace{11:14})} } \\ \\ [-1em] % Si on veut ajouter les bordures latérales, remplacer {7}{c} par {7}{|c|}
\cline{4-4} \\
\cline{4-4}
&  &  & &  &  & \\ [-0.9em]
&  & 8 & \foreignlanguage{greek}{απεθανεν και χαιρω δι υμαϲ ινα πιϲτευ} & 6 & \textbf{15} &  \\
&  & 6 & \foreignlanguage{greek}{ϲηται οτι ουκ ημην εκει αλλα αγωμε̅} & 12 &  &  \\
&  & 13 & \foreignlanguage{greek}{προϲ αυτον ειπεν ουν θωμαϲ ο λε} & 5 & \textbf{16} &  \\
&  & 5 & \foreignlanguage{greek}{γομενοϲ διδυμοϲ τοιϲ ϲυνμαθηταιϲ} & 8 &  &  \\
&  & 9 & \foreignlanguage{greek}{αγωμεν και ημειϲ ινα αποθανωμε̅} & 13 &  &  \\
&  & 14 & \foreignlanguage{greek}{μετ αυτου ελθων ουν ο \textoverline{ιϲ} ευρεν} & 5 & \textbf{17} &  \\
&  & 6 & \foreignlanguage{greek}{αυτον τεϲϲαραϲ ημεραϲ ηδη εν τω} & 11 &  &  \\
&  & 12 & \foreignlanguage{greek}{μνημιω εχοντα ην δε η βηθανια} & 4 & \textbf{18} &  \\
&  & 5 & \foreignlanguage{greek}{εγγυϲ των ιεροϲολυμων ωϲ απο ϲταδιω̅} & 10 &  &  \\
&  & 11 & \foreignlanguage{greek}{δεκαπεντε πολλοι δε εκ των} & 4 & \textbf{19} &  \\
&  & 5 & \foreignlanguage{greek}{ιουδαιων εληλυθειϲαν προϲ την μαρ} & 9 &  &  \\
&  & 9 & \foreignlanguage{greek}{θαν και μαριαν ινα παραμυθη} & 13 &  &  \\
&  & 13 & \foreignlanguage{greek}{ϲωνται αυταϲ περι του αδελφου} & 17 &  &  \\
& \textbf{20} &  & \foreignlanguage{greek}{η ουν μαρθα ωϲ ηκουϲεν οτι \textoverline{ιϲ} ερχε} & 8 &  &  \\
&  & 8 & \foreignlanguage{greek}{ται υπηντηϲεν αυτω μαρια δε ε̅} & 13 &  &  \\
&  & 14 & \foreignlanguage{greek}{τω οικω εκαθητο ειπεν ουν η μαρ} & 4 & \textbf{21} &  \\
&  & 4 & \foreignlanguage{greek}{θα προϲ τον \textoverline{ιν} \textoverline{κε} ει ηϲ ωδε ουκ αν α} & 14 &  &  \\
&  & 14 & \foreignlanguage{greek}{πεθανεν ο αδελφοϲ μου αλλα και} & 2 & \textbf{22} &  \\
&  & 3 & \foreignlanguage{greek}{νυν οιδα οτι οϲα εαν αιτηϲηϲ τον \textoverline{θν}} & 10 &  &  \\
&  & 11 & \foreignlanguage{greek}{δωϲει ϲοι ο \textoverline{θϲ}} & 14 &  &  \\
& \textbf{23} &  & \foreignlanguage{greek}{λεγει αυτη ο \textoverline{ιϲ} αναϲτηϲεται ο αδελ} & 7 &  &  \\
&  & 7 & \foreignlanguage{greek}{φοϲ ϲου λεγει αυτω μαρθα οιδα} & 4 & \textbf{24} &  \\
&  & 5 & \foreignlanguage{greek}{οτι αναϲτηϲεται εν τη αναϲταϲει} & 9 &  &  \\
&  & 10 & \foreignlanguage{greek}{εν τη εϲχατη ημερα} & 13 &  &  \\
& \textbf{25} &  & \foreignlanguage{greek}{ειπεν αυτη ο \textoverline{ιϲ} εγω ειμει η αναϲτα} & 8 &  &  \\
&  & 8 & \foreignlanguage{greek}{ϲιϲ και η ζωη ο πιϲτευων ειϲ εμε} & 15 &  &  \\
&  & 16 & \foreignlanguage{greek}{καν αποθανη ζηϲεται και παϲ ο} & 3 & \textbf{26} &  \\
&  & 4 & \foreignlanguage{greek}{ζων και πιϲτευων ου μη αποθανη} & 9 &  &  \\
&  & 10 & \foreignlanguage{greek}{ειϲ τον αιωνα πιϲτευειϲ τουτο} & 14 &  &  \\
& \textbf{27} &  & \foreignlanguage{greek}{λεγει αυτω ναι \textoverline{κε} εγω πεπιϲτευκα} & 6 &  &  \\
[0.2em]
\cline{4-4}
\end{tabular}
\end{center}
\end{table}
}
\clearpage
\newpage
 {
 \setlength\arrayrulewidth{1pt}
\begin{table}
\begin{center}
\begin{tabular}{ccc|l|ccc}
\cline{4-4} \\ [-1em]
\multicolumn{7}{c}{\foreignlanguage{greek}{ευαγγελιον κατα ιωαννην} \textbf{(\nospace{11:27})} } \\ \\ [-1em] % Si on veut ajouter les bordures latérales, remplacer {7}{c} par {7}{|c|}
\cline{4-4} \\
\cline{4-4}
&  &  & &  &  & \\ [-0.9em]
&  & 7 & \foreignlanguage{greek}{οτι ϲυ ει ο \textoverline{χϲ} ο υιοϲ του \textoverline{θυ} ο ειϲ τον κοϲμον} & 19 &  &  \\
&  & 20 & \foreignlanguage{greek}{ερχομενοϲ και τουτο ειπουϲα απηλθε̅} & 4 & \textbf{28} &  \\
&  & 5 & \foreignlanguage{greek}{και εφωνηϲεν μαριαν την αδελφην} & 9 &  &  \\
&  & 10 & \foreignlanguage{greek}{αυτηϲ λαθρα ειπουϲα οτι ο διδαϲκαλοϲ} & 15 &  &  \\
&  & 16 & \foreignlanguage{greek}{παρεϲτιν και φωνει ϲε εκεινη δε} & 2 & \textbf{29} &  \\
&  & 3 & \foreignlanguage{greek}{ωϲ ηκουϲεν ηγερθη ταχυ και ηρχετο} & 8 &  &  \\
&  & 9 & \foreignlanguage{greek}{προϲ αυτον ουπω δε εληλυθει ο \textoverline{ιϲ} ειϲ} & 6 & \textbf{30} &  \\
&  & 7 & \foreignlanguage{greek}{την κωμην αλλ ην ετι εν τω τοπω ο} & 15 &  &  \\
&  & 15 & \foreignlanguage{greek}{που υπηντηϲεν αυτω μαρθα} & 18 &  &  \\
& \textbf{31} &  & \foreignlanguage{greek}{οι ουν ιουδαιοι οι οντεϲ μετ αυτηϲ εν} & 8 &  &  \\
&  & 9 & \foreignlanguage{greek}{τη οικεια και παραμυθουμενοι αυτη̅} & 13 &  &  \\
&  & 14 & \foreignlanguage{greek}{ιδοντεϲ την μαριαν οτι ταχεωϲ ανε} & 19 &  &  \\
&  & 19 & \foreignlanguage{greek}{ϲτη και εξηλθεν ηκολουθηϲαν αυτη} & 23 &  &  \\
&  & 24 & \foreignlanguage{greek}{δοξαντεϲ οτι υπαγει ειϲ το μνημιον} & 29 &  &  \\
&  & 30 & \foreignlanguage{greek}{ινα κλαυϲη εκει η ουν μαρια ωϲ ηλ} & 5 & \textbf{32} &  \\
&  & 5 & \foreignlanguage{greek}{θεν οπου ο \textoverline{ιϲ} ιδουϲα δε αυτον επεϲεν} & 12 &  &  \\
&  & 13 & \foreignlanguage{greek}{αυτου προϲ τουϲ ποδαϲ λεγουϲα αυτω} & 18 &  &  \\
&  & 19 & \foreignlanguage{greek}{\textoverline{κε} ει ηϲ ωδε ουκ αν μου απεθανεν ο α} & 28 &  &  \\
&  & 28 & \foreignlanguage{greek}{δελφοϲ} & 28 &  &  \\
& \textbf{33} &  & \foreignlanguage{greek}{\textoverline{ιϲ} ουν ωϲ ιδεν αυτην κλαιουϲαν και} & 7 &  &  \\
&  & 8 & \foreignlanguage{greek}{τουϲ ϲυνελθονταϲ αυτη ιουδαιουϲ κλαιον} & 12 &  &  \\
&  & 12 & \foreignlanguage{greek}{ταϲ ενεβριμηϲατο τω \textoverline{πνι} και ετα} & 17 &  &  \\
&  & 17 & \foreignlanguage{greek}{ραξεν εαυτον και ειπεν που τεθει} & 4 & \textbf{34} &  \\
&  & 4 & \foreignlanguage{greek}{κατε αυτον λεγουϲιν αυτω \textoverline{κε}} & 8 &  &  \\
&  & 9 & \foreignlanguage{greek}{ερχου και ειδε εδακρυϲεν ο \textoverline{ιϲ}} & 3 & \textbf{35} &  \\
& \textbf{36} &  & \foreignlanguage{greek}{ελεγον ουν οι ιουδαιοι ειδε πωϲ εφιλει} & 7 &  &  \\
&  & 8 & \foreignlanguage{greek}{αυτον τινεϲ δε εξ αυτων ειπον} & 5 & \textbf{37} &  \\
&  & 6 & \foreignlanguage{greek}{ουκ εδυνατο ουτοϲ ο ανοιξαϲ τουϲ ο} & 12 &  &  \\
&  & 12 & \foreignlanguage{greek}{φθαλμουϲ του τυφλου ποιηϲαι ινα} & 16 &  &  \\
&  & 17 & \foreignlanguage{greek}{και ουτοϲ μη αποθανη} & 20 &  &  \\
[0.2em]
\cline{4-4}
\end{tabular}
\end{center}
\end{table}
}
\clearpage
\newpage
 {
 \setlength\arrayrulewidth{1pt}
\begin{table}
\begin{center}
\begin{tabular}{ccc|l|ccc}
\cline{4-4} \\ [-1em]
\multicolumn{7}{c}{\foreignlanguage{greek}{ευαγγελιον κατα ιωαννην} \textbf{(\nospace{11:38})} } \\ \\ [-1em] % Si on veut ajouter les bordures latérales, remplacer {7}{c} par {7}{|c|}
\cline{4-4} \\
\cline{4-4}
&  &  & &  &  & \\ [-0.9em]
& \textbf{38} &  & \foreignlanguage{greek}{\textoverline{ιϲ} ουν παλιν ενβριμων εν εαυτω ερχε} & 7 &  &  \\
&  & 7 & \foreignlanguage{greek}{ται ειϲ το μνημιον ην δε ϲπηλεον} & 13 &  &  \\
&  & 14 & \foreignlanguage{greek}{και λιθοϲ επεκειτο επ αυτω} & 18 &  &  \\
& \textbf{39} &  & \foreignlanguage{greek}{λεγει ο \textoverline{ιϲ} αρατε τον λιθον λεγει αυτω} & 8 &  &  \\
&  & 9 & \foreignlanguage{greek}{η αδελφη του τετελευτηκοτοϲ μαρθα} & 13 &  &  \\
&  & 14 & \foreignlanguage{greek}{\textoverline{κε} ηδη οζει τεταρτεοϲ γαρ εϲτιν} & 19 &  &  \\
& \textbf{40} &  & \foreignlanguage{greek}{λεγει αυτη ο \textoverline{ιϲ} ουκ ειπον ϲοι οτι εαν πι} & 10 &  &  \\
&  & 10 & \foreignlanguage{greek}{ϲτευϲηϲ οψη την δοξαν του \textoverline{θυ}} & 15 &  &  \\
& \textbf{41} &  & \foreignlanguage{greek}{ηραν ουν τον λιθον ο δε \textoverline{ιϲ} ηρεν τουϲ} & 9 &  &  \\
&  & 10 & \foreignlanguage{greek}{οφθαλμουϲ ανω και ειπεν} & 13 &  &  \\
&  & 14 & \foreignlanguage{greek}{πατερ ευχαριϲτω ϲοι οτι ηκουϲαϲ μου} & 19 &  &  \\
& \textbf{42} &  & \foreignlanguage{greek}{εγω δε ηδιν οτι παντοτε μου ακουειϲ} & 7 &  &  \\
&  & 8 & \foreignlanguage{greek}{αλλα δια τον οχλον τον περιεϲτωτα} & 13 &  &  \\
&  & 14 & \foreignlanguage{greek}{ειπον ινα πιϲτευϲωϲιν οτι ϲυ με απε} & 20 &  &  \\
&  & 20 & \foreignlanguage{greek}{ϲτιλαϲ και ταυτα ειπων φωνη με} & 5 & \textbf{43} &  \\
&  & 5 & \foreignlanguage{greek}{γαλη εκραξεν λαζαρε δευρο εξω} & 9 &  &  \\
& \textbf{44} &  & \foreignlanguage{greek}{και εξηλθεν ο τεθνηκωϲ δεδεμενοϲ} & 5 &  &  \\
&  & 6 & \foreignlanguage{greek}{τουϲ ποδαϲ και ταϲ χειραϲ κιριαιϲ} & 11 &  &  \\
&  & 12 & \foreignlanguage{greek}{και η οψειϲ αυτου ϲουδαριω περιεδεδε} & 17 &  &  \\
&  & 17 & \foreignlanguage{greek}{το λεγει ο \textoverline{ιϲ} αυτοιϲ λυϲαται αυτο̅} & 23 &  &  \\
&  & 24 & \foreignlanguage{greek}{και αφεται υπαγειν} & 26 &  &  \\
& \textbf{45} &  & \foreignlanguage{greek}{πολλοι ουν εκ των ιουδαιων οι ελθο̅} & 7 &  &  \\
&  & 7 & \foreignlanguage{greek}{τεϲ προϲ την μαριαν και θεαϲαμε} & 12 &  &  \\
&  & 12 & \foreignlanguage{greek}{νοι α εποιηϲεν επιϲτευϲαν ειϲ αυτον} & 17 &  &  \\
& \textbf{46} &  & \foreignlanguage{greek}{τινεϲ δε εξ αυτων απηλθον προϲ τουϲ} & 7 &  &  \\
&  & 8 & \foreignlanguage{greek}{φαριϲαιουϲ και ειπον αυτοιϲ α εποιη} & 13 &  &  \\
&  & 13 & \foreignlanguage{greek}{ϲεν ο \textoverline{ιϲ} ϲυνηγαγον ουν οι αρχιερειϲ} & 4 & \textbf{47} &  \\
&  & 5 & \foreignlanguage{greek}{και οι φαριϲαιοι ϲυνεδριον και ελεγο̅} & 10 &  &  \\
&  & 11 & \foreignlanguage{greek}{τι ποιουμεν οτι ουτοϲ ο \textoverline{ανοϲ} πολλα ποι} & 18 &  &  \\
&  & 18 & \foreignlanguage{greek}{ει ϲημεια εαν αφωμεν αυτον ουτωϲ} & 4 & \textbf{48} &  \\
[0.2em]
\cline{4-4}
\end{tabular}
\end{center}
\end{table}
}
\clearpage
\newpage
 {
 \setlength\arrayrulewidth{1pt}
\begin{table}
\begin{center}
\begin{tabular}{ccc|l|ccc}
\cline{4-4} \\ [-1em]
\multicolumn{7}{c}{\foreignlanguage{greek}{ευαγγελιον κατα ιωαννην} \textbf{(\nospace{11:48})} } \\ \\ [-1em] % Si on veut ajouter les bordures latérales, remplacer {7}{c} par {7}{|c|}
\cline{4-4} \\
\cline{4-4}
&  &  & &  &  & \\ [-0.9em]
&  & 5 & \foreignlanguage{greek}{παντεϲ πιϲτευϲουϲιν ειϲ αυτον και ελευ} & 10 &  &  \\
&  & 10 & \foreignlanguage{greek}{ϲονται οι ρωμαιοι και αρουϲιν ημων και} & 16 &  &  \\
&  & 17 & \foreignlanguage{greek}{την πολιν και το εθνοϲ ειϲ δε τιϲ εξ αυ} & 5 & \textbf{49} &  \\
&  & 5 & \foreignlanguage{greek}{των καιαφαϲ αρχιερευϲ του ενιαυτου} & 9 &  &  \\
&  & 10 & \foreignlanguage{greek}{εκεινου ειπεν αυτοιϲ υμειϲ ουκ οιδατε} & 15 &  &  \\
&  & 16 & \foreignlanguage{greek}{ουδεν ουδε λογιζεϲθαι οτι ϲυμφερει ημι̅} & 5 & \textbf{50} &  \\
&  & 6 & \foreignlanguage{greek}{ινα ειϲ \textoverline{ανοϲ} αποθανη υπερ του λαου και} & 13 &  &  \\
&  & 14 & \foreignlanguage{greek}{μη ολον το εθνοϲ αποληται τουτο δε} & 2 & \textbf{51} &  \\
&  & 3 & \foreignlanguage{greek}{αφ εαυτου ουκ ειπεν αλλα αρχων ων} & 9 &  &  \\
&  & 10 & \foreignlanguage{greek}{του ενιαυτου εκεινου προεφητευϲεν} & 13 &  &  \\
&  & 14 & \foreignlanguage{greek}{οτι ημελλεν αποθνηϲκειν \textoverline{ιϲ} υπερ του} & 19 &  &  \\
&  & 20 & \foreignlanguage{greek}{εθνουϲ και ουχ υπερ του εθνουϲ μο} & 6 & \textbf{52} &  \\
&  & 6 & \foreignlanguage{greek}{νον αλλ ινα και τα τεκνα του \textoverline{θυ} τα δι} & 15 &  &  \\
&  & 15 & \foreignlanguage{greek}{εϲκορπιϲμενα ϲυναγαγη ειϲ εν} & 18 &  &  \\
& \textbf{53} &  & \foreignlanguage{greek}{απ εκεινηϲ ουν τηϲ ημεραϲ εβουλευϲα̅} & 6 &  &  \\
&  & 6 & \foreignlanguage{greek}{το ινα αποκτινωϲιν αυτον} & 9 &  &  \\
& \textbf{54} &  & \foreignlanguage{greek}{ο ουν \textoverline{ιϲ} ουκετι παρρηϲια περιεπατει εν} & 7 &  &  \\
&  & 8 & \foreignlanguage{greek}{τοιϲ ιουδαιοιϲ αλλα απηλθεν εκειθε̅} & 12 &  &  \\
&  & 13 & \foreignlanguage{greek}{ειϲ την χωραν εγγυϲ τηϲ ερημου ειϲ ε} & 20 &  &  \\
&  & 20 & \foreignlanguage{greek}{φρεμ λεγομενην πολιν και εκει εμει} & 25 &  &  \\
&  & 25 & \foreignlanguage{greek}{νεν μετα των μαθητων} & 28 &  &  \\
& \textbf{55} &  & \foreignlanguage{greek}{ην δε εγγυϲ το παϲχα των ιουδαιων} & 7 &  &  \\
&  & 8 & \foreignlanguage{greek}{και ανεβηϲαν πολλοι ειϲ ιεροϲολυμα εκ} & 13 &  &  \\
&  & 14 & \foreignlanguage{greek}{τηϲ χωραϲ προ του παϲχα ινα αγνιϲωϲι̅} & 20 &  &  \\
&  & 21 & \foreignlanguage{greek}{εαυτουϲ εζητουν ουν τον \textoverline{ιν} και ελε} & 6 & \textbf{56} &  \\
&  & 6 & \foreignlanguage{greek}{γον μετ αλληλων εν τω ιερω εϲτηκο} & 12 &  &  \\
&  & 12 & \foreignlanguage{greek}{τεϲ τι δοκει υμιν οτι ου μη ελθη ειϲ τη̅} & 21 &  &  \\
&  & 22 & \foreignlanguage{greek}{εορτην δεδωκειϲαν δε οι αρχιερειϲ} & 4 & \textbf{57} &  \\
&  & 5 & \foreignlanguage{greek}{και οι φαριϲαιοι εντολαϲ ινα εαν τιϲ γνω} & 12 &  &  \\
&  & 13 & \foreignlanguage{greek}{που εϲτιν μηνυϲη οπωϲ πιαϲωϲιν αυτο̅} & 18 &  &  \\
[0.2em]
\cline{4-4}
\end{tabular}
\end{center}
\end{table}
}
\clearpage
\newpage
 {
 \setlength\arrayrulewidth{1pt}
\begin{table}
\begin{center}
\begin{tabular}{ccc|l|ccc}
\cline{4-4} \\ [-1em]
\multicolumn{7}{c}{\foreignlanguage{greek}{ευαγγελιον κατα ιωαννην} \textbf{(\nospace{12:1})} } \\ \\ [-1em] % Si on veut ajouter les bordures latérales, remplacer {7}{c} par {7}{|c|}
\cline{4-4} \\
\cline{4-4}
&  &  & &  &  & \\ [-0.9em]
& \mygospelchapter &  & \foreignlanguage{greek}{ο ουν \textoverline{ιϲ} προ εξ ημερων του παϲχα ηλθε̅} & 9 &  &  \\
&  & 10 & \foreignlanguage{greek}{ειϲ βηθανιαν οπου ην λαζαροϲ ον ηγει} & 16 &  &  \\
&  & 16 & \foreignlanguage{greek}{ρεν εκ νεκρων ο \textoverline{ιϲ} εποιηϲαν ουν αυτω} & 3 & \textbf{2} &  \\
&  & 4 & \foreignlanguage{greek}{διπνον εκει και η μαρθα διηκονει} & 9 &  &  \\
&  & 10 & \foreignlanguage{greek}{αυτω ο δε λαζαροϲ ειϲ ην των ϲυνα} & 17 &  &  \\
&  & 17 & \foreignlanguage{greek}{νακειμενων αυτω} & 18 &  &  \\
& \textbf{3} &  & \foreignlanguage{greek}{η ουν μαρια λαβουϲα λιτραν μυρου} & 6 &  &  \\
&  & 7 & \foreignlanguage{greek}{ναρδου πιϲτικηϲ πολυτιμου ηλιψε̅} & 10 &  &  \\
&  & 11 & \foreignlanguage{greek}{τουϲ ποδαϲ του \textoverline{ιυ} και εξεμαξεν ταιϲ} & 17 &  &  \\
&  & 18 & \foreignlanguage{greek}{θριξιν αυτηϲ τουϲ ποδαϲ αυτου} & 22 &  &  \\
&  & 23 & \foreignlanguage{greek}{η δε οικεια επληρωθη τηϲ οϲμηϲ του} & 29 &  &  \\
&  & 30 & \foreignlanguage{greek}{μυρου λεγει δε ιουδαϲ ο ιϲκαριω} & 5 & \textbf{4} &  \\
&  & 5 & \foreignlanguage{greek}{τηϲ ειϲ των μαθητων αυτου ο μελλω̅} & 11 &  &  \\
&  & 12 & \foreignlanguage{greek}{αυτον παραδιδοναι δια τι τουτο το} & 4 & \textbf{5} &  \\
&  & 5 & \foreignlanguage{greek}{μυρον ουκ επραθη τριακοϲιων δηνα} & 9 &  &  \\
&  & 9 & \foreignlanguage{greek}{ριων και εδοθη πτωχοιϲ} & 12 &  &  \\
& \textbf{6} &  & \foreignlanguage{greek}{ειπεν δε τουτο ουχ οτι περι των πτω} & 8 &  &  \\
&  & 8 & \foreignlanguage{greek}{χων εμελεν αυτω αλλ οτι κλεπτηϲ} & 13 &  &  \\
&  & 14 & \foreignlanguage{greek}{ην και το γλωϲϲοκομιον εχον και τα βαλ} & 21 &  &  \\
&  & 21 & \foreignlanguage{greek}{λομενα εβαϲταζεν} & 22 &  &  \\
& \textbf{7} &  & \foreignlanguage{greek}{ειπεν ουν ο \textoverline{ιϲ} αφεϲ αυτην ινα ειϲ τη̅} & 9 &  &  \\
&  & 10 & \foreignlanguage{greek}{ημεραν του ενταφιαϲμου μου τηρη} & 14 &  &  \\
&  & 14 & \foreignlanguage{greek}{ϲη αυτο τουϲ πτωχουϲ γαρ παντοτε} & 4 & \textbf{8} &  \\
&  & 5 & \foreignlanguage{greek}{εχεται μεθ εαυτων εμε δε ου παν} & 11 &  &  \\
&  & 11 & \foreignlanguage{greek}{τοτε εχεται εγνω ουν ο οχλοϲ ο πο} & 6 & \textbf{9} &  \\
&  & 6 & \foreignlanguage{greek}{λυϲ των ιουδαιων οτι εκει εϲτιν} & 11 &  &  \\
&  & 12 & \foreignlanguage{greek}{και ηλθον ου δια τον \textoverline{ιν} μονον αλλ} & 19 &  &  \\
&  & 20 & \foreignlanguage{greek}{ινα και τον λαζαρον ιδωϲιν ον ηγειρε̅} & 26 &  &  \\
& \textbf{10} &  & \foreignlanguage{greek}{εβουλευϲαντο δε οι αρχιερειϲ ινα και} & 6 &  &  \\
&  & 7 & \foreignlanguage{greek}{τον λαζαρον αποκτινωϲιν οτι πολ} & 2 & \textbf{11} &  \\
[0.2em]
\cline{4-4}
\end{tabular}
\end{center}
\end{table}
}
\clearpage
\newpage
 {
 \setlength\arrayrulewidth{1pt}
\begin{table}
\begin{center}
\begin{tabular}{ccc|l|ccc}
\cline{4-4} \\ [-1em]
\multicolumn{7}{c}{\foreignlanguage{greek}{ευαγγελιον κατα ιωαννην} \textbf{(\nospace{12:11})} } \\ \\ [-1em] % Si on veut ajouter les bordures latérales, remplacer {7}{c} par {7}{|c|}
\cline{4-4} \\
\cline{4-4}
&  &  & &  &  & \\ [-0.9em]
&  & 2 & \foreignlanguage{greek}{λοι δι αυτον υπηγον των ιουδαιων} & 7 &  &  \\
&  & 8 & \foreignlanguage{greek}{και επιϲτευον ειϲ τον \textoverline{ιν}} & 12 &  &  \\
& \textbf{12} &  & \foreignlanguage{greek}{τη επαυριον οχλοϲ πολυϲ ο ελθων ειϲ} & 7 &  &  \\
&  & 8 & \foreignlanguage{greek}{την εορτην ακουϲαντεϲ οτι ερχε} & 12 &  &  \\
&  & 12 & \foreignlanguage{greek}{ται \textoverline{ιϲ} ειϲ ιεροϲολυμα ελαβον τα βαια} & 3 & \textbf{13} &  \\
&  & 4 & \foreignlanguage{greek}{των φοινικων και εξηλθον ειϲ υπα̅} & 9 &  &  \\
&  & 9 & \foreignlanguage{greek}{τηϲιν αυτω και εκραυγαζον ωϲαν} & 13 &  &  \\
&  & 13 & \foreignlanguage{greek}{να ευλογημενοϲ ο ερχομενοϲ εν ονο} & 18 &  &  \\
&  & 18 & \foreignlanguage{greek}{ματι \textoverline{κυ} και ο βαϲιλευϲ του \textoverline{ιηλ}} & 24 &  &  \\
& \textbf{14} &  & \foreignlanguage{greek}{ευρων δε ο \textoverline{ιϲ} οναριον εκαθειϲεν επ αυ} & 8 &  &  \\
&  & 8 & \foreignlanguage{greek}{το καθωϲ εϲτιν γεγραμμενον} & 11 &  &  \\
& \textbf{15} &  & \foreignlanguage{greek}{μη φοβου θυγατηρ ϲιων ιδου ο βαϲιλευϲ} & 7 &  &  \\
&  & 8 & \foreignlanguage{greek}{ϲου ερχεται καθημενοϲ επι πωλον ονου} & 13 &  &  \\
& \textbf{16} &  & \foreignlanguage{greek}{ταυτα ουκ εγνωϲαν οι μαθηται αυτου} & 6 &  &  \\
&  & 7 & \foreignlanguage{greek}{το πρωτον αλλ οτε εδοξαϲθη ο \textoverline{ιϲ} ε} & 14 &  &  \\
&  & 14 & \foreignlanguage{greek}{μνηϲθηϲαν οτι ταυτα ην επ αυτω γε} & 20 &  &  \\
&  & 20 & \foreignlanguage{greek}{γραμμενα και ταυτα εποιηϲαν αυτω} & 24 &  &  \\
& \textbf{17} &  & \foreignlanguage{greek}{εμαρτυρι ουν ο οχλοϲ ο ων μετ αυτου} & 8 &  &  \\
&  & 9 & \foreignlanguage{greek}{οτε τον λαζαρον εφωνηϲεν εκ του} & 14 &  &  \\
&  & 15 & \foreignlanguage{greek}{μνημιου και ηγειρεν αυτον εκ νε} & 20 &  &  \\
&  & 20 & \foreignlanguage{greek}{κρων δια τουτο και υπηντηϲεν} & 4 & \textbf{18} &  \\
&  & 5 & \foreignlanguage{greek}{αυτω οχλοϲ οτι ηκουϲαν τουτο αυ} & 10 &  &  \\
&  & 10 & \foreignlanguage{greek}{τον πεποιηκεναι το ϲημιον} & 13 &  &  \\
& \textbf{19} &  & \foreignlanguage{greek}{οι ουν φαριϲαιοι ειπον προϲ εαυτουϲ} & 6 &  &  \\
&  & 7 & \foreignlanguage{greek}{θεωρειται οτι ουκ ωφελειται ουδεν} & 11 &  &  \\
&  & 12 & \foreignlanguage{greek}{ειδε ο κοϲμοϲ οπιϲω αυτου απηλθεν} & 17 &  &  \\
& \textbf{20} &  & \foreignlanguage{greek}{ηϲαν δε ελληνεϲ τινεϲ εκ των ανα} & 7 &  &  \\
&  & 7 & \foreignlanguage{greek}{βαντων ινα προϲκυνηϲωϲιν εν τη} & 11 &  &  \\
&  & 12 & \foreignlanguage{greek}{εορτη ουτοι ουν προϲηλθον τω} & 4 & \textbf{21} &  \\
&  & 5 & \foreignlanguage{greek}{φιλιππω τω απο βηδϲαιδα τηϲ γαλι} & 10 &  &  \\
[0.2em]
\cline{4-4}
\end{tabular}
\end{center}
\end{table}
}
\clearpage
\newpage
 {
 \setlength\arrayrulewidth{1pt}
\begin{table}
\begin{center}
\begin{tabular}{ccc|l|ccc}
\cline{4-4} \\ [-1em]
\multicolumn{7}{c}{\foreignlanguage{greek}{ευαγγελιον κατα ιωαννην} \textbf{(\nospace{12:21})} } \\ \\ [-1em] % Si on veut ajouter les bordures latérales, remplacer {7}{c} par {7}{|c|}
\cline{4-4} \\
\cline{4-4}
&  &  & &  &  & \\ [-0.9em]
&  & 10 & \foreignlanguage{greek}{λαιαϲ και ηρωτων αυτον λεγοντεϲ} & 14 &  &  \\
&  & 15 & \foreignlanguage{greek}{\textoverline{κε} θελομεν τον \textoverline{ιν} ιδειν ερχεται φι} & 2 & \textbf{22} &  \\
&  & 2 & \foreignlanguage{greek}{λιπποϲ και λεγει τω ανδρεα και πα} & 8 &  &  \\
&  & 8 & \foreignlanguage{greek}{λιν ανδρεαϲ και ο φιλιπποϲ λεγουϲι̅} & 13 &  &  \\
&  & 14 & \foreignlanguage{greek}{τω \textoverline{ιυ} ο δε \textoverline{ιϲ} αποκρινεται αυτοιϲ} & 5 & \textbf{23} &  \\
&  & 6 & \foreignlanguage{greek}{λεγων εληλυθεν η ωρα ινα δοξαϲθη} & 11 &  &  \\
&  & 12 & \foreignlanguage{greek}{ο υιοϲ του \textoverline{ανου} αμην αμην λεγω υ} & 4 & \textbf{24} &  \\
&  & 4 & \foreignlanguage{greek}{μιν εαν μη ο κοκκοϲ του ϲιτου πεϲω̅} & 11 &  &  \\
&  & 12 & \foreignlanguage{greek}{ειϲ την γην αποθανη αυτοϲ μονοϲ με} & 18 &  &  \\
&  & 18 & \foreignlanguage{greek}{νει εαν δε αποθανη πολυν καρπον} & 23 &  &  \\
&  & 24 & \foreignlanguage{greek}{φερει ο φιλων την ψυχην αυτου} & 5 & \textbf{25} &  \\
&  & 6 & \foreignlanguage{greek}{απολλυει αυτην και ο μιϲων την} & 11 &  &  \\
&  & 12 & \foreignlanguage{greek}{ψυχην αυτου εν τω κοϲμω τουτω φυ} & 18 &  &  \\
&  & 18 & \foreignlanguage{greek}{λαξει αυτην ειϲ ζωην αιωνιον} & 22 &  &  \\
& \textbf{26} &  & \foreignlanguage{greek}{εαν εμοι τιϲ διακονη εμοι ακολουθει} & 6 &  &  \\
&  & 6 & \foreignlanguage{greek}{τω και οπου εγω ειμει εκει και ο δια} & 14 &  &  \\
&  & 14 & \foreignlanguage{greek}{κονοϲ ο εμοϲ εϲται εαν τιϲ εμοι δια} & 21 &  &  \\
&  & 21 & \foreignlanguage{greek}{κονη τιμηϲει αυτον ο \textoverline{πηρ}} & 25 &  &  \\
& \textbf{27} &  & \foreignlanguage{greek}{νυν η ψυχη μου τεταρακται και τι ει} & 8 &  &  \\
&  & 8 & \foreignlanguage{greek}{πω \textoverline{περ} ϲωϲον με εκ τηϲ ωραϲ ταυτηϲ} & 15 &  &  \\
&  & 16 & \foreignlanguage{greek}{αλλα δια τουτο ηλθον ειϲ την ωραν} & 22 &  &  \\
&  & 23 & \foreignlanguage{greek}{ταυτην \textoverline{περ} δοξαϲον ϲου το ονομα} & 5 & \textbf{28} &  \\
&  & 6 & \foreignlanguage{greek}{ηλθεν ουν φωνη εκ του ουρανου} & 11 &  &  \\
&  & 12 & \foreignlanguage{greek}{και εδοξαϲα και παλιν δοξαϲω} & 16 &  &  \\
& \textbf{29} &  & \foreignlanguage{greek}{ο δε οχλοϲ ο εϲτηκωϲ και ακουϲαϲ ε} & 8 &  &  \\
&  & 8 & \foreignlanguage{greek}{λεγεν βροντην γεγονεναι} & 10 &  &  \\
&  & 11 & \foreignlanguage{greek}{αλλοι δε ελεγον αγγελοϲ αυτω λε} & 16 &  &  \\
&  & 16 & \foreignlanguage{greek}{λαληκεν} & 16 &  &  \\
& \textbf{30} &  & \foreignlanguage{greek}{απεκριθη \textoverline{ιϲ} και ειπεν ου δι εμε η φω} & 9 &  &  \\
&  & 9 & \foreignlanguage{greek}{νη αυτη γεγονεν αλλα δι υμαϲ} & 14 &  &  \\
[0.2em]
\cline{4-4}
\end{tabular}
\end{center}
\end{table}
}
\clearpage
\newpage
 {
 \setlength\arrayrulewidth{1pt}
\begin{table}
\begin{center}
\begin{tabular}{ccc|l|ccc}
\cline{4-4} \\ [-1em]
\multicolumn{7}{c}{\foreignlanguage{greek}{ευαγγελιον κατα ιωαννην} \textbf{(\nospace{12:31})} } \\ \\ [-1em] % Si on veut ajouter les bordures latérales, remplacer {7}{c} par {7}{|c|}
\cline{4-4} \\
\cline{4-4}
&  &  & &  &  & \\ [-0.9em]
& \textbf{31} &  & \foreignlanguage{greek}{νυν κριϲειϲ εϲτιν του κοϲμου νυν ο αρ} & 8 &  &  \\
&  & 8 & \foreignlanguage{greek}{χων του κοϲμου τουτου εκβληθηϲεται} & 12 &  &  \\
&  & 13 & \foreignlanguage{greek}{εξω καγω εαν υψωθω εκ τηϲ γηϲ παν} & 7 & \textbf{32} &  \\
&  & 7 & \foreignlanguage{greek}{ταϲ ελκυϲω προϲ εμαυτον τουτο δε ε} & 3 & \textbf{33} &  \\
&  & 3 & \foreignlanguage{greek}{λεγεν ϲημαινων ποιω θανατω ημελ} & 7 &  &  \\
&  & 7 & \foreignlanguage{greek}{λεν αποθνηϲκειν} & 8 &  &  \\
& \textbf{34} &  & \foreignlanguage{greek}{απεκριθη ουν αυτω ο οχλοϲ ημειϲ ηκου} & 7 &  &  \\
&  & 7 & \foreignlanguage{greek}{ϲαμεν εκ του νομου οτι ο \textoverline{χϲ} μενει ειϲ} & 15 &  &  \\
&  & 16 & \foreignlanguage{greek}{τον αιωνα και πωϲ λεγειϲ ϲυ οτι δει υ} & 24 &  &  \\
&  & 24 & \foreignlanguage{greek}{ψωθηναι τον υιον του \textoverline{ανου}} & 28 &  &  \\
&  & 29 & \foreignlanguage{greek}{τιϲ εϲτιν ουτοϲ ο υιοϲ του \textoverline{ανου}} & 35 &  &  \\
& \textbf{35} &  & \foreignlanguage{greek}{ειπεν ουν αυτοιϲ ο \textoverline{ιϲ} ετι μικρον χρο} & 8 &  &  \\
&  & 8 & \foreignlanguage{greek}{νον το φωϲ εν υμιν εϲτιν περιπατει} & 14 &  &  \\
&  & 14 & \foreignlanguage{greek}{ται ωϲ το φωϲ εχεται ινα μη ϲκοτια υ} & 22 &  &  \\
&  & 22 & \foreignlanguage{greek}{μαϲ λαβη και ο περιπατων εν τη ϲκο} & 29 &  &  \\
&  & 29 & \foreignlanguage{greek}{τια ουκ οιδεν που υπαγει ωϲ το φωϲ ε} & 4 & \textbf{36} &  \\
&  & 4 & \foreignlanguage{greek}{χεται πιϲτευεται ειϲ το φωϲ ινα υιοι} & 10 &  &  \\
&  & 11 & \foreignlanguage{greek}{φωτοϲ γενηϲθαι} & 12 &  &  \\
&  & 13 & \foreignlanguage{greek}{ταυτα ελαληϲεν ο \textoverline{ιϲ} και απελθων εκρυ} & 19 &  &  \\
&  & 19 & \foreignlanguage{greek}{βη απ αυτων τοϲαυτα δε αυτου ϲη} & 4 & \textbf{37} &  \\
&  & 4 & \foreignlanguage{greek}{μια πεποιηκοτοϲ εμπροϲθεν αυτων} & 7 &  &  \\
&  & 8 & \foreignlanguage{greek}{ουκ επιϲτευον ειϲ αυτον ινα ο λογοϲ} & 3 & \textbf{38} &  \\
&  & 4 & \foreignlanguage{greek}{ηϲαιου του προφητου πληρωθη ον ει} & 9 &  &  \\
&  & 9 & \foreignlanguage{greek}{πεν \textoverline{κε} τιϲ επιϲτευϲεν τη ακοη η} & 15 &  &  \\
&  & 15 & \foreignlanguage{greek}{μων και ο βραχιων \textoverline{κυ} τινι απεκα} & 21 &  &  \\
&  & 21 & \foreignlanguage{greek}{λυφθη δια τουτο ουκ ηδυναντο πι} & 5 & \textbf{39} &  \\
&  & 5 & \foreignlanguage{greek}{ϲτευειν οτι παλιν ειπεν ηϲαιαϲ} & 9 &  &  \\
& \textbf{40} &  & \foreignlanguage{greek}{τετυφλωκεν αυτων τουϲ οφθαλμουϲ} & 4 &  &  \\
&  & 5 & \foreignlanguage{greek}{και επηρωϲεν αυτων την καρδιαν} & 9 &  &  \\
&  & 10 & \foreignlanguage{greek}{ινα μη ειδωϲιν τοιϲ οφθαλμοιϲ} & 14 &  &  \\
[0.2em]
\cline{4-4}
\end{tabular}
\end{center}
\end{table}
}
\clearpage
\newpage
 {
 \setlength\arrayrulewidth{1pt}
\begin{table}
\begin{center}
\begin{tabular}{ccc|l|ccc}
\cline{4-4} \\ [-1em]
\multicolumn{7}{c}{\foreignlanguage{greek}{ευαγγελιον κατα ιωαννην} \textbf{(\nospace{12:40})} } \\ \\ [-1em] % Si on veut ajouter les bordures latérales, remplacer {7}{c} par {7}{|c|}
\cline{4-4} \\
\cline{4-4}
&  &  & &  &  & \\ [-0.9em]
&  & 15 & \foreignlanguage{greek}{και νοηϲωϲιν τη καρδια και επιϲτρε} & 20 &  &  \\
&  & 20 & \foreignlanguage{greek}{ψωϲιν και ιαϲομαι αυτουϲ ταυτα ειπε̅} & 2 & \textbf{41} &  \\
&  & 3 & \foreignlanguage{greek}{ηϲαιαϲ επει ειδεν την δοξαν αυτου} & 8 &  &  \\
&  & 9 & \foreignlanguage{greek}{και ελαληϲεν περι αυτου ομωϲ μεν} & 2 & \textbf{42} &  \\
&  & 2 & \foreignlanguage{greek}{τοι πολλοι των αρχοντων επιϲτευϲαν} & 6 &  &  \\
&  & 7 & \foreignlanguage{greek}{ειϲ αυτον αλλα δια τουϲ φαριϲαιουϲ} & 12 &  &  \\
&  & 13 & \foreignlanguage{greek}{ουχ ωμολογουν ινα μη αποϲυναγω} & 17 &  &  \\
&  & 17 & \foreignlanguage{greek}{γοι γενωνται ηγαπηϲαν γαρ την δο} & 4 & \textbf{43} &  \\
&  & 4 & \foreignlanguage{greek}{ξαν των \textoverline{ανων} μαλλον υπερ την δο} & 10 &  &  \\
&  & 10 & \foreignlanguage{greek}{ξαν του \textoverline{θυ} εκραξεν δε ο \textoverline{ιϲ} και ειπε̅} & 6 & \textbf{44} &  \\
&  & 7 & \foreignlanguage{greek}{ο πιϲτευων ειϲ εμε ου πιϲτευει ειϲ εμε} & 14 &  &  \\
&  & 15 & \foreignlanguage{greek}{αλλα ειϲ τον πεμψαντα με και ο θε} & 3 & \textbf{45} &  \\
&  & 3 & \foreignlanguage{greek}{ωρων εμε θεωρει τον πεμψαντα με} & 8 &  &  \\
& \textbf{46} &  & \foreignlanguage{greek}{εγω φωϲ ειϲ τον κοϲμον εληλυθα ινα} & 7 &  &  \\
&  & 8 & \foreignlanguage{greek}{παϲ ο πιϲτευων ειϲ εμε εν τη ϲκοτια} & 15 &  &  \\
&  & 16 & \foreignlanguage{greek}{μη μεινη και εαν τιϲ μου μη ακου} & 6 & \textbf{47} &  \\
&  & 6 & \foreignlanguage{greek}{ϲη των ρηματων μηδε φυλαξη} & 10 &  &  \\
&  & 11 & \foreignlanguage{greek}{εγω ου κρινω αυτον ου γαρ ηλθον} & 17 &  &  \\
&  & 18 & \foreignlanguage{greek}{ινα κρινω τον κοϲμον αλλα ινα ϲω} & 24 &  &  \\
&  & 24 & \foreignlanguage{greek}{ϲω τον κοϲμον ο αθετων εμε και} & 4 & \textbf{48} &  \\
&  & 5 & \foreignlanguage{greek}{μη λαμβανων τα ρηματα μου εχει} & 10 &  &  \\
&  & 11 & \foreignlanguage{greek}{τον κρινοντα αυτον ο λογοϲ ον ε} & 17 &  &  \\
&  & 17 & \foreignlanguage{greek}{λαληϲα εκεινοϲ κρινει αυτον εν ε} & 22 &  &  \\
&  & 22 & \foreignlanguage{greek}{ϲχατη ημερα οτι εγω εξ εμαυτου} & 4 & \textbf{49} &  \\
&  & 5 & \foreignlanguage{greek}{ουκ ελαληϲα αλλ ο πεμψαϲ με \textoverline{πηρ}} & 11 &  &  \\
&  & 12 & \foreignlanguage{greek}{αυτοϲ εντολην μοι δεδωκεν τι ει} & 17 &  &  \\
&  & 17 & \foreignlanguage{greek}{πω και τι λαληϲω και οιδα οτι η εν} & 5 & \textbf{50} &  \\
&  & 5 & \foreignlanguage{greek}{τολη αυτου ζωη αιωνιοϲ εϲτιν} & 9 &  &  \\
&  & 10 & \foreignlanguage{greek}{α ουν εγω λαλω καθωϲ ειρηκεν μοι ο \textoverline{πηρ}} & 18 &  &  \\
&  & 19 & \foreignlanguage{greek}{ουτωϲ λαλω} & 20 &  &  \\
[0.2em]
\cline{4-4}
\end{tabular}
\end{center}
\end{table}
}
\clearpage
\newpage
 {
 \setlength\arrayrulewidth{1pt}
\begin{table}
\begin{center}
\begin{tabular}{ccc|l|ccc}
\cline{4-4} \\ [-1em]
\multicolumn{7}{c}{\foreignlanguage{greek}{ευαγγελιον κατα ιωαννην} \textbf{(\nospace{13:1})} } \\ \\ [-1em] % Si on veut ajouter les bordures latérales, remplacer {7}{c} par {7}{|c|}
\cline{4-4} \\
\cline{4-4}
&  &  & &  &  & \\ [-0.9em]
& \mygospelchapter &  & \foreignlanguage{greek}{προ δε τηϲ εορτηϲ του παϲχα ειδωϲ ο \textoverline{ιϲ}} & 9 &  &  \\
&  & 10 & \foreignlanguage{greek}{οτι ηλθεν αυτου η ωρα ινα μεταβη εκ} & 17 &  &  \\
&  & 18 & \foreignlanguage{greek}{του κοϲμου τουτου προϲ τον \textoverline{πρα} αγα} & 24 &  &  \\
&  & 24 & \foreignlanguage{greek}{πηϲαϲ τουϲ ιδιουϲ τουϲ εν τω κοϲμω} & 30 &  &  \\
&  & 31 & \foreignlanguage{greek}{ειϲ τελοϲ ηγαπηϲεν αυτουϲ και δι} & 2 & \textbf{2} &  \\
&  & 2 & \foreignlanguage{greek}{πνου γεινομενου του διαβολου ηδη} & 6 &  &  \\
&  & 7 & \foreignlanguage{greek}{βεβληκοτοϲ ειϲ την καρδιαν ινα πα} & 12 &  &  \\
&  & 12 & \foreignlanguage{greek}{ραδω αυτον ιδα ϲειμωνοϲ ιϲκαριωτη} & 16 &  &  \\
& \textbf{3} &  & \foreignlanguage{greek}{ειδωϲ οτι παντα εδωκεν αυτω ο \textoverline{πηρ} ειϲ} & 8 &  &  \\
&  & 9 & \foreignlanguage{greek}{ταϲ χειραϲ και οτι απο \textoverline{θυ} εξηλθεν και} & 16 &  &  \\
&  & 17 & \foreignlanguage{greek}{προϲ τον \textoverline{θν} υπαγει εγειρεται εκ του} & 3 & \textbf{4} &  \\
&  & 4 & \foreignlanguage{greek}{διπνου και τιθηϲιν τα ιματια και λα} & 10 &  &  \\
&  & 10 & \foreignlanguage{greek}{βων λεντιον διεζωϲεν εαυτον} & 13 &  &  \\
& \textbf{5} &  & \foreignlanguage{greek}{ειτα βαλλει υδωρ ειϲ τον νιπτηρα} & 6 &  &  \\
&  & 7 & \foreignlanguage{greek}{και ηρξατο νιπτειν τουϲ ποδαϲ των} & 12 &  &  \\
&  & 13 & \foreignlanguage{greek}{μαθητων και εκμαϲϲιν τω λεντιω} & 17 &  &  \\
&  & 18 & \foreignlanguage{greek}{ω ην διεζωϲμενοϲ} & 20 &  &  \\
& \textbf{6} &  & \foreignlanguage{greek}{ερχεται ουν προϲ ϲιμωνα πετρον και} & 6 &  &  \\
&  & 7 & \foreignlanguage{greek}{λεγει αυτω εκεινοϲ \textoverline{κε} ϲυ μου νιπτιϲ} & 13 &  &  \\
&  & 14 & \foreignlanguage{greek}{τουϲ ποδαϲ απεκριθη \textoverline{ιϲ} και ειπεν} & 4 & \textbf{7} &  \\
&  & 5 & \foreignlanguage{greek}{αυτω ο εγω ποιω ϲοι ουκ οιδαϲ τι} & 12 &  &  \\
&  & 13 & \foreignlanguage{greek}{γνωϲη δε μετα ταυτα} & 16 &  &  \\
& \textbf{8} &  & \foreignlanguage{greek}{λεγει αυτω πετροϲ ου μη νιψηϲ μου} & 7 &  &  \\
&  & 8 & \foreignlanguage{greek}{τουϲ ποδαϲ ειϲ τον αιωνα} & 12 &  &  \\
&  & 13 & \foreignlanguage{greek}{απεκριθη αυτω \textoverline{ιϲ} εαν μη νιψω ϲε} & 19 &  &  \\
&  & 20 & \foreignlanguage{greek}{ουκ εχειϲ μεροϲ μετ εμου λεγει αυ} & 2 & \textbf{9} &  \\
&  & 2 & \foreignlanguage{greek}{τω πετροϲ ϲειμων \textoverline{κε} μη τουϲ ποδαϲ} & 8 &  &  \\
&  & 9 & \foreignlanguage{greek}{μου μονον αλλα και ταϲ χειραϲ και τη̅} & 16 &  &  \\
&  & 17 & \foreignlanguage{greek}{κεφαλην λεγει αυτω ο \textoverline{ιϲ} ο λε} & 6 & \textbf{10} &  \\
&  & 6 & \foreignlanguage{greek}{λουμενοϲ ουκ εχει χριαν ει μη τουϲ} & 12 &  &  \\
[0.2em]
\cline{4-4}
\end{tabular}
\end{center}
\end{table}
}
\clearpage
\newpage
 {
 \setlength\arrayrulewidth{1pt}
\begin{table}
\begin{center}
\begin{tabular}{ccc|l|ccc}
\cline{4-4} \\ [-1em]
\multicolumn{7}{c}{\foreignlanguage{greek}{ευαγγελιον κατα ιωαννην} \textbf{(\nospace{13:10})} } \\ \\ [-1em] % Si on veut ajouter les bordures latérales, remplacer {7}{c} par {7}{|c|}
\cline{4-4} \\
\cline{4-4}
&  &  & &  &  & \\ [-0.9em]
&  & 13 & \foreignlanguage{greek}{ποδαϲ νιψαϲθαι αλλ εϲτιν καθαροϲ ο} & 18 &  &  \\
&  & 18 & \foreignlanguage{greek}{λοϲ και υμειϲ καθαροι εϲται αλλ ουχι} & 24 &  &  \\
&  & 25 & \foreignlanguage{greek}{παντεϲ ηδει γαρ τον παραδιδουντα} & 4 & \textbf{11} &  \\
&  & 5 & \foreignlanguage{greek}{αυτον δια τουτο ειπεν οτι ουχει πα̅} & 11 &  &  \\
&  & 11 & \foreignlanguage{greek}{τεϲ καθαροι εϲται} & 13 &  &  \\
& \textbf{12} &  & \foreignlanguage{greek}{οτε ουν ενιψεν τουϲ ποδαϲ αυτων} & 6 &  &  \\
&  & 7 & \foreignlanguage{greek}{και ελαβεν τα ιματια εαυτου και} & 12 &  &  \\
&  & 13 & \foreignlanguage{greek}{ανεπεϲεν παλιν ειπεν αυτοιϲ} & 16 &  &  \\
&  & 17 & \foreignlanguage{greek}{γινωϲκεται τι πεποιηκα υμιν} & 20 &  &  \\
& \textbf{13} &  & \foreignlanguage{greek}{υμειϲ φωνειται με ο διδαϲκαλοϲ} & 5 &  &  \\
&  & 6 & \foreignlanguage{greek}{και ο \textoverline{κϲ} και καλωϲ λεγεται ειμει γαρ} & 13 &  &  \\
& \textbf{14} &  & \foreignlanguage{greek}{ει ουν εγω ενειψα υμων τουϲ ποδαϲ} & 7 &  &  \\
&  & 8 & \foreignlanguage{greek}{ο \textoverline{κϲ} και ο διδαϲκαλοϲ και υμειϲ ο} & 15 &  &  \\
&  & 15 & \foreignlanguage{greek}{φειλεται αλληλων νιπτειν τουϲ} & 18 &  &  \\
&  & 19 & \foreignlanguage{greek}{ποδαϲ υποδιγμα γαρ εδωκα υμιν} & 4 & \textbf{15} &  \\
&  & 5 & \foreignlanguage{greek}{ινα καθωϲ εγω εποιηϲα υμιν και υ} & 11 &  &  \\
&  & 11 & \foreignlanguage{greek}{μειϲ ποιηται αμην αμην λε} & 3 & \textbf{16} &  \\
&  & 3 & \foreignlanguage{greek}{γω υμιν ουκ εϲτιν δουλοϲ μειζω̅} & 8 &  &  \\
&  & 9 & \foreignlanguage{greek}{του \textoverline{κυ} αυτου ουδε αποϲτολοϲ μει} & 14 &  &  \\
&  & 14 & \foreignlanguage{greek}{ζον του πεμψαντοϲ αυτον ει ταυ} & 2 & \textbf{17} &  \\
&  & 2 & \foreignlanguage{greek}{τα οιδατε μακαριοι εϲται εαν ποι} & 7 &  &  \\
&  & 7 & \foreignlanguage{greek}{ηται αυτα ου περι παντων υμω̅} & 4 & \textbf{18} &  \\
&  & 5 & \foreignlanguage{greek}{λεγω εγω οιδα ουϲ εξελεξαμην} & 9 &  &  \\
&  & 10 & \foreignlanguage{greek}{αλλα ινα η γραφη πληρωθη ο τρω} & 16 &  &  \\
&  & 16 & \foreignlanguage{greek}{γων μετ εμου τον αρτον επηρκεν} & 21 &  &  \\
&  & 22 & \foreignlanguage{greek}{επ εμε την πτερναν αυτου} & 26 &  &  \\
& \textbf{19} &  & \foreignlanguage{greek}{απ αρτι λεγω υμιν προ του γενεϲθαι} & 7 &  &  \\
&  & 8 & \foreignlanguage{greek}{ινα οταν γενηται πιϲτευϲηται ο} & 12 &  &  \\
&  & 12 & \foreignlanguage{greek}{τι εγω ειμει} & 14 &  &  \\
& \textbf{20} &  & \foreignlanguage{greek}{αμην αμην λεγω υμιν ο λαμβανω̅} & 6 &  &  \\
[0.2em]
\cline{4-4}
\end{tabular}
\end{center}
\end{table}
}
\clearpage
\newpage
 {
 \setlength\arrayrulewidth{1pt}
\begin{table}
\begin{center}
\begin{tabular}{ccc|l|ccc}
\cline{4-4} \\ [-1em]
\multicolumn{7}{c}{\foreignlanguage{greek}{ευαγγελιον κατα ιωαννην} \textbf{(\nospace{13:20})} } \\ \\ [-1em] % Si on veut ajouter les bordures latérales, remplacer {7}{c} par {7}{|c|}
\cline{4-4} \\
\cline{4-4}
&  &  & &  &  & \\ [-0.9em]
&  & 7 & \foreignlanguage{greek}{αν τινα πεμψω εμε λαμβανει ο δε εμε} & 14 &  &  \\
&  & 15 & \foreignlanguage{greek}{λαμβανων λαμβανει τον πεμψαντα με} & 19 &  &  \\
& \textbf{21} &  & \foreignlanguage{greek}{ταυτα ειπων ο \textoverline{ιϲ} εταραχθη τω \textoverline{πνι} και ε} & 9 &  &  \\
&  & 9 & \foreignlanguage{greek}{μαρτυρηϲεν και ειπεν αμην αμην} & 13 &  &  \\
&  & 14 & \foreignlanguage{greek}{λεγω υμιν οτι ειϲ εξ υμων παραδωϲι με} & 21 &  &  \\
& \textbf{22} &  & \foreignlanguage{greek}{εβλεπον ουν ειϲ αλληλουϲ οι μαθηται} & 6 &  &  \\
&  & 7 & \foreignlanguage{greek}{απορουμενοι περι τινοϲ λεγει} & 10 &  &  \\
& \textbf{23} &  & \foreignlanguage{greek}{ην δε ανακειμενοϲ ειϲ εκ των μαθη} & 7 &  &  \\
&  & 7 & \foreignlanguage{greek}{των εν τω κολπω του \textoverline{ιυ} ον ηγαπα ο \textoverline{ιϲ}} & 16 &  &  \\
& \textbf{24} &  & \foreignlanguage{greek}{νευει ουν τουτω ϲιμων πετροϲ πυθε} & 6 &  &  \\
&  & 6 & \foreignlanguage{greek}{ϲθαι τιϲ αν ειη περι ου λεγει} & 12 &  &  \\
& \textbf{25} &  & \foreignlanguage{greek}{επιπεϲων ουν εκεινοϲ επι το ϲτηθοϲ} & 6 &  &  \\
&  & 7 & \foreignlanguage{greek}{του \textoverline{ιυ} λεγει αυτω \textoverline{κε} τιϲ εϲτιν} & 13 &  &  \\
& \textbf{26} &  & \foreignlanguage{greek}{αποκρινεται \textoverline{ιϲ} εκεινοϲ εϲτιν ω εγω} & 6 &  &  \\
&  & 7 & \foreignlanguage{greek}{δωϲω ενβαψαϲ το ψωμιον και εν} & 12 &  &  \\
&  & 12 & \foreignlanguage{greek}{βαψαϲ το ψωμιον διδωϲιν ιουδα ϲιμω} & 17 &  &  \\
&  & 17 & \foreignlanguage{greek}{νοϲ ιϲκαριωτη και μετα το ψωμι} & 4 & \textbf{27} &  \\
&  & 4 & \foreignlanguage{greek}{ον τοτε ειϲηλθεν ειϲ εκεινον ο ϲαταναϲ} & 10 &  &  \\
&  & 11 & \foreignlanguage{greek}{λεγει ουν αυτω ο \textoverline{ιϲ} ο ποιειϲ ποιηϲον} & 18 &  &  \\
&  & 19 & \foreignlanguage{greek}{ταχιον τουτο ουδειϲ εγνω των α} & 5 & \textbf{28} &  \\
&  & 5 & \foreignlanguage{greek}{νακειμενων προϲ τι ειπεν αυτω} & 9 &  &  \\
& \textbf{29} &  & \foreignlanguage{greek}{τινεϲ γαρ εδοκουν επι το γλωϲϲοκο} & 6 &  &  \\
&  & 6 & \foreignlanguage{greek}{μιον ειχεν ιουδαϲ οτι λεγει αυτω ο \textoverline{ιϲ}} & 13 &  &  \\
&  & 14 & \foreignlanguage{greek}{αγοραϲον ων χρειαν εχομεν ειϲ την} & 19 &  &  \\
&  & 20 & \foreignlanguage{greek}{εορτην η τοιϲ πτωχοιϲ ινα τι δω} & 26 &  &  \\
& \textbf{30} &  & \foreignlanguage{greek}{λαβων ουν το ψωμιον εκεινοϲ εξηλ} & 6 &  &  \\
&  & 6 & \foreignlanguage{greek}{θεν ευθυϲ ην δε νυξ οτε ουν εξηλ} & 3 & \textbf{31} &  \\
&  & 3 & \foreignlanguage{greek}{θεν λεγει ο \textoverline{ιϲ} νυν εδοξαϲθη ο υιοϲ του} & 11 &  &  \\
&  & 12 & \foreignlanguage{greek}{\textoverline{ανου} και ο \textoverline{θϲ} εδοξαϲθη εν αυτω και ο} & 2 & \textbf{32} &  \\
&  & 3 & \foreignlanguage{greek}{\textoverline{θϲ} δοξαϲει αυτον εν εαυτω} & 7 &  &  \\
[0.2em]
\cline{4-4}
\end{tabular}
\end{center}
\end{table}
}
\clearpage
\newpage
 {
 \setlength\arrayrulewidth{1pt}
\begin{table}
\begin{center}
\begin{tabular}{ccc|l|ccc}
\cline{4-4} \\ [-1em]
\multicolumn{7}{c}{\foreignlanguage{greek}{ευαγγελιον κατα ιωαννην} \textbf{(\nospace{13:33})} } \\ \\ [-1em] % Si on veut ajouter les bordures latérales, remplacer {7}{c} par {7}{|c|}
\cline{4-4} \\
\cline{4-4}
&  &  & &  &  & \\ [-0.9em]
& \textbf{33} &  & \foreignlanguage{greek}{τεκνια ετι μεικρον μεθ υμων ειμει} & 6 &  &  \\
&  & 7 & \foreignlanguage{greek}{ζητηϲεται με και καθωϲ ειρηκα τοιϲ} & 12 &  &  \\
&  & 13 & \foreignlanguage{greek}{ιουδαιοιϲ οπου υπαγω υμειϲ ου δυνα} & 18 &  &  \\
&  & 18 & \foreignlanguage{greek}{ϲθαι ελθειν και υμιν λεγω αρτι} & 23 &  &  \\
& \textbf{34} &  & \foreignlanguage{greek}{εντολην κενην διδωμι υμιν ινα} & 5 &  &  \\
&  & 6 & \foreignlanguage{greek}{αγαπατε αλληλουϲ καθωϲ ηγαπηϲα} & 9 &  &  \\
&  & 10 & \foreignlanguage{greek}{υμαϲ ινα και υμειϲ αγαπαται αλληλουϲ} & 15 &  &  \\
& \textbf{35} &  & \foreignlanguage{greek}{εν τουτω γνωϲονται παντεϲ οτι εμοι} & 6 &  &  \\
&  & 7 & \foreignlanguage{greek}{μαθηται εϲται εαν αγαπην εχηται εν} & 12 &  &  \\
&  & 13 & \foreignlanguage{greek}{αλληλοιϲ λεγει αυτω ϲιμων πετροϲ} & 4 & \textbf{36} &  \\
&  & 5 & \foreignlanguage{greek}{\textoverline{κε} που υπαγειϲ απεκριθη αυτω} & 9 &  &  \\
&  & 10 & \foreignlanguage{greek}{ο \textoverline{ιϲ} οπου υπαγω ου δυναϲαι μοι νυν} & 17 &  &  \\
&  & 18 & \foreignlanguage{greek}{ακολουθηϲαι ακολουθηϲειϲ δε υϲτερο̅} & 21 &  &  \\
& \textbf{37} &  & \foreignlanguage{greek}{λεγει αυτω ο πετροϲ \textoverline{κε} δια τι ου δυνα} & 9 &  &  \\
&  & 9 & \foreignlanguage{greek}{μαι ϲοι νυν ακολουθηϲαι αρτι υπερ} & 14 &  &  \\
&  & 15 & \foreignlanguage{greek}{ϲου την ψυχην μου θηϲω} & 19 &  &  \\
& \textbf{38} &  & \foreignlanguage{greek}{απεκρινεται ο \textoverline{ιϲ} την ψυχην ϲου υπερ} & 7 &  &  \\
&  & 8 & \foreignlanguage{greek}{εμου θηϲειϲ αμην αμην λεγω ϲοι} & 13 &  &  \\
&  & 14 & \foreignlanguage{greek}{ου μη αλεκτωρ φωνηϲη εωϲ ου ϲυ με} & 21 &  &  \\
&  & 22 & \foreignlanguage{greek}{απαρνηϲη τριϲ} & 23 &  &  \\
& \mygospelchapter &  & \foreignlanguage{greek}{μη ταραϲϲεϲθω υμων η καρδια πιϲτευ} & 6 &  &  \\
&  & 6 & \foreignlanguage{greek}{εται ειϲ τον \textoverline{θν} και ειϲ εμε πιϲτευεται} & 13 &  &  \\
& \textbf{2} &  & \foreignlanguage{greek}{εν τη οικεια του \textoverline{πρϲ} μου μοναι πολ} & 8 &  &  \\
&  & 8 & \foreignlanguage{greek}{λαι ειϲιν ει δε μη ειπον υμιν οτι} & 15 &  &  \\
&  & 16 & \foreignlanguage{greek}{πορευομαι ετοιμαϲαι τοπον υμιν} & 19 &  &  \\
& \textbf{3} &  & \foreignlanguage{greek}{και εαν πορευθω ετοιμαϲω υμιν το} & 6 &  &  \\
&  & 6 & \foreignlanguage{greek}{πον παλιν ερχομαι και παραλημ} & 10 &  &  \\
&  & 10 & \foreignlanguage{greek}{ψομαι υμαϲ προϲ εμαυτον ινα οπου} & 15 &  &  \\
&  & 16 & \foreignlanguage{greek}{εγω ειμει και υμειϲ ητε και οπου υ} & 3 & \textbf{4} &  \\
&  & 3 & \foreignlanguage{greek}{παγω οιδατε την οδον} & 6 &  &  \\
[0.2em]
\cline{4-4}
\end{tabular}
\end{center}
\end{table}
}
\clearpage
\newpage
 {
 \setlength\arrayrulewidth{1pt}
\begin{table}
\begin{center}
\begin{tabular}{ccc|l|ccc}
\cline{4-4} \\ [-1em]
\multicolumn{7}{c}{\foreignlanguage{greek}{ευαγγελιον κατα ιωαννην} \textbf{(\nospace{14:5})} } \\ \\ [-1em] % Si on veut ajouter les bordures latérales, remplacer {7}{c} par {7}{|c|}
\cline{4-4} \\
\cline{4-4}
&  &  & &  &  & \\ [-0.9em]
& \textbf{5} &  & \foreignlanguage{greek}{λεγει αυτω θωμαϲ \textoverline{κε} ουκ οιδαμεν που} & 7 &  &  \\
&  & 8 & \foreignlanguage{greek}{υπαγειϲ πωϲ δυναμεθα την οδον ειδεναι} & 13 &  &  \\
& \textbf{6} &  & \foreignlanguage{greek}{λεγει αυτω ο \textoverline{ιϲ} εγω ειμει η οδοϲ και η} & 10 &  &  \\
&  & 11 & \foreignlanguage{greek}{αληθεια και η ζωη ουδειϲ ερχεται} & 16 &  &  \\
&  & 17 & \foreignlanguage{greek}{προϲ τον \textoverline{πρα} ει μη δι εμου ει εγνωκει} & 2 & \textbf{7} &  \\
&  & 2 & \foreignlanguage{greek}{ται με και τον \textoverline{πρα} μου γνωϲεϲθαι και α} & 10 &  &  \\
&  & 10 & \foreignlanguage{greek}{π αρτι γιγνωϲκεται αυτον και εωρα} & 15 &  &  \\
&  & 15 & \foreignlanguage{greek}{κατε αυτον} & 16 &  &  \\
& \textbf{8} &  & \foreignlanguage{greek}{λεγει αυτω φιλιπποϲ \textoverline{κε} διξον ημιν το̅} & 7 &  &  \\
&  & 8 & \foreignlanguage{greek}{\textoverline{πρα} και αρκει ημιν} & 11 &  &  \\
& \textbf{9} &  & \foreignlanguage{greek}{λεγει αυτω ο \textoverline{ιϲ} τοϲουτω χρονω μεθ υ} & 8 &  &  \\
&  & 8 & \foreignlanguage{greek}{μων ειμει και ουκ εγνωκαϲ με φιλιππε} & 14 &  &  \\
&  & 15 & \foreignlanguage{greek}{ο εωρακωϲ εμε εωρακεν τον \textoverline{πρα}} & 20 &  &  \\
&  & 21 & \foreignlanguage{greek}{πωϲ ϲυ λεγειϲ δειξον ημιν τον \textoverline{πρα}} & 27 &  &  \\
& \textbf{10} &  & \foreignlanguage{greek}{ου πιϲτευειϲ οτι εγω εν τω \textoverline{πρι} και ο \textoverline{πηρ}} & 10 &  &  \\
&  & 11 & \foreignlanguage{greek}{εν εμοι εϲτιν τα ρηματα α εγω λαλω} & 18 &  &  \\
&  & 19 & \foreignlanguage{greek}{υμιν απ εμαυτου ου λαλω ο δε \textoverline{πηρ}} & 26 &  &  \\
&  & 27 & \foreignlanguage{greek}{ο εν εμοι μενων ποιει τα εργα αυτοϲ} & 34 &  &  \\
& \textbf{11} &  & \foreignlanguage{greek}{πιϲτευεται μοι οτι εγω εν τω \textoverline{πρι}} & 7 &  &  \\
&  & 8 & \foreignlanguage{greek}{και ο \textoverline{πηρ} εν εμοι ει δε μη γε δια τα} & 18 &  &  \\
&  & 19 & \foreignlanguage{greek}{εργα αυτα πιϲτευεται} & 21 &  &  \\
& \textbf{12} &  & \foreignlanguage{greek}{αμην αμην λεγω υμιν ο πιϲτευων ειϲ} & 7 &  &  \\
&  & 8 & \foreignlanguage{greek}{εμε τα εργα α εγω ποιω κακεινοϲ ποιηϲει} & 15 &  &  \\
&  & 16 & \foreignlanguage{greek}{και μειζονα τουτων ποιηϲει οτι εγω} & 21 &  &  \\
&  & 22 & \foreignlanguage{greek}{προϲ τον \textoverline{πρα} πορευομαι και ο τι αν αι} & 5 & \textbf{13} &  \\
&  & 5 & \foreignlanguage{greek}{τηϲηται εν τω ονοματι μου τουτο} & 10 &  &  \\
&  & 11 & \foreignlanguage{greek}{ποιηϲω ινα δοξαϲθη ο \textoverline{πηρ} εν τω \textoverline{υω}} & 18 &  &  \\
& \textbf{14} &  & \foreignlanguage{greek}{εαν τι αιτηϲηται με εν τω ονοματι} & 7 &  &  \\
&  & 8 & \foreignlanguage{greek}{μου εγω ποιηϲω εαν αγαπαται με} & 3 & \textbf{15} &  \\
&  & 4 & \foreignlanguage{greek}{ταϲ εντολαϲ ταϲ εμαϲ τηρηϲετε} & 8 &  &  \\
[0.2em]
\cline{4-4}
\end{tabular}
\end{center}
\end{table}
}
\clearpage
\newpage
 {
 \setlength\arrayrulewidth{1pt}
\begin{table}
\begin{center}
\begin{tabular}{ccc|l|ccc}
\cline{4-4} \\ [-1em]
\multicolumn{7}{c}{\foreignlanguage{greek}{ευαγγελιον κατα ιωαννην} \textbf{(\nospace{14:16})} } \\ \\ [-1em] % Si on veut ajouter les bordures latérales, remplacer {7}{c} par {7}{|c|}
\cline{4-4} \\
\cline{4-4}
&  &  & &  &  & \\ [-0.9em]
& \textbf{16} &  & \foreignlanguage{greek}{και εγω ερωτηϲω τον \textoverline{πρα} και αλλον} & 7 &  &  \\
&  & 8 & \foreignlanguage{greek}{παρακλητον δωϲει υμιν ινα μενη μεθ} & 13 &  &  \\
&  & 14 & \foreignlanguage{greek}{υμων ειϲ τον αιωνα το \textoverline{πνα} τηϲ αλη} & 4 & \textbf{17} &  \\
&  & 4 & \foreignlanguage{greek}{θειαϲ ο ο κοϲμοϲ ου δυναται λαβειν} & 10 &  &  \\
&  & 11 & \foreignlanguage{greek}{οτι ου θεωρει αυτο ουδε γιγνωϲκει} & 16 &  &  \\
&  & 17 & \foreignlanguage{greek}{υμειϲ γινωϲκεται αυτον οτι παρ υ} & 22 &  &  \\
&  & 22 & \foreignlanguage{greek}{μιν μενει και εν υμιν εϲτιν ουκ αφη} & 2 & \textbf{18} &  \\
&  & 2 & \foreignlanguage{greek}{ϲω υμαϲ ορφανουϲ ερχομαι προϲ υμαϲ} & 7 &  &  \\
& \textbf{19} &  & \foreignlanguage{greek}{ετι μικρον και ο κοϲμοϲ με ουκετι θε} & 8 &  &  \\
&  & 8 & \foreignlanguage{greek}{ωρει υμειϲ δε θεωρειται με οτι εγω ζω} & 15 &  &  \\
&  & 16 & \foreignlanguage{greek}{και υμειϲ ζηϲεϲθαι εκεινη τη ημερα} & 3 & \textbf{20} &  \\
&  & 4 & \foreignlanguage{greek}{γνωϲεϲθαι υμειϲ οτι εγω εν τω \textoverline{πρι} μου} & 11 &  &  \\
&  & 12 & \foreignlanguage{greek}{και υμειϲ εν εμοι καγω εν υμιν} & 18 &  &  \\
& \textbf{21} &  & \foreignlanguage{greek}{ο εχων ταϲ εντολαϲ μου και τηρων} & 7 &  &  \\
&  & 8 & \foreignlanguage{greek}{αυταϲ εκεινοϲ εϲτιν ο αγαπων με} & 13 &  &  \\
&  & 14 & \foreignlanguage{greek}{ο δε αγαπων με αγαπηθηϲεται υπο του} & 20 &  &  \\
&  & 21 & \foreignlanguage{greek}{\textoverline{πρϲ} μου και εγω αγαπηϲω αυτον και} & 27 &  &  \\
&  & 28 & \foreignlanguage{greek}{εμφανιϲω αυτω εμαυτον} & 30 &  &  \\
& \textbf{22} &  & \foreignlanguage{greek}{λεγει αυτω ιουδαϲ ουχ ο ιϲκαριωτηϲ} & 6 &  &  \\
&  & 7 & \foreignlanguage{greek}{\textoverline{κε} και τι γεγονεν οτι μελλειϲ ημιν} & 13 &  &  \\
&  & 14 & \foreignlanguage{greek}{εμφανιζειν ϲεαυτον και ουχι τω} & 18 &  &  \\
&  & 19 & \foreignlanguage{greek}{κοϲμω απεκριθη \textoverline{ιϲ} και ειπεν αυ} & 5 & \textbf{23} &  \\
&  & 5 & \foreignlanguage{greek}{τω εαν τιϲ αγαπα με τον λογον μου} & 12 &  &  \\
&  & 13 & \foreignlanguage{greek}{τηρηϲει και ο \textoverline{πηρ} μου αγαπηϲει αυτο̅} & 19 &  &  \\
&  & 20 & \foreignlanguage{greek}{και προϲ αυτον ελευϲομεθα και μο} & 25 &  &  \\
&  & 25 & \foreignlanguage{greek}{νην παρ αυτω ποιηϲομεθα} & 28 &  &  \\
& \textbf{24} &  & \foreignlanguage{greek}{ο μη αγαπων με τουϲ λογουϲ μου ου τη} & 9 &  &  \\
&  & 9 & \foreignlanguage{greek}{ρει και ο λογοϲ ον ακουεται ουκ εϲτι̅} & 16 &  &  \\
&  & 17 & \foreignlanguage{greek}{εμοϲ αλλα του πεμψαντοϲ με \textoverline{πρϲ}} & 22 &  &  \\
& \textbf{25} &  & \foreignlanguage{greek}{ταυτα λελαληκα υμιν παρ υμιν μενω̅} & 6 &  &  \\
[0.2em]
\cline{4-4}
\end{tabular}
\end{center}
\end{table}
}
\clearpage
\newpage
 {
 \setlength\arrayrulewidth{1pt}
\begin{table}
\begin{center}
\begin{tabular}{ccc|l|ccc}
\cline{4-4} \\ [-1em]
\multicolumn{7}{c}{\foreignlanguage{greek}{ευαγγελιον κατα ιωαννην} \textbf{(\nospace{16:7})} } \\ \\ [-1em] % Si on veut ajouter les bordures latérales, remplacer {7}{c} par {7}{|c|}
\cline{4-4} \\
\cline{4-4}
&  &  & &  &  & \\ [-0.9em]
& \refstepcounter{gospelchapter} \mygospelchapter &  & \foreignlanguage{greek}{εαν δε πορευθω πεμψω αυτον προϲ υμαϲ} & 7 &  &  \\
& \textbf{8} &  & \foreignlanguage{greek}{και ελθων εκεινοϲ ελεγξει τον κοϲμον} & 6 &  &  \\
&  & 7 & \foreignlanguage{greek}{περι αμαρτιαϲ και περι δικαιοϲυνηϲ} & 11 &  &  \\
&  & 12 & \foreignlanguage{greek}{και περι κριϲεωϲ περι αμαρτιαϲ μεν} & 3 & \textbf{9} &  \\
&  & 4 & \foreignlanguage{greek}{οτι ου πιϲτευουϲιν ειϲ εμε περι δικαι} & 2 & \textbf{10} &  \\
&  & 2 & \foreignlanguage{greek}{οϲυνηϲ δε οτι προϲ τον \textoverline{πρα} υπαγω και} & 9 &  &  \\
&  & 10 & \foreignlanguage{greek}{ουκετι θεωριται με περι δε κριϲεωϲ} & 3 & \textbf{11} &  \\
&  & 4 & \foreignlanguage{greek}{οτι ο αρχων του κοϲμου τουτου κεκριτε} & 10 &  &  \\
& \textbf{12} &  & \foreignlanguage{greek}{ετι πολλα εχω λεγειν υμιν αλλ ου δυ} & 8 &  &  \\
&  & 8 & \foreignlanguage{greek}{ναϲθαι βαϲταζειν αρτι οταν ελθη} & 2 & \textbf{13} &  \\
&  & 3 & \foreignlanguage{greek}{εκεινοϲ το \textoverline{πνα} τηϲ αληθειαϲ οδηγηϲει} & 8 &  &  \\
&  & 9 & \foreignlanguage{greek}{υμαϲ εν τη αληθεια παϲη ου γαρ λαλη} & 16 &  &  \\
&  & 16 & \foreignlanguage{greek}{ϲει αφ εαυτου αλλ οϲα ακουϲει λαλη} & 22 &  &  \\
&  & 22 & \foreignlanguage{greek}{ϲει και τα ερχομενα αναγγελει υμιν} & 27 &  &  \\
& \textbf{14} &  & \foreignlanguage{greek}{εκεινοϲ εμε δοξαϲει οτι εκ του εμου} & 7 &  &  \\
&  & 8 & \foreignlanguage{greek}{λημψεται και αναγγελει υμιν παντα} & 1 & \textbf{15} &  \\
&  & 2 & \foreignlanguage{greek}{οϲα εχει ο \textoverline{πηρ} εμα εϲτιν δια τουτο ει} & 10 &  &  \\
&  & 10 & \foreignlanguage{greek}{πον οτι εκ του εμου λαμβανει και α} & 17 &  &  \\
&  & 17 & \foreignlanguage{greek}{ναγγελει υμιν μικρον και ουκετι} & 3 & \textbf{16} &  \\
&  & 4 & \foreignlanguage{greek}{θεωριται με και παλιν μικρον και ο} & 10 &  &  \\
&  & 10 & \foreignlanguage{greek}{ψεϲθαι με ειπαν ουν εκ των μα} & 5 & \textbf{17} &  \\
&  & 5 & \foreignlanguage{greek}{θητων αυτου προϲ αλληλουϲ τι εϲτι̅} & 10 &  &  \\
&  & 11 & \foreignlanguage{greek}{τουτο ο λεγει ημιν μικρον και ουκε} & 17 &  &  \\
&  & 17 & \foreignlanguage{greek}{τι θεωριται με και παλιν μικρον και} & 23 &  &  \\
&  & 24 & \foreignlanguage{greek}{οψεϲθαι με και εγω υπαγω προϲ τον} & 30 &  &  \\
&  & 31 & \foreignlanguage{greek}{\textoverline{πρα} ελεγον ουν τι εϲτιν τουτο} & 5 & \textbf{18} &  \\
&  & 6 & \foreignlanguage{greek}{το μικρον ουκ οιδαμεν τι λαλει} & 11 &  &  \\
& \textbf{19} &  & \foreignlanguage{greek}{εγνοι \textoverline{ιϲ} οτι ημελλον αυτον ερωταν} & 6 &  &  \\
&  & 7 & \foreignlanguage{greek}{και ειπεν αυτοιϲ περι τουτου ζητι} & 12 &  &  \\
&  & 12 & \foreignlanguage{greek}{ται μετ αλληλων οτι ειπον μικρον} & 17 &  &  \\
[0.2em]
\cline{4-4}
\end{tabular}
\end{center}
\end{table}
}
\clearpage
\newpage
 {
 \setlength\arrayrulewidth{1pt}
\begin{table}
\begin{center}
\begin{tabular}{ccc|l|ccc}
\cline{4-4} \\ [-1em]
\multicolumn{7}{c}{\foreignlanguage{greek}{ευαγγελιον κατα ιωαννην} \textbf{(\nospace{16:19})} } \\ \\ [-1em] % Si on veut ajouter les bordures latérales, remplacer {7}{c} par {7}{|c|}
\cline{4-4} \\
\cline{4-4}
&  &  & &  &  & \\ [-0.9em]
&  & 18 & \foreignlanguage{greek}{και ου θεωρειται με και παλιν μικρον} & 24 &  &  \\
&  & 25 & \foreignlanguage{greek}{και οψεϲθαι με αμην αμην λεγω υμι̅} & 4 & \textbf{20} &  \\
&  & 5 & \foreignlanguage{greek}{οτι κλαυϲεται και θρηνηϲεται υμειϲ} & 9 &  &  \\
&  & 10 & \foreignlanguage{greek}{ο δε κοϲμοϲ χαρηϲεται υμειϲ δε λυπη} & 16 &  &  \\
&  & 16 & \foreignlanguage{greek}{θηϲεϲθαι αλλ η λυπη υμων ειϲ χαρα̅} & 22 &  &  \\
&  & 23 & \foreignlanguage{greek}{γενηϲεται} & 23 &  &  \\
& \textbf{21} &  & \foreignlanguage{greek}{η γυνη οταν τικτη λυπην εχει οτι} & 7 &  &  \\
&  & 8 & \foreignlanguage{greek}{ηλθεν η ωρα αυτηϲ οταν δε γεννη} & 14 &  &  \\
&  & 14 & \foreignlanguage{greek}{ϲη το παιδιον ουκετι μνημονευει} & 18 &  &  \\
&  & 19 & \foreignlanguage{greek}{τηϲ θλιψεωϲ δια την χαραν οτι εγε̅} & 25 &  &  \\
&  & 25 & \foreignlanguage{greek}{νηθη \textoverline{ανοϲ} ειϲ τον κοϲμον} & 29 &  &  \\
& \textbf{22} &  & \foreignlanguage{greek}{και υμειϲ ουν νυν μεν λυπην εχεται} & 7 &  &  \\
&  & 9 & \foreignlanguage{greek}{παλιν δε οψομαι υμαϲ και χαρηϲε} & 14 &  &  \\
&  & 14 & \foreignlanguage{greek}{ται υμων η καρδια και την χαραν υ} & 21 &  &  \\
&  & 21 & \foreignlanguage{greek}{μων ουδειϲ αφερει αφ υμων και ε} & 2 & \textbf{23} &  \\
&  & 2 & \foreignlanguage{greek}{κεινη τη ημερα εμε ουκ ερωτηϲεται} & 7 &  &  \\
&  & 8 & \foreignlanguage{greek}{ουδεν αμην αμην λεγω υμιν} & 12 &  &  \\
&  & 13 & \foreignlanguage{greek}{οτι αν αιτηϲηται τον \textoverline{πρα} εν τω ονο} & 20 &  &  \\
&  & 20 & \foreignlanguage{greek}{ματι μου δωϲει υμιν εωϲ αρτι ου} & 3 & \textbf{24} &  \\
&  & 3 & \foreignlanguage{greek}{κ ητηϲατε ουδεν εν τω ονοματι μου} & 9 &  &  \\
&  & 10 & \foreignlanguage{greek}{αιτηϲαϲθαι και ληψεϲθαι ινα η χα} & 15 &  &  \\
&  & 15 & \foreignlanguage{greek}{ρα υμων πεπληρωμενη ην} & 18 &  &  \\
& \textbf{25} &  & \foreignlanguage{greek}{ταυτα εν παροιμιαιϲ λελαληκα υμι̅} & 5 &  &  \\
&  & 6 & \foreignlanguage{greek}{ερχεται ωρα οτε ουκετι εν παροιμι} & 11 &  &  \\
&  & 11 & \foreignlanguage{greek}{αιϲ λαληϲω υμιν αλλα παρρηϲια πε} & 16 &  &  \\
&  & 16 & \foreignlanguage{greek}{ρι του \textoverline{πρϲ} απαγγελω υμιν} & 20 &  &  \\
& \textbf{26} &  & \foreignlanguage{greek}{εν εκεινη τη ημερα αιτηϲαϲθαι εν} & 6 &  &  \\
&  & 7 & \foreignlanguage{greek}{τω ονοματι μου και ου λεγω υμιν} & 13 &  &  \\
&  & 14 & \foreignlanguage{greek}{οτι εγω ερωτηϲω τον \textoverline{πρα} περι υμω̅} & 20 &  &  \\
& \textbf{27} &  & \foreignlanguage{greek}{αυτοϲ γαρ ο \textoverline{πηρ} φιλει υμαϲ οτι υμειϲ} & 8 &  &  \\
[0.2em]
\cline{4-4}
\end{tabular}
\end{center}
\end{table}
}
\clearpage
\newpage
 {
 \setlength\arrayrulewidth{1pt}
\begin{table}
\begin{center}
\begin{tabular}{ccc|l|ccc}
\cline{4-4} \\ [-1em]
\multicolumn{7}{c}{\foreignlanguage{greek}{ευαγγελιον κατα ιωαννην} \textbf{(\nospace{16:27})} } \\ \\ [-1em] % Si on veut ajouter les bordures latérales, remplacer {7}{c} par {7}{|c|}
\cline{4-4} \\
\cline{4-4}
&  &  & &  &  & \\ [-0.9em]
&  & 9 & \foreignlanguage{greek}{εμε πεφιληκατε και πεπιϲτευκατε} & 12 &  &  \\
&  & 13 & \foreignlanguage{greek}{οτι εγω παρα του \textoverline{θυ} εξηλθον και εληλυ} & 2 & \textbf{28} &  \\
&  & 2 & \foreignlanguage{greek}{θα ειϲ τον κοϲμον παλιν αφιημει το̅} & 8 &  &  \\
&  & 9 & \foreignlanguage{greek}{κοϲμον και πορευομαι προϲ τον \textoverline{πρα}} & 14 &  &  \\
& \textbf{29} &  & \foreignlanguage{greek}{λεγουϲιν αυτω οι μαθηται ειδε νυ̅} & 6 &  &  \\
&  & 7 & \foreignlanguage{greek}{εν παρρηϲια λαλειϲ και παροιμιαν ου} & 12 &  &  \\
&  & 12 & \foreignlanguage{greek}{δεμιαν λεγειϲ νυν οιδαμεν οτι} & 3 & \textbf{30} &  \\
&  & 4 & \foreignlanguage{greek}{οιδαϲ παντα και ου χρειαν εχειϲ ινα} & 10 &  &  \\
&  & 11 & \foreignlanguage{greek}{τιϲ ϲε ερωτα εν τουτω πιϲτευομεν} & 16 &  &  \\
&  & 17 & \foreignlanguage{greek}{οτι απο \textoverline{θυ} εξηλθεϲ} & 20 &  &  \\
& \textbf{31} &  & \foreignlanguage{greek}{απεκριθη αυτοιϲ \textoverline{ιϲ} αρτι πιϲτευεται} & 5 &  &  \\
& \textbf{32} &  & \foreignlanguage{greek}{ιδου ερχεται ωρα και εληλυθεν ινα} & 6 &  &  \\
&  & 7 & \foreignlanguage{greek}{ϲκορπιϲθηται εκαϲτοϲ ειϲ τα ιδια} & 11 &  &  \\
&  & 12 & \foreignlanguage{greek}{και εμε μονον αφηται και ουκ ειμει} & 18 &  &  \\
&  & 19 & \foreignlanguage{greek}{μονοϲ οτι ο \textoverline{πηρ} μετ εμου εϲτιν} & 25 &  &  \\
& \textbf{33} &  & \foreignlanguage{greek}{ταυτα λελαληκα υμιν ινα εν εμοι ει} & 7 &  &  \\
&  & 7 & \foreignlanguage{greek}{ρηνην εχηται εν τω κοϲμω θλιψιν} & 12 &  &  \\
&  & 13 & \foreignlanguage{greek}{εχεται αλλα θαρϲειται εγω νενι} & 17 &  &  \\
&  & 17 & \foreignlanguage{greek}{κηκα τον κοϲμον} & 19 &  &  \\
& \mygospelchapter &  & \foreignlanguage{greek}{ταυτα λελαληκεν ο \textoverline{ιϲ} και επαραϲ τουϲ} & 7 &  &  \\
&  & 8 & \foreignlanguage{greek}{οφθαλμουϲ αυτου ειϲ τον ουρανον} & 12 &  &  \\
&  & 13 & \foreignlanguage{greek}{ειπεν \textoverline{περ} εληλυθεν η ωρα δοξα} & 18 &  &  \\
&  & 18 & \foreignlanguage{greek}{ϲον ϲου τον υιον ινα ο υιοϲ δοξαϲη ϲε} & 26 &  &  \\
& \textbf{2} &  & \foreignlanguage{greek}{καθωϲ εδωκαϲ αυτω εξουϲιαν πα} & 5 &  &  \\
&  & 5 & \foreignlanguage{greek}{ϲηϲ ϲαρκοϲ ινα παν ο δεδωκαϲ αυτω} & 11 &  &  \\
&  & 12 & \foreignlanguage{greek}{δωϲ αυτω ζωην αιωνιον} & 15 &  &  \\
& \textbf{3} &  & \foreignlanguage{greek}{αυτη δε εϲτιν η αιωνιοϲ ζωη ινα γι} & 8 &  &  \\
&  & 8 & \foreignlanguage{greek}{νωϲκουϲιν τον μονον αληθεινον} & 11 &  &  \\
&  & 12 & \foreignlanguage{greek}{\textoverline{θν} και ον απεϲτιλεν \textoverline{ιν} \textoverline{χν}} & 17 &  &  \\
& \textbf{4} &  & \foreignlanguage{greek}{εγω ϲε εδοξαϲα επι τηϲ γηϲ το εργον} & 8 &  &  \\
[0.2em]
\cline{4-4}
\end{tabular}
\end{center}
\end{table}
}
\clearpage
\newpage
 {
 \setlength\arrayrulewidth{1pt}
\begin{table}
\begin{center}
\begin{tabular}{ccc|l|ccc}
\cline{4-4} \\ [-1em]
\multicolumn{7}{c}{\foreignlanguage{greek}{ευαγγελιον κατα ιωαννην} \textbf{(\nospace{17:4})} } \\ \\ [-1em] % Si on veut ajouter les bordures latérales, remplacer {7}{c} par {7}{|c|}
\cline{4-4} \\
\cline{4-4}
&  &  & &  &  & \\ [-0.9em]
&  & 9 & \foreignlanguage{greek}{ϲου τελιωϲαϲ ο εδωκαϲ μοι ινα ποιηϲω} & 15 &  &  \\
& \textbf{5} &  & \foreignlanguage{greek}{και νυν δοξαϲον με ϲυ \textoverline{περ} παρα ϲεαυτω} & 8 &  &  \\
&  & 9 & \foreignlanguage{greek}{τη δοξη η ειχον προ του τον κοϲμον ει} & 17 &  &  \\
&  & 17 & \foreignlanguage{greek}{ναι παρα ϲοι εφανερωϲα ϲου το ονομα} & 4 & \textbf{6} &  \\
&  & 5 & \foreignlanguage{greek}{τοιϲ \textoverline{ανοιϲ} ουϲ εδωκαϲ μοι εκ του κοϲ} & 12 &  &  \\
&  & 12 & \foreignlanguage{greek}{μου ϲου ηϲαν και εμοι αυτουϲ εδωκαϲ} & 18 &  &  \\
&  & 19 & \foreignlanguage{greek}{και τον λογον ϲου τετηρηκαν} & 23 &  &  \\
& \textbf{7} &  & \foreignlanguage{greek}{νυν εγνωκα οτι παντα οϲα δεδω} & 6 &  &  \\
&  & 6 & \foreignlanguage{greek}{καϲ μοι παρα ϲου ειϲιν οτι τα ρημα} & 3 & \textbf{8} &  \\
&  & 3 & \foreignlanguage{greek}{τα α εδωκαϲ μοι εδωκα αυτοιϲ} & 8 &  &  \\
&  & 9 & \foreignlanguage{greek}{και αυτο ελαβον αληθωϲ οτι παρα ϲου} & 15 &  &  \\
&  & 16 & \foreignlanguage{greek}{εξηλθον και επιϲτευϲαν οτι ϲυ με} & 21 &  &  \\
&  & 22 & \foreignlanguage{greek}{απεϲτιλαϲ εγω περι αυτων ερω} & 4 & \textbf{9} &  \\
&  & 4 & \foreignlanguage{greek}{τω ου περι του κοϲμου ερωτω αλλα} & 10 &  &  \\
&  & 11 & \foreignlanguage{greek}{περι ων εδωκαϲ μοι οτι ϲοι ειϲιν και} & 1 & \textbf{10} &  \\
&  & 2 & \foreignlanguage{greek}{τα εμα παντα ϲα εϲτιν και τα ϲα εμα} & 10 &  &  \\
&  & 11 & \foreignlanguage{greek}{και δεδοξαϲμαι εν αυτοιϲ και ουκε} & 2 & \textbf{11} &  \\
&  & 2 & \foreignlanguage{greek}{τι ειμει εν τω κοϲμω και ουτοι εν} & 9 &  &  \\
&  & 10 & \foreignlanguage{greek}{τω κοϲμω ειϲιν και εγω προϲ ϲε ερ} & 17 &  &  \\
&  & 17 & \foreignlanguage{greek}{χομαι} & 17 &  &  \\
&  & 18 & \foreignlanguage{greek}{πατερ αγιε τηρηϲον αυτουϲ εν τω} & 23 &  &  \\
&  & 24 & \foreignlanguage{greek}{ονοματι ϲου ω εδωκαϲ μοι ινα ωϲιν} & 30 &  &  \\
&  & 31 & \foreignlanguage{greek}{εν καθωϲ ημειϲ οτε ημην μετ αυ} & 4 & \textbf{12} &  \\
&  & 4 & \foreignlanguage{greek}{των εγω ετηρουν αυτουϲ εν τω ο} & 10 &  &  \\
&  & 10 & \foreignlanguage{greek}{νοματι ϲου ω εδωκαϲ μοι και εφυ} & 16 &  &  \\
&  & 16 & \foreignlanguage{greek}{λαξα και ουδειϲ εξ αυτων απωλετο} & 21 &  &  \\
&  & 22 & \foreignlanguage{greek}{ει μη ο υιοϲ τηϲ απωλειαϲ ινα η γρα} & 30 &  &  \\
&  & 30 & \foreignlanguage{greek}{φη πληρωθη νυν δε προϲ ϲε ερχο} & 5 & \textbf{13} &  \\
&  & 5 & \foreignlanguage{greek}{μαι και ταυτα λαλω εν τω κοϲμω} & 11 &  &  \\
&  & 12 & \foreignlanguage{greek}{ινα εχωϲιν την χαραν την εμην πε} & 18 &  &  \\
[0.2em]
\cline{4-4}
\end{tabular}
\end{center}
\end{table}
}
\clearpage
\newpage
 {
 \setlength\arrayrulewidth{1pt}
\begin{table}
\begin{center}
\begin{tabular}{ccc|l|ccc}
\cline{4-4} \\ [-1em]
\multicolumn{7}{c}{\foreignlanguage{greek}{ευαγγελιον κατα ιωαννην} \textbf{(\nospace{17:13})} } \\ \\ [-1em] % Si on veut ajouter les bordures latérales, remplacer {7}{c} par {7}{|c|}
\cline{4-4} \\
\cline{4-4}
&  &  & &  &  & \\ [-0.9em]
&  & 18 & \foreignlanguage{greek}{πληρωμενην εν εαυτοιϲ} & 20 &  &  \\
& \textbf{14} &  & \foreignlanguage{greek}{εγω εδωκα αυτοιϲ τον λογον ϲου και ο} & 8 &  &  \\
&  & 9 & \foreignlanguage{greek}{κοϲμοϲ εμειϲηϲεν αυτουϲ οτι ουκ ειϲι̅} & 14 &  &  \\
&  & 15 & \foreignlanguage{greek}{εκ του κοϲμου καθωϲ εγω ουκ ιμει εκ} & 22 &  &  \\
&  & 23 & \foreignlanguage{greek}{του κοϲμου ουκ ερωτω ινα αρηϲ αυ} & 5 & \textbf{15} &  \\
&  & 5 & \foreignlanguage{greek}{τουϲ εκ του κοϲμου αλλ ινα τηρηϲηϲ} & 11 &  &  \\
&  & 12 & \foreignlanguage{greek}{αυτουϲ εκ του πονηρου} & 15 &  &  \\
& \textbf{16} &  & \foreignlanguage{greek}{εκ του κοϲμου ουκ ειϲιν καθωϲ εγω} & 7 &  &  \\
&  & 8 & \foreignlanguage{greek}{ουκ ειμει εκ του κοϲμου αγιαϲον αυ} & 2 & \textbf{17} &  \\
&  & 2 & \foreignlanguage{greek}{τουϲ εν τη αληθεια} & 5 &  &  \\
&  & 6 & \foreignlanguage{greek}{ο λογοϲ ο ϲοϲ η αληθεια εϲτιν καθωϲ} & 1 & \textbf{18} &  \\
&  & 2 & \foreignlanguage{greek}{εμε απεϲτιλαϲ ειϲ τον κοϲμον καγω} & 7 &  &  \\
&  & 8 & \foreignlanguage{greek}{απεϲτιλα αυτουϲ ειϲ τον κοϲμον} & 12 &  &  \\
& \textbf{19} &  & \foreignlanguage{greek}{και υπερ αυτων αγιαζω εμαυτον ινα} & 6 &  &  \\
&  & 7 & \foreignlanguage{greek}{ωϲιν και αυτοι ηγιαϲμενοι εν αλη} & 12 &  &  \\
&  & 12 & \foreignlanguage{greek}{θεια ου περι τουτων δε μονων ε} & 6 & \textbf{20} &  \\
&  & 6 & \foreignlanguage{greek}{ρωτω αλλα και υπερ των πιϲτευϲον} & 11 &  &  \\
&  & 11 & \foreignlanguage{greek}{των δια του λογου αυτων ειϲ εμε} & 17 &  &  \\
& \textbf{21} &  & \foreignlanguage{greek}{ινα παντεϲ εν ωϲιν καθωϲ ϲυ \textoverline{πηρ}} & 7 &  &  \\
&  & 8 & \foreignlanguage{greek}{εν εμοι καγω εν ϲοι ινα και αυτοι} & 15 &  &  \\
&  & 16 & \foreignlanguage{greek}{εν ημιν ωϲιν ινα ο κοϲμοϲ πιϲτευη} & 22 &  &  \\
&  & 23 & \foreignlanguage{greek}{οτι ϲυ με απεϲτιλαϲ καγω την δο} & 3 & \textbf{22} &  \\
&  & 3 & \foreignlanguage{greek}{ξαν μου ην εδωκαϲ μοι δεδωκα αυ} & 10 &  &  \\
&  & 10 & \foreignlanguage{greek}{τοιϲ ινα ωϲιν εν καθωϲ ημειϲ εν} & 16 &  &  \\
& \textbf{23} &  & \foreignlanguage{greek}{εγω εν αυτοιϲ και ϲυ εν εμοι ινα ω} & 9 &  &  \\
&  & 9 & \foreignlanguage{greek}{ϲιν τετελιωμενοι ειϲ εν και γινω} & 14 &  &  \\
&  & 14 & \foreignlanguage{greek}{ϲκη ο κοϲμοϲ οτι ϲυ με απεϲτιλαϲ} & 20 &  &  \\
&  & 21 & \foreignlanguage{greek}{και ηγαπηϲαϲ αυτουϲ καθωϲ καμε η} & 26 &  &  \\
&  & 26 & \foreignlanguage{greek}{γαπηϲαϲ} & 26 &  &  \\
& \textbf{24} &  & \foreignlanguage{greek}{πατερ ο δεδωκαϲ μοι θελω ινα οπου} & 7 &  &  \\
[0.2em]
\cline{4-4}
\end{tabular}
\end{center}
\end{table}
}
\clearpage
\newpage
 {
 \setlength\arrayrulewidth{1pt}
\begin{table}
\begin{center}
\begin{tabular}{ccc|l|ccc}
\cline{4-4} \\ [-1em]
\multicolumn{7}{c}{\foreignlanguage{greek}{ευαγγελιον κατα ιωαννην} \textbf{(\nospace{17:24})} } \\ \\ [-1em] % Si on veut ajouter les bordures latérales, remplacer {7}{c} par {7}{|c|}
\cline{4-4} \\
\cline{4-4}
&  &  & &  &  & \\ [-0.9em]
&  & 8 & \foreignlanguage{greek}{ειμει εγω και εκεινοι ωϲιν μετ εμου} & 14 &  &  \\
&  & 15 & \foreignlanguage{greek}{ινα θεωρωϲιν την δοξαν την εμην η̅} & 22 &  &  \\
&  & 23 & \foreignlanguage{greek}{δεδωκαϲ μοι οτι ηγαπηϲαϲ με προ κα} & 29 &  &  \\
&  & 29 & \foreignlanguage{greek}{ταβοληϲ κοϲμου} & 30 &  &  \\
& \textbf{25} &  & \foreignlanguage{greek}{πατερ δικαιε και ο κοϲμοϲ ϲε ουκ εγνω} & 8 &  &  \\
&  & 9 & \foreignlanguage{greek}{εγω δε εγνων ϲε και ουτοι εγνωϲαν οτι} & 16 &  &  \\
&  & 17 & \foreignlanguage{greek}{ϲυ με απεϲτιλαϲ και εγνωριϲα αυτοιϲ} & 3 & \textbf{26} &  \\
&  & 4 & \foreignlanguage{greek}{το ονομα ϲου και γνωριϲω ινα η αγα} & 11 &  &  \\
&  & 11 & \foreignlanguage{greek}{πη ην ηγαπηϲαϲ με εν αυτοιϲ η και ε} & 19 &  &  \\
&  & 19 & \foreignlanguage{greek}{γω εν αυτοιϲ} & 21 &  &  \\
& \mygospelchapter &  & \foreignlanguage{greek}{ταυτα ειπων ο \textoverline{ιϲ} εξηλθεν ϲυν τοιϲ μα} & 8 &  &  \\
&  & 8 & \foreignlanguage{greek}{θηταιϲ αυτου περαν του χειμαρρου} & 12 &  &  \\
&  & 13 & \foreignlanguage{greek}{του κεδρου οπου ην κηποϲ ειϲ ον ειϲ} & 20 &  &  \\
&  & 20 & \foreignlanguage{greek}{εληλυθεν αυτοϲ και οι μαθηται αυτου} & 25 &  &  \\
& \textbf{2} &  & \foreignlanguage{greek}{ηδει δε και ιουδαϲ ο παραδιδουϲ αυτο̅} & 7 &  &  \\
&  & 8 & \foreignlanguage{greek}{τον τοπον οτι πολλακειϲ ϲυνηχθη} & 12 &  &  \\
&  & 13 & \foreignlanguage{greek}{ο \textoverline{ιϲ} εκει μετα των μαθητων αυτου} & 19 &  &  \\
& \textbf{3} &  & \foreignlanguage{greek}{ο ουν ιουδαϲ λαβων την ϲπειραν και} & 7 &  &  \\
&  & 8 & \foreignlanguage{greek}{εκ των αρχιερεων και φαριϲαιων υ} & 13 &  &  \\
&  & 13 & \foreignlanguage{greek}{πηρεταϲ ερχεται εκει μετα φανω̅} & 17 &  &  \\
&  & 18 & \foreignlanguage{greek}{και λαμπαδων και οπλων} & 21 &  &  \\
& \textbf{4} &  & \foreignlanguage{greek}{\textoverline{ιϲ} δε ιδωϲ παντα τα ερχομενα επ αυ} & 8 &  &  \\
&  & 8 & \foreignlanguage{greek}{τον εξελθων ειπεν αυτοιϲ τινα ζη} & 13 &  &  \\
&  & 13 & \foreignlanguage{greek}{τειτε απεκριθηϲαν αυτω \textoverline{ιν} τον να} & 5 & \textbf{5} &  \\
&  & 5 & \foreignlanguage{greek}{ζωραιον λεγει αυτοιϲ ο \textoverline{ιϲ} εγω} & 10 &  &  \\
&  & 11 & \foreignlanguage{greek}{ειμει ειϲτηκει δε και ιουδαϲ ο παρα} & 17 &  &  \\
&  & 17 & \foreignlanguage{greek}{διδουϲ αυτον μετ αυτων} & 20 &  &  \\
& \textbf{6} &  & \foreignlanguage{greek}{ωϲ ουν ειπεν αυτοιϲ εγω ειμει απηλ} & 7 &  &  \\
&  & 7 & \foreignlanguage{greek}{θαν ειϲ τα οπιϲω και επεϲαν χαμε} & 13 &  &  \\
& \textbf{7} &  & \foreignlanguage{greek}{παλιν ουν αυτουϲ επηρωτηϲεν τι} & 5 &  &  \\
[0.2em]
\cline{4-4}
\end{tabular}
\end{center}
\end{table}
}
\clearpage
\newpage
 {
 \setlength\arrayrulewidth{1pt}
\begin{table}
\begin{center}
\begin{tabular}{ccc|l|ccc}
\cline{4-4} \\ [-1em]
\multicolumn{7}{c}{\foreignlanguage{greek}{ευαγγελιον κατα ιωαννην} \textbf{(\nospace{18:7})} } \\ \\ [-1em] % Si on veut ajouter les bordures latérales, remplacer {7}{c} par {7}{|c|}
\cline{4-4} \\
\cline{4-4}
&  &  & &  &  & \\ [-0.9em]
&  & 5 & \foreignlanguage{greek}{να ζητειται οι δε ειπον \textoverline{ιν} τον ναζωραιο̅} & 12 &  &  \\
& \textbf{8} &  & \foreignlanguage{greek}{απεκριθη \textoverline{ιϲ} ειπον υμιν οτι εγω ειμει} & 7 &  &  \\
&  & 8 & \foreignlanguage{greek}{ει ουν εμε ζητειται αφεται τουτουϲ υπα} & 14 &  &  \\
&  & 14 & \foreignlanguage{greek}{γειν ινα πληρωθη ο λογοϲ ον ειπεν οτι} & 7 & \textbf{9} &  \\
&  & 8 & \foreignlanguage{greek}{ουϲ δεδωκαϲ μοι ουκ απωλεϲα εξ αυτω̅} & 14 &  &  \\
&  & 15 & \foreignlanguage{greek}{ουδενα ϲιμων ουν πετροϲ εχων} & 4 & \textbf{10} &  \\
&  & 5 & \foreignlanguage{greek}{μαχαιραν ειλκυϲεν αυτην και επε} & 9 &  &  \\
&  & 9 & \foreignlanguage{greek}{ϲεν τον του αρχιερεωϲ δουλον και α} & 15 &  &  \\
&  & 15 & \foreignlanguage{greek}{πεκοψεν αυτου το ωταριον το δεξιον} & 20 &  &  \\
&  & 21 & \foreignlanguage{greek}{ην δε ονομα τω δουλω μαλχοϲ} & 26 &  &  \\
& \textbf{11} &  & \foreignlanguage{greek}{ειπεν ουν ο \textoverline{ιϲ} τω πετρω βαλε την μα} & 9 &  &  \\
&  & 9 & \foreignlanguage{greek}{χαιραν ειϲ την θηκην το ποτηριον} & 14 &  &  \\
&  & 15 & \foreignlanguage{greek}{ο δεδωκεν μοι ο \textoverline{πηρ} ου μη πιω αυτο} & 23 &  &  \\
& \textbf{12} &  & \foreignlanguage{greek}{η ουν ϲπειρα και ο χειλιαρχοϲ και οι υ} & 9 &  &  \\
&  & 9 & \foreignlanguage{greek}{πηρεται των ιουδαιων ϲυνελαβον} & 12 &  &  \\
&  & 13 & \foreignlanguage{greek}{τον \textoverline{ιν} και εδηϲαν αυτον και ηγαγο̅} & 2 & \textbf{13} &  \\
&  & 3 & \foreignlanguage{greek}{προϲ ανναν πρωτον ην γαρ πενθε} & 8 &  &  \\
&  & 8 & \foreignlanguage{greek}{ροϲ του καιαφα οϲ ην αρχιερευϲ του} & 14 &  &  \\
&  & 15 & \foreignlanguage{greek}{ενιαυτου εκεινου} & 16 &  &  \\
& \textbf{14} &  & \foreignlanguage{greek}{ην δε καιαφαϲ ο ϲυνβουλευϲαϲ τοιϲ} & 6 &  &  \\
&  & 7 & \foreignlanguage{greek}{ιουδαιοιϲ οτι ϲυμφερι ενα \textoverline{ανον} απο} & 12 &  &  \\
&  & 12 & \foreignlanguage{greek}{θανειν υπερ του λαου} & 15 &  &  \\
& \textbf{15} &  & \foreignlanguage{greek}{ηκολουθει δε τω \textoverline{ιυ} ϲιμων πετροϲ και} & 7 &  &  \\
&  & 8 & \foreignlanguage{greek}{αλλοϲ μαθητηϲ ο δε μαθητηϲ εκει} & 13 &  &  \\
&  & 13 & \foreignlanguage{greek}{νοϲ γνωϲτοϲ ην τω αρχιερει και ϲυν} & 19 &  &  \\
&  & 19 & \foreignlanguage{greek}{ειϲηλθεν τω \textoverline{ιυ} ειϲ την αυλην του αρ} & 26 &  &  \\
&  & 26 & \foreignlanguage{greek}{χιερεωϲ ο δε πετροϲ ιϲτηκει προϲ} & 5 & \textbf{16} &  \\
&  & 6 & \foreignlanguage{greek}{τη θυρα εξω εξηλθεν ουν ο μαθη} & 12 &  &  \\
&  & 12 & \foreignlanguage{greek}{τηϲ ο αλλοϲ οϲ ην γνωϲτοϲ τω αρχιερει} & 19 &  &  \\
&  & 20 & \foreignlanguage{greek}{και ειπεν τω θυρωρω και ειϲηνεγκε̅} & 25 &  &  \\
[0.2em]
\cline{4-4}
\end{tabular}
\end{center}
\end{table}
}
\clearpage
\newpage
 {
 \setlength\arrayrulewidth{1pt}
\begin{table}
\begin{center}
\begin{tabular}{ccc|l|ccc}
\cline{4-4} \\ [-1em]
\multicolumn{7}{c}{\foreignlanguage{greek}{ευαγγελιον κατα ιωαννην} \textbf{(\nospace{18:16})} } \\ \\ [-1em] % Si on veut ajouter les bordures latérales, remplacer {7}{c} par {7}{|c|}
\cline{4-4} \\
\cline{4-4}
&  &  & &  &  & \\ [-0.9em]
&  & 26 & \foreignlanguage{greek}{τον πετρον λεγει ουν αυτω η παιδιϲκη} & 5 & \textbf{17} &  \\
&  & 6 & \foreignlanguage{greek}{η θυρωροϲ τω πετρω μη και ϲυ εκ των} & 14 &  &  \\
&  & 15 & \foreignlanguage{greek}{μαθητων ει του \textoverline{ανου} τουτου λεγει ε} & 21 &  &  \\
&  & 21 & \foreignlanguage{greek}{κεινοϲ ουκ ειμει ιϲτηκειϲαν δε} & 2 & \textbf{18} &  \\
&  & 3 & \foreignlanguage{greek}{οι δουλοι και οι υπηρεται ανθρακιαν} & 8 &  &  \\
&  & 9 & \foreignlanguage{greek}{πεποιηκοτεϲ οτι ψυχοϲ ην και εθερμε} & 14 &  &  \\
&  & 14 & \foreignlanguage{greek}{νοντο ην δε και πετροϲ μετ αυτων} & 20 &  &  \\
&  & 21 & \foreignlanguage{greek}{εϲτωϲ και θερμενομενοϲ} & 23 &  &  \\
& \textbf{19} &  & \foreignlanguage{greek}{ο ουν αρχιερευϲ ηρωτηϲεν τον \textoverline{ιν} πε} & 7 &  &  \\
&  & 7 & \foreignlanguage{greek}{ρι των μαθητων αυτου και περι} & 12 &  &  \\
&  & 13 & \foreignlanguage{greek}{τηϲ διδαχηϲ αυτου} & 15 &  &  \\
& \textbf{20} &  & \foreignlanguage{greek}{απεκριθη αυτω ο \textoverline{ιϲ} εγω παρρηϲια ελα} & 7 &  &  \\
&  & 7 & \foreignlanguage{greek}{ληϲα τω κοϲμω εγω παντοτε εδιδα} & 12 &  &  \\
&  & 12 & \foreignlanguage{greek}{ξα εν ϲυναγωγη και εν τω ιερω οπου} & 19 &  &  \\
&  & 20 & \foreignlanguage{greek}{παντεϲ οι ιουδαιοι ϲυνερχονται} & 23 &  &  \\
&  & 24 & \foreignlanguage{greek}{και εν κρυπτω ελαληϲα ουδεν τι με ε} & 3 & \textbf{21} &  \\
&  & 3 & \foreignlanguage{greek}{ρωταϲ ερωτηϲον τουϲ ακηκοοταϲ} & 6 &  &  \\
&  & 7 & \foreignlanguage{greek}{τι ελαληϲα αυτοιϲ ειδε ουτοι οιδα} & 12 &  &  \\
&  & 12 & \foreignlanguage{greek}{ϲιν α ειπον εγω} & 15 &  &  \\
& \textbf{22} &  & \foreignlanguage{greek}{ταυτα δε αυτου ειποντοϲ ειϲ παρε} & 6 &  &  \\
&  & 6 & \foreignlanguage{greek}{ϲτηκωϲ των υπηρετων εδωκεν} & 9 &  &  \\
&  & 10 & \foreignlanguage{greek}{ραπιϲμα τω \textoverline{ιυ} ειπων ουτωϲ απο} & 15 &  &  \\
&  & 15 & \foreignlanguage{greek}{κρινη τω αρχιερει ο δε \textoverline{ιϲ} ειπεν} & 4 & \textbf{23} &  \\
&  & 5 & \foreignlanguage{greek}{αυτω ει κακωϲ ειπον μαρτυρηϲον} & 9 &  &  \\
&  & 10 & \foreignlanguage{greek}{περι του κακου ει δε καλωϲ τι με δε} & 18 &  &  \\
&  & 18 & \foreignlanguage{greek}{ρειϲ απεϲτιλεν ουν αυτον ο αν} & 5 & \textbf{24} &  \\
&  & 5 & \foreignlanguage{greek}{ναϲ δεδεμενον προϲ καιαφαν το̅} & 9 &  &  \\
&  & 10 & \foreignlanguage{greek}{αρχιερεα ην δε ϲιμων πετροϲ} & 4 & \textbf{25} &  \\
&  & 5 & \foreignlanguage{greek}{εϲτωϲ και θερμενομενοϲ} & 7 &  &  \\
&  & 8 & \foreignlanguage{greek}{ειπον ουν αυτω μη και ϲυ εκ των} & 15 &  &  \\
[0.2em]
\cline{4-4}
\end{tabular}
\end{center}
\end{table}
}
\clearpage
\newpage
 {
 \setlength\arrayrulewidth{1pt}
\begin{table}
\begin{center}
\begin{tabular}{ccc|l|ccc}
\cline{4-4} \\ [-1em]
\multicolumn{7}{c}{\foreignlanguage{greek}{ευαγγελιον κατα ιωαννην} \textbf{(\nospace{18:25})} } \\ \\ [-1em] % Si on veut ajouter les bordures latérales, remplacer {7}{c} par {7}{|c|}
\cline{4-4} \\
\cline{4-4}
&  &  & &  &  & \\ [-0.9em]
&  & 16 & \foreignlanguage{greek}{μαθητων αυτου ει ηρνηϲατο εκει} & 20 &  &  \\
&  & 20 & \foreignlanguage{greek}{νοϲ και ειπεν ουκ ειμει} & 24 &  &  \\
& \textbf{26} &  & \foreignlanguage{greek}{λεγει ειϲ εκ των δουλων του αρχιερε} & 7 &  &  \\
&  & 7 & \foreignlanguage{greek}{ωϲ ϲυγγενηϲ ων ου απεκοψεν πετροϲ} & 12 &  &  \\
&  & 13 & \foreignlanguage{greek}{το ωτιον ουκ εγω ϲε ειδον εν τω κη} & 21 &  &  \\
&  & 21 & \foreignlanguage{greek}{πω μετ αυτου παλιν ουν ηρνηϲατο} & 3 & \textbf{27} &  \\
&  & 4 & \foreignlanguage{greek}{πετροϲ και ευθυϲ αλεκτωρ εφωνηϲε̅} & 8 &  &  \\
& \textbf{28} &  & \foreignlanguage{greek}{αγουϲιν ουν τον \textoverline{ιν} απο του καιαφα} & 7 &  &  \\
&  & 8 & \foreignlanguage{greek}{ειϲ το πρετωριον ην δε πρωει και αυ} & 15 &  &  \\
&  & 15 & \foreignlanguage{greek}{τοι ουκ ειϲηλθον ειϲ το πρετωριον ινα} & 21 &  &  \\
&  & 22 & \foreignlanguage{greek}{μη μιανθωϲιν αλλα φαγωϲιν το παϲχα} & 27 &  &  \\
& \textbf{29} &  & \foreignlanguage{greek}{εξηλθεν ουν προϲ αυτουϲ ο πειλατοϲ} & 6 &  &  \\
&  & 7 & \foreignlanguage{greek}{εξω και φηϲιν τινα κατηγοριαν φε} & 12 &  &  \\
&  & 12 & \foreignlanguage{greek}{ρεται κατα του \textoverline{ανου} τουτου} & 16 &  &  \\
& \textbf{30} &  & \foreignlanguage{greek}{απεκριθηϲαν και ειπον αυτω ει μη} & 6 &  &  \\
&  & 7 & \foreignlanguage{greek}{ην ουτοϲ κακον ποιων ουκ αν ϲοι πα} & 14 &  &  \\
&  & 14 & \foreignlanguage{greek}{ραδεδωκειμεν αυτον} & 15 &  &  \\
& \textbf{31} &  & \foreignlanguage{greek}{ειπεν ουν αυτοιϲ ο πειλατοϲ λαβεται} & 6 &  &  \\
&  & 7 & \foreignlanguage{greek}{αυτον υμειϲ και κατα τον νομον υ} & 13 &  &  \\
&  & 13 & \foreignlanguage{greek}{μων κρεινατε ειπον ουν αυτω} & 17 &  &  \\
&  & 18 & \foreignlanguage{greek}{οι ιουδαιοι ημιν ουκ εξεϲτιν απο} & 23 &  &  \\
&  & 23 & \foreignlanguage{greek}{κτιναι ουδενα ινα πληρωθη ο λο} & 4 & \textbf{32} &  \\
&  & 4 & \foreignlanguage{greek}{γοϲ του \textoverline{ιυ} ον ειπεν ϲημαινων ποι} & 10 &  &  \\
&  & 10 & \foreignlanguage{greek}{ω θανατω ημελλεν αποθνηϲκειν} & 13 &  &  \\
& \textbf{33} &  & \foreignlanguage{greek}{ειϲηλθεν ουν παλιν ειϲ το πρετωριο̅} & 6 &  &  \\
&  & 7 & \foreignlanguage{greek}{ο πειλατοϲ και εφωνηϲεν τον \textoverline{ιν} και} & 13 &  &  \\
&  & 14 & \foreignlanguage{greek}{ειπεν αυτω ϲυ ει ο βαϲιλευϲ των ιου} & 21 &  &  \\
&  & 21 & \foreignlanguage{greek}{δαιων και απεκρινατο ο \textoverline{ιϲ} αφ εαυ} & 6 & \textbf{34} &  \\
&  & 6 & \foreignlanguage{greek}{του ϲυ τουτο λεγειϲ η αλλοι ειπον ϲοι} & 13 &  &  \\
&  & 14 & \foreignlanguage{greek}{περι εμου} & 15 &  &  \\
[0.2em]
\cline{4-4}
\end{tabular}
\end{center}
\end{table}
}
\clearpage
\newpage
 {
 \setlength\arrayrulewidth{1pt}
\begin{table}
\begin{center}
\begin{tabular}{ccc|l|ccc}
\cline{4-4} \\ [-1em]
\multicolumn{7}{c}{\foreignlanguage{greek}{ευαγγελιον κατα ιωαννην} \textbf{(\nospace{18:35})} } \\ \\ [-1em] % Si on veut ajouter les bordures latérales, remplacer {7}{c} par {7}{|c|}
\cline{4-4} \\
\cline{4-4}
&  &  & &  &  & \\ [-0.9em]
& \textbf{35} &  & \foreignlanguage{greek}{απεκριθη ο πειλατοϲ μη εγω ιουδαιοϲ} & 6 &  &  \\
&  & 7 & \foreignlanguage{greek}{ειμει το εθνοϲ το ϲον και οι αρχιερειϲ} & 14 &  &  \\
&  & 15 & \foreignlanguage{greek}{παρεδωκαν ϲε εμοι τι εποιηϲαϲ} & 19 &  &  \\
& \textbf{36} &  & \foreignlanguage{greek}{απεκριθη \textoverline{ιϲ} η βαϲιλεια η εμη ουκ εϲτι̅} & 8 &  &  \\
&  & 9 & \foreignlanguage{greek}{εκ του κοϲμου τουτου ει ην εκ του} & 16 &  &  \\
&  & 17 & \foreignlanguage{greek}{κοϲμου τουτου η βαϲιλεια η εμη} & 22 &  &  \\
&  & 23 & \foreignlanguage{greek}{οι υπηρεται οι εμοι ηγωνιζοντο αν} & 28 &  &  \\
&  & 29 & \foreignlanguage{greek}{ινα μη παραδοθω τοιϲ ιουδαιοιϲ} & 33 &  &  \\
&  & 34 & \foreignlanguage{greek}{νυν δε η βαϲιλεια η εμη ουκ εϲτιν ε̅} & 42 &  &  \\
&  & 42 & \foreignlanguage{greek}{τευθεν ειπεν ουν αυτω ο πει} & 5 & \textbf{37} &  \\
&  & 5 & \foreignlanguage{greek}{λατοϲ ουκουν βαϲιλευϲ ει ϲυ} & 9 &  &  \\
&  & 10 & \foreignlanguage{greek}{απεκριθη \textoverline{ιϲ} ϲυ λεγειϲ οτι βαϲιλευϲ} & 15 &  &  \\
&  & 16 & \foreignlanguage{greek}{ειμει εγω ειϲ τουτο γεγενημαι και} & 21 &  &  \\
&  & 22 & \foreignlanguage{greek}{ειϲ τουτο εληλυθα ειϲ τον κοϲμον} & 27 &  &  \\
&  & 28 & \foreignlanguage{greek}{ινα μαρτυρηϲω τη αληθεια} & 31 &  &  \\
&  & 32 & \foreignlanguage{greek}{παϲ ο ων εκ τηϲ αληθειαϲ ακουει μου} & 39 &  &  \\
&  & 40 & \foreignlanguage{greek}{τηϲ φωνηϲ} & 41 &  &  \\
& \textbf{38} &  & \foreignlanguage{greek}{λεγει αυτω ο πειλατοϲ τι εϲτιν αληθει} & 7 &  &  \\
&  & 7 & \foreignlanguage{greek}{α και τουτο ειπων παλιν εξηλθεν} & 12 &  &  \\
&  & 13 & \foreignlanguage{greek}{προϲ τουϲ ιουδαιουϲ και λεγει αυτοιϲ} & 18 &  &  \\
&  & 19 & \foreignlanguage{greek}{εγω ουδεμιαν αιτιαν ευριϲκω εν αυ} & 24 &  &  \\
&  & 24 & \foreignlanguage{greek}{τω εϲτιν δε ϲυνηθεια υμιν ινα ε} & 6 & \textbf{39} &  \\
&  & 6 & \foreignlanguage{greek}{να απολυω υμιν εν τω παϲχα} & 11 &  &  \\
&  & 12 & \foreignlanguage{greek}{βουλεϲθαι ουν ινα απολυϲω υμιν} & 16 &  &  \\
&  & 17 & \foreignlanguage{greek}{τον βαϲιλεα των ιουδαιων} & 20 &  &  \\
& \textbf{40} &  & \foreignlanguage{greek}{εκραυγαϲαν ουν παλιν λεγοντεϲ} & 4 &  &  \\
&  & 5 & \foreignlanguage{greek}{μη τουτον αλλα τον βαραββαν} & 9 &  &  \\
&  & 11 & \foreignlanguage{greek}{ην δε ο βαραββαϲ ληϲτηϲ} & 15 &  &  \\
& \mygospelchapter &  & \foreignlanguage{greek}{τοτε ουν λαβων ο πειλατοϲ τον \textoverline{ιν}} & 7 &  &  \\
&  & 8 & \foreignlanguage{greek}{εμαϲτιγωϲεν και οι ϲτρατιωται} & 3 & \textbf{2} &  \\
[0.2em]
\cline{4-4}
\end{tabular}
\end{center}
\end{table}
}
\clearpage
\newpage
 {
 \setlength\arrayrulewidth{1pt}
\begin{table}
\begin{center}
\begin{tabular}{ccc|l|ccc}
\cline{4-4} \\ [-1em]
\multicolumn{7}{c}{\foreignlanguage{greek}{ευαγγελιον κατα ιωαννην} \textbf{(\nospace{19:2})} } \\ \\ [-1em] % Si on veut ajouter les bordures latérales, remplacer {7}{c} par {7}{|c|}
\cline{4-4} \\
\cline{4-4}
&  &  & &  &  & \\ [-0.9em]
&  & 4 & \foreignlanguage{greek}{πλεξαντεϲ ϲτεφανον εξ ακανθων} & 7 &  &  \\
&  & 8 & \foreignlanguage{greek}{επεθηκαν αυτου τη κεφαλη και ιμα} & 13 &  &  \\
&  & 13 & \foreignlanguage{greek}{τιον πορφυρουν περιεβαλον αυτον} & 16 &  &  \\
& \textbf{3} &  & \foreignlanguage{greek}{και ηρχοντο προϲ αυτον και ελεγον χαι} & 7 &  &  \\
&  & 7 & \foreignlanguage{greek}{ρε ο βαϲιλευϲ των ιουδαιων και εδιδο} & 13 &  &  \\
&  & 13 & \foreignlanguage{greek}{ϲαν αυτω ραπιϲματα} & 15 &  &  \\
& \textbf{4} &  & \foreignlanguage{greek}{εξηλθεν ουν ο πειλατοϲ εξω και λεγει} & 7 &  &  \\
&  & 8 & \foreignlanguage{greek}{αυτοιϲ ειδε αγω υμιν αυτον εξω ινα} & 14 &  &  \\
&  & 15 & \foreignlanguage{greek}{γνωται οτι αιτιαν εν αυτω ουχ ευριϲκω} & 21 &  &  \\
& \textbf{5} &  & \foreignlanguage{greek}{εξηλθεν ουν ο \textoverline{ιϲ} εξω φορων τον ακαν} & 8 &  &  \\
&  & 8 & \foreignlanguage{greek}{θινον ϲτεφανον και το πορφυρουν} & 12 &  &  \\
&  & 13 & \foreignlanguage{greek}{ιματιον και λεγει αυτοιϲ ιδου ο \textoverline{ανοϲ}} & 19 &  &  \\
& \textbf{6} &  & \foreignlanguage{greek}{οτε ουν ιδον αυτον οι αρχιερειϲ και οι υ} & 9 &  &  \\
&  & 9 & \foreignlanguage{greek}{πηρεται εκραυγαϲαν λεγοντεϲ} & 11 &  &  \\
&  & 12 & \foreignlanguage{greek}{ϲταυρωϲον ϲταυρωϲον} & 13 &  &  \\
&  & 14 & \foreignlanguage{greek}{λεγει αυτοιϲ ο πειλατοϲ λαβεται υμειϲ} & 19 &  &  \\
&  & 20 & \foreignlanguage{greek}{αυτον και ϲταυρωϲαται εγω γαρ ουχι} & 25 &  &  \\
&  & 26 & \foreignlanguage{greek}{ευριϲκω εν αυτω αιτιαν} & 29 &  &  \\
& \textbf{7} &  & \foreignlanguage{greek}{απεκριθηϲαν οι ιουδαιοι ημειϲ νομο̅} & 5 &  &  \\
&  & 6 & \foreignlanguage{greek}{εχομεν και κατα τον νομον οφιλει} & 11 &  &  \\
&  & 12 & \foreignlanguage{greek}{αποθανειν οτι υιον του \textoverline{θυ} εαυτο̅} & 17 &  &  \\
&  & 18 & \foreignlanguage{greek}{εποιηϲεν} & 18 &  &  \\
& \textbf{8} &  & \foreignlanguage{greek}{οτε ουν ηκουϲεν ο πειλατοϲ τουτον} & 6 &  &  \\
&  & 7 & \foreignlanguage{greek}{τον λογον μαλλον εφοβηθη και ειϲ} & 2 & \textbf{9} &  \\
&  & 2 & \foreignlanguage{greek}{ηλθεν ειϲ το πρετωριον παλιν} & 6 &  &  \\
&  & 8 & \foreignlanguage{greek}{και λεγει τω \textoverline{ιυ} ποθεν ει ϲυ} & 14 &  &  \\
&  & 15 & \foreignlanguage{greek}{ο δε \textoverline{ιϲ} αποκριϲιν ουκ εδωκεν αυτω} & 21 &  &  \\
& \textbf{10} &  & \foreignlanguage{greek}{λεγει ουν αυτω ο πειλατοϲ εμοι ου λα} & 8 &  &  \\
&  & 8 & \foreignlanguage{greek}{λειϲ ουκ οιδαϲ οτι εξουϲιαν εχω ϲταυ} & 14 &  &  \\
[0.2em]
\cline{4-4}
\end{tabular}
\end{center}
\end{table}
}
\clearpage
\newpage
 {
 \setlength\arrayrulewidth{1pt}
\begin{table}
\begin{center}
\begin{tabular}{ccc|l|ccc}
\cline{4-4} \\ [-1em]
\multicolumn{7}{c}{\foreignlanguage{greek}{ευαγγελιον κατα ιωαννην} \textbf{(\nospace{19:10})} } \\ \\ [-1em] % Si on veut ajouter les bordures latérales, remplacer {7}{c} par {7}{|c|}
\cline{4-4} \\
\cline{4-4}
&  &  & &  &  & \\ [-0.9em]
&  & 14 & \foreignlanguage{greek}{ρωϲαι ϲε και εξουϲιαν εχω απολυϲαι ϲε} & 20 &  &  \\
& \textbf{11} &  & \foreignlanguage{greek}{απεκριθη αυτω ο \textoverline{ιϲ} ουκ ειχεϲ εξουϲιαν} & 7 &  &  \\
&  & 8 & \foreignlanguage{greek}{κατ εμου ουδεμιαν ει μη ην δεδομενο̅} & 14 &  &  \\
&  & 15 & \foreignlanguage{greek}{ϲοι ανωθεν δια τουτο ο παραδιδουϲ} & 20 &  &  \\
&  & 21 & \foreignlanguage{greek}{με ϲοι μιζονα αμαρτιαν εχει} & 25 &  &  \\
& \textbf{12} &  & \foreignlanguage{greek}{εκ τουτου ο πειλατοϲ εζητει αυτον απο} & 7 &  &  \\
&  & 7 & \foreignlanguage{greek}{λυϲαι αυτον οι δε ιουδαιοι εκραυγαζον} & 12 &  &  \\
&  & 13 & \foreignlanguage{greek}{λεγοντεϲ εαν τουτον απολυϲηϲ} & 16 &  &  \\
&  & 17 & \foreignlanguage{greek}{ουκ ει φιλοϲ του καιϲαροϲ παϲ ο βαϲι} & 24 &  &  \\
&  & 24 & \foreignlanguage{greek}{λεα ποιων εαυτον αντιλεγει τω και} & 29 &  &  \\
&  & 29 & \foreignlanguage{greek}{ϲαρι ο ουν πειλατοϲ ακουϲαϲ τω̅} & 5 & \textbf{13} &  \\
&  & 6 & \foreignlanguage{greek}{λογων τουτων ηγαγεν τον \textoverline{ιν} εξω} & 11 &  &  \\
&  & 12 & \foreignlanguage{greek}{και εκαθειϲεν επι του βηματοϲ ειϲ} & 17 &  &  \\
&  & 18 & \foreignlanguage{greek}{τοπον λεγομενον λιθοϲτρωτον} & 20 &  &  \\
&  & 21 & \foreignlanguage{greek}{εβραιϲτι δε γαββαθα ην δε παραϲκευ} & 3 & \textbf{14} &  \\
&  & 3 & \foreignlanguage{greek}{η του παϲχα ωρα ην ωϲ εκτη και ελε} & 11 &  &  \\
&  & 11 & \foreignlanguage{greek}{γεν τοιϲ ιουδαιοιϲ ειδε ο βαϲιλευϲ υμω̅} & 17 &  &  \\
& \textbf{15} &  & \foreignlanguage{greek}{οι δε ελεγον αρον αρον ϲταυρωϲον αυτο̅} & 7 &  &  \\
&  & 8 & \foreignlanguage{greek}{λεγει αυτοιϲ ο πειλατοϲ τον βαϲιλεα υ} & 14 &  &  \\
&  & 14 & \foreignlanguage{greek}{μων ϲταυρωϲω απεκριθηϲαν οι} & 17 &  &  \\
&  & 18 & \foreignlanguage{greek}{αρχιερειϲ ουκ εχομεν βαϲειλεα ει} & 22 &  &  \\
&  & 23 & \foreignlanguage{greek}{μη καιϲαρα τοτε ουν παρεδωκεν} & 3 & \textbf{16} &  \\
&  & 4 & \foreignlanguage{greek}{αυτον αυτοιϲ ινα ϲταυρωθη} & 7 &  &  \\
&  & 8 & \foreignlanguage{greek}{οι δε παραλαβοντεϲ τον \textoverline{ιν} απηγαγο̅} & 13 &  &  \\
& \textbf{17} &  & \foreignlanguage{greek}{και βαϲταζων εαυτω τον ϲταυρον} & 5 &  &  \\
&  & 6 & \foreignlanguage{greek}{εξηλθεν ειϲ τον λεγομενον κρανι} & 10 &  &  \\
&  & 10 & \foreignlanguage{greek}{ου τοπον ο λεγεται εβραιϲτι γολ} & 15 &  &  \\
&  & 15 & \foreignlanguage{greek}{γοθα οπου αυτον εϲταυρωϲαν και} & 4 & \textbf{18} &  \\
&  & 5 & \foreignlanguage{greek}{μετ αυτου αλλουϲ δυο εντευθεν ϗ} & 10 &  &  \\
[0.2em]
\cline{4-4}
\end{tabular}
\end{center}
\end{table}
}
\clearpage
\newpage
 {
 \setlength\arrayrulewidth{1pt}
\begin{table}
\begin{center}
\begin{tabular}{ccc|l|ccc}
\cline{4-4} \\ [-1em]
\multicolumn{7}{c}{\foreignlanguage{greek}{ευαγγελιον κατα ιωαννην} \textbf{(\nospace{19:18})} } \\ \\ [-1em] % Si on veut ajouter les bordures latérales, remplacer {7}{c} par {7}{|c|}
\cline{4-4} \\
\cline{4-4}
&  &  & &  &  & \\ [-0.9em]
&  & 11 & \foreignlanguage{greek}{εντευθεν μεϲον δε τον \textoverline{ιν}} & 15 &  &  \\
& \textbf{19} &  & \foreignlanguage{greek}{εγραψεν δε και τιτλον ο πειλατοϲ και} & 7 &  &  \\
&  & 8 & \foreignlanguage{greek}{εθηκεν επι του ϲταυρου} & 11 &  &  \\
&  & 12 & \foreignlanguage{greek}{ην δε γεγραμμενον \textoverline{ιϲ} ο ναζωραιοϲ} & 17 &  &  \\
&  & 18 & \foreignlanguage{greek}{ο βαϲιλευϲ των ιουδαιων} & 21 &  &  \\
& \textbf{20} &  & \foreignlanguage{greek}{τοτε ουν τον τιτλον ανεγνωϲαν πολ} & 6 &  &  \\
&  & 6 & \foreignlanguage{greek}{λοι των ιουδαιων οτι εγγυϲ ην τηϲ} & 12 &  &  \\
&  & 13 & \foreignlanguage{greek}{πολεωϲ ο τοποϲ οπου εϲταυρωθη ο \textoverline{ιϲ}} & 19 &  &  \\
&  & 20 & \foreignlanguage{greek}{και ην γεγραμμενον εβραιϲτι ρω} & 24 &  &  \\
&  & 24 & \foreignlanguage{greek}{μαειϲτι εβραειϲτι} & 25 &  &  \\
& \textbf{21} &  & \foreignlanguage{greek}{ελεγον ουν τω πειλατω οι αρχιερειϲ} & 6 &  &  \\
&  & 7 & \foreignlanguage{greek}{των ιουδαιων μη γραφε ο βαϲιλευϲ} & 12 &  &  \\
&  & 13 & \foreignlanguage{greek}{των ιουδαιων αλλ οτι εκεινοϲ ειπεν} & 18 &  &  \\
&  & 19 & \foreignlanguage{greek}{βαϲιλευϲ ειμει των ιουδαιων} & 22 &  &  \\
& \textbf{22} &  & \foreignlanguage{greek}{απεκριθη ο πειλατοϲ ο γεγραφα γεγρα} & 6 &  &  \\
&  & 6 & \foreignlanguage{greek}{φα οι ουν ϲτρατιωται οτε εϲταυ} & 5 & \textbf{23} &  \\
&  & 5 & \foreignlanguage{greek}{ρωϲαν τον \textoverline{ιν} ελαβον τα ιματια αυτου} & 11 &  &  \\
&  & 12 & \foreignlanguage{greek}{και εποιηϲαν τεϲϲαρα μερη εκαϲτω} & 16 &  &  \\
&  & 17 & \foreignlanguage{greek}{ϲτρατιωτη μεροϲ και τον χειτωνα} & 21 &  &  \\
&  & 22 & \foreignlanguage{greek}{ην δε ο χειτων αραφοϲ εκ των ανω} & 29 &  &  \\
&  & 29 & \foreignlanguage{greek}{θεν υφαντοϲ δι ολου} & 32 &  &  \\
& \textbf{24} &  & \foreignlanguage{greek}{ειπαν ουν προϲ αλληλουϲ μη ϲχιϲωμε̅} & 6 &  &  \\
&  & 7 & \foreignlanguage{greek}{αυτον αλλα λαχωμεν περι αυτου} & 11 &  &  \\
&  & 12 & \foreignlanguage{greek}{τινοϲ εϲται ινα η γραφη πληρωθη η} & 18 &  &  \\
&  & 19 & \foreignlanguage{greek}{λεγουϲα διεμεριϲαντο τα ιμα} & 22 &  &  \\
&  & 22 & \foreignlanguage{greek}{τια μου και επι τον ιματιϲμον μου} & 28 &  &  \\
&  & 29 & \foreignlanguage{greek}{εβαλον κληρον} & 30 &  &  \\
&  & 31 & \foreignlanguage{greek}{οι μεν ουν ϲτρατιωται ταυτα εποι} & 36 &  &  \\
&  & 36 & \foreignlanguage{greek}{ηϲαν ειϲτηκειϲαν δε παρα} & 3 & \textbf{25} &  \\
&  & 4 & \foreignlanguage{greek}{τω ϲταυρω η μητηρ αυτου και η α} & 11 &  &  \\
[0.2em]
\cline{4-4}
\end{tabular}
\end{center}
\end{table}
}
\clearpage
\newpage
 {
 \setlength\arrayrulewidth{1pt}
\begin{table}
\begin{center}
\begin{tabular}{ccc|l|ccc}
\cline{4-4} \\ [-1em]
\multicolumn{7}{c}{\foreignlanguage{greek}{ευαγγελιον κατα ιωαννην} \textbf{(\nospace{19:25})} } \\ \\ [-1em] % Si on veut ajouter les bordures latérales, remplacer {7}{c} par {7}{|c|}
\cline{4-4} \\
\cline{4-4}
&  &  & &  &  & \\ [-0.9em]
&  & 11 & \foreignlanguage{greek}{δελφη τηϲ μητροϲ αυτου μαρια η του} & 17 &  &  \\
&  & 18 & \foreignlanguage{greek}{κλωπα και μαρια η μαγδαληνη} & 22 &  &  \\
& \textbf{26} &  & \foreignlanguage{greek}{\textoverline{ιϲ} ουν ιδων την μητερα και τον μα} & 8 &  &  \\
&  & 8 & \foreignlanguage{greek}{θητην ον ηγαπα} & 10 &  &  \\
&  & 11 & \foreignlanguage{greek}{λεγει τη μητρι γυναι ιδου ο υιοϲ ϲου} & 18 &  &  \\
& \textbf{27} &  & \foreignlanguage{greek}{ειτα λεγει τω μαθητη ειδε η μη} & 7 &  &  \\
&  & 7 & \foreignlanguage{greek}{τηρ ϲου και απ εκεινηϲ τηϲ ωραϲ} & 13 &  &  \\
&  & 14 & \foreignlanguage{greek}{ελαβεν αυτην ο μαθητηϲ ειϲ τα ι} & 20 &  &  \\
&  & 20 & \foreignlanguage{greek}{δια μετα τουτο ιδωϲ ο \textoverline{ιϲ} οτι} & 6 & \textbf{28} &  \\
&  & 7 & \foreignlanguage{greek}{παντα τετελεϲται ινα τελιωθη} & 10 &  &  \\
&  & 11 & \foreignlanguage{greek}{η γραφη λεγει διψω} & 14 &  &  \\
& \textbf{29} &  & \foreignlanguage{greek}{ϲκευοϲ εκειτο οξουϲ μεϲτον ϲπογ} & 5 &  &  \\
&  & 5 & \foreignlanguage{greek}{γον ουν μεϲτον του οξουϲ υϲϲωπω} & 10 &  &  \\
&  & 11 & \foreignlanguage{greek}{περιθεντεϲ προϲηνεγκαν αυτου} & 13 &  &  \\
&  & 14 & \foreignlanguage{greek}{τω ϲτοματι} & 15 &  &  \\
& \textbf{30} &  & \foreignlanguage{greek}{οτε ουν ελαβεν το οξοϲ \textoverline{ιϲ} ειπεν τε} & 8 &  &  \\
&  & 8 & \foreignlanguage{greek}{τελεϲται και κλειναϲ την κεφα} & 12 &  &  \\
&  & 12 & \foreignlanguage{greek}{λην παραδεδωκεν το \textoverline{πνα}} & 15 &  &  \\
& \textbf{31} &  & \foreignlanguage{greek}{οι ουν ιουδαιοι επι παραϲκευη ην ι} & 7 &  &  \\
&  & 7 & \foreignlanguage{greek}{να μη μεινη επι του ϲταυρου τα ϲω} & 14 &  &  \\
&  & 14 & \foreignlanguage{greek}{ματα εν τω ϲαββατω ην γαρ μεγαλη} & 20 &  &  \\
&  & 21 & \foreignlanguage{greek}{ημερα εκεινου του ϲαββατου} & 24 &  &  \\
&  & 25 & \foreignlanguage{greek}{ηρωτηϲαν τον πιλατον ινα κατεα} & 29 &  &  \\
&  & 29 & \foreignlanguage{greek}{γωϲιν αυτων τα ϲκελη και αρθωϲιν} & 34 &  &  \\
& \textbf{32} &  & \foreignlanguage{greek}{ηλθον ουν οι ϲτρατιωται και του με̅} & 7 &  &  \\
&  & 8 & \foreignlanguage{greek}{πρωτου κατεαξαν τα ϲκελη και του} & 13 &  &  \\
&  & 14 & \foreignlanguage{greek}{αλλου του ϲυνϲταυρωθεντοϲ αυτω} & 17 &  &  \\
& \textbf{33} &  & \foreignlanguage{greek}{επει δε τον \textoverline{ιν} ελθοντεϲ ωϲ ιδον η} & 8 &  &  \\
&  & 8 & \foreignlanguage{greek}{δη αυτον τεθνηκοτα ου κατεαξαν} & 12 &  &  \\
&  & 13 & \foreignlanguage{greek}{αυτου τα ϲκελη} & 15 &  &  \\
[0.2em]
\cline{4-4}
\end{tabular}
\end{center}
\end{table}
}
\clearpage
\newpage
 {
 \setlength\arrayrulewidth{1pt}
\begin{table}
\begin{center}
\begin{tabular}{ccc|l|ccc}
\cline{4-4} \\ [-1em]
\multicolumn{7}{c}{\foreignlanguage{greek}{ευαγγελιον κατα ιωαννην} \textbf{(\nospace{19:34})} } \\ \\ [-1em] % Si on veut ajouter les bordures latérales, remplacer {7}{c} par {7}{|c|}
\cline{4-4} \\
\cline{4-4}
&  &  & &  &  & \\ [-0.9em]
& \textbf{34} &  & \foreignlanguage{greek}{αλλα ειϲ των ϲτρατιωτων λογχη αυτου} & 6 &  &  \\
&  & 7 & \foreignlanguage{greek}{την πλευραν ενυξεν και εξηλθεν} & 11 &  &  \\
&  & 12 & \foreignlanguage{greek}{ευθυϲ αιμα και υδωρ και ο εωρακωϲ} & 3 & \textbf{35} &  \\
&  & 4 & \foreignlanguage{greek}{μεμαρτυρηκεν και αληθεινη αυτου} & 7 &  &  \\
&  & 8 & \foreignlanguage{greek}{εϲτιν η μαρτυρια και εκεινοϲ οιδεν} & 13 &  &  \\
&  & 14 & \foreignlanguage{greek}{οτι αληθη λεγει ινα και υμειϲ πιϲτευ} & 20 &  &  \\
&  & 20 & \foreignlanguage{greek}{ϲηται εγενετο γαρ ταυτα ινα η γρα} & 6 & \textbf{36} &  \\
&  & 6 & \foreignlanguage{greek}{φη πληρωθη οϲτουν ου ϲυντριβη} & 10 &  &  \\
&  & 10 & \foreignlanguage{greek}{ϲεται αυτου και παλιν ετερα γρα} & 4 & \textbf{37} &  \\
&  & 4 & \foreignlanguage{greek}{φη λεγει οψονται ειϲ ον εξεκεντη} & 9 &  &  \\
&  & 9 & \foreignlanguage{greek}{ϲαν μετα δε ταυτα ηρωτη} & 4 & \textbf{38} &  \\
&  & 4 & \foreignlanguage{greek}{ϲεν τον πιλατον ιωϲηφ ο απο αριμα} & 10 &  &  \\
&  & 10 & \foreignlanguage{greek}{θιαϲ ων μαθητηϲ του \textoverline{ιυ} κεκρυμ} & 15 &  &  \\
&  & 15 & \foreignlanguage{greek}{μενοϲ δε δια τον φοβον των ιουδαι} & 21 &  &  \\
&  & 21 & \foreignlanguage{greek}{ων ινα αρη το ϲωμα του \textoverline{ιυ}} & 27 &  &  \\
&  & 28 & \foreignlanguage{greek}{και επετρεψεν ο πειλατοϲ ηλθον} & 32 &  &  \\
&  & 33 & \foreignlanguage{greek}{ουν και ηραν αυτον ηλθεν δε και} & 3 & \textbf{39} &  \\
&  & 4 & \foreignlanguage{greek}{νικοδημοϲ ο ελθων προϲ τον \textoverline{ιν} νυ} & 10 &  &  \\
&  & 10 & \foreignlanguage{greek}{κτοϲ το πρωτον εχων ελιγμα ζμυρ} & 15 &  &  \\
&  & 15 & \foreignlanguage{greek}{νηϲ και αλοηϲ ωϲει λιτραϲ εκατον} & 20 &  &  \\
& \textbf{40} &  & \foreignlanguage{greek}{ελαβον ουν το ϲωμα του \textoverline{ιυ} και εδηϲα̅} & 8 &  &  \\
&  & 9 & \foreignlanguage{greek}{αυτο οθονιοιϲ μετα των αρωματων} & 13 &  &  \\
&  & 14 & \foreignlanguage{greek}{καθωϲ εθοϲ ην τοιϲ ιουδαιοιϲ εντα} & 19 &  &  \\
&  & 19 & \foreignlanguage{greek}{φιαζειν ην δε εν τω τοπω οπου εϲταυρωθη} & 7 & \textbf{41} &  \\
&  & 8 & \foreignlanguage{greek}{κηποϲ και εν τω κηπω μνημιον} & 13 &  &  \\
&  & 14 & \foreignlanguage{greek}{καινον εν ω ουδεπω ουδειϲ ην τε} & 20 &  &  \\
&  & 20 & \foreignlanguage{greek}{θειμενοϲ εκει ουν δια την πα} & 5 & \textbf{42} &  \\
&  & 5 & \foreignlanguage{greek}{ραϲκευην των ιουδαιων οτι εγγυϲ} & 9 &  &  \\
&  & 10 & \foreignlanguage{greek}{ην το μνημιον εθηκαν τον \textoverline{ιν}} & 15 &  &  \\
[0.2em]
\cline{4-4}
\end{tabular}
\end{center}
\end{table}
}
\clearpage
\newpage
 {
 \setlength\arrayrulewidth{1pt}
\begin{table}
\begin{center}
\begin{tabular}{ccc|l|ccc}
\cline{4-4} \\ [-1em]
\multicolumn{7}{c}{\foreignlanguage{greek}{ευαγγελιον κατα ιωαννην} \textbf{(\nospace{20:1})} } \\ \\ [-1em] % Si on veut ajouter les bordures latérales, remplacer {7}{c} par {7}{|c|}
\cline{4-4} \\
\cline{4-4}
&  &  & &  &  & \\ [-0.9em]
& \mygospelchapter &  & \foreignlanguage{greek}{τη δε μια των ϲαββατων μαριαμ η μαγδα} & 8 &  &  \\
&  & 8 & \foreignlanguage{greek}{ληνη ερχεται ϲκοτιαϲ ετι ουϲηϲ επι το} & 14 &  &  \\
&  & 15 & \foreignlanguage{greek}{μνημιον κα βλεπει τον λιθον ηρμε} & 20 &  &  \\
&  & 20 & \foreignlanguage{greek}{νον απο τηϲ θυραϲ εκ του μνημιου} & 26 &  &  \\
& \textbf{2} &  & \foreignlanguage{greek}{τρεχει ουν και ερχεται προϲ ϲιμωνα πε} & 7 &  &  \\
&  & 7 & \foreignlanguage{greek}{τρον και προϲ τον αλλον μαθητην} & 12 &  &  \\
&  & 13 & \foreignlanguage{greek}{ον εφιλει ο \textoverline{ιϲ} και λεγει αυτοιϲ} & 19 &  &  \\
&  & 20 & \foreignlanguage{greek}{ηραν τον \textoverline{κν} εκ του μνημιου και ουκ οι} & 28 &  &  \\
&  & 28 & \foreignlanguage{greek}{δαμεν που εθηκαν αυτον} & 31 &  &  \\
& \textbf{3} &  & \foreignlanguage{greek}{εξηλθεν ουν ο πετροϲ και ο αλλοϲ μα} & 8 &  &  \\
&  & 8 & \foreignlanguage{greek}{θητηϲ και ηρχοντο ειϲ το μνημειον} & 13 &  &  \\
& \textbf{4} &  & \foreignlanguage{greek}{ετρεχον δε οι δυο ομου και ο αλλοϲ μα} & 9 &  &  \\
&  & 9 & \foreignlanguage{greek}{θητηϲ προεδραμεν ταχιον του πετρου} & 13 &  &  \\
&  & 14 & \foreignlanguage{greek}{και ηλθεν πρωτοϲ επι το μνημιον} & 19 &  &  \\
& \textbf{5} &  & \foreignlanguage{greek}{και παρακυψαϲ βλεπει κειμενα τα} & 5 &  &  \\
&  & 6 & \foreignlanguage{greek}{οθονια ου μεντοιϲ ειϲηλθεν ερχε} & 1 & \textbf{6} &  \\
&  & 1 & \foreignlanguage{greek}{ται ουν και ο ϲιμων πετροϲ ακολου} & 7 &  &  \\
&  & 7 & \foreignlanguage{greek}{θων αυτω και ειϲηλθεν ειϲ το μνη} & 13 &  &  \\
&  & 13 & \foreignlanguage{greek}{μιον και θεωρει τα οθονια κειμενα} & 18 &  &  \\
& \textbf{7} &  & \foreignlanguage{greek}{και το ϲουδαριον ο ην επι τηϲ κεφα} & 8 &  &  \\
&  & 8 & \foreignlanguage{greek}{ληϲ αυτου ου μετα των οθονιων} & 13 &  &  \\
&  & 14 & \foreignlanguage{greek}{κειμενον αλλα χωριϲ εντετυλι} & 17 &  &  \\
&  & 17 & \foreignlanguage{greek}{γμενον ειϲ ενα τοπον} & 20 &  &  \\
& \textbf{8} &  & \foreignlanguage{greek}{τοτε ουν ειϲηλθεν και ο αλλοϲ μα} & 7 &  &  \\
&  & 7 & \foreignlanguage{greek}{θητηϲ ο ελθων πρωτοϲ ειϲ το μνη} & 13 &  &  \\
&  & 13 & \foreignlanguage{greek}{μιον και ειδεν και επιϲτευϲεν} & 17 &  &  \\
& \textbf{9} &  & \foreignlanguage{greek}{ουδεπω γαρ ηδιϲαν την γραφην} & 5 &  &  \\
&  & 6 & \foreignlanguage{greek}{οτι δει αυτον εκ νεκρων αναϲτηναι} & 11 &  &  \\
& \textbf{10} &  & \foreignlanguage{greek}{απηλθον ουν παλιν προϲ εαυτουϲ} & 5 &  &  \\
&  & 6 & \foreignlanguage{greek}{οι μαθηται μαρια δε ιϲτηκει προϲ} & 4 & \textbf{11} &  \\
[0.2em]
\cline{4-4}
\end{tabular}
\end{center}
\end{table}
}
\clearpage
\newpage
 {
 \setlength\arrayrulewidth{1pt}
\begin{table}
\begin{center}
\begin{tabular}{ccc|l|ccc}
\cline{4-4} \\ [-1em]
\multicolumn{7}{c}{\foreignlanguage{greek}{ευαγγελιον κατα ιωαννην} \textbf{(\nospace{20:11})} } \\ \\ [-1em] % Si on veut ajouter les bordures latérales, remplacer {7}{c} par {7}{|c|}
\cline{4-4} \\
\cline{4-4}
&  &  & &  &  & \\ [-0.9em]
&  & 5 & \foreignlanguage{greek}{τω μνημιω εξω κλαιουϲα} & 8 &  &  \\
&  & 9 & \foreignlanguage{greek}{ωϲ ουν εκλαιεν παρεκυψεν ειϲ το} & 14 &  &  \\
&  & 15 & \foreignlanguage{greek}{μνημιον και θεωρει δυο αγγελουϲ} & 4 & \textbf{12} &  \\
&  & 5 & \foreignlanguage{greek}{εν λευκοιϲ καθεζομενουϲ ενα προϲ} & 9 &  &  \\
&  & 10 & \foreignlanguage{greek}{τη κεφαλη και ενα προϲ τοιϲ ποϲιν} & 16 &  &  \\
&  & 17 & \foreignlanguage{greek}{οπου εκειτο το ϲωμα του \textoverline{ιυ}} & 22 &  &  \\
& \textbf{13} &  & \foreignlanguage{greek}{και λεγουϲιν αυτη εκεινοι γυναι τι} & 6 &  &  \\
&  & 7 & \foreignlanguage{greek}{κλαιειϲ λεγει αυτοιϲ οτι ηραν το̅} & 12 &  &  \\
&  & 13 & \foreignlanguage{greek}{\textoverline{κν} μου και ουκ οιδα που τεθεικαϲιν} & 19 &  &  \\
&  & 20 & \foreignlanguage{greek}{αυτον ταυτα ειπουϲα εϲτρα} & 3 & \textbf{14} &  \\
&  & 3 & \foreignlanguage{greek}{φη ειϲ τα οπιϲω και ειδεν τον \textoverline{ιν} ε} & 11 &  &  \\
&  & 11 & \foreignlanguage{greek}{ϲτωτα και ουκ ηδει οτι \textoverline{ιϲ} εϲτιν} & 17 &  &  \\
& \textbf{15} &  & \foreignlanguage{greek}{λεγει αυτη \textoverline{ιϲ} γυναι τι κλαιειϲ τινα ζη} & 8 &  &  \\
&  & 8 & \foreignlanguage{greek}{τειϲ εκεινη δοκουϲα οτι ο κηπου} & 13 &  &  \\
&  & 13 & \foreignlanguage{greek}{ροϲ εϲτιν λεγει αυτω \textoverline{κε} ϲυ εβαϲτα} & 19 &  &  \\
&  & 19 & \foreignlanguage{greek}{ξαϲ αυτον ειπε μοι που εθηκαϲ αυτο̅} & 25 &  &  \\
&  & 26 & \foreignlanguage{greek}{καγω αρω αυτον} & 28 &  &  \\
& \textbf{16} &  & \foreignlanguage{greek}{λεγει αυτη ο \textoverline{ιϲ} μαριαμ ϲτραφειϲα} & 6 &  &  \\
&  & 7 & \foreignlanguage{greek}{εκεινη λεγει αυτω εβραιϲτι ραββου} & 11 &  &  \\
&  & 11 & \foreignlanguage{greek}{νι ο λεγεται διδαϲκαλε} & 14 &  &  \\
& \textbf{17} &  & \foreignlanguage{greek}{λεγει αυτη ο \textoverline{ιϲ} μη μου απτου ουπω γαρ} & 9 &  &  \\
&  & 10 & \foreignlanguage{greek}{αναβεβηκα προϲ τον \textoverline{πρα} πορευου} & 14 &  &  \\
&  & 15 & \foreignlanguage{greek}{δε προϲ τουϲ αδελφουϲ και ειπε αυτοιϲ} & 21 &  &  \\
&  & 22 & \foreignlanguage{greek}{αναβενω προϲ τον \textoverline{πρα} μου και \textoverline{πρα}} & 28 &  &  \\
&  & 29 & \foreignlanguage{greek}{υμων και \textoverline{θν} μου και \textoverline{θν} υμων} & 35 &  &  \\
& \textbf{18} &  & \foreignlanguage{greek}{ερχεται μαρια η μαγδαληνη αναγγελ} & 5 &  &  \\
&  & 5 & \foreignlanguage{greek}{λουϲα τοιϲ μαθηταιϲ οτι εωρακα} & 9 &  &  \\
&  & 10 & \foreignlanguage{greek}{τον \textoverline{κν} και ταυτα ειπεν αυτη} & 15 &  &  \\
& \textbf{19} &  & \foreignlanguage{greek}{ουϲηϲ ουν οψειαϲ τη ημερα εκεινη} & 6 &  &  \\
&  & 7 & \foreignlanguage{greek}{μιαϲ ϲαββατων και των θυρων κε} & 12 &  &  \\
[0.2em]
\cline{4-4}
\end{tabular}
\end{center}
\end{table}
}
\clearpage
\newpage
 {
 \setlength\arrayrulewidth{1pt}
\begin{table}
\begin{center}
\begin{tabular}{ccc|l|ccc}
\cline{4-4} \\ [-1em]
\multicolumn{7}{c}{\foreignlanguage{greek}{ευαγγελιον κατα ιωαννην} \textbf{(\nospace{20:19})} } \\ \\ [-1em] % Si on veut ajouter les bordures latérales, remplacer {7}{c} par {7}{|c|}
\cline{4-4} \\
\cline{4-4}
&  &  & &  &  & \\ [-0.9em]
&  & 12 & \foreignlanguage{greek}{κλιϲμενων οπου ηϲαν οι μαθηται δι} & 17 &  &  \\
&  & 17 & \foreignlanguage{greek}{α τον φοβον των ιουδαιων ηλθεν} & 22 &  &  \\
&  & 23 & \foreignlanguage{greek}{ο \textoverline{ιϲ} και εϲτη ειϲ το μεϲον και λεγει αυ} & 32 &  &  \\
&  & 32 & \foreignlanguage{greek}{τοιϲ ειρηνη υμιν και ταυτα ειπω̅} & 3 & \textbf{20} &  \\
&  & 4 & \foreignlanguage{greek}{εδειξεν ταϲ χειραϲ και την πλευραν} & 9 &  &  \\
&  & 10 & \foreignlanguage{greek}{αυτοιϲ εχαρηϲαν ουν οι μαθη} & 14 &  &  \\
&  & 14 & \foreignlanguage{greek}{ται ιδοντεϲ τον \textoverline{κν}} & 17 &  &  \\
& \textbf{21} &  & \foreignlanguage{greek}{ειπεν ουν παλιν αυτοιϲ ειρηνη υμιν} & 6 &  &  \\
&  & 7 & \foreignlanguage{greek}{καθωϲ απεϲταλκεν με ο \textoverline{πηρ} καγω} & 12 &  &  \\
&  & 13 & \foreignlanguage{greek}{πεμπω υμαϲ και τουτο ειπων ε} & 4 & \textbf{22} &  \\
&  & 4 & \foreignlanguage{greek}{νεφυϲηϲεν αυτοιϲ και λεγει λαβε} & 8 &  &  \\
&  & 8 & \foreignlanguage{greek}{ται \textoverline{πνα} αγιον αν τινων αφητε} & 3 & \textbf{23} &  \\
&  & 4 & \foreignlanguage{greek}{ταϲ αμαρτιαϲ αφιενται αυτοιϲ αν τι} & 9 &  &  \\
&  & 9 & \foreignlanguage{greek}{νων κρατητε κεκρατηνται} & 11 &  &  \\
& \textbf{24} &  & \foreignlanguage{greek}{θωμαϲ δε ειϲ εκ των δωδεκα ο λεγο} & 8 &  &  \\
&  & 8 & \foreignlanguage{greek}{μενοϲ διδυμοϲ ουκ ην μετ αυτων οτε} & 14 &  &  \\
&  & 15 & \foreignlanguage{greek}{ηλθεν ο \textoverline{ιϲ} ελεγον ουν αυτω οι} & 4 & \textbf{25} &  \\
&  & 5 & \foreignlanguage{greek}{αλλοι μαθηται εορακαμεν τον \textoverline{κν}} & 9 &  &  \\
&  & 10 & \foreignlanguage{greek}{ο δε ειπεν αυτοιϲ εαν μη ιδω εν ταιϲ} & 18 &  &  \\
&  & 19 & \foreignlanguage{greek}{χερϲιν αυτου τον τυπον των ηλων} & 24 &  &  \\
&  & 25 & \foreignlanguage{greek}{και βαλω μου τον δακτυλον ειϲ τον} & 31 &  &  \\
&  & 32 & \foreignlanguage{greek}{τυπον των ηλων και βαλω μου} & 37 &  &  \\
&  & 38 & \foreignlanguage{greek}{την χειρα ειϲ την πλευραν αυτου} & 43 &  &  \\
&  & 44 & \foreignlanguage{greek}{ου μη πιϲτευϲω και μετα ημε} & 3 & \textbf{26} &  \\
&  & 3 & \foreignlanguage{greek}{ραϲ οκτω παλιν ηϲαν εϲω οι μαθηται} & 9 &  &  \\
&  & 10 & \foreignlanguage{greek}{και θωμαϲ μετ αυτων} & 13 &  &  \\
&  & 14 & \foreignlanguage{greek}{ερχεται \textoverline{ιϲ} των θυρων κεκλιϲμενω̅} & 18 &  &  \\
&  & 19 & \foreignlanguage{greek}{και εϲτη ειϲ το μεϲον και ειπεν ειρη} & 26 &  &  \\
&  & 26 & \foreignlanguage{greek}{νη υμιν ειτα λεγει τω θωμα φε} & 5 & \textbf{27} &  \\
&  & 5 & \foreignlanguage{greek}{ρε τον δακτυλον ϲου ωδε και ειδε} & 11 &  &  \\
[0.2em]
\cline{4-4}
\end{tabular}
\end{center}
\end{table}
}
\clearpage
\newpage
 {
 \setlength\arrayrulewidth{1pt}
\begin{table}
\begin{center}
\begin{tabular}{ccc|l|ccc}
\cline{4-4} \\ [-1em]
\multicolumn{7}{c}{\foreignlanguage{greek}{ευαγγελιον κατα ιωαννην} \textbf{(\nospace{20:27})} } \\ \\ [-1em] % Si on veut ajouter les bordures latérales, remplacer {7}{c} par {7}{|c|}
\cline{4-4} \\
\cline{4-4}
&  &  & &  &  & \\ [-0.9em]
&  & 12 & \foreignlanguage{greek}{ταϲ χειραϲ μου και φερε την χειρα ϲου} & 19 &  &  \\
&  & 20 & \foreignlanguage{greek}{και βαλε ειϲ την πλευραν μου και μη} & 27 &  &  \\
&  & 28 & \foreignlanguage{greek}{γινου απιϲτοϲ αλλα πιϲτοϲ} & 31 &  &  \\
& \textbf{28} &  & \foreignlanguage{greek}{απεκριθη θωμαϲ και ειπεν αυτω \textoverline{κϲ}} & 6 &  &  \\
&  & 7 & \foreignlanguage{greek}{μου και ο \textoverline{θϲ} μου} & 11 &  &  \\
& \textbf{29} &  & \foreignlanguage{greek}{ειπεν δε αυτω ο \textoverline{ιϲ} οτι εορακαϲ με πε} & 9 &  &  \\
&  & 9 & \foreignlanguage{greek}{πιϲτευκαϲ μακαριοι οι μη ειδοντεϲ} & 13 &  &  \\
&  & 14 & \foreignlanguage{greek}{και πιϲτευϲαντεϲ} & 15 &  &  \\
& \textbf{30} &  & \foreignlanguage{greek}{πολλα μεν ουν και αλλα ϲημια πεποι} & 7 &  &  \\
&  & 7 & \foreignlanguage{greek}{ηκεν ο \textoverline{ιϲ} ενωπιον των μαθητων} & 12 &  &  \\
&  & 13 & \foreignlanguage{greek}{αυτου α ουκ εϲτιν γεγραμμενα εν} & 18 &  &  \\
&  & 19 & \foreignlanguage{greek}{τω βιβλιω τουτω} & 21 &  &  \\
& \textbf{31} &  & \foreignlanguage{greek}{ταυτα δε γεγραπται ινα πιϲτευϲη} & 5 &  &  \\
&  & 5 & \foreignlanguage{greek}{ται οτι \textoverline{ιϲ} ο \textoverline{χϲ} εϲτιν ο \textoverline{υϲ} του \textoverline{θυ}} & 14 &  &  \\
&  & 15 & \foreignlanguage{greek}{και ινα πιϲτευοντεϲ ζωην εχηται} & 19 &  &  \\
&  & 20 & \foreignlanguage{greek}{εν τω ονοματι αυτου} & 23 &  &  \\
& \mygospelchapter &  & \foreignlanguage{greek}{μετα ταυτα εφανερωϲεν εαυτον} & 4 &  &  \\
&  & 5 & \foreignlanguage{greek}{ο \textoverline{ιϲ} παλιν τοιϲ μαθηταιϲ επι τηϲ θα} & 12 &  &  \\
&  & 12 & \foreignlanguage{greek}{λαϲϲηϲ τηϲ τιβεριαδοϲ εφανε} & 15 &  &  \\
&  & 15 & \foreignlanguage{greek}{ρωϲεν δε ουτωϲ} & 17 &  &  \\
& \textbf{2} &  & \foreignlanguage{greek}{ηϲαν ομου ϲιμων πετροϲ και θω} & 6 &  &  \\
&  & 6 & \foreignlanguage{greek}{μαϲ ο λεγομενοϲ διδυμοϲ και να} & 11 &  &  \\
&  & 11 & \foreignlanguage{greek}{θαναηλ ο απο κανα τηϲ γαλιλαιαϲ} & 16 &  &  \\
&  & 17 & \foreignlanguage{greek}{και οι του ζεβεδεου και αλλοι εκ} & 23 &  &  \\
&  & 24 & \foreignlanguage{greek}{των μαθητων αυτου δυο} & 27 &  &  \\
& \textbf{3} &  & \foreignlanguage{greek}{λεγει αυτοιϲ ϲιμων πετροϲ υπαγω} & 5 &  &  \\
&  & 6 & \foreignlanguage{greek}{αλιευειν λεγουϲιν αυτω ερχομε} & 9 &  &  \\
&  & 9 & \foreignlanguage{greek}{θα και ημειϲ ϲυν ϲοι} & 13 &  &  \\
&  & 14 & \foreignlanguage{greek}{εξηλθον και ενεβηϲαν ειϲ το πλοιο̅} & 19 &  &  \\
&  & 20 & \foreignlanguage{greek}{και εν εκεινη τη νυκτι επιαϲαν ουδε ε̅} & 27 &  &  \\
[0.2em]
\cline{4-4}
\end{tabular}
\end{center}
\end{table}
}
\clearpage
\newpage
 {
 \setlength\arrayrulewidth{1pt}
\begin{table}
\begin{center}
\begin{tabular}{ccc|l|ccc}
\cline{4-4} \\ [-1em]
\multicolumn{7}{c}{\foreignlanguage{greek}{ευαγγελιον κατα ιωαννην} \textbf{(\nospace{21:4})} } \\ \\ [-1em] % Si on veut ajouter les bordures latérales, remplacer {7}{c} par {7}{|c|}
\cline{4-4} \\
\cline{4-4}
&  &  & &  &  & \\ [-0.9em]
& \textbf{4} &  & \foreignlanguage{greek}{πρωιαϲ δε ηδη γενομενηϲ εϲτη \textoverline{ιϲ}} & 6 &  &  \\
& \textbf{5} &  & \foreignlanguage{greek}{και λεγει αυτοιϲ παιδια μη προϲφα} & 6 &  &  \\
&  & 6 & \foreignlanguage{greek}{γιον εχεται απεκριθηϲαν αυτω ου} & 10 &  &  \\
& \textbf{6} &  & \foreignlanguage{greek}{λεγει αυτοιϲ βαλεται ειϲ τα δεξια} & 6 &  &  \\
&  & 7 & \foreignlanguage{greek}{μερη του πλοιου το δικτυον και} & 12 &  &  \\
&  & 13 & \foreignlanguage{greek}{ευρηϲεται οι δε εβαλον και ουκε} & 18 &  &  \\
&  & 18 & \foreignlanguage{greek}{τι αυτο ιϲχυϲαν ελκυϲαι απο του} & 23 &  &  \\
&  & 24 & \foreignlanguage{greek}{πληθουϲ των ιχθυων} & 26 &  &  \\
& \textbf{7} &  & \foreignlanguage{greek}{λεγει ουν ο μαθητηϲ εκεινοϲ ον η} & 7 &  &  \\
&  & 7 & \foreignlanguage{greek}{γαπα ο \textoverline{ιϲ} τω πετρω ο \textoverline{κϲ} εϲτιν} & 14 &  &  \\
&  & 15 & \foreignlanguage{greek}{ϲιμων ουν πετροϲ ακουϲαϲ οτι ο \textoverline{κϲ} ε} & 22 &  &  \\
&  & 22 & \foreignlanguage{greek}{ϲτιν τον επενδυτην διεζωϲατο} & 25 &  &  \\
&  & 26 & \foreignlanguage{greek}{ην γαρ γυμνοϲ και εβαλεν εαυτον} & 31 &  &  \\
&  & 32 & \foreignlanguage{greek}{ειϲ την θαλαϲϲαν οι δε αλλοι μα} & 4 & \textbf{8} &  \\
&  & 4 & \foreignlanguage{greek}{θηται τω πλοιω ηλθον ου γαρ ηϲα̅} & 10 &  &  \\
&  & 11 & \foreignlanguage{greek}{μακραν απο τηϲ γηϲ αλλα ωϲ απο} & 17 &  &  \\
&  & 18 & \foreignlanguage{greek}{πηχεων διακοϲιων ϲυροντεϲ το} & 21 &  &  \\
&  & 22 & \foreignlanguage{greek}{δικτυον των ιχθυων} & 24 &  &  \\
& \textbf{9} &  & \foreignlanguage{greek}{ωϲ ουν ανεβηϲαν ειϲ την γην βλε} & 7 &  &  \\
&  & 7 & \foreignlanguage{greek}{πουϲιν ανθρακιαν κειμενην και} & 10 &  &  \\
&  & 11 & \foreignlanguage{greek}{οψαριον επικειμενον και αρτον} & 14 &  &  \\
& \textbf{10} &  & \foreignlanguage{greek}{λεγει αυτοιϲ ο \textoverline{ιϲ} ενεγκαται απο των} & 7 &  &  \\
&  & 8 & \foreignlanguage{greek}{οψαριων ων επιαϲαται νυν} & 11 &  &  \\
& \textbf{11} &  & \foreignlanguage{greek}{ενεβη ουν ϲιμων πετροϲ και ειλκυ} & 6 &  &  \\
&  & 6 & \foreignlanguage{greek}{ϲεν το δικτυον ειϲ την γην μεϲτον} & 12 &  &  \\
&  & 13 & \foreignlanguage{greek}{μεγαλων ιχθυων εκατον πεντη} & 16 &  &  \\
&  & 16 & \foreignlanguage{greek}{κοντα τριων και τοϲουτων οντω̅} & 20 &  &  \\
&  & 21 & \foreignlanguage{greek}{ουκ εϲχιϲθη το δικτυον} & 24 &  &  \\
& \textbf{12} &  & \foreignlanguage{greek}{λεγει αυτοιϲ ο \textoverline{ιϲ} δευτε αριϲταται} & 6 &  &  \\
&  & 7 & \foreignlanguage{greek}{ουδειϲ δε ετολμα των μαθητων εξε} & 12 &  &  \\
[0.2em]
\cline{4-4}
\end{tabular}
\end{center}
\end{table}
}
\clearpage
\newpage
 {
 \setlength\arrayrulewidth{1pt}
\begin{table}
\begin{center}
\begin{tabular}{ccc|l|ccc}
\cline{4-4} \\ [-1em]
\multicolumn{7}{c}{\foreignlanguage{greek}{ευαγγελιον κατα ιωαννην} \textbf{(\nospace{21:12})} } \\ \\ [-1em] % Si on veut ajouter les bordures latérales, remplacer {7}{c} par {7}{|c|}
\cline{4-4} \\
\cline{4-4}
&  &  & &  &  & \\ [-0.9em]
&  & 12 & \foreignlanguage{greek}{ταϲαι αυτον ϲυ τιϲ ει ειδοτεϲ οτι ο \textoverline{κϲ} εϲτι̅} & 21 &  &  \\
& \textbf{13} &  & \foreignlanguage{greek}{ερχεται \textoverline{ιϲ} και λαμβανει τον αρτον} & 6 &  &  \\
&  & 7 & \foreignlanguage{greek}{και διδωϲιν αυτοιϲ και το οψαριον ο} & 13 &  &  \\
&  & 13 & \foreignlanguage{greek}{μοιωϲ τουτο ηδη τριτον εφανερω} & 4 & \textbf{14} &  \\
&  & 4 & \foreignlanguage{greek}{θη τοιϲ μαθηταιϲ εγερθειϲ εκ νεκρω̅} & 9 &  &  \\
& \textbf{15} &  & \foreignlanguage{greek}{οτε ουν ηριϲτηϲαν λεγει τω ϲιμωνι πε} & 7 &  &  \\
&  & 7 & \foreignlanguage{greek}{τρω ο \textoverline{ιϲ} ϲιμων ιωαννου αγαπαϲ με} & 13 &  &  \\
&  & 14 & \foreignlanguage{greek}{πλειον παντων τουτων} & 16 &  &  \\
&  & 17 & \foreignlanguage{greek}{λεγει αυτω ναι \textoverline{κε} ϲυ οιδαϲ οτι φιλω ϲε} & 25 &  &  \\
&  & 26 & \foreignlanguage{greek}{λεγει αυτω βοϲκε τα αρνια μου} & 31 &  &  \\
& \textbf{16} &  & \foreignlanguage{greek}{παλιν λεγει αυτω δευτερον ϲιμω̅} & 5 &  &  \\
&  & 6 & \foreignlanguage{greek}{ιωαννου αγαπαϲ με λεγει αυτω} & 10 &  &  \\
&  & 11 & \foreignlanguage{greek}{ναι \textoverline{κε} ϲυ οιδαϲ οτι φιλω ϲε} & 17 &  &  \\
&  & 18 & \foreignlanguage{greek}{λεγει αυτω ποιμαινε τα προβατα μου} & 23 &  &  \\
& \textbf{17} &  & \foreignlanguage{greek}{λεγει αυτω το τριτον ϲιμων ιω} & 6 &  &  \\
&  & 6 & \foreignlanguage{greek}{αννου αγαπαϲ με ελυπηθη ο πε} & 11 &  &  \\
&  & 11 & \foreignlanguage{greek}{τροϲ οτι ειπεν αυτω το τριτον φι} & 17 &  &  \\
&  & 17 & \foreignlanguage{greek}{λειϲ με και λεγει αυτω \textoverline{κε} παν} & 23 &  &  \\
&  & 23 & \foreignlanguage{greek}{τα ϲυ οιδαϲ ϲυ γιγνωϲκειϲ οτι φι} & 29 &  &  \\
&  & 29 & \foreignlanguage{greek}{λω ϲε} & 30 &  &  \\
&  & 31 & \foreignlanguage{greek}{λεγει αυτω βοϲκε τα προβατια μου} & 37 &  &  \\
& \textbf{18} &  & \foreignlanguage{greek}{αμην αμην λεγω ϲοι οτε ηϲ νεω} & 7 &  &  \\
&  & 7 & \foreignlanguage{greek}{τεροϲ εζωννυεϲ ϲεαυτον και πε} & 11 &  &  \\
&  & 11 & \foreignlanguage{greek}{ριεπατειϲ οπου ηθελεϲ} & 13 &  &  \\
&  & 14 & \foreignlanguage{greek}{οταν δε γηραϲηϲ εκτενειϲ ταϲ χει} & 19 &  &  \\
&  & 19 & \foreignlanguage{greek}{ραϲ ϲου και αλλοι ϲε ζωϲουϲιν και α} & 26 &  &  \\
&  & 26 & \foreignlanguage{greek}{ποιϲουϲιν οπου ϲυ ου θελειϲ} & 30 &  &  \\
& \textbf{19} &  & \foreignlanguage{greek}{τουτο δε ελεγεν ϲημαινων ποιω} & 5 &  &  \\
&  & 6 & \foreignlanguage{greek}{θανατω δοξαϲει τον \textoverline{θν}} & 9 &  &  \\
&  & 10 & \foreignlanguage{greek}{και τουτο ειπων λεγει αυτω ακο} & 15 &  &  \\
[0.2em]
\cline{4-4}
\end{tabular}
\end{center}
\end{table}
}
\clearpage
\newpage
 {
 \setlength\arrayrulewidth{1pt}
\begin{table}
\begin{center}
\begin{tabular}{ccc|l|ccc}
\cline{4-4} \\ [-1em]
\multicolumn{7}{c}{\foreignlanguage{greek}{ευαγγελιον κατα ιωαννην} \textbf{(\nospace{21:19})} } \\ \\ [-1em] % Si on veut ajouter les bordures latérales, remplacer {7}{c} par {7}{|c|}
\cline{4-4} \\
\cline{4-4}
&  &  & &  &  & \\ [-0.9em]
&  & 15 & \foreignlanguage{greek}{λουθει μοι επιϲτραφειϲ ο πετροϲ βλεπει τον} & 5 & \textbf{20} &  \\
&  & 6 & \foreignlanguage{greek}{μαθητην ον ηγαπα ο \textoverline{ιϲ} οϲ και ανε} & 13 &  &  \\
&  & 13 & \foreignlanguage{greek}{πεϲεν εν τω διπνω επι το ϲτηθοϲ} & 19 &  &  \\
&  & 20 & \foreignlanguage{greek}{αυτου και ειπεν αυτω \textoverline{κε} τιϲ εϲτι̅} & 26 &  &  \\
&  & 27 & \foreignlanguage{greek}{ο παραδιδουϲ ϲε} & 29 &  &  \\
& \textbf{21} &  & \foreignlanguage{greek}{τουτον ιδων ο πετροϲ ειπεν τω \textoverline{ιυ}} & 7 &  &  \\
&  & 8 & \foreignlanguage{greek}{\textoverline{κε} ουτοϲ δε τι} & 11 &  &  \\
& \textbf{22} &  & \foreignlanguage{greek}{λεγει αυτω ο \textoverline{ιϲ} εαν αυτον θελω με} & 8 &  &  \\
&  & 8 & \foreignlanguage{greek}{νειν εωϲ ερχομαι τι προϲ ϲε ϲυ} & 14 &  &  \\
&  & 15 & \foreignlanguage{greek}{μοι ακολουθει εξηλθεν ουν ουτοϲ ο λογοϲ ειϲ} & 6 & \textbf{23} &  \\
&  & 7 & \foreignlanguage{greek}{τουϲ αδελφουϲ οτι ο μαθητηϲ} & 11 &  &  \\
&  & 12 & \foreignlanguage{greek}{εκεινοϲ ουκ αποθνηϲκει} & 14 &  &  \\
&  & 15 & \foreignlanguage{greek}{ουκ ειπεν δε αυτω ο \textoverline{ιϲ} οτι ουκ α} & 23 &  &  \\
&  & 23 & \foreignlanguage{greek}{ποθνηϲκει αλλ εαν αυτον θε} & 27 &  &  \\
&  & 27 & \foreignlanguage{greek}{λω μενειν εωϲ ερχομαι τι προϲ} & 32 &  &  \\
&  & 33 & \foreignlanguage{greek}{ϲε ουτοϲ εϲτιν ο μαθη} & 4 & \textbf{24} &  \\
&  & 4 & \foreignlanguage{greek}{τηϲ ο και μαρτυρων περι τουτω̅} & 9 &  &  \\
&  & 10 & \foreignlanguage{greek}{και γραψαϲ ταυτα} & 12 &  &  \\
&  & 13 & \foreignlanguage{greek}{και οιδαμεν οτι αληθηϲ αυτου} & 17 &  &  \\
&  & 18 & \foreignlanguage{greek}{η μαρτυρια εϲτιν} & 20 &  &  \\
& \textbf{25} &  & \foreignlanguage{greek}{εϲτιν δε και αλλα πολλα οϲα ε} & 7 &  &  \\
&  & 7 & \foreignlanguage{greek}{ποιηϲεν ο \textoverline{ιϲ} ατινα εαν γραφηται} & 12 &  &  \\
&  & 13 & \foreignlanguage{greek}{καθ εν ουδε αυτον οιμαι τον κοϲ} & 19 &  &  \\
&  & 19 & \foreignlanguage{greek}{μον χωρηϲαι τα γραφομενα βι} & 23 &  &  \\
&  & 23 & \foreignlanguage{greek}{βλια} & 23 &  &  \\
[0.2em]
\cline{4-4}
\end{tabular}
\end{center}
\end{table}
}
\clearpage
\newpage
 {
 \setlength\arrayrulewidth{1pt}
\begin{table}
\begin{center}
\begin{tabular}{ccc|l|ccc}
\cline{4-4} \\ [-1em]
\multicolumn{7}{c}{\agospelbook{\foreignlanguage{greek}{ευαγγελιον κατα λουκαν}} \textbf{(\nospace{1:1})} } \\ \\ [-1em] % Si on veut ajouter les bordures latérales, remplacer {7}{c} par {7}{|c|}
\cline{4-4} \\
\cline{4-4}
&  &  & &  &  & \\ [-0.9em]
& \mygospelchapter &  & \foreignlanguage{greek}{επειδηπερ πολλοι επεχειρηϲαν αναταξα} & 4 &  &  \\
&  & 4 & \foreignlanguage{greek}{ϲθαι διηγηϲιν περι των πεπληροφορημε} & 8 &  &  \\
&  & 8 & \foreignlanguage{greek}{νων ημιν πραγματων καθωϲ παρεδοϲαν} & 2 & \textbf{2} &  \\
&  & 3 & \foreignlanguage{greek}{ημιν οι απ αρχηϲ αυθοπται και υπηρεται} & 9 &  &  \\
&  & 10 & \foreignlanguage{greek}{γενομενοι του λογου εδοξε καμοι παρη} & 3 & \textbf{3} &  \\
&  & 3 & \foreignlanguage{greek}{κολουθηκοτι ανωθε παϲιν ακριβωϲ καθε} & 7 &  &  \\
&  & 7 & \foreignlanguage{greek}{ξηϲ ϲοι γραψαι κρατιϲτε θεοφιλε ινα επι} & 2 & \textbf{4} &  \\
&  & 2 & \foreignlanguage{greek}{γνωϲ περι ων κατηχηθηϲ λογων την αϲφαλια} & 8 &  &  \\
& \textbf{5} &  & \foreignlanguage{greek}{εγενετο εν ταιϲ ημεραιϲ ηρωδου βαϲιλεωϲ} & 6 &  &  \\
&  & 7 & \foreignlanguage{greek}{τηϲ ιουδαιαϲ ιερευϲ τιϲ ονοματι ζαχαριαϲ} & 12 &  &  \\
&  & 13 & \foreignlanguage{greek}{εξ εφημεριαϲ αβιλ και γυνη αυτω εκ} & 19 &  &  \\
&  & 20 & \foreignlanguage{greek}{των θυγατερων ααρων και το ονομα αυτη} & 26 &  &  \\
&  & 27 & \foreignlanguage{greek}{ελιϲαβετ ηϲαν δε δικαιοι αμφοτεροι ενω} & 5 & \textbf{6} &  \\
&  & 5 & \foreignlanguage{greek}{πιον του \textoverline{θυ} πορευομενοι εν παϲαιϲ ταιϲ} & 12 &  &  \\
&  & 13 & \foreignlanguage{greek}{εντολαιϲ και δικαιωμαϲιν του \textoverline{κυ} αμεμπτοι} & 18 &  &  \\
& \textbf{7} &  & \foreignlanguage{greek}{και ουκ ην αυτοιϲ τεκνον καθοτι ην ε} & 8 &  &  \\
&  & 8 & \foreignlanguage{greek}{λιϲαβετ ϲτειρα και αμφοτεροι προβεβη} & 12 &  &  \\
&  & 12 & \foreignlanguage{greek}{κοτεϲ εν ταιϲ ημεραιϲ αυτων ηϲαν} & 17 &  &  \\
& \textbf{8} &  & \foreignlanguage{greek}{εγενετο δε εν τω ιερατευειν αυτον εν τη} & 8 &  &  \\
&  & 9 & \foreignlanguage{greek}{ταξει τηϲ εφημεριαϲ αυτου εναντι του \textoverline{θυ}} & 15 &  &  \\
& \textbf{9} &  & \foreignlanguage{greek}{κατα το εθοϲ τηϲ ιερατιαϲ ελαχε του θυμι} & 8 &  &  \\
&  & 8 & \foreignlanguage{greek}{αϲαι ειϲελθων ειϲ τον ναον του \textoverline{κυ}} & 14 &  &  \\
& \textbf{10} &  & \foreignlanguage{greek}{και παν το πληθοϲ ην του λαου προϲευχο} & 8 &  &  \\
&  & 8 & \foreignlanguage{greek}{μενον εξω τη ωρα του θυμιαματοϲ} & 13 &  &  \\
& \textbf{11} &  & \foreignlanguage{greek}{ωφθη δε αυτω αγγελοϲ \textoverline{κυ} εϲτωϲ εκ δεξιων} & 8 &  &  \\
&  & 9 & \foreignlanguage{greek}{του θυϲιαϲτηριου του θυμιαματοϲ} & 12 &  &  \\
& \textbf{12} &  & \foreignlanguage{greek}{και εταραχθη ζαχαριαϲ ιδων και φοβοϲ} & 6 &  &  \\
&  & 7 & \foreignlanguage{greek}{επεπεϲεν επ αυτον} & 9 &  &  \\
& \textbf{13} &  & \foreignlanguage{greek}{ειπεν δε προϲ αυτον ο αγγελοϲ μη φοβου} & 8 &  &  \\
&  & 9 & \foreignlanguage{greek}{ζαχαρια διοτι ειϲηκουϲθη η δεηϲιϲ ϲου} & 14 &  &  \\
[0.2em]
\cline{4-4}
\end{tabular}
\end{center}
\end{table}
}
\clearpage
\newpage
 {
 \setlength\arrayrulewidth{1pt}
\begin{table}
\begin{center}
\begin{tabular}{ccc|l|ccc}
\cline{4-4} \\ [-1em]
\multicolumn{7}{c}{\foreignlanguage{greek}{ευαγγελιον κατα λουκαν} \textbf{(\nospace{1:13})} } \\ \\ [-1em] % Si on veut ajouter les bordures latérales, remplacer {7}{c} par {7}{|c|}
\cline{4-4} \\
\cline{4-4}
&  &  & &  &  & \\ [-0.9em]
&  & 15 & \foreignlanguage{greek}{και η γυνη ϲου ελιϲαβετ γεννηϲει υιον ϲοι} & 22 &  &  \\
&  & 23 & \foreignlanguage{greek}{και καλεϲειϲ το ονομα αυτου ιωαννην} & 28 &  &  \\
& \textbf{14} &  & \foreignlanguage{greek}{και εϲται χαρα ϲοι και αγαλλιαϲιϲ} & 6 &  &  \\
&  & 7 & \foreignlanguage{greek}{και πολλοι επι τη γενεϲει αυτου χαρηϲονται} & 13 &  &  \\
& \textbf{15} &  & \foreignlanguage{greek}{εϲται γαρ μεγαϲ ενωπιον του \textoverline{κυ}} & 6 &  &  \\
&  & 7 & \foreignlanguage{greek}{και οινον και ϲικαιρα ου μη πιη και \textoverline{πνϲ}} & 15 &  &  \\
&  & 16 & \foreignlanguage{greek}{αγιου πληϲθηϲεται ετι εν κοιλια μητροϲ} & 21 &  &  \\
&  & 22 & \foreignlanguage{greek}{αυτου και πολλουϲ των υιων ιϲραηλ} & 5 & \textbf{16} &  \\
&  & 6 & \foreignlanguage{greek}{επιϲτρεψει επι \textoverline{κν} τον \textoverline{θν} αυτων} & 11 &  &  \\
& \textbf{17} &  & \foreignlanguage{greek}{και αυτοϲ προελευϲεται ενωπιον αυτου} & 5 &  &  \\
&  & 6 & \foreignlanguage{greek}{εν \textoverline{πνι} και δυναμει ηλια} & 10 &  &  \\
&  & 11 & \foreignlanguage{greek}{επιϲτρεψαι καρδιαϲ πατερων επι τεκνα} & 15 &  &  \\
&  & 16 & \foreignlanguage{greek}{και απιθειϲ εν φρονηϲει δικαιων} & 20 &  &  \\
&  & 21 & \foreignlanguage{greek}{ετοιμαϲαι \textoverline{κω} λαον κατεϲκευαϲμενον} & 24 &  &  \\
& \textbf{18} &  & \foreignlanguage{greek}{και ειπεν ζαχαριαϲ προϲ τον αγγελον} & 6 &  &  \\
&  & 7 & \foreignlanguage{greek}{κατα τι γνωϲομαι τουτο εγω γαρ ειμει} & 13 &  &  \\
&  & 14 & \foreignlanguage{greek}{πρεϲβυτηϲ και η γυνη μου προβεβηκυ} & 19 &  &  \\
&  & 19 & \foreignlanguage{greek}{ια εν ταιϲ ημεραιϲ αυτηϲ} & 23 &  &  \\
& \textbf{19} &  & \foreignlanguage{greek}{και αποκριθειϲ ο αγγελοϲ ειπεν αυτω} & 6 &  &  \\
&  & 7 & \foreignlanguage{greek}{εγω ειμει γαβριηλ ο παρεϲτηκωϲ ε} & 12 &  &  \\
&  & 12 & \foreignlanguage{greek}{νωπιον του \textoverline{θυ} και απεϲταλην λα} & 17 &  &  \\
&  & 17 & \foreignlanguage{greek}{ληϲαι προϲ ϲε και ευαγγελιϲαϲθαι ϲοι} & 22 &  &  \\
&  & 23 & \foreignlanguage{greek}{ταυτα και ιδου εϲη ϲιωπων και μη} & 6 & \textbf{20} &  \\
&  & 7 & \foreignlanguage{greek}{δυναμενοϲ λαληϲαι αχριϲ ημεραϲ γε} & 11 &  &  \\
&  & 11 & \foreignlanguage{greek}{νηται ταυτα ανθ ων ουκ επιϲτευϲαϲ} & 16 &  &  \\
&  & 17 & \foreignlanguage{greek}{τοιϲ λογοιϲ μου οιτινεϲ πληϲθηϲον ειϲ} & 22 &  &  \\
&  & 23 & \foreignlanguage{greek}{τον καιρον αυτων} & 25 &  &  \\
& \textbf{21} &  & \foreignlanguage{greek}{και ην ο λαοϲ προϲδοκων τον ζαχαριαν} & 7 &  &  \\
&  & 8 & \foreignlanguage{greek}{και εθαυμαζον εν τω χρονιζειν εν τω} & 14 &  &  \\
&  & 15 & \foreignlanguage{greek}{ναω αυτον} & 16 &  &  \\
[0.2em]
\cline{4-4}
\end{tabular}
\end{center}
\end{table}
}
\clearpage
\newpage
 {
 \setlength\arrayrulewidth{1pt}
\begin{table}
\begin{center}
\begin{tabular}{ccc|l|ccc}
\cline{4-4} \\ [-1em]
\multicolumn{7}{c}{\foreignlanguage{greek}{ευαγγελιον κατα λουκαν} \textbf{(\nospace{1:22})} } \\ \\ [-1em] % Si on veut ajouter les bordures latérales, remplacer {7}{c} par {7}{|c|}
\cline{4-4} \\
\cline{4-4}
&  &  & &  &  & \\ [-0.9em]
& \textbf{22} &  & \foreignlanguage{greek}{εξελθων δε ουκ ηδυνατο λαληϲαι αυτοιϲ} & 6 &  &  \\
&  & 7 & \foreignlanguage{greek}{και επεγνωϲαν οτι οπταϲιαν εωρακεν εν} & 12 &  &  \\
&  & 13 & \foreignlanguage{greek}{τω ναω και αυτοϲ ην διανευων αυτοιϲ} & 19 &  &  \\
&  & 20 & \foreignlanguage{greek}{και διεμενεν κωφοϲ} & 22 &  &  \\
& \textbf{23} &  & \foreignlanguage{greek}{και εγενετο ωϲ επληϲθηϲαν αι ημεραι τηϲ} & 7 &  &  \\
&  & 8 & \foreignlanguage{greek}{λιτουργιαϲ αυτου απηλθεν ειϲ τον οικο̅} & 13 &  &  \\
&  & 14 & \foreignlanguage{greek}{αυτου μετα δε ταυταϲ ταϲ ημεραϲ} & 5 & \textbf{24} &  \\
&  & 6 & \foreignlanguage{greek}{ϲυνελαβεν ελιϲαβετ η γυνη αυτου} & 10 &  &  \\
&  & 11 & \foreignlanguage{greek}{και περιεκρυβεν εαυτην μηναϲ πεντε} & 15 &  &  \\
&  & 16 & \foreignlanguage{greek}{λεγουϲα οτι ουτωϲ μοι πεποιηκεν \textoverline{κϲ} εν} & 6 & \textbf{25} &  \\
&  & 7 & \foreignlanguage{greek}{ημεραιϲ αιϲ εφειδεν αφελειν ονει} & 11 &  &  \\
&  & 11 & \foreignlanguage{greek}{δοϲ μου εν ανθρωποιϲ} & 14 &  &  \\
& \textbf{26} &  & \foreignlanguage{greek}{εν δε τω μηνι τω εκτω απεϲταλη ο αγγε} & 9 &  &  \\
&  & 9 & \foreignlanguage{greek}{λοϲ γαβριηλ απο του \textoverline{θυ} ειϲ πολιν τηϲ} & 16 &  &  \\
&  & 17 & \foreignlanguage{greek}{γαλιλαιαϲ η ονομα ναζαρετ προϲ παρ} & 2 & \textbf{27} &  \\
&  & 2 & \foreignlanguage{greek}{θενον εμνηϲτευμενην ανδρει ω ονο} & 6 &  &  \\
&  & 6 & \foreignlanguage{greek}{μα ιωϲηφ εξ οικου δαυειδ και το ονο} & 13 &  &  \\
&  & 13 & \foreignlanguage{greek}{μα τηϲ παρθενου μαριαμ} & 16 &  &  \\
& \textbf{28} &  & \foreignlanguage{greek}{και ειϲελθων προϲ αυτην ειπεν χαιρε} & 6 &  &  \\
&  & 7 & \foreignlanguage{greek}{κεχαριτωμενη ο \textoverline{κϲ} μετα ϲου} & 11 &  &  \\
& \textbf{29} &  & \foreignlanguage{greek}{η δε επι τω λογω διεταραχθη και διελογι} & 8 &  &  \\
&  & 8 & \foreignlanguage{greek}{ζετο ποταποϲ ειη ο αϲπαϲμοϲ ουτοϲ} & 13 &  &  \\
& \textbf{30} &  & \foreignlanguage{greek}{και ειπεν ο αγγελοϲ αυτη μη φοβου μα} & 8 &  &  \\
&  & 8 & \foreignlanguage{greek}{ριαμ ευρεϲ γαρ χαριν παρα τω \textoverline{θω} και ιδου} & 2 & \textbf{31} &  \\
&  & 3 & \foreignlanguage{greek}{ϲυνληψη εν γαϲτρι και τεξη υιον και} & 9 &  &  \\
&  & 10 & \foreignlanguage{greek}{καλεϲειϲ το ονομα αυτου \textoverline{ιν}} & 14 &  &  \\
& \textbf{32} &  & \foreignlanguage{greek}{αυτοϲ εϲται μεγαϲ και υιοϲ υψιϲτου κλη} & 7 &  &  \\
&  & 7 & \foreignlanguage{greek}{θηϲεται και δωϲη αυτω \textoverline{κϲ} ο \textoverline{θϲ} τον θρο} & 15 &  &  \\
&  & 15 & \foreignlanguage{greek}{νον δαυειδ του πατροϲ αυτου και βαϲι} & 2 & \textbf{33} &  \\
&  & 2 & \foreignlanguage{greek}{λευϲει επι τον οικον ιακωβ ειϲ τουϲ αιωναϲ} & 9 &  &  \\
[0.2em]
\cline{4-4}
\end{tabular}
\end{center}
\end{table}
}
\clearpage
\newpage
 {
 \setlength\arrayrulewidth{1pt}
\begin{table}
\begin{center}
\begin{tabular}{ccc|l|ccc}
\cline{4-4} \\ [-1em]
\multicolumn{7}{c}{\foreignlanguage{greek}{ευαγγελιον κατα λουκαν} \textbf{(\nospace{1:33})} } \\ \\ [-1em] % Si on veut ajouter les bordures latérales, remplacer {7}{c} par {7}{|c|}
\cline{4-4} \\
\cline{4-4}
&  &  & &  &  & \\ [-0.9em]
&  & 10 & \foreignlanguage{greek}{και τηϲ βαϲιλειαϲ αυτου ουκ εϲται τελοϲ} & 16 &  &  \\
& \textbf{34} &  & \foreignlanguage{greek}{ειπεν δε μαριαμ προϲ τον αγγελον πωϲ ε} & 8 &  &  \\
&  & 8 & \foreignlanguage{greek}{ϲτι τουτο επι ανδρα ου γινωϲκω} & 13 &  &  \\
& \textbf{35} &  & \foreignlanguage{greek}{και αποκριθειϲ ο αγγελοϲ ειπεν αυτη} & 6 &  &  \\
&  & 7 & \foreignlanguage{greek}{\textoverline{πνα} αγιον επελευϲεται επι ϲε και δυ} & 13 &  &  \\
&  & 13 & \foreignlanguage{greek}{ναμειϲ υψιϲτου επιϲκιαϲει ϲοι διοτι} & 17 &  &  \\
&  & 18 & \foreignlanguage{greek}{και το γεννωμενον αγιον κληθηϲεται} & 22 &  &  \\
&  & 23 & \foreignlanguage{greek}{υιοϲ \textoverline{θυ}} & 24 &  &  \\
& \textbf{36} &  & \foreignlanguage{greek}{και ιδου ελιϲαβετ η ϲυνγενηϲ ϲου και} & 7 &  &  \\
&  & 8 & \foreignlanguage{greek}{αυτη ϲυνειληφεν υιον εν γηρει αυτηϲ} & 13 &  &  \\
&  & 14 & \foreignlanguage{greek}{και ουτοϲ μην εκτοϲ εϲτιν αυτη τη κα} & 21 &  &  \\
&  & 21 & \foreignlanguage{greek}{λουμενη ϲτειρα οτι ουκ αδυνατηϲει} & 3 & \textbf{37} &  \\
&  & 4 & \foreignlanguage{greek}{παρα του \textoverline{θυ} παν ρημα} & 8 &  &  \\
& \textbf{38} &  & \foreignlanguage{greek}{ειπεν δε μαριαμ ιδου η δουλη \textoverline{κυ} γενοι} & 8 &  &  \\
&  & 8 & \foreignlanguage{greek}{το μοι κατα το ρημα ϲου και απηλθε̅} & 15 &  &  \\
&  & 16 & \foreignlanguage{greek}{απ αυτηϲ ο αγγελοϲ} & 19 &  &  \\
& \textbf{39} &  & \foreignlanguage{greek}{αναϲταϲα δε μαριαμ εν ταιϲ ημεραιϲ ταυ} & 7 &  &  \\
&  & 7 & \foreignlanguage{greek}{ταιϲ επορευθη ειϲ την ορινην μετα} & 12 &  &  \\
&  & 13 & \foreignlanguage{greek}{ϲπουδηϲ ειϲ πολιν ιουδα και ειϲηλθεν} & 2 & \textbf{40} &  \\
&  & 3 & \foreignlanguage{greek}{ειϲ τον οικον ζαχαριου και ηϲπαϲατο} & 8 &  &  \\
&  & 9 & \foreignlanguage{greek}{την ελιϲαβετ} & 10 &  &  \\
& \textbf{41} &  & \foreignlanguage{greek}{και εγενετο ωϲ ηκουϲεν η ελιϲαβετ τον α} & 8 &  &  \\
&  & 8 & \foreignlanguage{greek}{ϲπαϲμον τηϲ μαριαϲ εϲκειρτηϲεν το} & 12 &  &  \\
&  & 13 & \foreignlanguage{greek}{βρεφοϲ εν τη κοιλια αυτηϲ και επληϲθη} & 19 &  &  \\
&  & 20 & \foreignlanguage{greek}{\textoverline{πνϲ} αγιου η ελιϲαβετ και ανεφωνη} & 2 & \textbf{42} &  \\
&  & 2 & \foreignlanguage{greek}{ϲεν κραυγη μεγαλη και ειπεν} & 6 &  &  \\
&  & 7 & \foreignlanguage{greek}{ευλογημενη ϲυ εν γυναιξιν και ευλο} & 12 &  &  \\
&  & 12 & \foreignlanguage{greek}{γημενοϲ ο καρποϲ τηϲ κοιλιαϲ ϲου} & 17 &  &  \\
& \textbf{43} &  & \foreignlanguage{greek}{και ποθεν μοι τουτο ινα ελθη η μητηρ} & 8 &  &  \\
&  & 9 & \foreignlanguage{greek}{του \textoverline{κυ} προϲ με ιδου γαρ ωϲ εγενετο} & 4 & \textbf{44} &  \\
[0.2em]
\cline{4-4}
\end{tabular}
\end{center}
\end{table}
}
\clearpage
\newpage
 {
 \setlength\arrayrulewidth{1pt}
\begin{table}
\begin{center}
\begin{tabular}{ccc|l|ccc}
\cline{4-4} \\ [-1em]
\multicolumn{7}{c}{\foreignlanguage{greek}{ευαγγελιον κατα λουκαν} \textbf{(\nospace{1:44})} } \\ \\ [-1em] % Si on veut ajouter les bordures latérales, remplacer {7}{c} par {7}{|c|}
\cline{4-4} \\
\cline{4-4}
&  &  & &  &  & \\ [-0.9em]
&  & 5 & \foreignlanguage{greek}{η φωνη του αϲπαϲμου ϲου ειϲ τα ωτα μου} & 13 &  &  \\
&  & 14 & \foreignlanguage{greek}{εϲκιρτηϲεν εν αγαλλιαϲει το βρεφοϲ ε̅} & 19 &  &  \\
&  & 20 & \foreignlanguage{greek}{τη κοιλια μου και η καρδια η πιϲτευϲα} & 5 & \textbf{45} &  \\
&  & 5 & \foreignlanguage{greek}{ϲα οτι εϲται τελιωϲειϲ τοιϲ λελαλημε} & 10 &  &  \\
&  & 10 & \foreignlanguage{greek}{νοιϲ αυτη παρα \textoverline{κυ}} & 13 &  &  \\
& \textbf{46} &  & \foreignlanguage{greek}{και ειπεν μαριαμ μεγαλυνει η ψυχη} & 6 &  &  \\
&  & 7 & \foreignlanguage{greek}{μου τον \textoverline{κν} και ηγαλλιαϲεν το \textoverline{πνα} μου} & 5 & \textbf{47} &  \\
&  & 6 & \foreignlanguage{greek}{επι τω \textoverline{θω} τω ϲωτηρι μου οτι επεβλεψε̅} & 2 & \textbf{48} &  \\
&  & 3 & \foreignlanguage{greek}{επι την ταπινωϲιν τηϲ δουληϲ αυτου} & 8 &  &  \\
&  & 9 & \foreignlanguage{greek}{ιδου γαρ απο του νυν μακαριουϲιν με πα} & 16 &  &  \\
&  & 16 & \foreignlanguage{greek}{ϲαι αι γενεαι οτι εποιηϲεν μοι μεγαλα ο} & 5 & \textbf{49} &  \\
&  & 6 & \foreignlanguage{greek}{δυνατοϲ και αγιον το ονομα αυτου} & 11 &  &  \\
& \textbf{50} &  & \foreignlanguage{greek}{και το ελεοϲ αυτου ειϲ γενεαϲ και γενε} & 8 &  &  \\
&  & 8 & \foreignlanguage{greek}{αϲ τοιϲ φοβουμενοιϲ αυτον} & 11 &  &  \\
& \textbf{51} &  & \foreignlanguage{greek}{εποιηϲεν κρατοϲ εν βραχιονι αυτου} & 5 &  &  \\
&  & 6 & \foreignlanguage{greek}{διεϲκορπιϲεν υπερηφανουϲ διανοια} & 8 &  &  \\
&  & 9 & \foreignlanguage{greek}{καρδιαϲ αυτων καθειλεν δυναϲταϲ} & 2 & \textbf{52} &  \\
&  & 3 & \foreignlanguage{greek}{απο θρονων και υψωϲεν ταπινουϲ} & 7 &  &  \\
& \textbf{53} &  & \foreignlanguage{greek}{πινωνταϲ ενεπληϲεν αγαθων} & 3 &  &  \\
&  & 4 & \foreignlanguage{greek}{και πλουτουνταϲ εξαπεϲτιλεν κε} & 7 &  &  \\
&  & 7 & \foreignlanguage{greek}{νουϲ αντελαβετο ιϲραηλ παιδοϲ αυ} & 4 & \textbf{54} &  \\
&  & 4 & \foreignlanguage{greek}{του μνηϲθηναι ελεουϲ καθωϲ ελα} & 2 & \textbf{55} &  \\
&  & 2 & \foreignlanguage{greek}{ληϲεν προϲ τουϲ πατεραϲ ημων τω} & 7 &  &  \\
&  & 8 & \foreignlanguage{greek}{αβρααμ και τω ϲπερματι αυτου} & 12 &  &  \\
&  & 13 & \foreignlanguage{greek}{ειϲ τον αιωνα} & 15 &  &  \\
& \textbf{56} &  & \foreignlanguage{greek}{εμεινεν δε μαριαμ ϲυν αυτη ωϲ μη} & 7 &  &  \\
&  & 7 & \foreignlanguage{greek}{ναϲ τριϲ και υπεϲτρεψεν ειϲ τον} & 12 &  &  \\
&  & 13 & \foreignlanguage{greek}{οικον αυτηϲ} & 14 &  &  \\
& \textbf{57} &  & \foreignlanguage{greek}{τη δε ελιϲαβετ επληϲθη ο χρονοϲ του} & 7 &  &  \\
&  & 8 & \foreignlanguage{greek}{τεκειν αυτην και εγεννηϲεν υιον} & 12 &  &  \\
[0.2em]
\cline{4-4}
\end{tabular}
\end{center}
\end{table}
}
\clearpage
\newpage
 {
 \setlength\arrayrulewidth{1pt}
\begin{table}
\begin{center}
\begin{tabular}{ccc|l|ccc}
\cline{4-4} \\ [-1em]
\multicolumn{7}{c}{\foreignlanguage{greek}{ευαγγελιον κατα λουκαν} \textbf{(\nospace{1:58})} } \\ \\ [-1em] % Si on veut ajouter les bordures latérales, remplacer {7}{c} par {7}{|c|}
\cline{4-4} \\
\cline{4-4}
&  &  & &  &  & \\ [-0.9em]
& \textbf{58} &  & \foreignlanguage{greek}{και ηκουϲαν οι περιοικοι και οι ϲυγγε} & 7 &  &  \\
&  & 7 & \foreignlanguage{greek}{νειϲ αυτηϲ οτι εμεγαλυνεν \textoverline{κϲ} το ε} & 13 &  &  \\
&  & 13 & \foreignlanguage{greek}{λεοϲ αυτου μετ αυτηϲ και ϲυνεχαι} & 18 &  &  \\
&  & 18 & \foreignlanguage{greek}{ρον αυτη} & 19 &  &  \\
& \textbf{59} &  & \foreignlanguage{greek}{και εγενετο εν τη ημερα τη ογδοη} & 7 &  &  \\
&  & 8 & \foreignlanguage{greek}{ηλθον περιτεμειν το παιδιον και} & 12 &  &  \\
&  & 13 & \foreignlanguage{greek}{εκαλουν αυτο επι τω ονοματι του} & 18 &  &  \\
&  & 19 & \foreignlanguage{greek}{πατροϲ αυτου ζαχαριαν} & 21 &  &  \\
& \textbf{60} &  & \foreignlanguage{greek}{και αποκριθιϲα η μητηρ αυτου ειπεν} & 6 &  &  \\
&  & 7 & \foreignlanguage{greek}{ουχι αλλα κληθηϲεται ιωαννηϲ} & 10 &  &  \\
& \textbf{61} &  & \foreignlanguage{greek}{και ειπαν προϲ αυτην οτι ουδειϲ ε} & 7 &  &  \\
&  & 7 & \foreignlanguage{greek}{ϲτιν εκ τηϲ ϲυγγενιαϲ ϲου οϲ καλει} & 13 &  &  \\
&  & 13 & \foreignlanguage{greek}{ται τω ονοματι τουτω} & 16 &  &  \\
& \textbf{62} &  & \foreignlanguage{greek}{ενενευον δε τω πατρι αυτου το τι} & 7 &  &  \\
&  & 8 & \foreignlanguage{greek}{αν θελοι καλειϲθαι αυτον} & 11 &  &  \\
& \textbf{63} &  & \foreignlanguage{greek}{και αιτηϲαϲ πινακιδιον εγραψεν λεγω̅} & 5 &  &  \\
&  & 6 & \foreignlanguage{greek}{ιωαννηϲ εϲτιν το ονομα αυτου και} & 11 &  &  \\
&  & 12 & \foreignlanguage{greek}{εθαυμαϲαν παντεϲ} & 13 &  &  \\
& \textbf{64} &  & \foreignlanguage{greek}{ανεωχθη δε το ϲτομα αυτου παραχρη} & 6 &  &  \\
&  & 6 & \foreignlanguage{greek}{μα και η γλωϲϲα αυτου και ελαλει} & 12 &  &  \\
&  & 13 & \foreignlanguage{greek}{ευλογων τον \textoverline{θν}} & 15 &  &  \\
& \textbf{65} &  & \foreignlanguage{greek}{και εγενετο επι πανταϲ φοβοϲ τουϲ} & 6 &  &  \\
&  & 7 & \foreignlanguage{greek}{περιοικουνταϲ αυτουϲ εν ολη τη} & 11 &  &  \\
&  & 12 & \foreignlanguage{greek}{ορινη τηϲ ιουδαιαϲ και διελαλειτο} & 16 &  &  \\
&  & 17 & \foreignlanguage{greek}{παντα τα ρηματα ταυτα} & 20 &  &  \\
& \textbf{66} &  & \foreignlanguage{greek}{και εθεντο παντεϲ οι ακουϲαντεϲ ε̅} & 6 &  &  \\
&  & 7 & \foreignlanguage{greek}{ταιϲ καρδιαιϲ αυτων λεγοντεϲ} & 10 &  &  \\
&  & 11 & \foreignlanguage{greek}{τι αρα το παιδιον τουτο εϲται και γαρ} & 18 &  &  \\
&  & 19 & \foreignlanguage{greek}{χειρ \textoverline{κυ} ην μετ αυτου} & 23 &  &  \\
& \textbf{67} &  & \foreignlanguage{greek}{και ζαχαριαϲ ο πατηρ αυτου επληϲθη} & 6 &  &  \\
[0.2em]
\cline{4-4}
\end{tabular}
\end{center}
\end{table}
}
\clearpage
\newpage
 {
 \setlength\arrayrulewidth{1pt}
\begin{table}
\begin{center}
\begin{tabular}{ccc|l|ccc}
\cline{4-4} \\ [-1em]
\multicolumn{7}{c}{\foreignlanguage{greek}{ευαγγελιον κατα λουκαν} \textbf{(\nospace{1:67})} } \\ \\ [-1em] % Si on veut ajouter les bordures latérales, remplacer {7}{c} par {7}{|c|}
\cline{4-4} \\
\cline{4-4}
&  &  & &  &  & \\ [-0.9em]
&  & 7 & \foreignlanguage{greek}{\textoverline{πνϲ} αγιου και επροφητευϲεν λεγων} & 11 &  &  \\
& \textbf{68} &  & \foreignlanguage{greek}{ευλογητοϲ ο \textoverline{θϲ} του ιϲραηλ οτι επεϲκε} & 7 &  &  \\
&  & 7 & \foreignlanguage{greek}{ψατο και εποιηϲεν λυτρωϲιν του λαου} & 12 &  &  \\
&  & 13 & \foreignlanguage{greek}{αυτου και ηγειρεν κεραϲ ϲωτηριαϲ ημι̅} & 5 & \textbf{69} &  \\
&  & 6 & \foreignlanguage{greek}{εν οικω δαυειδ παιδοϲ αυτου καθωϲ} & 1 & \textbf{70} &  \\
&  & 2 & \foreignlanguage{greek}{ελαληϲεν δια ϲτοματοϲ των αγιων απ} & 7 &  &  \\
&  & 8 & \foreignlanguage{greek}{αιωνοϲ αυτου προφητων ϲωτηρια̅} & 1 & \textbf{71} &  \\
&  & 2 & \foreignlanguage{greek}{εξ εχθρων ημων και εκ χειροϲ παν} & 8 &  &  \\
&  & 8 & \foreignlanguage{greek}{των των μιϲουντων ημαϲ ποιηϲαι} & 1 & \textbf{72} &  \\
&  & 2 & \foreignlanguage{greek}{ελεοϲ μετα των πατερων ημων και} & 7 &  &  \\
&  & 8 & \foreignlanguage{greek}{μνηϲθηναι διαθηκηϲ αγιαϲ αυτου} & 11 &  &  \\
& \textbf{73} &  & \foreignlanguage{greek}{ορκον ον ωμοϲεν προϲ αβρααμ τον πα} & 7 &  &  \\
&  & 7 & \foreignlanguage{greek}{τερα ημων του δουναι ημιν αφοβωϲ} & 1 & \textbf{74} &  \\
&  & 2 & \foreignlanguage{greek}{εκ χειροϲ εχθρων ρυϲθενταϲ λατρευ} & 6 &  &  \\
&  & 6 & \foreignlanguage{greek}{ειν αυτω εν οϲιοτητι και δικαιοϲυ} & 4 & \textbf{75} &  \\
&  & 4 & \foreignlanguage{greek}{νη ενωπιον αυτου παϲαιϲ ταιϲ ημε} & 9 &  &  \\
&  & 9 & \foreignlanguage{greek}{ραιϲ ημων} & 10 &  &  \\
& \textbf{76} &  & \foreignlanguage{greek}{και ϲυ δε παιδιον προφητηϲ υψιϲτου} & 6 &  &  \\
&  & 7 & \foreignlanguage{greek}{κληθηϲη προπορευϲη γαρ ενωπιο̅} & 10 &  &  \\
&  & 11 & \foreignlanguage{greek}{\textoverline{κυ} ετοιμαϲαι οδουϲ αυτου δουναι γνω} & 2 & \textbf{77} &  \\
&  & 2 & \foreignlanguage{greek}{ϲιν ϲωτηριαϲ τω λαω αυτου εν αφε} & 8 &  &  \\
&  & 8 & \foreignlanguage{greek}{ϲει αμαρτιων αυτου δια ϲπλαγχνα} & 2 & \textbf{78} &  \\
&  & 3 & \foreignlanguage{greek}{ελεουϲ \textoverline{θυ} ημων εν οιϲ επεϲκεψε} & 8 &  &  \\
&  & 8 & \foreignlanguage{greek}{ται ημαϲ ανατολη εξ υψουϲ επιφα} & 1 & \textbf{79} &  \\
&  & 1 & \foreignlanguage{greek}{ναι τοιϲ εν ϲκοτι και ϲκια θανατου} & 7 &  &  \\
&  & 8 & \foreignlanguage{greek}{καθημενοιϲ του κατευθυναι τουϲ} & 11 &  &  \\
&  & 12 & \foreignlanguage{greek}{ποδαϲ ημων ειϲ οδον ειρηνηϲ} & 16 &  &  \\
& \textbf{80} &  & \foreignlanguage{greek}{το δε παιδιον ηυξανεν και εκρατεου} & 6 &  &  \\
&  & 6 & \foreignlanguage{greek}{το \textoverline{πνι} και ην εν ταιϲ ερημοιϲ εωϲ η} & 14 &  &  \\
&  & 14 & \foreignlanguage{greek}{μεραϲ αναδειξεωϲ αυτου προϲ τον ιϲραηλ} & 19 &  &  \\
[0.2em]
\cline{4-4}
\end{tabular}
\end{center}
\end{table}
}
\clearpage
\newpage
 {
 \setlength\arrayrulewidth{1pt}
\begin{table}
\begin{center}
\begin{tabular}{ccc|l|ccc}
\cline{4-4} \\ [-1em]
\multicolumn{7}{c}{\foreignlanguage{greek}{ευαγγελιον κατα λουκαν} \textbf{(\nospace{2:1})} } \\ \\ [-1em] % Si on veut ajouter les bordures latérales, remplacer {7}{c} par {7}{|c|}
\cline{4-4} \\
\cline{4-4}
&  &  & &  &  & \\ [-0.9em]
& \mygospelchapter &  & \foreignlanguage{greek}{εγενετο δε εν ταιϲ ημεραιϲ εκειναιϲ} & 6 &  &  \\
&  & 7 & \foreignlanguage{greek}{εξηλθεν δογμα παρα καιϲαροϲ αυγουϲτου} & 11 &  &  \\
&  & 12 & \foreignlanguage{greek}{του απογραφεϲθαι παϲαν την οικουμενη̅} & 16 &  &  \\
& \textbf{2} &  & \foreignlanguage{greek}{αυτη η απογραφη πρωτη εγενετο ηγε} & 6 &  &  \\
&  & 6 & \foreignlanguage{greek}{μονευοντοϲ τηϲ ϲυριαϲ κυρινου} & 9 &  &  \\
& \textbf{3} &  & \foreignlanguage{greek}{και επορευοντο παντεϲ απογραφεϲθαι} & 4 &  &  \\
&  & 5 & \foreignlanguage{greek}{εκαϲτοϲ ειϲ την εαυτου πολιν} & 9 &  &  \\
& \textbf{4} &  & \foreignlanguage{greek}{ανεβη δε και ιωϲηφ απο τηϲ γαλιλαιαϲ} & 7 &  &  \\
&  & 8 & \foreignlanguage{greek}{εκ πολεωϲ ναζαρετ ειϲ την ιουδαιαν} & 13 &  &  \\
&  & 14 & \foreignlanguage{greek}{ειϲ πολιν δαυειδ ητιϲ καλειται βηθλε} & 19 &  &  \\
&  & 19 & \foreignlanguage{greek}{εμ δια το ειναι αυτον εξ οικου και πα} & 27 &  &  \\
&  & 27 & \foreignlanguage{greek}{τριαϲ δαυειδ απογραφεϲθαι ϲυν μαριαμ} & 3 & \textbf{5} &  \\
&  & 4 & \foreignlanguage{greek}{τη εμνηϲτευμενη αυτω ουϲη νεκυω} & 8 &  &  \\
& \textbf{6} &  & \foreignlanguage{greek}{εγενετο δε εν τω ειναι αυτουϲ εκει επλη} & 8 &  &  \\
&  & 8 & \foreignlanguage{greek}{ϲθηϲαν αι ημεραι του τεκειν αυτην} & 13 &  &  \\
& \textbf{7} &  & \foreignlanguage{greek}{και ετεκεν τον υιον αυτηϲ και εϲπαρ} & 7 &  &  \\
&  & 7 & \foreignlanguage{greek}{γανωϲεν αυτον και ανεκλινεν αυτον} & 11 &  &  \\
&  & 12 & \foreignlanguage{greek}{εν φατνη διοτι ουκ ην αυτοιϲ τοποϲ} & 18 &  &  \\
&  & 19 & \foreignlanguage{greek}{εν τω καταλυματι} & 21 &  &  \\
& \textbf{8} &  & \foreignlanguage{greek}{και ποιμενεϲ ηϲαν εν τη χωρα τη αυτη} & 8 &  &  \\
&  & 9 & \foreignlanguage{greek}{αγραυλουντεϲ και φυλαϲϲοντεϲ φυ} & 12 &  &  \\
&  & 12 & \foreignlanguage{greek}{λακαϲ τηϲ νυκτοϲ επι την ποιμνην} & 17 &  &  \\
&  & 18 & \foreignlanguage{greek}{αυτων και αγγελοϲ \textoverline{κυ} επεϲτη αυτοιϲ} & 5 & \textbf{9} &  \\
&  & 6 & \foreignlanguage{greek}{και δοξα \textoverline{κυ} περιελαμψεν αυτουϲ} & 10 &  &  \\
&  & 11 & \foreignlanguage{greek}{και εφοβηθηϲαν φοβον μεγαν ϲφοδρα} & 15 &  &  \\
& \textbf{10} &  & \foreignlanguage{greek}{και ειπεν αυτοιϲ ο αγγελοϲ μη φοβειϲθαι} & 7 &  &  \\
&  & 8 & \foreignlanguage{greek}{ιδου γαρ ευαγγελιζομαι υμιν χαραν με} & 13 &  &  \\
&  & 13 & \foreignlanguage{greek}{γαλην ητιϲ εϲται παντι τω λαω} & 18 &  &  \\
& \textbf{11} &  & \foreignlanguage{greek}{οτι ετεχθη υμιν ϲημερον ϲωτηρ οϲ} & 6 &  &  \\
&  & 7 & \foreignlanguage{greek}{εϲτιν \textoverline{κϲ} \textoverline{χϲ} εν πολει δαυειδ και τουτο} & 2 & \textbf{12} &  \\
[0.2em]
\cline{4-4}
\end{tabular}
\end{center}
\end{table}
}
\clearpage
\newpage
 {
 \setlength\arrayrulewidth{1pt}
\begin{table}
\begin{center}
\begin{tabular}{ccc|l|ccc}
\cline{4-4} \\ [-1em]
\multicolumn{7}{c}{\foreignlanguage{greek}{ευαγγελιον κατα λουκαν} \textbf{(\nospace{2:12})} } \\ \\ [-1em] % Si on veut ajouter les bordures latérales, remplacer {7}{c} par {7}{|c|}
\cline{4-4} \\
\cline{4-4}
&  &  & &  &  & \\ [-0.9em]
&  & 3 & \foreignlanguage{greek}{υμιν το ϲημιον ευρηϲεται βρεφοϲ εϲπαρ} & 8 &  &  \\
&  & 8 & \foreignlanguage{greek}{γανωμενον και κειμενον εν φατνη} & 12 &  &  \\
& \textbf{13} &  & \foreignlanguage{greek}{και εξεφνηϲ εγενετο ϲυν τω αγγελω} & 6 &  &  \\
&  & 7 & \foreignlanguage{greek}{πληθοϲ ϲτρατιαϲ ουρανιου αινουντω̅} & 10 &  &  \\
&  & 11 & \foreignlanguage{greek}{τον \textoverline{θν} και λεγοντων δοξα εν υψιϲτοιϲ \textoverline{θω}} & 4 & \textbf{14} &  \\
&  & 5 & \foreignlanguage{greek}{και επι γηϲ ειρηνη εν \textoverline{ανοιϲ} ευδοκειαϲ} & 11 &  &  \\
& \textbf{15} &  & \foreignlanguage{greek}{και εγενετο ωϲ απηλθον απ αυτων ειϲ το̅} & 8 &  &  \\
&  & 9 & \foreignlanguage{greek}{ουρανον οι αγγελοι οι ποιμενεϲ ελαλου̅} & 14 &  &  \\
&  & 15 & \foreignlanguage{greek}{προϲ αλληλουϲ διελθωμεν δη εωϲ βη} & 20 &  &  \\
&  & 20 & \foreignlanguage{greek}{θλεεμ και ιδωμεν το ρημα τουτο το γε} & 27 &  &  \\
&  & 27 & \foreignlanguage{greek}{γονοϲ ο ο \textoverline{κϲ} εγνωριϲεν ημιν} & 32 &  &  \\
& \textbf{16} &  & \foreignlanguage{greek}{και ηλθον ϲπευϲαντεϲ και ευρον την τε} & 7 &  &  \\
&  & 8 & \foreignlanguage{greek}{μαριαμ και τον ιωϲηφ και το βρεφοϲ} & 14 &  &  \\
&  & 15 & \foreignlanguage{greek}{κειμενον εν τη φατνη ιδοντεϲ δε ε} & 3 & \textbf{17} &  \\
&  & 3 & \foreignlanguage{greek}{γνωριϲαν περι του ρηματοϲ του λαλη} & 8 &  &  \\
&  & 8 & \foreignlanguage{greek}{θεντοϲ αυτοιϲ περι του παιδιου τουτου} & 13 &  &  \\
& \textbf{18} &  & \foreignlanguage{greek}{και παντεϲ οι ακουϲαντεϲ εθαυμαϲα̅} & 5 &  &  \\
&  & 6 & \foreignlanguage{greek}{περι των λαληθεντων υπο των ποι} & 11 &  &  \\
&  & 11 & \foreignlanguage{greek}{μενων προϲ αυτουϲ} & 13 &  &  \\
& \textbf{19} &  & \foreignlanguage{greek}{η δε μαριαμ παντα ϲυνετηρει τα ρημα} & 7 &  &  \\
&  & 7 & \foreignlanguage{greek}{τα ταυτα ϲυνβαλλουϲα εν τη καρδια} & 12 &  &  \\
&  & 13 & \foreignlanguage{greek}{αυτηϲ και υπεϲτρεψαν οι ποιμε} & 4 & \textbf{20} &  \\
&  & 4 & \foreignlanguage{greek}{νεϲ δοξαζοντεϲ και αινουντεϲ τον \textoverline{θν}} & 9 &  &  \\
&  & 10 & \foreignlanguage{greek}{επι παϲιν οιϲ ηκουϲαν και ιδον καθωϲ} & 16 &  &  \\
&  & 17 & \foreignlanguage{greek}{ελαληθη προϲ αυτουϲ} & 19 &  &  \\
& \textbf{21} &  & \foreignlanguage{greek}{και οτε επληϲθηϲαν ημεραι οκτω του} & 6 &  &  \\
&  & 7 & \foreignlanguage{greek}{περιτεμιν αυτον και εκληθη το ο} & 12 &  &  \\
&  & 12 & \foreignlanguage{greek}{νομα αυτου \textoverline{ιϲ} το κληθεν υπο του} & 18 &  &  \\
&  & 19 & \foreignlanguage{greek}{αγγελου προ του ϲυνλημφθηναι αυ} & 23 &  &  \\
&  & 23 & \foreignlanguage{greek}{τον εν τη κοιλια} & 26 &  &  \\
[0.2em]
\cline{4-4}
\end{tabular}
\end{center}
\end{table}
}
\clearpage
\newpage
 {
 \setlength\arrayrulewidth{1pt}
\begin{table}
\begin{center}
\begin{tabular}{ccc|l|ccc}
\cline{4-4} \\ [-1em]
\multicolumn{7}{c}{\foreignlanguage{greek}{ευαγγελιον κατα λουκαν} \textbf{(\nospace{2:22})} } \\ \\ [-1em] % Si on veut ajouter les bordures latérales, remplacer {7}{c} par {7}{|c|}
\cline{4-4} \\
\cline{4-4}
&  &  & &  &  & \\ [-0.9em]
& \textbf{22} &  & \foreignlanguage{greek}{και οτε επληϲθηϲαν αι ημεραι του κα} & 7 &  &  \\
&  & 7 & \foreignlanguage{greek}{θαριϲμου αυτων κατα τον νομον μω} & 12 &  &  \\
&  & 12 & \foreignlanguage{greek}{υϲεωϲ ανηγαγον αυτον ειϲ ιεροϲολυμα} & 16 &  &  \\
&  & 17 & \foreignlanguage{greek}{παραϲτηϲαι τω \textoverline{κω} καθωϲ γεγραπται} & 2 & \textbf{23} &  \\
&  & 3 & \foreignlanguage{greek}{εν νομω \textoverline{κυ} οτι παν αρϲεν διανοι} & 9 &  &  \\
&  & 9 & \foreignlanguage{greek}{γον μητραν αγιον τω \textoverline{κω} κληθηϲεται} & 14 &  &  \\
& \textbf{24} &  & \foreignlanguage{greek}{και του δουναι θυϲιαν κατα το ειρη} & 7 &  &  \\
&  & 7 & \foreignlanguage{greek}{μενον εν τω νομω \textoverline{κυ} ζευγοϲ τρυ} & 13 &  &  \\
&  & 13 & \foreignlanguage{greek}{γονων η δυο νοϲϲουϲ περιϲτερων} & 17 &  &  \\
& \textbf{25} &  & \foreignlanguage{greek}{και ιδου ανθρωποϲ ην εν ιερουϲαλημ} & 6 &  &  \\
&  & 7 & \foreignlanguage{greek}{ω ονομα ϲυμεων και ο ανθρωποϲ} & 12 &  &  \\
&  & 13 & \foreignlanguage{greek}{ουτοϲ δικαιοϲ και ευλαβηϲ προϲδε} & 17 &  &  \\
&  & 17 & \foreignlanguage{greek}{χομενοϲ παρακληϲιν του ιϲραηλ και} & 21 &  &  \\
&  & 22 & \foreignlanguage{greek}{\textoverline{πνα} ην αγιον επ αυτον και ην αυτω} & 3 & \textbf{26} &  \\
&  & 4 & \foreignlanguage{greek}{κεχρηματιϲμενον υπο του \textoverline{πνϲ} του} & 8 &  &  \\
&  & 9 & \foreignlanguage{greek}{αγιου μη ιδιν θανατον πριν ειδη} & 14 &  &  \\
&  & 15 & \foreignlanguage{greek}{\textoverline{χν} \textoverline{κυ} και ηλθεν εν τω \textoverline{πνι} ειϲ το} & 7 & \textbf{27} &  \\
&  & 8 & \foreignlanguage{greek}{ιερον και εν τω ειϲαγειν τουϲ γονειϲ} & 14 &  &  \\
&  & 15 & \foreignlanguage{greek}{το παιδιον \textoverline{ιν} του ποιηϲαι αυτουϲ κα} & 21 &  &  \\
&  & 21 & \foreignlanguage{greek}{τα το ειθειϲμενον του νομου περι} & 26 &  &  \\
&  & 27 & \foreignlanguage{greek}{αυτου και αυτοϲ εδεξατο αυτο} & 4 & \textbf{28} &  \\
&  & 5 & \foreignlanguage{greek}{ειϲ ταϲ ανκαλαϲ και ηυλογηϲεν τον} & 10 &  &  \\
&  & 11 & \foreignlanguage{greek}{\textoverline{θν} και ειπεν} & 13 &  &  \\
& \textbf{29} &  & \foreignlanguage{greek}{νυν απολυειϲ τον δουλον ϲου δεϲ} & 6 &  &  \\
&  & 6 & \foreignlanguage{greek}{ποτα κατα το ρημα ϲου εν ειρηνη} & 12 &  &  \\
& \textbf{30} &  & \foreignlanguage{greek}{οτι ειδον οι οφθαλμοι μου το ϲωτη} & 7 &  &  \\
&  & 7 & \foreignlanguage{greek}{ριον ϲου ο ητοιμαϲαϲ κατα προϲω} & 4 & \textbf{31} &  \\
&  & 4 & \foreignlanguage{greek}{πον παντων των λαων φωϲ ειϲ α} & 3 & \textbf{32} &  \\
&  & 3 & \foreignlanguage{greek}{ποκαλυψιν εθνων και δοξαν λα} & 7 &  &  \\
&  & 7 & \foreignlanguage{greek}{ου ϲου ιϲραηλ και ην ο πατηρ αυ} & 5 & \textbf{33} &  \\
[0.2em]
\cline{4-4}
\end{tabular}
\end{center}
\end{table}
}
\clearpage
\newpage
 {
 \setlength\arrayrulewidth{1pt}
\begin{table}
\begin{center}
\begin{tabular}{ccc|l|ccc}
\cline{4-4} \\ [-1em]
\multicolumn{7}{c}{\foreignlanguage{greek}{ευαγγελιον κατα λουκαν} \textbf{(\nospace{2:33})} } \\ \\ [-1em] % Si on veut ajouter les bordures latérales, remplacer {7}{c} par {7}{|c|}
\cline{4-4} \\
\cline{4-4}
&  &  & &  &  & \\ [-0.9em]
&  & 5 & \foreignlanguage{greek}{του και η μητηρ θαυμαζοντεϲ επι τοιϲ} & 11 &  &  \\
&  & 12 & \foreignlanguage{greek}{λαλουμενοιϲ περι αυτου} & 14 &  &  \\
& \textbf{34} &  & \foreignlanguage{greek}{και ηυλογηϲεν αυτουϲ ϲυμεων και} & 5 &  &  \\
&  & 6 & \foreignlanguage{greek}{ειπεν προϲ μαριαμ την \textoverline{μρα} αυτου} & 11 &  &  \\
&  & 12 & \foreignlanguage{greek}{ιδου ουτοϲ κειται ειϲ πτωϲιν και α} & 18 &  &  \\
&  & 18 & \foreignlanguage{greek}{ναϲταϲιν πολλων εν τω ιϲραηλ} & 22 &  &  \\
&  & 23 & \foreignlanguage{greek}{και ειϲ ϲημειον αντιλεγομενον και} & 1 & \textbf{35} &  \\
&  & 2 & \foreignlanguage{greek}{ϲου αυτηϲ την ψυχην διελευϲεται} & 6 &  &  \\
&  & 7 & \foreignlanguage{greek}{ρομφαια οπωϲ αν αποκαλυφθωϲιν} & 10 &  &  \\
&  & 11 & \foreignlanguage{greek}{εκ πολλων καρδιων διαλογιϲμοι} & 14 &  &  \\
& \textbf{36} &  & \foreignlanguage{greek}{και ην αννα προφητιϲ θυγατηρ φανου} & 6 &  &  \\
&  & 6 & \foreignlanguage{greek}{ηλ εκ φυληϲ αϲηρ αυτη προβεβηκυ} & 11 &  &  \\
&  & 11 & \foreignlanguage{greek}{ια εν ημεραιϲ πολλαιϲ ζηϲαϲα μετα α̅} & 17 &  &  \\
&  & 17 & \foreignlanguage{greek}{δροϲ ετη \textoverline{ζ} απο τηϲ παρθενιαϲ αυτηϲ} & 23 &  &  \\
& \textbf{37} &  & \foreignlanguage{greek}{και ην αυτη χηρα ωϲ ετων \textoverline{π} \textoverline{δ} η ουκ α} & 11 &  &  \\
&  & 11 & \foreignlanguage{greek}{φιϲτατο του ιερου νηϲτιαιϲ τε και} & 16 &  &  \\
&  & 17 & \foreignlanguage{greek}{δεηϲεϲιν λατρευουϲα νυκτα και η} & 21 &  &  \\
&  & 21 & \foreignlanguage{greek}{μεραν και αυτη τη ωρα επιϲταϲα} & 5 & \textbf{38} &  \\
&  & 6 & \foreignlanguage{greek}{ανθωμολογειτο τω \textoverline{θω} και ελαλει πε} & 11 &  &  \\
&  & 11 & \foreignlanguage{greek}{ρι αυτου παϲι τοιϲ προϲδεχομενοιϲ} & 15 &  &  \\
&  & 16 & \foreignlanguage{greek}{λυτρωϲιν ιερουϲαλημ} & 17 &  &  \\
& \textbf{39} &  & \foreignlanguage{greek}{και ωϲ ετελεϲαν παντα τα κατα τον νο} & 8 &  &  \\
&  & 8 & \foreignlanguage{greek}{μον \textoverline{κυ} επεϲτρεψαν ειϲ την γαλιλαι} & 13 &  &  \\
&  & 13 & \foreignlanguage{greek}{αν ειϲ πολιν εαυτων ναζαρετ} & 17 &  &  \\
& \textbf{40} &  & \foreignlanguage{greek}{το δε παιδιον ηυξανεν και εκραται} & 6 &  &  \\
&  & 6 & \foreignlanguage{greek}{ουτο πληρουμενον ϲοφια και χα} & 10 &  &  \\
&  & 10 & \foreignlanguage{greek}{ριϲ \textoverline{θυ} ην επ αυτο} & 14 &  &  \\
& \textbf{41} &  & \foreignlanguage{greek}{και επορευοντο οι γονειϲ αυτου καθ ε} & 7 &  &  \\
&  & 7 & \foreignlanguage{greek}{τοϲ ειϲ ιερουϲαλημ τη εορτη του παϲχα} & 13 &  &  \\
& \textbf{42} &  & \foreignlanguage{greek}{και οτε εγενετο ετων δεκαδυο ανα} & 6 &  &  \\
[0.2em]
\cline{4-4}
\end{tabular}
\end{center}
\end{table}
}
\clearpage
\newpage
 {
 \setlength\arrayrulewidth{1pt}
\begin{table}
\begin{center}
\begin{tabular}{ccc|l|ccc}
\cline{4-4} \\ [-1em]
\multicolumn{7}{c}{\foreignlanguage{greek}{ευαγγελιον κατα λουκαν} \textbf{(\nospace{2:42})} } \\ \\ [-1em] % Si on veut ajouter les bordures latérales, remplacer {7}{c} par {7}{|c|}
\cline{4-4} \\
\cline{4-4}
&  &  & &  &  & \\ [-0.9em]
&  & 6 & \foreignlanguage{greek}{βαινοντων αυτων κατα το εθοϲ τηϲ εορ} & 12 &  &  \\
&  & 12 & \foreignlanguage{greek}{τηϲ και τελιωϲαντων ταϲ ημεραϲ εν} & 5 & \textbf{43} &  \\
&  & 6 & \foreignlanguage{greek}{τω υποϲτρεφειν αυτουϲ υπεμεινεν} & 9 &  &  \\
&  & 10 & \foreignlanguage{greek}{\textoverline{ιϲ} ο παιϲ εν ιερουϲαλημ και ουκ εγνω} & 17 &  &  \\
&  & 17 & \foreignlanguage{greek}{ϲαν οι γονειϲ αυτου νομιϲαντεϲ δε} & 2 & \textbf{44} &  \\
&  & 3 & \foreignlanguage{greek}{αυτον ειναι εν τη ϲυνοδια ηλθον η} & 9 &  &  \\
&  & 9 & \foreignlanguage{greek}{μεραϲ οδον και ανεζητουν αυτον} & 13 &  &  \\
&  & 14 & \foreignlanguage{greek}{εν τοιϲ ϲυγγενευϲιν και τοιϲ γνωϲτοιϲ} & 19 &  &  \\
& \textbf{45} &  & \foreignlanguage{greek}{και μη ευροντεϲ επεϲτρεψαν ειϲ ιε} & 6 &  &  \\
&  & 6 & \foreignlanguage{greek}{ρουϲαλημ αναζητουντεϲ αυτον} & 8 &  &  \\
& \textbf{46} &  & \foreignlanguage{greek}{και εγενετο μετα ημεραϲ τριϲ ευρον} & 6 &  &  \\
&  & 7 & \foreignlanguage{greek}{αυτον εν τω ιερω καθεζομενον εν} & 12 &  &  \\
&  & 13 & \foreignlanguage{greek}{μεϲω των διδαϲκαλων και ακουον} & 17 &  &  \\
&  & 17 & \foreignlanguage{greek}{τα αυτων και επερωτωντα αυτουϲ} & 21 &  &  \\
& \textbf{47} &  & \foreignlanguage{greek}{εξιϲταντο δε παντεϲ επι τη ϲυνεϲει} & 6 &  &  \\
&  & 7 & \foreignlanguage{greek}{και ταιϲ αποκριϲεϲιν αυτου και ειδο̅} & 2 & \textbf{48} &  \\
&  & 2 & \foreignlanguage{greek}{τεϲ αυτον εξεπλαγηϲαν} & 4 &  &  \\
&  & 5 & \foreignlanguage{greek}{και ειπεν προϲ αυτον η μητηρ αυτου} & 11 &  &  \\
&  & 12 & \foreignlanguage{greek}{τεκνον τι εποιηϲαϲ ημιν ουτωϲ} & 16 &  &  \\
&  & 17 & \foreignlanguage{greek}{ιδου ο πατηρ ϲου καγω οδυνουμε} & 22 &  &  \\
&  & 22 & \foreignlanguage{greek}{νοι εζητουμεν ϲε} & 24 &  &  \\
& \textbf{49} &  & \foreignlanguage{greek}{και ειπεν προϲ αυτουϲ τι οτι ζητειτε} & 7 &  &  \\
&  & 8 & \foreignlanguage{greek}{με ουκ οιδατε οτι εν τοιϲ του πατροϲ} & 15 &  &  \\
&  & 16 & \foreignlanguage{greek}{δει με ειναι και αυτοι ου ϲυνηκαν} & 4 & \textbf{50} &  \\
&  & 5 & \foreignlanguage{greek}{το ρημα ο ελαληϲεν αυτοιϲ} & 9 &  &  \\
& \textbf{51} &  & \foreignlanguage{greek}{και κατεβη μετ αυτων και ηλθεν ειϲ} & 7 &  &  \\
&  & 8 & \foreignlanguage{greek}{ναζαρετ και ην υποταϲϲομενοϲ αυτοιϲ} & 12 &  &  \\
&  & 13 & \foreignlanguage{greek}{και η μητηρ αυτου ετηρει παντα τα} & 19 &  &  \\
&  & 20 & \foreignlanguage{greek}{ρηματα εν τη καρδια αυτηϲ} & 24 &  &  \\
& \textbf{52} &  & \foreignlanguage{greek}{και ο \textoverline{ιϲ} προεκοπτεν τη ϲοφια ϗ ηλικια} & 8 &  &  \\
[0.2em]
\cline{4-4}
\end{tabular}
\end{center}
\end{table}
}
\clearpage
\newpage
 {
 \setlength\arrayrulewidth{1pt}
\begin{table}
\begin{center}
\begin{tabular}{ccc|l|ccc}
\cline{4-4} \\ [-1em]
\multicolumn{7}{c}{\foreignlanguage{greek}{ευαγγελιον κατα λουκαν} \textbf{(\nospace{2:52})} } \\ \\ [-1em] % Si on veut ajouter les bordures latérales, remplacer {7}{c} par {7}{|c|}
\cline{4-4} \\
\cline{4-4}
&  &  & &  &  & \\ [-0.9em]
&  & 9 & \foreignlanguage{greek}{και χαριτι παρα \textoverline{θω} και ανθρωποιϲ} & 14 &  &  \\
& \mygospelchapter &  & \foreignlanguage{greek}{εν ετι δε πεντεκαιδεκατω τηϲ ηγεμο} & 6 &  &  \\
&  & 6 & \foreignlanguage{greek}{νιαϲ τιβαιριου καιϲαροϲ ηγεμονευο̅} & 9 &  &  \\
&  & 9 & \foreignlanguage{greek}{τοϲ ποντιου πειλατου τηϲ ιουδαιαϲ} & 13 &  &  \\
&  & 14 & \foreignlanguage{greek}{και τετραρχουντοϲ τηϲ γαλιλαιαϲ} & 17 &  &  \\
&  & 18 & \foreignlanguage{greek}{ηρωδου φιλιππου δε του αδελφου} & 22 &  &  \\
&  & 23 & \foreignlanguage{greek}{αυτου τετραρχουντοϲ τηϲ ιουδαιαϲ} & 26 &  &  \\
&  & 27 & \foreignlanguage{greek}{και τραχωνιτιδοϲ χωραϲ λυϲανιου} & 30 &  &  \\
&  & 31 & \foreignlanguage{greek}{τηϲ αβιληνηϲ τετραρχουντοϲ επι} & 1 & \textbf{2} &  \\
&  & 2 & \foreignlanguage{greek}{αρχιερεωϲ αννα και καιαφα} & 5 &  &  \\
&  & 6 & \foreignlanguage{greek}{εγενετο ρημα \textoverline{θυ} επι ιωαννην τον} & 11 &  &  \\
&  & 12 & \foreignlanguage{greek}{ζαχαριου υιον εν τη ερημω και ηλ} & 2 & \textbf{3} &  \\
&  & 2 & \foreignlanguage{greek}{θεν ειϲ παϲαν περιχωρον του ιορδα} & 7 &  &  \\
&  & 7 & \foreignlanguage{greek}{νου κηρυϲϲων βαπτιϲμα μετανοι} & 10 &  &  \\
&  & 10 & \foreignlanguage{greek}{αϲ ειϲ αφεϲιν αμαρτιων} & 13 &  &  \\
& \textbf{4} &  & \foreignlanguage{greek}{ωϲ γεγραπται εν βιβλω λογων ηϲαιου} & 6 &  &  \\
&  & 7 & \foreignlanguage{greek}{του προφητου φωνη βοωντοϲ εν} & 11 &  &  \\
&  & 12 & \foreignlanguage{greek}{τη ερημω ετοιμαϲατε την οδον \textoverline{κυ}} & 17 &  &  \\
&  & 18 & \foreignlanguage{greek}{ευθειαϲ ποιειται ταϲ τριβουϲ αυτου} & 22 &  &  \\
& \textbf{5} &  & \foreignlanguage{greek}{παϲα φαραγξ πληρωθηϲεται και πα̅} & 5 &  &  \\
&  & 6 & \foreignlanguage{greek}{οροϲ και βουνοϲ ταπινωθηϲεται} & 9 &  &  \\
&  & 10 & \foreignlanguage{greek}{και εϲται τα ϲκολεια ειϲ ευθειαν και} & 16 &  &  \\
&  & 17 & \foreignlanguage{greek}{αι τραχειαι ειϲ οδουϲ λειαϲ και οψεται} & 2 & \textbf{6} &  \\
&  & 3 & \foreignlanguage{greek}{παϲα ϲαρξ το ϲωτηριον του \textoverline{θυ}} & 8 &  &  \\
& \textbf{7} &  & \foreignlanguage{greek}{ελεγεν ουν τοιϲ εκπορευομενοιϲ ο} & 5 &  &  \\
&  & 5 & \foreignlanguage{greek}{χλοιϲ βαπτιϲθηναι υπ αυτου} & 8 &  &  \\
&  & 9 & \foreignlanguage{greek}{γεννηματα εχιδνων τιϲ υπεδειξε̅} & 12 &  &  \\
&  & 13 & \foreignlanguage{greek}{υμιν απο τηϲ μελλουϲηϲ ποιηϲατε} & 1 & \textbf{8} &  \\
&  & 2 & \foreignlanguage{greek}{ουν καρπον αξιον τηϲ μετανοιαϲ} & 6 &  &  \\
&  & 7 & \foreignlanguage{greek}{και μη αρξηϲθαι λεγειν εν εαυτοιϲ} & 12 &  &  \\
[0.2em]
\cline{4-4}
\end{tabular}
\end{center}
\end{table}
}
\clearpage
\newpage
 {
 \setlength\arrayrulewidth{1pt}
\begin{table}
\begin{center}
\begin{tabular}{ccc|l|ccc}
\cline{4-4} \\ [-1em]
\multicolumn{7}{c}{\foreignlanguage{greek}{ευαγγελιον κατα λουκαν} \textbf{(\nospace{3:8})} } \\ \\ [-1em] % Si on veut ajouter les bordures latérales, remplacer {7}{c} par {7}{|c|}
\cline{4-4} \\
\cline{4-4}
&  &  & &  &  & \\ [-0.9em]
&  & 13 & \foreignlanguage{greek}{πατερα εχομεν τον αβρααμ λεγω γαρ υ} & 19 &  &  \\
&  & 19 & \foreignlanguage{greek}{μιν οτι δυναται ο \textoverline{θϲ} εκ των λιθων τουτω̅} & 27 &  &  \\
&  & 28 & \foreignlanguage{greek}{εγειρε τεκνα τω αβρααμ} & 31 &  &  \\
& \textbf{9} &  & \foreignlanguage{greek}{ηδη δε και η αξινη προϲ την ριζαν των} & 9 &  &  \\
&  & 10 & \foreignlanguage{greek}{δενδρων κειται παν ουν δενδρον} & 14 &  &  \\
&  & 15 & \foreignlanguage{greek}{μη ποιουν καρπον καλον εκκοπτεται} & 19 &  &  \\
&  & 20 & \foreignlanguage{greek}{και ειϲ πυρ βαλλεται} & 23 &  &  \\
& \textbf{10} &  & \foreignlanguage{greek}{και επηρωηϲαν αυτον οι οχλοι λεγοντεϲ} & 6 &  &  \\
&  & 7 & \foreignlanguage{greek}{τι ουν ποιηϲωμεν αποκριθειϲ δε} & 2 & \textbf{11} &  \\
&  & 3 & \foreignlanguage{greek}{ειπεν αυτοιϲ ο εχων δυο χειτωναϲ} & 8 &  &  \\
&  & 9 & \foreignlanguage{greek}{μεταδοτω τω μη εχοντι και ο εχων} & 15 &  &  \\
&  & 16 & \foreignlanguage{greek}{βρωματα ομοιωϲ ποιειτω} & 18 &  &  \\
& \textbf{12} &  & \foreignlanguage{greek}{ηλθον δε και τελωναι βαπτιϲθηναι} & 5 &  &  \\
&  & 6 & \foreignlanguage{greek}{και ειπαν προϲ αυτον διδαϲκαλε τι} & 11 &  &  \\
&  & 12 & \foreignlanguage{greek}{ποιηϲωμεν ο δε ειπεν προϲ αυτουϲ} & 5 & \textbf{13} &  \\
&  & 6 & \foreignlanguage{greek}{μηδεν πλεον παρα το διατεταγμενο̅} & 10 &  &  \\
&  & 11 & \foreignlanguage{greek}{υμιν πραϲϲεται} & 12 &  &  \\
& \textbf{14} &  & \foreignlanguage{greek}{επηρωτων δε αυτον και ϲτρατευο} & 5 &  &  \\
&  & 5 & \foreignlanguage{greek}{μενοι λεγοντεϲ τι ποιηϲωμεν ϗ ημειϲ} & 10 &  &  \\
&  & 11 & \foreignlanguage{greek}{και ειπεν προϲ αυτουϲ μηδενα δια} & 16 &  &  \\
&  & 16 & \foreignlanguage{greek}{ϲειϲηται μηδε ϲυκοφαντηϲηται} & 18 &  &  \\
&  & 19 & \foreignlanguage{greek}{και αρκειϲθαι τοιϲ οψωνιοιϲ υμων} & 23 &  &  \\
& \textbf{15} &  & \foreignlanguage{greek}{προϲδοκωντοϲ δε του λαου και διαλο} & 6 &  &  \\
&  & 6 & \foreignlanguage{greek}{γιζομενων παντων εν ταιϲ καρδιαιϲ} & 10 &  &  \\
&  & 11 & \foreignlanguage{greek}{αυτων περι του ιωαννου μηποτε αυ} & 16 &  &  \\
&  & 16 & \foreignlanguage{greek}{τοϲ ειη ο \textoverline{χϲ}} & 19 &  &  \\
& \textbf{16} &  & \foreignlanguage{greek}{απεκρινατο λεγων παϲιν ο ιωαννηϲ} & 5 &  &  \\
&  & 6 & \foreignlanguage{greek}{εγω μεν υδατι βαπτιζω υμαϲ ερχε} & 11 &  &  \\
&  & 11 & \foreignlanguage{greek}{ται δε ο ιϲχυροτεροϲ μου ου ουκ ειμει} & 18 &  &  \\
&  & 19 & \foreignlanguage{greek}{εικανοϲ λυϲαι τον ιμαντα των υπο} & 24 &  &  \\
[0.2em]
\cline{4-4}
\end{tabular}
\end{center}
\end{table}
}
\clearpage
\newpage
 {
 \setlength\arrayrulewidth{1pt}
\begin{table}
\begin{center}
\begin{tabular}{ccc|l|ccc}
\cline{4-4} \\ [-1em]
\multicolumn{7}{c}{\foreignlanguage{greek}{ευαγγελιον κατα λουκαν} \textbf{(\nospace{3:16})} } \\ \\ [-1em] % Si on veut ajouter les bordures latérales, remplacer {7}{c} par {7}{|c|}
\cline{4-4} \\
\cline{4-4}
&  &  & &  &  & \\ [-0.9em]
&  & 24 & \foreignlanguage{greek}{δηματων αυτου αυτοϲ υμαϲ βαπτι} & 28 &  &  \\
&  & 28 & \foreignlanguage{greek}{ϲει εν \textoverline{πνι} αγιω και πυρι ου το πτυον εν} & 4 & \textbf{17} &  \\
&  & 5 & \foreignlanguage{greek}{τη χειρι αυτου και διακαθαριει την α} & 11 &  &  \\
&  & 11 & \foreignlanguage{greek}{λωνα αυτου και ϲυναξει τον ϲειτον} & 16 &  &  \\
&  & 17 & \foreignlanguage{greek}{ειϲ την αποθηκην αυτου το δε αχυρο̅} & 23 &  &  \\
&  & 24 & \foreignlanguage{greek}{κατακαυϲει πυρι αϲβεϲτω} & 26 &  &  \\
& \textbf{18} &  & \foreignlanguage{greek}{πολλα μεν ουν και ετερα παρακαλων} & 6 &  &  \\
&  & 7 & \foreignlanguage{greek}{ευηγγελειζετο τον λαον} & 9 &  &  \\
& \textbf{19} &  & \foreignlanguage{greek}{ο δε ηρωδηϲ ο τετραρχηϲ ελεγχομενοϲ} & 6 &  &  \\
&  & 7 & \foreignlanguage{greek}{υπ αυτου περι ηρωδιαδοϲ τηϲ γυναι} & 12 &  &  \\
&  & 12 & \foreignlanguage{greek}{κοϲ φιλιππου του αδελφου αυτου} & 16 &  &  \\
&  & 17 & \foreignlanguage{greek}{και περι παντων των πονηρων ων εποι} & 23 &  &  \\
&  & 23 & \foreignlanguage{greek}{ηϲεν ο ηρωδηϲ προϲεθηκεν και του} & 3 & \textbf{20} &  \\
&  & 3 & \foreignlanguage{greek}{το επι παϲιν και κατεκλειϲεν τον ι} & 9 &  &  \\
&  & 9 & \foreignlanguage{greek}{ωαννην εν τη φυλακη} & 12 &  &  \\
& \textbf{21} &  & \foreignlanguage{greek}{εγενετο δε εν τω βαπτιϲθηναι παν} & 6 &  &  \\
&  & 6 & \foreignlanguage{greek}{τα τον λαον και \textoverline{ιυ} βαπτιϲθεντοϲ} & 11 &  &  \\
&  & 12 & \foreignlanguage{greek}{και προϲευχομενου ανεωχθηναι} & 14 &  &  \\
&  & 15 & \foreignlanguage{greek}{τον ουρανον και καταβηναι το \textoverline{πνα}} & 4 & \textbf{22} &  \\
&  & 5 & \foreignlanguage{greek}{το αγιον ϲωματικω ειδι ωϲ περι} & 10 &  &  \\
&  & 10 & \foreignlanguage{greek}{ϲτεραν επ αυτον} & 12 &  &  \\
&  & 13 & \foreignlanguage{greek}{και φωνην εξ ουρανου γενεϲθαι} & 17 &  &  \\
&  & 18 & \foreignlanguage{greek}{ϲυ ει ο υιοϲ μου ο αγαπητοϲ εν ϲοι} & 26 &  &  \\
&  & 27 & \foreignlanguage{greek}{ηυδοκηϲα και αυτοϲ ην \textoverline{ιϲ} αρ} & 5 & \textbf{23} &  \\
&  & 5 & \foreignlanguage{greek}{χομενοϲ ωϲει ετων \textoverline{λ} ων υιοϲ ωϲ} & 11 &  &  \\
&  & 12 & \foreignlanguage{greek}{ενομιζετο ιωϲηφ} & 13 &  &  \\
& \mygospelchapter &  & \foreignlanguage{greek}{\textoverline{ιϲ} δε πληρηϲ \textoverline{πνϲ} αγιου υπεϲτρεψεν} & 6 &  &  \\
&  & 7 & \foreignlanguage{greek}{απο του ιορδανου και ηγετο εν τω} & 13 &  &  \\
&  & 14 & \foreignlanguage{greek}{\textoverline{πνι} εν τη ερημω ημεραϲ \textoverline{μ} πειρα} & 3 & \textbf{2} &  \\
&  & 3 & \foreignlanguage{greek}{ζομενοϲ υπο του διαβολου} & 6 &  &  \\
[0.2em]
\cline{4-4}
\end{tabular}
\end{center}
\end{table}
}
\clearpage
\newpage
 {
 \setlength\arrayrulewidth{1pt}
\begin{table}
\begin{center}
\begin{tabular}{ccc|l|ccc}
\cline{4-4} \\ [-1em]
\multicolumn{7}{c}{\foreignlanguage{greek}{ευαγγελιον κατα λουκαν} \textbf{(\nospace{4:2})} } \\ \\ [-1em] % Si on veut ajouter les bordures latérales, remplacer {7}{c} par {7}{|c|}
\cline{4-4} \\
\cline{4-4}
&  &  & &  &  & \\ [-0.9em]
&  & 7 & \foreignlanguage{greek}{και ουκ εφαγεν ουδεν εν ταιϲ ημεραιϲ} & 13 &  &  \\
&  & 14 & \foreignlanguage{greek}{εκειναιϲ και ϲυντελεϲθειϲων αυτων} & 17 &  &  \\
&  & 18 & \foreignlanguage{greek}{επιναϲεν} & 18 &  &  \\
& \textbf{3} &  & \foreignlanguage{greek}{ειπεν δε αυτω ο διαβολοϲ ει υιοϲ ει του} & 9 &  &  \\
&  & 10 & \foreignlanguage{greek}{\textoverline{θυ} ειπε τω λιθω τουτω ινα γενηται αρτοϲ} & 17 &  &  \\
& \textbf{4} &  & \foreignlanguage{greek}{και απεκριθη προϲ αυτον ο \textoverline{ιϲ} γεγραπται} & 7 &  &  \\
&  & 8 & \foreignlanguage{greek}{οτι ουκ επ αρτω ζηϲεται ο \textoverline{ανοϲ}} & 14 &  &  \\
& \textbf{5} &  & \foreignlanguage{greek}{και αναγαγων αυτον ειϲ οροϲ εδειξεν} & 6 &  &  \\
&  & 7 & \foreignlanguage{greek}{αυτω παϲαϲ ταϲ βαϲιλειαϲ τηϲ γηϲ εν} & 13 &  &  \\
&  & 14 & \foreignlanguage{greek}{ϲτιγμη χρονου και ειπεν αυτω ο δι} & 5 & \textbf{6} &  \\
&  & 5 & \foreignlanguage{greek}{αβολοϲ ϲοι δωϲω την εξουϲιαν παϲα̅} & 10 &  &  \\
&  & 11 & \foreignlanguage{greek}{ταυτην και την δοξαν αυτων οτι ε} & 17 &  &  \\
&  & 17 & \foreignlanguage{greek}{μοι παραδεδοται και ω εαν θελω πα} & 23 &  &  \\
&  & 23 & \foreignlanguage{greek}{ραδιδωμι αυτην ϲυ ουν εαν προϲκυ} & 4 & \textbf{7} &  \\
&  & 4 & \foreignlanguage{greek}{νηϲηϲ ενωπιον εμου εϲται ϲου παντα} & 9 &  &  \\
& \textbf{8} &  & \foreignlanguage{greek}{και αποκριθειϲ ο \textoverline{ιϲ} ειπεν αυτω γεγραπται} & 7 &  &  \\
&  & 8 & \foreignlanguage{greek}{\textoverline{κν} τον \textoverline{θν} ϲου προϲκυνηϲειϲ και αυτω} & 14 &  &  \\
&  & 15 & \foreignlanguage{greek}{μονω λατρευϲειϲ} & 16 &  &  \\
& \textbf{9} &  & \foreignlanguage{greek}{ηγαγεν δε αυτον ειϲ ιερουϲαλημ και ε} & 7 &  &  \\
&  & 7 & \foreignlanguage{greek}{ϲτηϲεν αυτον επι το πτερυγιον του ι} & 13 &  &  \\
&  & 13 & \foreignlanguage{greek}{ερου και ειπεν αυτω ει υιοϲ ει του \textoverline{θυ}} & 21 &  &  \\
&  & 22 & \foreignlanguage{greek}{βαλε ϲεαυτον εντευθεν κατω γεγρα} & 1 & \textbf{10} &  \\
&  & 1 & \foreignlanguage{greek}{πται γαρ οτι τοιϲ αγγελοιϲ αυτου εντε} & 7 &  &  \\
&  & 7 & \foreignlanguage{greek}{λειται περι ϲου περι ϲου του διαφυλα} & 13 &  &  \\
&  & 13 & \foreignlanguage{greek}{ξαι ϲε και οτι επι χειρων αρουϲιν ϲε} & 6 & \textbf{11} &  \\
&  & 7 & \foreignlanguage{greek}{μηποτε προϲκοψηϲ προϲ λιθον τον πο} & 12 &  &  \\
&  & 12 & \foreignlanguage{greek}{δα ϲου} & 13 &  &  \\
& \textbf{12} &  & \foreignlanguage{greek}{και αποκριθειϲ ειπεν αυτω ο \textoverline{ιϲ} γεγραπται} & 7 &  &  \\
&  & 8 & \foreignlanguage{greek}{ουκ εκπειραϲειϲ \textoverline{κν} τον \textoverline{θν} ϲου} & 13 &  &  \\
& \textbf{13} &  & \foreignlanguage{greek}{και ϲυντελεϲαϲ παντα πιραϲμον ο δια} & 6 &  &  \\
[0.2em]
\cline{4-4}
\end{tabular}
\end{center}
\end{table}
}
\clearpage
\newpage
 {
 \setlength\arrayrulewidth{1pt}
\begin{table}
\begin{center}
\begin{tabular}{ccc|l|ccc}
\cline{4-4} \\ [-1em]
\multicolumn{7}{c}{\foreignlanguage{greek}{ευαγγελιον κατα λουκαν} \textbf{(\nospace{4:13})} } \\ \\ [-1em] % Si on veut ajouter les bordures latérales, remplacer {7}{c} par {7}{|c|}
\cline{4-4} \\
\cline{4-4}
&  &  & &  &  & \\ [-0.9em]
&  & 6 & \foreignlanguage{greek}{βολοϲ απεϲτη απ αυτου αχρι καιρου} & 11 &  &  \\
& \textbf{14} &  & \foreignlanguage{greek}{και υπεϲτρεψεν ο \textoverline{ιϲ} εν τη δυναμει του} & 8 &  &  \\
&  & 9 & \foreignlanguage{greek}{\textoverline{πνϲ} ειϲ την γαλιλαιαν και φημη εξηλ} & 15 &  &  \\
&  & 15 & \foreignlanguage{greek}{θεν καθ οληϲ τηϲ περιχωρου περι αυτου} & 21 &  &  \\
& \textbf{15} &  & \foreignlanguage{greek}{και αυτοϲ εδιδαϲκεν εν ταιϲ ϲυναγω} & 6 &  &  \\
&  & 6 & \foreignlanguage{greek}{γαιϲ αυτων δοξαζομενοϲ υπο παντων} & 10 &  &  \\
& \textbf{16} &  & \foreignlanguage{greek}{και ηλθεν ειϲ ναζαρεθ ου ην ανατεθραμ} & 7 &  &  \\
&  & 7 & \foreignlanguage{greek}{μενοϲ και ειϲηλθεν κατα το ιωθοϲ αυτω} & 13 &  &  \\
&  & 14 & \foreignlanguage{greek}{εν τη ημερα των ϲαββατων ειϲ την ϲυν} & 21 &  &  \\
&  & 21 & \foreignlanguage{greek}{αγωγην και ανεϲτη αναγνωναι και ε} & 2 & \textbf{17} &  \\
&  & 2 & \foreignlanguage{greek}{πεδοθη αυτω βιβλιον του προφητου ηϲαιου} & 7 &  &  \\
&  & 8 & \foreignlanguage{greek}{και ανοιξαϲ το βιβλιον ευρεν τοπον ου} & 14 &  &  \\
&  & 15 & \foreignlanguage{greek}{ην γεγραμμενον \textoverline{πνα} \textoverline{κυ} επ εμε ου εινε} & 6 & \textbf{18} &  \\
&  & 6 & \foreignlanguage{greek}{κεν εχριϲεν με ευαγγελιϲαϲθαι πτωχοιϲ} & 10 &  &  \\
&  & 11 & \foreignlanguage{greek}{απεϲταλκεν με κηρυξαι αιχμαλωτοιϲ} & 14 &  &  \\
&  & 15 & \foreignlanguage{greek}{αφεϲιν και τυφλοιϲ αναβλεψιν απο} & 19 &  &  \\
&  & 19 & \foreignlanguage{greek}{ϲτιλαι τεθρωμενουϲ εν αφεϲει κη} & 1 & \textbf{19} &  \\
&  & 1 & \foreignlanguage{greek}{ρυξαι ενιαυτον \textoverline{κυ} δεκτον και πτυ} & 2 & \textbf{20} &  \\
&  & 2 & \foreignlanguage{greek}{ξαϲ το βιβλιον και αποδουϲ τω υπηρε} & 8 &  &  \\
&  & 8 & \foreignlanguage{greek}{τη εκαθειϲεν και παντων οι οφθαλ} & 13 &  &  \\
&  & 13 & \foreignlanguage{greek}{μοι εν τη ϲυναγωγη ηϲαν ατενιζον} & 18 &  &  \\
&  & 18 & \foreignlanguage{greek}{τεϲ αυτω} & 19 &  &  \\
& \textbf{21} &  & \foreignlanguage{greek}{ηρξατο δε λεγειν προϲ αυτουϲ ϲημε} & 6 &  &  \\
&  & 6 & \foreignlanguage{greek}{ρον πεπληρωται η γραφη αυτη εν τοιϲ} & 12 &  &  \\
&  & 13 & \foreignlanguage{greek}{ωϲιν υμων και παντεϲ εμαρτυρουν} & 3 & \textbf{22} &  \\
&  & 4 & \foreignlanguage{greek}{αυτω και εθαυμαζον επι τοιϲ λογοιϲ} & 9 &  &  \\
&  & 10 & \foreignlanguage{greek}{τηϲ χαριτοϲ τοιϲ εκπορευομενοιϲ εκ} & 14 &  &  \\
&  & 15 & \foreignlanguage{greek}{του ϲτοματοϲ αυτου και ελεγον ουχι υ} & 21 &  &  \\
&  & 21 & \foreignlanguage{greek}{ιοϲ εϲτιν ιωϲηφ ουτοϲ} & 24 &  &  \\
& \textbf{23} &  & \foreignlanguage{greek}{και ειπεν προϲ αυτουϲ παντωϲ ερειται μοι} & 7 &  &  \\
[0.2em]
\cline{4-4}
\end{tabular}
\end{center}
\end{table}
}
\clearpage
\newpage
 {
 \setlength\arrayrulewidth{1pt}
\begin{table}
\begin{center}
\begin{tabular}{ccc|l|ccc}
\cline{4-4} \\ [-1em]
\multicolumn{7}{c}{\foreignlanguage{greek}{ευαγγελιον κατα λουκαν} \textbf{(\nospace{4:23})} } \\ \\ [-1em] % Si on veut ajouter les bordures latérales, remplacer {7}{c} par {7}{|c|}
\cline{4-4} \\
\cline{4-4}
&  &  & &  &  & \\ [-0.9em]
&  & 8 & \foreignlanguage{greek}{την παραβολην ταυτην ιατρε θεραπευ} & 12 &  &  \\
&  & 12 & \foreignlanguage{greek}{ϲον ϲεαυτον οϲα ηκουϲαμεν γενομε} & 16 &  &  \\
&  & 16 & \foreignlanguage{greek}{να ειϲ την καφαρναουμ ποιηϲον και} & 21 &  &  \\
&  & 22 & \foreignlanguage{greek}{ωδε εν τη πατριδι ϲου} & 26 &  &  \\
& \textbf{24} &  & \foreignlanguage{greek}{ειπεν δε αμην λεγω υμιν οτι ουδειϲ} & 7 &  &  \\
&  & 8 & \foreignlanguage{greek}{προφητηϲ δεκτοϲ εϲτιν εν τη πατριδι ε} & 14 &  &  \\
&  & 14 & \foreignlanguage{greek}{αυτου επ αληθειαϲ δε λεγω υμιν} & 5 & \textbf{25} &  \\
&  & 6 & \foreignlanguage{greek}{οτι πολλαι χηραι ηϲαν εν ταιϲ ημεραιϲ} & 12 &  &  \\
&  & 13 & \foreignlanguage{greek}{ηλιου εν τω ιϲραηλ οτε εκλιϲθη ο ου} & 20 &  &  \\
&  & 20 & \foreignlanguage{greek}{ρανοϲ επι ετη τρια και μηναϲ εξ ωϲ} & 27 &  &  \\
&  & 28 & \foreignlanguage{greek}{εγενετο λιμοϲ μεγαλη επι παϲαν τη̅} & 33 &  &  \\
&  & 34 & \foreignlanguage{greek}{γην και προϲ ουδεμιαν αυτων επεμ} & 5 & \textbf{26} &  \\
&  & 5 & \foreignlanguage{greek}{φθη ηλιαϲ ει μη ειϲ ϲαραπτα τηϲ ϲει} & 12 &  &  \\
&  & 12 & \foreignlanguage{greek}{δωνιαϲ προϲ γυναικα χηραν} & 15 &  &  \\
& \textbf{27} &  & \foreignlanguage{greek}{και πολλοι λεπροι ηϲαν εν τω ιϲραηλ} & 7 &  &  \\
&  & 8 & \foreignlanguage{greek}{επι ελειϲεου του προφητου και ουδειϲ} & 13 &  &  \\
&  & 14 & \foreignlanguage{greek}{αυτων εκαθαριϲθη ει μη ναιμαν} & 18 &  &  \\
&  & 19 & \foreignlanguage{greek}{ο ϲυροϲ και επληϲθηϲαν παν} & 3 & \textbf{28} &  \\
&  & 3 & \foreignlanguage{greek}{τεϲ θυμου εν τη ϲυναγωγη ακουον} & 8 &  &  \\
&  & 8 & \foreignlanguage{greek}{τεϲ ταυτα και αναϲταντεϲ εξεβα} & 3 & \textbf{29} &  \\
&  & 3 & \foreignlanguage{greek}{λον αυτον εξω τηϲ πολεωϲ και ηγα} & 9 &  &  \\
&  & 9 & \foreignlanguage{greek}{γον αυτον εωϲ οφρυοϲ του ορουϲ εφ ου} & 16 &  &  \\
&  & 17 & \foreignlanguage{greek}{η πολειϲ ωκοδομητο αυτων ωϲτε} & 21 &  &  \\
&  & 22 & \foreignlanguage{greek}{κατακρημνιϲαι αυτον αυτοϲ δε δι} & 3 & \textbf{30} &  \\
&  & 3 & \foreignlanguage{greek}{ελθων δια μεϲου αυτων επορευετο} & 7 &  &  \\
& \textbf{31} &  & \foreignlanguage{greek}{και κατηλθεν ειϲ καφαρναουμ πολι̅} & 5 &  &  \\
&  & 6 & \foreignlanguage{greek}{τηϲ γαλιλαιαϲ και ην διδαϲκων αυ} & 11 &  &  \\
&  & 11 & \foreignlanguage{greek}{τουϲ εν τοιϲ ϲαββαϲιν και εξεπληϲ} & 2 & \textbf{32} &  \\
&  & 2 & \foreignlanguage{greek}{ϲοντο επι τη διδαχη αυτου οτι εν ε} & 9 &  &  \\
&  & 9 & \foreignlanguage{greek}{ξουϲια ην ο λογοϲ αυτου} & 13 &  &  \\
[0.2em]
\cline{4-4}
\end{tabular}
\end{center}
\end{table}
}
\clearpage
\newpage
 {
 \setlength\arrayrulewidth{1pt}
\begin{table}
\begin{center}
\begin{tabular}{ccc|l|ccc}
\cline{4-4} \\ [-1em]
\multicolumn{7}{c}{\foreignlanguage{greek}{ευαγγελιον κατα λουκαν} \textbf{(\nospace{4:33})} } \\ \\ [-1em] % Si on veut ajouter les bordures latérales, remplacer {7}{c} par {7}{|c|}
\cline{4-4} \\
\cline{4-4}
&  &  & &  &  & \\ [-0.9em]
& \textbf{33} &  & \foreignlanguage{greek}{και εν τη ϲυναγωγη ην ανθρωποϲ εχω̅} & 7 &  &  \\
&  & 8 & \foreignlanguage{greek}{\textoverline{πνα} δαιμονιου ακαθαρτου και ανε} & 12 &  &  \\
&  & 12 & \foreignlanguage{greek}{κραξεν φωνη μεγαλη εα τι ημιν και} & 4 & \textbf{34} &  \\
&  & 5 & \foreignlanguage{greek}{ϲοι \textoverline{ιυ} ναζαρηνε ηλθεϲ απολεϲαι ημαϲ} & 10 &  &  \\
&  & 11 & \foreignlanguage{greek}{οιδα ϲε τιϲ ει ο αγιοϲ του \textoverline{θυ}} & 18 &  &  \\
& \textbf{35} &  & \foreignlanguage{greek}{και επετιμηϲεν αυτω ο \textoverline{ιϲ} λεγων φι} & 7 &  &  \\
&  & 7 & \foreignlanguage{greek}{μωθητι και εξελθε απ αυτου και ρι} & 13 &  &  \\
&  & 13 & \foreignlanguage{greek}{ψαν αυτον το δαιμονιον ειϲ το μεϲο̅} & 19 &  &  \\
&  & 20 & \foreignlanguage{greek}{εξηλθεν απ αυτου} & 22 &  &  \\
& \textbf{36} &  & \foreignlanguage{greek}{και εγενετο θαμβοϲ επι πανταϲ και} & 6 &  &  \\
&  & 7 & \foreignlanguage{greek}{ϲυνελαλουν προϲ αλληλουϲ λεγοντεϲ} & 10 &  &  \\
&  & 11 & \foreignlanguage{greek}{τιϲ ο λογοϲ ουτοϲ οτι εν εξουϲια και δυ} & 19 &  &  \\
&  & 19 & \foreignlanguage{greek}{ναμει επιταϲϲει τοιϲ ακαθαρτοιϲ} & 23 &  &  \\
&  & 24 & \foreignlanguage{greek}{πνευμαϲιν και εξερχονται} & 26 &  &  \\
& \textbf{37} &  & \foreignlanguage{greek}{και εξεπορευετο ηχοϲ περι αυτου} & 5 &  &  \\
&  & 6 & \foreignlanguage{greek}{ειϲ παντα τοπον τηϲ περιχωρου} & 10 &  &  \\
& \textbf{38} &  & \foreignlanguage{greek}{αναϲταϲ δε απο τηϲ ϲυναγωγηϲ ειϲηλ} & 6 &  &  \\
&  & 6 & \foreignlanguage{greek}{θεν ειϲ την οικειαν ϲιμωνοϲ η πεν} & 12 &  &  \\
&  & 12 & \foreignlanguage{greek}{θερα δε του ϲιμωνοϲ ην ϲυνεχομενη} & 17 &  &  \\
&  & 18 & \foreignlanguage{greek}{πυρετω μεγαλω και ηρωτηϲα̅} & 21 &  &  \\
&  & 22 & \foreignlanguage{greek}{αυτον περι αυτηϲ και επιϲταϲ ε} & 3 & \textbf{39} &  \\
&  & 3 & \foreignlanguage{greek}{πανω αυτηϲ επετιμηϲεν τω πυρε} & 7 &  &  \\
&  & 7 & \foreignlanguage{greek}{τω και αφηκεν αυτην παραχρημα} & 11 &  &  \\
&  & 12 & \foreignlanguage{greek}{δε αναϲταϲα διηκονι αυτοιϲ} & 15 &  &  \\
& \textbf{40} &  & \foreignlanguage{greek}{δυνοντοϲ δε του ηλιου παντεϲ οϲοι} & 6 &  &  \\
&  & 7 & \foreignlanguage{greek}{ειχον αϲθενουνταϲ νοϲοιϲ ποικει} & 10 &  &  \\
&  & 10 & \foreignlanguage{greek}{λαιϲ ηγον αυτουϲ προϲ αυτον ο δε} & 16 &  &  \\
&  & 17 & \foreignlanguage{greek}{ενι εκαϲτω αυτων ταϲ χειραϲ επιτι} & 22 &  &  \\
&  & 22 & \foreignlanguage{greek}{θειϲ εθεραπευεν αυτουϲ} & 24 &  &  \\
& \textbf{41} &  & \foreignlanguage{greek}{εξηρχετο δε και δαιμονια πολλων} & 5 &  &  \\
[0.2em]
\cline{4-4}
\end{tabular}
\end{center}
\end{table}
}
\clearpage
\newpage
 {
 \setlength\arrayrulewidth{1pt}
\begin{table}
\begin{center}
\begin{tabular}{ccc|l|ccc}
\cline{4-4} \\ [-1em]
\multicolumn{7}{c}{\foreignlanguage{greek}{ευαγγελιον κατα λουκαν} \textbf{(\nospace{4:41})} } \\ \\ [-1em] % Si on veut ajouter les bordures latérales, remplacer {7}{c} par {7}{|c|}
\cline{4-4} \\
\cline{4-4}
&  &  & &  &  & \\ [-0.9em]
&  & 6 & \foreignlanguage{greek}{κραυγαζοντα και λεγοντα οτι ϲυ ει ο υιοϲ} & 13 &  &  \\
&  & 14 & \foreignlanguage{greek}{του \textoverline{θυ} και επιτιμων ουκ ηα λαλειν} & 20 &  &  \\
&  & 21 & \foreignlanguage{greek}{αυτα οτι ηδιϲαν τον \textoverline{χν} αυτον ειναι} & 27 &  &  \\
& \textbf{42} &  & \foreignlanguage{greek}{γενομενηϲ δε ημεραϲ εξελθων επο} & 5 &  &  \\
&  & 5 & \foreignlanguage{greek}{ρευθη ειϲ ερημον τοπον και οι οχλοι} & 11 &  &  \\
&  & 12 & \foreignlanguage{greek}{επεζητουν αυτον και ηλθον εωϲ αυ} & 17 &  &  \\
&  & 17 & \foreignlanguage{greek}{του και κατειχον αυτον του μη πο} & 23 &  &  \\
&  & 23 & \foreignlanguage{greek}{ρευεϲθαι απ αυτων} & 25 &  &  \\
& \textbf{43} &  & \foreignlanguage{greek}{ο δε ειπεν προϲ αυτουϲ οτι και ταιϲ ετε} & 9 &  &  \\
&  & 9 & \foreignlanguage{greek}{ραιϲ πολεϲιν ευαγγελιϲαϲθαι δει με} & 13 &  &  \\
&  & 14 & \foreignlanguage{greek}{την βαϲιλειαν του \textoverline{θυ} οτι επι του} & 20 &  &  \\
&  & 20 & \foreignlanguage{greek}{το απεϲταλην} & 21 &  &  \\
& \textbf{44} &  & \foreignlanguage{greek}{και ην κηρυϲϲων ειϲ ταϲ ϲυναγωγαϲ τω̅} & 7 &  &  \\
&  & 8 & \foreignlanguage{greek}{ιουδαιων εγενετο δε εν τω το̅} & 5 & \mygospelchapter &  \\
&  & 6 & \foreignlanguage{greek}{οχλον επικειϲθαι αυτω και ακουειν} & 10 &  &  \\
&  & 11 & \foreignlanguage{greek}{τον λογον του \textoverline{θυ} και αυτοϲ ην εϲτωϲ} & 18 &  &  \\
&  & 19 & \foreignlanguage{greek}{παρα την λιμνην γεννηϲαρετ} & 22 &  &  \\
& \textbf{2} &  & \foreignlanguage{greek}{και ειδεν πλοια δυο εϲτωτα παρα την} & 7 &  &  \\
&  & 8 & \foreignlanguage{greek}{λιμνην οι δε αλιειϲ απ αυτων απο} & 14 &  &  \\
&  & 14 & \foreignlanguage{greek}{βαντεϲ επλυναν τα δικτυα} & 17 &  &  \\
& \textbf{3} &  & \foreignlanguage{greek}{εμβαϲ δε ειϲ εν των πλοιων ο ην ϲιμω} & 9 &  &  \\
&  & 9 & \foreignlanguage{greek}{νοϲ ηρωτηϲεν αυτον απο τηϲ γηϲ ε} & 15 &  &  \\
&  & 15 & \foreignlanguage{greek}{παναγαγειν ολειγον καθειϲαϲ δε} & 18 &  &  \\
&  & 19 & \foreignlanguage{greek}{εδιδαϲκεν εκ του πλοιου τουϲ οχλουϲ} & 24 &  &  \\
& \textbf{4} &  & \foreignlanguage{greek}{ωϲ δε επαυϲατο λαλων ειπεν προϲ το̅} & 7 &  &  \\
&  & 8 & \foreignlanguage{greek}{ϲιμωνα επαναγαγεται ειϲ το βαθοϲ} & 12 &  &  \\
&  & 13 & \foreignlanguage{greek}{και χαλαϲαται τα δικτυα υμων ειϲ αγρα̅} & 19 &  &  \\
& \textbf{5} &  & \foreignlanguage{greek}{και αποκριθειϲ ο ϲιμων ειπεν αυτω} & 6 &  &  \\
&  & 7 & \foreignlanguage{greek}{επιϲτατα δι οληϲ νυκτοϲ κοπιαϲαν} & 11 &  &  \\
&  & 11 & \foreignlanguage{greek}{τεϲ ουδεν ελαβομεν επει δε τω ϲω} & 17 &  &  \\
[0.2em]
\cline{4-4}
\end{tabular}
\end{center}
\end{table}
}
\clearpage
\newpage
 {
 \setlength\arrayrulewidth{1pt}
\begin{table}
\begin{center}
\begin{tabular}{ccc|l|ccc}
\cline{4-4} \\ [-1em]
\multicolumn{7}{c}{\foreignlanguage{greek}{ευαγγελιον κατα λουκαν} \textbf{(\nospace{5:5})} } \\ \\ [-1em] % Si on veut ajouter les bordures latérales, remplacer {7}{c} par {7}{|c|}
\cline{4-4} \\
\cline{4-4}
&  &  & &  &  & \\ [-0.9em]
&  & 18 & \foreignlanguage{greek}{ρηματι χαλαϲω τα δικτυα} & 21 &  &  \\
& \textbf{6} &  & \foreignlanguage{greek}{και τουτο ποιηϲαντεϲ ϲυνεκλειϲαν} & 4 &  &  \\
&  & 5 & \foreignlanguage{greek}{πληθοϲ ιχθυων πολυ διερρηϲϲοντο} & 8 &  &  \\
&  & 9 & \foreignlanguage{greek}{δε τα δικτυα αυτων και κατενευϲα̅} & 2 & \textbf{7} &  \\
&  & 3 & \foreignlanguage{greek}{τοιϲ μετοχοιϲ εν τω ετερω πλοιω} & 8 &  &  \\
&  & 9 & \foreignlanguage{greek}{του ελθονταϲ ϲυνλαβεϲθαι αυτοιϲ} & 12 &  &  \\
&  & 13 & \foreignlanguage{greek}{και ηλθαν και επληϲθηϲαν αμφοτερα} & 17 &  &  \\
&  & 18 & \foreignlanguage{greek}{τα πλοια ωϲτε βυθιζεϲθαι αυτα} & 22 &  &  \\
& \textbf{8} &  & \foreignlanguage{greek}{ιδων δε ο ϲιμων προϲεπεϲεν τοιϲ γο} & 7 &  &  \\
&  & 7 & \foreignlanguage{greek}{ναϲιν \textoverline{ιυ} λεγων εξελθε απ εμου οτι} & 13 &  &  \\
&  & 14 & \foreignlanguage{greek}{ανηρ αμαρτωλοϲ ειμει \textoverline{κε} θαμβοϲ γαρ} & 2 & \textbf{9} &  \\
&  & 3 & \foreignlanguage{greek}{περιεϲχεν αυτον και πανταϲ τουϲ ϲυ̅} & 8 &  &  \\
&  & 9 & \foreignlanguage{greek}{αυτω επι τη αγρα των ιχθυων η ϲυνε} & 16 &  &  \\
&  & 16 & \foreignlanguage{greek}{λαβον ομοιωϲ δε και ιακωβον και} & 5 & \textbf{10} &  \\
&  & 6 & \foreignlanguage{greek}{ιωαννην υιουϲ ζεβεδεου οι ηϲαν κοι} & 11 &  &  \\
&  & 11 & \foreignlanguage{greek}{νωνοι τω ϲιμωνι} & 13 &  &  \\
&  & 14 & \foreignlanguage{greek}{και ειπεν προϲ τον ϲιμωνα ο \textoverline{ιϲ} μη φοβου} & 22 &  &  \\
&  & 23 & \foreignlanguage{greek}{απο του νυν \textoverline{ανουϲ} εϲη ζωγρων} & 28 &  &  \\
& \textbf{11} &  & \foreignlanguage{greek}{και καταγαγοντεϲ τα πλοια και επι τη̅} & 7 &  &  \\
&  & 8 & \foreignlanguage{greek}{γην αφεντεϲ απαντα ηκολουθηϲα̅} & 11 &  &  \\
&  & 12 & \foreignlanguage{greek}{αυτω} & 12 &  &  \\
& \textbf{12} &  & \foreignlanguage{greek}{και εγενετο εν τω ειναι αυτον εν μια} & 8 &  &  \\
&  & 9 & \foreignlanguage{greek}{των πολεων και ιδου ανηρ πληρηϲ} & 14 &  &  \\
&  & 15 & \foreignlanguage{greek}{λεπραϲ και ιδων τον \textoverline{ιν} πεϲων επι προ} & 22 &  &  \\
&  & 22 & \foreignlanguage{greek}{ϲωπον εδεηθη αυτου λεγων \textoverline{κε} εαν} & 27 &  &  \\
&  & 28 & \foreignlanguage{greek}{θεληϲ δυναϲαι με καθαριϲαι} & 31 &  &  \\
& \textbf{13} &  & \foreignlanguage{greek}{και εκτειναϲ την χειρα ηψατο αυτου λε} & 7 &  &  \\
&  & 7 & \foreignlanguage{greek}{γων θελω καθαριϲθητει και ευθεωϲ} & 11 &  &  \\
&  & 12 & \foreignlanguage{greek}{η λεπρα απηλθεν απ αυτου και παρηγ} & 2 & \textbf{14} &  \\
&  & 2 & \foreignlanguage{greek}{γειλεν αυτω μηδενι ειπειν αλλα α} & 7 &  &  \\
[0.2em]
\cline{4-4}
\end{tabular}
\end{center}
\end{table}
}
\clearpage
\newpage
 {
 \setlength\arrayrulewidth{1pt}
\begin{table}
\begin{center}
\begin{tabular}{ccc|l|ccc}
\cline{4-4} \\ [-1em]
\multicolumn{7}{c}{\foreignlanguage{greek}{ευαγγελιον κατα λουκαν} \textbf{(\nospace{5:14})} } \\ \\ [-1em] % Si on veut ajouter les bordures latérales, remplacer {7}{c} par {7}{|c|}
\cline{4-4} \\
\cline{4-4}
&  &  & &  &  & \\ [-0.9em]
&  & 7 & \foreignlanguage{greek}{πελθων δειξον ϲεαυτον τω ιερει} & 11 &  &  \\
&  & 12 & \foreignlanguage{greek}{και προϲενεγκε περι του καθαριϲμου} & 16 &  &  \\
&  & 17 & \foreignlanguage{greek}{ϲου καθωϲ προϲεταξεν μωυϲηϲ ειϲ} & 21 &  &  \\
&  & 22 & \foreignlanguage{greek}{μαρτυριον αυτοιϲ} & 23 &  &  \\
& \textbf{15} &  & \foreignlanguage{greek}{διηρχετο δε μαλλον ο λογοϲ περι αυτου} & 7 &  &  \\
&  & 8 & \foreignlanguage{greek}{και ϲυνηρχοντο οχλοι πολλοι ακουειν} & 12 &  &  \\
&  & 13 & \foreignlanguage{greek}{και θεραπευεϲθαι απο των αϲθενι} & 17 &  &  \\
&  & 17 & \foreignlanguage{greek}{ων αυτων αυτοϲ δε ην υποχωρω̅} & 4 & \textbf{16} &  \\
&  & 5 & \foreignlanguage{greek}{εν ταιϲ ερημοιϲ ϗ προϲευχομενοϲ} & 9 &  &  \\
& \textbf{17} &  & \foreignlanguage{greek}{και εγενετο εν μια των ημερων και} & 7 &  &  \\
&  & 8 & \foreignlanguage{greek}{αυτοϲ ην διδαϲκων και ηϲαν καθη} & 13 &  &  \\
&  & 13 & \foreignlanguage{greek}{μενοι φαριϲαιοι και νομοδιδαϲκαλοι} & 16 &  &  \\
&  & 17 & \foreignlanguage{greek}{οι ηϲαν εληλυθοτεϲ εκ παϲηϲ χωραϲ} & 22 &  &  \\
&  & 23 & \foreignlanguage{greek}{τηϲ γαλιλαιαϲ και ιουδαιαϲ και ιε} & 28 &  &  \\
&  & 28 & \foreignlanguage{greek}{ρουϲαλημ και δυναμειϲ \textoverline{κυ} ην ειϲ} & 33 &  &  \\
&  & 34 & \foreignlanguage{greek}{το ειαϲθαι αυτον} & 36 &  &  \\
& \textbf{18} &  & \foreignlanguage{greek}{και ιδου ανδρεϲ φεροντεϲ επι κλινηϲ} & 6 &  &  \\
&  & 7 & \foreignlanguage{greek}{ανθρωπον οϲ ην παραλελυμενοϲ} & 10 &  &  \\
&  & 11 & \foreignlanguage{greek}{και εζητουν αυτον ειϲενεγκειν και} & 15 &  &  \\
&  & 16 & \foreignlanguage{greek}{θειναι ενωπιον αυτου και μη ευρο̅} & 3 & \textbf{19} &  \\
&  & 3 & \foreignlanguage{greek}{τεϲ ποιαϲ ειϲενεγκωϲιν αυτον δια} & 7 &  &  \\
&  & 8 & \foreignlanguage{greek}{τον οχλον αναβαντεϲ επι το δωμα} & 13 &  &  \\
&  & 14 & \foreignlanguage{greek}{δια των κεραμων καθηκαν αυτον} & 18 &  &  \\
&  & 19 & \foreignlanguage{greek}{ϲυν τω κλεινιδιω ειϲ το μεϲον εμ} & 25 &  &  \\
&  & 25 & \foreignlanguage{greek}{προϲθεν του \textoverline{ιυ}} & 27 &  &  \\
& \textbf{20} &  & \foreignlanguage{greek}{και ιδων την πιϲτιν αυτων ειπεν αυ} & 7 &  &  \\
&  & 7 & \foreignlanguage{greek}{τω \textoverline{ανε} αφεωνται ϲου αι αμαρτιαι} & 12 &  &  \\
& \textbf{21} &  & \foreignlanguage{greek}{και ηρξαντο διαλογιζεϲθαι οι γραμ} & 5 &  &  \\
&  & 5 & \foreignlanguage{greek}{ματιϲ και οι φαριϲαιοι λεγοντεϲ} & 9 &  &  \\
&  & 10 & \foreignlanguage{greek}{τιϲ εϲτιν ουτοϲ οϲ λαλει βλαϲφημειαϲ} & 15 &  &  \\
[0.2em]
\cline{4-4}
\end{tabular}
\end{center}
\end{table}
}
\clearpage
\newpage
 {
 \setlength\arrayrulewidth{1pt}
\begin{table}
\begin{center}
\begin{tabular}{ccc|l|ccc}
\cline{4-4} \\ [-1em]
\multicolumn{7}{c}{\foreignlanguage{greek}{ευαγγελιον κατα λουκαν} \textbf{(\nospace{5:21})} } \\ \\ [-1em] % Si on veut ajouter les bordures latérales, remplacer {7}{c} par {7}{|c|}
\cline{4-4} \\
\cline{4-4}
&  &  & &  &  & \\ [-0.9em]
&  & 16 & \foreignlanguage{greek}{τιϲ δυναται αφειεναι αμαρτιαϲ ει μη} & 21 &  &  \\
&  & 22 & \foreignlanguage{greek}{μονοϲ ο \textoverline{θϲ}} & 24 &  &  \\
& \textbf{22} &  & \foreignlanguage{greek}{επιγνουϲ δε ο \textoverline{ιϲ} τουϲ διαλογιϲμουϲ αυτω̅} & 7 &  &  \\
&  & 8 & \foreignlanguage{greek}{αποκριθειϲ ειπεν προϲ αυτουϲ τι δια} & 13 &  &  \\
&  & 13 & \foreignlanguage{greek}{λογιζεϲθαι εν ταιϲ καρδιαιϲ υμων τι ε} & 2 & \textbf{23} &  \\
&  & 2 & \foreignlanguage{greek}{ϲτιν ευκοπωτερον ειπειν αφεωνται} & 5 &  &  \\
&  & 6 & \foreignlanguage{greek}{ϲου αι αμαρτιαι η ειπειν εγειρε και} & 12 &  &  \\
&  & 13 & \foreignlanguage{greek}{περιπατι ινα δε ειδηται οτι ο υιοϲ του} & 7 & \textbf{24} &  \\
&  & 8 & \foreignlanguage{greek}{ανθρωπου εξουϲιαν εχει επι τηϲ γηϲ} & 13 &  &  \\
&  & 14 & \foreignlanguage{greek}{αφιεναι αμαρτιαϲ ειπεν τω παρα} & 18 &  &  \\
&  & 18 & \foreignlanguage{greek}{λυτικω ϲοι λεγω εγειρε και αραϲ το} & 24 &  &  \\
&  & 25 & \foreignlanguage{greek}{κλινιδιον ϲου πορευου ειϲ τον οικον ϲου} & 31 &  &  \\
& \textbf{25} &  & \foreignlanguage{greek}{και παραχρημα αναϲταϲ ενωπιον αυ} & 5 &  &  \\
&  & 5 & \foreignlanguage{greek}{των αραϲ εφ ο κατεκειτο απηλθεν} & 11 &  &  \\
&  & 12 & \foreignlanguage{greek}{ειϲ τον οικον αυτου δοξαζων τον \textoverline{θν}} & 18 &  &  \\
& \textbf{26} &  & \foreignlanguage{greek}{και επληϲθηϲαν φοβου λεγοντεϲ οτι} & 5 &  &  \\
&  & 6 & \foreignlanguage{greek}{ειδομεν παραδοξα ϲημερον} & 8 &  &  \\
& \textbf{27} &  & \foreignlanguage{greek}{και μετα ταυτα εξηλθεν και εθεαϲα} & 6 &  &  \\
&  & 6 & \foreignlanguage{greek}{το τελωνην ονοματι λευειν καθημε} & 10 &  &  \\
&  & 10 & \foreignlanguage{greek}{νον επι το τελωνιον και ειπεν αυτω} & 16 &  &  \\
&  & 17 & \foreignlanguage{greek}{ακολουθει μοι και καταλιπων παν} & 3 & \textbf{28} &  \\
&  & 3 & \foreignlanguage{greek}{τα αναϲταϲ ηκολουθει αυτω} & 6 &  &  \\
& \textbf{29} &  & \foreignlanguage{greek}{και εποιηϲεν δοχην μεγαλην λευειϲ} & 5 &  &  \\
&  & 6 & \foreignlanguage{greek}{αυτω εν τη οικεια αυτου και ην οχλοϲ} & 13 &  &  \\
&  & 14 & \foreignlanguage{greek}{πολυϲ τελωνων και αμαρτωλων οι η} & 19 &  &  \\
&  & 19 & \foreignlanguage{greek}{ϲαν μετ αυτων κατακειμενοι} & 22 &  &  \\
& \textbf{30} &  & \foreignlanguage{greek}{και εγογγυζον οι φαριϲαιοι και οι γραμ} & 7 &  &  \\
&  & 7 & \foreignlanguage{greek}{ματειϲ αυτων προϲ τουϲ μαθηταϲ αυ} & 12 &  &  \\
&  & 12 & \foreignlanguage{greek}{του λεγοντεϲ δια τι μετα των τελω} & 18 &  &  \\
&  & 18 & \foreignlanguage{greek}{νων και αμαρτωλων εϲθιεται κα πει} & 23 &  &  \\
&  & 23 & \foreignlanguage{greek}{νεται} & 23 &  &  \\
[0.2em]
\cline{4-4}
\end{tabular}
\end{center}
\end{table}
}
\clearpage
\newpage
 {
 \setlength\arrayrulewidth{1pt}
\begin{table}
\begin{center}
\begin{tabular}{ccc|l|ccc}
\cline{4-4} \\ [-1em]
\multicolumn{7}{c}{\foreignlanguage{greek}{ευαγγελιον κατα λουκαν} \textbf{(\nospace{5:31})} } \\ \\ [-1em] % Si on veut ajouter les bordures latérales, remplacer {7}{c} par {7}{|c|}
\cline{4-4} \\
\cline{4-4}
&  &  & &  &  & \\ [-0.9em]
& \textbf{31} &  & \foreignlanguage{greek}{και αποκριθειϲ ειπεν προϲ αυτουϲ ου χρι} & 7 &  &  \\
&  & 7 & \foreignlanguage{greek}{αν εχουϲιν οι υγειαινοντεϲ ιατρου} & 11 &  &  \\
&  & 12 & \foreignlanguage{greek}{αλλα οι κακωϲ εχοντεϲ ουκ εληλυθα} & 2 & \textbf{32} &  \\
&  & 3 & \foreignlanguage{greek}{καλεϲαι δικαιουϲ αλλα αμαρτωλουϲ} & 6 &  &  \\
&  & 7 & \foreignlanguage{greek}{ειϲ μετανοιαν} & 8 &  &  \\
& \textbf{33} &  & \foreignlanguage{greek}{οι δε ειπαν προϲ αυτον οι μαθηται ιω} & 8 &  &  \\
&  & 8 & \foreignlanguage{greek}{αννου νηϲτευουϲιν πυκνα και δεη} & 12 &  &  \\
&  & 12 & \foreignlanguage{greek}{ϲειϲ ποιουνται ομοιωϲ και οι των φα} & 18 &  &  \\
&  & 18 & \foreignlanguage{greek}{ριϲαιων οι δε ϲοι εϲθιουϲιν και πινουϲι̅} & 24 &  &  \\
& \textbf{34} &  & \foreignlanguage{greek}{ο δε \textoverline{ιϲ} ειπεν προϲ αυτουϲ μη δυναϲθαι} & 8 &  &  \\
&  & 9 & \foreignlanguage{greek}{τουϲ υιουϲ του νυμφωνοϲ εν ω ο νυμ} & 16 &  &  \\
&  & 16 & \foreignlanguage{greek}{φιοϲ μετ αυτων εϲτιν ποιηϲαι νη} & 21 &  &  \\
&  & 21 & \foreignlanguage{greek}{ϲτευειν ελευϲονται δε ημεραι και} & 4 & \textbf{35} &  \\
&  & 5 & \foreignlanguage{greek}{οταν απαρθη απ αυτων ο νυμφιοϲ το} & 11 &  &  \\
&  & 11 & \foreignlanguage{greek}{τε νηϲτευϲουϲιν εν εκειναιϲ ταιϲ} & 15 &  &  \\
&  & 16 & \foreignlanguage{greek}{ημεραιϲ} & 16 &  &  \\
& \textbf{36} &  & \foreignlanguage{greek}{ελεγεν δε και παραβολην προϲ αυτουϲ} & 6 &  &  \\
&  & 7 & \foreignlanguage{greek}{οτι ουδειϲ επιβλημα απο ιματιου και} & 12 &  &  \\
&  & 12 & \foreignlanguage{greek}{νου ϲχιϲαϲ επιβαλλει επι ιματιον πα} & 17 &  &  \\
&  & 17 & \foreignlanguage{greek}{λαιον ει δε μη γε και το καινον ϲχειϲει} & 25 &  &  \\
&  & 26 & \foreignlanguage{greek}{και τω παλαιω ου ϲυμφωνηϲει το ε} & 32 &  &  \\
&  & 32 & \foreignlanguage{greek}{πιβλημα το απο του καινου} & 36 &  &  \\
& \textbf{37} &  & \foreignlanguage{greek}{και ουδειϲ βαλλει οινον νεον ειϲ αϲκουϲ} & 7 &  &  \\
&  & 8 & \foreignlanguage{greek}{παλαιουϲ ει δε μη ρηξει ο οινοϲ ο νεοϲ} & 16 &  &  \\
&  & 17 & \foreignlanguage{greek}{τουϲ αϲκουϲ και αυτοϲ εκχυθηϲεται} & 21 &  &  \\
&  & 22 & \foreignlanguage{greek}{και οι αϲκοι απολουνται αλλα οινο̅} & 2 & \textbf{38} &  \\
&  & 3 & \foreignlanguage{greek}{νεον ειϲ αϲκουϲ καινουϲ βαλληται} & 7 &  &  \\
& \textbf{39} &  & \foreignlanguage{greek}{και ουδειϲ πιων παλαιον θελει ναιον} & 6 &  &  \\
&  & 7 & \foreignlanguage{greek}{λεγει γαρ ο παλαιοϲ χρηϲτοϲ εϲτιν} & 12 &  &  \\
& \mygospelchapter &  & \foreignlanguage{greek}{εγενετο δε εν ϲαββατω διαπορευεϲθαι} & 5 &  &  \\
[0.2em]
\cline{4-4}
\end{tabular}
\end{center}
\end{table}
}
\clearpage
\newpage
 {
 \setlength\arrayrulewidth{1pt}
\begin{table}
\begin{center}
\begin{tabular}{ccc|l|ccc}
\cline{4-4} \\ [-1em]
\multicolumn{7}{c}{\foreignlanguage{greek}{ευαγγελιον κατα λουκαν} \textbf{(\nospace{6:1})} } \\ \\ [-1em] % Si on veut ajouter les bordures latérales, remplacer {7}{c} par {7}{|c|}
\cline{4-4} \\
\cline{4-4}
&  &  & &  &  & \\ [-0.9em]
&  & 6 & \foreignlanguage{greek}{αυτον δια ϲποριμων και ετιλλον οι} & 12 &  &  \\
&  & 13 & \foreignlanguage{greek}{μαθηται αυτου τουϲ ϲταχυαϲ και ηϲθειο̅} & 18 &  &  \\
&  & 19 & \foreignlanguage{greek}{ψωχοντεϲ ταιϲ χερϲιν} & 21 &  &  \\
& \textbf{2} &  & \foreignlanguage{greek}{τινεϲ δε των φαριϲαιων ειπον τι ποιειτε} & 7 &  &  \\
&  & 8 & \foreignlanguage{greek}{ο ουκ εξεϲτιν ποιειν τοιϲ ϲαββαϲιν} & 13 &  &  \\
& \textbf{3} &  & \foreignlanguage{greek}{και αποκριθειϲ ο \textoverline{ιϲ} προϲ αυτουϲ ειπεν} & 7 &  &  \\
&  & 8 & \foreignlanguage{greek}{ουδε τουτο ανεγνωται ο εποιηϲεν δαυειδ} & 13 &  &  \\
&  & 14 & \foreignlanguage{greek}{οτε επιναϲεν αυτοϲ και οι μετ αυτου} & 20 &  &  \\
& \textbf{4} &  & \foreignlanguage{greek}{ωϲ ειϲηλθεν ειϲ τον οικον του \textoverline{θυ} και} & 8 &  &  \\
&  & 9 & \foreignlanguage{greek}{τουϲ αρτουϲ τηϲ προθεϲεωϲ εφαγεν} & 13 &  &  \\
&  & 14 & \foreignlanguage{greek}{και εδωκεν τοιϲ μετ αυτου ουϲ ουκ ε} & 21 &  &  \\
&  & 21 & \foreignlanguage{greek}{ξεϲτιν φαγειν ει μη μονουϲ τουϲ ιερειϲ} & 27 &  &  \\
& \textbf{5} &  & \foreignlanguage{greek}{και ελεγεν αυτοιϲ \textoverline{κϲ} εϲτιν του ϲαββα} & 7 &  &  \\
&  & 7 & \foreignlanguage{greek}{του ο υιοϲ του ανθρωπου} & 11 &  &  \\
& \textbf{6} &  & \foreignlanguage{greek}{εγενετο δε εν ετερω ϲαββατω ειϲελ} & 6 &  &  \\
&  & 6 & \foreignlanguage{greek}{θειν αυτον ειϲ την ϲυναγωγην και δι} & 12 &  &  \\
&  & 12 & \foreignlanguage{greek}{δαϲκειν και ην ανθρωποϲ εκει και} & 17 &  &  \\
&  & 18 & \foreignlanguage{greek}{η χειρ αυτου η δεξια ην ξηρα} & 24 &  &  \\
& \textbf{7} &  & \foreignlanguage{greek}{παρετηρουν δε αυτον οι γραμματιϲ} & 5 &  &  \\
&  & 6 & \foreignlanguage{greek}{και οι φαριϲαιοι ει εν τω ϲαββατω θε} & 13 &  &  \\
&  & 13 & \foreignlanguage{greek}{ραπευει ινα ευρωϲιν κατηγοριαν} & 16 &  &  \\
&  & 17 & \foreignlanguage{greek}{κατ αυτου αυτοϲ δε ηδει τουϲ δι} & 5 & \textbf{8} &  \\
&  & 5 & \foreignlanguage{greek}{αλογιϲμουϲ αυτων} & 6 &  &  \\
&  & 7 & \foreignlanguage{greek}{ειπεν δε τω ανθρωπω τω ξηραν εχον} & 13 &  &  \\
&  & 13 & \foreignlanguage{greek}{τι την χειρα εγειρε και ϲτηθει ειϲ το} & 21 &  &  \\
&  & 22 & \foreignlanguage{greek}{μεϲον και αναϲταϲ εϲτη} & 25 &  &  \\
& \textbf{9} &  & \foreignlanguage{greek}{ειπεν δε προϲ αυτουϲ ο \textoverline{ιϲ} επερωτω υ} & 8 &  &  \\
&  & 8 & \foreignlanguage{greek}{μαϲ ει εξεϲτιν τω ϲαββατω αγαθοποι} & 13 &  &  \\
&  & 13 & \foreignlanguage{greek}{ηϲαι η κακοποιηϲαι ψυχην ϲωϲαι} & 17 &  &  \\
&  & 18 & \foreignlanguage{greek}{η απολεϲαι} & 19 &  &  \\
[0.2em]
\cline{4-4}
\end{tabular}
\end{center}
\end{table}
}
\clearpage
\newpage
 {
 \setlength\arrayrulewidth{1pt}
\begin{table}
\begin{center}
\begin{tabular}{ccc|l|ccc}
\cline{4-4} \\ [-1em]
\multicolumn{7}{c}{\foreignlanguage{greek}{ευαγγελιον κατα λουκαν} \textbf{(\nospace{6:10})} } \\ \\ [-1em] % Si on veut ajouter les bordures latérales, remplacer {7}{c} par {7}{|c|}
\cline{4-4} \\
\cline{4-4}
&  &  & &  &  & \\ [-0.9em]
& \textbf{10} &  & \foreignlanguage{greek}{και περιβλεψαμενοϲ πανταϲ ειπεν τω} & 5 &  &  \\
&  & 6 & \foreignlanguage{greek}{ανθρωπω εκτινον την χειρα ϲου} & 10 &  &  \\
&  & 11 & \foreignlanguage{greek}{και εξετινεν και απεκατεϲταθη η} & 15 &  &  \\
&  & 16 & \foreignlanguage{greek}{χειρ αυτου υγιηϲ αυτοι δε επληϲθη} & 3 & \textbf{11} &  \\
&  & 3 & \foreignlanguage{greek}{ϲαν ανοιαϲ και διελαλουν προϲ αλληλουϲ} & 8 &  &  \\
&  & 9 & \foreignlanguage{greek}{τι αν ποιηϲειεν τω \textoverline{ιυ}} & 13 &  &  \\
& \textbf{12} &  & \foreignlanguage{greek}{εγενετο δε εν ταιϲ ημεραιϲ ταυταιϲ εξηλ} & 7 &  &  \\
&  & 7 & \foreignlanguage{greek}{θειν αυτον ειϲ το οροϲ προϲευξαϲθαι} & 12 &  &  \\
&  & 13 & \foreignlanguage{greek}{και ην διανυκτερευων εν τη προϲευ} & 18 &  &  \\
&  & 18 & \foreignlanguage{greek}{χη του \textoverline{θυ} και οτε εγενετο ημερα προϲ} & 5 & \textbf{13} &  \\
&  & 5 & \foreignlanguage{greek}{εφωνηϲεν τουϲ μαθηταϲ αυτου} & 8 &  &  \\
&  & 9 & \foreignlanguage{greek}{και εκλεξαμενοϲ απ αυτων δωδεκα} & 13 &  &  \\
&  & 14 & \foreignlanguage{greek}{ουϲ και αποϲτολουϲ ωνομαϲεν} & 17 &  &  \\
& \textbf{14} &  & \foreignlanguage{greek}{ϲιμων ον και ωνομαϲεν πετρον} & 5 &  &  \\
&  & 6 & \foreignlanguage{greek}{και ανδρεαν τον αδελφον αυτου και} & 11 &  &  \\
&  & 12 & \foreignlanguage{greek}{ιακωβον και ιωαννην και φιλιππο̅} & 16 &  &  \\
&  & 17 & \foreignlanguage{greek}{και ματθολομεον και μαθθεον και} & 3 & \textbf{15} &  \\
&  & 4 & \foreignlanguage{greek}{θωμαν ιακωβον αλφαιου και ϲιμω} & 8 &  &  \\
&  & 8 & \foreignlanguage{greek}{να τον καλουμενον ζηλωτην και ι} & 2 & \textbf{16} &  \\
&  & 2 & \foreignlanguage{greek}{ουδαν ιακωβου και ιουδαν ιϲκαριωτη̅} & 6 &  &  \\
&  & 7 & \foreignlanguage{greek}{οϲ εγενετο προδοτηϲ} & 9 &  &  \\
& \textbf{17} &  & \foreignlanguage{greek}{και καταβαϲ μετ αυτων εϲτη επι το} & 7 &  &  \\
&  & 7 & \foreignlanguage{greek}{που πεδινου και οχλοϲ πολυϲ μαθη} & 12 &  &  \\
&  & 12 & \foreignlanguage{greek}{των αυτου και πληθοϲ πολυ του λαου} & 18 &  &  \\
&  & 19 & \foreignlanguage{greek}{απο παϲηϲ τηϲ ιουδαιαϲ και ιερουϲαλημ} & 24 &  &  \\
&  & 25 & \foreignlanguage{greek}{και τηϲ περεαϲ και τηϲ παραλιου τυ} & 31 &  &  \\
&  & 31 & \foreignlanguage{greek}{ρου και ϲιδωνοϲ οι ηλθον ακουϲαι} & 3 & \textbf{18} &  \\
&  & 4 & \foreignlanguage{greek}{αυτου και ιαθηναι απο των νοϲων} & 9 &  &  \\
&  & 10 & \foreignlanguage{greek}{αυτων και οι οχλουμενοι απο} & 14 &  &  \\
&  & 15 & \foreignlanguage{greek}{πνευματων ακαθαρτων εθεραπευ} & 17 &  &  \\
[0.2em]
\cline{4-4}
\end{tabular}
\end{center}
\end{table}
}
\clearpage
\newpage
 {
 \setlength\arrayrulewidth{1pt}
\begin{table}
\begin{center}
\begin{tabular}{ccc|l|ccc}
\cline{4-4} \\ [-1em]
\multicolumn{7}{c}{\foreignlanguage{greek}{ευαγγελιον κατα λουκαν} \textbf{(\nospace{6:18})} } \\ \\ [-1em] % Si on veut ajouter les bordures latérales, remplacer {7}{c} par {7}{|c|}
\cline{4-4} \\
\cline{4-4}
&  &  & &  &  & \\ [-0.9em]
&  & 17 & \foreignlanguage{greek}{οντο και παϲ ο οχλοϲ εζητουν απτεϲθε} & 6 & \textbf{19} &  \\
&  & 7 & \foreignlanguage{greek}{αυτου οτι δυναμιϲ παρ αυτου εξηρχετο} & 12 &  &  \\
&  & 13 & \foreignlanguage{greek}{και ιατο πανταϲ} & 15 &  &  \\
& \textbf{20} &  & \foreignlanguage{greek}{και αυτοϲ επαραϲ τουϲ οφθαλμουϲ αυ} & 6 &  &  \\
&  & 6 & \foreignlanguage{greek}{του ειϲ τουϲ μαθηταϲ αυτου ελεγεν} & 11 &  &  \\
&  & 12 & \foreignlanguage{greek}{μακαριοι οι πτωχοι οτι αυτων εϲτιν η} & 18 &  &  \\
&  & 19 & \foreignlanguage{greek}{βαϲιλεια του \textoverline{θυ} μακαριοι οι πι} & 3 & \textbf{21} &  \\
&  & 3 & \foreignlanguage{greek}{νωντεϲ νυν οτι χορταϲθηϲεϲθαι} & 6 &  &  \\
&  & 7 & \foreignlanguage{greek}{μακαριοι οι κλεοντεϲ νυν οτι γελαϲουϲι̅} & 12 &  &  \\
& \textbf{22} &  & \foreignlanguage{greek}{μακαριοι εϲται οταν μιϲηϲωϲιν υμαϲ οι} & 6 &  &  \\
&  & 7 & \foreignlanguage{greek}{ανθρωποι και αφοριϲωϲιν υμαϲ και} & 11 &  &  \\
&  & 12 & \foreignlanguage{greek}{ονιδιϲωϲιν και εκβαλωϲιν το ονομα} & 16 &  &  \\
&  & 17 & \foreignlanguage{greek}{υμων ωϲ πονηρον ενεκεν του υιου} & 22 &  &  \\
&  & 23 & \foreignlanguage{greek}{του ανθρωπου χαρητε εν εκεινη τη} & 4 & \textbf{23} &  \\
&  & 5 & \foreignlanguage{greek}{ημερα και ϲκιρτηϲατε ιδου γαρ ο μι} & 11 &  &  \\
&  & 11 & \foreignlanguage{greek}{ϲθοϲ υμων πολυϲ εν τω ουρανω} & 16 &  &  \\
&  & 17 & \foreignlanguage{greek}{κατα τα αυτα γαρ εποιουν τοιϲ προφη} & 23 &  &  \\
&  & 23 & \foreignlanguage{greek}{ταιϲ οι πατερεϲ αυτων} & 26 &  &  \\
& \textbf{24} &  & \foreignlanguage{greek}{πλην ουαι υμιν τοιϲ πλουϲιοιϲ οτι απε} & 7 &  &  \\
&  & 7 & \foreignlanguage{greek}{χεται την παρακληϲιν υμων} & 10 &  &  \\
& \textbf{25} &  & \foreignlanguage{greek}{ουαι υμιν οι ενπεπληϲμενοι νυν οτι} & 6 &  &  \\
&  & 7 & \foreignlanguage{greek}{πιναϲεται ουαι οι γελωντεϲ νυν} & 11 &  &  \\
&  & 12 & \foreignlanguage{greek}{οτι πενθηϲεται και κλαυϲεται} & 15 &  &  \\
& \textbf{26} &  & \foreignlanguage{greek}{ουαι οταν καλωϲ υμαϲ ειπωϲιν} & 6 &  &  \\
&  & 7 & \foreignlanguage{greek}{παντεϲ οι ανθρωποι κατα τα αυτα γαρ} & 13 &  &  \\
&  & 14 & \foreignlanguage{greek}{εποιουν τοιϲ ψευδοπροφηταιϲ οι πα} & 18 &  &  \\
&  & 18 & \foreignlanguage{greek}{τερεϲ αυτων} & 19 &  &  \\
& \textbf{27} &  & \foreignlanguage{greek}{αλλα υμιν λεγω τοιϲ ακουουϲιν μου} & 6 &  &  \\
&  & 7 & \foreignlanguage{greek}{αγαπατε τουϲ εχθρουϲ υμων και καλωϲ} & 12 &  &  \\
&  & 13 & \foreignlanguage{greek}{ποιειται τοιϲ μιϲουϲιν υμαϲ ευλογει} & 1 & \textbf{28} &  \\
[0.2em]
\cline{4-4}
\end{tabular}
\end{center}
\end{table}
}
\clearpage
\newpage
 {
 \setlength\arrayrulewidth{1pt}
\begin{table}
\begin{center}
\begin{tabular}{ccc|l|ccc}
\cline{4-4} \\ [-1em]
\multicolumn{7}{c}{\foreignlanguage{greek}{ευαγγελιον κατα λουκαν} \textbf{(\nospace{6:28})} } \\ \\ [-1em] % Si on veut ajouter les bordures latérales, remplacer {7}{c} par {7}{|c|}
\cline{4-4} \\
\cline{4-4}
&  &  & &  &  & \\ [-0.9em]
&  & 1 & \foreignlanguage{greek}{ται τουϲ καταρωμενουϲ υμαϲ και} & 5 &  &  \\
&  & 6 & \foreignlanguage{greek}{προϲευχεϲθαι περι των επηρεαζον} & 9 &  &  \\
&  & 9 & \foreignlanguage{greek}{των υμαϲ τω τυπτοντι ϲε ειϲ} & 4 & \textbf{29} &  \\
&  & 5 & \foreignlanguage{greek}{την ϲιαγονα παρεχε και την αλλην} & 10 &  &  \\
&  & 11 & \foreignlanguage{greek}{και απο του εροντοϲ ϲου το ιματιον} & 17 &  &  \\
&  & 18 & \foreignlanguage{greek}{και τον χειτωνα μη κωλυϲηϲ} & 22 &  &  \\
& \textbf{30} &  & \foreignlanguage{greek}{παντι αιτουντι ϲε διδου και απο του} & 7 &  &  \\
&  & 8 & \foreignlanguage{greek}{εροντοϲ τα ϲα μη απετει} & 12 &  &  \\
& \textbf{31} &  & \foreignlanguage{greek}{και καθωϲ θελεται ινα ποιωϲιν υμιν} & 6 &  &  \\
&  & 7 & \foreignlanguage{greek}{οι ανθρωποι και υμειϲ ποιειται αυ} & 12 &  &  \\
&  & 12 & \foreignlanguage{greek}{τοιϲ ομοιωϲ} & 13 &  &  \\
& \textbf{32} &  & \foreignlanguage{greek}{και ει αγαπατε τουϲ αγαπωνταϲ υμαϲ} & 6 &  &  \\
&  & 7 & \foreignlanguage{greek}{ποια υμιν χαριϲ εϲτιν και γαρ οι αμαρ} & 14 &  &  \\
&  & 14 & \foreignlanguage{greek}{τωλοι τουϲ αγαπωνταϲ αυτουϲ αγαπωϲι̅} & 18 &  &  \\
& \textbf{33} &  & \foreignlanguage{greek}{και εαν αγαθοποιητε τουϲ αγαθοποι} & 5 &  &  \\
&  & 5 & \foreignlanguage{greek}{ουνταϲ υμαϲ ποια υμιν χαριϲ εϲτιν} & 10 &  &  \\
&  & 11 & \foreignlanguage{greek}{και οι αμαρτωλοι το αυτο ποιουϲιν} & 16 &  &  \\
& \textbf{34} &  & \foreignlanguage{greek}{και εαν δανιϲηται παρ ων ελπιζεται} & 6 &  &  \\
&  & 7 & \foreignlanguage{greek}{λαβειν ποια χαριϲ εϲτιν υμιν και} & 12 &  &  \\
&  & 13 & \foreignlanguage{greek}{αμαρτωλοι αμαρτωλοιϲ δανιζουϲι̅} & 15 &  &  \\
&  & 16 & \foreignlanguage{greek}{ινα απολαμβανωϲιν τα ιϲα} & 19 &  &  \\
& \textbf{35} &  & \foreignlanguage{greek}{πλην αγαπατε τουϲ εχθρουϲ υμων} & 5 &  &  \\
&  & 6 & \foreignlanguage{greek}{και αγαθοποιειτε και δανιζετε μη} & 10 &  &  \\
&  & 10 & \foreignlanguage{greek}{δενα απελπιζοντεϲ και εϲται ο μι} & 15 &  &  \\
&  & 15 & \foreignlanguage{greek}{ϲθοϲ υμων πολυϲ και εϲται υιοι} & 20 &  &  \\
&  & 21 & \foreignlanguage{greek}{υψιϲτου οτι αυτοϲ χρηϲτοϲ εϲτιν επι} & 26 &  &  \\
&  & 27 & \foreignlanguage{greek}{τουϲ αχαριϲτουϲ και πονηρουϲ} & 30 &  &  \\
& \textbf{36} &  & \foreignlanguage{greek}{γιγνεϲθαι οικτιρμονεϲ καθωϲ ο πα} & 5 &  &  \\
&  & 5 & \foreignlanguage{greek}{τηρ υμων οικτιρμων εϲτιν} & 8 &  &  \\
& \textbf{37} &  & \foreignlanguage{greek}{και μη κρινετε ινα μη κριθηται} & 6 &  &  \\
[0.2em]
\cline{4-4}
\end{tabular}
\end{center}
\end{table}
}
\clearpage
\newpage
 {
 \setlength\arrayrulewidth{1pt}
\begin{table}
\begin{center}
\begin{tabular}{ccc|l|ccc}
\cline{4-4} \\ [-1em]
\multicolumn{7}{c}{\foreignlanguage{greek}{ευαγγελιον κατα λουκαν} \textbf{(\nospace{6:37})} } \\ \\ [-1em] % Si on veut ajouter les bordures latérales, remplacer {7}{c} par {7}{|c|}
\cline{4-4} \\
\cline{4-4}
&  &  & &  &  & \\ [-0.9em]
&  & 7 & \foreignlanguage{greek}{και μη καταδικαζεται και ου μη κατα} & 14 &  &  \\
&  & 14 & \foreignlanguage{greek}{δικαϲθητε απολυετε και απολυ} & 17 &  &  \\
&  & 17 & \foreignlanguage{greek}{θηϲεϲθαι διδοτε και δοθηϲεται υμι̅} & 4 & \textbf{38} &  \\
&  & 5 & \foreignlanguage{greek}{μετρον καλον ϲεϲαλευμενον πεπιεϲ} & 8 &  &  \\
&  & 8 & \foreignlanguage{greek}{μενον υπερεκχυννομενον δωϲουϲιν} & 10 &  &  \\
&  & 11 & \foreignlanguage{greek}{ειϲ τον κολπον υμων ω γαρ μετρω με} & 18 &  &  \\
&  & 18 & \foreignlanguage{greek}{τριτε αντιμετρηθηϲεται υμιν} & 20 &  &  \\
& \textbf{39} &  & \foreignlanguage{greek}{ειπεν δε και παραβολην αυτοιϲ μη δυνα} & 7 &  &  \\
&  & 7 & \foreignlanguage{greek}{τε τυφλοϲ τυφλον οδηγειν ουχι αμφο} & 12 &  &  \\
&  & 12 & \foreignlanguage{greek}{τεροι ειϲ βοθυνον ενπεϲουνται} & 15 &  &  \\
& \textbf{40} &  & \foreignlanguage{greek}{ουκ εϲτιν μαθητηϲ υπερ τον διδαϲκαλο̅} & 6 &  &  \\
&  & 7 & \foreignlanguage{greek}{κατηρτιϲμενοϲ δε παϲ εϲται ωϲ ο διδα} & 13 &  &  \\
&  & 13 & \foreignlanguage{greek}{ϲκαλοϲ αυτου} & 14 &  &  \\
& \textbf{41} &  & \foreignlanguage{greek}{τι δε βλεπειϲ το καλφοϲ εν τω οφθαλμω} & 8 &  &  \\
&  & 9 & \foreignlanguage{greek}{του αδελφου ϲου την δε δοκον την εν} & 16 &  &  \\
&  & 17 & \foreignlanguage{greek}{τω ιδιω οφθαλμω ου κατανοειϲ} & 21 &  &  \\
& \textbf{42} &  & \foreignlanguage{greek}{η πωϲ δυναϲαι λεγειν τω αδελφω ϲου} & 7 &  &  \\
&  & 8 & \foreignlanguage{greek}{αδελφε αφεϲ εκβαλω το καρφοϲ το εν} & 14 &  &  \\
&  & 15 & \foreignlanguage{greek}{τω οφθαλμω ϲου αυτοϲ την εν τω ο} & 22 &  &  \\
&  & 22 & \foreignlanguage{greek}{φθαλμω ϲου δοκον ου βλεπων} & 26 &  &  \\
&  & 27 & \foreignlanguage{greek}{υποκριτα εκβαλε πρωτον την δοκον εκ} & 32 &  &  \\
&  & 33 & \foreignlanguage{greek}{του οφθαλμου ϲου και τοτε διαβλε} & 38 &  &  \\
&  & 38 & \foreignlanguage{greek}{ψειϲ το καρφοϲ το εν τω οφθαλμω του α} & 46 &  &  \\
&  & 46 & \foreignlanguage{greek}{δελφου ϲου εκβαλειν} & 48 &  &  \\
& \textbf{43} &  & \foreignlanguage{greek}{ου γαρ εϲτιν δενδρον καλον ποιουν} & 6 &  &  \\
&  & 7 & \foreignlanguage{greek}{καρπον κακον ουδε παλιν δενδρο̅} & 11 &  &  \\
&  & 12 & \foreignlanguage{greek}{ϲαπρον ποιουν καρπον καλον εκα} & 1 & \textbf{44} &  \\
&  & 1 & \foreignlanguage{greek}{ϲτον γαρ δενδρον εκ του ιδιου καρπου} & 7 &  &  \\
&  & 8 & \foreignlanguage{greek}{γιγνωϲκεται} & 8 &  &  \\
&  & 9 & \foreignlanguage{greek}{ου γαρ εξ ακανθων ϲυλλεγουϲιν ϲυκα} & 14 &  &  \\
[0.2em]
\cline{4-4}
\end{tabular}
\end{center}
\end{table}
}
\clearpage
\newpage
 {
 \setlength\arrayrulewidth{1pt}
\begin{table}
\begin{center}
\begin{tabular}{ccc|l|ccc}
\cline{4-4} \\ [-1em]
\multicolumn{7}{c}{\foreignlanguage{greek}{ευαγγελιον κατα λουκαν} \textbf{(\nospace{6:44})} } \\ \\ [-1em] % Si on veut ajouter les bordures latérales, remplacer {7}{c} par {7}{|c|}
\cline{4-4} \\
\cline{4-4}
&  &  & &  &  & \\ [-0.9em]
&  & 15 & \foreignlanguage{greek}{ουδε εκ βατου ϲταφυλην τρυγωϲιν} & 19 &  &  \\
& \textbf{45} &  & \foreignlanguage{greek}{ο αγαθοϲ ανθρωποϲ εκ του αγαθου θη} & 7 &  &  \\
&  & 7 & \foreignlanguage{greek}{ϲαυρου τηϲ καρδιαϲ αυτου προφερει α} & 12 &  &  \\
&  & 12 & \foreignlanguage{greek}{γαθον και ο πονηροϲ ανθρωποϲ εκ} & 17 &  &  \\
&  & 18 & \foreignlanguage{greek}{του πονηρου προφερει πονηρον} & 21 &  &  \\
&  & 22 & \foreignlanguage{greek}{εκ γαρ περιϲευματοϲ καρδιαϲ λαλει} & 26 &  &  \\
&  & 27 & \foreignlanguage{greek}{το ϲτομα αυτου τι δε με καλει} & 4 & \textbf{46} &  \\
&  & 4 & \foreignlanguage{greek}{τε \textoverline{κε} \textoverline{κε} και ου ποιειτε α λεγω} & 11 &  &  \\
& \textbf{47} &  & \foreignlanguage{greek}{παϲ ο ερχομενοϲ προϲ με και ακουων} & 7 &  &  \\
&  & 8 & \foreignlanguage{greek}{μου των λογων και ποιων αυτουϲ} & 13 &  &  \\
&  & 14 & \foreignlanguage{greek}{υποδιξω υμιν τινι εϲτιν ομοιοϲ} & 18 &  &  \\
& \textbf{48} &  & \foreignlanguage{greek}{ομοιοϲ εϲτιν ανθρωπω οικοδομουντι} & 4 &  &  \\
&  & 5 & \foreignlanguage{greek}{οικειαν οϲ εϲκαψεν και εβαθυνεν} & 9 &  &  \\
&  & 10 & \foreignlanguage{greek}{και εθηκεν θεμελιον επι την πετρα̅} & 15 &  &  \\
&  & 16 & \foreignlanguage{greek}{πλημυρηϲ δε γενομενηϲ προϲερη} & 19 &  &  \\
&  & 19 & \foreignlanguage{greek}{ξεν ο ποταμοϲ τη οικεια εκεινη και} & 25 &  &  \\
&  & 26 & \foreignlanguage{greek}{ουκ ιϲχυϲεν ϲαλευϲαι αυτην δια το} & 32 &  &  \\
&  & 33 & \foreignlanguage{greek}{καλωϲ οικοδομηϲθαι αυτην} & 35 &  &  \\
& \textbf{49} &  & \foreignlanguage{greek}{ο δε ακουϲαϲ και μη ποιηϲαϲ ομοιοϲ} & 7 &  &  \\
&  & 8 & \foreignlanguage{greek}{εϲτιν ανθρωπω οικοδομουντι οικει} & 11 &  &  \\
&  & 11 & \foreignlanguage{greek}{αν επι την γην χωριϲ θεμελιου} & 16 &  &  \\
&  & 17 & \foreignlanguage{greek}{και προϲερρηξεν αυτη ο ποταμοϲ και} & 22 &  &  \\
&  & 23 & \foreignlanguage{greek}{ευθεωϲ επεϲεν και εγενετο το ρη} & 28 &  &  \\
&  & 28 & \foreignlanguage{greek}{γμα τηϲ οικειαϲ εκεινηϲ μεγα} & 32 &  &  \\
& \mygospelchapter &  & \foreignlanguage{greek}{επειδη επληρωϲεν παντα τα ρημα} & 5 &  &  \\
&  & 5 & \foreignlanguage{greek}{τα αυτου ειϲ ταϲ ακοαϲ του λαου} & 11 &  &  \\
&  & 12 & \foreignlanguage{greek}{ειϲηλθεν ειϲ καφαρναουμ} & 14 &  &  \\
& \textbf{2} &  & \foreignlanguage{greek}{εκατονταρχου δε τινοϲ δουλοϲ κα} & 5 &  &  \\
&  & 5 & \foreignlanguage{greek}{κωϲ εχων ημελλεν τελευταν} & 8 &  &  \\
&  & 9 & \foreignlanguage{greek}{οϲ ην αυτω εντιμοϲ} & 12 &  &  \\
[0.2em]
\cline{4-4}
\end{tabular}
\end{center}
\end{table}
}
\clearpage
\newpage
 {
 \setlength\arrayrulewidth{1pt}
\begin{table}
\begin{center}
\begin{tabular}{ccc|l|ccc}
\cline{4-4} \\ [-1em]
\multicolumn{7}{c}{\foreignlanguage{greek}{ευαγγελιον κατα λουκαν} \textbf{(\nospace{7:3})} } \\ \\ [-1em] % Si on veut ajouter les bordures latérales, remplacer {7}{c} par {7}{|c|}
\cline{4-4} \\
\cline{4-4}
&  &  & &  &  & \\ [-0.9em]
& \textbf{3} &  & \foreignlanguage{greek}{ακουϲαϲ δε περι του \textoverline{ιυ} απεϲτιλεν προϲ αυ} & 8 &  &  \\
&  & 8 & \foreignlanguage{greek}{τον πρεϲβυτερουϲ των ιουδαιων} & 11 &  &  \\
&  & 12 & \foreignlanguage{greek}{ερωτων αυτο οπωϲ ελθων διαϲωϲη} & 16 &  &  \\
&  & 17 & \foreignlanguage{greek}{τον δουλον αυτου} & 19 &  &  \\
& \textbf{4} &  & \foreignlanguage{greek}{οι δε παραγενομενοι προϲ τον \textoverline{ιν} παρε} & 7 &  &  \\
&  & 7 & \foreignlanguage{greek}{καλουν αυτον ϲπουδεωϲ λεγοντεϲ} & 10 &  &  \\
&  & 11 & \foreignlanguage{greek}{οτι αξιοϲ εϲτιν ω παρεξη τουτο αγα} & 1 & \textbf{5} &  \\
&  & 1 & \foreignlanguage{greek}{πα γαρ το εθνοϲ ημων και την ϲυναγω} & 8 &  &  \\
&  & 8 & \foreignlanguage{greek}{γην αυτοϲ εποιηϲεν ημιν} & 11 &  &  \\
& \textbf{6} &  & \foreignlanguage{greek}{ο δε \textoverline{ιϲ} επορευετο ϲυν αυτοιϲ ηδη δε αυ} & 9 &  &  \\
&  & 9 & \foreignlanguage{greek}{του ου μακραν εχοντοϲ απο τηϲ οικειαϲ} & 15 &  &  \\
&  & 16 & \foreignlanguage{greek}{επεμψεν προϲ αυτουϲ φιλουϲ ο εκατο̅} & 21 &  &  \\
&  & 21 & \foreignlanguage{greek}{ταρχηϲ λεγων αυτω \textoverline{κε} μη ϲκυλλου} & 26 &  &  \\
&  & 27 & \foreignlanguage{greek}{ου γαρ εικανοϲ ειμει ινα μου υπο την} & 34 &  &  \\
&  & 35 & \foreignlanguage{greek}{ϲτεγην ειϲελθηϲ διο ουδε εμαυτον} & 3 & \textbf{7} &  \\
&  & 4 & \foreignlanguage{greek}{ηξιωϲα προϲ ϲε ελθειν αλλα ειπε λο} & 10 &  &  \\
&  & 10 & \foreignlanguage{greek}{γω και ιαθηϲεται ο παιϲ μου} & 15 &  &  \\
& \textbf{8} &  & \foreignlanguage{greek}{και γαρ εγω ανθρωποϲ ειμει υπο εξου} & 7 &  &  \\
&  & 7 & \foreignlanguage{greek}{ϲιαν ταϲϲομενοϲ εχων υπ εμαυτον} & 11 &  &  \\
&  & 12 & \foreignlanguage{greek}{ϲτρατιωταϲ και λεγω τουτω πορευ} & 16 &  &  \\
&  & 16 & \foreignlanguage{greek}{θητι και πορευεται και αλλω ερχου} & 21 &  &  \\
&  & 22 & \foreignlanguage{greek}{και ερχεται και τω δουλω μου ποιη} & 28 &  &  \\
&  & 28 & \foreignlanguage{greek}{ϲον τουτο και ποιει} & 31 &  &  \\
& \textbf{9} &  & \foreignlanguage{greek}{ακουϲαϲ δε ταυτα ο \textoverline{ιϲ} εθαυμαϲεν αυτο̅} & 7 &  &  \\
&  & 8 & \foreignlanguage{greek}{και ϲτραφειϲ τω οχλω ειπεν λεγω υ} & 14 &  &  \\
&  & 14 & \foreignlanguage{greek}{μιν ουδε εν τω ιϲραηλ τοϲαυτην πι} & 20 &  &  \\
&  & 20 & \foreignlanguage{greek}{ϲτιν ευρον και υποϲτρεψαντεϲ} & 2 & \textbf{10} &  \\
&  & 3 & \foreignlanguage{greek}{ειϲ τον οικον οι πεμφθεντεϲ ευρον} & 8 &  &  \\
&  & 9 & \foreignlanguage{greek}{τον δουλον υγιαινοντα} & 11 &  &  \\
& \textbf{11} &  & \foreignlanguage{greek}{και εγενετο τη εξηϲ επορευθη ειϲ πολι̅} & 7 &  &  \\
[0.2em]
\cline{4-4}
\end{tabular}
\end{center}
\end{table}
}
\clearpage
\newpage
 {
 \setlength\arrayrulewidth{1pt}
\begin{table}
\begin{center}
\begin{tabular}{ccc|l|ccc}
\cline{4-4} \\ [-1em]
\multicolumn{7}{c}{\foreignlanguage{greek}{ευαγγελιον κατα λουκαν} \textbf{(\nospace{7:11})} } \\ \\ [-1em] % Si on veut ajouter les bordures latérales, remplacer {7}{c} par {7}{|c|}
\cline{4-4} \\
\cline{4-4}
&  &  & &  &  & \\ [-0.9em]
&  & 8 & \foreignlanguage{greek}{καλουμενην ναιν και ϲυνεπορευον} & 11 &  &  \\
&  & 11 & \foreignlanguage{greek}{το αυτω οι μαθηται αυτου και οχλοϲ πολυϲ} & 18 &  &  \\
& \textbf{12} &  & \foreignlanguage{greek}{ωϲ δε ηγγειζεν τη πυλη τηϲ πολεωϲ} & 7 &  &  \\
&  & 8 & \foreignlanguage{greek}{και ιδου εξεκομιζετο τεθνηκωϲ μο} & 12 &  &  \\
&  & 12 & \foreignlanguage{greek}{νογενηϲ υιοϲ τη μητρι αυτου και αυ} & 18 &  &  \\
&  & 18 & \foreignlanguage{greek}{τη χηρα και οχλοϲ τηϲ πολεωϲ ικα} & 24 &  &  \\
&  & 24 & \foreignlanguage{greek}{νοϲ ην ϲυν αυτη} & 27 &  &  \\
& \textbf{13} &  & \foreignlanguage{greek}{και ιδων αυτην ο \textoverline{ιϲ} εϲπλαγχνιϲθη επ αυ} & 8 &  &  \\
&  & 8 & \foreignlanguage{greek}{τη και ειπεν αυτη μη κλεε} & 13 &  &  \\
& \textbf{14} &  & \foreignlanguage{greek}{και προϲελθων ηψατο τηϲ ϲορου οι δε} & 7 &  &  \\
&  & 8 & \foreignlanguage{greek}{βαϲταζοντεϲ εϲτηϲαν και ειπεν} & 11 &  &  \\
&  & 12 & \foreignlanguage{greek}{νεανιϲκε ϲοι λεγω εγερθητι και ανε} & 2 & \textbf{15} &  \\
&  & 2 & \foreignlanguage{greek}{καθειϲεν ο νεκροϲ και ηρξατο λαλειν} & 7 &  &  \\
&  & 8 & \foreignlanguage{greek}{και εδωκεν αυτον τη μητρι αυτου} & 13 &  &  \\
& \textbf{16} &  & \foreignlanguage{greek}{ελαβεν δε φοβοϲ απανταϲ και εδοξα} & 6 &  &  \\
&  & 6 & \foreignlanguage{greek}{ζον τον \textoverline{θν} λεγοντεϲ οτι προφητηϲ} & 11 &  &  \\
&  & 12 & \foreignlanguage{greek}{μεγαϲ εγηγερται εν ημιν και οτι επε} & 18 &  &  \\
&  & 18 & \foreignlanguage{greek}{ϲκεψατο ο \textoverline{θϲ} τον λαον αυτου} & 23 &  &  \\
& \textbf{17} &  & \foreignlanguage{greek}{και εξηλθεν ο λογοϲ ουτοϲ εν ολη τη ιου} & 9 &  &  \\
&  & 9 & \foreignlanguage{greek}{δαια περι αυτου και παϲη τη περιχωρω} & 15 &  &  \\
& \textbf{18} &  & \foreignlanguage{greek}{και απηγγειλον ιωαννη οι μαθηται αυ} & 6 &  &  \\
&  & 6 & \foreignlanguage{greek}{του περι παντων τουτων} & 9 &  &  \\
&  & 10 & \foreignlanguage{greek}{και προϲκαλεϲαμενοϲ δυο τιναϲ των} & 14 &  &  \\
&  & 15 & \foreignlanguage{greek}{μαθητων αυτου ο ιωαννηϲ επεμ} & 1 & \textbf{19} &  \\
&  & 1 & \foreignlanguage{greek}{ψεν προϲ τον \textoverline{ιν} λεγων ϲυ ει ο ερχομε} & 9 &  &  \\
&  & 9 & \foreignlanguage{greek}{νοϲ η ετερον προϲδοκωμεν} & 12 &  &  \\
& \textbf{20} &  & \foreignlanguage{greek}{παραγενομενοι δε προϲ αυτον οι αν} & 6 &  &  \\
&  & 6 & \foreignlanguage{greek}{δρεϲ ειπον ιωαννηϲ ο βαπτιϲτηϲ α} & 11 &  &  \\
&  & 11 & \foreignlanguage{greek}{πεϲτιλεν ημαϲ προϲ ϲε λεγων} & 15 &  &  \\
&  & 16 & \foreignlanguage{greek}{ϲυ ει ο ερχομενοϲ η ετερον προϲδοκωμε̅} & 22 &  &  \\
[0.2em]
\cline{4-4}
\end{tabular}
\end{center}
\end{table}
}
\clearpage
\newpage
 {
 \setlength\arrayrulewidth{1pt}
\begin{table}
\begin{center}
\begin{tabular}{ccc|l|ccc}
\cline{4-4} \\ [-1em]
\multicolumn{7}{c}{\foreignlanguage{greek}{ευαγγελιον κατα λουκαν} \textbf{(\nospace{7:21})} } \\ \\ [-1em] % Si on veut ajouter les bordures latérales, remplacer {7}{c} par {7}{|c|}
\cline{4-4} \\
\cline{4-4}
&  &  & &  &  & \\ [-0.9em]
& \textbf{21} &  & \foreignlanguage{greek}{εν εκεινη τη ωρα εθεραπευϲεν πολλουϲ} & 6 &  &  \\
&  & 7 & \foreignlanguage{greek}{απο νοϲων και μαϲτιγων και πνευμα} & 12 &  &  \\
&  & 12 & \foreignlanguage{greek}{των πονηρων και τυφλοιϲ πολλοιϲ ε} & 17 &  &  \\
&  & 17 & \foreignlanguage{greek}{χαριϲατο το βλεπειν} & 19 &  &  \\
& \textbf{22} &  & \foreignlanguage{greek}{και αποκριθειϲ ειπεν αυτοιϲ πορευθεν} & 5 &  &  \\
&  & 5 & \foreignlanguage{greek}{τεϲ ειπατε ιωαννη α ειδατε και ηκου} & 11 &  &  \\
&  & 11 & \foreignlanguage{greek}{ϲατε τυφλοι αναβλεπουϲιν και χω} & 15 &  &  \\
&  & 15 & \foreignlanguage{greek}{λοι περιπατουϲιν λεπροι καθαριζονται} & 18 &  &  \\
&  & 19 & \foreignlanguage{greek}{και κωφοι ακουουϲιν νεκροι εγειρο̅} & 23 &  &  \\
&  & 23 & \foreignlanguage{greek}{ται πτωχοι ευαγγελιζονται και μα} & 2 & \textbf{23} &  \\
&  & 2 & \foreignlanguage{greek}{καριοϲ εϲτιν οϲ αν μη ϲκανδαλιϲθη} & 7 &  &  \\
&  & 8 & \foreignlanguage{greek}{εν εμοι} & 9 &  &  \\
& \textbf{24} &  & \foreignlanguage{greek}{απελθοντων δε των αγγελων ιωαννου} & 5 &  &  \\
&  & 6 & \foreignlanguage{greek}{ηρξατο λεγειν προϲ τουϲ οχλουϲ περι} & 11 &  &  \\
&  & 12 & \foreignlanguage{greek}{ιωαννου τι εξηλθατε ειϲ την ερη} & 17 &  &  \\
&  & 17 & \foreignlanguage{greek}{μον θεαϲαϲθαι καλαμον υπο ανεμου} & 21 &  &  \\
&  & 22 & \foreignlanguage{greek}{ϲαλευομενον} & 22 &  &  \\
& \textbf{25} &  & \foreignlanguage{greek}{αλλα τι εξηλθατε ιδειν ανθρωπον} & 5 &  &  \\
&  & 6 & \foreignlanguage{greek}{εν μαλακοιϲ ιματιοιϲ ημφιεϲμενον} & 9 &  &  \\
&  & 10 & \foreignlanguage{greek}{ιδου οι εν ιματιϲμω ενδοξω και τρυ} & 16 &  &  \\
&  & 16 & \foreignlanguage{greek}{φη υπαρχοντεϲ εν τοιϲ βαϲιλειοιϲ} & 20 &  &  \\
&  & 21 & \foreignlanguage{greek}{ειϲιν αλλα τι εξεληλυθατε ιδει̅} & 4 & \textbf{26} &  \\
&  & 5 & \foreignlanguage{greek}{προφητην νε λεγω υμιν και περιϲ} & 10 &  &  \\
&  & 10 & \foreignlanguage{greek}{ϲοτερον προφητου} & 11 &  &  \\
& \textbf{27} &  & \foreignlanguage{greek}{ουτοϲ εϲτιν περι ου γεγραπται ιδου α} & 7 &  &  \\
&  & 7 & \foreignlanguage{greek}{ποϲτελλω τον αγγελον μου προ προ} & 12 &  &  \\
&  & 12 & \foreignlanguage{greek}{ϲωπου ϲου οϲ καταϲκευαϲει την ο} & 17 &  &  \\
&  & 17 & \foreignlanguage{greek}{δον ϲου εμπροϲθεν ϲου λεγω δε} & 2 & \textbf{28} &  \\
&  & 3 & \foreignlanguage{greek}{υμιν οτι μιζων εν γεννητοιϲ γυναι} & 8 &  &  \\
&  & 8 & \foreignlanguage{greek}{κων ιωαννου ουδειϲ εϲτιν} & 11 &  &  \\
[0.2em]
\cline{4-4}
\end{tabular}
\end{center}
\end{table}
}
\clearpage
\newpage
 {
 \setlength\arrayrulewidth{1pt}
\begin{table}
\begin{center}
\begin{tabular}{ccc|l|ccc}
\cline{4-4} \\ [-1em]
\multicolumn{7}{c}{\foreignlanguage{greek}{ευαγγελιον κατα λουκαν} \textbf{(\nospace{7:28})} } \\ \\ [-1em] % Si on veut ajouter les bordures latérales, remplacer {7}{c} par {7}{|c|}
\cline{4-4} \\
\cline{4-4}
&  &  & &  &  & \\ [-0.9em]
&  & 12 & \foreignlanguage{greek}{και ο μικροτεροϲ εν τη βαϲιλεια του \textoverline{θυ}} & 19 &  &  \\
&  & 20 & \foreignlanguage{greek}{μιζων αυτου εϲτιν} & 22 &  &  \\
& \textbf{29} &  & \foreignlanguage{greek}{και παϲ ο λαοϲ ακουϲαϲ και οι τελωναι} & 8 &  &  \\
&  & 9 & \foreignlanguage{greek}{εδικαιωϲαν τον \textoverline{θν} βαπτιϲθεντεϲ} & 12 &  &  \\
&  & 13 & \foreignlanguage{greek}{το βαπτιϲμα ιωαννου} & 15 &  &  \\
& \textbf{30} &  & \foreignlanguage{greek}{οι δε φαριϲαιοι και οι νομικοι την βου} & 8 &  &  \\
&  & 8 & \foreignlanguage{greek}{λην του \textoverline{θυ} ηθετηϲαν ειϲ εαυτουϲ} & 13 &  &  \\
&  & 15 & \foreignlanguage{greek}{μη βαπτιϲθεντεϲ υπ αυτου το βα} & 2 & \textbf{31} &  \\
&  & 2 & \foreignlanguage{greek}{πτιϲμα ιωαννου} & 3 &  &  \\
&  & 4 & \foreignlanguage{greek}{τινι ουν ομοιωϲω τουϲ ανθρωπουϲ} & 8 &  &  \\
&  & 9 & \foreignlanguage{greek}{τηϲ γενεαϲ ταυτηϲ και τινι ειϲιν} & 14 &  &  \\
&  & 15 & \foreignlanguage{greek}{ομοιοι ομοιοι ειϲιν παιδιοιϲ τοιϲ ε̅} & 5 & \textbf{32} &  \\
&  & 6 & \foreignlanguage{greek}{αγοραιϲ καθημενοιϲ και προϲφω} & 9 &  &  \\
&  & 9 & \foreignlanguage{greek}{νουϲιν αλληλοιϲ λεγοντα} & 11 &  &  \\
&  & 12 & \foreignlanguage{greek}{ηυληϲαμεν υμιν και ουκ ωρχηϲαϲθε} & 16 &  &  \\
&  & 17 & \foreignlanguage{greek}{εθρηνηϲαμεν και ουκ εκλαυϲατε} & 20 &  &  \\
& \textbf{33} &  & \foreignlanguage{greek}{εληλυθεν γαρ ο ιωαννηϲ ο βαπτιϲτηϲ} & 6 &  &  \\
&  & 7 & \foreignlanguage{greek}{μη εϲθιων αρτον μηδε πινων οινο̅} & 12 &  &  \\
&  & 13 & \foreignlanguage{greek}{και λεγεται δαιμονιον εχει} & 16 &  &  \\
& \textbf{34} &  & \foreignlanguage{greek}{εληλυθεν ο υιοϲ του ανθρωπου εϲθι} & 6 &  &  \\
&  & 6 & \foreignlanguage{greek}{ων και πινων και λεγεται ιδου αν} & 12 &  &  \\
&  & 12 & \foreignlanguage{greek}{θρωποϲ φαγοϲ και οινοποτηϲ φιλοϲ} & 16 &  &  \\
&  & 17 & \foreignlanguage{greek}{τελωνων και αμαρτωλων και εδι} & 2 & \textbf{35} &  \\
&  & 2 & \foreignlanguage{greek}{καιωθη η ϲοφια απο παντων των} & 7 &  &  \\
&  & 8 & \foreignlanguage{greek}{τεκνων αυτηϲ} & 9 &  &  \\
& \textbf{36} &  & \foreignlanguage{greek}{ηρωτα δε τιϲ αυτον των φαριϲαιων} & 6 &  &  \\
&  & 7 & \foreignlanguage{greek}{ινα φαγη μετ αυτου και ειϲελθων} & 12 &  &  \\
&  & 13 & \foreignlanguage{greek}{ειϲ τον οικον του φαριϲαιου ανεκλιθη} & 18 &  &  \\
& \textbf{37} &  & \foreignlanguage{greek}{και ιδου γυνη τιϲ ην εν τη πολει αμαρ} & 9 &  &  \\
&  & 9 & \foreignlanguage{greek}{τωλοϲ και επιγνουϲα οτι κατακει} & 13 &  &  \\
[0.2em]
\cline{4-4}
\end{tabular}
\end{center}
\end{table}
}
\clearpage
\newpage
 {
 \setlength\arrayrulewidth{1pt}
\begin{table}
\begin{center}
\begin{tabular}{ccc|l|ccc}
\cline{4-4} \\ [-1em]
\multicolumn{7}{c}{\foreignlanguage{greek}{ευαγγελιον κατα λουκαν} \textbf{(\nospace{7:37})} } \\ \\ [-1em] % Si on veut ajouter les bordures latérales, remplacer {7}{c} par {7}{|c|}
\cline{4-4} \\
\cline{4-4}
&  &  & &  &  & \\ [-0.9em]
&  & 13 & \foreignlanguage{greek}{ται εν τη οικεια του φαριϲαιου} & 18 &  &  \\
&  & 19 & \foreignlanguage{greek}{κομιϲαϲα αλαβαϲτρον μυρου και ϲτα} & 2 & \textbf{38} &  \\
&  & 2 & \foreignlanguage{greek}{ϲα οπιϲω παρα τουϲ ποδαϲ αυτου κλαι} & 8 &  &  \\
&  & 8 & \foreignlanguage{greek}{ουϲα τοιϲ δακρυϲιν ηρξατο βρεχειν} & 12 &  &  \\
&  & 13 & \foreignlanguage{greek}{τουϲ ποδαϲ αυτου και ταιϲ θριξιν τηϲ} & 19 &  &  \\
&  & 20 & \foreignlanguage{greek}{κεφαληϲ αυτηϲ εξεμαξεν και κα} & 25 &  &  \\
&  & 25 & \foreignlanguage{greek}{τεφιλει τουϲ ποδαϲ αυτου και ηλι} & 30 &  &  \\
&  & 30 & \foreignlanguage{greek}{φεν τω μυρω} & 32 &  &  \\
& \textbf{39} &  & \foreignlanguage{greek}{ιδων δε ο φαριϲαιοϲ ο καλεϲαϲ αυτον} & 7 &  &  \\
&  & 8 & \foreignlanguage{greek}{ειπεν εν εαυτω ουτοϲ ει ην προφη} & 14 &  &  \\
&  & 14 & \foreignlanguage{greek}{τηϲ εγιγνωϲκεν αν τιϲ και ποταπη} & 19 &  &  \\
&  & 20 & \foreignlanguage{greek}{η γυνη ητιϲ απτεται αυτου οτι α} & 26 &  &  \\
&  & 26 & \foreignlanguage{greek}{μαρτωλοϲ εϲτιν} & 27 &  &  \\
& \textbf{40} &  & \foreignlanguage{greek}{και αποκριθειϲ ειπεν ο \textoverline{ιϲ} προϲ αυτον} & 7 &  &  \\
&  & 8 & \foreignlanguage{greek}{ϲιμων εχω ϲοι τι ειπειν} & 12 &  &  \\
&  & 13 & \foreignlanguage{greek}{ο δε διδαϲκαλε φηϲι ειπε δυο} & 1 & \textbf{41} &  \\
&  & 2 & \foreignlanguage{greek}{χρεοφιλεται ηϲαν δανιϲτη τινι ο} & 6 &  &  \\
&  & 7 & \foreignlanguage{greek}{ειϲ ωφιλεν δηναρια πεντακοϲια} & 10 &  &  \\
&  & 11 & \foreignlanguage{greek}{ο δε ετεροϲ πεντηκοντα μη εχον} & 2 & \textbf{42} &  \\
&  & 2 & \foreignlanguage{greek}{των δε αυτων αποδουναι αμφοτε} & 6 &  &  \\
&  & 6 & \foreignlanguage{greek}{ροιϲ εχαριϲατο τιϲ ουν αυτων πλε} & 11 &  &  \\
&  & 11 & \foreignlanguage{greek}{ον αγαπηϲει αυτον} & 13 &  &  \\
& \textbf{43} &  & \foreignlanguage{greek}{ο δε ϲιμων ειπεν υπολαμβανω οτι} & 6 &  &  \\
&  & 7 & \foreignlanguage{greek}{ω το πλιον εχαριϲατο ο δε \textoverline{ιϲ} ειπε̅} & 14 &  &  \\
&  & 15 & \foreignlanguage{greek}{αυτω ορθωϲ εκριναϲ} & 17 &  &  \\
& \textbf{44} &  & \foreignlanguage{greek}{και ϲτραφειϲ προϲ την γυναικα τω ϲι} & 7 &  &  \\
&  & 7 & \foreignlanguage{greek}{μωνι εφη βλεπειϲ ταυτην την γυ} & 12 &  &  \\
&  & 12 & \foreignlanguage{greek}{ναικα ειϲηλθον ϲου ειϲ τον οικον} & 17 &  &  \\
&  & 18 & \foreignlanguage{greek}{υδωρ υπο ποδαϲ μοι ουκ επεδωκαϲ} & 23 &  &  \\
&  & 24 & \foreignlanguage{greek}{αυτη δε τοιϲ δακρυϲιν εβρεξεν μου} & 29 &  &  \\
[0.2em]
\cline{4-4}
\end{tabular}
\end{center}
\end{table}
}
\clearpage
\newpage
 {
 \setlength\arrayrulewidth{1pt}
\begin{table}
\begin{center}
\begin{tabular}{ccc|l|ccc}
\cline{4-4} \\ [-1em]
\multicolumn{7}{c}{\foreignlanguage{greek}{ευαγγελιον κατα λουκαν} \textbf{(\nospace{7:44})} } \\ \\ [-1em] % Si on veut ajouter les bordures latérales, remplacer {7}{c} par {7}{|c|}
\cline{4-4} \\
\cline{4-4}
&  &  & &  &  & \\ [-0.9em]
&  & 30 & \foreignlanguage{greek}{τουϲ ποδαϲ και ταιϲ θριξιν αυτηϲ εξε} & 36 &  &  \\
&  & 36 & \foreignlanguage{greek}{μαξεν φιλημα μοι ουκ εδωκαϲ αυ} & 5 & \textbf{45} &  \\
&  & 5 & \foreignlanguage{greek}{τη δε αφ ηϲ ειϲηλθον ου διελειπεν κα} & 12 &  &  \\
&  & 12 & \foreignlanguage{greek}{ταφιλουϲα μου τουϲ ποδαϲ} & 15 &  &  \\
& \textbf{46} &  & \foreignlanguage{greek}{ελεω την κεφαλην μου ουκ ηλιψαϲ} & 6 &  &  \\
&  & 7 & \foreignlanguage{greek}{αυτη δε μυρω ηλιψεν ου χαριν λε} & 3 & \textbf{47} &  \\
&  & 3 & \foreignlanguage{greek}{γω ϲοι αφιενται αυτηϲ αι αμαρτιαι} & 8 &  &  \\
&  & 9 & \foreignlanguage{greek}{αι πολλαι οτι ηγαπηϲεν πολυ ω δε} & 15 &  &  \\
&  & 16 & \foreignlanguage{greek}{ολειγον αφιεται ολιγον αγαπα} & 19 &  &  \\
& \textbf{48} &  & \foreignlanguage{greek}{ειπεν δε αυτη αφιενται ϲου αι αμαρτιαι} & 7 &  &  \\
& \textbf{49} &  & \foreignlanguage{greek}{και ηρξαντο οι ϲυνανακειμενοι λε} & 5 &  &  \\
&  & 5 & \foreignlanguage{greek}{γειν προϲ εαυτουϲ τιϲ ουτοϲ εϲτιν οϲ} & 12 &  &  \\
&  & 13 & \foreignlanguage{greek}{και αμαρτιαϲ αφιηϲιν} & 15 &  &  \\
& \textbf{50} &  & \foreignlanguage{greek}{ειπεν δε προϲ την γυναικα η πιϲτιϲ} & 7 &  &  \\
&  & 8 & \foreignlanguage{greek}{ϲου ϲεϲωκεν ϲε πορευου ειϲ ειρηνην} & 13 &  &  \\
& \mygospelchapter &  & \foreignlanguage{greek}{και εγενετο εν τω καθεξηϲ και αυ} & 7 &  &  \\
&  & 7 & \foreignlanguage{greek}{τοϲ διωδευεν κατα πολιν και κωμη̅} & 12 &  &  \\
&  & 13 & \foreignlanguage{greek}{κηρυϲϲων και ευαγγελιζομενοϲ} & 15 &  &  \\
&  & 16 & \foreignlanguage{greek}{την βαϲιλειαν του \textoverline{θυ} και οι δεκα} & 22 &  &  \\
&  & 22 & \foreignlanguage{greek}{δυο ϲυν αυτω και γυναικεϲ τινεϲ} & 3 & \textbf{2} &  \\
&  & 4 & \foreignlanguage{greek}{αι ηϲαν τεθεραπευμεναι απο πνευ} & 8 &  &  \\
&  & 8 & \foreignlanguage{greek}{ματων πονηρων και αϲθενιων} & 11 &  &  \\
&  & 12 & \foreignlanguage{greek}{μαρια η καλουμενη μαγδαληνη} & 15 &  &  \\
&  & 16 & \foreignlanguage{greek}{αφ ηϲ \textoverline{ζ} δαιμονια εξεληλυθει και ι} & 2 & \textbf{3} &  \\
&  & 2 & \foreignlanguage{greek}{ωαννα γυνη χουζα επιτροπου ηρω} & 6 &  &  \\
&  & 6 & \foreignlanguage{greek}{δου και ϲουϲαννα και ετεραι πολλαι} & 11 &  &  \\
&  & 12 & \foreignlanguage{greek}{αιτινεϲ διηκονουν αυτοιϲ εκ των} & 16 &  &  \\
&  & 17 & \foreignlanguage{greek}{υπαρχοντων αυταιϲ} & 18 &  &  \\
& \textbf{4} &  & \foreignlanguage{greek}{ϲυνιοντοϲ δε οχλου πολλου και των} & 6 &  &  \\
&  & 7 & \foreignlanguage{greek}{κατα πολιν ειϲπορευομενων προϲ αυ} & 11 &  &  \\
[0.2em]
\cline{4-4}
\end{tabular}
\end{center}
\end{table}
}
\clearpage
\newpage
 {
 \setlength\arrayrulewidth{1pt}
\begin{table}
\begin{center}
\begin{tabular}{ccc|l|ccc}
\cline{4-4} \\ [-1em]
\multicolumn{7}{c}{\foreignlanguage{greek}{ευαγγελιον κατα λουκαν} \textbf{(\nospace{8:4})} } \\ \\ [-1em] % Si on veut ajouter les bordures latérales, remplacer {7}{c} par {7}{|c|}
\cline{4-4} \\
\cline{4-4}
&  &  & &  &  & \\ [-0.9em]
&  & 11 & \foreignlanguage{greek}{τον ειπεν δια παραβοληϲ} & 14 &  &  \\
& \textbf{5} &  & \foreignlanguage{greek}{εξηλθεν ο ϲπειρων ϲπειραι τον ϲπο} & 6 &  &  \\
&  & 6 & \foreignlanguage{greek}{ρον αυτου και εν τω ϲπιρειν αυτον} & 12 &  &  \\
&  & 13 & \foreignlanguage{greek}{α μεν επεϲεν παρα την οδον και κατε} & 20 &  &  \\
&  & 20 & \foreignlanguage{greek}{πατηθη και τα πετινα κατεφαγεν} & 24 &  &  \\
&  & 25 & \foreignlanguage{greek}{αυτο και ετερον επεϲεν επι την πε} & 6 & \textbf{6} &  \\
&  & 6 & \foreignlanguage{greek}{τραν και φυεν εξηρανθη δια το} & 11 &  &  \\
&  & 12 & \foreignlanguage{greek}{μη εχειν ικμαδα και ετερον επε} & 3 & \textbf{7} &  \\
&  & 3 & \foreignlanguage{greek}{ϲεν εν μεϲω των ακανθων και ϲυν} & 9 &  &  \\
&  & 9 & \foreignlanguage{greek}{φυειϲαι αι ακανθαι αποπνιξαν αυτο} & 14 &  &  \\
& \textbf{8} &  & \foreignlanguage{greek}{και ετερον επεϲεν επι την γην την} & 7 &  &  \\
&  & 8 & \foreignlanguage{greek}{αγαθην και φυεν εποιηϲεν καρπον} & 12 &  &  \\
&  & 13 & \foreignlanguage{greek}{εκατονταπλαϲιονα ταυτα λεγων ε} & 16 &  &  \\
&  & 16 & \foreignlanguage{greek}{φωνι ο εχων ωτα ακουειν ακουετω} & 21 &  &  \\
& \textbf{9} &  & \foreignlanguage{greek}{επηρωτων δε αυτον οι μαθηται τιϲ αυ} & 7 &  &  \\
&  & 7 & \foreignlanguage{greek}{τη ειη η παραβολη ο δε ειπεν} & 3 & \textbf{10} &  \\
&  & 4 & \foreignlanguage{greek}{υμιν δεδοτε γνωναι τα μυϲτηρια} & 8 &  &  \\
&  & 9 & \foreignlanguage{greek}{του \textoverline{θυ} τοιϲ δε λοιποιϲ εν παραβολαιϲ} & 15 &  &  \\
&  & 16 & \foreignlanguage{greek}{ινα βλεποντεϲ μη ιδωϲιν και ακου} & 21 &  &  \\
&  & 21 & \foreignlanguage{greek}{οντεϲ μη ϲυνιωϲιν} & 23 &  &  \\
& \textbf{11} &  & \foreignlanguage{greek}{εϲτιν δε αυτη η παραβολη ο ϲποροϲ} & 7 &  &  \\
&  & 8 & \foreignlanguage{greek}{εϲτιν ο λογοϲ του \textoverline{θυ} οι δε παρα την} & 4 & \textbf{12} &  \\
&  & 5 & \foreignlanguage{greek}{οδον ειϲιν οι ακουοντεϲ ειτα ερχε} & 10 &  &  \\
&  & 10 & \foreignlanguage{greek}{ται ο διαβολοϲ και ερει τον λογον α} & 17 &  &  \\
&  & 17 & \foreignlanguage{greek}{πο τηϲ καρδιαϲ αυτων ινα μη πι} & 23 &  &  \\
&  & 23 & \foreignlanguage{greek}{ϲτευϲαντεϲ ϲωθωϲιν οι δε επι τηϲ} & 4 & \textbf{13} &  \\
&  & 5 & \foreignlanguage{greek}{πετραϲ οι οταν ακουϲωϲιν μετα χα} & 10 &  &  \\
&  & 10 & \foreignlanguage{greek}{ραϲ δεχονται τον λογον και ουτοι ρι} & 16 &  &  \\
&  & 16 & \foreignlanguage{greek}{ζαν ουκ εχουϲιν οι προϲ καιρον πι} & 22 &  &  \\
&  & 22 & \foreignlanguage{greek}{ϲτευουϲιν και εν καιρω πιραϲμου αφι} & 27 &  &  \\
&  & 27 & \foreignlanguage{greek}{ϲτανται} & 27 &  &  \\
[0.2em]
\cline{4-4}
\end{tabular}
\end{center}
\end{table}
}
\clearpage
\newpage
 {
 \setlength\arrayrulewidth{1pt}
\begin{table}
\begin{center}
\begin{tabular}{ccc|l|ccc}
\cline{4-4} \\ [-1em]
\multicolumn{7}{c}{\foreignlanguage{greek}{ευαγγελιον κατα λουκαν} \textbf{(\nospace{8:14})} } \\ \\ [-1em] % Si on veut ajouter les bordures latérales, remplacer {7}{c} par {7}{|c|}
\cline{4-4} \\
\cline{4-4}
&  &  & &  &  & \\ [-0.9em]
& \textbf{14} &  & \foreignlanguage{greek}{το δε ειϲ ταϲ ακανθαϲ πεϲον ουτοι ειϲι̅} & 8 &  &  \\
&  & 9 & \foreignlanguage{greek}{οι ακουϲαντεϲ και υπο μεριμνων} & 13 &  &  \\
&  & 14 & \foreignlanguage{greek}{και πλουτου και ηδονων του βιου} & 19 &  &  \\
&  & 20 & \foreignlanguage{greek}{πορευομενοι ϲυνπνιγονται και ου} & 23 &  &  \\
&  & 24 & \foreignlanguage{greek}{τελεϲφορουϲιν το δε εν τη κα} & 5 & \textbf{15} &  \\
&  & 5 & \foreignlanguage{greek}{λη γη ουτοι ειϲιν οιτινεϲ εν καρδια} & 11 &  &  \\
&  & 12 & \foreignlanguage{greek}{καλη και αγαθη ακουϲαντεϲ τον λο} & 17 &  &  \\
&  & 17 & \foreignlanguage{greek}{γον κατεχουϲιν και καρποφο} & 20 &  &  \\
&  & 20 & \foreignlanguage{greek}{ρουϲιν εν υπομονη} & 22 &  &  \\
& \textbf{16} &  & \foreignlanguage{greek}{ουδειϲ δε λυχνον αψαϲ καλυπτει} & 5 &  &  \\
&  & 6 & \foreignlanguage{greek}{αυτον ϲκευει η υποκατω κλεινηϲ} & 10 &  &  \\
&  & 11 & \foreignlanguage{greek}{τιθηϲιν αλλ επι λυχνιαϲ επιτιθη} & 15 &  &  \\
&  & 15 & \foreignlanguage{greek}{ϲιν ινα οι ειϲπορευομενοι βλεπωϲι̅} & 19 &  &  \\
&  & 20 & \foreignlanguage{greek}{το φωϲ ου εϲτιν κρυπτον ο ου φα} & 6 & \textbf{17} &  \\
&  & 6 & \foreignlanguage{greek}{νερον γενηϲεται ουδε αποκρυ} & 9 &  &  \\
&  & 9 & \foreignlanguage{greek}{φον ο ου γνωϲθηϲεται και ειϲ φα} & 15 &  &  \\
&  & 15 & \foreignlanguage{greek}{νερον ελθη} & 16 &  &  \\
& \textbf{18} &  & \foreignlanguage{greek}{βλεπεται ουν πωϲ ακουεται οϲ γαρ} & 6 &  &  \\
&  & 7 & \foreignlanguage{greek}{εχη δοθηϲεται αυτω και οϲ εαν} & 12 &  &  \\
&  & 13 & \foreignlanguage{greek}{μη εχη και ο δοκει εχειν αρθηϲε} & 19 &  &  \\
&  & 19 & \foreignlanguage{greek}{ται απ αυτου} & 21 &  &  \\
& \textbf{19} &  & \foreignlanguage{greek}{παρεγενοντο δε προϲ αυτον η μη} & 6 &  &  \\
&  & 6 & \foreignlanguage{greek}{τηρ και οι αδελφοι αυτου και ου} & 12 &  &  \\
&  & 12 & \foreignlanguage{greek}{κ ηδυναντο ϲυντυχειν αυτω δι} & 16 &  &  \\
&  & 16 & \foreignlanguage{greek}{α τον οχλον και απηγγελθη} & 2 & \textbf{20} &  \\
&  & 3 & \foreignlanguage{greek}{αυτω η μητηρ ϲου και οι αδελ} & 9 &  &  \\
&  & 9 & \foreignlanguage{greek}{φοι ϲου εϲτηκαϲιν εξω ιδειν ϲε θε} & 15 &  &  \\
&  & 15 & \foreignlanguage{greek}{λοντεϲ ο δε αποκριθειϲ ει} & 4 & \textbf{21} &  \\
&  & 4 & \foreignlanguage{greek}{πεν προϲ αυτουϲ} & 6 &  &  \\
&  & 8 & \foreignlanguage{greek}{μητηρ μου και αδελφοι μου ουτοι} & 13 &  &  \\
[0.2em]
\cline{4-4}
\end{tabular}
\end{center}
\end{table}
}
\clearpage
\newpage
 {
 \setlength\arrayrulewidth{1pt}
\begin{table}
\begin{center}
\begin{tabular}{ccc|l|ccc}
\cline{4-4} \\ [-1em]
\multicolumn{7}{c}{\foreignlanguage{greek}{ευαγγελιον κατα λουκαν} \textbf{(\nospace{8:21})} } \\ \\ [-1em] % Si on veut ajouter les bordures latérales, remplacer {7}{c} par {7}{|c|}
\cline{4-4} \\
\cline{4-4}
&  &  & &  &  & \\ [-0.9em]
&  & 14 & \foreignlanguage{greek}{ειϲιν οι τον λογον του \textoverline{θυ} ακουοντεϲ} & 20 &  &  \\
&  & 21 & \foreignlanguage{greek}{και ποιουντεϲ} & 22 &  &  \\
& \textbf{22} &  & \foreignlanguage{greek}{εγενετο δε εν μια των ημερων και} & 7 &  &  \\
&  & 8 & \foreignlanguage{greek}{αυτοϲ ενεβη ειϲ πλοιον και οι μαθη} & 14 &  &  \\
&  & 14 & \foreignlanguage{greek}{ται αυτου και ειπεν προϲ αυτουϲ} & 19 &  &  \\
&  & 20 & \foreignlanguage{greek}{διελθωμεν ειϲ το περαν τηϲ λιμνηϲ} & 25 &  &  \\
&  & 26 & \foreignlanguage{greek}{και ανηχθηϲαν} & 27 &  &  \\
& \textbf{23} &  & \foreignlanguage{greek}{πλεοντων δε αυτων αφυπνωϲεν} & 4 &  &  \\
&  & 5 & \foreignlanguage{greek}{και κατεβη λελαψ ανεμου ειϲ την λι} & 11 &  &  \\
&  & 11 & \foreignlanguage{greek}{μνην και ϲυνεπληρουντο και εκιν} & 15 &  &  \\
&  & 15 & \foreignlanguage{greek}{δυνευον προϲελθοντεϲ δε} & 2 & \textbf{24} &  \\
&  & 3 & \foreignlanguage{greek}{διηγειραν αυτον λεγοντεϲ επιϲτα} & 6 &  &  \\
&  & 6 & \foreignlanguage{greek}{τα απολλυμεθα ο δε εγερθειϲ ε} & 11 &  &  \\
&  & 11 & \foreignlanguage{greek}{πετιμηϲεν τω ανεμω και τω κλυ} & 16 &  &  \\
&  & 16 & \foreignlanguage{greek}{δωνι του υδατοϲ και επαυϲατο και} & 21 &  &  \\
&  & 22 & \foreignlanguage{greek}{εγενετο γαληνη ειπεν δε αυτοιϲ} & 3 & \textbf{25} &  \\
&  & 4 & \foreignlanguage{greek}{που η πιϲτιϲ υμων} & 7 &  &  \\
&  & 8 & \foreignlanguage{greek}{φοβηθεντεϲ δε εθαυμαϲαν λεγον} & 11 &  &  \\
&  & 11 & \foreignlanguage{greek}{τεϲ προϲ αλληλουϲ τιϲ αρα ουτοϲ ε} & 17 &  &  \\
&  & 17 & \foreignlanguage{greek}{ϲτιν οτι και τοιϲ ανεμοιϲ επιταϲϲει} & 22 &  &  \\
&  & 23 & \foreignlanguage{greek}{και τω υδατι και υπακουουϲιν αυτω} & 28 &  &  \\
& \textbf{26} &  & \foreignlanguage{greek}{και κατεπλευϲεν ειϲ την χωραν τω̅} & 6 &  &  \\
&  & 7 & \foreignlanguage{greek}{γαδαρηνων ητιϲ εϲτιν αντιπε} & 10 &  &  \\
&  & 10 & \foreignlanguage{greek}{ρα τηϲ γαλειλαιαϲ} & 12 &  &  \\
& \textbf{27} &  & \foreignlanguage{greek}{εξελθοντι δε αυτω επι την γην υ} & 7 &  &  \\
&  & 7 & \foreignlanguage{greek}{πηντηϲεν ανηρ τιϲ εκ τηϲ πολεωϲ} & 12 &  &  \\
&  & 13 & \foreignlanguage{greek}{οϲ ειχεν δαιμονια εκ χρονων ικανω̅} & 18 &  &  \\
&  & 19 & \foreignlanguage{greek}{και ιματιον ουκ ενεδιδυϲκετο και} & 23 &  &  \\
&  & 24 & \foreignlanguage{greek}{εν οικεια ουκ εμενεν αλλ εν τοιϲ} & 30 &  &  \\
&  & 31 & \foreignlanguage{greek}{μνημαϲιν} & 31 &  &  \\
[0.2em]
\cline{4-4}
\end{tabular}
\end{center}
\end{table}
}
\clearpage
\newpage
 {
 \setlength\arrayrulewidth{1pt}
\begin{table}
\begin{center}
\begin{tabular}{ccc|l|ccc}
\cline{4-4} \\ [-1em]
\multicolumn{7}{c}{\foreignlanguage{greek}{ευαγγελιον κατα λουκαν} \textbf{(\nospace{8:28})} } \\ \\ [-1em] % Si on veut ajouter les bordures latérales, remplacer {7}{c} par {7}{|c|}
\cline{4-4} \\
\cline{4-4}
&  &  & &  &  & \\ [-0.9em]
& \textbf{28} &  & \foreignlanguage{greek}{ιδων δε τον \textoverline{ιν} και ανακραξαϲ προϲεπε} & 7 &  &  \\
&  & 7 & \foreignlanguage{greek}{ϲεν αυτω και φωνη μεγαλη ειπεν} & 12 &  &  \\
&  & 13 & \foreignlanguage{greek}{αυτω τι εμοι και ϲοι \textoverline{ιυ} υιε του \textoverline{θυ}} & 21 &  &  \\
&  & 22 & \foreignlanguage{greek}{του υψιϲτου δεομαι ϲου μη με βαϲα} & 28 &  &  \\
&  & 28 & \foreignlanguage{greek}{νιϲηϲ παρηγγελλεν γαρ τω \textoverline{πνι}} & 4 & \textbf{29} &  \\
&  & 5 & \foreignlanguage{greek}{τω ακαθαρτω εξελθειν απο του \textoverline{ανου}} & 10 &  &  \\
&  & 11 & \foreignlanguage{greek}{πολλοιϲ γαρ χρονοιϲ ϲυνηρπακει} & 14 &  &  \\
&  & 15 & \foreignlanguage{greek}{αυτον και εδεϲμιτο αλυϲεϲιν και} & 19 &  &  \\
&  & 20 & \foreignlanguage{greek}{πεδεϲ φυλαϲϲομενοϲ και διαρρηϲ} & 23 &  &  \\
&  & 23 & \foreignlanguage{greek}{ϲων τα δεϲμα ηλαυνετο υπο του δαι} & 29 &  &  \\
&  & 29 & \foreignlanguage{greek}{μονοϲ ειϲ ταϲ ερημουϲ} & 32 &  &  \\
& \textbf{30} &  & \foreignlanguage{greek}{επηρωτηϲεν δε αυτον ο \textoverline{ιϲ} λεγων τι} & 7 &  &  \\
&  & 8 & \foreignlanguage{greek}{ϲοι εϲτιν ονομα ο δε ειπεν λεγεων} & 14 &  &  \\
&  & 15 & \foreignlanguage{greek}{οτι δαιμονια πολλα ειϲηλθεν ειϲ αυτο̅} & 20 &  &  \\
& \textbf{31} &  & \foreignlanguage{greek}{ινα μη επιταξη αυτοιϲ ειϲ την αβυϲ} & 7 &  &  \\
&  & 7 & \foreignlanguage{greek}{ϲον απελθειν ην δε αγελη χοιρων} & 4 & \textbf{32} &  \\
&  & 5 & \foreignlanguage{greek}{ικανων βοϲκομενων εν τω ορι του} & 10 &  &  \\
&  & 10 & \foreignlanguage{greek}{τω και παρεκαλουν αυτον ινα επι} & 15 &  &  \\
&  & 15 & \foreignlanguage{greek}{τρεψη αυτοιϲ ειϲ εκεινουϲ ειϲελθειν} & 19 &  &  \\
&  & 20 & \foreignlanguage{greek}{και επετρεψεν αυτοιϲ} & 22 &  &  \\
& \textbf{33} &  & \foreignlanguage{greek}{εξελθοντα δε τα δαιμονια απο του} & 6 &  &  \\
&  & 7 & \foreignlanguage{greek}{ανθρωπου ειϲηλθεν ειϲ τουϲ χοιρουϲ} & 11 &  &  \\
&  & 12 & \foreignlanguage{greek}{και ωρμηϲεν η αγελη κατα του κρη} & 18 &  &  \\
&  & 18 & \foreignlanguage{greek}{μνου ειϲ την λιμνην και απεπνιγη} & 23 &  &  \\
& \textbf{34} &  & \foreignlanguage{greek}{ιδοντεϲ δε οι βοϲκοντεϲ το γεγονωϲ} & 6 &  &  \\
&  & 7 & \foreignlanguage{greek}{εφυγαν και απηγγειλαν ειϲ την πολι̅} & 12 &  &  \\
&  & 13 & \foreignlanguage{greek}{και ειϲ τουϲ αγρουϲ εξηλθον δε} & 2 & \textbf{35} &  \\
&  & 3 & \foreignlanguage{greek}{ιδειν το γεγονοϲ και ηλθον προϲ τον \textoverline{ιν}} & 10 &  &  \\
&  & 11 & \foreignlanguage{greek}{και ευρον τον ανθρωπον καθημε} & 15 &  &  \\
&  & 15 & \foreignlanguage{greek}{νον αφ ου τα δαιμονια εξεληλυθει} & 20 &  &  \\
[0.2em]
\cline{4-4}
\end{tabular}
\end{center}
\end{table}
}
\clearpage
\newpage
 {
 \setlength\arrayrulewidth{1pt}
\begin{table}
\begin{center}
\begin{tabular}{ccc|l|ccc}
\cline{4-4} \\ [-1em]
\multicolumn{7}{c}{\foreignlanguage{greek}{ευαγγελιον κατα λουκαν} \textbf{(\nospace{8:35})} } \\ \\ [-1em] % Si on veut ajouter les bordures latérales, remplacer {7}{c} par {7}{|c|}
\cline{4-4} \\
\cline{4-4}
&  &  & &  &  & \\ [-0.9em]
&  & 21 & \foreignlanguage{greek}{ιματιϲμενον και ϲωφρονουντα πα} & 24 &  &  \\
&  & 24 & \foreignlanguage{greek}{ρα τουϲ ποδαϲ του \textoverline{ιυ} και εφοβηθηϲαν} & 30 &  &  \\
& \textbf{36} &  & \foreignlanguage{greek}{απηγγειλαν δε αυτοιϲ και οι ειδον} & 6 &  &  \\
&  & 6 & \foreignlanguage{greek}{τεϲ πωϲ εϲωθη ο δαιμονιϲθειϲ και η} & 2 & \textbf{37} &  \\
&  & 2 & \foreignlanguage{greek}{ρωτηϲαν αυτον παν το πληθοϲ τηϲ} & 7 &  &  \\
&  & 8 & \foreignlanguage{greek}{περιχωρου των γαδαρηνων απελθει̅} & 11 &  &  \\
&  & 12 & \foreignlanguage{greek}{απ αυτων οτι φοβω μεγαλω ϲυνειχοντο} & 17 &  &  \\
&  & 18 & \foreignlanguage{greek}{αυτοϲ δε ενβαϲ ειϲ το πλοιον υπεϲτρε} & 24 &  &  \\
&  & 24 & \foreignlanguage{greek}{ψεν εδιδαϲκεν δε αυτον ο \textoverline{ιϲ} λεγω̅} & 6 & \textbf{38} &  \\
& \textbf{39} &  & \foreignlanguage{greek}{υποϲτρεφε ειϲ τον οικον ϲου και διη} & 7 &  &  \\
&  & 7 & \foreignlanguage{greek}{γου οϲα ϲοι εποιηϲεν ο \textoverline{θϲ}} & 12 &  &  \\
&  & 13 & \foreignlanguage{greek}{και απηλθεν καθ ολην την πολιν κη} & 19 &  &  \\
&  & 19 & \foreignlanguage{greek}{ρυϲϲων οϲα εποιηϲεν αυτω ο \textoverline{ιϲ}} & 24 &  &  \\
& \textbf{40} &  & \foreignlanguage{greek}{εγενετο δε εν τω υποϲτρεψαι τον \textoverline{ιν}} & 7 &  &  \\
&  & 8 & \foreignlanguage{greek}{απεδεξατο αυτον ο οχλοϲ ηϲαν γαρ} & 13 &  &  \\
&  & 14 & \foreignlanguage{greek}{παντεϲ προϲδοκωντεϲ αυτον} & 16 &  &  \\
& \textbf{41} &  & \foreignlanguage{greek}{και ιδου ηλθεν ανηρ ω ονομα ιαειροϲ} & 7 &  &  \\
&  & 8 & \foreignlanguage{greek}{και αυτοϲ αρχων τηϲ ϲυναγωγηϲ υπηρχε̅} & 13 &  &  \\
&  & 14 & \foreignlanguage{greek}{και πεϲων παρα τουϲ ποδαϲ του \textoverline{ιυ} πα} & 21 &  &  \\
&  & 21 & \foreignlanguage{greek}{ρεκαλει αυτον ειϲελθειν ειϲ τον οι} & 26 &  &  \\
&  & 26 & \foreignlanguage{greek}{κον αυτου οτι θυγατηρ μονογενηϲ} & 3 & \textbf{42} &  \\
&  & 4 & \foreignlanguage{greek}{ην αυτω ωϲ ετων δωδεκα και αυ} & 10 &  &  \\
&  & 10 & \foreignlanguage{greek}{τη απεθνηϲκεν} & 11 &  &  \\
&  & 12 & \foreignlanguage{greek}{εν δε τω υπαγειν αυτον οι οχλοι ϲυν} & 19 &  &  \\
&  & 19 & \foreignlanguage{greek}{επνιγον αυτον και γυνη ουϲα ε̅} & 4 & \textbf{43} &  \\
&  & 5 & \foreignlanguage{greek}{ρυϲει αιματοϲ απο ετων δωδεκα η} & 10 &  &  \\
&  & 10 & \foreignlanguage{greek}{τιϲ ιατροιϲ προϲαναλωϲαϲα ολον το̅} & 14 &  &  \\
&  & 15 & \foreignlanguage{greek}{βιον ουκ ιϲχυϲεν υπ ουδενοϲ θερα} & 20 &  &  \\
&  & 20 & \foreignlanguage{greek}{πευθηναι προϲελθουϲα οπιϲθεν} & 2 & \textbf{44} &  \\
&  & 3 & \foreignlanguage{greek}{ηψατο του κραϲπεδου του ιματιου} & 7 &  &  \\
[0.2em]
\cline{4-4}
\end{tabular}
\end{center}
\end{table}
}
\clearpage
\newpage
 {
 \setlength\arrayrulewidth{1pt}
\begin{table}
\begin{center}
\begin{tabular}{ccc|l|ccc}
\cline{4-4} \\ [-1em]
\multicolumn{7}{c}{\foreignlanguage{greek}{ευαγγελιον κατα λουκαν} \textbf{(\nospace{8:44})} } \\ \\ [-1em] % Si on veut ajouter les bordures latérales, remplacer {7}{c} par {7}{|c|}
\cline{4-4} \\
\cline{4-4}
&  &  & &  &  & \\ [-0.9em]
&  & 8 & \foreignlanguage{greek}{αυτου και παραχρημα εϲτη η ρυϲιϲ} & 13 &  &  \\
&  & 14 & \foreignlanguage{greek}{του αιματοϲ αυτηϲ} & 16 &  &  \\
& \textbf{45} &  & \foreignlanguage{greek}{και ειπεν ο \textoverline{ιϲ} τιϲ ο αψαμενοϲ μου} & 8 &  &  \\
&  & 9 & \foreignlanguage{greek}{αρνουμενων δε παντων ειπεν ο} & 13 &  &  \\
&  & 14 & \foreignlanguage{greek}{πετροϲ και οι ϲυν αυτω επιϲτατα} & 19 &  &  \\
&  & 20 & \foreignlanguage{greek}{οι οχλοι ϲυνεχουϲιν ϲε και αποθλι} & 25 &  &  \\
&  & 25 & \foreignlanguage{greek}{βουϲιν και λεγειϲ τιϲ ο αψαμενοϲ μου} & 31 &  &  \\
& \textbf{46} &  & \foreignlanguage{greek}{ο δε \textoverline{ιϲ} ειπεν ηψατο μου τιϲ εγω γαρ} & 9 &  &  \\
&  & 10 & \foreignlanguage{greek}{εγνων δυναμιν εξελθουϲαν απ εμου} & 14 &  &  \\
& \textbf{47} &  & \foreignlanguage{greek}{ιδουϲα δε η γυνη οτι ουκ ελαθεν} & 7 &  &  \\
&  & 8 & \foreignlanguage{greek}{τρεμουϲα ηλθεν και προϲπεϲουϲα} & 11 &  &  \\
&  & 12 & \foreignlanguage{greek}{αυτω δι ην αιτιαν ηψατο αυτου} & 17 &  &  \\
&  & 18 & \foreignlanguage{greek}{απηγγειλεν εναντιον παντοϲ} & 20 &  &  \\
&  & 21 & \foreignlanguage{greek}{του λαου και πωϲ ειαθη παραχρημα} & 26 &  &  \\
& \textbf{48} &  & \foreignlanguage{greek}{ο δε ειπεν αυτη θαρϲει θυγατηρ η πι} & 8 &  &  \\
&  & 8 & \foreignlanguage{greek}{ϲτιϲ ϲου ϲεϲωκεν ϲε πορευου ειϲ} & 13 &  &  \\
&  & 14 & \foreignlanguage{greek}{ειρηνην} & 14 &  &  \\
& \textbf{49} &  & \foreignlanguage{greek}{ετι αυτου λαλουντοϲ ερχεται τιϲ} & 5 &  &  \\
&  & 6 & \foreignlanguage{greek}{παρα απο του αρχιϲυναγωγου λεγων αυ} & 11 &  &  \\
&  & 11 & \foreignlanguage{greek}{τω οτι τεθνηκεν η θυγατηρ ϲου} & 16 &  &  \\
&  & 17 & \foreignlanguage{greek}{μη ϲκυλλε τον διδαϲκαλον} & 20 &  &  \\
& \textbf{50} &  & \foreignlanguage{greek}{ο δε \textoverline{ιϲ} ακουϲαϲ απεκριθη αυτω λεγω̅} & 7 &  &  \\
&  & 8 & \foreignlanguage{greek}{μη φοβου μονον πιϲτευε και ϲωθη} & 13 &  &  \\
&  & 13 & \foreignlanguage{greek}{ϲεται ελθων δε ειϲ την οικια} & 5 & \textbf{51} &  \\
&  & 6 & \foreignlanguage{greek}{ουκ αφηκεν ειϲελθειν ουδενα ει} & 10 &  &  \\
&  & 11 & \foreignlanguage{greek}{μη πετρον και ιωαννην και ιακωβο̅} & 16 &  &  \\
&  & 17 & \foreignlanguage{greek}{και τον \textoverline{πρα} τηϲ παιδοϲ και την μη} & 24 &  &  \\
&  & 24 & \foreignlanguage{greek}{τερα εκλεον δε παντεϲ και ε} & 5 & \textbf{52} &  \\
&  & 5 & \foreignlanguage{greek}{κοπτοντο αυτην} & 6 &  &  \\
&  & 7 & \foreignlanguage{greek}{ο δε ειπεν μη κλαιεται ου γαρ απεθα} & 14 &  &  \\
[0.2em]
\cline{4-4}
\end{tabular}
\end{center}
\end{table}
}
\clearpage
\newpage
 {
 \setlength\arrayrulewidth{1pt}
\begin{table}
\begin{center}
\begin{tabular}{ccc|l|ccc}
\cline{4-4} \\ [-1em]
\multicolumn{7}{c}{\foreignlanguage{greek}{ευαγγελιον κατα λουκαν} \textbf{(\nospace{8:52})} } \\ \\ [-1em] % Si on veut ajouter les bordures latérales, remplacer {7}{c} par {7}{|c|}
\cline{4-4} \\
\cline{4-4}
&  &  & &  &  & \\ [-0.9em]
&  & 14 & \foreignlanguage{greek}{νεν αλλα καθευδει και κατεγελω̅} & 2 & \textbf{53} &  \\
&  & 3 & \foreignlanguage{greek}{αυτου ειδοτεϲ οτι απεθανεν} & 6 &  &  \\
& \textbf{54} &  & \foreignlanguage{greek}{αυτοϲ δε εκβαλων πανταϲ εξω και} & 6 &  &  \\
&  & 7 & \foreignlanguage{greek}{κρατηϲαϲ τηϲ χειροϲ αυτηϲ εφωνηϲε̅} & 11 &  &  \\
&  & 12 & \foreignlanguage{greek}{λεγων η παιϲ εγειρου και επεϲτρε} & 2 & \textbf{55} &  \\
&  & 2 & \foreignlanguage{greek}{ψεν το \textoverline{πνα} αυτηϲ και ανεϲτη παρα} & 8 &  &  \\
&  & 8 & \foreignlanguage{greek}{χρημα και διεταξεν δοθηναι αυ} & 12 &  &  \\
&  & 12 & \foreignlanguage{greek}{τη φαγειν και εξεϲτηϲαν οι γονειϲ} & 4 & \textbf{56} &  \\
&  & 5 & \foreignlanguage{greek}{αυτηϲ ο δε παρηγγειλεν αυτοιϲ} & 9 &  &  \\
&  & 10 & \foreignlanguage{greek}{μηδενει ειπειν το γεγονοϲ} & 13 &  &  \\
& \mygospelchapter &  & \foreignlanguage{greek}{ϲυνκαλεϲαμενοϲ δε τουϲ δωδεκα ε} & 5 &  &  \\
&  & 5 & \foreignlanguage{greek}{δωκεν αυτοιϲ δυναμιν και εξουϲια̅} & 9 &  &  \\
&  & 10 & \foreignlanguage{greek}{επι παντα τα δαιμονια και νοϲουϲ} & 15 &  &  \\
&  & 16 & \foreignlanguage{greek}{θεραπευειν και απεϲτιλεν αυτουϲ} & 3 & \textbf{2} &  \\
&  & 4 & \foreignlanguage{greek}{κηρυϲϲιν την βαϲιλειαν του \textoverline{θυ} και ει} & 10 &  &  \\
&  & 10 & \foreignlanguage{greek}{αϲαϲθαι τουϲ αϲθενουνταϲ} & 12 &  &  \\
& \textbf{3} &  & \foreignlanguage{greek}{και ειπεν προϲ αυτουϲ μηδεν ερεται} & 6 &  &  \\
&  & 7 & \foreignlanguage{greek}{ειϲ την οδον μητε ραβδον μητε} & 12 &  &  \\
&  & 13 & \foreignlanguage{greek}{πηραν μητε αρτον μητε αργυριον} & 17 &  &  \\
&  & 18 & \foreignlanguage{greek}{μητε ανα δυο χειθωναϲ εχειν και} & 1 & \textbf{4} &  \\
&  & 2 & \foreignlanguage{greek}{ειϲ ην αν οικειαν ειϲελθηται εκει με} & 8 &  &  \\
&  & 8 & \foreignlanguage{greek}{νεται και εκειθεν εξερχεϲθαι} & 11 &  &  \\
& \textbf{5} &  & \foreignlanguage{greek}{και οϲοι αν μη δεχωνται υμαϲ εξερ} & 7 &  &  \\
&  & 7 & \foreignlanguage{greek}{χομενοι απο τηϲ πολεωϲ εκεινηϲ} & 11 &  &  \\
&  & 12 & \foreignlanguage{greek}{τον κονιορτον απο των ποδων υμω̅} & 17 &  &  \\
&  & 18 & \foreignlanguage{greek}{αποτιναξατε ειϲ μαρτυριον επ αυτουϲ} & 22 &  &  \\
& \textbf{6} &  & \foreignlanguage{greek}{εξερχομενοι δε διηρχοντο κατα} & 4 &  &  \\
&  & 5 & \foreignlanguage{greek}{ταϲ κωμαϲ ευαγγελιζομενοι και} & 8 &  &  \\
&  & 9 & \foreignlanguage{greek}{θεραπευοντεϲ πανταχου} & 10 &  &  \\
& \textbf{7} &  & \foreignlanguage{greek}{ηκουϲεν δε ηρωδηϲ ο τετραρχηϲ} & 5 &  &  \\
[0.2em]
\cline{4-4}
\end{tabular}
\end{center}
\end{table}
}
\clearpage
\newpage
 {
 \setlength\arrayrulewidth{1pt}
\begin{table}
\begin{center}
\begin{tabular}{ccc|l|ccc}
\cline{4-4} \\ [-1em]
\multicolumn{7}{c}{\foreignlanguage{greek}{ευαγγελιον κατα λουκαν} \textbf{(\nospace{9:7})} } \\ \\ [-1em] % Si on veut ajouter les bordures latérales, remplacer {7}{c} par {7}{|c|}
\cline{4-4} \\
\cline{4-4}
&  &  & &  &  & \\ [-0.9em]
&  & 6 & \foreignlanguage{greek}{τα γεινομενα υπ αυτου παντα και διη} & 12 &  &  \\
&  & 12 & \foreignlanguage{greek}{πορει δια το λεγεϲθαι υπο τινων οτι} & 18 &  &  \\
&  & 19 & \foreignlanguage{greek}{ιωαννηϲ εγηγερται εκ νεκρων} & 22 &  &  \\
& \textbf{8} &  & \foreignlanguage{greek}{υπο τινων δε λεγοντων οτι ηλιαϲ} & 6 &  &  \\
&  & 7 & \foreignlanguage{greek}{εφανη αλλων δε οτι προφητηϲ ειϲ} & 12 &  &  \\
&  & 13 & \foreignlanguage{greek}{των αρχαιων ανεϲτη} & 15 &  &  \\
& \textbf{9} &  & \foreignlanguage{greek}{και ειπεν ηρωδηϲ ιωαννην εγω απε} & 6 &  &  \\
&  & 6 & \foreignlanguage{greek}{κεφαλιϲα τιϲ δε εϲτιν ουτοϲ περι ου} & 13 &  &  \\
&  & 14 & \foreignlanguage{greek}{εγω ακουω τοιαυτα και εζητει ιδει̅} & 19 &  &  \\
&  & 20 & \foreignlanguage{greek}{αυτον και υποϲτρεψαντεϲ οι} & 3 & \textbf{10} &  \\
&  & 4 & \foreignlanguage{greek}{αποϲτολοι διηγηϲαντο αυτω οϲα ε} & 8 &  &  \\
&  & 8 & \foreignlanguage{greek}{ποιηϲαν} & 8 &  &  \\
&  & 9 & \foreignlanguage{greek}{και παραλαβων αυτουϲ υπεχωρηϲεν} & 12 &  &  \\
&  & 13 & \foreignlanguage{greek}{κατ ιδιαν ειϲ τοπον ερημον πολε} & 18 &  &  \\
&  & 18 & \foreignlanguage{greek}{ωϲ καλουμενηϲ βηθϲαιδαν} & 20 &  &  \\
& \textbf{11} &  & \foreignlanguage{greek}{οι δε οχλοι γνοντεϲ ηκολουθηϲαν} & 5 &  &  \\
&  & 6 & \foreignlanguage{greek}{αυτω και δεξομενοϲ αυτουϲ ελα} & 10 &  &  \\
&  & 10 & \foreignlanguage{greek}{λει αυτοιϲ περι τηϲ βαϲιλειαϲ του \textoverline{θυ}} & 16 &  &  \\
&  & 17 & \foreignlanguage{greek}{και τουϲ χρειαν εχονταϲ θεραπειαϲ} & 21 &  &  \\
&  & 22 & \foreignlanguage{greek}{ειατο η δε ημερα ηρξατο κλεινειν προϲ} & 6 & \textbf{12} &  \\
&  & 6 & \foreignlanguage{greek}{ελθοντεϲ οι δωδεκα ειπον αυτω} & 10 &  &  \\
&  & 11 & \foreignlanguage{greek}{απολυϲον τον οχλον ινα απελθον} & 15 &  &  \\
&  & 15 & \foreignlanguage{greek}{τεϲ ειϲ ταϲ κυκλω κωμαϲ και τουϲ} & 21 &  &  \\
&  & 22 & \foreignlanguage{greek}{αγρουϲ καταλυϲωϲιν και ευρωϲιν} & 25 &  &  \\
&  & 26 & \foreignlanguage{greek}{επιϲιτιϲμον οτι ωδε εν ερημω το} & 31 &  &  \\
&  & 31 & \foreignlanguage{greek}{πω εϲμεν} & 32 &  &  \\
& \textbf{13} &  & \foreignlanguage{greek}{ειπεν δε προϲ αυτουϲ δοτε αυτοιϲ υ} & 7 &  &  \\
&  & 7 & \foreignlanguage{greek}{μειϲ φαγειν οι δε ειπον ουκ ειϲιν} & 13 &  &  \\
&  & 14 & \foreignlanguage{greek}{ημιν πλειον η πεντε αρτων και ι} & 20 &  &  \\
[0.2em]
\cline{4-4}
\end{tabular}
\end{center}
\end{table}
}
\clearpage
\newpage
 {
 \setlength\arrayrulewidth{1pt}
\begin{table}
\begin{center}
\begin{tabular}{ccc|l|ccc}
\cline{4-4} \\ [-1em]
\multicolumn{7}{c}{\foreignlanguage{greek}{ευαγγελιον κατα λουκαν} \textbf{(\nospace{9:13})} } \\ \\ [-1em] % Si on veut ajouter les bordures latérales, remplacer {7}{c} par {7}{|c|}
\cline{4-4} \\
\cline{4-4}
&  &  & &  &  & \\ [-0.9em]
&  & 20 & \foreignlanguage{greek}{χθυεϲ δυο ει μητι πορευθεντεϲ ημειϲ} & 25 &  &  \\
&  & 26 & \foreignlanguage{greek}{αγοραϲωμεν ειϲ παντα τον λαον του} & 31 &  &  \\
&  & 31 & \foreignlanguage{greek}{τον βρωματα ηϲαν γαρ ωϲει ανδρεϲ} & 4 & \textbf{14} &  \\
&  & 5 & \foreignlanguage{greek}{πεντακειϲχειλιοι} & 5 &  &  \\
&  & 6 & \foreignlanguage{greek}{ειπεν δε προϲ τουϲ μαθηταϲ αυτου} & 11 &  &  \\
&  & 12 & \foreignlanguage{greek}{κατακλεινατε αυτουϲ κλιϲιαϲ ανα πε̅} & 16 &  &  \\
&  & 16 & \foreignlanguage{greek}{τηκοντα και εποιηϲαν ουτωϲ και} & 4 & \textbf{15} &  \\
&  & 5 & \foreignlanguage{greek}{ανεκλειναν απανταϲ} & 6 &  &  \\
& \textbf{16} &  & \foreignlanguage{greek}{λαβων δε τουϲ πεντε αρτουϲ και τουϲ} & 7 &  &  \\
&  & 8 & \foreignlanguage{greek}{δυο ιχθυαϲ αναβλεψαϲ ειϲ τον ουρα} & 13 &  &  \\
&  & 13 & \foreignlanguage{greek}{νον ηυλογηϲεν αυτουϲ και κατεκλα} & 17 &  &  \\
&  & 17 & \foreignlanguage{greek}{ϲεν και εδιδου τοιϲ μαθηταιϲ παρατι} & 22 &  &  \\
&  & 22 & \foreignlanguage{greek}{θεναι τω οχλω και εφαγον και ε} & 4 & \textbf{17} &  \\
&  & 4 & \foreignlanguage{greek}{χορταϲθηϲαν παντεϲ και ηρθη το} & 8 &  &  \\
&  & 9 & \foreignlanguage{greek}{περιϲϲευμα αυτων των κλαϲματω̅} & 12 &  &  \\
&  & 13 & \foreignlanguage{greek}{κοφινουϲ δωδεκα} & 14 &  &  \\
& \textbf{18} &  & \foreignlanguage{greek}{και εγενετο εν τω ειναι αυτον προϲ} & 7 &  &  \\
&  & 7 & \foreignlanguage{greek}{ευχομενον κατα μοναϲ ϲυνηϲαν αυ} & 11 &  &  \\
&  & 11 & \foreignlanguage{greek}{τω οι μαθηται αυτου και επηρωτη} & 16 &  &  \\
&  & 16 & \foreignlanguage{greek}{ϲεν αυτουϲ λεγων τινα με λεγουϲι̅} & 21 &  &  \\
&  & 22 & \foreignlanguage{greek}{οι οχλοι ειναι οι δε αποκριθεντεϲ} & 3 & \textbf{19} &  \\
&  & 4 & \foreignlanguage{greek}{ειπον ιωαννην τον βαπτιϲτην} & 7 &  &  \\
&  & 8 & \foreignlanguage{greek}{αλλοι δε ηλιαν αλλοι δε οτι προφη} & 14 &  &  \\
&  & 14 & \foreignlanguage{greek}{τηϲ τιϲ των αρχεων ανεϲτη} & 18 &  &  \\
& \textbf{20} &  & \foreignlanguage{greek}{ειπεν δε αυτοιϲ υμειϲ δε τινα με} & 7 &  &  \\
&  & 8 & \foreignlanguage{greek}{λεγεται ειναι αποκριθειϲ δε πε} & 12 &  &  \\
&  & 12 & \foreignlanguage{greek}{τροϲ ειπεν τον \textoverline{χρν} του \textoverline{θυ}} & 17 &  &  \\
& \textbf{21} &  & \foreignlanguage{greek}{ο δε επιτιμηϲαϲ αυτοιϲ παρηγγειλε̅} & 5 &  &  \\
&  & 6 & \foreignlanguage{greek}{μηδενι λεγειν τουτο ειπων οτι δει} & 3 & \textbf{22} &  \\
&  & 4 & \foreignlanguage{greek}{τον υιον του ανθρωπου πολλα παθει̅} & 9 &  &  \\
[0.2em]
\cline{4-4}
\end{tabular}
\end{center}
\end{table}
}
\clearpage
\newpage
 {
 \setlength\arrayrulewidth{1pt}
\begin{table}
\begin{center}
\begin{tabular}{ccc|l|ccc}
\cline{4-4} \\ [-1em]
\multicolumn{7}{c}{\foreignlanguage{greek}{ευαγγελιον κατα λουκαν} \textbf{(\nospace{9:22})} } \\ \\ [-1em] % Si on veut ajouter les bordures latérales, remplacer {7}{c} par {7}{|c|}
\cline{4-4} \\
\cline{4-4}
&  &  & &  &  & \\ [-0.9em]
&  & 10 & \foreignlanguage{greek}{και αποδοκιμαϲθηναι απο των πρεϲ} & 14 &  &  \\
&  & 14 & \foreignlanguage{greek}{βυτερων και αρχιερεων και γραμμα} & 18 &  &  \\
&  & 18 & \foreignlanguage{greek}{τεων και αποκτανθηναι και τη τρι} & 23 &  &  \\
&  & 23 & \foreignlanguage{greek}{τη ημερα εγερθηναι} & 25 &  &  \\
& \textbf{23} &  & \foreignlanguage{greek}{ελεγεν δε προϲ πανταϲ ει τιϲ θελει} & 7 &  &  \\
&  & 8 & \foreignlanguage{greek}{οπιϲω μου ερχεϲθαι απαρνηϲαϲθω} & 11 &  &  \\
&  & 12 & \foreignlanguage{greek}{εαυτον και αρατω τον ϲταυρον αυ} & 17 &  &  \\
&  & 17 & \foreignlanguage{greek}{του καθ ημεραν και ακολουθειτω μοι} & 22 &  &  \\
& \textbf{24} &  & \foreignlanguage{greek}{οϲ γαρ αν θελη την ψυχην ϲωϲαι απο} & 8 &  &  \\
&  & 8 & \foreignlanguage{greek}{λεϲει αυτην οϲ δ αν απολεϲει την} & 14 &  &  \\
&  & 15 & \foreignlanguage{greek}{ψυχην αυτου ενεκεν εμου ουτοϲ} & 19 &  &  \\
&  & 20 & \foreignlanguage{greek}{ϲωϲει αυτην τι γαρ ωφελειται αν} & 4 & \textbf{25} &  \\
&  & 4 & \foreignlanguage{greek}{θρωποϲ κερδηϲαϲ τον κοϲμον ολον} & 8 &  &  \\
&  & 9 & \foreignlanguage{greek}{εαυτον δε απολεϲαϲ η ζημιωθειϲ} & 13 &  &  \\
& \textbf{26} &  & \foreignlanguage{greek}{οϲ γαρ αν επεϲχυνθη με και τουϲ ε} & 8 &  &  \\
&  & 8 & \foreignlanguage{greek}{μουϲ λογουϲ τουτον ο υιοϲ του αν} & 14 &  &  \\
&  & 14 & \foreignlanguage{greek}{θρωπου επεϲχυνθηϲεται οταν ελ} & 17 &  &  \\
&  & 17 & \foreignlanguage{greek}{θη εν τη δοξη αυτου και του \textoverline{πρϲ}} & 24 &  &  \\
&  & 25 & \foreignlanguage{greek}{και των αγιων αγγελων} & 28 &  &  \\
& \textbf{27} &  & \foreignlanguage{greek}{λεγω δε υμιν αληθωϲ ειϲιν τινεϲ} & 6 &  &  \\
&  & 7 & \foreignlanguage{greek}{των ωδε εϲτωτων οι ου μη γευϲω̅} & 13 &  &  \\
&  & 13 & \foreignlanguage{greek}{ται θανατου εωϲ αν ιδωϲιν την βαϲι} & 19 &  &  \\
&  & 19 & \foreignlanguage{greek}{λειαν του \textoverline{θυ}} & 21 &  &  \\
& \textbf{28} &  & \foreignlanguage{greek}{εγενετο δε μετα τουϲ λογουϲ τουτουϲ} & 6 &  &  \\
&  & 7 & \foreignlanguage{greek}{ωϲει ημεραι οκτω και παραλαβων} & 11 &  &  \\
&  & 12 & \foreignlanguage{greek}{πετρον και ιωαννην και ιακωβον} & 16 &  &  \\
&  & 17 & \foreignlanguage{greek}{ανεβη ειϲ το οροϲ προϲευξαϲθαι} & 21 &  &  \\
& \textbf{29} &  & \foreignlanguage{greek}{και εγενετο εν τω προϲευχεϲθαι} & 5 &  &  \\
&  & 6 & \foreignlanguage{greek}{αυτον το ειδοϲ του προϲωπου αυ} & 11 &  &  \\
&  & 11 & \foreignlanguage{greek}{του ετερον και ο ιματιϲμοϲ αυτου} & 16 &  &  \\
[0.2em]
\cline{4-4}
\end{tabular}
\end{center}
\end{table}
}
\clearpage
\newpage
 {
 \setlength\arrayrulewidth{1pt}
\begin{table}
\begin{center}
\begin{tabular}{ccc|l|ccc}
\cline{4-4} \\ [-1em]
\multicolumn{7}{c}{\foreignlanguage{greek}{ευαγγελιον κατα λουκαν} \textbf{(\nospace{9:29})} } \\ \\ [-1em] % Si on veut ajouter les bordures latérales, remplacer {7}{c} par {7}{|c|}
\cline{4-4} \\
\cline{4-4}
&  &  & &  &  & \\ [-0.9em]
&  & 17 & \foreignlanguage{greek}{λευκοϲ εξαϲτραπτων και ιδου αν} & 3 & \textbf{30} &  \\
&  & 3 & \foreignlanguage{greek}{δρεϲ δυο ϲυνελαλουν αυτω οιτινεϲ} & 7 &  &  \\
&  & 8 & \foreignlanguage{greek}{ηϲαν μωυϲηϲ και ηλιαϲ οι οφθεντεϲ} & 2 & \textbf{31} &  \\
&  & 3 & \foreignlanguage{greek}{εν τη δοξη ελεγον την εξοδον αυτου} & 9 &  &  \\
&  & 10 & \foreignlanguage{greek}{ην ημελλεν πληρουν εν ιερουϲαλημ} & 14 &  &  \\
& \textbf{32} &  & \foreignlanguage{greek}{ο δε πετροϲ και οι ϲυν αυτω ηϲαν βεβα} & 9 &  &  \\
&  & 9 & \foreignlanguage{greek}{ρημενοι υπνω διαγρηγορηϲαντεϲ} & 11 &  &  \\
&  & 12 & \foreignlanguage{greek}{δε ειδον την δοξαν αυτου και τουϲ} & 18 &  &  \\
&  & 19 & \foreignlanguage{greek}{δυο ανδραϲ τουϲ ϲυνεϲτωταϲ αυτω} & 23 &  &  \\
& \textbf{33} &  & \foreignlanguage{greek}{και εγενετο εν τω διαχωριζεϲθαι} & 5 &  &  \\
&  & 6 & \foreignlanguage{greek}{αυτουϲ απ αυτου ειπεν πετροϲ προϲ} & 11 &  &  \\
&  & 12 & \foreignlanguage{greek}{τον \textoverline{ιν} επιϲτατα καλον εϲτιν ημαϲ} & 17 &  &  \\
&  & 18 & \foreignlanguage{greek}{ωδε ειναι και ποιηϲωμεν ϲκηναϲ} & 22 &  &  \\
&  & 23 & \foreignlanguage{greek}{τριϲ μιαν ϲοι και μιαν μωυϲει και} & 29 &  &  \\
&  & 30 & \foreignlanguage{greek}{μιαν ηλεια μη ειδωϲ ο λεγει} & 35 &  &  \\
& \textbf{34} &  & \foreignlanguage{greek}{ταυτα δε αυτου λεγοντοϲ εγενετο} & 5 &  &  \\
&  & 6 & \foreignlanguage{greek}{λεφελη και επεϲκιαϲεν αυτουϲ} & 9 &  &  \\
&  & 10 & \foreignlanguage{greek}{εφοβηθηϲαν δε εν τω εκεινουϲ ειϲ} & 15 &  &  \\
&  & 15 & \foreignlanguage{greek}{ελθειν ειϲ την νεφελην} & 18 &  &  \\
& \textbf{35} &  & \foreignlanguage{greek}{και φωνη εγενετο εκ τηϲ νεφεληϲ} & 6 &  &  \\
&  & 7 & \foreignlanguage{greek}{λεγουϲα ουτοϲ εϲτιν ο υιοϲ μου} & 12 &  &  \\
&  & 13 & \foreignlanguage{greek}{ο αγαπητοϲ αυτου ακουεται} & 16 &  &  \\
& \textbf{36} &  & \foreignlanguage{greek}{και εν τω γενεϲθαι την φωνην ευ} & 7 &  &  \\
&  & 7 & \foreignlanguage{greek}{ρεθη ο \textoverline{ιϲ} μονοϲ και αυτοι εϲειγηϲα̅} & 13 &  &  \\
&  & 14 & \foreignlanguage{greek}{και ουδενι απηγγειλον εν εκειναιϲ} & 18 &  &  \\
&  & 19 & \foreignlanguage{greek}{ταιϲ ημεραιϲ ουδεν ων εορακαϲιν} & 23 &  &  \\
& \textbf{37} &  & \foreignlanguage{greek}{εγενετο δε τη εξηϲ ημερα κατελθο̅} & 6 &  &  \\
&  & 6 & \foreignlanguage{greek}{των αυτων απο του ορουϲ ϲυνηντη} & 11 &  &  \\
&  & 11 & \foreignlanguage{greek}{ϲεν αυτω οχλοϲ πολυϲ και ιδου} & 2 & \textbf{38} &  \\
&  & 3 & \foreignlanguage{greek}{ανηρ απο του οχλου ανεβοηϲεν λεγω̅} & 8 &  &  \\
[0.2em]
\cline{4-4}
\end{tabular}
\end{center}
\end{table}
}
\clearpage
\newpage
 {
 \setlength\arrayrulewidth{1pt}
\begin{table}
\begin{center}
\begin{tabular}{ccc|l|ccc}
\cline{4-4} \\ [-1em]
\multicolumn{7}{c}{\foreignlanguage{greek}{ευαγγελιον κατα λουκαν} \textbf{(\nospace{9:38})} } \\ \\ [-1em] % Si on veut ajouter les bordures latérales, remplacer {7}{c} par {7}{|c|}
\cline{4-4} \\
\cline{4-4}
&  &  & &  &  & \\ [-0.9em]
&  & 9 & \foreignlanguage{greek}{διδαϲκαλε δεομαι ϲου επιβλεψον ε} & 14 &  &  \\
&  & 14 & \foreignlanguage{greek}{πι τον υιον μου οτι μονογενηϲ ε} & 20 &  &  \\
&  & 20 & \foreignlanguage{greek}{ϲτιν μοι και ιδου \textoverline{πνα} λαμβανει αυτο̅} & 5 & \textbf{39} &  \\
&  & 6 & \foreignlanguage{greek}{και εξεφνηϲ κραζει και ϲπαραϲϲει αυ} & 11 &  &  \\
&  & 11 & \foreignlanguage{greek}{τον μετα αφρου και μολιϲ αποχω} & 16 &  &  \\
&  & 16 & \foreignlanguage{greek}{ρει απ αυτου ϲυντριβον αυτον} & 20 &  &  \\
& \textbf{40} &  & \foreignlanguage{greek}{και εδεηθην των μαθητων ϲου ινα} & 6 &  &  \\
&  & 7 & \foreignlanguage{greek}{εκβαλωϲιν αυτο και ουκ ηδυνηθηϲαν} & 11 &  &  \\
& \textbf{41} &  & \foreignlanguage{greek}{αποκριθειϲ δε ο \textoverline{ιϲ} ειπεν ω γενεα απι} & 8 &  &  \\
&  & 8 & \foreignlanguage{greek}{ϲτοϲ και διεϲτραμμενη εωϲ ποτε} & 12 &  &  \\
&  & 13 & \foreignlanguage{greek}{εϲομαι προϲ υμαϲ και ανεξομαι υμω̅} & 18 &  &  \\
&  & 19 & \foreignlanguage{greek}{προϲαγαγε τον υιον ϲου ωδε} & 23 &  &  \\
& \textbf{42} &  & \foreignlanguage{greek}{ετι δε προϲερχομενου αυτου ερη} & 5 &  &  \\
&  & 5 & \foreignlanguage{greek}{ξεν αυτον το δαιμονιον και ϲυνε} & 10 &  &  \\
&  & 10 & \foreignlanguage{greek}{ϲπαραξεν επετιμηϲεν δε ο \textoverline{ιϲ} τω} & 15 &  &  \\
&  & 16 & \foreignlanguage{greek}{\textoverline{πνι} τω ακαθαρτω και ιαϲατο τον} & 21 &  &  \\
&  & 22 & \foreignlanguage{greek}{παιδα και απεδωκεν αυτον τω πα} & 27 &  &  \\
&  & 27 & \foreignlanguage{greek}{τρι αυτου εξεπληϲϲοντο δε πα̅} & 3 & \textbf{43} &  \\
&  & 3 & \foreignlanguage{greek}{τεϲ επι τη μεγαλιοτητι του \textoverline{θυ}} & 8 &  &  \\
&  & 9 & \foreignlanguage{greek}{παντων δε θαυμαζοντων επι παϲι̅} & 13 &  &  \\
&  & 14 & \foreignlanguage{greek}{οιϲ εποιηϲεν ο \textoverline{ιϲ} ειπεν προϲ τουϲ μα} & 21 &  &  \\
&  & 21 & \foreignlanguage{greek}{θηταϲ αυτου θεϲθαι υμειϲ ειϲ τα} & 4 & \textbf{44} &  \\
&  & 5 & \foreignlanguage{greek}{ωτα υμων τουϲ λογουϲ τουτουϲ} & 9 &  &  \\
&  & 10 & \foreignlanguage{greek}{ο γαρ υιοϲ του ανθρωπου μελλει πα} & 16 &  &  \\
&  & 16 & \foreignlanguage{greek}{ραδιδοϲθαι ειϲ χειραϲ ανθρωπων} & 19 &  &  \\
& \textbf{45} &  & \foreignlanguage{greek}{οι δε ηγνοουν το ρημα τουτο και η̅} & 8 &  &  \\
&  & 9 & \foreignlanguage{greek}{παρακεκαλυμμενον απ αυτων ι} & 12 &  &  \\
&  & 12 & \foreignlanguage{greek}{να μη αιϲθωνται αυτο} & 15 &  &  \\
&  & 16 & \foreignlanguage{greek}{και εφοβουντο ερωτηϲαι αυτον πε} & 20 &  &  \\
&  & 20 & \foreignlanguage{greek}{ρι του ρηματοϲ τουτου ειϲηλθεν} & 1 & \textbf{46} &  \\
[0.2em]
\cline{4-4}
\end{tabular}
\end{center}
\end{table}
}
\clearpage
\newpage
 {
 \setlength\arrayrulewidth{1pt}
\begin{table}
\begin{center}
\begin{tabular}{ccc|l|ccc}
\cline{4-4} \\ [-1em]
\multicolumn{7}{c}{\foreignlanguage{greek}{ευαγγελιον κατα λουκαν} \textbf{(\nospace{9:46})} } \\ \\ [-1em] % Si on veut ajouter les bordures latérales, remplacer {7}{c} par {7}{|c|}
\cline{4-4} \\
\cline{4-4}
&  &  & &  &  & \\ [-0.9em]
&  & 2 & \foreignlanguage{greek}{δε διαλογιϲμοϲ αυτοιϲ το τιϲ αν ειη} & 8 &  &  \\
&  & 9 & \foreignlanguage{greek}{μειζων αυτων} & 10 &  &  \\
& \textbf{47} &  & \foreignlanguage{greek}{ο δε \textoverline{ιϲ} ιδων τον διαλογιϲμον τηϲ καρ} & 8 &  &  \\
&  & 8 & \foreignlanguage{greek}{διαϲ αυτων επιλαβομενοϲ παιδιου} & 11 &  &  \\
&  & 12 & \foreignlanguage{greek}{εϲτηϲεν αυτο παρ εαυτω και ειπεν} & 2 & \textbf{48} &  \\
&  & 3 & \foreignlanguage{greek}{αυτοιϲ οϲ εαν δεξηται τουτο το} & 8 &  &  \\
&  & 9 & \foreignlanguage{greek}{παιδιον επι τω ονοματι μου εμε δεχε} & 15 &  &  \\
&  & 15 & \foreignlanguage{greek}{ται και οϲ εαν εμε δεξηται δεχεται} & 21 &  &  \\
&  & 22 & \foreignlanguage{greek}{τον αποϲτιλαντα με ο γαρ μεικροτε} & 27 &  &  \\
&  & 27 & \foreignlanguage{greek}{ροϲ εν παϲιν υμιν υπαρχων ουτοϲ} & 32 &  &  \\
&  & 33 & \foreignlanguage{greek}{εϲται μεγαϲ} & 34 &  &  \\
& \textbf{49} &  & \foreignlanguage{greek}{αποκριθειϲ δε ιωαννηϲ ειπεν επιϲτα} & 5 &  &  \\
&  & 5 & \foreignlanguage{greek}{τα ιδομεν τινα επι τω ονοματι ϲου} & 11 &  &  \\
&  & 12 & \foreignlanguage{greek}{εκβαλλοντα δαιμονια και εκωλυϲα} & 15 &  &  \\
&  & 15 & \foreignlanguage{greek}{μεν αυτον οτι ουκ ακολουθει μεθ ημω̅} & 21 &  &  \\
& \textbf{50} &  & \foreignlanguage{greek}{και ειπεν προϲ αυτον ο \textoverline{ιϲ} μη κωλυεται} & 8 &  &  \\
&  & 9 & \foreignlanguage{greek}{οϲ γαρ ουκ εϲτιν καθ υμων υπερ υμω̅} & 16 &  &  \\
&  & 17 & \foreignlanguage{greek}{εϲτιν} & 17 &  &  \\
& \textbf{51} &  & \foreignlanguage{greek}{εγενετο δε εν τω ϲυνπληρουϲθαι ταϲ} & 6 &  &  \\
&  & 7 & \foreignlanguage{greek}{ημεραϲ τηϲ αναλημψεωϲ αυτου και} & 11 &  &  \\
&  & 12 & \foreignlanguage{greek}{αυτοϲ το προϲωπον εϲτηριξεν αυτου} & 16 &  &  \\
&  & 17 & \foreignlanguage{greek}{του πορευεϲθαι ειϲ ιερουϲαλημ} & 20 &  &  \\
& \textbf{52} &  & \foreignlanguage{greek}{και απεϲτιλεν τουϲ αγγελουϲ προ προ} & 6 &  &  \\
&  & 6 & \foreignlanguage{greek}{ϲωπου εαυτου και πορευθεντεϲ ειϲ} & 11 &  &  \\
&  & 11 & \foreignlanguage{greek}{ηλθον ειϲ κωμην ϲαμαριτων ωϲτε} & 15 &  &  \\
&  & 16 & \foreignlanguage{greek}{ετοιμαϲαι αυτω και ουκ εξεδεξα̅} & 3 & \textbf{53} &  \\
&  & 3 & \foreignlanguage{greek}{το αυτον οτι το προϲωπον αυτου ην} & 9 &  &  \\
&  & 10 & \foreignlanguage{greek}{πορευομενον ειϲ ιερουϲαλημ} & 12 &  &  \\
& \textbf{54} &  & \foreignlanguage{greek}{ιδοντεϲ δε οι μαθηται αυτου ιακωβοϲ} & 6 &  &  \\
&  & 7 & \foreignlanguage{greek}{και ιωαννηϲ ειπον \textoverline{κε} θελειϲ ειπωμε̅} & 12 &  &  \\
[0.2em]
\cline{4-4}
\end{tabular}
\end{center}
\end{table}
}
\clearpage
\newpage
 {
 \setlength\arrayrulewidth{1pt}
\begin{table}
\begin{center}
\begin{tabular}{ccc|l|ccc}
\cline{4-4} \\ [-1em]
\multicolumn{7}{c}{\foreignlanguage{greek}{ευαγγελιον κατα λουκαν} \textbf{(\nospace{9:54})} } \\ \\ [-1em] % Si on veut ajouter les bordures latérales, remplacer {7}{c} par {7}{|c|}
\cline{4-4} \\
\cline{4-4}
&  &  & &  &  & \\ [-0.9em]
&  & 13 & \foreignlanguage{greek}{πυρ καταβηναι απο του ουρανου και α} & 19 &  &  \\
&  & 19 & \foreignlanguage{greek}{ναλωϲαι αυτουϲ ωϲ και ηλιαϲ εποιηϲεν} & 24 &  &  \\
& \textbf{55} &  & \foreignlanguage{greek}{ϲτραφειϲ δε επετιμηϲεν αυτοιϲ και} & 1 &  &  \\
&  & 2 & \foreignlanguage{greek}{επορευθηϲαν ειϲ ετεραν κωμην} & 5 &  &  \\
& \textbf{57} &  & \foreignlanguage{greek}{εγενετο δε πορευομενων αυτων ε̅} & 5 &  &  \\
&  & 6 & \foreignlanguage{greek}{τη οδω ειπεν τιϲ προϲ αυτον ακο} & 12 &  &  \\
&  & 12 & \foreignlanguage{greek}{λουθηϲω ϲοι οπου αν απερχη \textoverline{κε}} & 17 &  &  \\
& \textbf{58} &  & \foreignlanguage{greek}{και ειπεν αυτω ο \textoverline{ιϲ} αι αλωπεκεϲ φω} & 8 &  &  \\
&  & 8 & \foreignlanguage{greek}{λεουϲ εχουϲιν και τα πετινα του ου} & 14 &  &  \\
&  & 14 & \foreignlanguage{greek}{ρανου καταϲκηνωϲειϲ ο δε υιοϲ του} & 19 &  &  \\
&  & 20 & \foreignlanguage{greek}{\textoverline{ανου} ουκ εχει που την κεφαλη κλινη} & 26 &  &  \\
& \textbf{59} &  & \foreignlanguage{greek}{ειπεν δε προϲ ετερον ακολουθει μοι} & 6 &  &  \\
&  & 7 & \foreignlanguage{greek}{ο δε ειπεν \textoverline{κε} επιτρεψον μοι απελ} & 13 &  &  \\
&  & 13 & \foreignlanguage{greek}{θοντι θαψαι τον πατερα μου} & 17 &  &  \\
& \textbf{60} &  & \foreignlanguage{greek}{ειπεν δε αυτω ο \textoverline{ιϲ} αφεϲ τουϲ νεκρουϲ} & 8 &  &  \\
&  & 9 & \foreignlanguage{greek}{θαψαι τουϲ νεκρουϲ εαυτων ϲυ δε α} & 15 &  &  \\
&  & 15 & \foreignlanguage{greek}{πελθων διαγγελλε την βαϲιλειαν} & 18 &  &  \\
&  & 19 & \foreignlanguage{greek}{του \textoverline{θυ}} & 20 &  &  \\
& \textbf{61} &  & \foreignlanguage{greek}{ειπεν δε και ετεροϲ ακολουθηϲω ϲοι} & 6 &  &  \\
&  & 7 & \foreignlanguage{greek}{\textoverline{κε} πρωτον δε επιτρεψον μοι αποτα} & 12 &  &  \\
&  & 12 & \foreignlanguage{greek}{ξαϲθαι τοιϲ ειϲ τον οικον μου} & 17 &  &  \\
& \textbf{62} &  & \foreignlanguage{greek}{ειπεν δε ο \textoverline{ιϲ} προϲ αυτον ουδειϲ επι} & 8 &  &  \\
&  & 8 & \foreignlanguage{greek}{βαλλων την χειρα αυτου επ αροτρο̅} & 13 &  &  \\
&  & 14 & \foreignlanguage{greek}{και βλεπων ειϲ τα οπιϲω ευθετοϲ ε} & 20 &  &  \\
&  & 20 & \foreignlanguage{greek}{ϲτιν ειϲ την βαϲιλειαν του \textoverline{θυ}} & 25 &  &  \\
& \mygospelchapter &  & \foreignlanguage{greek}{μετα δε ταυτα ανεδειξεν ο \textoverline{κϲ} και} & 7 &  &  \\
&  & 8 & \foreignlanguage{greek}{ετερουϲ εβδομηκοντα και απε} & 11 &  &  \\
&  & 11 & \foreignlanguage{greek}{ϲτιλεν αυτουϲ ανα δυο προ προϲω} & 16 &  &  \\
&  & 16 & \foreignlanguage{greek}{που αυτου ειϲ παϲαν πολιν και τοπο̅} & 22 &  &  \\
&  & 23 & \foreignlanguage{greek}{ου ημελλεν αυτοϲ ερχεϲθαι} & 26 &  &  \\
[0.2em]
\cline{4-4}
\end{tabular}
\end{center}
\end{table}
}
\clearpage
\newpage
 {
 \setlength\arrayrulewidth{1pt}
\begin{table}
\begin{center}
\begin{tabular}{ccc|l|ccc}
\cline{4-4} \\ [-1em]
\multicolumn{7}{c}{\foreignlanguage{greek}{ευαγγελιον κατα λουκαν} \textbf{(\nospace{10:2})} } \\ \\ [-1em] % Si on veut ajouter les bordures latérales, remplacer {7}{c} par {7}{|c|}
\cline{4-4} \\
\cline{4-4}
&  &  & &  &  & \\ [-0.9em]
& \textbf{2} &  & \foreignlanguage{greek}{ελεγεν ουν προϲ αυτουϲ ο μεν θεριϲ} & 7 &  &  \\
&  & 7 & \foreignlanguage{greek}{μοϲ πολυϲ οι δε εργαται ολειγοι} & 12 &  &  \\
&  & 13 & \foreignlanguage{greek}{δεηθηται ουν του \textoverline{κυ} του θεριϲμου} & 18 &  &  \\
&  & 19 & \foreignlanguage{greek}{οπωϲ εκβαλη εργαταϲ ειϲ τον θεριϲμον} & 24 &  &  \\
&  & 25 & \foreignlanguage{greek}{αυτου υπαγεται ιδου εγω απο} & 4 & \textbf{3} &  \\
&  & 4 & \foreignlanguage{greek}{ϲτελλω υμαϲ ωϲ αρναϲ εν μεϲω λυκω̅} & 10 &  &  \\
& \textbf{4} &  & \foreignlanguage{greek}{μη βαϲταζεται βαλαντιον μη πηρα̅} & 5 &  &  \\
&  & 6 & \foreignlanguage{greek}{μηδε υποδηματα και μηδενα κα} & 10 &  &  \\
&  & 10 & \foreignlanguage{greek}{τα την οδον αϲπαϲαϲθαι} & 13 &  &  \\
& \textbf{5} &  & \foreignlanguage{greek}{ειϲ ην δ αν οικειαν ειϲερχηϲθαι πρω} & 7 &  &  \\
&  & 7 & \foreignlanguage{greek}{τον λεγεται ειρηνη τω οικω τουτω} & 12 &  &  \\
& \textbf{6} &  & \foreignlanguage{greek}{και εαν η εκει υιοϲ ειρηνηϲ επανα} & 7 &  &  \\
&  & 7 & \foreignlanguage{greek}{παυϲηται επ αυτον η ειρηνη υμων} & 12 &  &  \\
&  & 13 & \foreignlanguage{greek}{ει δε μη γε εφ υμαϲ ανακαμψει} & 19 &  &  \\
& \textbf{7} &  & \foreignlanguage{greek}{εν αυτη δε τη οικεια μενεται εϲθιο̅} & 7 &  &  \\
&  & 7 & \foreignlanguage{greek}{τεϲ τα παρ αυτων αξιοϲ γαρ ο εργα} & 14 &  &  \\
&  & 14 & \foreignlanguage{greek}{τηϲ του μιϲθου αυτου εϲτιν μη μετα} & 20 &  &  \\
&  & 20 & \foreignlanguage{greek}{βαινεται εξ οικειαϲ ειϲ οικιαν} & 24 &  &  \\
& \textbf{8} &  & \foreignlanguage{greek}{και ειϲ ην αν πολιν ειϲερχηϲθαι και} & 7 &  &  \\
&  & 8 & \foreignlanguage{greek}{δεχονται υμαϲ εϲθιεται τα παρατι} & 12 &  &  \\
&  & 12 & \foreignlanguage{greek}{θεμενα υμιν και θεραπευεται τουϲ} & 3 & \textbf{9} &  \\
&  & 4 & \foreignlanguage{greek}{εν αυτη αϲθενειϲ και λεγεται αυτοιϲ} & 9 &  &  \\
&  & 10 & \foreignlanguage{greek}{ηγγικεν εφ υμαϲ η βαϲιλεια του \textoverline{θυ}} & 16 &  &  \\
& \textbf{10} &  & \foreignlanguage{greek}{ειϲ ην δ αν πολιν ειϲερχηϲθαι και μη} & 8 &  &  \\
&  & 9 & \foreignlanguage{greek}{δεχωνται υμαϲ εξελθοντεϲ ειϲ ταϲ} & 13 &  &  \\
&  & 14 & \foreignlanguage{greek}{πλατιουϲ αυτηϲ ειπατε και τον κο} & 3 & \textbf{11} &  \\
&  & 3 & \foreignlanguage{greek}{νιορτον τον κολληθεντα ημιν ημιν} & 7 &  &  \\
&  & 9 & \foreignlanguage{greek}{τηϲ πολεωϲ υμων} & 11 &  &  \\
&  & 16 & \foreignlanguage{greek}{απομαϲϲομεθα υμιν πλην τουτο} & 19 &  &  \\
&  & 20 & \foreignlanguage{greek}{γινωϲκεται οτι ηγγεικεν εφ υμαϲ} & 24 &  &  \\
[0.2em]
\cline{4-4}
\end{tabular}
\end{center}
\end{table}
}
\clearpage
\newpage
 {
 \setlength\arrayrulewidth{1pt}
\begin{table}
\begin{center}
\begin{tabular}{ccc|l|ccc}
\cline{4-4} \\ [-1em]
\multicolumn{7}{c}{\foreignlanguage{greek}{ευαγγελιον κατα λουκαν} \textbf{(\nospace{10:11})} } \\ \\ [-1em] % Si on veut ajouter les bordures latérales, remplacer {7}{c} par {7}{|c|}
\cline{4-4} \\
\cline{4-4}
&  &  & &  &  & \\ [-0.9em]
&  & 25 & \foreignlanguage{greek}{η βαϲιλεια του \textoverline{θυ} λεγω υμιν οτι ϲοδο} & 4 & \textbf{12} &  \\
&  & 4 & \foreignlanguage{greek}{μοιϲ εν τη ημερα εκεινη ανεκτοτε} & 9 &  &  \\
&  & 9 & \foreignlanguage{greek}{ρον εϲται η τη πολει εκεινη} & 14 &  &  \\
& \textbf{13} &  & \foreignlanguage{greek}{ουα ϲοι χωρεζειν ουαι ϲοι βηθϲαιδα̅} & 6 &  &  \\
&  & 7 & \foreignlanguage{greek}{οτι εν τυρω και ϲιδονει εγενοντο} & 12 &  &  \\
&  & 13 & \foreignlanguage{greek}{αι δυναμειϲ αι γενομεναι εν υμιν} & 18 &  &  \\
&  & 19 & \foreignlanguage{greek}{παλαι αν εν ϲακκω και ϲποδω καθη} & 25 &  &  \\
&  & 25 & \foreignlanguage{greek}{μεναι μετενοηϲαν πλην τυρω και} & 3 & \textbf{14} &  \\
&  & 4 & \foreignlanguage{greek}{ϲιδονι ανεκτοτερον εϲται εν τη} & 8 &  &  \\
&  & 9 & \foreignlanguage{greek}{κριϲει η υμιν και ϲυ καπερναουμ} & 3 & \textbf{15} &  \\
&  & 4 & \foreignlanguage{greek}{η εωϲ του ουρανου υψωθειϲα εωϲ α} & 10 &  &  \\
&  & 10 & \foreignlanguage{greek}{δου καταβιβαϲθηϲη} & 11 &  &  \\
& \textbf{16} &  & \foreignlanguage{greek}{ο ακουων υμων εμου ακουει και ο} & 7 &  &  \\
&  & 8 & \foreignlanguage{greek}{αθετων υμαϲ εμε αθετει ο δε εμε} & 14 &  &  \\
&  & 15 & \foreignlanguage{greek}{αθετων αθετει τον αποϲτιλαντα με} & 19 &  &  \\
& \textbf{17} &  & \foreignlanguage{greek}{υπεϲτρεψαν δε οι εβδομηκοντα με} & 5 &  &  \\
&  & 5 & \foreignlanguage{greek}{τα χαραϲ λεγοντεϲ \textoverline{κε} και τα δαιμο} & 11 &  &  \\
&  & 11 & \foreignlanguage{greek}{νια υποταϲϲεται ημιν εν ω ονοματι ϲου} & 17 &  &  \\
& \textbf{18} &  & \foreignlanguage{greek}{ειπεν δε αυτοιϲ εθεωρουν τον ϲατανα̅} & 6 &  &  \\
&  & 7 & \foreignlanguage{greek}{ωϲ αϲτραπην εκ του ουρανου πεϲον} & 12 &  &  \\
&  & 12 & \foreignlanguage{greek}{τα ιδου δεδωκα υμιν την εξουϲια̅} & 5 & \textbf{19} &  \\
&  & 6 & \foreignlanguage{greek}{πατιν επανω οφεων και ϲκορπιω̅} & 10 &  &  \\
&  & 11 & \foreignlanguage{greek}{και επι παϲαν την δυναμιν του εχθρου} & 17 &  &  \\
&  & 18 & \foreignlanguage{greek}{και ουδεν υμαϲ ου μη αδικηϲει} & 23 &  &  \\
& \textbf{20} &  & \foreignlanguage{greek}{πλην εν τουτω μη χαιρεται οτι τα} & 7 &  &  \\
&  & 8 & \foreignlanguage{greek}{\textoverline{πνα} υμιν υποταϲϲεται χαιρεται} & 11 &  &  \\
&  & 12 & \foreignlanguage{greek}{δε οτι τα ονοματα υμων εγραφη} & 17 &  &  \\
&  & 18 & \foreignlanguage{greek}{εν τοιϲ ουρανοιϲ} & 20 &  &  \\
& \textbf{21} &  & \foreignlanguage{greek}{εν αυτη τη ωρα ηγαλλιαϲατο τω \textoverline{πνι}} & 7 &  &  \\
&  & 8 & \foreignlanguage{greek}{ο \textoverline{ιϲ} και ειπεν εξομολογουμαι ϲοι} & 13 &  &  \\
[0.2em]
\cline{4-4}
\end{tabular}
\end{center}
\end{table}
}
\clearpage
\newpage
 {
 \setlength\arrayrulewidth{1pt}
\begin{table}
\begin{center}
\begin{tabular}{ccc|l|ccc}
\cline{4-4} \\ [-1em]
\multicolumn{7}{c}{\foreignlanguage{greek}{ευαγγελιον κατα λουκαν} \textbf{(\nospace{10:21})} } \\ \\ [-1em] % Si on veut ajouter les bordures latérales, remplacer {7}{c} par {7}{|c|}
\cline{4-4} \\
\cline{4-4}
&  &  & &  &  & \\ [-0.9em]
&  & 14 & \foreignlanguage{greek}{πατερ \textoverline{κε} του ουρανου και τηϲ γηϲ} & 20 &  &  \\
&  & 21 & \foreignlanguage{greek}{οτι απεκρυψαϲ ταυτα απο ϲοφων και} & 26 &  &  \\
&  & 27 & \foreignlanguage{greek}{ϲυνετων και απεκαλυψαϲ αυτα νη} & 31 &  &  \\
&  & 31 & \foreignlanguage{greek}{πιοιϲ ναι ο \textoverline{πηρ} οτι ουτωϲ εγενετο ευ} & 38 &  &  \\
&  & 38 & \foreignlanguage{greek}{δοκεια εμπροϲθεν ϲου} & 40 &  &  \\
& \textbf{22} &  & \foreignlanguage{greek}{και ϲτραφειϲ προϲ τουϲ μαθηταϲ ειπε̅} & 6 &  &  \\
&  & 7 & \foreignlanguage{greek}{παντα μοι παρεδοθη υπο του \textoverline{πρϲ} μου} & 13 &  &  \\
&  & 14 & \foreignlanguage{greek}{και ουδειϲ γιγνωϲκει τιϲ εϲτιν ο υι} & 20 &  &  \\
&  & 20 & \foreignlanguage{greek}{οϲ ει μη ο \textoverline{πηρ} και τιϲ εϲτιν ο \textoverline{πηρ} ει} & 30 &  &  \\
&  & 31 & \foreignlanguage{greek}{μη ο υιοϲ και ω εαν βουλεται ο υιοϲ} & 39 &  &  \\
&  & 40 & \foreignlanguage{greek}{αποκαλυψαι} & 40 &  &  \\
& \textbf{23} &  & \foreignlanguage{greek}{και ϲτραφειϲ προϲ τουϲ μαθηταϲ καθ} & 6 &  &  \\
&  & 7 & \foreignlanguage{greek}{ιδιαν ειπεν μακαριοι οι οφθαλμοι} & 11 &  &  \\
&  & 12 & \foreignlanguage{greek}{οι βλεποντεϲ α βλεπεται λεγω γαρ} & 2 & \textbf{24} &  \\
&  & 3 & \foreignlanguage{greek}{υμιν οτι πολλοι προφηται και βαϲι} & 8 &  &  \\
&  & 8 & \foreignlanguage{greek}{λειϲ ηθεληϲαν ιδειν α υμειϲ βλεπε} & 13 &  &  \\
&  & 13 & \foreignlanguage{greek}{ται και ουχ ειδον και ακουϲαι α α} & 20 &  &  \\
&  & 20 & \foreignlanguage{greek}{κουεται και ουκ ηκουϲαν} & 23 &  &  \\
& \textbf{25} &  & \foreignlanguage{greek}{και ιδου νομικοϲ τιϲ ανεϲτη εκπειρα} & 6 &  &  \\
&  & 6 & \foreignlanguage{greek}{ζων αυτον και λεγων διδαϲκαλε} & 10 &  &  \\
&  & 11 & \foreignlanguage{greek}{τι ποιηϲαϲ ζωην αιωνιον κληρονο} & 15 &  &  \\
&  & 15 & \foreignlanguage{greek}{μηϲω ο δε ειπεν προϲ αυτον} & 5 & \textbf{26} &  \\
&  & 6 & \foreignlanguage{greek}{εν τω νομω τι γεγραπται πωϲ ανα} & 12 &  &  \\
&  & 12 & \foreignlanguage{greek}{γιγνωϲκειϲ ο δε αποκριθειϲ ειπε̅} & 4 & \textbf{27} &  \\
&  & 5 & \foreignlanguage{greek}{αγαπηϲειϲ \textoverline{κν} τον \textoverline{θν} ϲου εξ οληϲ τηϲ} & 12 &  &  \\
&  & 13 & \foreignlanguage{greek}{καρδιαϲ ϲου και εξ οληϲ τηϲ ψυχηϲ ϲου} & 20 &  &  \\
&  & 21 & \foreignlanguage{greek}{και εξ οληϲ τηϲ ιϲχυοϲ ϲου και εξ ο} & 29 &  &  \\
&  & 29 & \foreignlanguage{greek}{ληϲ τηϲ διανοιαϲ ϲου και τον πληϲι} & 35 &  &  \\
&  & 35 & \foreignlanguage{greek}{ον ϲου ωϲ ϲεαυτον} & 38 &  &  \\
& \textbf{28} &  & \foreignlanguage{greek}{ειπεν δε αυτω ορθωϲ απεκριθηϲ} & 5 &  &  \\
[0.2em]
\cline{4-4}
\end{tabular}
\end{center}
\end{table}
}
\clearpage
\newpage
 {
 \setlength\arrayrulewidth{1pt}
\begin{table}
\begin{center}
\begin{tabular}{ccc|l|ccc}
\cline{4-4} \\ [-1em]
\multicolumn{7}{c}{\foreignlanguage{greek}{ευαγγελιον κατα λουκαν} \textbf{(\nospace{10:28})} } \\ \\ [-1em] % Si on veut ajouter les bordures latérales, remplacer {7}{c} par {7}{|c|}
\cline{4-4} \\
\cline{4-4}
&  &  & &  &  & \\ [-0.9em]
&  & 6 & \foreignlanguage{greek}{τουτο ποιει και ζηϲη ο δε θελων δι} & 4 & \textbf{29} &  \\
&  & 4 & \foreignlanguage{greek}{καιουν εαυτον ειπεν προϲ τον \textoverline{ιν} και} & 10 &  &  \\
&  & 11 & \foreignlanguage{greek}{τιϲ εϲτιν μου πληϲιον} & 14 &  &  \\
& \textbf{30} &  & \foreignlanguage{greek}{υπολαβων δε ο \textoverline{ιϲ} ειπεν ανθρωποϲ} & 6 &  &  \\
&  & 7 & \foreignlanguage{greek}{τιϲ κατεβαινεν απο ιερουϲαλημ ειϲ} & 11 &  &  \\
&  & 12 & \foreignlanguage{greek}{ιεριχω και ληϲταιϲ περιεπεϲεν οι} & 16 &  &  \\
&  & 17 & \foreignlanguage{greek}{και εκδυϲαντεϲ αυτον και πληγαϲ} & 21 &  &  \\
&  & 22 & \foreignlanguage{greek}{επιθεντεϲ απηλθον αφεντεϲ ημι} & 25 &  &  \\
&  & 25 & \foreignlanguage{greek}{θανη τυγχανοντα κατα ϲυνκυ} & 2 & \textbf{31} &  \\
&  & 2 & \foreignlanguage{greek}{ριαν δε ιερευϲ τιϲ καταβαινων εν} & 7 &  &  \\
&  & 8 & \foreignlanguage{greek}{τη οδω εκεινη και ιδων αυτον αν} & 14 &  &  \\
&  & 14 & \foreignlanguage{greek}{τιπαρηλθεν ομοιωϲ και λευ} & 3 & \textbf{32} &  \\
&  & 3 & \foreignlanguage{greek}{ειτηϲ γενομενοϲ κατα τον τοπον} & 7 &  &  \\
&  & 8 & \foreignlanguage{greek}{ελθων και ιδων αντιπαρηλθεν} & 11 &  &  \\
& \textbf{33} &  & \foreignlanguage{greek}{ϲαμαριτηϲ δε τιϲ οδευων ηλθεν} & 5 &  &  \\
&  & 6 & \foreignlanguage{greek}{κατ αυτον και ιδων αυτον εϲπλαγ} & 11 &  &  \\
&  & 11 & \foreignlanguage{greek}{χνιϲθη και προϲελθων κατεδη} & 3 & \textbf{34} &  \\
&  & 3 & \foreignlanguage{greek}{ϲεν τα τραυματα αυτου επιχεων} & 7 &  &  \\
&  & 8 & \foreignlanguage{greek}{ελαιον και οινον επιβιβαϲαϲ δε} & 12 &  &  \\
&  & 13 & \foreignlanguage{greek}{αυτον επι το ιδιον κτηνοϲ ηγαγε̅} & 18 &  &  \\
&  & 19 & \foreignlanguage{greek}{αυτον ειϲ πανδοχιον και επεμε} & 23 &  &  \\
&  & 23 & \foreignlanguage{greek}{ληθη αυτου} & 24 &  &  \\
& \textbf{35} &  & \foreignlanguage{greek}{και επι την αυριον εξελθων εκ} & 6 &  &  \\
&  & 6 & \foreignlanguage{greek}{βαλων δυο δηναρια εδωκεν τω} & 10 &  &  \\
&  & 11 & \foreignlanguage{greek}{πανδοχει και ειπεν αυτω επι} & 15 &  &  \\
&  & 15 & \foreignlanguage{greek}{μεληθητι αυτου και ο τι αν προϲ} & 21 &  &  \\
&  & 21 & \foreignlanguage{greek}{δαπανηϲηϲ εγω εν τω επανερ} & 25 &  &  \\
&  & 25 & \foreignlanguage{greek}{χεϲθαι με αποδωϲω ϲοι} & 28 &  &  \\
& \textbf{36} &  & \foreignlanguage{greek}{τιϲ ουν τουτων των τριων πληϲι} & 6 &  &  \\
&  & 6 & \foreignlanguage{greek}{ον δοκει ϲοι γεγονεναι του εμπε} & 11 &  &  \\
[0.2em]
\cline{4-4}
\end{tabular}
\end{center}
\end{table}
}
\clearpage
\newpage
 {
 \setlength\arrayrulewidth{1pt}
\begin{table}
\begin{center}
\begin{tabular}{ccc|l|ccc}
\cline{4-4} \\ [-1em]
\multicolumn{7}{c}{\foreignlanguage{greek}{ευαγγελιον κατα λουκαν} \textbf{(\nospace{10:36})} } \\ \\ [-1em] % Si on veut ajouter les bordures latérales, remplacer {7}{c} par {7}{|c|}
\cline{4-4} \\
\cline{4-4}
&  &  & &  &  & \\ [-0.9em]
&  & 11 & \foreignlanguage{greek}{ϲοντοϲ ειϲ τουϲ ληϲταϲ ο δε ειπεν} & 3 & \textbf{37} &  \\
&  & 4 & \foreignlanguage{greek}{ο ποιηϲαϲ το ελεοϲ μετ αυτου} & 9 &  &  \\
&  & 10 & \foreignlanguage{greek}{ειπεν ουν ο \textoverline{ιϲ} πορευου και ϲυ ποιει ομοιωϲ} & 18 &  &  \\
& \textbf{38} &  & \foreignlanguage{greek}{εγενετο δε εν τω πορευεϲθαι αυτουϲ} & 6 &  &  \\
&  & 7 & \foreignlanguage{greek}{και αυτοϲ ειϲηλθεν ειϲ κωμην τινα} & 12 &  &  \\
&  & 13 & \foreignlanguage{greek}{γυνη δε τιϲ ονοματι μαρθα υπεδε} & 18 &  &  \\
&  & 18 & \foreignlanguage{greek}{ξατο αυτον ειϲ τον οικον αυτηϲ} & 23 &  &  \\
& \textbf{39} &  & \foreignlanguage{greek}{και ταυτη ην αδελφη καλουμενη} & 5 &  &  \\
&  & 6 & \foreignlanguage{greek}{μαριαμ η και παρακαθειϲαϲα πα} & 10 &  &  \\
&  & 10 & \foreignlanguage{greek}{ρα τουϲ ποδαϲ του \textoverline{ιυ} ηκουεν τον} & 16 &  &  \\
&  & 17 & \foreignlanguage{greek}{λογον αυτου η δε μαρθα περιε} & 4 & \textbf{40} &  \\
&  & 4 & \foreignlanguage{greek}{ϲπατο περι πολλην διακονιαν} & 7 &  &  \\
&  & 8 & \foreignlanguage{greek}{επιϲταϲα δε ειπεν \textoverline{κε} ου μελει ϲοι} & 14 &  &  \\
&  & 15 & \foreignlanguage{greek}{οτι η αδελφη μου μονην με ενκα} & 21 &  &  \\
&  & 21 & \foreignlanguage{greek}{τελιψεν διακονειν ειπε ουν αυ} & 25 &  &  \\
&  & 25 & \foreignlanguage{greek}{τη ινα μοι ϲυναντιλαβηται} & 28 &  &  \\
& \textbf{41} &  & \foreignlanguage{greek}{αποκριθειϲ δε ειπεν αυτη ο \textoverline{ιϲ} μαρ} & 7 &  &  \\
&  & 7 & \foreignlanguage{greek}{θα μαρθα μεριμναϲ και θορυβαζη} & 11 &  &  \\
&  & 12 & \foreignlanguage{greek}{περι πολλα ενοϲ δε εϲτιν χρεια} & 4 & \textbf{42} &  \\
&  & 5 & \foreignlanguage{greek}{μαρια δε την αγαθην μεριδα εξε} & 10 &  &  \\
&  & 10 & \foreignlanguage{greek}{λεξατο ητιϲ ουκ αφερεθηϲεται} & 13 &  &  \\
&  & 14 & \foreignlanguage{greek}{απ αυτηϲ} & 15 &  &  \\
& \mygospelchapter &  & \foreignlanguage{greek}{και εγενετο εν τω ειναι αυτον εν} & 7 &  &  \\
&  & 8 & \foreignlanguage{greek}{τοπω τινι προϲευχομενον ωϲ ε} & 12 &  &  \\
&  & 12 & \foreignlanguage{greek}{παυϲατο ειπεν τιϲ των μαθητων} & 16 &  &  \\
&  & 17 & \foreignlanguage{greek}{αυτου προϲ αυτον \textoverline{κε} διδαξον} & 21 &  &  \\
&  & 22 & \foreignlanguage{greek}{ημαϲ προϲευχεϲθαι καθωϲ και ι} & 26 &  &  \\
&  & 26 & \foreignlanguage{greek}{ωαννηϲ εδιδαξεν τουϲ μαθηταϲ} & 29 &  &  \\
&  & 30 & \foreignlanguage{greek}{αυτου ειπεν δε αυτοιϲ} & 3 & \textbf{2} &  \\
&  & 4 & \foreignlanguage{greek}{οταν προϲευχεϲθαι λεγεται πατερ} & 7 &  &  \\
[0.2em]
\cline{4-4}
\end{tabular}
\end{center}
\end{table}
}
\clearpage
\newpage
 {
 \setlength\arrayrulewidth{1pt}
\begin{table}
\begin{center}
\begin{tabular}{ccc|l|ccc}
\cline{4-4} \\ [-1em]
\multicolumn{7}{c}{\foreignlanguage{greek}{ευαγγελιον κατα λουκαν} \textbf{(\nospace{11:2})} } \\ \\ [-1em] % Si on veut ajouter les bordures latérales, remplacer {7}{c} par {7}{|c|}
\cline{4-4} \\
\cline{4-4}
&  &  & &  &  & \\ [-0.9em]
&  & 8 & \foreignlanguage{greek}{ημων ο εν τοιϲ ουρανοιϲ αγιαϲθητω} & 13 &  &  \\
&  & 14 & \foreignlanguage{greek}{το ονομα ϲου ελθατω η βαϲιλεια ϲου} & 20 &  &  \\
&  & 21 & \foreignlanguage{greek}{γενηθητω το θελημα ϲου ωϲ εν ου} & 27 &  &  \\
&  & 27 & \foreignlanguage{greek}{ρανω και επι γηϲ τον αρτον ημων} & 3 & \textbf{3} &  \\
&  & 4 & \foreignlanguage{greek}{τον επιουϲιον διδου ημιν το καθ η} & 10 &  &  \\
&  & 10 & \foreignlanguage{greek}{μεραν και αφεϲ ημιν ταϲ αμαρτι} & 5 & \textbf{4} &  \\
&  & 5 & \foreignlanguage{greek}{αϲ ημων και γαρ αυτοι αφειολομεν} & 10 &  &  \\
&  & 12 & \foreignlanguage{greek}{παντι οφειλοντι ημιν και μη ειϲ} & 17 &  &  \\
&  & 17 & \foreignlanguage{greek}{ενεγκηϲ ημαϲ ειϲ πειραϲμον αλλα} & 21 &  &  \\
&  & 22 & \foreignlanguage{greek}{ρυϲαι ημαϲ απο του πονηρου} & 26 &  &  \\
& \textbf{5} &  & \foreignlanguage{greek}{και ειπεν προϲ αυτουϲ τιϲ εξ υμων} & 7 &  &  \\
&  & 8 & \foreignlanguage{greek}{εξει φιλον και πορευϲεται προϲ αυτο̅} & 13 &  &  \\
&  & 14 & \foreignlanguage{greek}{μεϲονυκτιου και ερει αυτω} & 17 &  &  \\
&  & 18 & \foreignlanguage{greek}{φιλε χρηϲον μοι τριϲ αρτουϲ επειδη} & 1 & \textbf{6} &  \\
&  & 2 & \foreignlanguage{greek}{φιλοϲ μου παρεγενετο εξ οδου προϲ} & 7 &  &  \\
&  & 8 & \foreignlanguage{greek}{με και ουκ εχω ο παραθηϲω αυτω} & 14 &  &  \\
& \textbf{7} &  & \foreignlanguage{greek}{κακεινοϲ εϲωθεν αποκριθειϲ ειπη} & 4 &  &  \\
&  & 5 & \foreignlanguage{greek}{μη μοι κοπουϲ παρεχε ηδη η θυρα} & 11 &  &  \\
&  & 12 & \foreignlanguage{greek}{κεκλειϲται και τα παιδια μου με} & 17 &  &  \\
&  & 17 & \foreignlanguage{greek}{τ εμου ειϲ την κοιτην εϲτιν ου δυ} & 24 &  &  \\
&  & 24 & \foreignlanguage{greek}{ναμαι αναϲταϲ δουναι ϲοι} & 27 &  &  \\
& \textbf{8} &  & \foreignlanguage{greek}{λεγω υμιν ει και ου δωϲει αυτω} & 7 &  &  \\
&  & 8 & \foreignlanguage{greek}{αναϲταϲ δια το ειναι αυτου φιλοϲ} & 13 &  &  \\
&  & 14 & \foreignlanguage{greek}{δια γε την αναιδιαν αυτου εγερ} & 19 &  &  \\
&  & 19 & \foreignlanguage{greek}{θειϲ δωϲη αυτω οϲων χρηζει} & 23 &  &  \\
& \textbf{9} &  & \foreignlanguage{greek}{καγω υμιν λεγω αιτιται και δοθη} & 6 &  &  \\
&  & 6 & \foreignlanguage{greek}{ϲεται υμιν ζητειται και ευρηϲεται} & 10 &  &  \\
&  & 11 & \foreignlanguage{greek}{κρουεται και ανυχθηϲεται υμιν} & 14 &  &  \\
& \textbf{10} &  & \foreignlanguage{greek}{παϲ γαρ ο αιτων λαμβανει και ο ζη} & 8 &  &  \\
&  & 8 & \foreignlanguage{greek}{των ευριϲκει ϗ τω κρουοντι ανηχθη} & 13 &  &  \\
&  & 13 & \foreignlanguage{greek}{ϲεται} & 13 &  &  \\
[0.2em]
\cline{4-4}
\end{tabular}
\end{center}
\end{table}
}
\clearpage
\newpage
 {
 \setlength\arrayrulewidth{1pt}
\begin{table}
\begin{center}
\begin{tabular}{ccc|l|ccc}
\cline{4-4} \\ [-1em]
\multicolumn{7}{c}{\foreignlanguage{greek}{ευαγγελιον κατα λουκαν} \textbf{(\nospace{11:11})} } \\ \\ [-1em] % Si on veut ajouter les bordures latérales, remplacer {7}{c} par {7}{|c|}
\cline{4-4} \\
\cline{4-4}
&  &  & &  &  & \\ [-0.9em]
& \textbf{11} &  & \foreignlanguage{greek}{τινα δε εξ υμων τον \textoverline{πρα} ο υιοϲ αιτηϲει} & 9 &  &  \\
&  & 10 & \foreignlanguage{greek}{αρτον μη λιθον επιδωϲει αυτω} & 14 &  &  \\
&  & 15 & \foreignlanguage{greek}{η και ιχθυν μη αντι ιχθυοϲ οφιν επι} & 22 &  &  \\
&  & 22 & \foreignlanguage{greek}{δωϲει αυτω η και αν αιτηϲη ωον μη} & 6 & \textbf{12} &  \\
&  & 7 & \foreignlanguage{greek}{επιδωϲη αυτω ϲκορπιον} & 9 &  &  \\
& \textbf{13} &  & \foreignlanguage{greek}{ει ουν υμειϲ πονηροι υπαρχοντεϲ οι} & 6 &  &  \\
&  & 6 & \foreignlanguage{greek}{δατε δοματα αγαθα διδοναι τοιϲ} & 10 &  &  \\
&  & 11 & \foreignlanguage{greek}{τεκνοιϲ υμων ποϲω μαλλον ο \textoverline{πηρ}} & 16 &  &  \\
&  & 17 & \foreignlanguage{greek}{ο εξ ουρανου δωϲει \textoverline{πνα} αγιον τοιϲ} & 23 &  &  \\
&  & 24 & \foreignlanguage{greek}{αιτουϲιν αυτον} & 25 &  &  \\
& \textbf{14} &  & \foreignlanguage{greek}{και ην εκβαλλων δαιμονιον και} & 5 &  &  \\
&  & 6 & \foreignlanguage{greek}{αυτο ην κωφον εγενετο δε του} & 11 &  &  \\
&  & 12 & \foreignlanguage{greek}{δαιμονιου εξελθοντοϲ ελαληϲεν} & 14 &  &  \\
&  & 15 & \foreignlanguage{greek}{ο κωφοϲ και εθαυμαϲαν οι οχλοι} & 20 &  &  \\
& \textbf{15} &  & \foreignlanguage{greek}{τινεϲ δε εξ αυτων ειπον εν βεελ} & 7 &  &  \\
&  & 7 & \foreignlanguage{greek}{ζεβουλ τω αρχοντι των δαιμονιων} & 11 &  &  \\
&  & 12 & \foreignlanguage{greek}{εκβαλλειν τα δαιμονια} & 14 &  &  \\
& \textbf{16} &  & \foreignlanguage{greek}{ετεροι δε πειραζοντεϲ ϲημιον παρ} & 5 &  &  \\
&  & 6 & \foreignlanguage{greek}{αυτου εζητουν εξ ουρανου} & 9 &  &  \\
& \textbf{17} &  & \foreignlanguage{greek}{αυτοϲ δε ειδωϲ αυτων τα διανοη} & 6 &  &  \\
&  & 6 & \foreignlanguage{greek}{ματα ειπεν αυτοιϲ παϲα βαϲιλει} & 10 &  &  \\
&  & 10 & \foreignlanguage{greek}{α εφ εαυτην μεριϲθειϲα ερημουται} & 14 &  &  \\
&  & 15 & \foreignlanguage{greek}{και οικοϲ επι οικον πιπτει} & 19 &  &  \\
& \textbf{18} &  & \foreignlanguage{greek}{ει δε και ο ϲαταναϲ εφ εαυτον εμε} & 8 &  &  \\
&  & 8 & \foreignlanguage{greek}{ριϲθη πωϲ ϲταθηϲεται η βαϲιλεια αυ} & 13 &  &  \\
&  & 13 & \foreignlanguage{greek}{του οτι λεγεται εν βεελζεβουλ εκ} & 18 &  &  \\
&  & 18 & \foreignlanguage{greek}{βαλλει τα δαιμονια οι υιοι υμω̅} & 3 & \textbf{19} &  \\
&  & 4 & \foreignlanguage{greek}{εν τινι εκβαλουϲιν δια τουτο αυ} & 9 &  &  \\
&  & 9 & \foreignlanguage{greek}{τοι κριται υμων εϲονται} & 12 &  &  \\
& \textbf{20} &  & \foreignlanguage{greek}{ει δε εν δακτυλω \textoverline{θυ} εκβαλλω τα δαι} & 8 &  &  \\
[0.2em]
\cline{4-4}
\end{tabular}
\end{center}
\end{table}
}
\clearpage
\newpage
 {
 \setlength\arrayrulewidth{1pt}
\begin{table}
\begin{center}
\begin{tabular}{ccc|l|ccc}
\cline{4-4} \\ [-1em]
\multicolumn{7}{c}{\foreignlanguage{greek}{ευαγγελιον κατα λουκαν} \textbf{(\nospace{11:20})} } \\ \\ [-1em] % Si on veut ajouter les bordures latérales, remplacer {7}{c} par {7}{|c|}
\cline{4-4} \\
\cline{4-4}
&  &  & &  &  & \\ [-0.9em]
&  & 8 & \foreignlanguage{greek}{μονια αρα εφθαϲεν εφ υμαϲ η βαϲι} & 14 &  &  \\
&  & 14 & \foreignlanguage{greek}{λεια του \textoverline{θυ} οταν ο ιϲχυροϲ κα} & 4 & \textbf{21} &  \\
&  & 4 & \foreignlanguage{greek}{θωπλιϲμενοϲ φυλαϲϲη την εαυτου} & 7 &  &  \\
&  & 8 & \foreignlanguage{greek}{αυλην εν ειρηνη εϲτιν τα υπαρχον} & 13 &  &  \\
&  & 13 & \foreignlanguage{greek}{τα αυτου επαν δε ο ιϲχυροτεροϲ αυ} & 5 & \textbf{22} &  \\
&  & 5 & \foreignlanguage{greek}{του επελθων νεικηϲει αυτον τη̅} & 9 &  &  \\
&  & 10 & \foreignlanguage{greek}{πανοπλειαν αυτου ερει εφ η επε} & 15 &  &  \\
&  & 15 & \foreignlanguage{greek}{ποιθει και τα ϲκυλα αυτου διαδι} & 20 &  &  \\
&  & 20 & \foreignlanguage{greek}{δωϲιν ο μη ων μετ εμου κατ εμου} & 7 & \textbf{23} &  \\
&  & 8 & \foreignlanguage{greek}{εϲτιν και ο μη ϲυναγων μετ εμου} & 14 &  &  \\
&  & 15 & \foreignlanguage{greek}{ϲκορπιζει} & 15 &  &  \\
& \textbf{24} &  & \foreignlanguage{greek}{οταν δε το ακαθαρτον \textoverline{πνα} εξελθη} & 6 &  &  \\
&  & 7 & \foreignlanguage{greek}{απο του \textoverline{ανου} διερχεται δι ανυδρω̅} & 12 &  &  \\
&  & 13 & \foreignlanguage{greek}{τοπων ζητουν αναπαυϲιν και μη} & 17 &  &  \\
&  & 18 & \foreignlanguage{greek}{ευριϲκον αναυπαϲιν λεγει υπο} & 21 &  &  \\
&  & 21 & \foreignlanguage{greek}{ϲτρεψω ειϲ τον οικον μου οθεν ε} & 27 &  &  \\
&  & 27 & \foreignlanguage{greek}{ξηλθον και ελθον ευριϲκει ϲε} & 4 & \textbf{25} &  \\
&  & 4 & \foreignlanguage{greek}{ϲαρωμενον και κεκοϲμημενον} & 6 &  &  \\
& \textbf{26} &  & \foreignlanguage{greek}{τοτε πορευεται και παραλαμβα} & 4 &  &  \\
&  & 4 & \foreignlanguage{greek}{νει επτα ετερα πνευματα πονη} & 8 &  &  \\
&  & 8 & \foreignlanguage{greek}{ροτερα εαυτου και ειϲελθον} & 11 &  &  \\
&  & 11 & \foreignlanguage{greek}{τα κατοικει εκει και γεινεται} & 15 &  &  \\
&  & 16 & \foreignlanguage{greek}{τα εϲχατα του ανθρωπου εκεινου} & 20 &  &  \\
&  & 21 & \foreignlanguage{greek}{χειρονα των πρωτων} & 23 &  &  \\
& \textbf{27} &  & \foreignlanguage{greek}{εγενετο δε εν τω λεγειν αυτον} & 6 &  &  \\
&  & 7 & \foreignlanguage{greek}{ταυτα επαραϲα τιϲ γυνη φωνη̅} & 11 &  &  \\
&  & 12 & \foreignlanguage{greek}{εκ του οχλου ειπεν αυτω} & 16 &  &  \\
&  & 17 & \foreignlanguage{greek}{μακαρια η κοιλια η βαϲταϲαϲα ϲε} & 22 &  &  \\
&  & 23 & \foreignlanguage{greek}{και μαϲτοι ουϲ εθηλαϲαϲ} & 26 &  &  \\
& \textbf{28} &  & \foreignlanguage{greek}{αυτοϲ δε ειπεν μενουν μακαρι} & 5 &  &  \\
[0.2em]
\cline{4-4}
\end{tabular}
\end{center}
\end{table}
}
\clearpage
\newpage
 {
 \setlength\arrayrulewidth{1pt}
\begin{table}
\begin{center}
\begin{tabular}{ccc|l|ccc}
\cline{4-4} \\ [-1em]
\multicolumn{7}{c}{\foreignlanguage{greek}{ευαγγελιον κατα λουκαν} \textbf{(\nospace{11:28})} } \\ \\ [-1em] % Si on veut ajouter les bordures latérales, remplacer {7}{c} par {7}{|c|}
\cline{4-4} \\
\cline{4-4}
&  &  & &  &  & \\ [-0.9em]
&  & 5 & \foreignlanguage{greek}{οι οι ακουοντεϲ τον λογον του \textoverline{θυ} και} & 12 &  &  \\
&  & 13 & \foreignlanguage{greek}{φυλαϲϲοντεϲ των δε οχλων επα} & 4 & \textbf{29} &  \\
&  & 4 & \foreignlanguage{greek}{θροιζομενων ηρξατο λεγειν} & 6 &  &  \\
&  & 7 & \foreignlanguage{greek}{η γενεα αυτη πονηρα εϲτιν ϲημιο̅} & 12 &  &  \\
&  & 13 & \foreignlanguage{greek}{επιζητει και ϲημιον ου δοθηϲεται} & 17 &  &  \\
&  & 18 & \foreignlanguage{greek}{αυτη ει μη το ϲημιον ιωνα του προ} & 25 &  &  \\
&  & 25 & \foreignlanguage{greek}{φητου καθωϲ γαρ εγενετο ιω} & 4 & \textbf{30} &  \\
&  & 4 & \foreignlanguage{greek}{ναϲ ϲημιον τοιϲ νινευειταιϲ ουτωϲ} & 8 &  &  \\
&  & 9 & \foreignlanguage{greek}{εϲται και ο υιοϲ του ανθρωπου τη γε} & 16 &  &  \\
&  & 16 & \foreignlanguage{greek}{νεα ταυτη} & 17 &  &  \\
& \textbf{31} &  & \foreignlanguage{greek}{βαϲιλιϲϲα νοτου εγερθηϲεται εν τη} & 5 &  &  \\
&  & 6 & \foreignlanguage{greek}{κριϲει μετα των ανδρων τηϲ γενε} & 11 &  &  \\
&  & 11 & \foreignlanguage{greek}{αϲ ταυτηϲ και κατακρινει αυτουϲ} & 15 &  &  \\
&  & 16 & \foreignlanguage{greek}{οτι ηλθεν εκ των περατων τηϲ γηϲ} & 22 &  &  \\
&  & 23 & \foreignlanguage{greek}{ακουϲαι την ϲοφιαν ϲολομωντοϲ} & 26 &  &  \\
&  & 27 & \foreignlanguage{greek}{και ιδου πλιον ϲολομωντοϲ ωδε} & 31 &  &  \\
& \textbf{32} &  & \foreignlanguage{greek}{ανδρεϲ νινευειται αναϲτηϲονται} & 3 &  &  \\
&  & 4 & \foreignlanguage{greek}{εν τη κριϲει μετα τηϲ γενεαϲ ταυτηϲ} & 10 &  &  \\
&  & 11 & \foreignlanguage{greek}{και κατακρινουϲιν αυτην οτι με} & 15 &  &  \\
&  & 15 & \foreignlanguage{greek}{τενοηϲαν ειϲ το κηρυγμα ιωνα} & 19 &  &  \\
&  & 20 & \foreignlanguage{greek}{και ιδου πλειον ιωνα ωδε} & 24 &  &  \\
& \textbf{33} &  & \foreignlanguage{greek}{ουδειϲ δε λυχνον αψαϲ ειϲ κρυπτη̅} & 6 &  &  \\
&  & 7 & \foreignlanguage{greek}{τιθηϲιν ουδε υπο τον μοδιον αλλ} & 12 &  &  \\
&  & 13 & \foreignlanguage{greek}{επι την λυχνιαν ινα οι ειϲπορευο} & 18 &  &  \\
&  & 18 & \foreignlanguage{greek}{μενοι το φεγγοϲ βλεπωϲιν} & 21 &  &  \\
& \textbf{34} &  & \foreignlanguage{greek}{ο λυχνοϲ του ϲωματοϲ εϲτιν ο οφθαλ} & 7 &  &  \\
&  & 7 & \foreignlanguage{greek}{μοϲ ϲου οταν ο οφθαλμοϲ ϲου α} & 13 &  &  \\
&  & 13 & \foreignlanguage{greek}{πλουϲ η και ολον το ϲωμα ϲου φωτι} & 20 &  &  \\
&  & 20 & \foreignlanguage{greek}{νον εϲτιν επαν δε πονηροϲ η και} & 26 &  &  \\
&  & 27 & \foreignlanguage{greek}{το ϲωμα ϲου ϲκοτινον} & 30 &  &  \\
[0.2em]
\cline{4-4}
\end{tabular}
\end{center}
\end{table}
}
\clearpage
\newpage
 {
 \setlength\arrayrulewidth{1pt}
\begin{table}
\begin{center}
\begin{tabular}{ccc|l|ccc}
\cline{4-4} \\ [-1em]
\multicolumn{7}{c}{\foreignlanguage{greek}{ευαγγελιον κατα λουκαν} \textbf{(\nospace{11:35})} } \\ \\ [-1em] % Si on veut ajouter les bordures latérales, remplacer {7}{c} par {7}{|c|}
\cline{4-4} \\
\cline{4-4}
&  &  & &  &  & \\ [-0.9em]
& \textbf{35} &  & \foreignlanguage{greek}{ϲκοπει ουν μη το φωϲ το εν ϲοι ϲκοτοϲ} & 9 &  &  \\
&  & 10 & \foreignlanguage{greek}{εϲτιν ει ουν το ϲωμα ϲου ολον φωτι} & 7 & \textbf{36} &  \\
&  & 7 & \foreignlanguage{greek}{νον μη εχον μεροϲ τι ϲκοτινον ε} & 13 &  &  \\
&  & 13 & \foreignlanguage{greek}{ϲται φωτινον ολον ωϲ οταν ο λυχνοϲ} & 19 &  &  \\
&  & 20 & \foreignlanguage{greek}{τη αϲτραπη φωτιζη ϲε} & 23 &  &  \\
& \textbf{37} &  & \foreignlanguage{greek}{εν δε τω λαληϲαι ερωτα αυτον φαρι} & 7 &  &  \\
&  & 7 & \foreignlanguage{greek}{ϲαιοϲ τιϲ οπωϲ αριϲτηϲει παρ αυτω} & 12 &  &  \\
&  & 13 & \foreignlanguage{greek}{ειϲελθων δε ανεπεϲεν ο δε φαρι} & 3 & \textbf{38} &  \\
&  & 3 & \foreignlanguage{greek}{ϲαιοϲ ειδων εθαυμαϲεν οτι ου πρω} & 8 &  &  \\
&  & 8 & \foreignlanguage{greek}{τον εβαπτιϲθη προ του αριϲτου} & 12 &  &  \\
& \textbf{39} &  & \foreignlanguage{greek}{ειπεν δε ο \textoverline{κϲ} προϲ αυτον νυν υμειϲ} & 8 &  &  \\
&  & 9 & \foreignlanguage{greek}{οι φαριϲαιοι το εξωθεν του ποτηρι} & 14 &  &  \\
&  & 14 & \foreignlanguage{greek}{ου και του πινακοϲ καθαριζεται} & 18 &  &  \\
&  & 19 & \foreignlanguage{greek}{το δε εϲωθεν υμων γεμει αρπαγηϲ} & 24 &  &  \\
&  & 25 & \foreignlanguage{greek}{και πονηριαϲ αφρονεϲ ουχ ο ποι} & 4 & \textbf{40} &  \\
&  & 4 & \foreignlanguage{greek}{ηϲαϲ το εξωθεν κα το εϲωθεν εποι} & 10 &  &  \\
&  & 10 & \foreignlanguage{greek}{ηϲεν πλην τα ενοντα δοτε ελεη} & 5 & \textbf{41} &  \\
&  & 5 & \foreignlanguage{greek}{μοϲυνην και ιδου παντα καθαρα} & 9 &  &  \\
&  & 10 & \foreignlanguage{greek}{υμιν εϲτιν} & 11 &  &  \\
& \textbf{42} &  & \foreignlanguage{greek}{αλλα ουαι υμιν τοιϲ φαριϲαιοιϲ οτι α} & 7 &  &  \\
&  & 7 & \foreignlanguage{greek}{ποδεκατουτε το ηδυοϲμον και το} & 11 &  &  \\
&  & 12 & \foreignlanguage{greek}{πηγανον και παν λαχανον και} & 16 &  &  \\
&  & 17 & \foreignlanguage{greek}{παρερχεϲθαι την κριϲιν και την α} & 22 &  &  \\
&  & 22 & \foreignlanguage{greek}{γαπην του \textoverline{θυ} ταυτα εδει ποιηϲαι} & 27 &  &  \\
&  & 28 & \foreignlanguage{greek}{κακεινα μη αφιεναι} & 30 &  &  \\
& \textbf{43} &  & \foreignlanguage{greek}{ουαι υμιν τοιϲ φαριϲαιοιϲ οτι αγαπα} & 6 &  &  \\
&  & 6 & \foreignlanguage{greek}{ται την πρωτοκαθεδριαν εν ταιϲ} & 10 &  &  \\
&  & 11 & \foreignlanguage{greek}{ϲυναγωγαιϲ και τουϲ αϲπαϲμουϲ} & 14 &  &  \\
&  & 15 & \foreignlanguage{greek}{εν ταιϲ αγοραιϲ ουαι υμιν γραμ} & 3 & \textbf{44} &  \\
&  & 3 & \foreignlanguage{greek}{ματειϲ και φαριϲαιοι υποκριται} & 6 &  &  \\
[0.2em]
\cline{4-4}
\end{tabular}
\end{center}
\end{table}
}
\clearpage
\newpage
 {
 \setlength\arrayrulewidth{1pt}
\begin{table}
\begin{center}
\begin{tabular}{ccc|l|ccc}
\cline{4-4} \\ [-1em]
\multicolumn{7}{c}{\foreignlanguage{greek}{ευαγγελιον κατα λουκαν} \textbf{(\nospace{11:44})} } \\ \\ [-1em] % Si on veut ajouter les bordures latérales, remplacer {7}{c} par {7}{|c|}
\cline{4-4} \\
\cline{4-4}
&  &  & &  &  & \\ [-0.9em]
&  & 7 & \foreignlanguage{greek}{οτι εϲται ωϲ μνημια τα αδηλα και οι} & 14 &  &  \\
&  & 15 & \foreignlanguage{greek}{\textoverline{ανοι} περιπατουντεϲ επανω ουκ οιδαϲιν} & 19 &  &  \\
& \textbf{45} &  & \foreignlanguage{greek}{αποκριθειϲ δε τιϲ των νομικων λεγει} & 6 &  &  \\
&  & 7 & \foreignlanguage{greek}{αυτω διδαϲκαλε ταυτα λεγων και} & 11 &  &  \\
&  & 12 & \foreignlanguage{greek}{ημαϲ υβριζειϲ ο δε ειπεν} & 3 & \textbf{46} &  \\
&  & 4 & \foreignlanguage{greek}{και υμιν τοιϲ νομικοιϲ ουαι οτι φορ} & 10 &  &  \\
&  & 10 & \foreignlanguage{greek}{τιζεται τουϲ ανθρωπουϲ φορτια δυϲ} & 14 &  &  \\
&  & 14 & \foreignlanguage{greek}{βαϲτακτα και αυτοι ενι των δακτυ} & 19 &  &  \\
&  & 19 & \foreignlanguage{greek}{λων υμων ου προϲψαυεται τοιϲ φορ} & 24 &  &  \\
&  & 24 & \foreignlanguage{greek}{τιοιϲ} & 24 &  &  \\
& \textbf{47} &  & \foreignlanguage{greek}{ουαι υμιν οτι οικοδομειται τα μνημια} & 6 &  &  \\
&  & 7 & \foreignlanguage{greek}{των προφητων οι δε πατερεϲ υμω̅} & 12 &  &  \\
&  & 13 & \foreignlanguage{greek}{απεκτιναν αυτουϲ αρα μαρτυρει} & 2 & \textbf{48} &  \\
&  & 2 & \foreignlanguage{greek}{τε και ϲυνευδοκειται τοιϲ εργοιϲ τω̅} & 7 &  &  \\
&  & 8 & \foreignlanguage{greek}{πατερων υμων οτι αυτοι μεν απε} & 13 &  &  \\
&  & 13 & \foreignlanguage{greek}{κτιναν αυτουϲ υμειϲ δε οικοδομει} & 17 &  &  \\
&  & 17 & \foreignlanguage{greek}{ται αυτων τα μνημια} & 20 &  &  \\
& \textbf{49} &  & \foreignlanguage{greek}{δια τουτο και η ϲοφια του \textoverline{θυ} ειπεν} & 8 &  &  \\
&  & 9 & \foreignlanguage{greek}{αποϲτελω ειϲ αυτουϲ προφηταϲ και} & 13 &  &  \\
&  & 14 & \foreignlanguage{greek}{αποϲτολουϲ εξ αυτων αποκτιενουϲι̅} & 17 &  &  \\
&  & 19 & \foreignlanguage{greek}{εκδιωξουϲιν ινα εκζητηθη το αι} & 4 & \textbf{50} &  \\
&  & 4 & \foreignlanguage{greek}{μα παντων των προφητων το εκ} & 9 &  &  \\
&  & 9 & \foreignlanguage{greek}{χυννομενον απο καταβοληϲ κοϲμου} & 12 &  &  \\
&  & 13 & \foreignlanguage{greek}{απο τηϲ γενεαϲ ταυτηϲ απο του αι} & 3 & \textbf{51} &  \\
&  & 3 & \foreignlanguage{greek}{ματοϲ αβελ εωϲ του αιματοϲ ζαχαρι} & 8 &  &  \\
&  & 8 & \foreignlanguage{greek}{ου του απολομενου μεταξυ του θυ} & 13 &  &  \\
&  & 13 & \foreignlanguage{greek}{ϲιαϲτηριου και του οικου} & 16 &  &  \\
&  & 17 & \foreignlanguage{greek}{ναι λεγω υμιν εκζητηθηϲεται απο} & 21 &  &  \\
&  & 22 & \foreignlanguage{greek}{τηϲ γενεαϲ ταυτηϲ} & 24 &  &  \\
& \textbf{52} &  & \foreignlanguage{greek}{ουαι υμιν τοιϲ νομικοιϲ οτι ηρατε την} & 7 &  &  \\
[0.2em]
\cline{4-4}
\end{tabular}
\end{center}
\end{table}
}
\clearpage
\newpage
 {
 \setlength\arrayrulewidth{1pt}
\begin{table}
\begin{center}
\begin{tabular}{ccc|l|ccc}
\cline{4-4} \\ [-1em]
\multicolumn{7}{c}{\foreignlanguage{greek}{ευαγγελιον κατα λουκαν} \textbf{(\nospace{11:52})} } \\ \\ [-1em] % Si on veut ajouter les bordures latérales, remplacer {7}{c} par {7}{|c|}
\cline{4-4} \\
\cline{4-4}
&  &  & &  &  & \\ [-0.9em]
&  & 8 & \foreignlanguage{greek}{κλειδα τηϲ γνωϲεωϲ αυτοι ουκ ειϲηλ} & 13 &  &  \\
&  & 13 & \foreignlanguage{greek}{θατε και τουϲ ειϲερχομενουϲ εκωλυ} & 17 &  &  \\
&  & 17 & \foreignlanguage{greek}{ϲατε λεγοντοϲ δε αυτου ταυτα} & 4 & \textbf{53} &  \\
&  & 5 & \foreignlanguage{greek}{προϲ αυτουϲ ηρξαντο οι γραμματιϲ} & 9 &  &  \\
&  & 10 & \foreignlanguage{greek}{και οι φαριϲαιοι δεινωϲ ενεχειν και} & 15 &  &  \\
&  & 16 & \foreignlanguage{greek}{αποϲτοματιζειν αυτον περι πλει} & 19 &  &  \\
&  & 19 & \foreignlanguage{greek}{ονων ενεδρευοντεϲ αυτον ζητουν} & 3 & \textbf{54} &  \\
&  & 3 & \foreignlanguage{greek}{τεϲ θηρευϲαι τι εκ του ϲτοματοϲ αυτου} & 9 &  &  \\
&  & 10 & \foreignlanguage{greek}{ινα κατηγορηϲωϲιν αυτου} & 13 &  &  \\
& \mygospelchapter &  & \foreignlanguage{greek}{εν οιϲ επιϲυναχθιϲων των μυριαδω̅} & 5 &  &  \\
&  & 6 & \foreignlanguage{greek}{του οχλου ωϲτε καταπατειν αλληλουϲ} & 10 &  &  \\
&  & 11 & \foreignlanguage{greek}{ηρξατο λεγειν προϲ τουϲ μαθηταϲ αυτου} & 16 &  &  \\
&  & 17 & \foreignlanguage{greek}{πρωτον προϲεχεται εαυτοιϲ απο τηϲ} & 21 &  &  \\
&  & 22 & \foreignlanguage{greek}{ζυμηϲ των φαριϲαιων ητιϲ εϲτιν} & 26 &  &  \\
&  & 27 & \foreignlanguage{greek}{υποκριϲειϲ ουδεν δε ϲυνκεκαλυμμενον εϲτι̅} & 4 & \textbf{2} &  \\
&  & 5 & \foreignlanguage{greek}{ο ουκ αποκαλυφθηϲεται και κρυ} & 9 &  &  \\
&  & 9 & \foreignlanguage{greek}{πτον ο ου γνωϲθηϲεται ανθ ων ο} & 3 & \textbf{3} &  \\
&  & 3 & \foreignlanguage{greek}{ϲα εν τη ϲκοτια ειπατε εν τω φωτι} & 10 &  &  \\
&  & 11 & \foreignlanguage{greek}{ακουϲθηϲεται και ο προϲ το ουϲ ε} & 17 &  &  \\
&  & 17 & \foreignlanguage{greek}{λαληϲατε εν τοιϲ ταμιοιϲ κηρυχθη} & 21 &  &  \\
&  & 21 & \foreignlanguage{greek}{ϲεται επι των δωματων} & 24 &  &  \\
& \textbf{4} &  & \foreignlanguage{greek}{λεγω δε υμιν τοιϲ φιλοιϲ μου μη φο} & 8 &  &  \\
&  & 8 & \foreignlanguage{greek}{βηθηται απο των αποκτενοντων} & 11 &  &  \\
&  & 12 & \foreignlanguage{greek}{το ϲωμα και μετα ταυτα μη εχοντω̅} & 18 &  &  \\
&  & 19 & \foreignlanguage{greek}{περιϲϲοτερον τι ποιηϲαι} & 21 &  &  \\
& \textbf{5} &  & \foreignlanguage{greek}{υποδειξω δε υμιν τινα φοβηθητε} & 5 &  &  \\
&  & 6 & \foreignlanguage{greek}{φοβηθητε τον μετα το αποκτιναι} & 10 &  &  \\
&  & 11 & \foreignlanguage{greek}{εχοντα εξουϲιαν βαλιν ειϲ την γεεννα̅} & 16 &  &  \\
&  & 17 & \foreignlanguage{greek}{ναι λεγω υμιν τουτον φοβηθητε} & 21 &  &  \\
[0.2em]
\cline{4-4}
\end{tabular}
\end{center}
\end{table}
}
\clearpage
\newpage
 {
 \setlength\arrayrulewidth{1pt}
\begin{table}
\begin{center}
\begin{tabular}{ccc|l|ccc}
\cline{4-4} \\ [-1em]
\multicolumn{7}{c}{\foreignlanguage{greek}{ευαγγελιον κατα λουκαν} \textbf{(\nospace{12:6})} } \\ \\ [-1em] % Si on veut ajouter les bordures latérales, remplacer {7}{c} par {7}{|c|}
\cline{4-4} \\
\cline{4-4}
&  &  & &  &  & \\ [-0.9em]
& \textbf{6} &  & \foreignlanguage{greek}{ουχι πεντε ϲτρουθεια πωλειται δυο} & 5 &  &  \\
&  & 6 & \foreignlanguage{greek}{αϲϲαριων και εν εξ αυτων ουκ εϲτιν} & 12 &  &  \\
&  & 13 & \foreignlanguage{greek}{επιλεληϲμενον ενωπιον του \textoverline{θυ}} & 16 &  &  \\
& \textbf{7} &  & \foreignlanguage{greek}{αλλα και αι τριχεϲ τηϲ κεφαληϲ υμων} & 7 &  &  \\
&  & 8 & \foreignlanguage{greek}{παϲαι ηριθμηνται μη ουν φοβιϲθαι} & 12 &  &  \\
&  & 13 & \foreignlanguage{greek}{πολλων ϲτρουθιων διαφερετε} & 15 &  &  \\
& \textbf{8} &  & \foreignlanguage{greek}{λεγω δε υμιν παϲ οϲ αν ομολογηϲη εν} & 8 &  &  \\
&  & 9 & \foreignlanguage{greek}{εμοι εμπροϲθεν των ανθρωπων} & 12 &  &  \\
&  & 13 & \foreignlanguage{greek}{και ο υιοϲ του ανθρωπου ομολογηϲει} & 18 &  &  \\
&  & 19 & \foreignlanguage{greek}{εν αυτω εμπροϲθεν των αγγελων} & 23 &  &  \\
&  & 24 & \foreignlanguage{greek}{του \textoverline{θυ} ο δε αρνηϲαμενοϲ με ενω} & 5 & \textbf{9} &  \\
&  & 5 & \foreignlanguage{greek}{πιον των ανθρωπων απαρνηθηϲε} & 8 &  &  \\
&  & 8 & \foreignlanguage{greek}{ται ενωπιον των αγγελων του \textoverline{θυ}} & 13 &  &  \\
& \textbf{10} &  & \foreignlanguage{greek}{και παϲ οϲ ερει λογον ειϲ τον υιον του} & 9 &  &  \\
&  & 10 & \foreignlanguage{greek}{ανθρωπου αφεθηϲεται αυτω} & 12 &  &  \\
&  & 13 & \foreignlanguage{greek}{τω δε ειϲ το αγιον \textoverline{πνα} βλαϲφημηϲα̅} & 19 &  &  \\
&  & 19 & \foreignlanguage{greek}{τι ουκ αφεθηϲεται} & 21 &  &  \\
& \textbf{11} &  & \foreignlanguage{greek}{οταν δε προϲφερωϲιν υμαϲ επι ταϲ} & 6 &  &  \\
&  & 7 & \foreignlanguage{greek}{ϲυναγωγαϲ και ταϲ αρχαϲ και ταϲ ε} & 13 &  &  \\
&  & 13 & \foreignlanguage{greek}{ξουϲιαϲ μη μεριμνατε πωϲ η τι} & 18 &  &  \\
&  & 19 & \foreignlanguage{greek}{απολογηϲεϲθαι η τι ειπηται} & 22 &  &  \\
& \textbf{12} &  & \foreignlanguage{greek}{το γαρ αγιον \textoverline{πνα} διδαξει υμαϲ εν αυ} & 8 &  &  \\
&  & 8 & \foreignlanguage{greek}{τη τη ωρα α δει ειπειν} & 13 &  &  \\
& \textbf{13} &  & \foreignlanguage{greek}{ειπεν δε τιϲ αυτω εκ του οχλου δι} & 8 &  &  \\
&  & 8 & \foreignlanguage{greek}{δαϲκαλε ειπε τω αδελφω μου με} & 13 &  &  \\
&  & 13 & \foreignlanguage{greek}{ριϲαϲθαι μετ εμου την κληρονομιαν} & 17 &  &  \\
& \textbf{14} &  & \foreignlanguage{greek}{ο δε ειπεν αυτω ανθρωπε τιϲ με κα} & 8 &  &  \\
&  & 8 & \foreignlanguage{greek}{τεϲτηϲεν δικαϲτην η μεριϲτην} & 11 &  &  \\
&  & 12 & \foreignlanguage{greek}{εφ υμαϲ ειπεν δε προϲ αυτουϲ} & 4 & \textbf{15} &  \\
&  & 5 & \foreignlanguage{greek}{ορατε και φυλαϲϲεϲθαι απο παϲηϲ} & 9 &  &  \\
[0.2em]
\cline{4-4}
\end{tabular}
\end{center}
\end{table}
}
\clearpage
\newpage
 {
 \setlength\arrayrulewidth{1pt}
\begin{table}
\begin{center}
\begin{tabular}{ccc|l|ccc}
\cline{4-4} \\ [-1em]
\multicolumn{7}{c}{\foreignlanguage{greek}{ευαγγελιον κατα λουκαν} \textbf{(\nospace{12:15})} } \\ \\ [-1em] % Si on veut ajouter les bordures latérales, remplacer {7}{c} par {7}{|c|}
\cline{4-4} \\
\cline{4-4}
&  &  & &  &  & \\ [-0.9em]
&  & 10 & \foreignlanguage{greek}{πλεονεξιαϲ οτι ουκ εν τω περιϲϲευει̅} & 15 &  &  \\
&  & 16 & \foreignlanguage{greek}{τινι η ζωη αυτων εϲτιν εκ των υ} & 23 &  &  \\
&  & 23 & \foreignlanguage{greek}{παρχοντων αυτων} & 24 &  &  \\
& \textbf{16} &  & \foreignlanguage{greek}{ειπεν δε παραβολην προϲ αυτουϲ λεγω̅} & 6 &  &  \\
&  & 7 & \foreignlanguage{greek}{ανθρωπου τινοϲ πλουϲιου ηυφορη} & 10 &  &  \\
&  & 10 & \foreignlanguage{greek}{ϲεν η χωρα και διελογιζετο εν εαυτω} & 4 & \textbf{17} &  \\
&  & 5 & \foreignlanguage{greek}{λεγων τι ποιηϲω οτι ουκ εχω που ϲυ̅} & 12 &  &  \\
&  & 12 & \foreignlanguage{greek}{αξω τουϲ καρπουϲ μου και ειπεν} & 2 & \textbf{18} &  \\
&  & 3 & \foreignlanguage{greek}{τουτο ποιηϲω καθελω ταϲ αποθηκαϲ} & 7 &  &  \\
&  & 8 & \foreignlanguage{greek}{και μειζοναϲ οικοδομηϲω και ϲυν} & 12 &  &  \\
&  & 12 & \foreignlanguage{greek}{αξω εκει παντα τα γενηματα μου} & 17 &  &  \\
&  & 18 & \foreignlanguage{greek}{και τα αγαθα μου και ερω τη ψυχη μου} & 5 & \textbf{19} &  \\
&  & 6 & \foreignlanguage{greek}{ϲυ εχειϲ πολλα αγαθα κειμενα ειϲ ε} & 12 &  &  \\
&  & 12 & \foreignlanguage{greek}{τη πολλα αναπαυου φαγε πιε ευ} & 17 &  &  \\
&  & 17 & \foreignlanguage{greek}{φραινου ειπεν δε αυτω ο \textoverline{θϲ}} & 5 & \textbf{20} &  \\
&  & 6 & \foreignlanguage{greek}{αφρον ταυτη τη νυκτι την ψυχην} & 11 &  &  \\
&  & 12 & \foreignlanguage{greek}{ϲου απαιτουϲιν απο ϲου α δε ητοι} & 18 &  &  \\
&  & 18 & \foreignlanguage{greek}{μαϲαϲ τινι εϲται ουτωϲ ο θηϲαυρι} & 3 & \textbf{21} &  \\
&  & 3 & \foreignlanguage{greek}{ζων εν εαυτω και μη ειϲ \textoverline{θν} πλουτω̅} & 10 &  &  \\
& \textbf{22} &  & \foreignlanguage{greek}{ειπεν δε προϲ τουϲ μαθηταϲ αυτου} & 6 &  &  \\
&  & 7 & \foreignlanguage{greek}{δια τουτο υμιν λεγω μη μεριμναται} & 12 &  &  \\
&  & 13 & \foreignlanguage{greek}{τη ψυχη τι φαγηται μηδε τω ϲωμα} & 19 &  &  \\
&  & 19 & \foreignlanguage{greek}{τι τι ενδυϲηϲθαι η ψυχη πλιον εϲτι̅} & 4 & \textbf{23} &  \\
&  & 5 & \foreignlanguage{greek}{τηϲ τροφηϲ και το ϲωμα του ενδυ} & 11 &  &  \\
&  & 11 & \foreignlanguage{greek}{ματοϲ κατανοηϲατε τουϲ κορα} & 3 & \textbf{24} &  \\
&  & 3 & \foreignlanguage{greek}{καϲ οτι ου ϲπειρουϲιν ουδε θεριζουϲι̅} & 8 &  &  \\
&  & 9 & \foreignlanguage{greek}{οιϲ ουκ εϲτιν ταμιον ουδε αποθηκη} & 14 &  &  \\
&  & 15 & \foreignlanguage{greek}{και ο \textoverline{θϲ} τρεφει αυτουϲ ποϲω μαλλο̅} & 21 &  &  \\
&  & 22 & \foreignlanguage{greek}{υμειϲ διαφερετε των πετινων} & 25 &  &  \\
& \textbf{25} &  & \foreignlanguage{greek}{τιϲ δε εξ υμων μεριμνων δυναται} & 6 &  &  \\
[0.2em]
\cline{4-4}
\end{tabular}
\end{center}
\end{table}
}
\clearpage
\newpage
 {
 \setlength\arrayrulewidth{1pt}
\begin{table}
\begin{center}
\begin{tabular}{ccc|l|ccc}
\cline{4-4} \\ [-1em]
\multicolumn{7}{c}{\foreignlanguage{greek}{ευαγγελιον κατα λουκαν} \textbf{(\nospace{12:25})} } \\ \\ [-1em] % Si on veut ajouter les bordures latérales, remplacer {7}{c} par {7}{|c|}
\cline{4-4} \\
\cline{4-4}
&  &  & &  &  & \\ [-0.9em]
&  & 7 & \foreignlanguage{greek}{προϲθειναι επι την ηλικειαν αυτου} & 11 &  &  \\
&  & 12 & \foreignlanguage{greek}{πηχυν ενα ει ουν ουτε ελαχιϲτον δυ} & 5 & \textbf{26} &  \\
&  & 5 & \foreignlanguage{greek}{ναϲθαι τι περι των λοιπων μεριμνατε} & 10 &  &  \\
& \textbf{27} &  & \foreignlanguage{greek}{κατανοηϲατε τα κρινα πωϲ αυξανει} & 5 &  &  \\
&  & 6 & \foreignlanguage{greek}{ου κοπια ουδε νηθει λεγω δε υμιν} & 12 &  &  \\
&  & 13 & \foreignlanguage{greek}{ουδε ϲολομων εν παϲη τη δοξη αυτου} & 19 &  &  \\
&  & 20 & \foreignlanguage{greek}{περιεβαλετο ωϲ εν τουτων} & 23 &  &  \\
& \textbf{28} &  & \foreignlanguage{greek}{ει δε τον χορτον ϲημερον εν αγρω} & 7 &  &  \\
&  & 8 & \foreignlanguage{greek}{οντα και αυριον ειϲ κλειβανον βαλλο} & 13 &  &  \\
&  & 13 & \foreignlanguage{greek}{μενον ο \textoverline{θϲ} ουτωϲ αμφιεννυϲιν πο} & 18 &  &  \\
&  & 18 & \foreignlanguage{greek}{ϲω μαλλον υμαϲ ολιγοπιϲτοι} & 21 &  &  \\
& \textbf{29} &  & \foreignlanguage{greek}{και υμειϲ μη ζητειτε τι φαγηται η τι} & 8 &  &  \\
&  & 9 & \foreignlanguage{greek}{πιηται και μη μετεωριζεται ταυ} & 1 & \textbf{30} &  \\
&  & 1 & \foreignlanguage{greek}{τα γαρ παντα τα εθνη του κοϲμου επιζητει} & 8 &  &  \\
&  & 9 & \foreignlanguage{greek}{υμων δε ο \textoverline{πηρ} οιδεν οτι χρηζετε} & 15 &  &  \\
&  & 16 & \foreignlanguage{greek}{τουτων πλην ζητειτε την βαϲιλει} & 4 & \textbf{31} &  \\
&  & 4 & \foreignlanguage{greek}{αν του \textoverline{θυ} και προϲτεθηϲεται υμιν} & 9 &  &  \\
& \textbf{32} &  & \foreignlanguage{greek}{μη φοβου το μικρον ποιμνιον οτι ηυ} & 7 &  &  \\
&  & 7 & \foreignlanguage{greek}{δοκηϲεν ο \textoverline{πηρ} υμων δουναι υμιν} & 12 &  &  \\
&  & 13 & \foreignlanguage{greek}{την βαϲιλειαν πωληϲατε τα υπαρχοντα υμων} & 4 & \textbf{33} &  \\
&  & 5 & \foreignlanguage{greek}{και δοτε ελεημοϲυνην} & 7 &  &  \\
&  & 8 & \foreignlanguage{greek}{ποιηϲατε εαυτοιϲ βαλλαντια μη} & 11 &  &  \\
&  & 12 & \foreignlanguage{greek}{παλαιουμενα θηϲαυρον ανεκ} & 14 &  &  \\
&  & 14 & \foreignlanguage{greek}{λιπτον εν τοιϲ ουρανοιϲ} & 17 &  &  \\
&  & 18 & \foreignlanguage{greek}{οπου κλεπτηϲ ουκ ενγιζει ου} & 22 &  &  \\
&  & 22 & \foreignlanguage{greek}{δε ϲηϲ διαφθειρει οπου γαρ εϲτιν} & 3 & \textbf{34} &  \\
&  & 4 & \foreignlanguage{greek}{ο θηϲαυροϲ υμων εκει και η καρδια} & 10 &  &  \\
&  & 11 & \foreignlanguage{greek}{υμων εϲται} & 12 &  &  \\
& \textbf{35} &  & \foreignlanguage{greek}{εϲτωϲαν υμων αι οϲφυεϲ περιεζωϲμεναι} & 5 &  &  \\
[0.2em]
\cline{4-4}
\end{tabular}
\end{center}
\end{table}
}
\clearpage
\newpage
 {
 \setlength\arrayrulewidth{1pt}
\begin{table}
\begin{center}
\begin{tabular}{ccc|l|ccc}
\cline{4-4} \\ [-1em]
\multicolumn{7}{c}{\foreignlanguage{greek}{ευαγγελιον κατα λουκαν} \textbf{(\nospace{12:35})} } \\ \\ [-1em] % Si on veut ajouter les bordures latérales, remplacer {7}{c} par {7}{|c|}
\cline{4-4} \\
\cline{4-4}
&  &  & &  &  & \\ [-0.9em]
&  & 6 & \foreignlanguage{greek}{και οι λυχνοι καιομενοι και υμειϲ ο} & 3 & \textbf{36} &  \\
&  & 3 & \foreignlanguage{greek}{μοιοι ανθρωποιϲ προϲδεχομενοιϲ} & 5 &  &  \\
&  & 6 & \foreignlanguage{greek}{τον \textoverline{κν} αυτων ποτε αναλυϲει εκ τω̅} & 12 &  &  \\
&  & 13 & \foreignlanguage{greek}{γαμων ινα ελθοντοϲ και κρουϲαν} & 17 &  &  \\
&  & 17 & \foreignlanguage{greek}{τοϲ ευθεωϲ ανοιξωϲιν αυτω} & 20 &  &  \\
& \textbf{37} &  & \foreignlanguage{greek}{μακαριοι οι δουλοι εκεινοι ουϲ ελ} & 6 &  &  \\
&  & 6 & \foreignlanguage{greek}{θων ο \textoverline{κϲ} ευρηϲει γρηγορουνταϲ} & 10 &  &  \\
&  & 11 & \foreignlanguage{greek}{αμην λεγω υμιν οτι περιζωϲεται} & 15 &  &  \\
&  & 16 & \foreignlanguage{greek}{και ανακλινει αυτουϲ και παρελ} & 20 &  &  \\
&  & 20 & \foreignlanguage{greek}{θων διακονηϲει αυτοιϲ} & 22 &  &  \\
& \textbf{38} &  & \foreignlanguage{greek}{και εαν εν τη τριτη φυλακη ελθη ϗ} & 8 &  &  \\
&  & 9 & \foreignlanguage{greek}{ευρη ουτωϲ μακαριοι ειϲιν οι δου} & 14 &  &  \\
&  & 14 & \foreignlanguage{greek}{λοι εκεινοι} & 15 &  &  \\
& \textbf{39} &  & \foreignlanguage{greek}{τουτο δε γινωϲκεται οτι ει ηδει ο οι} & 8 &  &  \\
&  & 8 & \foreignlanguage{greek}{κοδεϲποτηϲ ποια ωρα ο κλεπτηϲ} & 12 &  &  \\
&  & 13 & \foreignlanguage{greek}{ερχεται εγρηγορηϲεν αν και ου} & 17 &  &  \\
&  & 17 & \foreignlanguage{greek}{κ αφηκεν διορυγηναι τον οικον αυτου} & 22 &  &  \\
& \textbf{40} &  & \foreignlanguage{greek}{και υμειϲ ουν γεινεϲθαι ετοιμοι} & 5 &  &  \\
&  & 6 & \foreignlanguage{greek}{οτι η ωρα ου δοκειται ο υιοϲ του αν} & 14 &  &  \\
&  & 14 & \foreignlanguage{greek}{θρωπου ερχεται} & 15 &  &  \\
& \textbf{41} &  & \foreignlanguage{greek}{ειπεν δε αυτω ο πετροϲ \textoverline{κε} προϲ η} & 8 &  &  \\
&  & 8 & \foreignlanguage{greek}{μαϲ την παραβολην ταυτην λεγειϲ} & 12 &  &  \\
&  & 13 & \foreignlanguage{greek}{η και προϲ πανταϲ} & 16 &  &  \\
& \textbf{42} &  & \foreignlanguage{greek}{ειπεν δε ο \textoverline{κϲ} τιϲ αρα εϲτιν ο πιϲτοϲ} & 9 &  &  \\
&  & 10 & \foreignlanguage{greek}{οικονομοϲ ο φρονιμοϲ ον κατα} & 14 &  &  \\
&  & 14 & \foreignlanguage{greek}{ϲτηϲει ο \textoverline{κϲ} επι τηϲ θεραπιαϲ αυ} & 20 &  &  \\
&  & 20 & \foreignlanguage{greek}{του δουναι εν καιρω το ϲιτομε} & 25 &  &  \\
&  & 25 & \foreignlanguage{greek}{τριον μακαριοϲ ο δουλοϲ ε} & 4 & \textbf{43} &  \\
&  & 4 & \foreignlanguage{greek}{κεινοϲ ον ελθων ο \textoverline{κϲ} αυτου ευρη} & 10 &  &  \\
&  & 10 & \foreignlanguage{greek}{ϲει ποιουντα ουτωϲ} & 12 &  &  \\
[0.2em]
\cline{4-4}
\end{tabular}
\end{center}
\end{table}
}
\clearpage
\newpage
 {
 \setlength\arrayrulewidth{1pt}
\begin{table}
\begin{center}
\begin{tabular}{ccc|l|ccc}
\cline{4-4} \\ [-1em]
\multicolumn{7}{c}{\foreignlanguage{greek}{ευαγγελιον κατα λουκαν} \textbf{(\nospace{12:44})} } \\ \\ [-1em] % Si on veut ajouter les bordures latérales, remplacer {7}{c} par {7}{|c|}
\cline{4-4} \\
\cline{4-4}
&  &  & &  &  & \\ [-0.9em]
& \textbf{44} &  & \foreignlanguage{greek}{αληθωϲ λεγω υμιν επι παϲιν τοιϲ υπαρ} & 7 &  &  \\
&  & 7 & \foreignlanguage{greek}{χουϲιν αυτω καταϲτηϲει αυτον} & 10 &  &  \\
& \textbf{45} &  & \foreignlanguage{greek}{εαν δε ειπη ο δουλοϲ εκεινοϲ εν τη καρ} & 9 &  &  \\
&  & 9 & \foreignlanguage{greek}{δια αυτου χρονιζει ο \textoverline{κϲ} μου ερχεϲθαι} & 15 &  &  \\
&  & 16 & \foreignlanguage{greek}{και αρξηται τυπτειν τουϲ παιδαϲ και} & 21 &  &  \\
&  & 22 & \foreignlanguage{greek}{ταϲ παιδιϲκαϲ αιϲθιειν τε και πινει̅} & 27 &  &  \\
&  & 28 & \foreignlanguage{greek}{και μεθυϲκεϲθαι ηξει ο \textoverline{κϲ} του δου} & 5 & \textbf{46} &  \\
&  & 5 & \foreignlanguage{greek}{λου εκεινου εν ημερα η ου προϲδοκα} & 11 &  &  \\
&  & 12 & \foreignlanguage{greek}{και εν ωρα η ου γιγνωϲκει και διχο} & 19 &  &  \\
&  & 19 & \foreignlanguage{greek}{τομηϲει αυτον και το μεροϲ αυτου} & 24 &  &  \\
&  & 25 & \foreignlanguage{greek}{μετα των απιϲτων θηϲει} & 28 &  &  \\
& \textbf{47} &  & \foreignlanguage{greek}{εκεινοϲ δε ο δουλοϲ ο γνουϲ το θελη} & 8 &  &  \\
&  & 8 & \foreignlanguage{greek}{μα του \textoverline{κυ} εαυτου και μη ετοιμαϲαϲ} & 15 &  &  \\
&  & 16 & \foreignlanguage{greek}{προϲ το θελημα αυτου δαρηϲεται} & 20 &  &  \\
&  & 21 & \foreignlanguage{greek}{πολλαϲ ο δε μη γνουϲ ποιηϲαϲ δε} & 6 & \textbf{48} &  \\
&  & 7 & \foreignlanguage{greek}{αξια πληγων δαρηϲεται ολειγα} & 10 &  &  \\
&  & 11 & \foreignlanguage{greek}{παντι δε ω εδοθη το πολυ πολυ ζη} & 18 &  &  \\
&  & 18 & \foreignlanguage{greek}{τηθηϲεται παρ αυτου και ω παρε} & 23 &  &  \\
&  & 23 & \foreignlanguage{greek}{θεντο το πολυ περιϲϲοτερον αιτη} & 27 &  &  \\
&  & 27 & \foreignlanguage{greek}{ϲουϲιν αυτον} & 28 &  &  \\
& \textbf{49} &  & \foreignlanguage{greek}{πυρ ηλθον βαλιν ειϲ την γην και} & 7 &  &  \\
&  & 8 & \foreignlanguage{greek}{τι θελω ει ηδη ανηφθη} & 12 &  &  \\
& \textbf{50} &  & \foreignlanguage{greek}{βαπτιϲμα δε εχω βαπτιϲθηναι και} & 5 &  &  \\
&  & 6 & \foreignlanguage{greek}{πωϲ ϲυνεχομαι εωϲ οπου τελεϲθη} & 10 &  &  \\
& \textbf{51} &  & \foreignlanguage{greek}{δοκειται οτι ειρηνην παρεγενομη̅} & 4 &  &  \\
&  & 5 & \foreignlanguage{greek}{δουναι εν τη γη ουχι λεγω υμιν} & 11 &  &  \\
&  & 12 & \foreignlanguage{greek}{αλλ η διαμεριϲμον εϲονται γαρ} & 2 & \textbf{52} &  \\
&  & 3 & \foreignlanguage{greek}{απο του νυν πεντε εν οικω ενι δι} & 10 &  &  \\
&  & 10 & \foreignlanguage{greek}{αμεμεριϲμενοι τριϲ επι δυϲιν και} & 14 &  &  \\
&  & 15 & \foreignlanguage{greek}{δυο επι τριϲιν} & 17 &  &  \\
[0.2em]
\cline{4-4}
\end{tabular}
\end{center}
\end{table}
}
\clearpage
\newpage
 {
 \setlength\arrayrulewidth{1pt}
\begin{table}
\begin{center}
\begin{tabular}{ccc|l|ccc}
\cline{4-4} \\ [-1em]
\multicolumn{7}{c}{\foreignlanguage{greek}{ευαγγελιον κατα λουκαν} \textbf{(\nospace{12:53})} } \\ \\ [-1em] % Si on veut ajouter les bordures latérales, remplacer {7}{c} par {7}{|c|}
\cline{4-4} \\
\cline{4-4}
&  &  & &  &  & \\ [-0.9em]
& \textbf{53} &  & \foreignlanguage{greek}{διαμεριϲθηϲονται \textoverline{πηρ} επι υιω και υιοϲ} & 6 &  &  \\
&  & 7 & \foreignlanguage{greek}{επι \textoverline{πρι} μητηρ επι θυγατρι πενθερα} & 12 &  &  \\
&  & 13 & \foreignlanguage{greek}{επι την νυμφην αυτηϲ και νυμφη} & 18 &  &  \\
&  & 19 & \foreignlanguage{greek}{επι την πενθεραν αυτηϲ} & 22 &  &  \\
& \textbf{54} &  & \foreignlanguage{greek}{ελεγεν δε και τοιϲ οχλοιϲ οταν ειδη} & 7 &  &  \\
&  & 7 & \foreignlanguage{greek}{ται την νεφελην ανατελλουϲαν απο} & 11 &  &  \\
&  & 12 & \foreignlanguage{greek}{δυϲμων ευθεωϲ λεγεται ομβροϲ ερ} & 16 &  &  \\
&  & 16 & \foreignlanguage{greek}{χεται και γεινεται ουτωϲ} & 19 &  &  \\
& \textbf{55} &  & \foreignlanguage{greek}{και οταν νοτον πνεοντα λεγεται ο} & 6 &  &  \\
&  & 6 & \foreignlanguage{greek}{τι καυϲων ερχεται και γεινεται} & 10 &  &  \\
& \textbf{56} &  & \foreignlanguage{greek}{υποκρειται το προϲωπον τηϲ γηϲ} & 5 &  &  \\
&  & 6 & \foreignlanguage{greek}{του ουρανου οιδατε δοκιμαζειν} & 9 &  &  \\
&  & 10 & \foreignlanguage{greek}{τον δε καιρον τουτον πωϲ ου δοκι} & 16 &  &  \\
&  & 16 & \foreignlanguage{greek}{μαζεται} & 16 &  &  \\
& \textbf{57} &  & \foreignlanguage{greek}{τι δε και αφ εαυτων ου κρεινεται το} & 8 &  &  \\
&  & 9 & \foreignlanguage{greek}{δικαιον ωϲ γαρ υπαγειϲ μετα του α̅} & 6 & \textbf{58} &  \\
&  & 6 & \foreignlanguage{greek}{τιδικου ϲου επ αρχοντα εν τη οδω} & 12 &  &  \\
&  & 13 & \foreignlanguage{greek}{δοϲ εργαϲιαν απηλλαχθαι απ αυτου} & 17 &  &  \\
&  & 18 & \foreignlanguage{greek}{μηποτε καταϲυρη ϲε προϲ τον κριτη̅} & 23 &  &  \\
&  & 24 & \foreignlanguage{greek}{και ο κριτηϲ ϲε παραδω τω πρακτορι} & 30 &  &  \\
&  & 31 & \foreignlanguage{greek}{και ο πρακτωρ ϲε βαλη ειϲ φυλακην} & 37 &  &  \\
& \textbf{59} &  & \foreignlanguage{greek}{λεγω ϲοι ου μη εξελθηϲ εκειθεν εωϲ} & 7 &  &  \\
&  & 8 & \foreignlanguage{greek}{ου και το εϲχατον λεπτον αποδωϲ} & 13 &  &  \\
& \mygospelchapter &  & \foreignlanguage{greek}{παρηϲαν δε τινεϲ εν αυτω τω καιρω} & 7 &  &  \\
&  & 8 & \foreignlanguage{greek}{απαγγελλοντεϲ αυτω περι των γαλι} & 12 &  &  \\
&  & 12 & \foreignlanguage{greek}{λαιων ων το αιμα πιλατοϲ εμειξε̅} & 17 &  &  \\
&  & 18 & \foreignlanguage{greek}{μετα των θυϲιων αυτων} & 21 &  &  \\
& \textbf{2} &  & \foreignlanguage{greek}{και αποκριθειϲ ο \textoverline{ιϲ} δοκειται οτι οι} & 7 &  &  \\
&  & 8 & \foreignlanguage{greek}{γαλιλαιοι ουτοι αμαρτωλοι παρα} & 11 &  &  \\
&  & 12 & \foreignlanguage{greek}{πανταϲ τουϲ γαλιλαιουϲ εγενοντο} & 15 &  &  \\
[0.2em]
\cline{4-4}
\end{tabular}
\end{center}
\end{table}
}
\clearpage
\newpage
 {
 \setlength\arrayrulewidth{1pt}
\begin{table}
\begin{center}
\begin{tabular}{ccc|l|ccc}
\cline{4-4} \\ [-1em]
\multicolumn{7}{c}{\foreignlanguage{greek}{ευαγγελιον κατα λουκαν} \textbf{(\nospace{13:2})} } \\ \\ [-1em] % Si on veut ajouter les bordures latérales, remplacer {7}{c} par {7}{|c|}
\cline{4-4} \\
\cline{4-4}
&  &  & &  &  & \\ [-0.9em]
&  & 16 & \foreignlanguage{greek}{οτι τοιαυτα πεπονθαϲιν ουχει λεγω υμι̅} & 3 & \textbf{3} &  \\
&  & 4 & \foreignlanguage{greek}{αλλ εαν μη μετανοητε παντεϲ ωϲαυ} & 9 &  &  \\
&  & 9 & \foreignlanguage{greek}{τωϲ απολειϲθαι η εκεινοι οι δεκα και} & 5 & \textbf{4} &  \\
&  & 6 & \foreignlanguage{greek}{οκτω εφ ουϲ επεϲεν ο πυργοϲ εν τω ϲι} & 14 &  &  \\
&  & 14 & \foreignlanguage{greek}{λωαμ και απεκτινεν αυτουϲ} & 17 &  &  \\
&  & 18 & \foreignlanguage{greek}{δοκειται οτι αυτοι οφειλεται εγενον} & 22 &  &  \\
&  & 22 & \foreignlanguage{greek}{το παρα πανταϲ ανθρωπουϲ τουϲ κα} & 27 &  &  \\
&  & 27 & \foreignlanguage{greek}{τοικουνταϲ εν ιερουϲαλημ ουχει λε} & 2 & \textbf{5} &  \\
&  & 2 & \foreignlanguage{greek}{γω υμιν αλλ εαν μη μετανοειτε παν} & 8 &  &  \\
&  & 8 & \foreignlanguage{greek}{τεϲ ομοιωϲ απολειϲθαι} & 10 &  &  \\
& \textbf{6} &  & \foreignlanguage{greek}{ελεγεν δε ταυτην την παραβολην} & 5 &  &  \\
&  & 6 & \foreignlanguage{greek}{ϲυκην ειχεν τιϲ πεφυτευμενην ε̅} & 10 &  &  \\
&  & 11 & \foreignlanguage{greek}{τω αμπελωνι αυτου και ηλθεν ζη} & 16 &  &  \\
&  & 16 & \foreignlanguage{greek}{των καρπον εν αυτη και ουχ ευρεν} & 22 &  &  \\
& \textbf{7} &  & \foreignlanguage{greek}{ειπεν δε προϲ τον αμπελουργον ιδου} & 6 &  &  \\
&  & 7 & \foreignlanguage{greek}{τρια ετη ερχομαι ζητων καρπον εν} & 12 &  &  \\
&  & 13 & \foreignlanguage{greek}{τη ϲυκη ταυτη και ουχ ευριϲκω} & 18 &  &  \\
&  & 19 & \foreignlanguage{greek}{εκκοψον αυτην ινα τι και την γην} & 25 &  &  \\
&  & 26 & \foreignlanguage{greek}{καταργει} & 26 &  &  \\
& \textbf{8} &  & \foreignlanguage{greek}{ο δε αποκριθειϲ λεγει αυτω \textoverline{κε} αφεϲ} & 7 &  &  \\
&  & 8 & \foreignlanguage{greek}{αυτην και τουτο το ετοϲ εωϲ οτου ϲκα} & 15 &  &  \\
&  & 15 & \foreignlanguage{greek}{ψω περι αυτην και βαλω κοπρια κα̅} & 1 & \textbf{9} &  \\
&  & 2 & \foreignlanguage{greek}{μεν ποιηϲη καρπον ει δε μη γε ειϲ το} & 10 &  &  \\
&  & 11 & \foreignlanguage{greek}{μελλον εκκοψειϲ αυτην} & 13 &  &  \\
& \textbf{10} &  & \foreignlanguage{greek}{ην δε διδαϲκων εν μια των ϲυναγω} & 7 &  &  \\
&  & 7 & \foreignlanguage{greek}{γων εν τοιϲ ϲαββαϲιν και ιδου ην γυ} & 4 & \textbf{11} &  \\
&  & 4 & \foreignlanguage{greek}{νη \textoverline{πνα} εχουϲα αϲθενιαϲ ετη δεκαο} & 9 &  &  \\
&  & 9 & \foreignlanguage{greek}{κτω και ην ϲυνκυπτουϲα και μη δυ} & 15 &  &  \\
&  & 15 & \foreignlanguage{greek}{ναμενη ανακυψαι ειϲ το παντελεϲ} & 19 &  &  \\
& \textbf{12} &  & \foreignlanguage{greek}{ιδων δε αυτην ο \textoverline{ιϲ} προϲεφωνηϲεν ϗ ειπε̅} & 8 &  &  \\
[0.2em]
\cline{4-4}
\end{tabular}
\end{center}
\end{table}
}
\clearpage
\newpage
 {
 \setlength\arrayrulewidth{1pt}
\begin{table}
\begin{center}
\begin{tabular}{ccc|l|ccc}
\cline{4-4} \\ [-1em]
\multicolumn{7}{c}{\foreignlanguage{greek}{ευαγγελιον κατα λουκαν} \textbf{(\nospace{13:12})} } \\ \\ [-1em] % Si on veut ajouter les bordures latérales, remplacer {7}{c} par {7}{|c|}
\cline{4-4} \\
\cline{4-4}
&  &  & &  &  & \\ [-0.9em]
&  & 9 & \foreignlanguage{greek}{αυτη γυναι απολελυϲαι τηϲ αϲθενιαϲ ϲου} & 14 &  &  \\
& \textbf{13} &  & \foreignlanguage{greek}{και επεθηκεν αυτη ταϲ χειραϲ και πα} & 7 &  &  \\
&  & 7 & \foreignlanguage{greek}{ραχρημα ανωρθωθη και εδοξαζεν τον \textoverline{θν}} & 12 &  &  \\
& \textbf{14} &  & \foreignlanguage{greek}{αποκριθειϲ δε ο αρχιϲυναγωγοϲ αγανα} & 5 &  &  \\
&  & 5 & \foreignlanguage{greek}{κτων οτι τω ϲαββατω εθεραπευϲεν} & 9 &  &  \\
&  & 10 & \foreignlanguage{greek}{ο \textoverline{ιϲ} ελεγεν τω οχλω εξ ημεραι ειϲιν} & 17 &  &  \\
&  & 18 & \foreignlanguage{greek}{εν αιϲ δει εργαζεϲθαι εν αυταιϲ ουν} & 24 &  &  \\
&  & 25 & \foreignlanguage{greek}{ερχομενοι θεραπευεϲθαι και μη τη} & 29 &  &  \\
&  & 30 & \foreignlanguage{greek}{ημερα του ϲαββατου} & 32 &  &  \\
& \textbf{15} &  & \foreignlanguage{greek}{απεκριθη ουν αυτω ο \textoverline{κϲ} και ειπεν} & 7 &  &  \\
&  & 8 & \foreignlanguage{greek}{υποκριτα εκαϲτοϲ υμων εν ϲαββα} & 12 &  &  \\
&  & 12 & \foreignlanguage{greek}{τω ου λυει τον βουν αυτου η τον ο} & 20 &  &  \\
&  & 20 & \foreignlanguage{greek}{νον απο τηϲ πατνηϲ και απαγαγων} & 25 &  &  \\
&  & 26 & \foreignlanguage{greek}{ποτιζει ταυτην δε θυγατερα} & 3 & \textbf{16} &  \\
&  & 4 & \foreignlanguage{greek}{αβρααμ ουϲαν ην εδηϲεν ο ϲατα} & 9 &  &  \\
&  & 9 & \foreignlanguage{greek}{ναϲ ιδου δεκα και οκτω ετη ουκ ε} & 16 &  &  \\
&  & 16 & \foreignlanguage{greek}{δει λυθηναι απο του δεϲμου τουτου} & 21 &  &  \\
&  & 22 & \foreignlanguage{greek}{τη ημερα του ϲαββατου} & 25 &  &  \\
& \textbf{17} &  & \foreignlanguage{greek}{και ταυτα λεγοντοϲ αυτου κατηϲχυ} & 5 &  &  \\
&  & 5 & \foreignlanguage{greek}{νοντο παντεϲ οι αντικειμενοι αυτω} & 9 &  &  \\
&  & 10 & \foreignlanguage{greek}{και παϲ ο οχλοϲ εχαιρεν επι παϲιν τοιϲ} & 17 &  &  \\
&  & 18 & \foreignlanguage{greek}{ενδοξοιϲ τοιϲ γεινομενοιϲ υπ αυτου} & 22 &  &  \\
& \textbf{18} &  & \foreignlanguage{greek}{ελεγεν δε τινι ομοια εϲτιν η βαϲιλει} & 7 &  &  \\
&  & 7 & \foreignlanguage{greek}{α του \textoverline{θυ} και τινι ομοιωϲω αυτην} & 13 &  &  \\
& \textbf{19} &  & \foreignlanguage{greek}{ομοια εϲτιν κοκκω ϲιναπεωϲ ον} & 5 &  &  \\
&  & 6 & \foreignlanguage{greek}{λαβων \textoverline{ανοϲ} εβαλεν ειϲ κηπον ε} & 11 &  &  \\
&  & 11 & \foreignlanguage{greek}{αυτου και ηυξηϲεν και εγενετο} & 15 &  &  \\
&  & 16 & \foreignlanguage{greek}{ειϲ δενδρον μεγα και τα πετινα} & 21 &  &  \\
&  & 22 & \foreignlanguage{greek}{του ουρανου κατεϲκηνωϲεν εν} & 25 &  &  \\
&  & 26 & \foreignlanguage{greek}{τοιϲ κλαδοιϲ αυτου} & 28 &  &  \\
[0.2em]
\cline{4-4}
\end{tabular}
\end{center}
\end{table}
}
\clearpage
\newpage
 {
 \setlength\arrayrulewidth{1pt}
\begin{table}
\begin{center}
\begin{tabular}{ccc|l|ccc}
\cline{4-4} \\ [-1em]
\multicolumn{7}{c}{\foreignlanguage{greek}{ευαγγελιον κατα λουκαν} \textbf{(\nospace{13:20})} } \\ \\ [-1em] % Si on veut ajouter les bordures latérales, remplacer {7}{c} par {7}{|c|}
\cline{4-4} \\
\cline{4-4}
&  &  & &  &  & \\ [-0.9em]
& \textbf{20} &  & \foreignlanguage{greek}{παλιν ειπεν τινι ομοιωϲω την βαϲι} & 6 &  &  \\
&  & 6 & \foreignlanguage{greek}{λειαν του \textoverline{θυ} ομοια εϲτιν ζυμη ην} & 4 & \textbf{21} &  \\
&  & 5 & \foreignlanguage{greek}{λαβουϲα γυνη ενεκρυψεν ειϲ αυτην} & 9 &  &  \\
&  & 10 & \foreignlanguage{greek}{ρου ϲατα τρια εωϲ ζυμωθη ολη} & 15 &  &  \\
& \textbf{22} &  & \foreignlanguage{greek}{και διεπορευετο κατα πολειϲ και κω} & 6 &  &  \\
&  & 6 & \foreignlanguage{greek}{μαϲ διδαϲκων και ποριαϲ ποιουμε} & 10 &  &  \\
&  & 10 & \foreignlanguage{greek}{νοϲ ειϲ ιερουϲαλημ ειπεν δε τιϲ} & 3 & \textbf{23} &  \\
&  & 4 & \foreignlanguage{greek}{αυτω \textoverline{κε} ει ολειγοι οι ϲωζομενοι} & 9 &  &  \\
&  & 10 & \foreignlanguage{greek}{ο δε ειπεν προϲ αυτουϲ αγωνιζεϲθε} & 1 & \textbf{24} &  \\
&  & 2 & \foreignlanguage{greek}{ειϲελθειν δια τηϲ ϲτενηϲ πυληϲ} & 6 &  &  \\
&  & 7 & \foreignlanguage{greek}{οτι πολλοι ζητηϲουϲιν ειϲελθειν} & 10 &  &  \\
&  & 11 & \foreignlanguage{greek}{και ουκ ιϲχυϲουϲιν αφ ου αν εγερθη} & 4 & \textbf{25} &  \\
&  & 5 & \foreignlanguage{greek}{ο οικοδεϲποτηϲ και αποκλειϲη την} & 9 &  &  \\
&  & 10 & \foreignlanguage{greek}{θυραν και αρξηϲθαι εξω εϲταναι} & 14 &  &  \\
&  & 15 & \foreignlanguage{greek}{και κρουειν την θυραν λεγοντεϲ} & 19 &  &  \\
&  & 20 & \foreignlanguage{greek}{\textoverline{κε} \textoverline{κε} ανοιξον ημιν} & 23 &  &  \\
&  & 24 & \foreignlanguage{greek}{και αποκριθειϲ ερει υμιν ουκ οιδα} & 29 &  &  \\
&  & 30 & \foreignlanguage{greek}{υμαϲ ποθεν εϲται τοτε αρξη} & 2 & \textbf{26} &  \\
&  & 2 & \foreignlanguage{greek}{ϲθαι λεγειν εφαγομεν ενωπιον ϲου} & 6 &  &  \\
&  & 7 & \foreignlanguage{greek}{και επιομεν και εν ταιϲ πλατιαιϲ} & 12 &  &  \\
&  & 13 & \foreignlanguage{greek}{ημων εδιδαξαϲ και ερει} & 2 & \textbf{27} &  \\
&  & 3 & \foreignlanguage{greek}{λεγω υμιν ουκ οιδα υμαϲ ποθεν εϲται} & 9 &  &  \\
&  & 10 & \foreignlanguage{greek}{αποϲτητε απ εμου παντεϲ εργατε} & 14 &  &  \\
&  & 15 & \foreignlanguage{greek}{τηϲ αδικειαϲ εκει εϲται ο κλαυθμοϲ} & 4 & \textbf{28} &  \\
&  & 5 & \foreignlanguage{greek}{και ο βρυγμοϲ των οδοντων} & 9 &  &  \\
&  & 10 & \foreignlanguage{greek}{οταν οψηϲθαι αβρααμ και ιϲαακ και} & 15 &  &  \\
&  & 16 & \foreignlanguage{greek}{ιακωβ και πανταϲ τουϲ προφηταϲ} & 20 &  &  \\
&  & 21 & \foreignlanguage{greek}{εν τη βαϲιλεια του \textoverline{θυ} υμαϲ δε εκ} & 28 &  &  \\
&  & 28 & \foreignlanguage{greek}{βαλλομενουϲ εξω} & 29 &  &  \\
& \textbf{29} &  & \foreignlanguage{greek}{και ηξουϲιν απο ανατολων και δυϲμω̅} & 6 &  &  \\
[0.2em]
\cline{4-4}
\end{tabular}
\end{center}
\end{table}
}
\clearpage
\newpage
 {
 \setlength\arrayrulewidth{1pt}
\begin{table}
\begin{center}
\begin{tabular}{ccc|l|ccc}
\cline{4-4} \\ [-1em]
\multicolumn{7}{c}{\foreignlanguage{greek}{ευαγγελιον κατα λουκαν} \textbf{(\nospace{13:29})} } \\ \\ [-1em] % Si on veut ajouter les bordures latérales, remplacer {7}{c} par {7}{|c|}
\cline{4-4} \\
\cline{4-4}
&  &  & &  &  & \\ [-0.9em]
&  & 7 & \foreignlanguage{greek}{και βορρα και νοτου και ανακλειθη} & 12 &  &  \\
&  & 12 & \foreignlanguage{greek}{ϲονται εν τη βαϲιλεια του \textoverline{θυ}} & 17 &  &  \\
& \textbf{30} &  & \foreignlanguage{greek}{και ιδου ειϲιν εϲχατοι οι εϲονται πρω} & 7 &  &  \\
&  & 7 & \foreignlanguage{greek}{τοι και ειϲιν πρωτοι οι εϲονται εϲχατοι} & 13 &  &  \\
& \textbf{31} &  & \foreignlanguage{greek}{εν ταυτη τη ημερα προϲηλθον τινεϲ} & 6 &  &  \\
&  & 7 & \foreignlanguage{greek}{φαριϲαιοι λεγοντεϲ αυτω εξελθε και} & 11 &  &  \\
&  & 12 & \foreignlanguage{greek}{πορευου εντευθεν οτι ηρωδηϲ ϲε} & 16 &  &  \\
&  & 17 & \foreignlanguage{greek}{θελει αποκτειναι} & 18 &  &  \\
& \textbf{32} &  & \foreignlanguage{greek}{και ειπεν αυτοιϲ πορευθεντεϲ ει} & 5 &  &  \\
&  & 5 & \foreignlanguage{greek}{πατε τη αλωπεκει ταυτη ιδου εκ} & 10 &  &  \\
&  & 10 & \foreignlanguage{greek}{βαλλω δαιμονια και ιαϲειϲ επιτελω} & 14 &  &  \\
&  & 15 & \foreignlanguage{greek}{ϲημερον και αυριον και τη τριτη} & 20 &  &  \\
&  & 21 & \foreignlanguage{greek}{τελιουμαι} & 21 &  &  \\
& \textbf{33} &  & \foreignlanguage{greek}{πλην δε με ϲημερον και αυριον και} & 7 &  &  \\
&  & 8 & \foreignlanguage{greek}{τη εχομενη πορευεϲθαι οτι ουκ εν} & 13 &  &  \\
&  & 13 & \foreignlanguage{greek}{δεχεται προφητην απολεϲθαι εξω} & 16 &  &  \\
&  & 17 & \foreignlanguage{greek}{ιερουϲαλημ} & 17 &  &  \\
& \textbf{34} &  & \foreignlanguage{greek}{ιερουϲαλημ ιερουϲαλημ η αποκτι} & 4 &  &  \\
&  & 4 & \foreignlanguage{greek}{νουϲα τουϲ προφηταϲ και λιθοβολου} & 8 &  &  \\
&  & 8 & \foreignlanguage{greek}{ϲα τουϲ απεϲταλμενουϲ προϲ αυτην} & 12 &  &  \\
&  & 13 & \foreignlanguage{greek}{ποϲακειϲ ηθεληϲα επιϲυναξαι τα τε} & 17 &  &  \\
&  & 17 & \foreignlanguage{greek}{κνα ϲου ον τροπον ορνιξ την εαυ} & 23 &  &  \\
&  & 23 & \foreignlanguage{greek}{τηϲ νοϲϲιαν υπο ταϲ πτερυγαϲ και} & 28 &  &  \\
&  & 29 & \foreignlanguage{greek}{ουκ ηθεληϲατε ιδου αφειεται} & 2 & \textbf{35} &  \\
&  & 3 & \foreignlanguage{greek}{υμιν ο οικοϲ υμων} & 6 &  &  \\
&  & 7 & \foreignlanguage{greek}{λεγω δε υμιν οτι ου μη ιδηται με} & 14 &  &  \\
&  & 15 & \foreignlanguage{greek}{εωϲ αν ηξει οτε ειπητε ευλογημε} & 21 &  &  \\
&  & 21 & \foreignlanguage{greek}{νοϲ ο ερχομενοϲ εν ονοματι \textoverline{κυ}} & 26 &  &  \\
& \mygospelchapter &  & \foreignlanguage{greek}{και εγενετο εν τω ελθειν αυτον ειϲ} & 7 &  &  \\
&  & 8 & \foreignlanguage{greek}{οικον τινοϲ των αρχοντων των φα} & 13 &  &  \\
[0.2em]
\cline{4-4}
\end{tabular}
\end{center}
\end{table}
}
\clearpage
\newpage
 {
 \setlength\arrayrulewidth{1pt}
\begin{table}
\begin{center}
\begin{tabular}{ccc|l|ccc}
\cline{4-4} \\ [-1em]
\multicolumn{7}{c}{\foreignlanguage{greek}{ευαγγελιον κατα λουκαν} \textbf{(\nospace{14:1})} } \\ \\ [-1em] % Si on veut ajouter les bordures latérales, remplacer {7}{c} par {7}{|c|}
\cline{4-4} \\
\cline{4-4}
&  &  & &  &  & \\ [-0.9em]
&  & 13 & \foreignlanguage{greek}{ριϲαιων ϲαββατω φαγειν αρτον και αυ} & 18 &  &  \\
&  & 18 & \foreignlanguage{greek}{τοι ηϲαν παρατηρουμενοι αυτον} & 21 &  &  \\
& \textbf{2} &  & \foreignlanguage{greek}{και ιδου ανθρωποϲ τιϲ ην υδρωπικοϲ} & 6 &  &  \\
&  & 7 & \foreignlanguage{greek}{εμπροϲθεν αυτου} & 8 &  &  \\
& \textbf{3} &  & \foreignlanguage{greek}{και αποκριθειϲ ο \textoverline{ιϲ} ειπεν προϲ αυτουϲ} & 7 &  &  \\
&  & 8 & \foreignlanguage{greek}{νομικουϲ και φαριϲαιουϲ λεγων ει ε} & 13 &  &  \\
&  & 13 & \foreignlanguage{greek}{ξεϲτιν τω ϲαββατω θεραπευειν} & 16 &  &  \\
& \textbf{4} &  & \foreignlanguage{greek}{οι δε ηϲυχαϲαν και επιλαβομενοϲ} & 5 &  &  \\
&  & 6 & \foreignlanguage{greek}{ιαϲατο αυτον και απελυϲεν} & 9 &  &  \\
& \textbf{5} &  & \foreignlanguage{greek}{και αποκριθειϲ ο \textoverline{ιϲ} ειπεν προϲ αυτουϲ} & 7 &  &  \\
&  & 8 & \foreignlanguage{greek}{τινοϲ υμων υιοϲ η βουϲ ειϲ φρεαρ πε} & 15 &  &  \\
&  & 15 & \foreignlanguage{greek}{ϲειται και ουκ ευθεωϲ αναϲπαϲι αυτο̅} & 20 &  &  \\
&  & 21 & \foreignlanguage{greek}{εν τη ημερα του ϲαββατου και ουκ ι} & 3 & \textbf{6} &  \\
&  & 3 & \foreignlanguage{greek}{ϲχυϲαν ανταποκριθηναι αυτω} & 5 &  &  \\
&  & 6 & \foreignlanguage{greek}{προϲ ταυτα} & 7 &  &  \\
& \textbf{7} &  & \foreignlanguage{greek}{ελεγεν δε προϲ τουϲ κεκλημενουϲ} & 5 &  &  \\
&  & 6 & \foreignlanguage{greek}{παραβολην επεχων πωϲ ταϲ πρω} & 10 &  &  \\
&  & 10 & \foreignlanguage{greek}{τοκλειϲιαϲ εξελεγοντο λεγων προϲ} & 13 &  &  \\
&  & 14 & \foreignlanguage{greek}{αυτουϲ οταν κληθηϲ υπο τινοϲ ειϲ} & 5 & \textbf{8} &  \\
&  & 6 & \foreignlanguage{greek}{γαμουϲ μη κατακλειθηϲ ειϲ την πρω} & 11 &  &  \\
&  & 11 & \foreignlanguage{greek}{τοκλιϲιαν μηποτε εντιμοτεροϲ ϲου} & 14 &  &  \\
&  & 15 & \foreignlanguage{greek}{η κεκλημενοϲ υπ αυτου και ελθω̅} & 2 & \textbf{9} &  \\
&  & 3 & \foreignlanguage{greek}{ο ϲε και αυτον καλεϲαϲ ερι ϲοι δοϲ του} & 11 &  &  \\
&  & 11 & \foreignlanguage{greek}{τω τοπον και τοτε αρξη μετα αιϲχυ} & 17 &  &  \\
&  & 17 & \foreignlanguage{greek}{νηϲ τον εϲχατον τοπον κατεχειν} & 21 &  &  \\
& \textbf{10} &  & \foreignlanguage{greek}{αλλ οταν κληθηϲ πορευθειϲ αναπε} & 5 &  &  \\
&  & 5 & \foreignlanguage{greek}{ϲε ειϲ τον εϲχατον τοπον ινα οταν} & 11 &  &  \\
&  & 12 & \foreignlanguage{greek}{ελθη ο κεκληκωϲ ϲε ειπη ϲοι φιλε} & 18 &  &  \\
&  & 19 & \foreignlanguage{greek}{προϲαναβηθει ανωτερον τοτε εϲται} & 22 &  &  \\
&  & 23 & \foreignlanguage{greek}{ϲοι δοξα ενωπιον των ϲυνανακειμενων} & 27 &  &  \\
&  & 28 & \foreignlanguage{greek}{ϲοι} & 28 &  &  \\
[0.2em]
\cline{4-4}
\end{tabular}
\end{center}
\end{table}
}
\clearpage
\newpage
 {
 \setlength\arrayrulewidth{1pt}
\begin{table}
\begin{center}
\begin{tabular}{ccc|l|ccc}
\cline{4-4} \\ [-1em]
\multicolumn{7}{c}{\foreignlanguage{greek}{ευαγγελιον κατα λουκαν} \textbf{(\nospace{14:11})} } \\ \\ [-1em] % Si on veut ajouter les bordures latérales, remplacer {7}{c} par {7}{|c|}
\cline{4-4} \\
\cline{4-4}
&  &  & &  &  & \\ [-0.9em]
& \textbf{11} &  & \foreignlanguage{greek}{οτι παϲ ο υψων εαυτον ταπινωθηϲε} & 6 &  &  \\
&  & 6 & \foreignlanguage{greek}{ται και ο ταπινων εαυτον υψωθηϲεται} & 11 &  &  \\
& \textbf{12} &  & \foreignlanguage{greek}{ελεγεν δε και τω κεκληκοτι αυτον} & 6 &  &  \\
&  & 7 & \foreignlanguage{greek}{οταν ποιηϲ αριϲτον η διπνον μη φω} & 13 &  &  \\
&  & 13 & \foreignlanguage{greek}{νει τουϲ φιλουϲ ϲου μηδε τουϲ αδελ} & 19 &  &  \\
&  & 19 & \foreignlanguage{greek}{φουϲ ϲου μηδε τουϲ ϲυγγενειϲ ϲου} & 24 &  &  \\
&  & 25 & \foreignlanguage{greek}{μηδε γειτοναϲ πλουϲιουϲ μηπο} & 28 &  &  \\
&  & 28 & \foreignlanguage{greek}{τε και αυτοι ϲε αντικαλεϲωϲιν και} & 33 &  &  \\
&  & 34 & \foreignlanguage{greek}{γενηται ϲοι ανταποδομα} & 36 &  &  \\
& \textbf{13} &  & \foreignlanguage{greek}{αλλα οταν ποιηϲ δοχην καλει πτω} & 6 &  &  \\
&  & 6 & \foreignlanguage{greek}{χουϲ αναπειρουϲ χωλουϲ τυφλουϲ} & 9 &  &  \\
& \textbf{14} &  & \foreignlanguage{greek}{και μακαριοϲ εϲη οτι ουκ εχουϲιν} & 6 &  &  \\
&  & 7 & \foreignlanguage{greek}{ανταποδουναι ϲοι ανταποδοθη} & 9 &  &  \\
&  & 9 & \foreignlanguage{greek}{ϲεται γαρ ϲοι εν τη αναϲταϲι των} & 15 &  &  \\
&  & 16 & \foreignlanguage{greek}{δικαιων} & 16 &  &  \\
& \textbf{15} &  & \foreignlanguage{greek}{ακουϲαϲ δε τιϲ των ϲυνανακειμενω̅} & 5 &  &  \\
&  & 6 & \foreignlanguage{greek}{ταυτα ειπεν αυτω μακαριοϲ οϲ} & 10 &  &  \\
&  & 11 & \foreignlanguage{greek}{φαγετε αριϲτον εν τη βαϲιλεια του \textoverline{θυ}} & 17 &  &  \\
& \textbf{16} &  & \foreignlanguage{greek}{ο δε ειπεν αυτω ανθρωποϲ τιϲ εποι} & 7 &  &  \\
&  & 7 & \foreignlanguage{greek}{ηϲεν διπνον μεγα και εκαλεϲεν} & 11 &  &  \\
&  & 12 & \foreignlanguage{greek}{πολλουϲ και απεϲτιλεν τον δου} & 4 & \textbf{17} &  \\
&  & 4 & \foreignlanguage{greek}{λον αυτου τη ωρα του διπνου ει} & 10 &  &  \\
&  & 10 & \foreignlanguage{greek}{πειν τοιϲ κεκλημενοιϲ ερχεϲθαι} & 13 &  &  \\
&  & 14 & \foreignlanguage{greek}{οτι ηδη ετοιμα εϲτιν παντα} & 18 &  &  \\
& \textbf{18} &  & \foreignlanguage{greek}{και ηρξαντο απο μιαϲ παραιτιϲθαι} & 5 &  &  \\
&  & 6 & \foreignlanguage{greek}{παντεϲ ο πρωτοϲ ειπεν αυτω α} & 11 &  &  \\
&  & 11 & \foreignlanguage{greek}{γρον ηγοραϲα και εχω αναγκην ε} & 16 &  &  \\
&  & 16 & \foreignlanguage{greek}{ξελθειν και ιδειν αυτον ερωτω ϲε} & 21 &  &  \\
&  & 22 & \foreignlanguage{greek}{εχε με παρητημενον} & 24 &  &  \\
& \textbf{19} &  & \foreignlanguage{greek}{και ετεροϲ ειπεν ζευγη βοων ηγο} & 6 &  &  \\
[0.2em]
\cline{4-4}
\end{tabular}
\end{center}
\end{table}
}
\clearpage
\newpage
 {
 \setlength\arrayrulewidth{1pt}
\begin{table}
\begin{center}
\begin{tabular}{ccc|l|ccc}
\cline{4-4} \\ [-1em]
\multicolumn{7}{c}{\foreignlanguage{greek}{ευαγγελιον κατα λουκαν} \textbf{(\nospace{14:19})} } \\ \\ [-1em] % Si on veut ajouter les bordures latérales, remplacer {7}{c} par {7}{|c|}
\cline{4-4} \\
\cline{4-4}
&  &  & &  &  & \\ [-0.9em]
&  & 6 & \foreignlanguage{greek}{ραϲα πεντε και πορευομαι δοκειμαϲαι} & 10 &  &  \\
&  & 11 & \foreignlanguage{greek}{αυτα ερωτω ϲε εχε με παρητημενον} & 16 &  &  \\
& \textbf{20} &  & \foreignlanguage{greek}{και ετεροϲ ειπεν γυναικα εγημα και} & 6 &  &  \\
&  & 7 & \foreignlanguage{greek}{δια τουτο ου δυναμαι ελθειν και πα} & 2 & \textbf{21} &  \\
&  & 2 & \foreignlanguage{greek}{ραγενομενοϲ ο δουλοϲ απηγγειλεν τω} & 6 &  &  \\
&  & 7 & \foreignlanguage{greek}{κυριω εαυτου ταυτα} & 9 &  &  \\
&  & 10 & \foreignlanguage{greek}{τοτε οργειϲθειϲ ο οικοδεϲποτηϲ ειπεν} & 14 &  &  \\
&  & 15 & \foreignlanguage{greek}{τω δουλω αυτου εξελθε ταχεωϲ ειϲ} & 20 &  &  \\
&  & 21 & \foreignlanguage{greek}{ταϲ πλατιαϲ και ρυμαϲ τηϲ πολεωϲ} & 26 &  &  \\
&  & 27 & \foreignlanguage{greek}{και τουϲ πτωχουϲ και αναπειρουϲ και} & 32 &  &  \\
&  & 33 & \foreignlanguage{greek}{τυφλουϲ και χωλουϲ ειϲαγαγε ωδε} & 37 &  &  \\
& \textbf{22} &  & \foreignlanguage{greek}{και ειπεν ο δουλοϲ \textoverline{κε} γεγονεν ωϲ ε} & 8 &  &  \\
&  & 8 & \foreignlanguage{greek}{πεταξαϲ και ετι τοποϲ εϲτιν} & 12 &  &  \\
& \textbf{23} &  & \foreignlanguage{greek}{και ειπεν ο \textoverline{κϲ} προϲ τον δουλον εξελθε} & 8 &  &  \\
&  & 9 & \foreignlanguage{greek}{ειϲ ταϲ οδουϲ και φραγμουϲ και αναγ} & 15 &  &  \\
&  & 15 & \foreignlanguage{greek}{καϲον ειϲελθειν ινα γεμιϲθη ο οικοϲ μου} & 21 &  &  \\
& \textbf{24} &  & \foreignlanguage{greek}{λεγω γαρ υμιν οτι ουδειϲ των ανδρω̅} & 7 &  &  \\
&  & 8 & \foreignlanguage{greek}{εκεινων των κεκλημενων γευϲη} & 11 &  &  \\
&  & 11 & \foreignlanguage{greek}{ται μου του διπνου} & 14 &  &  \\
& \textbf{25} &  & \foreignlanguage{greek}{ϲυνεπορευοντο δε αυτω οχλοι πολλοι} & 5 &  &  \\
&  & 6 & \foreignlanguage{greek}{και ϲτραφειϲ ειπεν προϲ αυτουϲ} & 10 &  &  \\
& \textbf{26} &  & \foreignlanguage{greek}{ει τιϲ ερχεται προϲ με και ου μιϲει το̅} & 9 &  &  \\
&  & 10 & \foreignlanguage{greek}{\textoverline{πρα} αυτου και την μητερα και την} & 16 &  &  \\
&  & 17 & \foreignlanguage{greek}{γυναικα και τα τεκνα και τουϲ α} & 23 &  &  \\
&  & 23 & \foreignlanguage{greek}{δελφουϲ και ταϲ αδελφαϲ ετι δε ϗ} & 29 &  &  \\
&  & 30 & \foreignlanguage{greek}{την εαυτου ψυχην ου δυναται μου} & 35 &  &  \\
&  & 36 & \foreignlanguage{greek}{μαθητηϲ ειναι και οϲτιϲ ου βαϲτα} & 4 & \textbf{27} &  \\
&  & 4 & \foreignlanguage{greek}{ζει τον ϲταυρον εαυτου και ερχεται} & 9 &  &  \\
&  & 10 & \foreignlanguage{greek}{οπιϲω μου ου δυναται ειναι μου μα} & 16 &  &  \\
&  & 16 & \foreignlanguage{greek}{θητηϲ} & 16 &  &  \\
[0.2em]
\cline{4-4}
\end{tabular}
\end{center}
\end{table}
}
\clearpage
\newpage
 {
 \setlength\arrayrulewidth{1pt}
\begin{table}
\begin{center}
\begin{tabular}{ccc|l|ccc}
\cline{4-4} \\ [-1em]
\multicolumn{7}{c}{\foreignlanguage{greek}{ευαγγελιον κατα λουκαν} \textbf{(\nospace{14:28})} } \\ \\ [-1em] % Si on veut ajouter les bordures latérales, remplacer {7}{c} par {7}{|c|}
\cline{4-4} \\
\cline{4-4}
&  &  & &  &  & \\ [-0.9em]
& \textbf{28} &  & \foreignlanguage{greek}{τιϲ γαρ εξ υμων ο θελων πυργον οκοδομη} & 8 &  &  \\
&  & 8 & \foreignlanguage{greek}{ϲαι ουχι πρωτον καθειϲαϲ ψηφιζει τη̅} & 13 &  &  \\
&  & 14 & \foreignlanguage{greek}{δαπανην ει εχει ειϲ απαρτιϲμον ινα} & 1 & \textbf{29} &  \\
&  & 2 & \foreignlanguage{greek}{μηποτε θεντοϲ αυτου θεμελιον και} & 6 &  &  \\
&  & 7 & \foreignlanguage{greek}{μη ιϲχυοντοϲ εκτελεϲαι παντεϲ οι} & 11 &  &  \\
&  & 12 & \foreignlanguage{greek}{θεωρουντεϲ αρξωνται αυτω ενπε} & 15 &  &  \\
&  & 15 & \foreignlanguage{greek}{ζειν λεγοντεϲ οτι ουτοϲ ο \textoverline{ανοϲ} ηρξα} & 6 & \textbf{30} &  \\
&  & 6 & \foreignlanguage{greek}{το οικοδομειν και ουκ ιϲχυϲεν εκ} & 11 &  &  \\
&  & 11 & \foreignlanguage{greek}{τελεϲαι} & 11 &  &  \\
& \textbf{31} &  & \foreignlanguage{greek}{η τιϲ βαϲιλευϲ πορευομενοϲ ϲυνβα} & 5 &  &  \\
&  & 5 & \foreignlanguage{greek}{λιν ετερω βαϲιλει ειϲ πολεμον} & 9 &  &  \\
&  & 10 & \foreignlanguage{greek}{ουχει καθειϲαϲ πρωτον βουλευεται} & 13 &  &  \\
&  & 14 & \foreignlanguage{greek}{ει δυνατοϲ εϲτιν εν δεκα χειλιαϲιν} & 19 &  &  \\
&  & 20 & \foreignlanguage{greek}{απαντηϲαι τω μετα εικοϲι χειλιαδω̅} & 24 &  &  \\
&  & 25 & \foreignlanguage{greek}{ερχομενω επ αυτον ει δε μη γε} & 4 & \textbf{32} &  \\
&  & 5 & \foreignlanguage{greek}{ετι αυτου πορρω οντοϲ πρεϲβειαν} & 9 &  &  \\
&  & 10 & \foreignlanguage{greek}{αποϲτιλαϲ ερωτα τα προϲ ειρηνην} & 14 &  &  \\
& \textbf{33} &  & \foreignlanguage{greek}{ουτωϲ παϲ εξ υμων οϲ ουκ αποταϲϲε} & 7 &  &  \\
&  & 7 & \foreignlanguage{greek}{ται παϲιν τοιϲ αυτου υπαρχουϲιν} & 11 &  &  \\
&  & 12 & \foreignlanguage{greek}{ου δυναται μου ειναι μαθητηϲ} & 16 &  &  \\
& \textbf{34} &  & \foreignlanguage{greek}{καλον το αλα εαν δε το αλα μωραν} & 8 &  &  \\
&  & 8 & \foreignlanguage{greek}{θη εν τινι αρτυθηϲεται ουτε ειϲ} & 2 & \textbf{35} &  \\
&  & 3 & \foreignlanguage{greek}{γην ουτε ειϲ κοπριαν ευθετον εϲτι̅} & 8 &  &  \\
&  & 9 & \foreignlanguage{greek}{εξω βαλλουϲιν αυτο ο εχων ωτα α} & 15 &  &  \\
&  & 15 & \foreignlanguage{greek}{κουειν ακουετω} & 16 &  &  \\
& \mygospelchapter &  & \foreignlanguage{greek}{ηϲαν δε αυτω εγγιζοντεϲ οι τελωναι} & 6 &  &  \\
&  & 7 & \foreignlanguage{greek}{και οι αμαρτωλοι ακουειν αυτου και} & 1 & \textbf{2} &  \\
&  & 2 & \foreignlanguage{greek}{διεγογγυζον οι φαριϲαιοι και οι γραμ} & 7 &  &  \\
&  & 7 & \foreignlanguage{greek}{ματιϲ λεγοντεϲ οτι ουτοϲ αμαρτω} & 11 &  &  \\
&  & 11 & \foreignlanguage{greek}{λουϲ προϲδεχεται ϗ ϲυνεϲθιει αυτοιϲ} & 15 &  &  \\
[0.2em]
\cline{4-4}
\end{tabular}
\end{center}
\end{table}
}
\clearpage
\newpage
 {
 \setlength\arrayrulewidth{1pt}
\begin{table}
\begin{center}
\begin{tabular}{ccc|l|ccc}
\cline{4-4} \\ [-1em]
\multicolumn{7}{c}{\foreignlanguage{greek}{ευαγγελιον κατα λουκαν} \textbf{(\nospace{15:3})} } \\ \\ [-1em] % Si on veut ajouter les bordures latérales, remplacer {7}{c} par {7}{|c|}
\cline{4-4} \\
\cline{4-4}
&  &  & &  &  & \\ [-0.9em]
& \textbf{3} &  & \foreignlanguage{greek}{ειπεν δε προϲ αυτουϲ παραβολην ταυτη̅} & 6 &  &  \\
&  & 7 & \foreignlanguage{greek}{λεγων τιϲ \textoverline{ανοϲ} εξ υμων εχων εκατον} & 6 & \textbf{4} &  \\
&  & 7 & \foreignlanguage{greek}{προβατα και απολεϲαϲ εξ αυτων εν ου} & 13 &  &  \\
&  & 14 & \foreignlanguage{greek}{καταλιπει τα \textoverline{ϟ} \textoverline{θ} εν τη ερημω και πορευ} & 22 &  &  \\
&  & 22 & \foreignlanguage{greek}{εται επι το απολωλοϲ εωϲ ευρη αυτο} & 28 &  &  \\
& \textbf{5} &  & \foreignlanguage{greek}{και ευρων επιτιθηϲιν επι τουϲ ωμουϲ ε} & 7 &  &  \\
&  & 7 & \foreignlanguage{greek}{αυτου χαιρων και ελθων ειϲ τον οικον} & 5 & \textbf{6} &  \\
&  & 6 & \foreignlanguage{greek}{ϲυνκαλει τουϲ φιλουϲ και τουϲ γειτοναϲ} & 11 &  &  \\
&  & 12 & \foreignlanguage{greek}{λεγων αυτοιϲ ϲυνχαρηται μοι οτι ευρον} & 17 &  &  \\
&  & 18 & \foreignlanguage{greek}{το προβατον μου το απολωλοϲ} & 22 &  &  \\
& \textbf{7} &  & \foreignlanguage{greek}{λεγω υμιν οτι ουτωϲ χαρα εϲται εν τω} & 8 &  &  \\
&  & 9 & \foreignlanguage{greek}{ουρανω επι ενι αμαρτωλω μετανοου̅} & 13 &  &  \\
&  & 13 & \foreignlanguage{greek}{τι η επι \textoverline{ϟ} \textoverline{θ} δικαιοιϲ οιτινεϲ ου χρειαν} & 21 &  &  \\
&  & 22 & \foreignlanguage{greek}{εχουϲιν μετανοιαϲ} & 23 &  &  \\
& \textbf{8} &  & \foreignlanguage{greek}{η τιϲ γυνη δραχμαϲ εχουϲα δεκα εαν α} & 8 &  &  \\
&  & 8 & \foreignlanguage{greek}{πολεϲη δραχμην μιαν ουχει απτι λυ} & 13 &  &  \\
&  & 13 & \foreignlanguage{greek}{χνον και ϲαροι την οικειαν και ζητι ε} & 20 &  &  \\
&  & 20 & \foreignlanguage{greek}{πιμελωϲ εωϲ οτου ευρη και ευρουϲα} & 2 & \textbf{9} &  \\
&  & 3 & \foreignlanguage{greek}{ϲυνκαλειται ταϲ φιλαϲ και ταϲ γειτο} & 8 &  &  \\
&  & 8 & \foreignlanguage{greek}{ναϲ λεγουϲα ϲυνχαρηται μοι οτι ευρον} & 13 &  &  \\
&  & 14 & \foreignlanguage{greek}{την δραχμην ην απωλεϲα} & 17 &  &  \\
& \textbf{10} &  & \foreignlanguage{greek}{ουτωϲ λεγω υμιν χαρα γεινεται ενωπιο̅} & 6 &  &  \\
&  & 7 & \foreignlanguage{greek}{των αγγελων του \textoverline{θυ} επι ενι αμαρτω} & 13 &  &  \\
&  & 13 & \foreignlanguage{greek}{λω μετανοουντι} & 14 &  &  \\
& \textbf{11} &  & \foreignlanguage{greek}{ειπεν δε ανθρωποϲ τιϲ εϲχεν δυο υιουϲ} & 7 &  &  \\
& \textbf{12} &  & \foreignlanguage{greek}{και ειπεν ο νεωτεροϲ αυτων τω \textoverline{πρι}} & 7 &  &  \\
&  & 8 & \foreignlanguage{greek}{πατερ δοϲ μοι το επιβαλλον μεροϲ τηϲ} & 14 &  &  \\
&  & 15 & \foreignlanguage{greek}{ουϲιαϲ και διειλεν αυτοιϲ τον βιο̅} & 20 &  &  \\
& \textbf{13} &  & \foreignlanguage{greek}{και μετ ου πολλαϲ ημεραϲ ϲυναγαγω̅} & 6 &  &  \\
&  & 7 & \foreignlanguage{greek}{απαντα ο νεωτεροϲ υιοϲ απεδημηϲεν} & 11 &  &  \\
[0.2em]
\cline{4-4}
\end{tabular}
\end{center}
\end{table}
}
\clearpage
\newpage
 {
 \setlength\arrayrulewidth{1pt}
\begin{table}
\begin{center}
\begin{tabular}{ccc|l|ccc}
\cline{4-4} \\ [-1em]
\multicolumn{7}{c}{\foreignlanguage{greek}{ευαγγελιον κατα λουκαν} \textbf{(\nospace{15:13})} } \\ \\ [-1em] % Si on veut ajouter les bordures latérales, remplacer {7}{c} par {7}{|c|}
\cline{4-4} \\
\cline{4-4}
&  &  & &  &  & \\ [-0.9em]
&  & 12 & \foreignlanguage{greek}{ειϲ χωραν μακραν και εκει διεϲκορπιϲε̅} & 17 &  &  \\
&  & 18 & \foreignlanguage{greek}{την ουϲιαν αυτου ζων αϲωτωϲ} & 22 &  &  \\
& \textbf{14} &  & \foreignlanguage{greek}{δαπανηϲαντοϲ δε αυτου παντα εγενε} & 5 &  &  \\
&  & 5 & \foreignlanguage{greek}{το λιμοϲ ιϲχυροϲ κατα την χωραν εκεινη̅} & 11 &  &  \\
&  & 12 & \foreignlanguage{greek}{και αυτοϲ ηρξατο υϲτεριϲθαι και πορευ} & 2 & \textbf{15} &  \\
&  & 2 & \foreignlanguage{greek}{θειϲ εκολληθη ενι των πολειτων τηϲ} & 7 &  &  \\
&  & 8 & \foreignlanguage{greek}{χωραϲ εκεινηϲ και επεμψεν αυτον} & 12 &  &  \\
&  & 13 & \foreignlanguage{greek}{ειϲ τουϲ αγρουϲ αυτου βοϲκειν χοιρουϲ} & 18 &  &  \\
& \textbf{16} &  & \foreignlanguage{greek}{και επεθυμει γεμιϲαι την κοιλιαν και} & 6 &  &  \\
&  & 7 & \foreignlanguage{greek}{χορταϲθηναι απο των κερατιων ων η} & 12 &  &  \\
&  & 12 & \foreignlanguage{greek}{ϲθιον οι χοιροι και ουδειϲ εδιδου αυτω} & 18 &  &  \\
& \textbf{17} &  & \foreignlanguage{greek}{ειϲ εαυτον δε ελθων ειπεν ποϲοι μι} & 7 &  &  \\
&  & 7 & \foreignlanguage{greek}{ϲθιου του \textoverline{πρϲ} μου περιϲϲευουϲιν αρτω̅} & 12 &  &  \\
&  & 13 & \foreignlanguage{greek}{εγω δε λιμω απολλυμαι αναϲταϲ πο} & 2 & \textbf{18} &  \\
&  & 2 & \foreignlanguage{greek}{ρευϲομαι προϲ τον \textoverline{πρα} μου και ερω αυ} & 9 &  &  \\
&  & 9 & \foreignlanguage{greek}{τω πατερ ημαρτον ειϲ τον ουρανον και} & 15 &  &  \\
&  & 16 & \foreignlanguage{greek}{ενωπιον ϲου ουκετι ειμει αξιοϲ κλη} & 4 & \textbf{19} &  \\
&  & 4 & \foreignlanguage{greek}{θηναι υιοϲ ϲου και αναϲταϲ ηλθεν} & 3 & \textbf{20} &  \\
&  & 4 & \foreignlanguage{greek}{προϲ τον \textoverline{πρα} εαυτου} & 7 &  &  \\
&  & 8 & \foreignlanguage{greek}{ετι δε αυτου μακραν απεχοντοϲ ειδε̅} & 13 &  &  \\
&  & 14 & \foreignlanguage{greek}{αυτον ο \textoverline{πηρ} αυτου και εϲπλαγχνιϲθη} & 19 &  &  \\
&  & 20 & \foreignlanguage{greek}{και δραμων επεϲεν επι τον τραχηλο̅} & 25 &  &  \\
&  & 26 & \foreignlanguage{greek}{αυτου και κατεφιληϲεν αυτον} & 29 &  &  \\
& \textbf{21} &  & \foreignlanguage{greek}{ειπεν δε αυτω ο υιοϲ \textoverline{περ} ημαρτον ει} & 8 &  &  \\
&  & 8 & \foreignlanguage{greek}{ϲ τον ουρανον και ενωπιον ϲου και ου} & 15 &  &  \\
&  & 15 & \foreignlanguage{greek}{κετι ειμι αξιοϲ κληθηναι υιοϲ ϲου} & 20 &  &  \\
& \textbf{22} &  & \foreignlanguage{greek}{ειπεν δε ο \textoverline{πηρ} προϲ τουϲ δουλουϲ αυτου} & 8 &  &  \\
&  & 9 & \foreignlanguage{greek}{εξενεγκατε την ϲτολην την πρωτην και} & 14 &  &  \\
&  & 15 & \foreignlanguage{greek}{ενδυϲατε αυτον και δοτε αυτω δα} & 20 &  &  \\
&  & 20 & \foreignlanguage{greek}{κτυλιον ειϲ την χειρα αυτου και υπο} & 26 &  &  \\
[0.2em]
\cline{4-4}
\end{tabular}
\end{center}
\end{table}
}
\clearpage
\newpage
 {
 \setlength\arrayrulewidth{1pt}
\begin{table}
\begin{center}
\begin{tabular}{ccc|l|ccc}
\cline{4-4} \\ [-1em]
\multicolumn{7}{c}{\foreignlanguage{greek}{ευαγγελιον κατα λουκαν} \textbf{(\nospace{15:22})} } \\ \\ [-1em] % Si on veut ajouter les bordures latérales, remplacer {7}{c} par {7}{|c|}
\cline{4-4} \\
\cline{4-4}
&  &  & &  &  & \\ [-0.9em]
&  & 26 & \foreignlanguage{greek}{δηματα ειϲ τουϲ ποδαϲ και ενεγκαντεϲ} & 2 & \textbf{23} &  \\
&  & 3 & \foreignlanguage{greek}{τον μοϲχον τον ϲιτευτον θυϲατε και} & 8 &  &  \\
&  & 9 & \foreignlanguage{greek}{φαγοντεϲ ευφρανθωμεν οτι ουτοϲ ο υι} & 4 & \textbf{24} &  \\
&  & 4 & \foreignlanguage{greek}{οϲ μου νεκροϲ ην και ανεζηϲεν και ηρ} & 11 &  &  \\
&  & 11 & \foreignlanguage{greek}{ξαντο ευφραινεϲθαι} & 12 &  &  \\
& \textbf{25} &  & \foreignlanguage{greek}{ην δε ο υιοϲ αυτου ο πρεϲβυτεροϲ εν} & 8 &  &  \\
&  & 9 & \foreignlanguage{greek}{αγρω και ωϲ ερχομενοϲ ηγγιϲεν τη οι} & 15 &  &  \\
&  & 15 & \foreignlanguage{greek}{κεια ηκουϲεν ϲυμφωνιαϲ και χορων} & 19 &  &  \\
& \textbf{26} &  & \foreignlanguage{greek}{και προϲκαλεϲαμενοϲ ενα των παιδω̅} & 5 &  &  \\
&  & 6 & \foreignlanguage{greek}{επυνθανετο τι ειη ταυτα ο δε ειπεν} & 3 & \textbf{27} &  \\
&  & 4 & \foreignlanguage{greek}{αυτω ο αδελφοϲ ϲου ηκει και εθυϲεν} & 10 &  &  \\
&  & 11 & \foreignlanguage{greek}{ο \textoverline{πηρ} ϲου τον μοϲχον τον ϲιτευτον οτι} & 18 &  &  \\
&  & 19 & \foreignlanguage{greek}{υγιαινοντα αυτον απελαβεν ωργι} & 1 & \textbf{28} &  \\
&  & 1 & \foreignlanguage{greek}{ϲθη δε και ουκ ηθελεν ειϲελθειν} & 6 &  &  \\
&  & 7 & \foreignlanguage{greek}{ο ουν \textoverline{πηρ} εξελθων παρεκαλει αυτον} & 12 &  &  \\
& \textbf{29} &  & \foreignlanguage{greek}{ο δε αποκριθειϲ ειπεν τω \textoverline{πρι} ιδου τοϲαυ} & 8 &  &  \\
&  & 8 & \foreignlanguage{greek}{τα ετη δουλευω ϲοι και ουδεποτε ϲου} & 14 &  &  \\
&  & 15 & \foreignlanguage{greek}{εντολην παρηλθον και εμοι ουδεπο} & 19 &  &  \\
&  & 19 & \foreignlanguage{greek}{τε εδωκαϲ εριφον ινα μετα των φιλων} & 25 &  &  \\
&  & 26 & \foreignlanguage{greek}{μου ευφρανθω οτε δε ο υιοϲ ϲου} & 5 & \textbf{30} &  \\
&  & 6 & \foreignlanguage{greek}{ουτοϲ ο καταφαγων ϲου τον βιον μετα} & 12 &  &  \\
&  & 13 & \foreignlanguage{greek}{πορνων ηλθεν εθυϲαϲ αυτω τον μοϲχο̅} & 18 &  &  \\
&  & 19 & \foreignlanguage{greek}{τον ϲιτιϲτον ο δε ειπεν αυτω} & 4 & \textbf{31} &  \\
&  & 5 & \foreignlanguage{greek}{τεκνον ϲυ παντοτε μετ εμου ει και πα̅} & 12 &  &  \\
&  & 12 & \foreignlanguage{greek}{τα τα εμα ϲα εϲτιν ευφρανθηναι δε} & 2 & \textbf{32} &  \\
&  & 3 & \foreignlanguage{greek}{και χαρηναι εδει οτι ο αδελφοϲ ϲου ουτοϲ} & 10 &  &  \\
&  & 11 & \foreignlanguage{greek}{νεκροϲ ην και ανεζηϲεν και απολω} & 16 &  &  \\
&  & 16 & \foreignlanguage{greek}{λωϲ και ευρεθη} & 18 &  &  \\
& \mygospelchapter &  & \foreignlanguage{greek}{ελεγεν δε και προϲ τουϲ μαθηταϲ εαυτου} & 7 &  &  \\
&  & 9 & \foreignlanguage{greek}{\textoverline{ανοϲ} τιϲ ην πλουϲιοϲ οϲ ειχεν οικονομον} & 15 &  &  \\
[0.2em]
\cline{4-4}
\end{tabular}
\end{center}
\end{table}
}
\clearpage
\newpage
 {
 \setlength\arrayrulewidth{1pt}
\begin{table}
\begin{center}
\begin{tabular}{ccc|l|ccc}
\cline{4-4} \\ [-1em]
\multicolumn{7}{c}{\foreignlanguage{greek}{ευαγγελιον κατα λουκαν} \textbf{(\nospace{16:1})} } \\ \\ [-1em] % Si on veut ajouter les bordures latérales, remplacer {7}{c} par {7}{|c|}
\cline{4-4} \\
\cline{4-4}
&  &  & &  &  & \\ [-0.9em]
&  & 16 & \foreignlanguage{greek}{και ουτοϲ διεβληθη αυτω ωϲ διαϲκορπι} & 21 &  &  \\
&  & 21 & \foreignlanguage{greek}{ζων τα υπαρχοντα αυτου και φωνηϲαϲ} & 2 & \textbf{2} &  \\
&  & 3 & \foreignlanguage{greek}{αυτον ειπεν αυτω τι τουτο ακουω περι ϲου} & 10 &  &  \\
&  & 11 & \foreignlanguage{greek}{αποδοϲ τον λογον τηϲ οικονομιαϲ ϲου ου} & 17 &  &  \\
&  & 18 & \foreignlanguage{greek}{γαρ δυνη ετι οικονομιν} & 21 &  &  \\
& \textbf{3} &  & \foreignlanguage{greek}{ειπεν δε αυτω ο οικονομοϲ τι ποιηϲω ο} & 8 &  &  \\
&  & 8 & \foreignlanguage{greek}{τι ο \textoverline{κϲ} μου αφερειται την οικονομιαν} & 14 &  &  \\
&  & 15 & \foreignlanguage{greek}{απ εμου ϲκαπτειν ουκ ιϲχυω επαι} & 20 &  &  \\
&  & 20 & \foreignlanguage{greek}{τειν αιϲχυνομαι εγνων τι ποιηϲω} & 3 & \textbf{4} &  \\
&  & 4 & \foreignlanguage{greek}{ινα οταν μεταϲταθω τηϲ οικονομιαϲ δε} & 9 &  &  \\
&  & 9 & \foreignlanguage{greek}{ξωνται με ειϲ τουϲ οικουϲ αυτων} & 14 &  &  \\
& \textbf{5} &  & \foreignlanguage{greek}{και προϲκαλεϲαμενοϲ ενα εκαϲτον των} & 5 &  &  \\
&  & 6 & \foreignlanguage{greek}{χρεωϲτων του \textoverline{κυ} εαυτου ελεγεν τω πρω} & 12 &  &  \\
&  & 12 & \foreignlanguage{greek}{τω ποϲον οφιλειϲ τω \textoverline{κω} μου ο δε ειπεν} & 3 & \textbf{6} &  \\
&  & 4 & \foreignlanguage{greek}{εκατον βαδουϲ ελαιου και ειπεν δεξε} & 9 &  &  \\
&  & 10 & \foreignlanguage{greek}{ϲου το γραμμα και καθειϲαϲ ταχεωϲ γρα} & 16 &  &  \\
&  & 16 & \foreignlanguage{greek}{ψον πεντηκοντα επειτα ετερω ειπε̅} & 3 & \textbf{7} &  \\
&  & 4 & \foreignlanguage{greek}{ϲυ δε ποϲον οφιλειϲ ο δε ειπεν εκατον} & 11 &  &  \\
&  & 12 & \foreignlanguage{greek}{κορουϲ ϲειτου και λεγει αυτω δεξε ϲου} & 18 &  &  \\
&  & 19 & \foreignlanguage{greek}{το γραμμα και γραψον ογδοηκοντα και} & 1 & \textbf{8} &  \\
&  & 2 & \foreignlanguage{greek}{επηνεϲεν ο \textoverline{κϲ} τον οικονομον τηϲ αδι} & 8 &  &  \\
&  & 8 & \foreignlanguage{greek}{κειαϲ οτι φρονιμωϲ εποιηϲεν οτι οι υιοι} & 14 &  &  \\
&  & 15 & \foreignlanguage{greek}{του αιωνοϲ τουτου φρονιμωτεροι υπερ} & 19 &  &  \\
&  & 20 & \foreignlanguage{greek}{τουϲ υιουϲ του φωτοϲ ειϲ την γενεαν την ε} & 28 &  &  \\
&  & 28 & \foreignlanguage{greek}{αυτων ειϲιν καγω υμιν λεγω ποιηϲα} & 4 & \textbf{9} &  \\
&  & 4 & \foreignlanguage{greek}{τε εαυτοιϲ φιλουϲ εκ του μαμωνα τηϲ α} & 12 &  &  \\
&  & 12 & \foreignlanguage{greek}{δικειαϲ ινα οταν εκλειπηται δεξωντε} & 16 &  &  \\
&  & 17 & \foreignlanguage{greek}{υμαϲ ειϲ ταϲ αιωνιουϲ ϲκηναϲ} & 21 &  &  \\
& \textbf{10} &  & \foreignlanguage{greek}{ο πιϲτοϲ εν ελαχιϲτω και εν πολλω πιϲτοϲ} & 8 &  &  \\
&  & 9 & \foreignlanguage{greek}{εϲτιν και ο εν ελαχιϲτω αδικοϲ και εν} & 16 &  &  \\
[0.2em]
\cline{4-4}
\end{tabular}
\end{center}
\end{table}
}
\clearpage
\newpage
 {
 \setlength\arrayrulewidth{1pt}
\begin{table}
\begin{center}
\begin{tabular}{ccc|l|ccc}
\cline{4-4} \\ [-1em]
\multicolumn{7}{c}{\foreignlanguage{greek}{ευαγγελιον κατα λουκαν} \textbf{(\nospace{16:10})} } \\ \\ [-1em] % Si on veut ajouter les bordures latérales, remplacer {7}{c} par {7}{|c|}
\cline{4-4} \\
\cline{4-4}
&  &  & &  &  & \\ [-0.9em]
&  & 17 & \foreignlanguage{greek}{πολλω αδικοϲ εϲτιν ει ουν εν τω αδικω} & 5 & \textbf{11} &  \\
&  & 6 & \foreignlanguage{greek}{μαμωνα πιϲτοι ουκ εγενεϲθαι το αληθεινο̅} & 11 &  &  \\
&  & 12 & \foreignlanguage{greek}{τιϲ υμιν πιϲτευϲει και ει εν τω αλλοτριω} & 5 & \textbf{12} &  \\
&  & 6 & \foreignlanguage{greek}{πιϲτοι ουκ εγενεϲθαι το υμετερον τιϲ υμι̅} & 12 &  &  \\
&  & 13 & \foreignlanguage{greek}{δωϲει ουδειϲ οικετηϲ δυναται δυ} & 4 & \textbf{13} &  \\
&  & 4 & \foreignlanguage{greek}{ϲι κυριοιϲ δουλευειν η γαρ τον ενα μειϲη} & 11 &  &  \\
&  & 11 & \foreignlanguage{greek}{ϲει και τον ετερον αγαπηϲει η ενοϲ ανθε} & 18 &  &  \\
&  & 18 & \foreignlanguage{greek}{ξεται και του ετερου καταφρονηϲει} & 22 &  &  \\
&  & 23 & \foreignlanguage{greek}{ου δυναϲθαι \textoverline{θω} δουλευειν και μαμωνα} & 28 &  &  \\
& \textbf{14} &  & \foreignlanguage{greek}{ηκουον δε ταυτα παντα και οι φαριϲαιοι} & 7 &  &  \\
&  & 8 & \foreignlanguage{greek}{φιλαργυροι υπαρχοντεϲ εξεμυκτηρι} & 10 &  &  \\
&  & 10 & \foreignlanguage{greek}{ζον αυτον και ειπεν αυτοιϲ υμειϲ εϲται} & 5 & \textbf{15} &  \\
&  & 6 & \foreignlanguage{greek}{οι δικαιουντεϲ εαυτουϲ ενωπιον των α̅} & 11 &  &  \\
&  & 11 & \foreignlanguage{greek}{θρωπων ο δε \textoverline{θϲ} γινωϲκει ταϲ καρδιαϲ υμω̅} & 18 &  &  \\
&  & 19 & \foreignlanguage{greek}{οτι το εν \textoverline{ανοιϲ} υψηλον βδελυϲμα ενωπι} & 25 &  &  \\
&  & 25 & \foreignlanguage{greek}{ον του \textoverline{θυ} ο νομοϲ και οι προφηται εωϲ} & 6 & \textbf{16} &  \\
&  & 7 & \foreignlanguage{greek}{ιωαννου απο τοτε η βαϲιλεια του \textoverline{θυ} ευ} & 14 &  &  \\
&  & 14 & \foreignlanguage{greek}{αγγελιζεται και παϲ ειϲ αυτην βιαζεται} & 19 &  &  \\
& \textbf{17} &  & \foreignlanguage{greek}{ευκοπωτερον δε εϲτιν τον ουρανον και} & 6 &  &  \\
&  & 7 & \foreignlanguage{greek}{την γην παρελθειν η του νομου μιαν} & 13 &  &  \\
&  & 14 & \foreignlanguage{greek}{κερεαν παρελθειν} & 15 &  &  \\
& \textbf{18} &  & \foreignlanguage{greek}{παϲ ο απολυων την γυναικα αυτου και} & 7 &  &  \\
&  & 8 & \foreignlanguage{greek}{γαμων ετεραν μοιχευει και παϲ ο απολε} & 14 &  &  \\
&  & 14 & \foreignlanguage{greek}{λυμενην απο ανδροϲ γαμων μοιχευει} & 18 &  &  \\
& \textbf{19} &  & \foreignlanguage{greek}{ανθρωποϲ δε τιϲ ην πλουϲιοϲ και ενεδιδυ} & 7 &  &  \\
&  & 7 & \foreignlanguage{greek}{ϲκετο πορφυραν και βυϲϲον ευφραινομε} & 11 &  &  \\
&  & 11 & \foreignlanguage{greek}{νοϲ καθ ημεραν λαμπρωϲ πτωχοϲ δε} & 2 & \textbf{20} &  \\
&  & 3 & \foreignlanguage{greek}{τιϲ ην ονοματι λαζαροϲ οϲ εβεβλητο προϲ} & 9 &  &  \\
&  & 10 & \foreignlanguage{greek}{τον πυλωνα αυτου ειλκωμενοϲ και επι} & 2 & \textbf{21} &  \\
&  & 2 & \foreignlanguage{greek}{θυμων χορταϲθηναι απο των ψιχιων τω̅} & 7 &  &  \\
[0.2em]
\cline{4-4}
\end{tabular}
\end{center}
\end{table}
}
\clearpage
\newpage
 {
 \setlength\arrayrulewidth{1pt}
\begin{table}
\begin{center}
\begin{tabular}{ccc|l|ccc}
\cline{4-4} \\ [-1em]
\multicolumn{7}{c}{\foreignlanguage{greek}{ευαγγελιον κατα λουκαν} \textbf{(\nospace{16:21})} } \\ \\ [-1em] % Si on veut ajouter les bordures latérales, remplacer {7}{c} par {7}{|c|}
\cline{4-4} \\
\cline{4-4}
&  &  & &  &  & \\ [-0.9em]
&  & 8 & \foreignlanguage{greek}{πιπτοντων απο τηϲ τραπεζηϲ του πλουϲιου} & 13 &  &  \\
&  & 14 & \foreignlanguage{greek}{αλλα και οι κυνεϲ ερχομενοι απελιχαν τα} & 20 &  &  \\
&  & 21 & \foreignlanguage{greek}{ελκη αυτου εγενετο δε αποθανειν τον} & 4 & \textbf{22} &  \\
&  & 5 & \foreignlanguage{greek}{πτωχον και απενεχθηναι αυτον υπο τω̅} & 10 &  &  \\
&  & 11 & \foreignlanguage{greek}{αγγελων ειϲ τον κολπον του αβρααμ} & 16 &  &  \\
&  & 17 & \foreignlanguage{greek}{απεθανεν δε και ο πλουϲιοϲ και εταφη ϗ} & 1 & \textbf{23} &  \\
&  & 2 & \foreignlanguage{greek}{εν τω αδη επαραϲ τουϲ οφθαλμουϲ αυτου} & 8 &  &  \\
&  & 9 & \foreignlanguage{greek}{υπαρχων εν βαϲανοιϲ ορα τον αβρααμ απο} & 15 &  &  \\
&  & 16 & \foreignlanguage{greek}{μακροθεν και λαζαρον εν τοιϲ κολποιϲ αυτου} & 22 &  &  \\
& \textbf{24} &  & \foreignlanguage{greek}{και αυτοϲ φωνηϲαϲ ειπεν \textoverline{περ} αβρααμ} & 6 &  &  \\
&  & 7 & \foreignlanguage{greek}{ελεηϲον με και πεμψον λαζαρον ινα βα} & 13 &  &  \\
&  & 13 & \foreignlanguage{greek}{ψη το ακρον του δακτυλου αυτου υδατοϲ} & 19 &  &  \\
&  & 20 & \foreignlanguage{greek}{και καταψυξη την γλωϲϲαν μου οτι οδυ} & 26 &  &  \\
&  & 26 & \foreignlanguage{greek}{νωμαι εν τη φλογει ταυτη} & 30 &  &  \\
& \textbf{25} &  & \foreignlanguage{greek}{ειπεν δε αβρααμ τεκνον μνηϲθητι} & 5 &  &  \\
&  & 6 & \foreignlanguage{greek}{οτι απελαβεϲ ϲυ τα αγαθα ϲου εν τη ζωη ϲου} & 15 &  &  \\
&  & 16 & \foreignlanguage{greek}{και λαζαροϲ ομοιωϲ τα κακα νυν δε ωδε} & 23 &  &  \\
&  & 24 & \foreignlanguage{greek}{παρακαλειται ϲυ δε οδυναϲαι και επι} & 2 & \textbf{26} &  \\
&  & 3 & \foreignlanguage{greek}{παϲιν τουτοιϲ μεταξυ υμων και ημων} & 8 &  &  \\
&  & 9 & \foreignlanguage{greek}{χαϲμα μεγα εϲτηρικτε οπωϲ οι θελοντεϲ} & 14 &  &  \\
&  & 15 & \foreignlanguage{greek}{διαβηναι προϲ υμαϲ μη δυνωνται μηδε} & 20 &  &  \\
&  & 21 & \foreignlanguage{greek}{οι εκειθεν προϲ ημαϲ διαπερωϲιν} & 25 &  &  \\
& \textbf{27} &  & \foreignlanguage{greek}{ειπεν δε ερωτω ϲε \textoverline{περ} ινα πεμψηϲ αυτο̅} & 8 &  &  \\
&  & 9 & \foreignlanguage{greek}{ειϲ τον οικον του \textoverline{πρϲ} μου εχω γαρ πεντε} & 3 & \textbf{28} &  \\
&  & 4 & \foreignlanguage{greek}{αδελφουϲ οπωϲ διαμαρτυρηται αυτοιϲ} & 7 &  &  \\
&  & 8 & \foreignlanguage{greek}{ινα μη και αυτοι ελθωϲιν ειϲ τον τοπον του} & 16 &  &  \\
&  & 16 & \foreignlanguage{greek}{τον τηϲ βαϲανου} & 18 &  &  \\
& \textbf{29} &  & \foreignlanguage{greek}{λεγει δε αυτω ο αβρααμ εχουϲιν μωϲεα} & 7 &  &  \\
&  & 8 & \foreignlanguage{greek}{και τουϲ προφηταϲ ακουϲατωϲαν αυτων} & 12 &  &  \\
& \textbf{30} &  & \foreignlanguage{greek}{ο δε ειπεν ουχει \textoverline{περ} αβρααμ αλλ εαν τιϲ} & 9 &  &  \\
[0.2em]
\cline{4-4}
\end{tabular}
\end{center}
\end{table}
}
\clearpage
\newpage
 {
 \setlength\arrayrulewidth{1pt}
\begin{table}
\begin{center}
\begin{tabular}{ccc|l|ccc}
\cline{4-4} \\ [-1em]
\multicolumn{7}{c}{\foreignlanguage{greek}{ευαγγελιον κατα λουκαν} \textbf{(\nospace{16:30})} } \\ \\ [-1em] % Si on veut ajouter les bordures latérales, remplacer {7}{c} par {7}{|c|}
\cline{4-4} \\
\cline{4-4}
&  &  & &  &  & \\ [-0.9em]
&  & 10 & \foreignlanguage{greek}{απο νεκρων πορευθη προϲ αυτουϲ μετα} & 15 &  &  \\
&  & 15 & \foreignlanguage{greek}{νοηϲουϲιν} & 15 &  &  \\
& \textbf{31} &  & \foreignlanguage{greek}{ειπεν δε αυτω ει μωυϲεωϲ και των προ} & 8 &  &  \\
&  & 8 & \foreignlanguage{greek}{φητων ουκ ακουουϲιν ουδε εαν τιϲ εκ} & 14 &  &  \\
&  & 15 & \foreignlanguage{greek}{νεκρων απελθη πιϲτευουϲιν} & 17 &  &  \\
& \mygospelchapter &  & \foreignlanguage{greek}{ειπεν δε προϲ τουϲ μαθηταϲ ανενδεκτο̅} & 6 &  &  \\
&  & 7 & \foreignlanguage{greek}{εϲτιν του μη ελθειν τα ϲκανδαλα ουδε δι ου} & 15 &  &  \\
&  & 16 & \foreignlanguage{greek}{ερχεται λυϲιτελει αυτω ει λιθοϲ ονικοϲ πε} & 6 & \textbf{2} &  \\
&  & 6 & \foreignlanguage{greek}{ρικειτε περι τον τραχηλον αυτου και εριπτε} & 12 &  &  \\
&  & 13 & \foreignlanguage{greek}{ειϲ την θαλαϲϲαν η ινα ϲκανδαλιϲη ενα τω̅} & 20 &  &  \\
&  & 21 & \foreignlanguage{greek}{μικρων τουτων προϲεχεται εαυτοιϲ} & 2 & \textbf{3} &  \\
&  & 3 & \foreignlanguage{greek}{εαν δε αμαρτη ο αδελφοϲ ϲου επιτιμηϲον αυ} & 10 &  &  \\
&  & 10 & \foreignlanguage{greek}{τω και εαν μετανοηϲη αφεϲ αυτω και εα̅} & 2 & \textbf{4} &  \\
&  & 3 & \foreignlanguage{greek}{επτακειϲ τηϲ ημεραϲ αμαρτηϲη ειϲ ϲε και} & 9 &  &  \\
&  & 10 & \foreignlanguage{greek}{επτακειϲ τηϲ ημεραϲ επιϲτρεψη λεγων} & 14 &  &  \\
&  & 15 & \foreignlanguage{greek}{μετανοω αφηϲιϲ αυτω} & 17 &  &  \\
& \textbf{5} &  & \foreignlanguage{greek}{και ειπον οι αποϲτολοι τω \textoverline{κω} προϲθεϲ ημι̅} & 8 &  &  \\
&  & 9 & \foreignlanguage{greek}{πιϲτιν ειπεν δε ο \textoverline{κϲ} ει εχεται πιϲτιν} & 7 & \textbf{6} &  \\
&  & 8 & \foreignlanguage{greek}{ωϲ κοκκον ϲινηπεωϲ ελεγεται αν τη ϲυ} & 14 &  &  \\
&  & 14 & \foreignlanguage{greek}{καμινω ταυτη εκριζωθητι και φυτευ} & 18 &  &  \\
&  & 18 & \foreignlanguage{greek}{θητι εν τη θαλαϲϲη και υπηκουϲεν αν υμι̅} & 25 &  &  \\
& \textbf{7} &  & \foreignlanguage{greek}{τιϲ δε εξ υμων δουλον εχων αροτριωντα} & 7 &  &  \\
&  & 8 & \foreignlanguage{greek}{η ποιμαινοντα οϲ ειϲελθοντι εκ του αγρου} & 14 &  &  \\
&  & 15 & \foreignlanguage{greek}{ερι ευθεωϲ παρελθων αναπεϲε αλλ ουχι} & 2 & \textbf{8} &  \\
&  & 3 & \foreignlanguage{greek}{ερει αυτω ετοιμαϲον τι διπνωϲω και πε} & 9 &  &  \\
&  & 9 & \foreignlanguage{greek}{ριζωϲαμενοϲ διακονει μοι εωϲ φαγω ϗ} & 14 &  &  \\
&  & 15 & \foreignlanguage{greek}{πιω και μετα ταυτα φαγεϲε και πιεϲε ϲυ} & 22 &  &  \\
& \textbf{9} &  & \foreignlanguage{greek}{μη χαριν εχει τω δουλω εκεινω οτι εποι} & 8 &  &  \\
&  & 8 & \foreignlanguage{greek}{ηϲεν τα διαταχθεντα ου δοκω ουτωϲ} & 1 & \textbf{10} &  \\
&  & 2 & \foreignlanguage{greek}{και υμειϲ οταν ποιηϲηται παντα τα δια} & 8 &  &  \\
[0.2em]
\cline{4-4}
\end{tabular}
\end{center}
\end{table}
}
\clearpage
\newpage
 {
 \setlength\arrayrulewidth{1pt}
\begin{table}
\begin{center}
\begin{tabular}{ccc|l|ccc}
\cline{4-4} \\ [-1em]
\multicolumn{7}{c}{\foreignlanguage{greek}{ευαγγελιον κατα λουκαν} \textbf{(\nospace{17:10})} } \\ \\ [-1em] % Si on veut ajouter les bordures latérales, remplacer {7}{c} par {7}{|c|}
\cline{4-4} \\
\cline{4-4}
&  &  & &  &  & \\ [-0.9em]
&  & 8 & \foreignlanguage{greek}{ταχθεντα υμιν λεγεται δουλοι αχριοι εϲμε̅} & 13 &  &  \\
&  & 14 & \foreignlanguage{greek}{οτι ο οφιλομεν ποιηϲαι πεποιηκαμεν} & 18 &  &  \\
& \textbf{11} &  & \foreignlanguage{greek}{και εγενετο εν τω πορευεϲθαι αυτον ειϲ ιε} & 8 &  &  \\
&  & 8 & \foreignlanguage{greek}{ρουϲαλημ και αυτοϲ διερχεται δια μεϲου} & 13 &  &  \\
&  & 14 & \foreignlanguage{greek}{ϲαμαριαϲ και γαλιλαιαϲ και ειϲερχομε} & 2 & \textbf{12} &  \\
&  & 2 & \foreignlanguage{greek}{νου αυτου ειϲ τινα κωμην απηντηϲαν αυ} & 8 &  &  \\
&  & 8 & \foreignlanguage{greek}{τω δεκα λεπροι ανδρεϲ οι εϲτηϲαν πορρω} & 14 &  &  \\
& \textbf{13} &  & \foreignlanguage{greek}{και αυτοι ηραν φωνην λεγοντεϲ \textoverline{ιυ} επι} & 7 &  &  \\
&  & 7 & \foreignlanguage{greek}{ϲτατα ελεηϲον ημαϲ και ιδων ειπεν αυ} & 4 & \textbf{14} &  \\
&  & 4 & \foreignlanguage{greek}{τοιϲ πορευθεντεϲ επιδειξαται εαυτουϲ} & 7 &  &  \\
&  & 8 & \foreignlanguage{greek}{τοιϲ ιερευϲιν και εγενετο εν τω υπαγει̅} & 14 &  &  \\
&  & 15 & \foreignlanguage{greek}{αυτουϲ εκαθαριϲθηϲαν ειϲ δε εξ αυτω̅} & 4 & \textbf{15} &  \\
&  & 5 & \foreignlanguage{greek}{ιδων οτι ειαθη υπεϲτρεψεν μετα φω} & 10 &  &  \\
&  & 10 & \foreignlanguage{greek}{νηϲ μεγαληϲ δοξαζων τον \textoverline{θν} και επεϲε̅} & 2 & \textbf{16} &  \\
&  & 3 & \foreignlanguage{greek}{επι προϲωπον παρα τουϲ ποδαϲ αυτου ευχα} & 9 &  &  \\
&  & 9 & \foreignlanguage{greek}{ριϲτων αυτω και αυτοϲ ην ϲαμαριτηϲ} & 14 &  &  \\
& \textbf{17} &  & \foreignlanguage{greek}{αποκριθειϲ δε ο \textoverline{ιϲ} ειπεν ουχ οι δεκα ουτοι ε} & 10 &  &  \\
&  & 10 & \foreignlanguage{greek}{καθαριϲθηϲαν οι δε εννεα που ουχ ευρεθη} & 2 & \textbf{18} &  \\
&  & 2 & \foreignlanguage{greek}{ϲαν υποϲτρεψαντεϲ δουναι δοξαν τω \textoverline{θω}} & 7 &  &  \\
&  & 8 & \foreignlanguage{greek}{ει μη ο αλλογενηϲ ουτοϲ και ειπεν αυτω} & 3 & \textbf{19} &  \\
&  & 4 & \foreignlanguage{greek}{αναϲταϲ πορευου η πιϲτιϲ ϲου ϲεϲωκεν ϲε} & 10 &  &  \\
& \textbf{20} &  & \foreignlanguage{greek}{επερωτηθειϲ δε υπο των φαριϲαιων} & 5 &  &  \\
&  & 7 & \foreignlanguage{greek}{ποτε ερχεται η βαϲιλεια του \textoverline{θυ} απεκρι} & 13 &  &  \\
&  & 13 & \foreignlanguage{greek}{θη αυτοιϲ και ειπεν ουκ ερχεται η βαϲιλεια} & 20 &  &  \\
&  & 21 & \foreignlanguage{greek}{του \textoverline{θυ} μετα παρατηρηϲεωϲ ουδε ερουϲι̅} & 2 & \textbf{21} &  \\
&  & 3 & \foreignlanguage{greek}{ιδου ωδε και ιδου εκει ιδου γαρ η βαϲιλει} & 11 &  &  \\
&  & 11 & \foreignlanguage{greek}{α του \textoverline{θυ} εντοϲ υμων εϲτιν} & 16 &  &  \\
& \textbf{22} &  & \foreignlanguage{greek}{ειπεν δε προϲ τουϲ μαθηταϲ ελευϲονται} & 6 &  &  \\
&  & 7 & \foreignlanguage{greek}{ημεραι οτε επιθυμηϲεται μιαν των ημε} & 12 &  &  \\
&  & 12 & \foreignlanguage{greek}{ρων του υιου του \textoverline{ανου} ιδιν ϗ ουχ οψεϲθαι} & 20 &  &  \\
[0.2em]
\cline{4-4}
\end{tabular}
\end{center}
\end{table}
}
\clearpage
\newpage
 {
 \setlength\arrayrulewidth{1pt}
\begin{table}
\begin{center}
\begin{tabular}{ccc|l|ccc}
\cline{4-4} \\ [-1em]
\multicolumn{7}{c}{\foreignlanguage{greek}{ευαγγελιον κατα λουκαν} \textbf{(\nospace{17:23})} } \\ \\ [-1em] % Si on veut ajouter les bordures latérales, remplacer {7}{c} par {7}{|c|}
\cline{4-4} \\
\cline{4-4}
&  &  & &  &  & \\ [-0.9em]
& \textbf{23} &  & \foreignlanguage{greek}{και ερουϲιν υμιν ιδου ωδε ιδου εκει μη απελ} & 9 &  &  \\
&  & 9 & \foreignlanguage{greek}{θητε μηδε διωξηται ωϲπερ γαρ η αϲτραπη} & 4 & \textbf{24} &  \\
&  & 5 & \foreignlanguage{greek}{αϲτραπτουϲα εκ τηϲ υπο τον ουρανον ειϲ την} & 12 &  &  \\
&  & 13 & \foreignlanguage{greek}{υπ ουρανον λαμπει ουτωϲ εϲται ο υιοϲ του} & 20 &  &  \\
&  & 21 & \foreignlanguage{greek}{\textoverline{ανου} εν τη ημερα αυτου πρωτον δε δι αυτο̅} & 4 & \textbf{25} &  \\
&  & 5 & \foreignlanguage{greek}{πολλα παθειν και αποδοκιμαϲθηναι απο} & 9 &  &  \\
&  & 10 & \foreignlanguage{greek}{τηϲ γενεαϲ ταυτηϲ και καθωϲ εγενετο} & 3 & \textbf{26} &  \\
&  & 4 & \foreignlanguage{greek}{εν ταιϲ ημεραιϲ νωε ουτωϲ εϲται και εν} & 11 &  &  \\
&  & 12 & \foreignlanguage{greek}{ταιϲ ημεραιϲ του υιου του \textoverline{ανου}} & 17 &  &  \\
& \textbf{27} &  & \foreignlanguage{greek}{ηϲθειον επινον εγαμουν εξεγαμιζοντο} & 4 &  &  \\
&  & 5 & \foreignlanguage{greek}{αχρι ηϲ ημεραϲ ειϲηλθεν νωε ειϲ την κιβω} & 12 &  &  \\
&  & 12 & \foreignlanguage{greek}{τον και ηλθεν ο κατακλυϲμοϲ και απωλε} & 18 &  &  \\
&  & 18 & \foreignlanguage{greek}{ϲεν απανταϲ ομοιωϲ και ωϲ εγενετο} & 4 & \textbf{28} &  \\
&  & 5 & \foreignlanguage{greek}{εν ταιϲ ημεραιϲ λωτ ηϲθιον επινον ηγο} & 11 &  &  \\
&  & 11 & \foreignlanguage{greek}{ραζον επωλουν εφυτευον ωκοδομουν} & 14 &  &  \\
& \textbf{29} &  & \foreignlanguage{greek}{η δε ημερα εξηλθεν λωθ απο ϲοδομων} & 7 &  &  \\
&  & 8 & \foreignlanguage{greek}{εβρεξεν θειον και πυρ απ ουρανου και α} & 15 &  &  \\
&  & 15 & \foreignlanguage{greek}{πωλεϲεν απανταϲ κατα ταυτα εϲται} & 3 & \textbf{30} &  \\
&  & 4 & \foreignlanguage{greek}{η ημερα ο υιοϲ του \textoverline{ανου} αποκαλυπτεται} & 10 &  &  \\
& \textbf{31} &  & \foreignlanguage{greek}{εν εκεινη τη ημερα οϲ εϲτιν επι του δωμα} & 9 &  &  \\
&  & 9 & \foreignlanguage{greek}{τοϲ και τα ϲκευη αυτου εν τη οικεια μη κα} & 18 &  &  \\
&  & 18 & \foreignlanguage{greek}{ταβατω αραι αυτα και ο εν τω αγρω ομοιωϲ} & 26 &  &  \\
&  & 27 & \foreignlanguage{greek}{μη επιϲτρεψατω ειϲ τα οπιϲω μνημονευ} & 1 & \textbf{32} &  \\
&  & 1 & \foreignlanguage{greek}{εται τηϲ γυναικοϲ λωθ} & 4 &  &  \\
& \textbf{33} &  & \foreignlanguage{greek}{οϲ εαν ζητηϲη την ψυχην αυτου ϲωϲαι απο} & 8 &  &  \\
&  & 8 & \foreignlanguage{greek}{λεϲη αυτην και οϲ εαν απολεϲη αυτην} & 14 &  &  \\
&  & 15 & \foreignlanguage{greek}{ζωογονηϲει αυτην λεγω υμιν} & 2 & \textbf{34} &  \\
&  & 3 & \foreignlanguage{greek}{αυτη τη νυκτι δυο εϲονται επι κλεινηϲ} & 9 &  &  \\
&  & 10 & \foreignlanguage{greek}{μιαϲ ειϲ παραλημφθηϲεται και ο ετεροϲ} & 15 &  &  \\
&  & 16 & \foreignlanguage{greek}{αφεθηϲεται} & 16 &  &  \\
[0.2em]
\cline{4-4}
\end{tabular}
\end{center}
\end{table}
}
\clearpage
\newpage
 {
 \setlength\arrayrulewidth{1pt}
\begin{table}
\begin{center}
\begin{tabular}{ccc|l|ccc}
\cline{4-4} \\ [-1em]
\multicolumn{7}{c}{\foreignlanguage{greek}{ευαγγελιον κατα λουκαν} \textbf{(\nospace{17:35})} } \\ \\ [-1em] % Si on veut ajouter les bordures latérales, remplacer {7}{c} par {7}{|c|}
\cline{4-4} \\
\cline{4-4}
&  &  & &  &  & \\ [-0.9em]
& \textbf{35} &  & \foreignlanguage{greek}{δυο εϲονται αληθουϲαι επι το αυτο μια} & 7 &  &  \\
&  & 8 & \foreignlanguage{greek}{παραλημφθηϲεται και η ετερα αφε} & 12 &  &  \\
&  & 12 & \foreignlanguage{greek}{θηϲεται} & 12 &  &  \\
& \textbf{37} &  & \foreignlanguage{greek}{αυτω που \textoverline{κε} ο δε ειπεν αυτοιϲ} & 10 &  &  \\
&  & 11 & \foreignlanguage{greek}{οπου το ϲωμα εκει ϲυναχθηϲοντε οι αετοι} & 17 &  &  \\
& \mygospelchapter &  & \foreignlanguage{greek}{ελεγεν δε και παραβολην αυτοιϲ προϲ} & 6 &  &  \\
&  & 7 & \foreignlanguage{greek}{το δειν παντοτε προϲευχεϲθαι αυτουϲ} & 11 &  &  \\
&  & 12 & \foreignlanguage{greek}{και μη εκκακειν λεγων} & 1 & \textbf{2} &  \\
&  & 2 & \foreignlanguage{greek}{κριτηϲ τιϲ ην εν τινι πολει τον \textoverline{θν} μη φο} & 11 &  &  \\
&  & 11 & \foreignlanguage{greek}{βουμενοϲ και \textoverline{ανουϲ} μη εντρεπομενοϲ} & 15 &  &  \\
& \textbf{3} &  & \foreignlanguage{greek}{χηρα δε ην εν τη πολει εκεινη και ηρχε} & 9 &  &  \\
&  & 9 & \foreignlanguage{greek}{το προϲ αυτον λεγουϲα εκδικηϲον με} & 14 &  &  \\
&  & 15 & \foreignlanguage{greek}{απο του αντιδικου μου και ουκ ηθελε̅} & 3 & \textbf{4} &  \\
&  & 4 & \foreignlanguage{greek}{επι χρονον μετα δε ταυτα ειπεν εν ε} & 11 &  &  \\
&  & 11 & \foreignlanguage{greek}{αυτω ει και τον \textoverline{θν} ου φοβουμαι και αν} & 19 &  &  \\
&  & 19 & \foreignlanguage{greek}{θρωπον ουκ εντρεπομαι δια γε το παρε} & 4 & \textbf{5} &  \\
&  & 4 & \foreignlanguage{greek}{χειν κοπον την χηραν ταυτην εκδικη} & 9 &  &  \\
&  & 9 & \foreignlanguage{greek}{ϲω αυτην ινα μη ειϲ τελοϲ ερχομενη υ} & 16 &  &  \\
&  & 16 & \foreignlanguage{greek}{ποπταζη με ειπεν δε ο \textoverline{κϲ}} & 4 & \textbf{6} &  \\
&  & 5 & \foreignlanguage{greek}{ακουϲατε τι ο κριτηϲ τηϲ αδικειαϲ λεγει} & 11 &  &  \\
& \textbf{7} &  & \foreignlanguage{greek}{ο δε \textoverline{θϲ} ου μη ποιηϲει την εκδικηϲιν τω̅} & 9 &  &  \\
&  & 10 & \foreignlanguage{greek}{εκλεκτων αυτου των βοωντων προϲ} & 14 &  &  \\
&  & 15 & \foreignlanguage{greek}{αυτον ημεραϲ και νυκτοϲ και μακρο} & 20 &  &  \\
&  & 20 & \foreignlanguage{greek}{θυμων επ αυτοιϲ λεγω υμιν οτι ποι} & 4 & \textbf{8} &  \\
&  & 4 & \foreignlanguage{greek}{ηϲει την εκδικηϲιν αυτων εν ταχει} & 9 &  &  \\
&  & 10 & \foreignlanguage{greek}{πλην ο υιοϲ του \textoverline{ανου} ελθων αρα ευρη} & 17 &  &  \\
&  & 17 & \foreignlanguage{greek}{ϲει την πιϲτιν επι τηϲ γηϲ} & 22 &  &  \\
& \textbf{9} &  & \foreignlanguage{greek}{ειπεν δε προϲ τιναϲ τουϲ πεποιθοταϲ} & 6 &  &  \\
&  & 7 & \foreignlanguage{greek}{εφ εαυτοιϲ οτι ειϲιν δικαιοι και εξου} & 13 &  &  \\
&  & 13 & \foreignlanguage{greek}{θενουνταϲ τουϲ λοιπουϲ την παραβολην} & 17 &  &  \\
&  & 18 & \foreignlanguage{greek}{ταυτην} & 18 &  &  \\
[0.2em]
\cline{4-4}
\end{tabular}
\end{center}
\end{table}
}
\clearpage
\newpage
 {
 \setlength\arrayrulewidth{1pt}
\begin{table}
\begin{center}
\begin{tabular}{ccc|l|ccc}
\cline{4-4} \\ [-1em]
\multicolumn{7}{c}{\foreignlanguage{greek}{ευαγγελιον κατα λουκαν} \textbf{(\nospace{18:10})} } \\ \\ [-1em] % Si on veut ajouter les bordures latérales, remplacer {7}{c} par {7}{|c|}
\cline{4-4} \\
\cline{4-4}
&  &  & &  &  & \\ [-0.9em]
& \textbf{10} &  & \foreignlanguage{greek}{ανθρωποι δυο ανεβηϲαν ειϲ το ιερον προϲ} & 7 &  &  \\
&  & 7 & \foreignlanguage{greek}{ευξαϲθαι ο ειϲ φαριϲαιοϲ και ο ετεροϲ τε} & 14 &  &  \\
&  & 14 & \foreignlanguage{greek}{λωνηϲ ο φαριϲαιοϲ ϲταθειϲ προϲ εαυτον} & 5 & \textbf{11} &  \\
&  & 6 & \foreignlanguage{greek}{ταυτα προϲευχεται ο \textoverline{θϲ} ευχαριϲτω ϲοι} & 11 &  &  \\
&  & 12 & \foreignlanguage{greek}{οτι ουκ ειμει ωϲπερ οι λοιποι των \textoverline{ανων}} & 19 &  &  \\
&  & 20 & \foreignlanguage{greek}{αρπαγεϲ αδικοι μοιχοι η και ωϲ ουτοϲ} & 26 &  &  \\
&  & 27 & \foreignlanguage{greek}{ο τελωνηϲ νηϲτευω δειϲ του ϲαββατου} & 4 & \textbf{12} &  \\
&  & 5 & \foreignlanguage{greek}{αποδεκατω παντα οϲα κτωμαι} & 8 &  &  \\
& \textbf{13} &  & \foreignlanguage{greek}{και ο τελωνηϲ μακροθεν εϲτωϲ ουκ ηδυνα} & 7 &  &  \\
&  & 7 & \foreignlanguage{greek}{το ουδε τουϲ οφθαλμουϲ ειϲ τον ουρανον ε} & 14 &  &  \\
&  & 14 & \foreignlanguage{greek}{παρε αλλ ετυπτεν ειϲ το ϲτηθοϲ αυτου} & 20 &  &  \\
&  & 21 & \foreignlanguage{greek}{λεγων ο \textoverline{θϲ} ειλαϲθητι μοι τω αμαρτωλω} & 27 &  &  \\
& \textbf{14} &  & \foreignlanguage{greek}{λεγω υμιν κατεβη ουτοϲ δεδικαιω} & 5 &  &  \\
&  & 5 & \foreignlanguage{greek}{μενοϲ ειϲ τον οικον αυτου η εκεινοϲ} & 11 &  &  \\
&  & 12 & \foreignlanguage{greek}{οτι παϲ ο υψων εαυτον ταπινωθη} & 17 &  &  \\
&  & 17 & \foreignlanguage{greek}{ϲεται ο δε ταπινων εαυτον} & 21 &  &  \\
&  & 22 & \foreignlanguage{greek}{υψωθηϲεται} & 22 &  &  \\
& \textbf{15} &  & \foreignlanguage{greek}{προϲεφερον δε αυτω και τα βρεφη ινα αυ} & 8 &  &  \\
&  & 8 & \foreignlanguage{greek}{των απτηται ιδοντεϲ δε οι μαθηται} & 13 &  &  \\
&  & 14 & \foreignlanguage{greek}{επετιμηϲαν αυτοιϲ ο δε \textoverline{ιϲ} προϲκαλε} & 4 & \textbf{16} &  \\
&  & 4 & \foreignlanguage{greek}{ϲαμενοϲ αυτα ειπεν αφεται τα παιδια ερ} & 10 &  &  \\
&  & 10 & \foreignlanguage{greek}{χεϲθαι προϲ εμε και μη κωλυεται αυτα} & 17 &  &  \\
&  & 18 & \foreignlanguage{greek}{των γαρ τοιουτων εϲτιν η βαϲιλεια του \textoverline{θυ}} & 25 &  &  \\
& \textbf{17} &  & \foreignlanguage{greek}{αμην λεγω υμιν οϲ αν μη δεξηται την βα} & 9 &  &  \\
&  & 9 & \foreignlanguage{greek}{ϲιλειαν του \textoverline{θυ} ωϲ παιδιον ου μη ειϲελθη} & 16 &  &  \\
&  & 17 & \foreignlanguage{greek}{ειϲ αυτην και επηρωτηϲεν τιϲ αυτον} & 4 & \textbf{18} &  \\
&  & 5 & \foreignlanguage{greek}{αρχων λεγων διδαϲκαλε αγαθε τι ποι} & 10 &  &  \\
&  & 10 & \foreignlanguage{greek}{ηϲαϲ ζωην αιωνιον κληρονομηϲω} & 13 &  &  \\
& \textbf{19} &  & \foreignlanguage{greek}{ειπεν δε αυτω ο \textoverline{ιϲ} τι με λεγειϲ αγαθον} & 9 &  &  \\
&  & 10 & \foreignlanguage{greek}{ουδειϲ αγαθοϲ ει μη ειϲ ο \textoverline{θϲ}} & 16 &  &  \\
[0.2em]
\cline{4-4}
\end{tabular}
\end{center}
\end{table}
}
\clearpage
\newpage
 {
 \setlength\arrayrulewidth{1pt}
\begin{table}
\begin{center}
\begin{tabular}{ccc|l|ccc}
\cline{4-4} \\ [-1em]
\multicolumn{7}{c}{\foreignlanguage{greek}{ευαγγελιον κατα λουκαν} \textbf{(\nospace{18:20})} } \\ \\ [-1em] % Si on veut ajouter les bordures latérales, remplacer {7}{c} par {7}{|c|}
\cline{4-4} \\
\cline{4-4}
&  &  & &  &  & \\ [-0.9em]
& \textbf{20} &  & \foreignlanguage{greek}{ταϲ εντολαϲ οιδαϲ μη μοιχευϲηϲ μη φο} & 7 &  &  \\
&  & 7 & \foreignlanguage{greek}{νευϲηϲ μη κλεψηϲ μη ψευδομαρτυρηϲηϲ} & 11 &  &  \\
&  & 12 & \foreignlanguage{greek}{τιμα τον \textoverline{πρα} ϲου και την μητερα ο δε ειπε̅} & 3 & \textbf{21} &  \\
&  & 4 & \foreignlanguage{greek}{ταυτα παντα εφυλαξαμην εκ νεοτητοϲ μου} & 9 &  &  \\
& \textbf{22} &  & \foreignlanguage{greek}{ακουϲαϲ δε ταυτα ο \textoverline{ιϲ} ειπεν αυτω ετι εν ϲοι} & 10 &  &  \\
&  & 11 & \foreignlanguage{greek}{λιπει παντα οϲα εχειϲ πωληϲον και διαδοϲ} & 17 &  &  \\
&  & 18 & \foreignlanguage{greek}{πτωχοιϲ και εξειϲ θηϲαυρον εν ουρανω} & 23 &  &  \\
&  & 24 & \foreignlanguage{greek}{και δευρο ακολουθει μοι} & 27 &  &  \\
& \textbf{23} &  & \foreignlanguage{greek}{ο δε ακουϲαϲ ταυτα περιλυποϲ εγενετο} & 6 &  &  \\
&  & 7 & \foreignlanguage{greek}{ην γαρ πλουϲιοϲ ϲφοδρα ιδων δε αυτο̅} & 3 & \textbf{24} &  \\
&  & 4 & \foreignlanguage{greek}{ο \textoverline{ιϲ} περιλυπον γενομενον ειπεν πωϲ} & 9 &  &  \\
&  & 10 & \foreignlanguage{greek}{δυϲκολωϲ οι τα χρηματα εχοντεϲ ειϲελευ} & 15 &  &  \\
&  & 15 & \foreignlanguage{greek}{ϲονται ειϲ την βαϲιλειαν του \textoverline{θυ}} & 20 &  &  \\
& \textbf{25} &  & \foreignlanguage{greek}{ευκοπωτερον γαρ εϲτιν καμηλο̅} & 4 &  &  \\
&  & 5 & \foreignlanguage{greek}{δια τρυμαλιαϲ ραφιδοϲ ειϲελθει̅} & 8 &  &  \\
&  & 9 & \foreignlanguage{greek}{η πλουϲιον ειϲ την βαϲιλειαν} & 13 &  &  \\
&  & 14 & \foreignlanguage{greek}{του \textoverline{θυ} ειϲελθειν} & 16 &  &  \\
& \textbf{26} &  & \foreignlanguage{greek}{ειπον δε οι ακουοντεϲ και τιϲ δυνατε ϲωθηνε} & 8 &  &  \\
& \textbf{27} &  & \foreignlanguage{greek}{ο δε ειπεν τα αδυνατα παρα \textoverline{ανοιϲ} δυνα} & 8 &  &  \\
&  & 8 & \foreignlanguage{greek}{τα παρα \textoverline{θω} εϲτιν ειπεν δε πετροϲ} & 3 & \textbf{28} &  \\
&  & 4 & \foreignlanguage{greek}{ιδου ημειϲ αφηκαμεν παντα και ηκολου} & 9 &  &  \\
&  & 9 & \foreignlanguage{greek}{θηϲαμεν ϲοι ο δε ειπεν αυτοιϲ} & 4 & \textbf{29} &  \\
&  & 5 & \foreignlanguage{greek}{αμην υμιν λεγω οτι ουδειϲ εϲτιν οϲ αφη} & 12 &  &  \\
&  & 12 & \foreignlanguage{greek}{κεν οικειαν η γονειϲ η αδελφουϲ η γυναικα} & 19 &  &  \\
&  & 20 & \foreignlanguage{greek}{η τεκνα ενεκεν τηϲ βαϲιλειαϲ του \textoverline{θυ}} & 26 &  &  \\
& \textbf{30} &  & \foreignlanguage{greek}{οϲ ου μη απολαβη πολλαπλαϲιονα εν τω} & 7 &  &  \\
&  & 8 & \foreignlanguage{greek}{καιρω τουτω και εν τω αιωνι τω ερχομενω} & 15 &  &  \\
&  & 16 & \foreignlanguage{greek}{ζωην αιωνιον} & 17 &  &  \\
& \textbf{31} &  & \foreignlanguage{greek}{παραλαβων δε τουϲ δωδεκα ειπεν προϲ} & 6 &  &  \\
&  & 7 & \foreignlanguage{greek}{αυτουϲ ιδου αναβαινομεν ειϲ ιεροϲολυμα} & 11 &  &  \\
[0.2em]
\cline{4-4}
\end{tabular}
\end{center}
\end{table}
}
\clearpage
\newpage
 {
 \setlength\arrayrulewidth{1pt}
\begin{table}
\begin{center}
\begin{tabular}{ccc|l|ccc}
\cline{4-4} \\ [-1em]
\multicolumn{7}{c}{\foreignlanguage{greek}{ευαγγελιον κατα λουκαν} \textbf{(\nospace{18:31})} } \\ \\ [-1em] % Si on veut ajouter les bordures latérales, remplacer {7}{c} par {7}{|c|}
\cline{4-4} \\
\cline{4-4}
&  &  & &  &  & \\ [-0.9em]
&  & 12 & \foreignlanguage{greek}{και τελεϲθηϲεται παντα τα γεγραμμενα δια} & 17 &  &  \\
&  & 18 & \foreignlanguage{greek}{των προφητων τω υιω του \textoverline{ανου}} & 23 &  &  \\
& \textbf{32} &  & \foreignlanguage{greek}{παραδοθηϲεται γαρ τοιϲ εθνεϲιν και ενπε} & 6 &  &  \\
&  & 6 & \foreignlanguage{greek}{χθηϲεται και υβριϲθηϲεται και ενπτυϲθηϲετε} & 10 &  &  \\
& \textbf{33} &  & \foreignlanguage{greek}{και μαϲτιγωϲαντεϲ αποκτινουϲιν αυτον και} & 5 &  &  \\
&  & 6 & \foreignlanguage{greek}{τη ημερα τη τριτη αναϲτηϲεται και αυτοι ου} & 3 & \textbf{34} &  \\
&  & 3 & \foreignlanguage{greek}{δεν τουτων ϲυνηκαν και ην το ρημα τουτο} & 10 &  &  \\
&  & 11 & \foreignlanguage{greek}{κεκρυμμενον απ αυτων και ουκ εγινω} & 16 &  &  \\
&  & 16 & \foreignlanguage{greek}{ϲκον τα λεγομενα} & 18 &  &  \\
& \textbf{35} &  & \foreignlanguage{greek}{εγενετο δε εν τω εγγιζειν αυτον ειϲ ιεριχω} & 8 &  &  \\
&  & 9 & \foreignlanguage{greek}{τυφλοϲ τιϲ εκαθητο παρα την οδον προϲετω̅} & 15 &  &  \\
& \textbf{36} &  & \foreignlanguage{greek}{ακουϲαϲ δε οχλου διαπορευομενου επυν} & 5 &  &  \\
&  & 5 & \foreignlanguage{greek}{θανετο τι ειη τουτο απηγγειλαν δε αυ} & 3 & \textbf{37} &  \\
&  & 3 & \foreignlanguage{greek}{τω οτι \textoverline{ιϲ} ο ναζωραιοϲ παρερχεται και εβο} & 2 & \textbf{38} &  \\
&  & 2 & \foreignlanguage{greek}{ηϲεν λεγων \textoverline{ιυ} υιε δαυειδ ελεηϲον με} & 8 &  &  \\
& \textbf{39} &  & \foreignlanguage{greek}{και οι προαγοντεϲ επετιμων αυτω ινα} & 6 &  &  \\
&  & 7 & \foreignlanguage{greek}{ϲειγηϲη αυτοϲ δε πολλω μαλλον εκρα} & 12 &  &  \\
&  & 12 & \foreignlanguage{greek}{ζεν υιε δαυειδ ελεηϲον με} & 16 &  &  \\
& \textbf{40} &  & \foreignlanguage{greek}{ϲταθειϲ δε ο \textoverline{ιϲ} εκελευϲεν αυτον αχθηναι} & 7 &  &  \\
&  & 8 & \foreignlanguage{greek}{προϲ αυτον ενγιϲαντοϲ δε αυτου επη} & 13 &  &  \\
&  & 13 & \foreignlanguage{greek}{ρωτηϲεν αυτον λεγων τι ϲοι θελειϲ} & 4 & \textbf{41} &  \\
&  & 5 & \foreignlanguage{greek}{ποιηϲω ο δε ειπεν \textoverline{κε} ινα αναβλεψω} & 11 &  &  \\
& \textbf{42} &  & \foreignlanguage{greek}{και ο \textoverline{ιϲ} ειπεν αναβλεψον η πιϲτιϲ ϲου ϲε} & 9 &  &  \\
&  & 9 & \foreignlanguage{greek}{ϲωκεν ϲε και παραχρημα ανεβλεψεν} & 3 & \textbf{43} &  \\
&  & 4 & \foreignlanguage{greek}{και ηκολουθει αυτω δοξαζων τον \textoverline{θν}} & 10 &  &  \\
&  & 11 & \foreignlanguage{greek}{και παϲ ο λαοϲ ιδων εδωκεν αινον τω \textoverline{θω}} & 19 &  &  \\
& \mygospelchapter &  & \foreignlanguage{greek}{και εξελθων διηρχετο την ιεριχω} & 6 &  &  \\
& \textbf{2} &  & \foreignlanguage{greek}{και ιδου ανηρ ονοματι καλουμενοϲ ζαχ} & 6 &  &  \\
&  & 6 & \foreignlanguage{greek}{χαιοϲ και αυτοϲ ην αρχιτελωνηϲ ουτοϲ} & 11 &  &  \\
&  & 12 & \foreignlanguage{greek}{ην πλουϲιοϲ και εζητει ιδειν τον \textoverline{ιν} τιϲ} & 6 & \textbf{3} &  \\
[0.2em]
\cline{4-4}
\end{tabular}
\end{center}
\end{table}
}
\clearpage
\newpage
 {
 \setlength\arrayrulewidth{1pt}
\begin{table}
\begin{center}
\begin{tabular}{ccc|l|ccc}
\cline{4-4} \\ [-1em]
\multicolumn{7}{c}{\foreignlanguage{greek}{ευαγγελιον κατα λουκαν} \textbf{(\nospace{19:3})} } \\ \\ [-1em] % Si on veut ajouter les bordures latérales, remplacer {7}{c} par {7}{|c|}
\cline{4-4} \\
\cline{4-4}
&  &  & &  &  & \\ [-0.9em]
&  & 7 & \foreignlanguage{greek}{εϲτιν και ουκ ηδυνατο απο του οχλου οτι} & 14 &  &  \\
&  & 15 & \foreignlanguage{greek}{τη ηλικεια μεικροϲ ην και προϲδραμων} & 2 & \textbf{4} &  \\
&  & 3 & \foreignlanguage{greek}{εμπροϲθεν ανεβη επι ϲυκομωραιαν ινα} & 7 &  &  \\
&  & 8 & \foreignlanguage{greek}{ειδη αυτον οτι εκεινηϲ ημελλεν διερχε} & 13 &  &  \\
&  & 13 & \foreignlanguage{greek}{ϲθαι και ωϲ ηλθεν επι τον τοπον ανα} & 7 & \textbf{5} &  \\
&  & 7 & \foreignlanguage{greek}{βλεψαϲ ο \textoverline{ιϲ} ειδεν αυτον και ειπεν προϲ} & 14 &  &  \\
&  & 15 & \foreignlanguage{greek}{αυτον ζαχχαιε ϲπευϲαϲ καταβηθει} & 18 &  &  \\
&  & 19 & \foreignlanguage{greek}{ϲημερον γαρ εν τω οικω ϲου δει με μει} & 27 &  &  \\
&  & 27 & \foreignlanguage{greek}{ναι και ϲπευϲαϲ κατεβη και υπεδεξα} & 5 & \textbf{6} &  \\
&  & 5 & \foreignlanguage{greek}{το αυτον χαιρων και ιδοντεϲ παν} & 3 & \textbf{7} &  \\
&  & 3 & \foreignlanguage{greek}{τεϲ διεγογγυζον λεγοντεϲ οτι παρα} & 7 &  &  \\
&  & 8 & \foreignlanguage{greek}{αμαρτωλω ανδρι ειϲηλθεν καταλυϲαι} & 11 &  &  \\
& \textbf{8} &  & \foreignlanguage{greek}{ϲταθειϲ δε ζαχχαιοϲ ειπεν προϲ τον \textoverline{κν}} & 7 &  &  \\
&  & 8 & \foreignlanguage{greek}{ιδου το ημιϲυ των υπαρχοντων μου} & 13 &  &  \\
&  & 14 & \foreignlanguage{greek}{\textoverline{κε} διδωμι τοιϲ πτωχοιϲ και ει τινοϲ} & 20 &  &  \\
&  & 21 & \foreignlanguage{greek}{τι εϲυκοφαντηϲα αποδιδωμι τετρα} & 24 &  &  \\
&  & 24 & \foreignlanguage{greek}{πλουν ειπεν δε προϲ αυτον ο \textoverline{ιϲ}} & 6 & \textbf{9} &  \\
&  & 7 & \foreignlanguage{greek}{οτι ϲημερον ϲωτηρια τω οικω τουτω} & 12 &  &  \\
&  & 13 & \foreignlanguage{greek}{εγενετο καθοτι και αυτοϲ υιοϲ αβρα} & 18 &  &  \\
&  & 18 & \foreignlanguage{greek}{αμ εϲτιν ηλθεν γαρ ο υιοϲ του \textoverline{ανου}} & 6 & \textbf{10} &  \\
&  & 7 & \foreignlanguage{greek}{ζητηϲαι και ϲωϲαι το απολωλοϲ} & 11 &  &  \\
& \textbf{11} &  & \foreignlanguage{greek}{ακουοντων δε αυτων ταυτα προϲθειϲ} & 5 &  &  \\
&  & 6 & \foreignlanguage{greek}{ειπεν παραβολην δια το εγγυϲ αυτον} & 11 &  &  \\
&  & 12 & \foreignlanguage{greek}{ειναι ιερουϲαλημ και δοκειν αυτουϲ} & 16 &  &  \\
&  & 17 & \foreignlanguage{greek}{οτι παραχρημα η βαϲιλεια του \textoverline{θυ} μελ} & 23 &  &  \\
&  & 23 & \foreignlanguage{greek}{λει αναφαινεϲθαι} & 24 &  &  \\
& \textbf{12} &  & \foreignlanguage{greek}{ειπεν ουν \textoverline{ανοϲ} τιϲ ην ευγενηϲ και επο} & 8 &  &  \\
&  & 8 & \foreignlanguage{greek}{ρευθη ειϲ χωραν μακραν λαβειν εαυτω} & 13 &  &  \\
&  & 14 & \foreignlanguage{greek}{βαϲιλειαν και υποϲτρεψαι καλεϲαϲ} & 1 & \textbf{13} &  \\
&  & 2 & \foreignlanguage{greek}{δε δεκα δουλουϲ εαυτου εδωκεν αυτοιϲ} & 7 &  &  \\
[0.2em]
\cline{4-4}
\end{tabular}
\end{center}
\end{table}
}
\clearpage
\newpage
 {
 \setlength\arrayrulewidth{1pt}
\begin{table}
\begin{center}
\begin{tabular}{ccc|l|ccc}
\cline{4-4} \\ [-1em]
\multicolumn{7}{c}{\foreignlanguage{greek}{ευαγγελιον κατα λουκαν} \textbf{(\nospace{19:13})} } \\ \\ [-1em] % Si on veut ajouter les bordures latérales, remplacer {7}{c} par {7}{|c|}
\cline{4-4} \\
\cline{4-4}
&  &  & &  &  & \\ [-0.9em]
&  & 8 & \foreignlanguage{greek}{δεκα μναϲ και ειπεν προϲ αυτουϲ πρα} & 14 &  &  \\
&  & 14 & \foreignlanguage{greek}{γματευϲαϲθαι εν ω ερχομαι οι δε πολει} & 3 & \textbf{14} &  \\
&  & 3 & \foreignlanguage{greek}{ται αυτου εμιϲουν αυτον και απεϲτιλα̅} & 8 &  &  \\
&  & 9 & \foreignlanguage{greek}{πρεϲβειαν οπιϲω αυτου λεγοντεϲ ου θε} & 14 &  &  \\
&  & 14 & \foreignlanguage{greek}{λομεν τουτον βαϲιλευϲαι εφ ημαϲ} & 18 &  &  \\
& \textbf{15} &  & \foreignlanguage{greek}{και εγενετο εν τω επανελθειν αυτον} & 6 &  &  \\
&  & 7 & \foreignlanguage{greek}{λαβοντα την βαϲιλειαν και ειπεν φω} & 12 &  &  \\
&  & 12 & \foreignlanguage{greek}{νηθηναι τουϲ δουλουϲ τουτουϲ οιϲ εδω} & 17 &  &  \\
&  & 17 & \foreignlanguage{greek}{κεν το αργυριον ινα γνω τιϲ πεπραγμα} & 23 &  &  \\
&  & 23 & \foreignlanguage{greek}{τευϲατο παρεγενετο δε ο πρωτοϲ λεγω̅} & 5 & \textbf{16} &  \\
&  & 6 & \foreignlanguage{greek}{\textoverline{κε} η μνα ϲου προϲειργαϲατο δεκα μναϲ} & 12 &  &  \\
& \textbf{17} &  & \foreignlanguage{greek}{και ειπεν αυτω ευ αγαθε δουλε οτι εν ε} & 9 &  &  \\
&  & 9 & \foreignlanguage{greek}{λαχιϲτω πιϲτοϲ εγενου ιϲθει εξουϲιαν} & 13 &  &  \\
&  & 14 & \foreignlanguage{greek}{εχων επανω δεκα πολεων} & 17 &  &  \\
& \textbf{18} &  & \foreignlanguage{greek}{και ηλθεν ο δευτεροϲ λεγων \textoverline{κε} η μνα ϲου} & 9 &  &  \\
&  & 10 & \foreignlanguage{greek}{εποιηϲεν πεντε μναϲ ειπεν δε και} & 3 & \textbf{19} &  \\
&  & 4 & \foreignlanguage{greek}{τουτω και ϲυ γενου επανω πεντε πολεω̅} & 10 &  &  \\
& \textbf{20} &  & \foreignlanguage{greek}{και ετεροϲ ηλθεν λεγων \textoverline{κε} ιδου η μνα} & 8 &  &  \\
&  & 9 & \foreignlanguage{greek}{ϲου ην ειχον αποκειμενην εν ϲουδαριω} & 14 &  &  \\
& \textbf{21} &  & \foreignlanguage{greek}{εφοβουμην γαρ ϲε οτι \textoverline{ανοϲ} ει αυϲτηροϲ} & 7 &  &  \\
&  & 8 & \foreignlanguage{greek}{ερειϲ ο ουκ εθηκαϲ και θεριζειϲ ο ουκ ε} & 16 &  &  \\
&  & 16 & \foreignlanguage{greek}{ϲπειραϲ λεγει δε αυτω εκ του ϲτομα} & 6 & \textbf{22} &  \\
&  & 6 & \foreignlanguage{greek}{τοϲ ϲου κρινω ϲε πονηρε δουλε ηδειϲ} & 12 &  &  \\
&  & 13 & \foreignlanguage{greek}{οτι εγω \textoverline{ανοϲ} αυϲτηροϲ ειμει ερων ο ου} & 20 &  &  \\
&  & 20 & \foreignlanguage{greek}{κ εθηκα και θεριζων ο ουκ εϲπειρα} & 26 &  &  \\
& \textbf{23} &  & \foreignlanguage{greek}{και δια τι ουκ εδωκαϲ το αργυριον} & 8 &  &  \\
&  & 9 & \foreignlanguage{greek}{επι τραπεζαν και εγω ελθων ϲυν τω το} & 16 &  &  \\
&  & 16 & \foreignlanguage{greek}{κω αν επραξα αυτο και τοιϲ παρεϲτω} & 3 & \textbf{24} &  \\
&  & 3 & \foreignlanguage{greek}{ϲιν ειπεν αρατε απ αυτου την μναν και} & 10 &  &  \\
&  & 11 & \foreignlanguage{greek}{δοτε τω ταϲ δεκα μναϲ εχοντι} & 16 &  &  \\
[0.2em]
\cline{4-4}
\end{tabular}
\end{center}
\end{table}
}
\clearpage
\newpage
 {
 \setlength\arrayrulewidth{1pt}
\begin{table}
\begin{center}
\begin{tabular}{ccc|l|ccc}
\cline{4-4} \\ [-1em]
\multicolumn{7}{c}{\foreignlanguage{greek}{ευαγγελιον κατα λουκαν} \textbf{(\nospace{19:26})} } \\ \\ [-1em] % Si on veut ajouter les bordures latérales, remplacer {7}{c} par {7}{|c|}
\cline{4-4} \\
\cline{4-4}
&  &  & &  &  & \\ [-0.9em]
& \textbf{26} &  & \foreignlanguage{greek}{λεγω γαρ υμιν οτι παντι τω εχοντι δοθη} & 8 &  &  \\
&  & 8 & \foreignlanguage{greek}{ϲεται απο δε του μη εχοντοϲ και ο εχει αρ} & 17 &  &  \\
&  & 17 & \foreignlanguage{greek}{θηϲεται απ αυτου πλην τουϲ εχθρουϲ μου} & 4 & \textbf{27} &  \\
&  & 5 & \foreignlanguage{greek}{εκεινουϲ τουϲ μη θεληϲανταϲ με βαϲιλευ} & 10 &  &  \\
&  & 10 & \foreignlanguage{greek}{ϲαι επ αυτουϲ αγαγετε ωδε και καταϲφα} & 16 &  &  \\
&  & 16 & \foreignlanguage{greek}{ξατε εμπροϲθεν μου και ειπων ταυτα επο} & 4 & \textbf{28} &  \\
&  & 4 & \foreignlanguage{greek}{ρευετο εμπροϲθεν αναβαινων ειϲ ιεροϲολυμα} & 8 &  &  \\
& \textbf{29} &  & \foreignlanguage{greek}{και εγενετο ωϲ ηγγιϲεν ειϲ βηθφαγη και} & 7 &  &  \\
&  & 8 & \foreignlanguage{greek}{βηθανιαν προϲ το οροϲ το καλουμενον ε} & 14 &  &  \\
&  & 14 & \foreignlanguage{greek}{λεωνα απεϲτιλεν δυο των μαθητων αυτου} & 19 &  &  \\
& \textbf{30} &  & \foreignlanguage{greek}{ειπων υπαγεται ειϲ την κατεναντι κωμη} & 6 &  &  \\
&  & 7 & \foreignlanguage{greek}{εν η ειϲπορευομενοι ευρηϲεται πωλον δεδε} & 12 &  &  \\
&  & 12 & \foreignlanguage{greek}{μενον εφ ον ουδειϲ πωποτε \textoverline{ανων} εκα} & 18 &  &  \\
&  & 18 & \foreignlanguage{greek}{θειϲεν λυϲαντεϲ αυτον αγαγεται και εαν} & 2 & \textbf{31} &  \\
&  & 3 & \foreignlanguage{greek}{τιϲ υμαϲ ερωτα δια τι λυεται ουτωϲ ερειτε} & 10 &  &  \\
&  & 11 & \foreignlanguage{greek}{αυτω οτι ο \textoverline{κϲ} αυτου χρειαν εχει} & 17 &  &  \\
& \textbf{32} &  & \foreignlanguage{greek}{απελθοντεϲ δε οι απεϲταλμενοι ευραν} & 5 &  &  \\
&  & 6 & \foreignlanguage{greek}{καθωϲ ειπεν αυτοιϲ λυοντων δε αυτω̅} & 3 & \textbf{33} &  \\
&  & 4 & \foreignlanguage{greek}{τον πωλον ειπον οι κυριοι αυτου προϲ αυτουϲ} & 11 &  &  \\
&  & 12 & \foreignlanguage{greek}{τι λυεται τον πωλον οι δε ειπον οτι ο \textoverline{κϲ}} & 6 & \textbf{34} &  \\
&  & 7 & \foreignlanguage{greek}{αυτου χρειαν εχει και ηγαγον αυτον προϲ} & 4 & \textbf{35} &  \\
&  & 5 & \foreignlanguage{greek}{τον \textoverline{ιν} και επιριψαντεϲ εαυτων τα ιμα} & 11 &  &  \\
&  & 11 & \foreignlanguage{greek}{τια επι τον πωλον επεβιβαϲαν τον \textoverline{ιν}} & 17 &  &  \\
& \textbf{36} &  & \foreignlanguage{greek}{πορευομενου δε αυτου υπεϲτρωννυον} & 4 &  &  \\
&  & 5 & \foreignlanguage{greek}{τα ιματια εαυτων εν τη οδω} & 10 &  &  \\
& \textbf{37} &  & \foreignlanguage{greek}{εγγιζοντοϲ δε αυτου ηδη προϲ τη καταβα} & 7 &  &  \\
&  & 7 & \foreignlanguage{greek}{ϲει του ορουϲ των ελεων ηρξατο απαν} & 13 &  &  \\
&  & 13 & \foreignlanguage{greek}{ταν το πληθοϲ των μαθητων χαιροντεϲ} & 18 &  &  \\
&  & 19 & \foreignlanguage{greek}{αινειν τον \textoverline{θν} φωνη μεγαλη περι παϲω̅} & 25 &  &  \\
&  & 26 & \foreignlanguage{greek}{ων ειδον δυναμεων λεγοντεϲ ευλογημε} & 2 & \textbf{38} &  \\
[0.2em]
\cline{4-4}
\end{tabular}
\end{center}
\end{table}
}
\clearpage
\newpage
 {
 \setlength\arrayrulewidth{1pt}
\begin{table}
\begin{center}
\begin{tabular}{ccc|l|ccc}
\cline{4-4} \\ [-1em]
\multicolumn{7}{c}{\foreignlanguage{greek}{ευαγγελιον κατα λουκαν} \textbf{(\nospace{19:38})} } \\ \\ [-1em] % Si on veut ajouter les bordures latérales, remplacer {7}{c} par {7}{|c|}
\cline{4-4} \\
\cline{4-4}
&  &  & &  &  & \\ [-0.9em]
&  & 2 & \foreignlanguage{greek}{νοϲ ο ερχομενοϲ εν ονοματι \textoverline{κυ} ειρηνη εν ου} & 10 &  &  \\
&  & 10 & \foreignlanguage{greek}{ρανω και δοξα εν υψιϲτοιϲ και τινεϲ φα} & 3 & \textbf{39} &  \\
&  & 3 & \foreignlanguage{greek}{ριϲαιοι απο του οχλου ειπον προϲ αυτον δι} & 10 &  &  \\
&  & 10 & \foreignlanguage{greek}{δαϲκαλε επιτιμηϲον τοιϲ μαθηταιϲ ϲου} & 14 &  &  \\
& \textbf{40} &  & \foreignlanguage{greek}{και αποκριθειϲ ειπεν αυτοιϲ λεγω υμιν εα̅} & 7 &  &  \\
&  & 8 & \foreignlanguage{greek}{ουτοι ϲιωπηϲουϲιν οι λιθοι κεκραξονται} & 12 &  &  \\
& \textbf{41} &  & \foreignlanguage{greek}{και ωϲ ηγγειϲεν ιδων την πολιν εκλαυϲε̅} & 7 &  &  \\
&  & 8 & \foreignlanguage{greek}{επ αυτην λεγων οτι ει εγνωϲ και ϲυ και} & 7 & \textbf{42} &  \\
&  & 8 & \foreignlanguage{greek}{γε εν τη ημερα ϲου ταυτη τα προϲ ειρηνην ϲου} & 17 &  &  \\
&  & 18 & \foreignlanguage{greek}{νυν δε εκρυβη απ οφθαλμων ϲου οτι ηξου} & 2 & \textbf{43} &  \\
&  & 2 & \foreignlanguage{greek}{ϲιν ημεραι επι ϲε και περιβαλουϲιν οι εχθροι} & 9 &  &  \\
&  & 10 & \foreignlanguage{greek}{ϲου χαρακα ϲοι και περικυκλωϲουϲιν ϲε πα̅} & 16 &  &  \\
&  & 16 & \foreignlanguage{greek}{τοθεν και εδαφιουϲιν ϲε και τα τεκνα ϲου εν ϲοι} & 9 & \textbf{44} &  \\
&  & 10 & \foreignlanguage{greek}{και ουκ αφηϲουϲιν εν ϲοι λιθον επι λιθω} & 17 &  &  \\
&  & 18 & \foreignlanguage{greek}{ανθ ων ουκ εγνωϲ τον καιρον τηϲ επιϲκο} & 25 &  &  \\
&  & 25 & \foreignlanguage{greek}{πηϲ ϲου και ειϲελθων ειϲ το ιερον ηρξατο} & 6 & \textbf{45} &  \\
&  & 7 & \foreignlanguage{greek}{εκβαλλειν τουϲ πωλουνταϲ εν αυτω και} & 12 &  &  \\
&  & 13 & \foreignlanguage{greek}{αγοραζονταϲ λεγων αυτοιϲ γεγραπται} & 3 & \textbf{46} &  \\
&  & 4 & \foreignlanguage{greek}{οτι ο οικοϲ μου οικοϲ προϲευχηϲ εϲτιν} & 10 &  &  \\
&  & 11 & \foreignlanguage{greek}{υμειϲ δε αυτον εποιηϲατε ϲπηλαιον ληϲτω̅} & 16 &  &  \\
& \textbf{47} &  & \foreignlanguage{greek}{και ην διδαϲκων το καθ ημεραν εν τω} & 8 &  &  \\
&  & 9 & \foreignlanguage{greek}{ιερω οι δε αρχιερειϲ και οι γραμματειϲ} & 15 &  &  \\
&  & 16 & \foreignlanguage{greek}{εζητουν αυτον απολεϲαι και οι πρωτοι} & 21 &  &  \\
&  & 22 & \foreignlanguage{greek}{του λαου και ουχ ηυριϲκον το τι ποιηϲουϲι̅} & 6 & \textbf{48} &  \\
&  & 7 & \foreignlanguage{greek}{ο λαοϲ γαρ απαϲ εξεκρεματο αυτου ακουων} & 13 &  &  \\
& \mygospelchapter &  & \foreignlanguage{greek}{και εγενετο εν μια των ημερων εκεινω̅} & 7 &  &  \\
&  & 8 & \foreignlanguage{greek}{διδαϲκοντοϲ αυτου τον λαον εν τω ιερω} & 14 &  &  \\
&  & 15 & \foreignlanguage{greek}{και ευαγγελιζομενου επεϲτηϲαν} & 17 &  &  \\
&  & 19 & \foreignlanguage{greek}{οι ιερειϲ και οι γραμματιϲ ϲυν τοιϲ πρεϲ} & 26 &  &  \\
&  & 26 & \foreignlanguage{greek}{βυτεροιϲ ϗ ειπον προϲ αυτον λεγοντεϲ} & 5 & \textbf{2} &  \\
[0.2em]
\cline{4-4}
\end{tabular}
\end{center}
\end{table}
}
\clearpage
\newpage
 {
 \setlength\arrayrulewidth{1pt}
\begin{table}
\begin{center}
\begin{tabular}{ccc|l|ccc}
\cline{4-4} \\ [-1em]
\multicolumn{7}{c}{\foreignlanguage{greek}{ευαγγελιον κατα λουκαν} \textbf{(\nospace{20:2})} } \\ \\ [-1em] % Si on veut ajouter les bordures latérales, remplacer {7}{c} par {7}{|c|}
\cline{4-4} \\
\cline{4-4}
&  &  & &  &  & \\ [-0.9em]
&  & 6 & \foreignlanguage{greek}{ειπε ημιν εν ποια εξουϲια ταυτα ποιειϲ η τιϲ} & 14 &  &  \\
&  & 15 & \foreignlanguage{greek}{εϲτιν ο δουϲ ϲοι την εξουϲιαν ταυτην} & 21 &  &  \\
& \textbf{3} &  & \foreignlanguage{greek}{αποκριθειϲ δε ειπεν προϲ αυτουϲ ερωτηϲω} & 6 &  &  \\
&  & 7 & \foreignlanguage{greek}{υμαϲ καγω λογον και ειπατε μοι} & 12 &  &  \\
& \textbf{4} &  & \foreignlanguage{greek}{το βαπτιϲμα ιωαννου εξ ουρανου ην η εξ} & 8 &  &  \\
&  & 9 & \foreignlanguage{greek}{ανθρωπων οι δε ϲυνελογιζοντο προϲ ε} & 5 & \textbf{5} &  \\
&  & 5 & \foreignlanguage{greek}{αυτουϲ λεγοντεϲ οτι εαν ειπωμεν εξ} & 10 &  &  \\
&  & 11 & \foreignlanguage{greek}{ουρανου ερει δια τι ουκ επιϲτευϲατε αυτω} & 17 &  &  \\
& \textbf{6} &  & \foreignlanguage{greek}{εαν δε ειπωμεν εξ ανθρωπου παϲ ο λαοϲ} & 8 &  &  \\
&  & 9 & \foreignlanguage{greek}{καταλιθαϲει ημαϲ πεπιϲμενοϲ γαρ εϲτι̅} & 13 &  &  \\
&  & 14 & \foreignlanguage{greek}{ιωαννην προφητην ειναι και απεκρι} & 2 & \textbf{7} &  \\
&  & 2 & \foreignlanguage{greek}{θηϲαν μη ειδεναι ποθεν} & 5 &  &  \\
& \textbf{8} &  & \foreignlanguage{greek}{και ο \textoverline{ιϲ} ειπεν αυτοιϲ ουδε εγω λεγω υμιν} & 9 &  &  \\
&  & 10 & \foreignlanguage{greek}{εν ποια εξουϲια ταυτα ποιω ηρξατο δε προϲ} & 3 & \textbf{9} &  \\
&  & 4 & \foreignlanguage{greek}{τον λαον λεγειν την παραβολην ταυτην} & 9 &  &  \\
&  & 10 & \foreignlanguage{greek}{\textoverline{ανοϲ} τιϲ εφυτευϲεν αμπελωνα και εξε} & 15 &  &  \\
&  & 15 & \foreignlanguage{greek}{δοτο αυτον γεωργοιϲ και απεδημηϲεν} & 19 &  &  \\
&  & 20 & \foreignlanguage{greek}{χρονουϲ ικανουϲ και εν καιρω απεϲτιλε̅} & 4 & \textbf{10} &  \\
&  & 5 & \foreignlanguage{greek}{προϲ τουϲ γεωργουϲ δουλον ινα απο του} & 11 &  &  \\
&  & 12 & \foreignlanguage{greek}{καρπου του αμπελωνοϲ δωϲιν αυτω} & 16 &  &  \\
&  & 17 & \foreignlanguage{greek}{οι δε γεωργοι διραντεϲ αυτον εξαπε} & 22 &  &  \\
&  & 22 & \foreignlanguage{greek}{ϲτιλαν αυτον κενον και προϲεθε} & 2 & \textbf{11} &  \\
&  & 2 & \foreignlanguage{greek}{το πεμψαι ετερον δουλον οι δε κακεινο̅} & 8 &  &  \\
&  & 9 & \foreignlanguage{greek}{δειραντεϲ και ατιμαϲαντεϲ εξαπεϲτι} & 12 &  &  \\
&  & 12 & \foreignlanguage{greek}{λαν κενον και προϲεθετο πεμψαι τρι} & 4 & \textbf{12} &  \\
&  & 4 & \foreignlanguage{greek}{τον οι δε και τουτον τραυματιϲαντεϲ} & 9 &  &  \\
&  & 10 & \foreignlanguage{greek}{εξεβαλον ειπεν δε ο \textoverline{κϲ} του αμπελω} & 6 & \textbf{13} &  \\
&  & 6 & \foreignlanguage{greek}{νοϲ τι ποιηϲω πεμψω τον υιον μου τον} & 13 &  &  \\
&  & 14 & \foreignlanguage{greek}{αγαπητον ιϲωϲ τουτον ιδοντεϲ εντρα} & 18 &  &  \\
&  & 18 & \foreignlanguage{greek}{πηϲονται ιδοντεϲ δε αυτον οι γεωργοι} & 5 & \textbf{14} &  \\
[0.2em]
\cline{4-4}
\end{tabular}
\end{center}
\end{table}
}
\clearpage
\newpage
 {
 \setlength\arrayrulewidth{1pt}
\begin{table}
\begin{center}
\begin{tabular}{ccc|l|ccc}
\cline{4-4} \\ [-1em]
\multicolumn{7}{c}{\foreignlanguage{greek}{ευαγγελιον κατα λουκαν} \textbf{(\nospace{20:14})} } \\ \\ [-1em] % Si on veut ajouter les bordures latérales, remplacer {7}{c} par {7}{|c|}
\cline{4-4} \\
\cline{4-4}
&  &  & &  &  & \\ [-0.9em]
&  & 6 & \foreignlanguage{greek}{διελογιζοντεϲ προϲ εαυτουϲ λεγοντεϲ} & 9 &  &  \\
&  & 10 & \foreignlanguage{greek}{ουτοϲ εϲτιν ο κληρονομοϲ αποκτινωμεν} & 14 &  &  \\
&  & 15 & \foreignlanguage{greek}{αυτον ινα ημων γενητε η κληρονομια} & 20 &  &  \\
& \textbf{15} &  & \foreignlanguage{greek}{και εκβαλοντεϲ αυτον εξω του αμπελωνοϲ} & 6 &  &  \\
&  & 7 & \foreignlanguage{greek}{απεκτειναν τι ουν ποιηϲει αυτοιϲ ο \textoverline{κϲ}} & 13 &  &  \\
&  & 14 & \foreignlanguage{greek}{του αμπελωνοϲ ελευϲεται και απολε} & 3 & \textbf{16} &  \\
&  & 3 & \foreignlanguage{greek}{ϲει τουϲ γεωργουϲ τουτουϲ και δωϲει} & 8 &  &  \\
&  & 10 & \foreignlanguage{greek}{τον αμπελωνα αλλοιϲ ακουϲαντεϲ δε} & 14 &  &  \\
&  & 15 & \foreignlanguage{greek}{ειπον μη γενοιτο ο δε εμβλεψαϲ αυ} & 4 & \textbf{17} &  \\
&  & 4 & \foreignlanguage{greek}{τοιϲ ειπεν τι ουν εϲτιν το γεγραμμενο̅} & 10 &  &  \\
&  & 11 & \foreignlanguage{greek}{τουτο λιθον ον απεδοκειμαϲαν οι οικο} & 16 &  &  \\
&  & 16 & \foreignlanguage{greek}{δομουντεϲ ουτοϲ εγενηθη ειϲ κεφαλη̅} & 20 &  &  \\
&  & 21 & \foreignlanguage{greek}{γωνιαϲ παϲ ο πεϲων επ εκεινον τον λιθο̅} & 7 & \textbf{18} &  \\
&  & 8 & \foreignlanguage{greek}{ϲυνθλαϲθηϲεται εφ ον δ αν πεϲειτε λι} & 14 &  &  \\
&  & 14 & \foreignlanguage{greek}{κμηϲει αυτον και εζητηϲαν οι γραμμα} & 4 & \textbf{19} &  \\
&  & 4 & \foreignlanguage{greek}{τιϲ και οι αρχιερειϲ επιβαλειν επ αυτον ταϲ} & 11 &  &  \\
&  & 12 & \foreignlanguage{greek}{χειραϲ εν αυτη τη ωρα και εφοβηθηϲαν} & 18 &  &  \\
&  & 19 & \foreignlanguage{greek}{τον οχλον εγνωϲαν γαρ οτι προϲ αυτουϲ} & 25 &  &  \\
&  & 26 & \foreignlanguage{greek}{την παραβολην ταυτην ειπεν} & 29 &  &  \\
& \textbf{20} &  & \foreignlanguage{greek}{και υποχωρηϲαντεϲ απεϲτιλαν ενκαθε} & 4 &  &  \\
&  & 4 & \foreignlanguage{greek}{τουϲ υποκρινομενουϲ εαυτουϲ δικαι} & 7 &  &  \\
&  & 7 & \foreignlanguage{greek}{ουϲ ειναι ινα επιλαβωνται αυτου λογου} & 12 &  &  \\
&  & 13 & \foreignlanguage{greek}{ειϲ το παραδουναι αυτον τη αρχη και τη} & 20 &  &  \\
&  & 21 & \foreignlanguage{greek}{εξουϲια του ηγεμονοϲ} & 23 &  &  \\
& \textbf{21} &  & \foreignlanguage{greek}{και επηρωτηϲαν αυτον λεγοντεϲ οιδαμε̅} & 5 &  &  \\
&  & 6 & \foreignlanguage{greek}{διδαϲκαλε οιδαμεν οτι ορθωϲ λεγειϲ και} & 11 &  &  \\
&  & 12 & \foreignlanguage{greek}{διδαϲκειϲ και ου λαμβανειϲ προϲωπον} & 16 &  &  \\
&  & 17 & \foreignlanguage{greek}{αλλ επ αληθειαϲ την οδον του \textoverline{θυ} διδαϲκειϲ} & 24 &  &  \\
& \textbf{22} &  & \foreignlanguage{greek}{εξεϲτιν ημιν καιϲαρι φορον δουναι η ου} & 7 &  &  \\
& \textbf{23} &  & \foreignlanguage{greek}{κατανοηϲαϲ δε αυτων την πανουργιαν} & 5 &  &  \\
[0.2em]
\cline{4-4}
\end{tabular}
\end{center}
\end{table}
}
\clearpage
\newpage
 {
 \setlength\arrayrulewidth{1pt}
\begin{table}
\begin{center}
\begin{tabular}{ccc|l|ccc}
\cline{4-4} \\ [-1em]
\multicolumn{7}{c}{\foreignlanguage{greek}{ευαγγελιον κατα λουκαν} \textbf{(\nospace{20:23})} } \\ \\ [-1em] % Si on veut ajouter les bordures latérales, remplacer {7}{c} par {7}{|c|}
\cline{4-4} \\
\cline{4-4}
&  &  & &  &  & \\ [-0.9em]
&  & 6 & \foreignlanguage{greek}{ειπεν προϲ αυτουϲ τι με πειραζεται δειξα} & 1 & \textbf{24} &  \\
&  & 1 & \foreignlanguage{greek}{τε μοι δηναριον τινοϲ εχει εικονα και ε} & 8 &  &  \\
&  & 8 & \foreignlanguage{greek}{πιγραφην αποκριθεντεϲ ειπον καιϲαροϲ} & 11 &  &  \\
& \textbf{25} &  & \foreignlanguage{greek}{ο δε ειπεν αυτοιϲ αποδοτε τοινυν τα και} & 8 &  &  \\
&  & 8 & \foreignlanguage{greek}{ϲαροϲ καιϲαρι και τα του \textoverline{θυ} τω \textoverline{θω} και ουκ ιϲχυ} & 3 & \textbf{26} &  \\
&  & 3 & \foreignlanguage{greek}{ϲαν επιλαβεϲθαι αυτου ρηματοϲ εναντι} & 7 &  &  \\
&  & 7 & \foreignlanguage{greek}{ον του λαου και θαυμαϲαντεϲ επι τη απο} & 14 &  &  \\
&  & 14 & \foreignlanguage{greek}{κριϲει αυτου εϲιγηϲαν} & 16 &  &  \\
& \textbf{27} &  & \foreignlanguage{greek}{προϲελθοντεϲ δε τινεϲ των ϲαδδουκεω̅} & 5 &  &  \\
&  & 6 & \foreignlanguage{greek}{οι αντιλεγοντεϲ αναϲταϲιν μη ειναι επη} & 11 &  &  \\
&  & 11 & \foreignlanguage{greek}{ρωτηϲαν αυτον λεγοντεϲ διδαϲκαλε} & 2 & \textbf{28} &  \\
&  & 3 & \foreignlanguage{greek}{μωυϲηϲ εγραψεν ημιν εαν τινοϲ αδελ} & 8 &  &  \\
&  & 8 & \foreignlanguage{greek}{φοϲ αποθανη εχων γυναικα και ουτοϲ} & 13 &  &  \\
&  & 14 & \foreignlanguage{greek}{ατεκνοϲ αποθανη ινα λαβη ο αδελφοϲ} & 19 &  &  \\
&  & 20 & \foreignlanguage{greek}{αυτου την γυναικα και εξαναϲτηϲει} & 24 &  &  \\
&  & 25 & \foreignlanguage{greek}{ϲπερμα τω αδελφω αυτου επτα ουν α} & 3 & \textbf{29} &  \\
&  & 3 & \foreignlanguage{greek}{δελφοι ηϲαν και ο πρωτοϲ λαβων γυναικα} & 9 &  &  \\
&  & 10 & \foreignlanguage{greek}{απεθανεν ατεκνοϲ και ελαβεν ο δευ} & 4 & \textbf{30} &  \\
&  & 4 & \foreignlanguage{greek}{τεροϲ την γυναικα και ουτοϲ απεθανεν} & 9 &  &  \\
&  & 10 & \foreignlanguage{greek}{ατεκνοϲ και ο τριτοϲ ελαβεν αυτην} & 5 & \textbf{31} &  \\
&  & 6 & \foreignlanguage{greek}{ωϲαυτωϲ δε και οι επτα και ου κατελιπο̅} & 13 &  &  \\
&  & 14 & \foreignlanguage{greek}{τεκνα και απεθανον υϲτερα δε παντω̅} & 3 & \textbf{32} &  \\
&  & 4 & \foreignlanguage{greek}{απεθανεν και η γυνη εν τη ουν αναϲταϲι} & 7 & \textbf{33} &  \\
&  & 8 & \foreignlanguage{greek}{τινοϲ αυτων γεινεται γυνη οι γαρ επτα εϲχο̅} & 15 &  &  \\
&  & 16 & \foreignlanguage{greek}{αυτην γυναικα} & 17 &  &  \\
& \textbf{34} &  & \foreignlanguage{greek}{και αποκριθειϲ ειπεν αυτοιϲ ο \textoverline{ιϲ} οι υιοι του} & 9 &  &  \\
&  & 10 & \foreignlanguage{greek}{αιωνοϲ τουτου γαμουϲιν και εκγαμιζονται} & 14 &  &  \\
& \textbf{35} &  & \foreignlanguage{greek}{οι δε καταξιωθεντεϲ του αιωνοϲ εκεινου τυχει̅} & 7 &  &  \\
&  & 8 & \foreignlanguage{greek}{και τηϲ αναϲταϲεωϲ τηϲ εκ νεκρων ουτε γα} & 16 &  &  \\
&  & 16 & \foreignlanguage{greek}{μουϲιν ουτε εκγαμιζονται ουδε γαρ αποθα} & 3 & \textbf{36} &  \\
[0.2em]
\cline{4-4}
\end{tabular}
\end{center}
\end{table}
}
\clearpage
\newpage
 {
 \setlength\arrayrulewidth{1pt}
\begin{table}
\begin{center}
\begin{tabular}{ccc|l|ccc}
\cline{4-4} \\ [-1em]
\multicolumn{7}{c}{\foreignlanguage{greek}{ευαγγελιον κατα λουκαν} \textbf{(\nospace{20:36})} } \\ \\ [-1em] % Si on veut ajouter les bordures latérales, remplacer {7}{c} par {7}{|c|}
\cline{4-4} \\
\cline{4-4}
&  &  & &  &  & \\ [-0.9em]
&  & 3 & \foreignlanguage{greek}{νειν μελλουϲιν ιϲαγγελοι γαρ ειϲιν και υιοι} & 9 &  &  \\
&  & 10 & \foreignlanguage{greek}{ειϲιν του \textoverline{θυ} τηϲ αναϲταϲεωϲ υιοι οντεϲ} & 16 &  &  \\
& \textbf{37} &  & \foreignlanguage{greek}{οτι δε εγειρονται οι νεκροι και μωυϲηϲ εδη} & 8 &  &  \\
&  & 8 & \foreignlanguage{greek}{λωϲεν επι τηϲ βατου ωϲ λεγει \textoverline{κν} τον \textoverline{θν}} & 16 &  &  \\
&  & 17 & \foreignlanguage{greek}{αβρααμ και τον \textoverline{θν} ιϲαακ ο \textoverline{θϲ} δε ουκ εϲτι̅} & 5 & \textbf{38} &  \\
&  & 6 & \foreignlanguage{greek}{νεκρων αλλα ζωντων παντεϲ γαρ αυτου} & 11 &  &  \\
&  & 12 & \foreignlanguage{greek}{ουτοι αποκριθεντεϲ δε τινεϲ τω̅} & 4 & \textbf{39} &  \\
&  & 5 & \foreignlanguage{greek}{γραμματεων ειπον διδαϲκαλε καλωϲ} & 8 &  &  \\
&  & 9 & \foreignlanguage{greek}{ειπαϲ ουκετι δε ετολμων επερωταν αυ} & 5 & \textbf{40} &  \\
&  & 5 & \foreignlanguage{greek}{τον ουδεν ειπεν δε προϲ αυτουϲ} & 4 & \textbf{41} &  \\
&  & 5 & \foreignlanguage{greek}{πωϲ λεγουϲιν τον \textoverline{χν} υιον δαυειδ ειναι} & 11 &  &  \\
& \textbf{42} &  & \foreignlanguage{greek}{και αυτοϲ δαυειδ λεγει εν βιβλω των} & 7 &  &  \\
&  & 8 & \foreignlanguage{greek}{ψαλμων ειπεν ο \textoverline{κϲ} τω \textoverline{κω} μου κα} & 15 &  &  \\
&  & 15 & \foreignlanguage{greek}{θου εκ δεξιων μου εωϲ αν θω τουϲ εχθρουϲ} & 5 & \textbf{43} &  \\
&  & 6 & \foreignlanguage{greek}{ϲου υποποδιον των ποδων ϲου} & 10 &  &  \\
& \textbf{44} &  & \foreignlanguage{greek}{δαυειδ ουν \textoverline{κν} αυτον καλει και πωϲ υ} & 8 &  &  \\
&  & 8 & \foreignlanguage{greek}{ιοϲ αυτου εϲτιν} & 10 &  &  \\
& \textbf{45} &  & \foreignlanguage{greek}{ακουοντοϲ δε παντοϲ του λαου ειπεν} & 6 &  &  \\
&  & 7 & \foreignlanguage{greek}{τοιϲ μαθηταιϲ αυτου προϲεχεται απο} & 2 & \textbf{46} &  \\
&  & 3 & \foreignlanguage{greek}{των γραμματεων των θελοντων περι} & 7 &  &  \\
&  & 7 & \foreignlanguage{greek}{πατειν εν ϲτολαιϲ και φιλουντων αϲπα} & 12 &  &  \\
&  & 12 & \foreignlanguage{greek}{ϲμουϲ εν ταιϲ αγοραιϲ και πρωτοκαθε} & 17 &  &  \\
&  & 17 & \foreignlanguage{greek}{δριαϲ εν ταιϲ ϲυναγωγαιϲ και πρωτο} & 22 &  &  \\
&  & 22 & \foreignlanguage{greek}{κλιϲιαϲ εν τοιϲ διπνοιϲ οι κατεϲθιουϲι̅} & 2 & \textbf{47} &  \\
&  & 3 & \foreignlanguage{greek}{ταϲ οικειαϲ των χηρων και προφαϲι} & 8 &  &  \\
&  & 9 & \foreignlanguage{greek}{μακρα προϲευχονται ουτοι λημψον} & 12 &  &  \\
&  & 12 & \foreignlanguage{greek}{ται περιϲϲοτερον κριμα} & 14 &  &  \\
& \mygospelchapter &  & \foreignlanguage{greek}{αναβλεψαϲ δε ειδεν τουϲ βαλλονταϲ} & 5 &  &  \\
&  & 6 & \foreignlanguage{greek}{τα δωρα αυτων ειϲ το γαζοφυλακιον} & 11 &  &  \\
&  & 12 & \foreignlanguage{greek}{πλουϲιουϲ ειδεν τινα και χηραν πενι} & 5 & \textbf{2} &  \\
[0.2em]
\cline{4-4}
\end{tabular}
\end{center}
\end{table}
}
\clearpage
\newpage
 {
 \setlength\arrayrulewidth{1pt}
\begin{table}
\begin{center}
\begin{tabular}{ccc|l|ccc}
\cline{4-4} \\ [-1em]
\multicolumn{7}{c}{\foreignlanguage{greek}{ευαγγελιον κατα λουκαν} \textbf{(\nospace{21:2})} } \\ \\ [-1em] % Si on veut ajouter les bordures latérales, remplacer {7}{c} par {7}{|c|}
\cline{4-4} \\
\cline{4-4}
&  &  & &  &  & \\ [-0.9em]
&  & 5 & \foreignlanguage{greek}{χραν βαλλουϲαν εκει δυο λεπτα και ειπεν} & 2 & \textbf{3} &  \\
&  & 3 & \foreignlanguage{greek}{αληθωϲ λεγω υμιν οτι η χηρα η πτωχη αυ} & 11 &  &  \\
&  & 11 & \foreignlanguage{greek}{τη πλιω παντων εβαλεν απαντεϲ γαρ} & 2 & \textbf{4} &  \\
&  & 3 & \foreignlanguage{greek}{ουτοι εκ του περιϲϲευοντοϲ αυτοιϲ εβα} & 8 &  &  \\
&  & 8 & \foreignlanguage{greek}{λον ειϲ τα δωρα του \textoverline{θυ} αυτη δε εκ του υϲτε} & 18 &  &  \\
&  & 18 & \foreignlanguage{greek}{ρηματοϲ αυτηϲ απαντα τον βιον ον ειχεν} & 24 &  &  \\
&  & 25 & \foreignlanguage{greek}{εβαλεν και τινων λεγοντων περι του} & 5 & \textbf{5} &  \\
&  & 6 & \foreignlanguage{greek}{ιερου οτι λιθοιϲ καλοιϲ και αναθεμαϲιν} & 11 &  &  \\
&  & 12 & \foreignlanguage{greek}{κεκοϲμητε κεκοϲμητο ειπεν ταυτα α θεωρειται} & 3 & \textbf{6} &  \\
&  & 4 & \foreignlanguage{greek}{ελευϲονται ημεραι εν αιϲ ουκ αφεθηϲε} & 9 &  &  \\
&  & 9 & \foreignlanguage{greek}{ται λιθοϲ επι λιθον οϲ ου καταλυθηϲεται} & 15 &  &  \\
& \textbf{7} &  & \foreignlanguage{greek}{επηρωτηϲαν δε αυτον λεγοντεϲ διδα} & 5 &  &  \\
&  & 5 & \foreignlanguage{greek}{ϲκαλε ποτε ουν ταυτα εϲται και τι το} & 12 &  &  \\
&  & 13 & \foreignlanguage{greek}{ϲημιον οταν μελλει ταυτα γεινεϲθαι} & 17 &  &  \\
& \textbf{8} &  & \foreignlanguage{greek}{ο δε ειπεν βλεπεται μη πλανηθηται πολ} & 7 &  &  \\
&  & 7 & \foreignlanguage{greek}{λοι γαρ ελευϲονται επι τω ονοματι μου} & 13 &  &  \\
&  & 14 & \foreignlanguage{greek}{λεγοντεϲ οτι εγω ειμει και ο καιροϲ ηγ} & 21 &  &  \\
&  & 21 & \foreignlanguage{greek}{γικεν μη ουν πορευθηται οπιϲω αυτω̅} & 26 &  &  \\
& \textbf{9} &  & \foreignlanguage{greek}{οταν δε ακουϲηται πολεμουϲ και ακατα} & 6 &  &  \\
&  & 6 & \foreignlanguage{greek}{ϲταϲιαϲ μη πτοηθηται δει γαρ ταυτα} & 11 &  &  \\
&  & 12 & \foreignlanguage{greek}{γενεϲθαι πρωτον αλλ ουκ ευθεωϲ} & 16 &  &  \\
&  & 17 & \foreignlanguage{greek}{το τελοϲ τοτε ελεγεν αυτοιϲ} & 3 & \textbf{10} &  \\
&  & 4 & \foreignlanguage{greek}{εγερθηϲεται εθνοϲ επι εθνοϲ και βαϲι} & 9 &  &  \\
&  & 9 & \foreignlanguage{greek}{λεια επι βαϲιλειαν ϲιϲμοι τε μεγαλοι} & 3 & \textbf{11} &  \\
&  & 4 & \foreignlanguage{greek}{κατα τοπουϲ και λιμοι και λοιμοι εϲονται} & 10 &  &  \\
&  & 11 & \foreignlanguage{greek}{φοβηθρα τε και ϲημια απ ουρανου με} & 17 &  &  \\
&  & 17 & \foreignlanguage{greek}{γαλα εϲται προ δε τουτων παντων} & 4 & \textbf{12} &  \\
&  & 5 & \foreignlanguage{greek}{επιβαλουϲιν εφ υμαϲ ταϲ χειραϲ αυτω̅} & 10 &  &  \\
&  & 11 & \foreignlanguage{greek}{και διωξουϲιν παραδιδοντεϲ ειϲ ϲυνα} & 15 &  &  \\
&  & 15 & \foreignlanguage{greek}{γωγαϲ και φυλακαϲ αγομενουϲ επι βα} & 20 &  &  \\
[0.2em]
\cline{4-4}
\end{tabular}
\end{center}
\end{table}
}
\clearpage
\newpage
 {
 \setlength\arrayrulewidth{1pt}
\begin{table}
\begin{center}
\begin{tabular}{ccc|l|ccc}
\cline{4-4} \\ [-1em]
\multicolumn{7}{c}{\foreignlanguage{greek}{ευαγγελιον κατα λουκαν} \textbf{(\nospace{21:12})} } \\ \\ [-1em] % Si on veut ajouter les bordures latérales, remplacer {7}{c} par {7}{|c|}
\cline{4-4} \\
\cline{4-4}
&  &  & &  &  & \\ [-0.9em]
&  & 20 & \foreignlanguage{greek}{ϲιλειϲ και ηγεμοναϲ ενεκεν του ονοματοϲ μου} & 26 &  &  \\
& \textbf{13} &  & \foreignlanguage{greek}{αποβηϲεται δε υμιν ειϲ μαρτυριον} & 5 &  &  \\
& \textbf{14} &  & \foreignlanguage{greek}{θετε ουν ειϲ ταϲ καρδιαϲ υμων μη προμελε} & 8 &  &  \\
&  & 8 & \foreignlanguage{greek}{ταν απολογηθηναι εγω γαρ δωϲω υμιν} & 4 & \textbf{15} &  \\
&  & 5 & \foreignlanguage{greek}{ϲτομα και ϲοφιαν η ου δυνηϲονται αντι} & 11 &  &  \\
&  & 11 & \foreignlanguage{greek}{πειν ουδε αντιϲτηναι παντεϲ οι αντι} & 16 &  &  \\
&  & 16 & \foreignlanguage{greek}{κειμενοι υμιν παραδοθηϲεϲθαι δε} & 2 & \textbf{16} &  \\
&  & 3 & \foreignlanguage{greek}{και υπο γονεων και αδελφων και ϲυγγε} & 9 &  &  \\
&  & 9 & \foreignlanguage{greek}{νεων και φιλων και θανατωϲουϲιν εξ} & 14 &  &  \\
&  & 15 & \foreignlanguage{greek}{υμων και εϲεϲθαι μιϲουμενοι υπο πα̅} & 5 & \textbf{17} &  \\
&  & 5 & \foreignlanguage{greek}{των δια το ονομα μου και θριξ εκ τηϲ κε} & 5 & \textbf{18} &  \\
&  & 5 & \foreignlanguage{greek}{φαληϲ υμων ου μη αποληται εν τη υπομο} & 3 & \textbf{19} &  \\
&  & 3 & \foreignlanguage{greek}{νη υμων κτηϲαϲθαι ταϲ ψυχαϲ υμων} & 8 &  &  \\
& \textbf{20} &  & \foreignlanguage{greek}{οταν δε ειδηται κυκλουμενην υπο ϲτρα} & 6 &  &  \\
&  & 6 & \foreignlanguage{greek}{τοπεδων ιερουϲαλημ τοτε γινωϲκεται} & 9 &  &  \\
&  & 10 & \foreignlanguage{greek}{οτι ηγγικεν η ερημωϲιϲ αυτηϲ} & 14 &  &  \\
& \textbf{21} &  & \foreignlanguage{greek}{τοτε οι εν τη ιουδαια φευγετωϲαν ειϲ τα} & 8 &  &  \\
&  & 9 & \foreignlanguage{greek}{ορη και οι εν μεϲω αυτηϲ εκχωριτωϲαν} & 15 &  &  \\
&  & 16 & \foreignlanguage{greek}{και εν ταιϲ χωραιϲ μη ειϲερχεϲθωϲαν ειϲ} & 22 &  &  \\
&  & 23 & \foreignlanguage{greek}{αυτην οτι ημεραι εκδικηϲεωϲ αυται ειϲι̅} & 5 & \textbf{22} &  \\
&  & 6 & \foreignlanguage{greek}{του πληϲθηναι παντα τα γεγραμμενα} & 10 &  &  \\
& \textbf{23} &  & \foreignlanguage{greek}{ουαι δε ταιϲ εν γαϲτρι εχουϲαιϲ και θηλα} & 8 &  &  \\
&  & 8 & \foreignlanguage{greek}{ζουϲαιϲ εν εκειναιϲ ταιϲ ημεραιϲ εϲται} & 13 &  &  \\
&  & 14 & \foreignlanguage{greek}{γαρ αναγκη μεγαλη επι τηϲ γηϲ και οργη} & 21 &  &  \\
&  & 22 & \foreignlanguage{greek}{εν τω λαω τουτω και πεϲουνται ϲτο} & 3 & \textbf{24} &  \\
&  & 3 & \foreignlanguage{greek}{ματι μαχαιραιϲ και εχμαλωτιϲθηϲον} & 6 &  &  \\
&  & 6 & \foreignlanguage{greek}{ται ειϲ παντα τα εθνη και ιερουϲαλημ} & 12 &  &  \\
&  & 13 & \foreignlanguage{greek}{εϲται πατουμενη υπο εθνων αχρι πλη} & 18 &  &  \\
&  & 18 & \foreignlanguage{greek}{ρωθωϲιν καιροι εθνων} & 20 &  &  \\
& \textbf{25} &  & \foreignlanguage{greek}{και εϲται ϲημια εν ηλιω και ϲεληνη και α} & 9 &  &  \\
[0.2em]
\cline{4-4}
\end{tabular}
\end{center}
\end{table}
}
\clearpage
\newpage
 {
 \setlength\arrayrulewidth{1pt}
\begin{table}
\begin{center}
\begin{tabular}{ccc|l|ccc}
\cline{4-4} \\ [-1em]
\multicolumn{7}{c}{\foreignlanguage{greek}{ευαγγελιον κατα λουκαν} \textbf{(\nospace{21:25})} } \\ \\ [-1em] % Si on veut ajouter les bordures latérales, remplacer {7}{c} par {7}{|c|}
\cline{4-4} \\
\cline{4-4}
&  &  & &  &  & \\ [-0.9em]
&  & 9 & \foreignlanguage{greek}{ϲτροιϲ και επι τηϲ γηϲ ϲυνοχη εθνων εν α} & 17 &  &  \\
&  & 17 & \foreignlanguage{greek}{πορεια η ωϲ ηχουϲηϲ θαλαϲϲηϲ και ϲαλουϲ} & 23 &  &  \\
& \textbf{26} &  & \foreignlanguage{greek}{αποψυχοντων \textoverline{ανων} απο φοβου και προϲ} & 6 &  &  \\
&  & 6 & \foreignlanguage{greek}{δοκειαϲ των επερχομενων τη οικουμε} & 10 &  &  \\
&  & 10 & \foreignlanguage{greek}{νηϲ αι γαρ δυναμειϲ των ουρανων} & 15 &  &  \\
&  & 16 & \foreignlanguage{greek}{ϲαλευθηϲονται και τοτε οψονται τον} & 4 & \textbf{27} &  \\
&  & 5 & \foreignlanguage{greek}{υιον του \textoverline{ανου} ερχομενον εν νεφελη με} & 11 &  &  \\
&  & 11 & \foreignlanguage{greek}{τα δυναμεωϲ και δοξηϲ πολληϲ} & 15 &  &  \\
& \textbf{28} &  & \foreignlanguage{greek}{αρχομενων δε τουτων γεινεϲθαι ανα} & 5 &  &  \\
&  & 6 & \foreignlanguage{greek}{καλυψατε και επαραται ταϲ κεφαλαϲ υμω̅} & 11 &  &  \\
&  & 12 & \foreignlanguage{greek}{διοτι εγγιζει η απολυτρωϲειϲ υμων και} & 1 & \textbf{29} &  \\
&  & 2 & \foreignlanguage{greek}{ειπεν παραβολην αυτοιϲ} & 4 &  &  \\
&  & 5 & \foreignlanguage{greek}{ιδετε την ϲυκην και παντα τα δενδρα} & 11 &  &  \\
& \textbf{30} &  & \foreignlanguage{greek}{οταν προβαλωϲιν ηδη βλεποντεϲ απ αυ} & 6 &  &  \\
&  & 6 & \foreignlanguage{greek}{των γινωϲκεται οτι ηδη εγγυϲ το θε} & 12 &  &  \\
&  & 12 & \foreignlanguage{greek}{ροϲ εϲτιν ουτωϲ και υμειϲ οταν ει} & 5 & \textbf{31} &  \\
&  & 5 & \foreignlanguage{greek}{δηται ταυτα γεινομενα γινωϲκεται ο} & 9 &  &  \\
&  & 9 & \foreignlanguage{greek}{τι εγγυϲ εϲτιν η βαϲιλεια του \textoverline{θυ}} & 15 &  &  \\
& \textbf{32} &  & \foreignlanguage{greek}{αμην λεγω υμιν οτι ου μη παρελθη η γε} & 9 &  &  \\
&  & 9 & \foreignlanguage{greek}{νεα αυτη εωϲ αν παντα γενηται} & 14 &  &  \\
& \textbf{33} &  & \foreignlanguage{greek}{ο ουρανοϲ και η γη παρελευϲηται οι δε} & 9 &  &  \\
&  & 10 & \foreignlanguage{greek}{λογοι μου ου μη παρελευϲονται προϲεχε} & 1 & \textbf{34} &  \\
&  & 1 & \foreignlanguage{greek}{ται δε εαυτοιϲ μηποτε βαρηθωϲιν αι καρ} & 7 &  &  \\
&  & 7 & \foreignlanguage{greek}{διαι υμων εν κραιπαλη και μεθη και με} & 14 &  &  \\
&  & 14 & \foreignlanguage{greek}{ριμναιϲ βιωτικαιϲ και εφνιδιοϲ εφ υ} & 19 &  &  \\
&  & 19 & \foreignlanguage{greek}{μαϲ επιϲτη η ημερα εκεινη ωϲ παγειϲ γαρ} & 3 & \textbf{35} &  \\
&  & 4 & \foreignlanguage{greek}{επελευϲεται επι πανταϲ τουϲ καθημε} & 8 &  &  \\
&  & 8 & \foreignlanguage{greek}{νουϲ επι προϲωπον τηϲ γηϲ παϲηϲ} & 13 &  &  \\
& \textbf{36} &  & \foreignlanguage{greek}{αγρυπνιται ουν εν παντι καιρω δεομε} & 6 &  &  \\
&  & 6 & \foreignlanguage{greek}{νοι ινα κατιϲχυϲατε εκφυγειν παντα ταυτα} & 11 &  &  \\
[0.2em]
\cline{4-4}
\end{tabular}
\end{center}
\end{table}
}
\clearpage
\newpage
 {
 \setlength\arrayrulewidth{1pt}
\begin{table}
\begin{center}
\begin{tabular}{ccc|l|ccc}
\cline{4-4} \\ [-1em]
\multicolumn{7}{c}{\foreignlanguage{greek}{ευαγγελιον κατα λουκαν} \textbf{(\nospace{21:36})} } \\ \\ [-1em] % Si on veut ajouter les bordures latérales, remplacer {7}{c} par {7}{|c|}
\cline{4-4} \\
\cline{4-4}
&  &  & &  &  & \\ [-0.9em]
&  & 12 & \foreignlanguage{greek}{μελλοντα γινεϲθαι και ϲταθηναι εμπρο} & 16 &  &  \\
&  & 16 & \foreignlanguage{greek}{ϲθεν του υιου του ανθρωπου} & 20 &  &  \\
& \textbf{37} &  & \foreignlanguage{greek}{ην δε ταϲ ημεραϲ εν τω ιερω διδαϲκων} & 8 &  &  \\
&  & 9 & \foreignlanguage{greek}{ταϲ δε νυκταϲ ηυλιζετο ειϲ το οροϲ το καλου} & 17 &  &  \\
&  & 17 & \foreignlanguage{greek}{μενον ελεων και παϲ ο λαοϲ ωρθριζεν} & 5 & \textbf{38} &  \\
&  & 6 & \foreignlanguage{greek}{προϲ αυτον εν τω ιερω ακουειν αυτου} & 12 &  &  \\
& \mygospelchapter &  & \foreignlanguage{greek}{ηγγιζεν δε η εορτη των αζυμων η λεγομε} & 8 &  &  \\
&  & 8 & \foreignlanguage{greek}{νη παϲχα και εζητουν οι αρχιερειϲ και} & 5 & \textbf{2} &  \\
&  & 6 & \foreignlanguage{greek}{οι γραμματιϲ το πωϲ ανελωϲιν αυτον εφο} & 12 &  &  \\
&  & 12 & \foreignlanguage{greek}{βουντο γαρ τον λαον} & 15 &  &  \\
& \textbf{3} &  & \foreignlanguage{greek}{ειϲηλθεν δε ϲαταναϲ ειϲ ιουδαν τον κα} & 7 &  &  \\
&  & 7 & \foreignlanguage{greek}{λουμενον ιϲκαριωτην οντα εκ του αρι} & 12 &  &  \\
&  & 12 & \foreignlanguage{greek}{θμου των δωδεκα και απελθων ϲυνε} & 3 & \textbf{4} &  \\
&  & 3 & \foreignlanguage{greek}{λαληϲεν τοιϲ αρχιερευϲιν και τοιϲ ϲτρα} & 8 &  &  \\
&  & 8 & \foreignlanguage{greek}{τηγοιϲ το πωϲ αυτον παραδω αυτοιϲ και} & 1 & \textbf{5} &  \\
&  & 2 & \foreignlanguage{greek}{εχαρηϲαν και ϲυνεθεντο αυτω αργυριο̅} & 6 &  &  \\
&  & 7 & \foreignlanguage{greek}{δουναι και εξωμολογηϲεν και εζητι} & 4 & \textbf{6} &  \\
&  & 5 & \foreignlanguage{greek}{ευκαιριαν του παραδουναι αυτον αυτοιϲ} & 9 &  &  \\
&  & 10 & \foreignlanguage{greek}{ατερ οχλου} & 11 &  &  \\
& \textbf{7} &  & \foreignlanguage{greek}{ηλθεν δε η ημερα των αζυμων εν η εδει} & 9 &  &  \\
&  & 10 & \foreignlanguage{greek}{θυεϲθαι το παϲχα και απεϲτιλεν πετρο̅} & 3 & \textbf{8} &  \\
&  & 4 & \foreignlanguage{greek}{και ιωαννην ειπων πορευθεντεϲ ετοι} & 8 &  &  \\
&  & 8 & \foreignlanguage{greek}{μαϲατε ημιν το παϲχα ινα φαγωμεν} & 13 &  &  \\
& \textbf{9} &  & \foreignlanguage{greek}{οι δε ειπον αυτω που θελειϲ ετοιμαϲωμεν} & 7 &  &  \\
& \textbf{10} &  & \foreignlanguage{greek}{ο δε ειπεν αυτοιϲ ιδου ειϲελθοντων υμω̅} & 7 &  &  \\
&  & 8 & \foreignlanguage{greek}{ειϲ την πολιν ϲυναντηϲει υμιν \textoverline{ανοϲ} κε} & 14 &  &  \\
&  & 14 & \foreignlanguage{greek}{ραμιον υδατοϲ βαϲταζων ακολουθηϲατε} & 17 &  &  \\
&  & 18 & \foreignlanguage{greek}{αυτω ειϲ την οικειαν ου ειϲπορευεται} & 23 &  &  \\
& \textbf{11} &  & \foreignlanguage{greek}{και ερειται τω οικοδεϲποτη τηϲ οικειαϲ λε} & 7 &  &  \\
&  & 7 & \foreignlanguage{greek}{γει ϲοι ο διδαϲκαλοϲ που εϲτιν το καταλυμα} & 14 &  &  \\
[0.2em]
\cline{4-4}
\end{tabular}
\end{center}
\end{table}
}
\clearpage
\newpage
 {
 \setlength\arrayrulewidth{1pt}
\begin{table}
\begin{center}
\begin{tabular}{ccc|l|ccc}
\cline{4-4} \\ [-1em]
\multicolumn{7}{c}{\foreignlanguage{greek}{ευαγγελιον κατα λουκαν} \textbf{(\nospace{22:11})} } \\ \\ [-1em] % Si on veut ajouter les bordures latérales, remplacer {7}{c} par {7}{|c|}
\cline{4-4} \\
\cline{4-4}
&  &  & &  &  & \\ [-0.9em]
&  & 15 & \foreignlanguage{greek}{οπου το παϲχα μετα των μαθητων μου φαγω} & 22 &  &  \\
& \textbf{12} &  & \foreignlanguage{greek}{κακεινοϲ υμιν δειξει αναγεον μεγα εϲτρω} & 7 &  &  \\
&  & 7 & \foreignlanguage{greek}{μενον εκει ετοιμαϲαται} & 9 &  &  \\
& \textbf{13} &  & \foreignlanguage{greek}{απελθοντεϲ δε ευρον καθωϲ ειρηκεν αυ} & 6 &  &  \\
&  & 6 & \foreignlanguage{greek}{τοιϲ και ητοιμαϲαν το παϲχα και οτε ε} & 3 & \textbf{14} &  \\
&  & 3 & \foreignlanguage{greek}{γενετο η ωρα ανεπεϲεν και οι δωδεκα} & 9 &  &  \\
&  & 10 & \foreignlanguage{greek}{αποϲτολοι ϲυν αυτω και ειπεν προϲ αυτουϲ} & 4 & \textbf{15} &  \\
&  & 5 & \foreignlanguage{greek}{επιθυμια επεθυμηϲα τουτο το παϲχα φα} & 10 &  &  \\
&  & 10 & \foreignlanguage{greek}{γειν μεθ υμων προ του παθειν λεγω} & 1 & \textbf{16} &  \\
&  & 2 & \foreignlanguage{greek}{γαρ υμιν οτι ουκεντι ου μη φαγω εξ αυ} & 10 &  &  \\
&  & 10 & \foreignlanguage{greek}{του εωϲ οτου πληρωθη εν τη βαϲιλεια} & 16 &  &  \\
&  & 17 & \foreignlanguage{greek}{του \textoverline{θυ} και δεξαμενοϲ το ποτηριον} & 4 & \textbf{17} &  \\
&  & 5 & \foreignlanguage{greek}{ευχαριϲτηϲαϲ ειπεν λαβεται τουτο} & 8 &  &  \\
&  & 9 & \foreignlanguage{greek}{και διαμεριϲαται εαυτοιϲ} & 11 &  &  \\
& \textbf{18} &  & \foreignlanguage{greek}{λεγω γαρ υμιν οτι ου μη πιω απο του νυ̅} & 10 &  &  \\
&  & 11 & \foreignlanguage{greek}{γεννηματοϲ τηϲ αμπελου εωϲ οτου η} & 17 &  &  \\
&  & 18 & \foreignlanguage{greek}{βαϲιλεια του \textoverline{θυ} ελθη και λαβων αρ} & 3 & \textbf{19} &  \\
&  & 3 & \foreignlanguage{greek}{τον ευχαριϲτηϲαϲ εκλαϲεν και εδω} & 7 &  &  \\
&  & 7 & \foreignlanguage{greek}{κεν αυτοιϲ λεγων τουτο εϲτιν το ϲω} & 13 &  &  \\
&  & 13 & \foreignlanguage{greek}{μα μου το υπερ υμων διδομενον τουτο} & 19 &  &  \\
&  & 20 & \foreignlanguage{greek}{ποιειται ειϲ την εμην αναμνηϲιν} & 24 &  &  \\
& \textbf{20} &  & \foreignlanguage{greek}{ωϲαυτωϲ και το ποτηριον μετα το δι} & 7 &  &  \\
&  & 7 & \foreignlanguage{greek}{πνηϲαι λεγων τουτο το ποτηριον η και} & 13 &  &  \\
&  & 13 & \foreignlanguage{greek}{νη διαθηκη εν τω ετι μου το υπερ υ} & 21 &  &  \\
&  & 21 & \foreignlanguage{greek}{μων εχχυννομενον} & 22 &  &  \\
& \textbf{21} &  & \foreignlanguage{greek}{πλην ιδου η χειρ του παραδιδοντοϲ με} & 7 &  &  \\
&  & 8 & \foreignlanguage{greek}{μετ εμου επι τηϲ τραπεζηϲ και ο μεν} & 3 & \textbf{22} &  \\
&  & 4 & \foreignlanguage{greek}{υιοϲ του \textoverline{ανου} πορευεται κατα το ωριϲμε} & 10 &  &  \\
&  & 10 & \foreignlanguage{greek}{νον πλην ουαι τω \textoverline{ανω} εκεινω δι ου πα} & 18 &  &  \\
&  & 18 & \foreignlanguage{greek}{ραδιδοται και ηρξατο ϲυνζητειν} & 3 & \textbf{23} &  \\
[0.2em]
\cline{4-4}
\end{tabular}
\end{center}
\end{table}
}
\clearpage
\newpage
 {
 \setlength\arrayrulewidth{1pt}
\begin{table}
\begin{center}
\begin{tabular}{ccc|l|ccc}
\cline{4-4} \\ [-1em]
\multicolumn{7}{c}{\foreignlanguage{greek}{ευαγγελιον κατα λουκαν} \textbf{(\nospace{22:23})} } \\ \\ [-1em] % Si on veut ajouter les bordures latérales, remplacer {7}{c} par {7}{|c|}
\cline{4-4} \\
\cline{4-4}
&  &  & &  &  & \\ [-0.9em]
&  & 4 & \foreignlanguage{greek}{προϲ αυτουϲ το τιϲ αρα ειη εξ αυτων ο του} & 13 &  &  \\
&  & 13 & \foreignlanguage{greek}{το μελλων πραϲϲιν εγενετο δε και φι} & 4 & \textbf{24} &  \\
&  & 4 & \foreignlanguage{greek}{λονικεια εν αυτοιϲ το τιϲ αυτων δοκει ει} & 11 &  &  \\
&  & 11 & \foreignlanguage{greek}{ναι μειζων ο δε ειπεν αυτοιϲ} & 4 & \textbf{25} &  \\
&  & 5 & \foreignlanguage{greek}{οι βαϲιλειϲ των εθνων κυριευουϲιν αυτω̅} & 10 &  &  \\
&  & 11 & \foreignlanguage{greek}{και εξουϲιαζουϲιν αυτων ευεργεται κα} & 15 &  &  \\
&  & 15 & \foreignlanguage{greek}{λουνται υμειϲ δε ουχ ουτωϲ αλλ ο μιζων} & 7 & \textbf{26} &  \\
&  & 8 & \foreignlanguage{greek}{εν υμιν γενεϲθω ωϲ ο νεωτεροϲ και ο ηγου} & 16 &  &  \\
&  & 16 & \foreignlanguage{greek}{μενοϲ ωϲ ο διακονων τιϲ γαρ μιζων} & 3 & \textbf{27} &  \\
&  & 4 & \foreignlanguage{greek}{ο ανακειμενοϲ η ο διακονων ουχει ο ανα} & 11 &  &  \\
&  & 11 & \foreignlanguage{greek}{κειμενοϲ εγω ειμει εν μεϲω υμων ωϲ} & 17 &  &  \\
&  & 18 & \foreignlanguage{greek}{ο διακονων υμειϲ δε εϲται οι διαμεμε} & 5 & \textbf{28} &  \\
&  & 5 & \foreignlanguage{greek}{νηκοτεϲ μετ εμου εν τοιϲ πειραϲμοιϲ μου} & 11 &  &  \\
& \textbf{29} &  & \foreignlanguage{greek}{καγω διατιθεμαι υμιν καθωϲ διεθετο} & 5 &  &  \\
&  & 6 & \foreignlanguage{greek}{μοι ο \textoverline{πηρ} μου βαϲιλειαν ινα εϲθειηται και} & 3 & \textbf{30} &  \\
&  & 4 & \foreignlanguage{greek}{πεινηται επι τηϲ τραπεζηϲ μου εν τη βα} & 11 &  &  \\
&  & 11 & \foreignlanguage{greek}{ϲιλεια μου και καθηϲηϲθαι επι θρονω̅} & 16 &  &  \\
&  & 17 & \foreignlanguage{greek}{κρινοντεϲ ταϲ δωδεκα φυλαϲ του ιϲραηλ} & 22 &  &  \\
& \textbf{31} &  & \foreignlanguage{greek}{ειπεν δε ο \textoverline{κϲ} ϲιμων ϲιμων ιδου ο ϲαταναϲ} & 9 &  &  \\
&  & 10 & \foreignlanguage{greek}{εξητηϲατο υμαϲ του ϲινιαϲαι ωϲ τον ϲιτο̅} & 16 &  &  \\
& \textbf{32} &  & \foreignlanguage{greek}{εγω δε εδεηθην περι ϲου ινα μη εκλει} & 8 &  &  \\
&  & 8 & \foreignlanguage{greek}{πη η πιϲτιϲ ϲου και ϲυ ποτε επιϲτρεψαϲ} & 15 &  &  \\
&  & 16 & \foreignlanguage{greek}{ϲτηριξον τουϲ αδελφουϲ ϲου} & 19 &  &  \\
& \textbf{33} &  & \foreignlanguage{greek}{ο δε ειπεν αυτω \textoverline{κε} μετα ϲου ειμει και ειϲ} & 10 &  &  \\
&  & 11 & \foreignlanguage{greek}{φυλακην και ειϲ θανατον πορευεϲθαι} & 15 &  &  \\
& \textbf{34} &  & \foreignlanguage{greek}{ο δε ειπεν λεγω ϲοι πετρε ου μη φωνηϲη} & 9 &  &  \\
&  & 10 & \foreignlanguage{greek}{ϲημερον αλεκτωρ πριν η τριϲ απαρνηϲη} & 15 &  &  \\
&  & 16 & \foreignlanguage{greek}{μη ειδεναι με} & 18 &  &  \\
& \textbf{35} &  & \foreignlanguage{greek}{και ειπεν αυτοιϲ οτε απεϲτιλα υμαϲ ατερ} & 7 &  &  \\
&  & 8 & \foreignlanguage{greek}{βαλλαντιου και πηραϲ και υποδηματων} & 12 &  &  \\
[0.2em]
\cline{4-4}
\end{tabular}
\end{center}
\end{table}
}
\clearpage
\newpage
 {
 \setlength\arrayrulewidth{1pt}
\begin{table}
\begin{center}
\begin{tabular}{ccc|l|ccc}
\cline{4-4} \\ [-1em]
\multicolumn{7}{c}{\foreignlanguage{greek}{ευαγγελιον κατα λουκαν} \textbf{(\nospace{22:35})} } \\ \\ [-1em] % Si on veut ajouter les bordures latérales, remplacer {7}{c} par {7}{|c|}
\cline{4-4} \\
\cline{4-4}
&  &  & &  &  & \\ [-0.9em]
&  & 13 & \foreignlanguage{greek}{μη τινοϲ υϲτερηϲατε οι δε ειπον ουθενοϲ} & 19 &  &  \\
& \textbf{36} &  & \foreignlanguage{greek}{ειπεν ουν αυτοιϲ αλλα νυν ο εχων βαλλαντι} & 8 &  &  \\
&  & 8 & \foreignlanguage{greek}{ον αρατω ομοιωϲ και πηραν και μη εχων} & 15 &  &  \\
&  & 16 & \foreignlanguage{greek}{πωληϲατω ιματιον αυτου και αγοραϲατω} & 20 &  &  \\
&  & 21 & \foreignlanguage{greek}{μαχαιραν λεγω γαρ υμιν οτι τουτο το γε} & 7 & \textbf{37} &  \\
&  & 7 & \foreignlanguage{greek}{γραμμενον δει τελεϲθηναι εν εμοι} & 12 &  &  \\
&  & 13 & \foreignlanguage{greek}{το και μετα ανομων ελογιϲθην και γαρ} & 19 &  &  \\
&  & 20 & \foreignlanguage{greek}{το περι εμου τελοϲ εχει οι δε ειπον και} & 4 & \textbf{38} &  \\
&  & 5 & \foreignlanguage{greek}{ιδου μαχαιρε ωδε δυο} & 8 &  &  \\
&  & 9 & \foreignlanguage{greek}{ο δε ειπεν αυτοιϲ εικανον εϲτιν και εξελ} & 2 & \textbf{39} &  \\
&  & 2 & \foreignlanguage{greek}{θων επορευθη κατα το εθοϲ ειϲ το οροϲ τω̅} & 11 &  &  \\
&  & 12 & \foreignlanguage{greek}{ελεων ηκολουθηϲαν δε αυτω και οι μα} & 18 &  &  \\
&  & 18 & \foreignlanguage{greek}{θηται γενομενοϲ δε επι του τοπου ειπε̅} & 6 & \textbf{40} &  \\
&  & 7 & \foreignlanguage{greek}{αυτοιϲ προϲευχεϲθαι μη ειϲελθειν ειϲ πει} & 12 &  &  \\
&  & 12 & \foreignlanguage{greek}{ραϲμον και αυτοϲ απεϲπαϲθη απ αυτων} & 5 & \textbf{41} &  \\
&  & 6 & \foreignlanguage{greek}{ωϲει λιθου βολην και θειϲ τα γονατα προϲ} & 13 &  &  \\
&  & 13 & \foreignlanguage{greek}{ηυχετο λεγων πατερ ει βουλει παρε} & 5 & \textbf{42} &  \\
&  & 5 & \foreignlanguage{greek}{νεγκε το ποτηριον τουτο απ εμου} & 10 &  &  \\
&  & 11 & \foreignlanguage{greek}{πλην μη το θελημα μου αλλα το ϲον γινε} & 19 &  &  \\
&  & 19 & \foreignlanguage{greek}{ϲθω ελ} & 6 & \textbf{45} &  \\
&  & 6 & \foreignlanguage{greek}{θων προϲ τουϲ μαθηταϲ ευρεν αυτουϲ} & 11 &  &  \\
&  & 12 & \foreignlanguage{greek}{κοιμωμενουϲ απο τηϲ λυπηϲ} & 15 &  &  \\
& \textbf{46} &  & \foreignlanguage{greek}{και ειπεν αυτοιϲ τι καθευδεται αναϲτα̅} & 6 &  &  \\
&  & 6 & \foreignlanguage{greek}{τεϲ προϲευχεϲθαι ινα μη ειϲελθηται ειϲ} & 11 &  &  \\
&  & 12 & \foreignlanguage{greek}{πειραϲμον ετι αυτου λαλουντοϲ ι} & 4 & \textbf{47} &  \\
&  & 4 & \foreignlanguage{greek}{δου οχλοϲ και ο λεγομενοϲ ιουδαϲ ειϲ των} & 11 &  &  \\
&  & 12 & \foreignlanguage{greek}{δωδεκα προηρχετο αυτου και ηγγιϲεν} & 16 &  &  \\
&  & 17 & \foreignlanguage{greek}{τω \textoverline{ιυ} φιληϲαι αυτον} & 20 &  &  \\
& \textbf{48} &  & \foreignlanguage{greek}{ο δε \textoverline{ιϲ} ειπεν αυτω ιουδα φιληματι τον} & 8 &  &  \\
&  & 9 & \foreignlanguage{greek}{υιον του ανθρωπου παραδιδωϲ} & 12 &  &  \\
[0.2em]
\cline{4-4}
\end{tabular}
\end{center}
\end{table}
}
\clearpage
\newpage
 {
 \setlength\arrayrulewidth{1pt}
\begin{table}
\begin{center}
\begin{tabular}{ccc|l|ccc}
\cline{4-4} \\ [-1em]
\multicolumn{7}{c}{\foreignlanguage{greek}{ευαγγελιον κατα λουκαν} \textbf{(\nospace{22:49})} } \\ \\ [-1em] % Si on veut ajouter les bordures latérales, remplacer {7}{c} par {7}{|c|}
\cline{4-4} \\
\cline{4-4}
&  &  & &  &  & \\ [-0.9em]
& \textbf{49} &  & \foreignlanguage{greek}{ιδοντεϲ δε οι περι αυτον το εϲομενον ειπο̅} & 8 &  &  \\
&  & 9 & \foreignlanguage{greek}{αυτω \textoverline{κε} επιταξομεν εν μαχαιρα και επα} & 2 & \textbf{50} &  \\
&  & 2 & \foreignlanguage{greek}{ταξεν ειϲ τιϲ εξ αυτων τον δουλον του αρ} & 10 &  &  \\
&  & 10 & \foreignlanguage{greek}{χιερεωϲ και αφειλεν αυτου το ουϲ το δεξιο̅} & 17 &  &  \\
& \textbf{51} &  & \foreignlanguage{greek}{αποκριθειϲ δε ο \textoverline{ιϲ} ειπεν εαϲατε εωϲ τουτου} & 8 &  &  \\
&  & 9 & \foreignlanguage{greek}{και αψαμενοϲ του ωτιου ιαϲατο αυτον} & 14 &  &  \\
& \textbf{52} &  & \foreignlanguage{greek}{ειπεν δε ο \textoverline{ιϲ} προϲ τουϲ παραγενομενουϲ} & 7 &  &  \\
&  & 8 & \foreignlanguage{greek}{επ αυτον αρχιερειϲ και ϲτρατηγουϲ του ιερου} & 14 &  &  \\
&  & 15 & \foreignlanguage{greek}{και πρεϲβυτερουϲ ωϲ επι ληϲτην εξελη} & 20 &  &  \\
&  & 20 & \foreignlanguage{greek}{λυθατε μετα μαχαιρων και ξυλων καθ η} & 2 & \textbf{53} &  \\
&  & 2 & \foreignlanguage{greek}{μεραν οντοϲ μου μεθ υμων εν τω ιερω} & 9 &  &  \\
&  & 10 & \foreignlanguage{greek}{ουκ εξετινατε ταϲ χειραϲ επ εμε αλλ αυ} & 17 &  &  \\
&  & 17 & \foreignlanguage{greek}{τη εϲτιν υμων η ωρα και η εξουϲια του ϲκο} & 26 &  &  \\
&  & 26 & \foreignlanguage{greek}{τουϲ ϲυνλαβοντεϲ δε αυτον ηγαγο̅} & 4 & \textbf{54} &  \\
&  & 5 & \foreignlanguage{greek}{και ϲυνηγαγον αυτον ειϲ τον οικον του αρ} & 12 &  &  \\
&  & 12 & \foreignlanguage{greek}{χιερεωϲ ο δε πετροϲ ηκολουθει μακροθε̅} & 17 &  &  \\
& \textbf{55} &  & \foreignlanguage{greek}{αψαντων δε πυρ εν μεϲω τηϲ αυληϲ και} & 8 &  &  \\
&  & 9 & \foreignlanguage{greek}{ϲυνκαθειϲαντων αυτων εκαθητο ο πε} & 13 &  &  \\
&  & 13 & \foreignlanguage{greek}{τροϲ εν μεϲω αυτων ιδουϲα δε αυτον} & 3 & \textbf{56} &  \\
&  & 4 & \foreignlanguage{greek}{παιδιϲκη τιϲ καθημενον προϲ το φωϲ} & 9 &  &  \\
&  & 10 & \foreignlanguage{greek}{και ατενιϲαϲα αυτω ειπεν και ουτοϲ} & 15 &  &  \\
&  & 16 & \foreignlanguage{greek}{ϲυν αυτω ην ο δε ηρνηϲατο αυτον} & 4 & \textbf{57} &  \\
&  & 5 & \foreignlanguage{greek}{λεγων γυναι ουκ οιδα αυτον} & 9 &  &  \\
& \textbf{58} &  & \foreignlanguage{greek}{και μετα βραχυ ετεροϲ ιδων αυτον εφη} & 7 &  &  \\
&  & 8 & \foreignlanguage{greek}{και ϲυ εξ αυτων ει ο δε πετροϲ ειπεν} & 16 &  &  \\
&  & 17 & \foreignlanguage{greek}{ανθρωπε ουκ ειμει και διαϲτηϲαϲηϲ} & 2 & \textbf{59} &  \\
&  & 3 & \foreignlanguage{greek}{ωϲει ωραϲ μιαϲ αλλοϲ τιϲ διιϲχυριζετο λεγω̅} & 9 &  &  \\
&  & 10 & \foreignlanguage{greek}{επ αληθειαϲ και ουτοϲ μετ αυτου ην και} & 17 &  &  \\
&  & 18 & \foreignlanguage{greek}{γαρ γαλιλαιοϲ εϲτιν ειπεν δε ο πετροϲ} & 4 & \textbf{60} &  \\
&  & 5 & \foreignlanguage{greek}{ανθρωπε ουκ οιδα ο λεγειϲ} & 9 &  &  \\
[0.2em]
\cline{4-4}
\end{tabular}
\end{center}
\end{table}
}
\clearpage
\newpage
 {
 \setlength\arrayrulewidth{1pt}
\begin{table}
\begin{center}
\begin{tabular}{ccc|l|ccc}
\cline{4-4} \\ [-1em]
\multicolumn{7}{c}{\foreignlanguage{greek}{ευαγγελιον κατα λουκαν} \textbf{(\nospace{22:60})} } \\ \\ [-1em] % Si on veut ajouter les bordures latérales, remplacer {7}{c} par {7}{|c|}
\cline{4-4} \\
\cline{4-4}
&  &  & &  &  & \\ [-0.9em]
&  & 10 & \foreignlanguage{greek}{και παραχρημα ετι λαλουντοϲ αυτου εφω} & 15 &  &  \\
&  & 15 & \foreignlanguage{greek}{νηϲεν αλεκτωρ και ϲτραφειϲ ο \textoverline{κϲ} ενε} & 5 & \textbf{61} &  \\
&  & 5 & \foreignlanguage{greek}{βλεψεν τω πετρω και υπεμνηϲθη ο πε} & 11 &  &  \\
&  & 11 & \foreignlanguage{greek}{τροϲ του λογου του \textoverline{κυ} ωϲ ειπεν αυτω} & 18 &  &  \\
&  & 19 & \foreignlanguage{greek}{οτι πριν αλεκτορα φωνηϲαι απαρνηϲη με} & 24 &  &  \\
&  & 25 & \foreignlanguage{greek}{τριϲ και εξελθων εξω ο πετροϲ εκλαυ} & 6 & \textbf{62} &  \\
&  & 6 & \foreignlanguage{greek}{ϲεν πικρωϲ} & 7 &  &  \\
& \textbf{63} &  & \foreignlanguage{greek}{και οι ανδρεϲ οι ϲυνεχοντεϲ τον \textoverline{ιν} ενεπε} & 8 &  &  \\
&  & 8 & \foreignlanguage{greek}{ζον αυτω δεροντεϲ και περικαλυψαντεϲ} & 2 & \textbf{64} &  \\
&  & 3 & \foreignlanguage{greek}{αυτον ετυπτον αυτου το προϲωπον και} & 8 &  &  \\
&  & 9 & \foreignlanguage{greek}{επηρωτων αυτον λεγοντεϲ προφητευ} & 12 &  &  \\
&  & 12 & \foreignlanguage{greek}{ϲον τιϲ εϲτιν ο πεϲαϲ ϲε και ετερα πολλα} & 3 & \textbf{65} &  \\
&  & 4 & \foreignlanguage{greek}{βλαϲφημουντεϲ ελεγον ειϲ αυτον} & 7 &  &  \\
& \textbf{66} &  & \foreignlanguage{greek}{και ωϲ εγενετο ημερα ϲυνηχθη το πρεϲ} & 7 &  &  \\
&  & 7 & \foreignlanguage{greek}{βυτεριον του λαου αρχιερειϲ τε και γραμ} & 13 &  &  \\
&  & 13 & \foreignlanguage{greek}{ματειϲ και ανηγαγον αυτον ειϲ το ϲυνε} & 19 &  &  \\
&  & 19 & \foreignlanguage{greek}{δριον εαυτων λεγοντεϲ ει ϲυ ει ο \textoverline{χϲ} ει} & 7 & \textbf{67} &  \\
&  & 7 & \foreignlanguage{greek}{πε ημιν ειπεν δε αυτοιϲ} & 11 &  &  \\
&  & 12 & \foreignlanguage{greek}{εαν υμιν ειπω ου μη πιϲτευϲηται εα̅} & 1 & \textbf{68} &  \\
&  & 2 & \foreignlanguage{greek}{δε και ερωτηϲω ου μη αποκριθηται μοι} & 8 &  &  \\
&  & 9 & \foreignlanguage{greek}{η απολυϲηται} & 10 &  &  \\
& \textbf{69} &  & \foreignlanguage{greek}{απο του νυν εϲται ο υιοϲ του \textoverline{ανου} καθη} & 9 &  &  \\
&  & 9 & \foreignlanguage{greek}{μενοϲ εκ δεξιων τηϲ δυναμεωϲ του \textoverline{θυ}} & 15 &  &  \\
& \textbf{70} &  & \foreignlanguage{greek}{ειπον ουν δε παντεϲ ϲυ ουν ει ο υιοϲ του \textoverline{θυ}} & 11 &  &  \\
&  & 12 & \foreignlanguage{greek}{ο δε προϲ αυτουϲ εφη υμειϲ λεγεται οτι ε} & 20 &  &  \\
&  & 20 & \foreignlanguage{greek}{γω ειμει οι δε ειπον τι ετι χρειαν εχο} & 7 & \textbf{71} &  \\
&  & 7 & \foreignlanguage{greek}{μεν μαρτυριαϲ αυτοι γαρ ηκουϲαμεν α} & 12 &  &  \\
&  & 12 & \foreignlanguage{greek}{πο του ϲτοματοϲ αυτου και αναϲταν} & 2 & \mygospelchapter &  \\
&  & 3 & \foreignlanguage{greek}{απαν το πληθοϲ αυτων ηγαγεν αυτον} & 8 &  &  \\
&  & 9 & \foreignlanguage{greek}{επι τον πειλατον} & 11 &  &  \\
[0.2em]
\cline{4-4}
\end{tabular}
\end{center}
\end{table}
}
\clearpage
\newpage
 {
 \setlength\arrayrulewidth{1pt}
\begin{table}
\begin{center}
\begin{tabular}{ccc|l|ccc}
\cline{4-4} \\ [-1em]
\multicolumn{7}{c}{\foreignlanguage{greek}{ευαγγελιον κατα λουκαν} \textbf{(\nospace{23:2})} } \\ \\ [-1em] % Si on veut ajouter les bordures latérales, remplacer {7}{c} par {7}{|c|}
\cline{4-4} \\
\cline{4-4}
&  &  & &  &  & \\ [-0.9em]
& \textbf{2} &  & \foreignlanguage{greek}{ηρξαντο δε κατηγορειν αυτου λεγοντεϲ} & 5 &  &  \\
&  & 6 & \foreignlanguage{greek}{τουτον ευρομεν διαϲτρεφοντα το εθνοϲ} & 10 &  &  \\
&  & 11 & \foreignlanguage{greek}{και κωλυοντα καιϲαρι φορουϲ διδοναι λε} & 16 &  &  \\
&  & 16 & \foreignlanguage{greek}{γοντα εαυτον \textoverline{χν} βαϲιλεα ειναι} & 20 &  &  \\
& \textbf{3} &  & \foreignlanguage{greek}{ο δε πειλατοϲ επηρωτηϲεν αυτον λεγων} & 6 &  &  \\
&  & 7 & \foreignlanguage{greek}{ϲυ ει ο βαϲιλευϲ των ιουδαιων αυτοϲ εφη} & 14 &  &  \\
&  & 15 & \foreignlanguage{greek}{ϲυ λεγειϲ ο δε πειλατοϲ ειπεν προϲ τουϲ} & 6 & \textbf{4} &  \\
&  & 7 & \foreignlanguage{greek}{αρχιερειϲ και τουϲ οχλουϲ ουδεν ευριϲκω} & 12 &  &  \\
&  & 13 & \foreignlanguage{greek}{αιτιον εν τω \textoverline{ανω} τουτω οι δε επιϲχυον} & 3 & \textbf{5} &  \\
&  & 4 & \foreignlanguage{greek}{λεγοντεϲ οτι αναϲιει τον λαον διδαϲκω̅} & 9 &  &  \\
&  & 10 & \foreignlanguage{greek}{καθ οληϲ τηϲ ιουδαιαϲ αρξαμενοϲ απο τηϲ} & 16 &  &  \\
&  & 17 & \foreignlanguage{greek}{γαλιλαιαϲ εωϲ ωδε πειλατοϲ δε ακου} & 3 & \textbf{6} &  \\
&  & 3 & \foreignlanguage{greek}{ϲαϲ γαλιλαιαν επηρωτηϲεν ει ο \textoverline{ανοϲ} γαλι} & 9 &  &  \\
&  & 9 & \foreignlanguage{greek}{λαιοϲ εϲτιν και επιγνουϲ οτι εκ τηϲ εξου} & 6 & \textbf{7} &  \\
&  & 6 & \foreignlanguage{greek}{ϲιαϲ ηρωδου εϲτιν ανεπεμψεν αυτον} & 10 &  &  \\
&  & 11 & \foreignlanguage{greek}{προϲ ηρωδην οντα και αυτον εν ιεροϲολυ} & 17 &  &  \\
&  & 17 & \foreignlanguage{greek}{μοιϲ εν ταυταιϲ ταιϲ ημεραιϲ} & 21 &  &  \\
& \textbf{8} &  & \foreignlanguage{greek}{ο δε ηρωδηϲ ιδων τον \textoverline{ιν} εχαρη λιαν ην γαρ} & 10 &  &  \\
&  & 11 & \foreignlanguage{greek}{θελων εξ ικανου χρονου ιδειν αυτον δια} & 17 &  &  \\
&  & 18 & \foreignlanguage{greek}{το ακουειν πολλα περι αυτου και ηλπιζεν} & 24 &  &  \\
&  & 25 & \foreignlanguage{greek}{τι ϲημιον ιδειν υπ αυτου γεινομενον} & 30 &  &  \\
& \textbf{9} &  & \foreignlanguage{greek}{επηρωτα δε αυτον εν λογοιϲ εικανοιϲ αυ} & 7 &  &  \\
&  & 7 & \foreignlanguage{greek}{τοϲ δε ουδεν απεκρινατο αυτω} & 11 &  &  \\
& \textbf{10} &  & \foreignlanguage{greek}{ιϲτηκειϲαν δε οι αρχιερειϲ και οι γραμμα} & 7 &  &  \\
&  & 7 & \foreignlanguage{greek}{τιϲ ευτονωϲ κατηγορουντεϲ αυτου} & 10 &  &  \\
& \textbf{11} &  & \foreignlanguage{greek}{εξουθενιϲαϲ δε αυτον ηρωδηϲ ϲυν τοιϲ} & 6 &  &  \\
&  & 7 & \foreignlanguage{greek}{ϲτρατευμαϲιν αυτου και ενπεξαϲ περι} & 11 &  &  \\
&  & 11 & \foreignlanguage{greek}{βαλων αυτον εϲθητα λαμπραν ανεπεμ} & 15 &  &  \\
&  & 15 & \foreignlanguage{greek}{ψεν αυτον πειλατω εγενοντο δε φιλοι} & 3 & \textbf{12} &  \\
&  & 4 & \foreignlanguage{greek}{ο τε πειλατοϲ και ο ηρωδηϲ εν αυτη τη ημερα} & 13 &  &  \\
[0.2em]
\cline{4-4}
\end{tabular}
\end{center}
\end{table}
}
\clearpage
\newpage
 {
 \setlength\arrayrulewidth{1pt}
\begin{table}
\begin{center}
\begin{tabular}{ccc|l|ccc}
\cline{4-4} \\ [-1em]
\multicolumn{7}{c}{\foreignlanguage{greek}{ευαγγελιον κατα λουκαν} \textbf{(\nospace{23:12})} } \\ \\ [-1em] % Si on veut ajouter les bordures latérales, remplacer {7}{c} par {7}{|c|}
\cline{4-4} \\
\cline{4-4}
&  &  & &  &  & \\ [-0.9em]
&  & 14 & \foreignlanguage{greek}{μετ αλληλων προυπηρχον γαρ εν εχθρα ον} & 20 &  &  \\
&  & 20 & \foreignlanguage{greek}{τεϲ προϲ εαυτουϲ πειλατοϲ δε ϲυνκαλε} & 3 & \textbf{13} &  \\
&  & 3 & \foreignlanguage{greek}{ϲαμενοϲ τουϲ αρχιερειϲ και τουϲ αρχονταϲ} & 8 &  &  \\
&  & 9 & \foreignlanguage{greek}{και τον λαον ειπεν προϲ αυτουϲ} & 3 & \textbf{14} &  \\
&  & 4 & \foreignlanguage{greek}{προϲηνεγκατε μοι τον \textoverline{ανον} τουτον ωϲ} & 9 &  &  \\
&  & 10 & \foreignlanguage{greek}{αποϲτρεφοντα τον λαον και ιδου εγω} & 15 &  &  \\
&  & 16 & \foreignlanguage{greek}{ενωπιον υμων ανακριναϲ ουδεν ευρο̅} & 20 &  &  \\
&  & 21 & \foreignlanguage{greek}{εν τω \textoverline{ανω} τουτω αιτιον ων κατηγορειται} & 27 &  &  \\
&  & 28 & \foreignlanguage{greek}{κατ αυτου αλλ ουδε ηρωδηϲ} & 3 & \textbf{15} &  \\
&  & 4 & \foreignlanguage{greek}{ανεπεμψα γαρ υμαϲ προϲ αυτον και ιδου} & 10 &  &  \\
&  & 11 & \foreignlanguage{greek}{ουδεν αξιον θανατου εϲτιν πεπραγμενο̅} & 15 &  &  \\
&  & 16 & \foreignlanguage{greek}{αυτω παιδευϲαϲ ουν αυτον απολυϲω} & 4 & \textbf{16} &  \\
& \textbf{17} &  & \foreignlanguage{greek}{αναγκην δε ειχεν απολυειν αυτοιϲ κατα} & 6 &  &  \\
&  & 7 & \foreignlanguage{greek}{εορτην ενα ανεκραξαν ουν πανπληθει} & 3 & \textbf{18} &  \\
&  & 4 & \foreignlanguage{greek}{λεγοντεϲ ερε τουτον απολυϲον δε ημιν βα} & 10 &  &  \\
&  & 10 & \foreignlanguage{greek}{ραββαν οϲτιϲ ην δια ϲταϲιν τινα γενομε} & 6 & \textbf{19} &  \\
&  & 6 & \foreignlanguage{greek}{νην εν τη πολει και φονον βεβλημενοϲ} & 12 &  &  \\
&  & 13 & \foreignlanguage{greek}{ειϲ την φυλακην} & 15 &  &  \\
& \textbf{20} &  & \foreignlanguage{greek}{παλιν ουν ο πειλατοϲ προϲεφωνηϲεν θε} & 6 &  &  \\
&  & 6 & \foreignlanguage{greek}{λων απολυϲαι τον \textoverline{ιν} οι δε επεφωνουν} & 3 & \textbf{21} &  \\
&  & 4 & \foreignlanguage{greek}{λεγοντεϲ ϲταυρωϲον αυτον ο δε τρι} & 3 & \textbf{22} &  \\
&  & 3 & \foreignlanguage{greek}{τον ειπεν προϲ αυτουϲ τι γαρ κακον εποι} & 10 &  &  \\
&  & 10 & \foreignlanguage{greek}{ηϲεν ουτοϲ ουδεν αιτιον θανατου ευρο̅} & 15 &  &  \\
&  & 16 & \foreignlanguage{greek}{εν αυτω παιδευϲαϲ ουν αυτον απολυϲω} & 21 &  &  \\
& \textbf{23} &  & \foreignlanguage{greek}{οι δε επεκιντο φωναιϲ μεγαλαιϲ αιτου} & 6 &  &  \\
&  & 6 & \foreignlanguage{greek}{μενοι αυτον ϲταυρωθηναι και κατιϲχυ} & 10 &  &  \\
&  & 10 & \foreignlanguage{greek}{ον αι φωναι αυτων και των αρχιερεων} & 16 &  &  \\
& \textbf{24} &  & \foreignlanguage{greek}{ο δε πειλατοϲ επεκρινεν γενεϲθαι το αι} & 7 &  &  \\
&  & 7 & \foreignlanguage{greek}{τημα αυτων απελυϲεν δε τον δια ϲτα} & 5 & \textbf{25} &  \\
&  & 5 & \foreignlanguage{greek}{ϲιν και φονον βεβλημενον εν τη φυλακη} & 11 &  &  \\
[0.2em]
\cline{4-4}
\end{tabular}
\end{center}
\end{table}
}
\clearpage
\newpage
 {
 \setlength\arrayrulewidth{1pt}
\begin{table}
\begin{center}
\begin{tabular}{ccc|l|ccc}
\cline{4-4} \\ [-1em]
\multicolumn{7}{c}{\foreignlanguage{greek}{ευαγγελιον κατα λουκαν} \textbf{(\nospace{23:25})} } \\ \\ [-1em] % Si on veut ajouter les bordures latérales, remplacer {7}{c} par {7}{|c|}
\cline{4-4} \\
\cline{4-4}
&  &  & &  &  & \\ [-0.9em]
&  & 12 & \foreignlanguage{greek}{ον ητουντο τον δε \textoverline{ιν} παρεδωκεν τω θε} & 19 &  &  \\
&  & 19 & \foreignlanguage{greek}{ληματι αυτων και ωϲ απηγαγον αυτον} & 4 & \textbf{26} &  \\
&  & 5 & \foreignlanguage{greek}{επιλαβομενοι ϲιμωνοϲ τινοϲ κυρηναιου} & 8 &  &  \\
&  & 9 & \foreignlanguage{greek}{ερχομενου απ αγρου επεθηκαν αυτω το̅} & 14 &  &  \\
&  & 15 & \foreignlanguage{greek}{ϲταυρον φερειν οπιϲθεν του \textoverline{ιυ}} & 19 &  &  \\
& \textbf{27} &  & \foreignlanguage{greek}{ηκολουθει δε αυτω πολυ πληθοϲ του λαου} & 7 &  &  \\
&  & 8 & \foreignlanguage{greek}{και γυναικων αι και εκοπτοντο και ε} & 14 &  &  \\
&  & 14 & \foreignlanguage{greek}{θρηνουν αυτον} & 15 &  &  \\
& \textbf{28} &  & \foreignlanguage{greek}{ϲτραφειϲ δε προϲ αυταϲ ο \textoverline{ιϲ} ειπεν θυγα} & 8 &  &  \\
&  & 8 & \foreignlanguage{greek}{τερεϲ ιερουϲαλημ μη κλεεται επ εμε} & 13 &  &  \\
&  & 14 & \foreignlanguage{greek}{πλην εφ εαυταϲ κλεεται και επι τα τε} & 21 &  &  \\
&  & 21 & \foreignlanguage{greek}{κνα υμων οτι ιδου ερχονται ημεραι εν} & 5 & \textbf{29} &  \\
&  & 6 & \foreignlanguage{greek}{αιϲ αιρουϲιν μακαριαι αι ϲτιραι και κοιλι} & 12 &  &  \\
&  & 12 & \foreignlanguage{greek}{αι αι ουκ εγεννηϲαν και μαϲτοι οι ουκ ε} & 20 &  &  \\
&  & 20 & \foreignlanguage{greek}{θηλαϲαν τοτε αρξονται λεγειν τοιϲ} & 4 & \textbf{30} &  \\
&  & 5 & \foreignlanguage{greek}{ορεϲιν πεϲατε εφ ημαϲ και τοιϲ βουνοιϲ} & 11 &  &  \\
&  & 12 & \foreignlanguage{greek}{καλυψαται ημαϲ οτι ει εν τω υγρω ξυλω} & 6 & \textbf{31} &  \\
&  & 7 & \foreignlanguage{greek}{ταυτα ποιουϲιν εν τω ξηρω τι γενηται} & 13 &  &  \\
& \textbf{32} &  & \foreignlanguage{greek}{ηγοντο δε και ετεροι δυο κακουργοι ϲυν} & 7 &  &  \\
&  & 8 & \foreignlanguage{greek}{αυτω αναιρεθηναι και οτε απηλθο̅} & 3 & \textbf{33} &  \\
&  & 4 & \foreignlanguage{greek}{επι τον τοπον τον καλουμενον κρανιον} & 9 &  &  \\
&  & 10 & \foreignlanguage{greek}{εκει εϲταυρωϲαν αυτον και τουϲ κακουργουϲ} & 15 &  &  \\
&  & 16 & \foreignlanguage{greek}{ον μεν εκ δεξιων τον δε εξ αριϲτερων} & 23 &  &  \\
& \textbf{34} &  & \foreignlanguage{greek}{διαμεριζομενοι δε τα ιματια αυτου εβα} & 7 &  &  \\
&  & 7 & \foreignlanguage{greek}{λον κληρον εν οιϲ και εϲτηκει ο λαοϲ θε} & 7 & \textbf{35} &  \\
&  & 7 & \foreignlanguage{greek}{ωρων εξεμυκτηριζον δε και οι αρχοντεϲ} & 12 &  &  \\
&  & 13 & \foreignlanguage{greek}{ϲυν αυτοιϲ λεγοντεϲ αλλουϲ εϲωϲεν} & 17 &  &  \\
&  & 18 & \foreignlanguage{greek}{ϲωϲατω εαυτον ει ουτοϲ εϲτιν ο \textoverline{χϲ} του \textoverline{θυ}} & 26 &  &  \\
&  & 27 & \foreignlanguage{greek}{ο εκλεκτοϲ ενεπεζον δε αυτω και οι} & 5 & \textbf{36} &  \\
&  & 6 & \foreignlanguage{greek}{ϲτρατιωται προϲευχομενοι και οξοϲ} & 9 &  &  \\
[0.2em]
\cline{4-4}
\end{tabular}
\end{center}
\end{table}
}
\clearpage
\newpage
 {
 \setlength\arrayrulewidth{1pt}
\begin{table}
\begin{center}
\begin{tabular}{ccc|l|ccc}
\cline{4-4} \\ [-1em]
\multicolumn{7}{c}{\foreignlanguage{greek}{ευαγγελιον κατα λουκαν} \textbf{(\nospace{23:36})} } \\ \\ [-1em] % Si on veut ajouter les bordures latérales, remplacer {7}{c} par {7}{|c|}
\cline{4-4} \\
\cline{4-4}
&  &  & &  &  & \\ [-0.9em]
&  & 10 & \foreignlanguage{greek}{προϲφεροντεϲ αυτω και λεγοντεϲ ει ϲυ} & 4 & \textbf{37} &  \\
&  & 5 & \foreignlanguage{greek}{ει ο βαϲιλευϲ των ιουδαιων ϲωϲον ϲεαυτο̅} & 11 &  &  \\
& \textbf{38} &  & \foreignlanguage{greek}{ην δε και επιγραφη γεγραμμενη επ αυτω} & 7 &  &  \\
&  & 8 & \foreignlanguage{greek}{γραμμαϲιν ελληνικοιϲ και ρωμαικοιϲ} & 11 &  &  \\
&  & 12 & \foreignlanguage{greek}{και εβραικοιϲ ουτοϲ εϲτιν ο βαϲιλευϲ} & 17 &  &  \\
&  & 18 & \foreignlanguage{greek}{των ιουδαιων} & 19 &  &  \\
& \textbf{39} &  & \foreignlanguage{greek}{ειϲ δε των κρεμαϲθεντων κακουργων} & 5 &  &  \\
&  & 6 & \foreignlanguage{greek}{εβλαϲφημει αυτον λεγων ει ϲυ ει ο \textoverline{χϲ} ϲω} & 14 &  &  \\
&  & 14 & \foreignlanguage{greek}{ϲον και αυτον και ημαϲ αποκριθειϲ δε} & 2 & \textbf{40} &  \\
&  & 3 & \foreignlanguage{greek}{ο ετεροϲ επετειμα αυτω λεγων ουδε} & 8 &  &  \\
&  & 9 & \foreignlanguage{greek}{φοβη ϲυ τον \textoverline{θν} οτι εν τω αυτω κριματι} & 17 &  &  \\
&  & 18 & \foreignlanguage{greek}{εϲμεν και ημειϲ μεν δικαιωϲ αξια γαρ} & 6 & \textbf{41} &  \\
&  & 7 & \foreignlanguage{greek}{ων επραξαμεν απολαμβανομεν} & 9 &  &  \\
&  & 10 & \foreignlanguage{greek}{ουτοϲ δε ουδεν ατοπον επραξεν και ε} & 2 & \textbf{42} &  \\
&  & 2 & \foreignlanguage{greek}{λεγεν τω \textoverline{ιυ} μνηϲθητι μου \textoverline{κε} οταν ελ} & 9 &  &  \\
&  & 9 & \foreignlanguage{greek}{θηϲ εν τη βαϲιλεια ϲου} & 13 &  &  \\
& \textbf{43} &  & \foreignlanguage{greek}{και ειπεν αυτω ο \textoverline{ιϲ} αμην λεγω ϲοι ϲη} & 9 &  &  \\
&  & 9 & \foreignlanguage{greek}{μερον μετ εμου εϲη εν τω παραδιϲω} & 16 &  &  \\
& \textbf{44} &  & \foreignlanguage{greek}{ην δε ωϲει ωρα εκτη και ϲκοτοϲ εγενετο} & 8 &  &  \\
&  & 9 & \foreignlanguage{greek}{εφ ολην την γην εωϲ ωραϲ ενατηϲ} & 15 &  &  \\
& \textbf{45} &  & \foreignlanguage{greek}{και εϲκοτιϲθη ο ηλιοϲ και εϲχιϲθη το} & 7 &  &  \\
&  & 8 & \foreignlanguage{greek}{καταπεταϲμα του ναου μεϲον και φω} & 2 & \textbf{46} &  \\
&  & 2 & \foreignlanguage{greek}{νηϲαϲ φωνη μεγαλη ο \textoverline{ιϲ} ειπεν \textoverline{περ} ειϲ} & 9 &  &  \\
&  & 10 & \foreignlanguage{greek}{χειραϲ ϲου παρατιθεμαι το \textoverline{πνα} μου του} & 16 &  &  \\
&  & 16 & \foreignlanguage{greek}{το δε ειπων εξεπνευϲεν} & 19 &  &  \\
& \textbf{47} &  & \foreignlanguage{greek}{ιδων δε ο εκατονταρχοϲ το γενομενο̅} & 6 &  &  \\
&  & 7 & \foreignlanguage{greek}{εδοξαϲεν τον \textoverline{θν} λεγων οντωϲ ο \textoverline{ανοϲ} ου} & 14 &  &  \\
&  & 14 & \foreignlanguage{greek}{τοϲ δικαιοϲ ην και παντεϲ οι ϲυνπαρα} & 4 & \textbf{48} &  \\
&  & 4 & \foreignlanguage{greek}{γενομενοι οχλοι επι την θεωριαν ταυτη̅} & 9 &  &  \\
&  & 10 & \foreignlanguage{greek}{θεωρουντεϲ τα γενομενα τυπτοντεϲ αυ} & 14 &  &  \\
[0.2em]
\cline{4-4}
\end{tabular}
\end{center}
\end{table}
}
\clearpage
\newpage
 {
 \setlength\arrayrulewidth{1pt}
\begin{table}
\begin{center}
\begin{tabular}{ccc|l|ccc}
\cline{4-4} \\ [-1em]
\multicolumn{7}{c}{\foreignlanguage{greek}{ευαγγελιον κατα λουκαν} \textbf{(\nospace{23:48})} } \\ \\ [-1em] % Si on veut ajouter les bordures latérales, remplacer {7}{c} par {7}{|c|}
\cline{4-4} \\
\cline{4-4}
&  &  & &  &  & \\ [-0.9em]
&  & 14 & \foreignlanguage{greek}{των τα ϲτηθη υπεϲτρεφον} & 17 &  &  \\
& \textbf{49} &  & \foreignlanguage{greek}{ιϲτηκειϲαν δε παντεϲ οι γνωϲτοι αυτου} & 6 &  &  \\
&  & 7 & \foreignlanguage{greek}{μακροθεν και γυναικεϲ αι ϲυνακολουθη} & 11 &  &  \\
&  & 11 & \foreignlanguage{greek}{ϲαϲαι αυτω απο τηϲ γαλιλαιαϲ ορωϲαι ταυτα} & 17 &  &  \\
& \textbf{50} &  & \foreignlanguage{greek}{και ιδου ανηρ ονοματι ιωϲηφ βουλευτηϲ} & 6 &  &  \\
&  & 7 & \foreignlanguage{greek}{υπαρχων ανηρ αγαθοϲ και δικαιοϲ} & 11 &  &  \\
& \textbf{51} &  & \foreignlanguage{greek}{ουτοϲ ουκ ην ϲυνκατατεθειμενοϲ τη βου} & 6 &  &  \\
&  & 6 & \foreignlanguage{greek}{λη και τη πραξει αυτων απο αριμαθιαϲ} & 12 &  &  \\
&  & 13 & \foreignlanguage{greek}{πολεωϲ των ιουδαιων οϲ και προϲε} & 18 &  &  \\
&  & 18 & \foreignlanguage{greek}{δεχετο και αυτοϲ την βαϲιλειαν του \textoverline{θυ}} & 24 &  &  \\
& \textbf{52} &  & \foreignlanguage{greek}{ουτοϲ προϲελθων τω πειλατω ητηϲατο} & 5 &  &  \\
&  & 6 & \foreignlanguage{greek}{το ϲωμα του \textoverline{ιυ} και καθελων αυτο ενε} & 4 & \textbf{53} &  \\
&  & 4 & \foreignlanguage{greek}{τυλιξεν ϲινδονι και εθηκεν αυτο εν} & 9 &  &  \\
&  & 10 & \foreignlanguage{greek}{μνηματι λαξευτω ου ουκ ην ουδειϲ ουδε} & 16 &  &  \\
&  & 16 & \foreignlanguage{greek}{πω κειμενοϲ και ημερα ην παραϲκευη} & 4 & \textbf{54} &  \\
&  & 5 & \foreignlanguage{greek}{ϲαββατον επιφαυϲκεν} & 6 &  &  \\
& \textbf{55} &  & \foreignlanguage{greek}{κατακολουθηϲαϲαι δε γυναικεϲ αιτινεϲ} & 4 &  &  \\
&  & 5 & \foreignlanguage{greek}{ηϲαν ϲυνεληλυθυειαι αυτω εκ τηϲ γαλιλαιαϲ} & 10 &  &  \\
&  & 11 & \foreignlanguage{greek}{εθεαϲαντο το μνημιον και ωϲ ετεθη το} & 17 &  &  \\
&  & 18 & \foreignlanguage{greek}{ϲωμα αυτου υποϲτρεψαϲαι δε ητοι} & 3 & \textbf{56} &  \\
&  & 3 & \foreignlanguage{greek}{μαϲαν αρωματα και μυρα και το μεν ϲαβ} & 10 &  &  \\
&  & 10 & \foreignlanguage{greek}{βατον ηϲυχαϲαν κατα την εντολην} & 14 &  &  \\
& \mygospelchapter &  & \foreignlanguage{greek}{τη δε μια των ϲαββατων ορθρου βαθεοϲ} & 7 &  &  \\
&  & 8 & \foreignlanguage{greek}{ηλθον επι το μνημα φερουϲαι α ητοιμαϲα̅} & 14 &  &  \\
&  & 15 & \foreignlanguage{greek}{αρωματα και τινεϲ ϲυν αυταιϲ ευρον} & 1 & \textbf{2} &  \\
&  & 2 & \foreignlanguage{greek}{δε τον λιθον αποκεκυλιϲμενον απο του} & 7 &  &  \\
&  & 8 & \foreignlanguage{greek}{μνημιου και ειϲελθουϲαι ουχ ευρον το} & 5 & \textbf{3} &  \\
&  & 6 & \foreignlanguage{greek}{ϲωμα του \textoverline{κυ} \textoverline{ιυ}} & 9 &  &  \\
& \textbf{4} &  & \foreignlanguage{greek}{και εγενετο εν τω διαποριϲθαι αυταϲ} & 6 &  &  \\
&  & 7 & \foreignlanguage{greek}{περι τουτου και ιδου ανδρεϲ δυο επε} & 13 &  &  \\
[0.2em]
\cline{4-4}
\end{tabular}
\end{center}
\end{table}
}
\clearpage
\newpage
 {
 \setlength\arrayrulewidth{1pt}
\begin{table}
\begin{center}
\begin{tabular}{ccc|l|ccc}
\cline{4-4} \\ [-1em]
\multicolumn{7}{c}{\foreignlanguage{greek}{ευαγγελιον κατα λουκαν} \textbf{(\nospace{24:4})} } \\ \\ [-1em] % Si on veut ajouter les bordures latérales, remplacer {7}{c} par {7}{|c|}
\cline{4-4} \\
\cline{4-4}
&  &  & &  &  & \\ [-0.9em]
&  & 13 & \foreignlanguage{greek}{ϲτηϲαν αυταιϲ εν αιϲθηϲεϲιν αϲτραπτουϲαιϲ} & 17 &  &  \\
& \textbf{5} &  & \foreignlanguage{greek}{ενφοβων δε γενομενων αυτων και κλει} & 6 &  &  \\
&  & 6 & \foreignlanguage{greek}{νουϲων το προϲωπον ειϲ την γην ειπον} & 12 &  &  \\
&  & 13 & \foreignlanguage{greek}{προϲ αυταϲ τι ζητειται τον ζωντα με} & 19 &  &  \\
&  & 19 & \foreignlanguage{greek}{τα των νεκρων ουκ εϲτιν ωδε αλλα} & 4 & \textbf{6} &  \\
&  & 5 & \foreignlanguage{greek}{ανεϲτη μνηϲθηται ωϲ ελαληϲεν} & 8 &  &  \\
&  & 9 & \foreignlanguage{greek}{υμιν ετι ων εν τη γαλιλαια λεγων} & 1 & \textbf{7} &  \\
&  & 2 & \foreignlanguage{greek}{οτι δει τον υιον του \textoverline{ανου} παραδοθηναι} & 8 &  &  \\
&  & 9 & \foreignlanguage{greek}{ειϲ χειραϲ \textoverline{ανων} αμαρτωλων και ϲταυρω} & 14 &  &  \\
&  & 14 & \foreignlanguage{greek}{θηναι και τη τριτη ημερα αναϲτηναι} & 19 &  &  \\
& \textbf{8} &  & \foreignlanguage{greek}{και εμνηϲθηϲαν των ρηματων αυτου} & 5 &  &  \\
& \textbf{9} &  & \foreignlanguage{greek}{και υποϲτρεψαϲαι απο του μνημιου απηγ} & 6 &  &  \\
&  & 6 & \foreignlanguage{greek}{γειλαν ταυτα παντα τοιϲ ενδεκα και} & 11 &  &  \\
&  & 12 & \foreignlanguage{greek}{παϲιν τοιϲ λοιποιϲ η μαγδαληνη μα} & 3 & \textbf{10} &  \\
&  & 3 & \foreignlanguage{greek}{ρια και ιωαννα και μαρια η ιακωβου} & 9 &  &  \\
&  & 10 & \foreignlanguage{greek}{και αι λοιπαι ϲυν αυταιϲ ελεγον προϲ τουϲ} & 17 &  &  \\
&  & 18 & \foreignlanguage{greek}{αποϲτολουϲ ταυτα και εφανηϲαν} & 2 & \textbf{11} &  \\
&  & 3 & \foreignlanguage{greek}{ενωπιον αυτων ωϲει ληροϲ τα ρηματα} & 8 &  &  \\
&  & 9 & \foreignlanguage{greek}{αυτων και ηπιϲτουν αυταιϲ} & 12 &  &  \\
& \textbf{12} &  & \foreignlanguage{greek}{ο δε πετροϲ αναϲταϲ εδραμεν επι το μνη} & 8 &  &  \\
&  & 8 & \foreignlanguage{greek}{μιον και παρακυψαϲ βλεπει τα οθονια μο} & 14 &  &  \\
&  & 14 & \foreignlanguage{greek}{να και απηλθεν προϲ εαυτον θαυμαζω̅} & 19 &  &  \\
&  & 20 & \foreignlanguage{greek}{το γεγονοϲ και ιδου δυο εξ αυτων} & 5 & \textbf{13} &  \\
&  & 6 & \foreignlanguage{greek}{ηϲαν πορευομενοι εν αυτη τη ημερα} & 11 &  &  \\
&  & 12 & \foreignlanguage{greek}{ειϲ κωμην απεχουϲαν ϲταδιουϲ εξηκο̅} & 16 &  &  \\
&  & 16 & \foreignlanguage{greek}{τα απο ιερουϲαλημ η ονομα εμμαουϲ} & 21 &  &  \\
& \textbf{14} &  & \foreignlanguage{greek}{και αυτοι ωμιλουν περι παντων προϲ} & 6 &  &  \\
&  & 7 & \foreignlanguage{greek}{αλληλουϲ των ϲυμβεβη} & 11 &  &  \\
&  & 11 & \foreignlanguage{greek}{κοτων τουτων και εγενετο εν τω ο} & 5 & \textbf{15} &  \\
&  & 5 & \foreignlanguage{greek}{μιλειν αυτουϲ και ϲυνζητειν} & 8 &  &  \\
[0.2em]
\cline{4-4}
\end{tabular}
\end{center}
\end{table}
}
\clearpage
\newpage
 {
 \setlength\arrayrulewidth{1pt}
\begin{table}
\begin{center}
\begin{tabular}{ccc|l|ccc}
\cline{4-4} \\ [-1em]
\multicolumn{7}{c}{\foreignlanguage{greek}{ευαγγελιον κατα λουκαν} \textbf{(\nospace{24:15})} } \\ \\ [-1em] % Si on veut ajouter les bordures latérales, remplacer {7}{c} par {7}{|c|}
\cline{4-4} \\
\cline{4-4}
&  &  & &  &  & \\ [-0.9em]
&  & 9 & \foreignlanguage{greek}{και αυτοϲ ο \textoverline{ιϲ} εγγιϲαϲ ϲυνεπορευετο αυτοιϲ} & 15 &  &  \\
& \textbf{16} &  & \foreignlanguage{greek}{οι δε οφθαλμοι αυτων εκρατουντο του} & 6 &  &  \\
&  & 7 & \foreignlanguage{greek}{μη επιγνωναι αυτον} & 9 &  &  \\
& \textbf{17} &  & \foreignlanguage{greek}{ειπεν δε προϲ αυτουϲ τινεϲ οι λογοι ουτοι} & 8 &  &  \\
&  & 9 & \foreignlanguage{greek}{ουϲ αντιβαλλεται προϲ αλληλουϲ περιπα} & 13 &  &  \\
&  & 13 & \foreignlanguage{greek}{τουντεϲ και εϲται ϲκυθρωποι} & 16 &  &  \\
& \textbf{18} &  & \foreignlanguage{greek}{αποκριθειϲ δε ο ειϲ ω ονομα κλεοπαϲ ειπε̅} & 8 &  &  \\
&  & 9 & \foreignlanguage{greek}{προϲ αυτον ϲυ μονοϲ παροικειϲ ιερουϲα} & 14 &  &  \\
&  & 14 & \foreignlanguage{greek}{λημ και ουκ εγνωϲ τα γενομενα εν αυτη} & 21 &  &  \\
&  & 22 & \foreignlanguage{greek}{εν ταιϲ ημεραιϲ ταυταιϲ και ειπεν αυ} & 3 & \textbf{19} &  \\
&  & 3 & \foreignlanguage{greek}{τοιϲ ποια οι δε ειπον αυτω τα περι \textoverline{ιυ}} & 11 &  &  \\
&  & 12 & \foreignlanguage{greek}{του ναζωραιου οϲ εγενετο ανηρ προφη} & 17 &  &  \\
&  & 17 & \foreignlanguage{greek}{τηϲ δυνατοϲ εν εργω και λογω εναντιο̅} & 23 &  &  \\
&  & 24 & \foreignlanguage{greek}{του \textoverline{θυ} και παντοϲ του λαου οπωϲ τε αυ} & 3 & \textbf{20} &  \\
&  & 3 & \foreignlanguage{greek}{τον παρεδωκαν οι αρχιερειϲ και οι αρχον} & 9 &  &  \\
&  & 9 & \foreignlanguage{greek}{τεϲ ημων ειϲ κριμα θανατου και εϲταυ} & 15 &  &  \\
&  & 15 & \foreignlanguage{greek}{ρωϲαν αυτον ημειϲ δε ηλπιζομεν ο} & 4 & \textbf{21} &  \\
&  & 4 & \foreignlanguage{greek}{τι αυτοϲ εϲτιν ο μελλων λυτρουϲθαι το̅} & 10 &  &  \\
&  & 11 & \foreignlanguage{greek}{ιϲραηλ αλλα γε ϲυμ παϲιν τουτοιϲ τρι} & 17 &  &  \\
&  & 17 & \foreignlanguage{greek}{την ταυτην ημεραν αγει ϲημερον αφ ου} & 23 &  &  \\
&  & 24 & \foreignlanguage{greek}{ταυτα εγενετο αλλα και γυναικεϲ} & 3 & \textbf{22} &  \\
&  & 4 & \foreignlanguage{greek}{τινεϲ εξ ημων εξεϲτηϲαν ημαϲ γενο} & 9 &  &  \\
&  & 9 & \foreignlanguage{greek}{μεναι ορθρειναι επι το μνημιον} & 14 &  &  \\
& \textbf{23} &  & \foreignlanguage{greek}{και μη ευρουϲαι το ϲωμα αυτου ηλθον} & 7 &  &  \\
&  & 8 & \foreignlanguage{greek}{λεγουϲαι και οπταϲιαν αγγελων εωρακε} & 12 &  &  \\
&  & 12 & \foreignlanguage{greek}{ναι οι λεγουϲιν αυτον ζην και απηλθο̅} & 2 & \textbf{24} &  \\
&  & 3 & \foreignlanguage{greek}{τινεϲ των ϲυν ημιν επι το μνημιον και ευ} & 11 &  &  \\
&  & 11 & \foreignlanguage{greek}{ρον ουτωϲ καθωϲ και αι γυναικεϲ ειπον} & 17 &  &  \\
&  & 18 & \foreignlanguage{greek}{αυτον δε ουχ ειδον} & 21 &  &  \\
& \textbf{25} &  & \foreignlanguage{greek}{και αυτοϲ ειπεν προϲ αυτουϲ ω ανοητοι και} & 8 &  &  \\
[0.2em]
\cline{4-4}
\end{tabular}
\end{center}
\end{table}
}
\clearpage
\newpage
 {
 \setlength\arrayrulewidth{1pt}
\begin{table}
\begin{center}
\begin{tabular}{ccc|l|ccc}
\cline{4-4} \\ [-1em]
\multicolumn{7}{c}{\foreignlanguage{greek}{ευαγγελιον κατα λουκαν} \textbf{(\nospace{24:25})} } \\ \\ [-1em] % Si on veut ajouter les bordures latérales, remplacer {7}{c} par {7}{|c|}
\cline{4-4} \\
\cline{4-4}
&  &  & &  &  & \\ [-0.9em]
&  & 9 & \foreignlanguage{greek}{βραδειϲ τη καρδια του πιϲτευειν επι παϲιν} & 15 &  &  \\
&  & 16 & \foreignlanguage{greek}{οιϲ ελαληϲαν οι προφηται ουχι ταυτα ε} & 3 & \textbf{26} &  \\
&  & 3 & \foreignlanguage{greek}{δει παθειν τον \textoverline{χν} και ειϲελθειν ειϲ την} & 10 &  &  \\
&  & 11 & \foreignlanguage{greek}{δοξαν αυτου και αρξαμενοϲ απο μωυ} & 4 & \textbf{27} &  \\
&  & 4 & \foreignlanguage{greek}{ϲεωϲ και απο παντων των προφητων} & 9 &  &  \\
&  & 10 & \foreignlanguage{greek}{διερμηνευειν αυτοιϲ εν παϲαιϲ ταιϲ γρα} & 15 &  &  \\
&  & 15 & \foreignlanguage{greek}{φαιϲ τα περι αυτου και ηγγειϲαν ειϲ τη̅} & 4 & \textbf{28} &  \\
&  & 5 & \foreignlanguage{greek}{κωμην ου επορευοντο και αυτοϲ προϲ} & 10 &  &  \\
&  & 10 & \foreignlanguage{greek}{εποιειτο πορρωτερω πορευεϲθαι και} & 1 & \textbf{29} &  \\
&  & 2 & \foreignlanguage{greek}{παρεβιαϲαντο αυτον λεγοντεϲ μεινο̅} & 5 &  &  \\
&  & 6 & \foreignlanguage{greek}{μεθ ημων οτι προϲ εϲπεραϲ εϲτιν και κε} & 13 &  &  \\
&  & 13 & \foreignlanguage{greek}{κλεικεν η ημερα και ειϲηλθεν του μει} & 19 &  &  \\
&  & 19 & \foreignlanguage{greek}{ναι ϲυν αυτοιϲ και εγενετο εν τω κα} & 5 & \textbf{30} &  \\
&  & 5 & \foreignlanguage{greek}{τακειϲθαι αυτον μετ αυτων λαβων ευλογη} & 10 &  &  \\
&  & 10 & \foreignlanguage{greek}{ϲεν και κλαϲαϲ επεδιδου αυτοιϲ} & 14 &  &  \\
& \textbf{31} &  & \foreignlanguage{greek}{αυτων δε διηνοιχθηϲαν οι οφθαλμοι} & 5 &  &  \\
&  & 6 & \foreignlanguage{greek}{και επεγνωϲαν αυτον και αυτοϲ αφα̅} & 11 &  &  \\
&  & 11 & \foreignlanguage{greek}{τοϲ εγενετο απ αυτων και ειπον προϲ} & 3 & \textbf{32} &  \\
&  & 4 & \foreignlanguage{greek}{αλληλουϲ ουχι η καρδια ημων και} & 9 &  &  \\
&  & 9 & \foreignlanguage{greek}{ομενη ην εν ημιν ωϲ ελαλει ημιν εν} & 16 &  &  \\
&  & 17 & \foreignlanguage{greek}{τη οδω και ωϲ διηνοιγεν ημιν ταϲ} & 23 &  &  \\
&  & 24 & \foreignlanguage{greek}{γραφαϲ και αναϲταντεϲ αυτη τη ω} & 5 & \textbf{33} &  \\
&  & 5 & \foreignlanguage{greek}{ρα υψεϲτρεψαν ειϲ ιερουϲαλημ και} & 9 &  &  \\
&  & 10 & \foreignlanguage{greek}{ευρον ϲυνηθροιϲμενουϲ τουϲ ενδεκα} & 13 &  &  \\
&  & 14 & \foreignlanguage{greek}{και τουϲ ϲυν αυτοιϲ λεγονταϲ} & 1 & \textbf{34} &  \\
&  & 2 & \foreignlanguage{greek}{οτι ηγερθη ο \textoverline{κϲ} οντωϲ και ωφθη ϲιμωνι και αυ} & 2 & \textbf{35} &  \\
&  & 2 & \foreignlanguage{greek}{τοι εξηγουντο το εν τη οδω και ωϲ ε} & 10 &  &  \\
&  & 10 & \foreignlanguage{greek}{γνωϲθη αυτοιϲ εν τη κλαϲι του αρτου} & 16 &  &  \\
& \textbf{36} &  & \foreignlanguage{greek}{ταυτα δε αυτων λαλουντων αυτοιϲ} & 5 &  &  \\
&  & 6 & \foreignlanguage{greek}{ο \textoverline{ιϲ} εϲτη εν μεϲω αυτων και λεγει αυτοιϲ} & 14 &  &  \\
[0.2em]
\cline{4-4}
\end{tabular}
\end{center}
\end{table}
}
\clearpage
\newpage
 {
 \setlength\arrayrulewidth{1pt}
\begin{table}
\begin{center}
\begin{tabular}{ccc|l|ccc}
\cline{4-4} \\ [-1em]
\multicolumn{7}{c}{\foreignlanguage{greek}{ευαγγελιον κατα λουκαν} \textbf{(\nospace{24:36})} } \\ \\ [-1em] % Si on veut ajouter les bordures latérales, remplacer {7}{c} par {7}{|c|}
\cline{4-4} \\
\cline{4-4}
&  &  & &  &  & \\ [-0.9em]
&  & 15 & \foreignlanguage{greek}{εγω ειμει μη φοβειϲθαι ειρηνη υμιν} & 20 &  &  \\
& \textbf{37} &  & \foreignlanguage{greek}{φοβηθεντεϲ δε και εμφοβοι γενομενοι ε} & 6 &  &  \\
&  & 6 & \foreignlanguage{greek}{δοκουν \textoverline{πνα} θεωρειν και ειπεν αυτοιϲ} & 3 & \textbf{38} &  \\
&  & 4 & \foreignlanguage{greek}{τι τεταραγμενοι εϲται και δια τι διαλογι} & 10 &  &  \\
&  & 10 & \foreignlanguage{greek}{ϲμοι αναβαινουϲιν εν ταιϲ καρδιαιϲ υμω̅} & 15 &  &  \\
& \textbf{39} &  & \foreignlanguage{greek}{ειδετε ταϲ χειραϲ μου και τουϲ ποδαϲ οτι} & 8 &  &  \\
&  & 9 & \foreignlanguage{greek}{αυτοϲ εγω ειμει ψηλαφηϲατε και ιδετε} & 14 &  &  \\
&  & 15 & \foreignlanguage{greek}{οτι \textoverline{πνα} ϲαρκα και οϲτεα ουκ εχει καθωϲ με} & 23 &  &  \\
&  & 24 & \foreignlanguage{greek}{θεωρειται εχοντα και τουτο ειπων επε} & 4 & \textbf{40} &  \\
&  & 4 & \foreignlanguage{greek}{δειξεν αυτοιϲ ταϲ χειραϲ και τουϲ ποδαϲ} & 10 &  &  \\
& \textbf{41} &  & \foreignlanguage{greek}{ετι δε απιϲτουντων αυτων απο τη χαραϲ} & 7 &  &  \\
&  & 8 & \foreignlanguage{greek}{και θαυμαζοντων ειπεν αυτοιϲ εχε} & 12 &  &  \\
&  & 12 & \foreignlanguage{greek}{τε τι βρωϲιμον ενθαδε} & 15 &  &  \\
& \textbf{42} &  & \foreignlanguage{greek}{οι δε επεδωκαν αυτω ιχθυοϲ οπτου με} & 7 &  &  \\
&  & 7 & \foreignlanguage{greek}{ροϲ και λαβων ενωπιον αυτων εφαγεν} & 5 & \textbf{43} &  \\
& \textbf{44} &  & \foreignlanguage{greek}{ειπεν δε αυτοιϲ ουτοι οι λογοι ουϲ ελαλη} & 8 &  &  \\
&  & 8 & \foreignlanguage{greek}{ϲα προϲ υμαϲ ετι ων ϲυν υμιν οτι δει} & 16 &  &  \\
&  & 17 & \foreignlanguage{greek}{πληρωθηναι παντα τα γεγραμμενα εν} & 21 &  &  \\
&  & 22 & \foreignlanguage{greek}{τω νομω μωυϲεωϲ και προφηταιϲ και} & 27 &  &  \\
&  & 28 & \foreignlanguage{greek}{ψαλμοιϲ περι εμου} & 30 &  &  \\
& \textbf{45} &  & \foreignlanguage{greek}{τοτε διηνοιξεν αυτων τον νουν του ϲυν} & 7 &  &  \\
&  & 7 & \foreignlanguage{greek}{ειεναι ταϲ γραφαϲ και ειπεν αυτοιϲ οτι} & 1 & \textbf{46} &  \\
&  & 2 & \foreignlanguage{greek}{ουτωϲ γεγραπται και ουτωϲ εδει πα} & 7 &  &  \\
&  & 7 & \foreignlanguage{greek}{θειν τον \textoverline{χν} και αναϲτηναι εκ νεκρων} & 13 &  &  \\
&  & 14 & \foreignlanguage{greek}{τη τριτη ημερα και κηρυχθηναι επι} & 3 & \textbf{47} &  \\
&  & 4 & \foreignlanguage{greek}{τω ονοματι αυτου μετανοιαν και αφε} & 9 &  &  \\
&  & 9 & \foreignlanguage{greek}{ϲιν αμαρτιων ειϲ παντα τα εθνη αρξα} & 15 &  &  \\
&  & 15 & \foreignlanguage{greek}{μενον απο ιερουϲαλημ} & 17 &  &  \\
& \textbf{48} &  & \foreignlanguage{greek}{υμειϲ δε εϲται μαρτυρεϲ τουτων} & 5 &  &  \\
& \textbf{49} &  & \foreignlanguage{greek}{και εγω ιδου αποϲτελλω την επαγγελεια̅} & 6 &  &  \\
[0.2em]
\cline{4-4}
\end{tabular}
\end{center}
\end{table}
}
\clearpage
\newpage
 {
 \setlength\arrayrulewidth{1pt}
\begin{table}
\begin{center}
\begin{tabular}{ccc|l|ccc}
\cline{4-4} \\ [-1em]
\multicolumn{7}{c}{\foreignlanguage{greek}{ευαγγελιον κατα λουκαν} \textbf{(\nospace{24:49})} } \\ \\ [-1em] % Si on veut ajouter les bordures latérales, remplacer {7}{c} par {7}{|c|}
\cline{4-4} \\
\cline{4-4}
&  &  & &  &  & \\ [-0.9em]
&  & 7 & \foreignlanguage{greek}{του \textoverline{πρϲ} μου εφ υμαϲ υμειϲ δε καθειϲατε} & 14 &  &  \\
&  & 15 & \foreignlanguage{greek}{εν τη πολει ιερουϲαλημ εωϲ ου ενδυϲηϲθαι} & 21 &  &  \\
&  & 22 & \foreignlanguage{greek}{δυναμιν εξ υψουϲ εξηγαγεν δε αυτουϲ} & 3 & \textbf{50} &  \\
&  & 4 & \foreignlanguage{greek}{εξω εωϲ ειϲ βηθανιαν και επαραϲ ταϲ χειραϲ} & 12 &  &  \\
&  & 13 & \foreignlanguage{greek}{ηυλογηϲεν αυτουϲ και εγενετο εν τω} & 4 & \textbf{51} &  \\
&  & 5 & \foreignlanguage{greek}{ευλογειν αυτον αυτουϲ διεϲτη απ αυτων} & 10 &  &  \\
&  & 11 & \foreignlanguage{greek}{και ανεφερετο ειϲ τον ουρανον} & 15 &  &  \\
& \textbf{52} &  & \foreignlanguage{greek}{και αυτοι προϲκυνηϲαντεϲ αυτον υπε} & 5 &  &  \\
&  & 5 & \foreignlanguage{greek}{ϲτρεψαν ειϲ ιερουϲαλημ μετα χαραϲ} & 9 &  &  \\
&  & 10 & \foreignlanguage{greek}{μεγαληϲ και ηϲαν δια του παντοϲ εν τω} & 7 & \textbf{53} &  \\
&  & 8 & \foreignlanguage{greek}{ιερω αινουντεϲ και ευλογουντεϲ} & 11 &  &  \\
&  & 12 & \foreignlanguage{greek}{τον \textoverline{θν}} & 13 &  &  \\
[0.2em]
\cline{4-4}
\end{tabular}
\end{center}
\end{table}
}
\clearpage
\newpage
 {
 \setlength\arrayrulewidth{1pt}
\begin{table}
\begin{center}
\begin{tabular}{ccc|l|ccc}
\cline{4-4} \\ [-1em]
\multicolumn{7}{c}{\agospelbook{\foreignlanguage{greek}{ευαγγελιον κατα μαρκον}} \textbf{(\nospace{1:1})} } \\ \\ [-1em] % Si on veut ajouter les bordures latérales, remplacer {7}{c} par {7}{|c|}
\cline{4-4} \\
\cline{4-4}
&  &  & &  &  & \\ [-0.9em]
& \mygospelchapter &  & \foreignlanguage{greek}{αρχη του ευαγγελιου \textoverline{ιυ} \textoverline{χυ} υιου \textoverline{θυ} ωϲ γε} & 2 &  &  \\
&  & 2 & \foreignlanguage{greek}{γραπται εν τοιϲ προφηταιϲ ιδου εγω α} & 8 &  &  \\
&  & 8 & \foreignlanguage{greek}{ποϲτελλω τον αγγελον μου προ προϲω} & 13 &  &  \\
&  & 13 & \foreignlanguage{greek}{που ϲου οϲ καταϲκευαϲει την οδον ϲου} & 19 &  &  \\
& \textbf{3} &  & \foreignlanguage{greek}{φωνη βοωντοϲ εν τη ερημω ετοιμαϲα} & 6 &  &  \\
&  & 6 & \foreignlanguage{greek}{τε την οδον \textoverline{κυ} ευθειαϲ ποιειται ταϲ τρι} & 13 &  &  \\
&  & 13 & \foreignlanguage{greek}{βουϲ αυτου παϲα φαραγξ πληρωθηϲε} & 17 &  &  \\
&  & 17 & \foreignlanguage{greek}{ται και παν οροϲ και βουνοϲ ταπινωθη} & 23 &  &  \\
&  & 23 & \foreignlanguage{greek}{ϲεται και εϲται παντα τα ϲκολια ειϲ ευ} & 30 &  &  \\
&  & 30 & \foreignlanguage{greek}{θειαν και η τραχεια ειϲ πεδιον και οφθη} & 38 &  &  \\
&  & 38 & \foreignlanguage{greek}{ϲεται η δοξα \textoverline{κυ} και οψεται παϲα ϲαρξ το} & 46 &  &  \\
&  & 47 & \foreignlanguage{greek}{ϲωτηριον του \textoverline{θυ} οτι \textoverline{κϲ} ελαληϲεν φωνη} & 53 &  &  \\
&  & 54 & \foreignlanguage{greek}{λεγοντοϲ βοηϲον και ειπα τι βοηϲω οτι} & 60 &  &  \\
&  & 61 & \foreignlanguage{greek}{παϲα ϲαρξ χορτοϲ και παϲα η δοξα αυτηϲ} & 68 &  &  \\
&  & 69 & \foreignlanguage{greek}{ωϲ ανθοϲ χορτου εξηρανθη ο χορτοϲ και} & 75 &  &  \\
&  & 76 & \foreignlanguage{greek}{το ανθοϲ εξεπεϲεν το δε ρημα \textoverline{κυ} μενει} & 83 &  &  \\
&  & 84 & \foreignlanguage{greek}{ειϲ τον αιωνα και εγενετο ιωαννηϲ} & 3 & \textbf{4} &  \\
&  & 4 & \foreignlanguage{greek}{βαπτιζων εν τη ερημω και κηρυϲϲω̅} & 9 &  &  \\
&  & 10 & \foreignlanguage{greek}{βαπτιϲμα μετανοιαϲ ειϲ αφεϲιν αμαρ} & 14 &  &  \\
&  & 14 & \foreignlanguage{greek}{τιων και εξεπορευετο προϲ αυτον πα} & 5 & \textbf{5} &  \\
&  & 5 & \foreignlanguage{greek}{ϲα η ιουδαια χωρα και οι ιεροϲολυμειται} & 11 &  &  \\
&  & 12 & \foreignlanguage{greek}{και εβαπτιζοντο παντεϲ εν τω ιορδα} & 17 &  &  \\
&  & 17 & \foreignlanguage{greek}{νη υπ αυτου εξομολογουμενοι ταϲ αμαρ} & 22 &  &  \\
&  & 22 & \foreignlanguage{greek}{τιαϲ αυτων ην δε ιωαννηϲ ενδεδυ} & 4 & \textbf{6} &  \\
&  & 4 & \foreignlanguage{greek}{μενοϲ τριχαϲ καμηλου και ζωνην δερ} & 9 &  &  \\
&  & 9 & \foreignlanguage{greek}{ματινην περι την οϲφυν αυτου} & 13 &  &  \\
&  & 14 & \foreignlanguage{greek}{και ην αιϲθιων ακριδαϲ και μελι αγριον} & 20 &  &  \\
& \textbf{7} &  & \foreignlanguage{greek}{και εκηρυϲϲεν λεγων ερχεται ο ιϲχυρο} & 6 &  &  \\
&  & 6 & \foreignlanguage{greek}{τεροϲ μου οπιϲω μου ου ουκ ειμει ικα} & 13 &  &  \\
&  & 13 & \foreignlanguage{greek}{νοϲ κυψαϲ λυϲαι τον ιμαντα του υπο} & 19 &  &  \\
[0.2em]
\cline{4-4}
\end{tabular}
\end{center}
\end{table}
}
\clearpage
\newpage
 {
 \setlength\arrayrulewidth{1pt}
\begin{table}
\begin{center}
\begin{tabular}{ccc|l|ccc}
\cline{4-4} \\ [-1em]
\multicolumn{7}{c}{\foreignlanguage{greek}{ευαγγελιον κατα μαρκον} \textbf{(\nospace{1:7})} } \\ \\ [-1em] % Si on veut ajouter les bordures latérales, remplacer {7}{c} par {7}{|c|}
\cline{4-4} \\
\cline{4-4}
&  &  & &  &  & \\ [-0.9em]
&  & 19 & \foreignlanguage{greek}{δηματοϲ αυτου εγω μεν εβαπτιϲα} & 3 & \textbf{8} &  \\
&  & 4 & \foreignlanguage{greek}{υμαϲ εν υδατι αυτοϲ δε βαπτιϲη υμαϲ} & 10 &  &  \\
&  & 11 & \foreignlanguage{greek}{εν \textoverline{πνι} αγιω εγενετο δε εν εκειναιϲ} & 4 & \textbf{9} &  \\
&  & 5 & \foreignlanguage{greek}{ταιϲ ημεραιϲ και ηλθεν \textoverline{ιϲ} απο ναζαρεθ τηϲ} & 12 &  &  \\
&  & 13 & \foreignlanguage{greek}{γαλιλαιαϲ και εβαπτιϲθη υπο ιωαννου} & 17 &  &  \\
&  & 18 & \foreignlanguage{greek}{ειϲ τον ιορδανην και ευθεωϲ αναβαινω̅} & 3 & \textbf{10} &  \\
&  & 4 & \foreignlanguage{greek}{εκ του υδατοϲ ειδεν ϲχιζομενουϲ τουϲ} & 9 &  &  \\
&  & 10 & \foreignlanguage{greek}{ουρανουϲ και το \textoverline{πνα} καταβαινον απο} & 15 &  &  \\
&  & 16 & \foreignlanguage{greek}{του ουρανου ωϲει περιϲτεραν και μενον} & 21 &  &  \\
&  & 22 & \foreignlanguage{greek}{επ αυτον και φωνη εγενετο εκ του ου} & 6 & \textbf{11} &  \\
&  & 6 & \foreignlanguage{greek}{ρανου ϲυ ει ο υιοϲ μου ο αγαπητοϲ εν ω ηυ} & 16 &  &  \\
&  & 16 & \foreignlanguage{greek}{δοκηϲα και ευθυϲ το \textoverline{πνα} αυτον εκβαλ} & 6 & \textbf{12} &  \\
&  & 6 & \foreignlanguage{greek}{λει ειϲ την ερημον και ην εκει εν τη} & 5 & \textbf{13} &  \\
&  & 6 & \foreignlanguage{greek}{ερημω \textoverline{μ} ημεραϲ πειραζομενοϲ υπο του} & 11 &  &  \\
&  & 12 & \foreignlanguage{greek}{ϲατανα και ην μετα των θηριων και} & 18 &  &  \\
&  & 19 & \foreignlanguage{greek}{οι αγγελοι διηκονουν αυτω} & 22 &  &  \\
& \textbf{14} &  & \foreignlanguage{greek}{μετα δε το παραδοθηναι τον ιωαννην} & 6 &  &  \\
&  & 7 & \foreignlanguage{greek}{ηλθεν \textoverline{ιϲ} ειϲ την γαλιλαιαν κηρυϲϲων} & 12 &  &  \\
&  & 13 & \foreignlanguage{greek}{το ευαγγελιον τηϲ βαϲιλειαϲ του \textoverline{θυ} και} & 1 & \textbf{15} &  \\
&  & 2 & \foreignlanguage{greek}{λεγων οτι πεπληρωται ο καιροϲ και ηγ} & 8 &  &  \\
&  & 8 & \foreignlanguage{greek}{γεικεν η βαϲιλεια των ουρανων μετα} & 13 &  &  \\
&  & 13 & \foreignlanguage{greek}{νοειται και πιϲτευεται εν τω ευαγγε} & 18 &  &  \\
&  & 18 & \foreignlanguage{greek}{λιω περιπατων δε παρα την θα} & 5 & \textbf{16} &  \\
&  & 5 & \foreignlanguage{greek}{λαϲϲαν τηϲ γαλιλαιαϲ ιδεν ϲιμωνα και} & 10 &  &  \\
&  & 11 & \foreignlanguage{greek}{ανδρεαν τον αδελφον αυτου αμφι} & 15 &  &  \\
&  & 15 & \foreignlanguage{greek}{βαλλονταϲ αμφιβληϲτρον εν τη θα} & 19 &  &  \\
&  & 19 & \foreignlanguage{greek}{λαϲϲη ηϲαν γαρ αλιειϲ και ειπεν αυ} & 3 & \textbf{17} &  \\
&  & 3 & \foreignlanguage{greek}{τοιϲ \textoverline{ιϲ} δευτε οπιϲω μου και ποιηϲω υ} & 10 &  &  \\
&  & 10 & \foreignlanguage{greek}{μαϲ γενεϲθαι αλιειϲ ανθρωπων} & 13 &  &  \\
& \textbf{18} &  & \foreignlanguage{greek}{και ευθεωϲ αφεντεϲ τα δικτυα ηκο} & 6 &  &  \\
[0.2em]
\cline{4-4}
\end{tabular}
\end{center}
\end{table}
}
\clearpage
\newpage
 {
 \setlength\arrayrulewidth{1pt}
\begin{table}
\begin{center}
\begin{tabular}{ccc|l|ccc}
\cline{4-4} \\ [-1em]
\multicolumn{7}{c}{\foreignlanguage{greek}{ευαγγελιον κατα μαρκον} \textbf{(\nospace{1:18})} } \\ \\ [-1em] % Si on veut ajouter les bordures latérales, remplacer {7}{c} par {7}{|c|}
\cline{4-4} \\
\cline{4-4}
&  &  & &  &  & \\ [-0.9em]
&  & 6 & \foreignlanguage{greek}{λουθηϲαν αυτω και προβαϲ ολιγο̅} & 3 & \textbf{19} &  \\
&  & 4 & \foreignlanguage{greek}{ειδεν ιακωβον τον του ζεβεδαιου και ι} & 10 &  &  \\
&  & 10 & \foreignlanguage{greek}{ωαννην τον αδελφον αυτου και αυτουϲ} & 15 &  &  \\
&  & 16 & \foreignlanguage{greek}{εν τω πλοιω καταρτιζονταϲ τα δικτυα} & 21 &  &  \\
& \textbf{20} &  & \foreignlanguage{greek}{και εκαλεϲεν αυτουϲ και ευθεωϲ αφε̅} & 6 &  &  \\
&  & 6 & \foreignlanguage{greek}{τεϲ τον πατερα αυτων ζεβεδαιον με} & 11 &  &  \\
&  & 11 & \foreignlanguage{greek}{τα των μιϲθωτων εν τω πλοιω ηκολου} & 17 &  &  \\
&  & 17 & \foreignlanguage{greek}{θηϲαν αυτω και ειϲπορευονται ειϲ κα} & 4 & \textbf{21} &  \\
&  & 4 & \foreignlanguage{greek}{φαρναουμ και ευθεωϲ τοιϲ ϲαββαϲιν} & 8 &  &  \\
&  & 9 & \foreignlanguage{greek}{ειϲελθων ειϲ την ϲυναγωγην εδιδαϲκε̅} & 13 &  &  \\
& \textbf{22} &  & \foreignlanguage{greek}{και εξεπληϲϲοντο επι τη διδαχη αυτου} & 6 &  &  \\
&  & 7 & \foreignlanguage{greek}{ην γαρ διδαϲκων αυτουϲ ωϲ εξουϲια̅} & 12 &  &  \\
&  & 13 & \foreignlanguage{greek}{εχων και ουχ ωϲ οι γραμματειϲ} & 18 &  &  \\
& \textbf{23} &  & \foreignlanguage{greek}{και ην εν τη ϲυναγωγη αυτων \textoverline{ανοϲ}} & 7 &  &  \\
&  & 8 & \foreignlanguage{greek}{εν \textoverline{πνι} ακαθαρτω και ανεκραξεν λε} & 1 & \textbf{24} &  \\
&  & 1 & \foreignlanguage{greek}{γων τι ημιν και ϲυ \textoverline{ιυ} ναζαρηνε ηλ} & 8 &  &  \\
&  & 8 & \foreignlanguage{greek}{θεϲ ημαϲ απολεϲαι ωδε οιδα ϲε τιϲ ει} & 15 &  &  \\
&  & 16 & \foreignlanguage{greek}{ο αγιοϲ του \textoverline{θυ} και επετιμηϲεν αυτω} & 3 & \textbf{25} &  \\
&  & 4 & \foreignlanguage{greek}{και ειπεν φιμωθητι και εξελθε εκ} & 9 &  &  \\
&  & 10 & \foreignlanguage{greek}{του ανθρωπου \textoverline{πνα} ακαθαρτον και} & 1 & \textbf{26} &  \\
&  & 2 & \foreignlanguage{greek}{εξηλθεν το \textoverline{πνα} ϲπαραξαν αυτον} & 6 &  &  \\
&  & 7 & \foreignlanguage{greek}{και ανεκραγεν φωνη μεγαλη και α} & 12 &  &  \\
&  & 12 & \foreignlanguage{greek}{πηλθεν απ αυτου και εθαυμαζον} & 2 & \textbf{27} &  \\
&  & 3 & \foreignlanguage{greek}{παντεϲ και ϲυνεζητουν προϲ εαυ} & 7 &  &  \\
&  & 7 & \foreignlanguage{greek}{τουϲ λεγοντεϲ τιϲ η διδαχη η κενη} & 13 &  &  \\
&  & 14 & \foreignlanguage{greek}{αυτη η εξουϲιαϲτικη αυτου και οτι} & 19 &  &  \\
&  & 20 & \foreignlanguage{greek}{τοιϲ πνευμαϲιν τοιϲ ακαθαρτοιϲ επι} & 24 &  &  \\
&  & 24 & \foreignlanguage{greek}{ταϲϲει και υπακουουϲιν αυτω} & 27 &  &  \\
& \textbf{28} &  & \foreignlanguage{greek}{και εξηλθεν η ακοη αυτου πανταχου} & 6 &  &  \\
&  & 7 & \foreignlanguage{greek}{ειϲ ολην την περιχωρον τηϲ γαλιλαιαϲ} & 12 &  &  \\
[0.2em]
\cline{4-4}
\end{tabular}
\end{center}
\end{table}
}
\clearpage
\newpage
 {
 \setlength\arrayrulewidth{1pt}
\begin{table}
\begin{center}
\begin{tabular}{ccc|l|ccc}
\cline{4-4} \\ [-1em]
\multicolumn{7}{c}{\foreignlanguage{greek}{ευαγγελιον κατα μαρκον} \textbf{(\nospace{1:29})} } \\ \\ [-1em] % Si on veut ajouter les bordures latérales, remplacer {7}{c} par {7}{|c|}
\cline{4-4} \\
\cline{4-4}
&  &  & &  &  & \\ [-0.9em]
& \textbf{29} &  & \foreignlanguage{greek}{εξελθων δε εκ τηϲ ϲυναγωγηϲ ηλθεν} & 6 &  &  \\
&  & 7 & \foreignlanguage{greek}{ειϲ την οικειαν ϲιμωνοϲ και ανδρεου} & 12 &  &  \\
&  & 13 & \foreignlanguage{greek}{μετα ιακωβου και ιακωβου και ιωαν} & 18 &  &  \\
&  & 18 & \foreignlanguage{greek}{νου κατεκειτο δε η πενθερα ϲιμωνοϲ} & 5 & \textbf{30} &  \\
&  & 6 & \foreignlanguage{greek}{πυρεϲϲουϲα και λεγουϲιν αυτω περι αυ} & 11 &  &  \\
&  & 11 & \foreignlanguage{greek}{τηϲ και προϲελθων εκτιναϲ την χειρα} & 5 & \textbf{31} &  \\
&  & 6 & \foreignlanguage{greek}{και επιλαβομενοϲ ηγειρεν αυτην} & 9 &  &  \\
&  & 10 & \foreignlanguage{greek}{και αφηκεν αυτην ο πυρετοϲ και διη} & 16 &  &  \\
&  & 16 & \foreignlanguage{greek}{κονι αυτω οψιαϲ δε γενομενηϲ οτε} & 4 & \textbf{32} &  \\
&  & 5 & \foreignlanguage{greek}{εδυ ο ηλιοϲ εφερον προϲ αυτον πανταϲ} & 11 &  &  \\
&  & 12 & \foreignlanguage{greek}{τουϲ κακωϲ εχονταϲ και η πολειϲ ολη} & 1 & \textbf{33} &  \\
&  & 2 & \foreignlanguage{greek}{ϲυνηγμενη ην προϲ ταϲ θυραϲ και εθε} & 2 & \textbf{34} &  \\
&  & 2 & \foreignlanguage{greek}{ραπευϲεν πολλουϲ κακωϲ εχονταϲ ποι} & 6 &  &  \\
&  & 6 & \foreignlanguage{greek}{κειλαιϲ νοϲοιϲ και δαιμονια πολλα ε} & 11 &  &  \\
&  & 11 & \foreignlanguage{greek}{ξεβαλεν απ αυτων και ουκ ηφιεν λα} & 17 &  &  \\
&  & 17 & \foreignlanguage{greek}{λιν τα δαιμονια οτι ηδιϲαν αυτον} & 22 &  &  \\
&  & 23 & \foreignlanguage{greek}{\textoverline{χν} ειναι και εννυχα αναϲταϲ απηλ} & 4 & \textbf{35} &  \\
&  & 4 & \foreignlanguage{greek}{θεν ειϲ ερημον τοπον και εκει προϲηυ} & 10 &  &  \\
&  & 10 & \foreignlanguage{greek}{χετο και κατεδιωξαν αυτον ϲιμων} & 4 & \textbf{36} &  \\
&  & 5 & \foreignlanguage{greek}{και οι μετ αυτου λεγοντεϲ αυτω ζητου} & 3 & \textbf{37} &  \\
&  & 3 & \foreignlanguage{greek}{ϲιν ϲε παντεϲ και λεγει αυτοιϲ αγω} & 4 & \textbf{38} &  \\
&  & 4 & \foreignlanguage{greek}{μεν ειϲ ταϲ εχομεναϲ κωμοπολειϲ κη} & 9 &  &  \\
&  & 9 & \foreignlanguage{greek}{ρυϲϲιν ειϲ τουτο γαρ εληλυθα} & 13 &  &  \\
& \textbf{39} &  & \foreignlanguage{greek}{και ην κηρυϲϲων ειϲ ταϲ ϲυναγωγαϲ} & 6 &  &  \\
&  & 7 & \foreignlanguage{greek}{αυτων ειϲ ολην την γαλιλαιαν και} & 1 & \textbf{40} &  \\
&  & 2 & \foreignlanguage{greek}{ερχεται προϲ αυτον λεπροϲ παρακα} & 6 &  &  \\
&  & 6 & \foreignlanguage{greek}{λων αυτον και λεγων \textoverline{κε} εαν θεληϲ} & 12 &  &  \\
&  & 13 & \foreignlanguage{greek}{δυναϲαι με καθαριϲαι ο δε \textoverline{ιϲ} ϲπλαγ} & 4 & \textbf{41} &  \\
&  & 4 & \foreignlanguage{greek}{χνιϲθειϲ εκτιναϲ την χειρα ηψατο αυ} & 9 &  &  \\
&  & 9 & \foreignlanguage{greek}{του λεγων θελω καθαριϲθητει και} & 1 & \textbf{42} &  \\
[0.2em]
\cline{4-4}
\end{tabular}
\end{center}
\end{table}
}
\clearpage
\newpage
 {
 \setlength\arrayrulewidth{1pt}
\begin{table}
\begin{center}
\begin{tabular}{ccc|l|ccc}
\cline{4-4} \\ [-1em]
\multicolumn{7}{c}{\foreignlanguage{greek}{ευαγγελιον κατα μαρκον} \textbf{(\nospace{1:42})} } \\ \\ [-1em] % Si on veut ajouter les bordures latérales, remplacer {7}{c} par {7}{|c|}
\cline{4-4} \\
\cline{4-4}
&  &  & &  &  & \\ [-0.9em]
&  & 2 & \foreignlanguage{greek}{ευθεωϲ απηλθεν απ αυτου η λεπρα} & 7 &  &  \\
& \textbf{44} &  & \foreignlanguage{greek}{και λεγει αυτω ορα μηδενει ειπηϲ αλ} & 7 &  &  \\
&  & 7 & \foreignlanguage{greek}{λα υπαγε δειξον ϲεαυτον τω ιερει και} & 13 &  &  \\
&  & 14 & \foreignlanguage{greek}{προϲενεγκε περι του καθαρϲιου ϲου} & 18 &  &  \\
&  & 19 & \foreignlanguage{greek}{ο προϲεταξεν μωυϲηϲ ειϲ μαρτυριον} & 23 &  &  \\
&  & 24 & \foreignlanguage{greek}{αυτοιϲ ο δε εξελθων ηρξατο κηρυϲ} & 5 & \textbf{45} &  \\
&  & 5 & \foreignlanguage{greek}{ϲιν και διαφημιζειν τον λογον ωϲτε} & 10 &  &  \\
&  & 11 & \foreignlanguage{greek}{μηκετι δυναϲθαι φανερωϲ ειϲ πολιν} & 15 &  &  \\
&  & 16 & \foreignlanguage{greek}{ειϲελθειν αλλ εξω επ ερημοιϲ τοποιϲ η̅} & 22 &  &  \\
&  & 23 & \foreignlanguage{greek}{και ηρχοντο προϲ αυτον παντοθεν} & 27 &  &  \\
& \mygospelchapter &  & \foreignlanguage{greek}{και παλιν ερχεται ειϲ καφαρναουμ και} & 6 &  &  \\
&  & 7 & \foreignlanguage{greek}{ηκουϲθη οτι εν οικω εϲτιν και ϲυνη} & 2 & \textbf{2} &  \\
&  & 2 & \foreignlanguage{greek}{χθηϲαν πολλοι ωϲτε μηκετι χωριν} & 6 &  &  \\
&  & 7 & \foreignlanguage{greek}{και ελαλει προϲ αυτουϲ τον λογον} & 12 &  &  \\
& \textbf{3} &  & \foreignlanguage{greek}{και ιδου ανδρεϲ ερχονται προϲ αυτον} & 6 &  &  \\
&  & 7 & \foreignlanguage{greek}{βαϲταζοντεϲ εν κρεβαττω παραλυ} & 10 &  &  \\
&  & 10 & \foreignlanguage{greek}{τικον και μη δυναμενοι προϲελθειν} & 4 & \textbf{4} &  \\
&  & 5 & \foreignlanguage{greek}{αυτω απο του οχλου απεϲτεγαϲαν την} & 10 &  &  \\
&  & 11 & \foreignlanguage{greek}{ϲτεγην οπου ην και χαλωϲιν τον κρα} & 17 &  &  \\
&  & 17 & \foreignlanguage{greek}{βαττον ειϲ ον ο παραλυτικοϲ κατεκειτο} & 22 &  &  \\
& \textbf{5} &  & \foreignlanguage{greek}{ιδων δε ο \textoverline{ιϲ} την πιϲτιν αυτων λεγει τω} & 9 &  &  \\
&  & 10 & \foreignlanguage{greek}{παραλυτικω τεκνον αφεωνται ϲου αι} & 14 &  &  \\
&  & 15 & \foreignlanguage{greek}{αμαρτιαι ηϲαν δε τινεϲ των γραμμα} & 5 & \textbf{6} &  \\
&  & 5 & \foreignlanguage{greek}{τεων εκει καθημενοι και διαλογιζομε} & 9 &  &  \\
&  & 9 & \foreignlanguage{greek}{νοι εν ταιϲ καρδιαιϲ αυτων λεγοντεϲ} & 14 &  &  \\
& \textbf{7} &  & \foreignlanguage{greek}{τι ουτοϲ ουτωϲ λαλει βλαϲφημιαϲ τιϲ} & 6 &  &  \\
&  & 7 & \foreignlanguage{greek}{δυναται αφειναι αμαρτιαϲ ει μη ειϲ ο \textoverline{θϲ}} & 14 &  &  \\
& \textbf{8} &  & \foreignlanguage{greek}{και επιγνουϲ ο \textoverline{ιϲ} τω \textoverline{πνι} οτι διαλογιζον} & 8 &  &  \\
&  & 8 & \foreignlanguage{greek}{ται λεγει αυτοιϲ τι διαλογειζεϲθαι εν} & 13 &  &  \\
&  & 14 & \foreignlanguage{greek}{ταιϲ καρδιαιϲ υμων τι γαρ εϲτιν ευκο} & 4 & \textbf{9} &  \\
[0.2em]
\cline{4-4}
\end{tabular}
\end{center}
\end{table}
}
\clearpage
\newpage
 {
 \setlength\arrayrulewidth{1pt}
\begin{table}
\begin{center}
\begin{tabular}{ccc|l|ccc}
\cline{4-4} \\ [-1em]
\multicolumn{7}{c}{\foreignlanguage{greek}{ευαγγελιον κατα μαρκον} \textbf{(\nospace{2:9})} } \\ \\ [-1em] % Si on veut ajouter les bordures latérales, remplacer {7}{c} par {7}{|c|}
\cline{4-4} \\
\cline{4-4}
&  &  & &  &  & \\ [-0.9em]
&  & 4 & \foreignlanguage{greek}{πωτερον ειπειν αφεωνται ϲου αι αμαρ} & 9 &  &  \\
&  & 9 & \foreignlanguage{greek}{τιαι η ειπειν εγειρε και περιπατει} & 14 &  &  \\
& \textbf{10} &  & \foreignlanguage{greek}{ινα δε ειδηται οτι εξουϲιαν εχει ο υιοϲ} & 8 &  &  \\
&  & 9 & \foreignlanguage{greek}{του ανθρωπου αφειεναι αμαρτιαϲ λε} & 13 &  &  \\
&  & 13 & \foreignlanguage{greek}{γει τω παραλυτικω εγειρε και αρον το̅} & 4 & \textbf{11} &  \\
&  & 5 & \foreignlanguage{greek}{κραβαττον ϲου και υπαγε ειϲ τον οικο̅} & 11 &  &  \\
&  & 12 & \foreignlanguage{greek}{ϲου ο δε εγερθειϲ και αραϲ αυτου τον} & 7 & \textbf{12} &  \\
&  & 8 & \foreignlanguage{greek}{κραβαττον εμπροϲθεν παντων απηλ} & 11 &  &  \\
&  & 11 & \foreignlanguage{greek}{θεν ωϲτε θαυμαζειν αυτουϲ και δο} & 16 &  &  \\
&  & 16 & \foreignlanguage{greek}{ξαζειν τον \textoverline{θν} οτι ουτωϲ ουδεποτε} & 21 &  &  \\
&  & 22 & \foreignlanguage{greek}{ειδον και εξηλθεν παλιν πα} & 4 & \textbf{13} &  \\
&  & 4 & \foreignlanguage{greek}{ρα την θαλαϲϲαν και παϲ ο οχλοϲ ηρχε} & 11 &  &  \\
&  & 11 & \foreignlanguage{greek}{το προϲ αυτον και εδιδαϲκεν αυτουϲ} & 16 &  &  \\
& \textbf{14} &  & \foreignlanguage{greek}{και παραγων ειδεν λευειν τον του αλ} & 7 &  &  \\
&  & 7 & \foreignlanguage{greek}{φεου καθημενον επι του τελωνιου} & 11 &  &  \\
&  & 12 & \foreignlanguage{greek}{και λεγει αυτω ακολουθει μοι και α} & 18 &  &  \\
&  & 18 & \foreignlanguage{greek}{ναϲταϲ ηκολουθει αυτω και γει} & 2 & \textbf{15} &  \\
&  & 2 & \foreignlanguage{greek}{νεται ανακειμενων αυτων εν τη οι} & 7 &  &  \\
&  & 7 & \foreignlanguage{greek}{κεια πολλοι τελωναι και αμαρτωλοι} & 11 &  &  \\
&  & 12 & \foreignlanguage{greek}{ϲυνανεκιντο τω \textoverline{ιυ} και τοιϲ μαθη} & 17 &  &  \\
&  & 17 & \foreignlanguage{greek}{ταιϲ αυτου ηϲαν γαρ πολλοι και ηκο} & 23 &  &  \\
&  & 23 & \foreignlanguage{greek}{λουθηϲαν αυτω και οι γραμματειϲ} & 3 & \textbf{16} &  \\
&  & 4 & \foreignlanguage{greek}{των φαριϲαιων ελεγον τοιϲ μαθηταιϲ} & 8 &  &  \\
&  & 9 & \foreignlanguage{greek}{αυτου δια τι μετα των τελωνων και} & 15 &  &  \\
&  & 16 & \foreignlanguage{greek}{αμαρτωλων εϲθιει και ακουϲαϲ} & 2 & \textbf{17} &  \\
&  & 3 & \foreignlanguage{greek}{ο \textoverline{ιϲ} λεγει ου χρειαν εχουϲιν οι ιϲχυο̅} & 10 &  &  \\
&  & 10 & \foreignlanguage{greek}{τεϲ ιατρου αλλα οι κακωϲ εχοντεϲ} & 15 &  &  \\
&  & 16 & \foreignlanguage{greek}{ουκ εληλυθα καλεϲαι δικαιουϲ αλλα} & 20 &  &  \\
&  & 21 & \foreignlanguage{greek}{αμαρτωλουϲ} & 21 &  &  \\
& \textbf{18} &  & \foreignlanguage{greek}{και ηϲαν οι μαθηται ιωαννου και οι} & 7 &  &  \\
[0.2em]
\cline{4-4}
\end{tabular}
\end{center}
\end{table}
}
\clearpage
\newpage
 {
 \setlength\arrayrulewidth{1pt}
\begin{table}
\begin{center}
\begin{tabular}{ccc|l|ccc}
\cline{4-4} \\ [-1em]
\multicolumn{7}{c}{\foreignlanguage{greek}{ευαγγελιον κατα μαρκον} \textbf{(\nospace{2:18})} } \\ \\ [-1em] % Si on veut ajouter les bordures latérales, remplacer {7}{c} par {7}{|c|}
\cline{4-4} \\
\cline{4-4}
&  &  & &  &  & \\ [-0.9em]
&  & 8 & \foreignlanguage{greek}{μαθηται των φαριϲαιων νηϲτευοντεϲ} & 11 &  &  \\
&  & 12 & \foreignlanguage{greek}{και ερχονται και λεγουϲιν αυτω δια τι οι} & 19 &  &  \\
&  & 20 & \foreignlanguage{greek}{μαθηται ιωαννου και των φαριϲαιων} & 24 &  &  \\
&  & 25 & \foreignlanguage{greek}{νηϲτευουϲιν οι δε ϲοι μαθηται ου νη} & 31 &  &  \\
&  & 31 & \foreignlanguage{greek}{ϲτευουϲιν και ειπεν αυτοιϲ μη δυ} & 5 & \textbf{19} &  \\
&  & 5 & \foreignlanguage{greek}{νανται οι νυμφιοι του νυμφωνοϲ εν ω} & 11 &  &  \\
&  & 12 & \foreignlanguage{greek}{ο νυμφιοϲ μετ αυτων εϲτιν νηϲτευει̅} & 17 &  &  \\
& \textbf{20} &  & \foreignlanguage{greek}{ελευϲονται δε ημεραι οταν απαρθη} & 5 &  &  \\
&  & 6 & \foreignlanguage{greek}{απ αυτων ο νυμφιοϲ και τοτε νηϲτευ} & 12 &  &  \\
&  & 12 & \foreignlanguage{greek}{ϲουϲιν εν εκεινη τη ημερα ουδειϲ επι} & 2 & \textbf{21} &  \\
&  & 2 & \foreignlanguage{greek}{βλημα ρακουϲ αγναφου επιϲυναπτι} & 5 &  &  \\
&  & 6 & \foreignlanguage{greek}{ιματιω παλαιω ει δε μη ερει απ αυτου} & 13 &  &  \\
&  & 14 & \foreignlanguage{greek}{το πληρωμα το καινον του παλαιου} & 19 &  &  \\
&  & 20 & \foreignlanguage{greek}{και πλειω ϲχιϲμα γεινεται και ουδειϲ} & 2 & \textbf{22} &  \\
&  & 3 & \foreignlanguage{greek}{βαλλει οινον νεον ειϲ αϲκουϲ παλαιουϲ} & 8 &  &  \\
&  & 9 & \foreignlanguage{greek}{αλλ ειϲ καινουϲ ει δε μη διαρρηϲϲον} & 15 &  &  \\
&  & 15 & \foreignlanguage{greek}{ται οι αϲκοι και ο οινοϲ εκχειται και} & 22 &  &  \\
&  & 23 & \foreignlanguage{greek}{οι αϲκοι απολλυνται αλλα οινον νε} & 28 &  &  \\
&  & 28 & \foreignlanguage{greek}{ον ειϲ αϲκουϲ καινουϲ βαλλουϲιν} & 32 &  &  \\
& \textbf{23} &  & \foreignlanguage{greek}{και εγενετο αυτον εν τοιϲ ϲαββαϲιν} & 6 &  &  \\
&  & 7 & \foreignlanguage{greek}{πορευεϲθαι δια των εϲπαρμενων} & 10 &  &  \\
&  & 11 & \foreignlanguage{greek}{και οι μαθηται αυτου ηρξαντο τιλλειν} & 16 &  &  \\
&  & 17 & \foreignlanguage{greek}{τουϲ ϲταχυαϲ οι δε φαριϲαιοι ελεγον} & 4 & \textbf{24} &  \\
&  & 5 & \foreignlanguage{greek}{αυτω ειδε τι ποιουϲιν τοιϲ ϲαββαϲιν} & 10 &  &  \\
&  & 11 & \foreignlanguage{greek}{ο ουκ εξεϲτιν και λεγει αυτοιϲ} & 3 & \textbf{25} &  \\
&  & 4 & \foreignlanguage{greek}{ουδε τουτο ανεγνωτε τι εποιηϲεν δαυ} & 10 &  &  \\
&  & 10 & \foreignlanguage{greek}{ειδ οτε χρειαν εϲχεν και επιναϲεν αυ} & 16 &  &  \\
&  & 16 & \foreignlanguage{greek}{τοϲ και οι μετ αυτου πωϲ ειϲελθων ειϲ} & 3 & \textbf{26} &  \\
&  & 4 & \foreignlanguage{greek}{τον οικον του \textoverline{θυ} εφαγεν τουϲ αρτουϲ} & 10 &  &  \\
&  & 11 & \foreignlanguage{greek}{τηϲ προθεϲεωϲ και εδωκεν και τοιϲ} & 16 &  &  \\
[0.2em]
\cline{4-4}
\end{tabular}
\end{center}
\end{table}
}
\clearpage
\newpage
 {
 \setlength\arrayrulewidth{1pt}
\begin{table}
\begin{center}
\begin{tabular}{ccc|l|ccc}
\cline{4-4} \\ [-1em]
\multicolumn{7}{c}{\foreignlanguage{greek}{ευαγγελιον κατα μαρκον} \textbf{(\nospace{2:26})} } \\ \\ [-1em] % Si on veut ajouter les bordures latérales, remplacer {7}{c} par {7}{|c|}
\cline{4-4} \\
\cline{4-4}
&  &  & &  &  & \\ [-0.9em]
&  & 17 & \foreignlanguage{greek}{μετ αυτου ουϲ ουκ εξεϲτιν φαγειν ει μη} & 24 &  &  \\
&  & 25 & \foreignlanguage{greek}{τοιϲ ιερευϲιν λεγω δε αυτοιϲ οτι το ϲαβ} & 6 & \textbf{27} &  \\
&  & 6 & \foreignlanguage{greek}{βατον δια τον ανθρωπον εκτιϲθη ωϲτε} & 1 & \textbf{28} &  \\
&  & 2 & \foreignlanguage{greek}{\textoverline{κϲ} εϲτιν ο υιοϲ του ανθρωπου και του ϲαβ} & 10 &  &  \\
&  & 10 & \foreignlanguage{greek}{βατου και ειϲελθοντοϲ αυτου ειϲ τη̅} & 5 & \mygospelchapter &  \\
&  & 6 & \foreignlanguage{greek}{ϲυναγωγην ερχεται ανθρωποϲ προϲ αυ} & 10 &  &  \\
&  & 10 & \foreignlanguage{greek}{τον εχων ξηραν την χειρα και παρετη} & 2 & \textbf{2} &  \\
&  & 2 & \foreignlanguage{greek}{ρουντο αυτον ει τοιϲ ϲαββαϲιν θεραπευ} & 7 &  &  \\
&  & 7 & \foreignlanguage{greek}{ει ινα κατηγορηϲωϲιν αυτου και λε} & 2 & \textbf{3} &  \\
&  & 2 & \foreignlanguage{greek}{γει τω ανθρωπω τω εχοντι την χειρα} & 8 &  &  \\
&  & 9 & \foreignlanguage{greek}{ξηραν εγειρε εκ του μεϲου και λεγει} & 2 & \textbf{4} &  \\
&  & 3 & \foreignlanguage{greek}{αυτοιϲ εξεϲτιν τοιϲ ϲαββαϲιν αγαθον} & 7 &  &  \\
&  & 8 & \foreignlanguage{greek}{ποιηϲαι η ου ψυχην ϲωϲαι η απολεϲαι} & 14 &  &  \\
&  & 15 & \foreignlanguage{greek}{οι δε εϲιωπων περιβλεψαμενοϲ δε} & 2 & \textbf{5} &  \\
&  & 3 & \foreignlanguage{greek}{αυτουϲ μετ οργηϲ επι τη πωρωϲει τηϲ} & 9 &  &  \\
&  & 10 & \foreignlanguage{greek}{καρδιαϲ αυτων λεγει τω ανθρωπω} & 14 &  &  \\
&  & 15 & \foreignlanguage{greek}{εκτεινον την χειρα ϲου και εξετινεν} & 20 &  &  \\
&  & 21 & \foreignlanguage{greek}{και απεκατεϲταθη η χειρ αυτου} & 25 &  &  \\
& \textbf{6} &  & \foreignlanguage{greek}{εξελθοντεϲ δε οι φαριϲαιοι μετα των} & 6 &  &  \\
&  & 7 & \foreignlanguage{greek}{ηρωδιανων ϲυνβουλιον εποιουντο} & 9 &  &  \\
&  & 10 & \foreignlanguage{greek}{κατ αυτου οπωϲ αυτον απολεϲωϲιν} & 14 &  &  \\
& \textbf{7} &  & \foreignlanguage{greek}{ο δε \textoverline{ιϲ} ανεχωρηϲεν μετα των μαθητω̅} & 7 &  &  \\
&  & 8 & \foreignlanguage{greek}{αυτου προϲ την θαλαϲϲαν και πολυ} & 13 &  &  \\
&  & 14 & \foreignlanguage{greek}{πληθοϲ απο τηϲ γαλιλαιαϲ και τηϲ ιου} & 20 &  &  \\
&  & 20 & \foreignlanguage{greek}{δαιαϲ και απο ιεροϲολυμων και περαν} & 5 & \textbf{8} &  \\
&  & 6 & \foreignlanguage{greek}{του ιορδανου και περι τυρον και ϲιδο} & 12 &  &  \\
&  & 12 & \foreignlanguage{greek}{να ηκολουθουν αυτω ακουοντεϲ οϲα} & 16 &  &  \\
&  & 17 & \foreignlanguage{greek}{εποιει και ειπεν τοιϲ μαθηταιϲ αυτου} & 5 & \textbf{9} &  \\
&  & 6 & \foreignlanguage{greek}{ινα πλοιαριον προϲκαρτερη αυτω δια} & 10 &  &  \\
&  & 11 & \foreignlanguage{greek}{τον οχλον ινα μη θλιβωϲιν αυτον} & 16 &  &  \\
[0.2em]
\cline{4-4}
\end{tabular}
\end{center}
\end{table}
}
\clearpage
\newpage
 {
 \setlength\arrayrulewidth{1pt}
\begin{table}
\begin{center}
\begin{tabular}{ccc|l|ccc}
\cline{4-4} \\ [-1em]
\multicolumn{7}{c}{\foreignlanguage{greek}{ευαγγελιον κατα μαρκον} \textbf{(\nospace{3:10})} } \\ \\ [-1em] % Si on veut ajouter les bordures latérales, remplacer {7}{c} par {7}{|c|}
\cline{4-4} \\
\cline{4-4}
&  &  & &  &  & \\ [-0.9em]
& \textbf{10} &  & \foreignlanguage{greek}{πολλουϲ γαρ εθεραπευεν ωϲτε επεπιπτο̅} & 5 &  &  \\
&  & 6 & \foreignlanguage{greek}{αυτω αυτου ινα αυτου αψωνται οϲοι ειχον μα} & 13 &  &  \\
&  & 13 & \foreignlanguage{greek}{ϲτιγαϲ τα πνευματα δε τα ακαθαρτα} & 5 & \textbf{11} &  \\
&  & 6 & \foreignlanguage{greek}{οταν αυτον ιδον προϲεπιπτον αυτω} & 10 &  &  \\
&  & 11 & \foreignlanguage{greek}{και εκραζον λεγοντεϲ ϲυ ει ο υιοϲ του \textoverline{θυ}} & 19 &  &  \\
& \textbf{12} &  & \foreignlanguage{greek}{και επετιμα αυτοιϲ ινα μη αυτον φα} & 7 &  &  \\
&  & 7 & \foreignlanguage{greek}{νερον ποιωϲιν και αναβαϲ ειϲ το οροϲ} & 5 & \textbf{13} &  \\
&  & 6 & \foreignlanguage{greek}{προϲεκαλεϲατο ουϲ ηθελεν και απηλ} & 10 &  &  \\
&  & 10 & \foreignlanguage{greek}{θον προϲ αυτον και εποιηϲεν \textoverline{ιβ} μαθηταϲ} & 4 & \textbf{14} &  \\
&  & 5 & \foreignlanguage{greek}{ινα ωϲιν μετ αυτου ουϲ και αποϲτολουϲ} & 11 &  &  \\
&  & 12 & \foreignlanguage{greek}{ωνομαϲεν ινα αποϲτιλη αυτουϲ κη} & 16 &  &  \\
&  & 16 & \foreignlanguage{greek}{ρυϲϲιν το ευαγγελιον και εδωκεν αυ} & 3 & \textbf{15} &  \\
&  & 3 & \foreignlanguage{greek}{τοιϲ εξουϲιαν θεραπευειν ταϲ νοϲουϲ} & 7 &  &  \\
&  & 8 & \foreignlanguage{greek}{και εκβαλλιν τα δαιμονια και περια} & 13 &  &  \\
&  & 13 & \foreignlanguage{greek}{γονταϲ κηρυϲϲιν το ευαγγελιον} & 16 &  &  \\
& \textbf{16} &  & \foreignlanguage{greek}{και επεθηκεν ονομα ϲιμωνι πετρον} & 5 &  &  \\
& \textbf{17} &  & \foreignlanguage{greek}{κοινωϲ δε αυτουϲ εκαλεϲεν βοανανηρ} & 5 &  &  \\
&  & 5 & \foreignlanguage{greek}{γε ο εϲτιν υιοι βροντηϲ ηϲαν δε ουτοι} & 3 & \textbf{18} &  \\
&  & 4 & \foreignlanguage{greek}{ϲιμων και ανδρεαϲ ιακωβοϲ και ιωα̅} & 9 &  &  \\
&  & 9 & \foreignlanguage{greek}{νηϲ φιλιπποϲ και μαρθολομεοϲ και} & 13 &  &  \\
&  & 14 & \foreignlanguage{greek}{μαθθεοϲ και θωμαϲ και ιακωβοϲ ο του} & 20 &  &  \\
&  & 21 & \foreignlanguage{greek}{αλφαιου και ϲιμων ο κανανεοϲ και ι} & 2 & \textbf{19} &  \\
&  & 2 & \foreignlanguage{greek}{ουδαϲ ιϲκαριωτηϲ ο και παραδουϲ αυτο} & 7 &  &  \\
& \textbf{20} &  & \foreignlanguage{greek}{και ερχεται ειϲ οικον και ϲυνερχεται πα} & 7 &  &  \\
&  & 7 & \foreignlanguage{greek}{λιν οχλοϲ ωϲτε μη δυναϲθαι αυτουϲ μη} & 13 &  &  \\
&  & 13 & \foreignlanguage{greek}{δε αρτον φαγειν και ακουϲαντεϲ} & 2 & \textbf{21} &  \\
&  & 3 & \foreignlanguage{greek}{περι αυτου οι γραμματειϲ και οι λοιποι} & 9 &  &  \\
&  & 10 & \foreignlanguage{greek}{εξηλθον κρατηϲαι αυτον ελεγαν γαρ} & 14 &  &  \\
&  & 15 & \foreignlanguage{greek}{οτι εξηρτηνται αυτου και οι απο ιεροϲο} & 4 & \textbf{22} &  \\
&  & 4 & \foreignlanguage{greek}{λυμων καταβαντεϲ γραμματιϲ} & 6 &  &  \\
[0.2em]
\cline{4-4}
\end{tabular}
\end{center}
\end{table}
}
\clearpage
\newpage
 {
 \setlength\arrayrulewidth{1pt}
\begin{table}
\begin{center}
\begin{tabular}{ccc|l|ccc}
\cline{4-4} \\ [-1em]
\multicolumn{7}{c}{\foreignlanguage{greek}{ευαγγελιον κατα μαρκον} \textbf{(\nospace{3:22})} } \\ \\ [-1em] % Si on veut ajouter les bordures latérales, remplacer {7}{c} par {7}{|c|}
\cline{4-4} \\
\cline{4-4}
&  &  & &  &  & \\ [-0.9em]
&  & 7 & \foreignlanguage{greek}{ελεγον οτι βεελζεβουλ εχει τον αρχον} & 12 &  &  \\
&  & 12 & \foreignlanguage{greek}{τα των δαιμονιων και δι αυτου εκβαλ} & 18 &  &  \\
&  & 18 & \foreignlanguage{greek}{λει τα δαιμονια και προϲκαλεϲαμε} & 2 & \textbf{23} &  \\
&  & 2 & \foreignlanguage{greek}{νοϲ αυτουϲ ειπεν αυτοιϲ εν παραβο} & 7 &  &  \\
&  & 7 & \foreignlanguage{greek}{λαιϲ πωϲ δυναται ϲαταναϲ ϲαταναν} & 11 &  &  \\
&  & 12 & \foreignlanguage{greek}{εκβαλλιν και εαν βαϲιλεια εφ εαυτην} & 5 & \textbf{24} &  \\
&  & 6 & \foreignlanguage{greek}{μεριϲθη ου δυναται ϲταθηναι η βαϲιλει} & 11 &  &  \\
&  & 11 & \foreignlanguage{greek}{α εκεινη καν οικεια εφ εαυτην μερι} & 5 & \textbf{25} &  \\
&  & 5 & \foreignlanguage{greek}{ϲθη ου δυναται ϲταθηναι και εαν ο ϲα} & 4 & \textbf{26} &  \\
&  & 4 & \foreignlanguage{greek}{ταναϲ εφ εαυτον εμεριϲθη ου δυναται} & 9 &  &  \\
&  & 10 & \foreignlanguage{greek}{ϲταθηναι η βαϲιλεια αυτου αλλα τελοϲ} & 15 &  &  \\
&  & 16 & \foreignlanguage{greek}{εχει ουδειϲ δυναται τα ϲκευη του} & 5 & \textbf{27} &  \\
&  & 6 & \foreignlanguage{greek}{ιϲχυρου διαρπαϲαι ειϲελθων ειϲ την οι} & 11 &  &  \\
&  & 11 & \foreignlanguage{greek}{κειαν εαν μη πρωτον τον ιϲχυρον δηϲη} & 17 &  &  \\
&  & 18 & \foreignlanguage{greek}{και τοτε τα ϲκευη αυτου διαρπαϲη} & 23 &  &  \\
& \textbf{28} &  & \foreignlanguage{greek}{αμην λεγω υμιν οτι παντα τα αμαρτη} & 7 &  &  \\
&  & 7 & \foreignlanguage{greek}{ματα αφεθηϲεται τοιϲ υιοιϲ των \textoverline{ανων}} & 12 &  &  \\
&  & 13 & \foreignlanguage{greek}{και αι βλαϲφημιαι οϲ δ αν βλαϲφημη} & 4 & \textbf{29} &  \\
&  & 4 & \foreignlanguage{greek}{ϲη το \textoverline{πνα} το αγιον ουκ εχει αφεϲιν αλλα} & 12 &  &  \\
&  & 13 & \foreignlanguage{greek}{ενοχοϲ εϲτιν αιωνιου αμαρτιαϲ οτι ε} & 2 & \textbf{30} &  \\
&  & 2 & \foreignlanguage{greek}{λεγον \textoverline{πνα} ακαθαρτον εχειν αυτον} & 6 &  &  \\
& \textbf{31} &  & \foreignlanguage{greek}{και ερχεται αυτου η μητηρ και οι αδελ} & 8 &  &  \\
&  & 8 & \foreignlanguage{greek}{φοι αυτου και εξω εϲτωτεϲ απεϲτιλα̅} & 13 &  &  \\
&  & 14 & \foreignlanguage{greek}{προϲ αυτον καλουντεϲ αυτον και εκα} & 2 & \textbf{32} &  \\
&  & 2 & \foreignlanguage{greek}{θητο περι αυτον οχλοϲ και λεγουϲιν αυ} & 8 &  &  \\
&  & 8 & \foreignlanguage{greek}{τω ιδου η μητηρ ϲου και οι αδελφοι ϲου} & 16 &  &  \\
&  & 17 & \foreignlanguage{greek}{εξω ϲτηκουϲιν ζητουντεϲ ϲε οϲ δε απε} & 3 & \textbf{33} &  \\
&  & 3 & \foreignlanguage{greek}{κριθη και ειπεν αυτοιϲ τιϲ εϲτιν η μη} & 10 &  &  \\
&  & 10 & \foreignlanguage{greek}{τηρ και οι αδελφοι μου και περιβλεψα} & 2 & \textbf{34} &  \\
&  & 2 & \foreignlanguage{greek}{μενοϲ κυκλω αυτου καθημενουϲ τουϲ} & 6 &  &  \\
[0.2em]
\cline{4-4}
\end{tabular}
\end{center}
\end{table}
}
\clearpage
\newpage
 {
 \setlength\arrayrulewidth{1pt}
\begin{table}
\begin{center}
\begin{tabular}{ccc|l|ccc}
\cline{4-4} \\ [-1em]
\multicolumn{7}{c}{\foreignlanguage{greek}{ευαγγελιον κατα μαρκον} \textbf{(\nospace{3:34})} } \\ \\ [-1em] % Si on veut ajouter les bordures latérales, remplacer {7}{c} par {7}{|c|}
\cline{4-4} \\
\cline{4-4}
&  &  & &  &  & \\ [-0.9em]
&  & 7 & \foreignlanguage{greek}{μαθηταϲ λεγει ειδε η μητηρ μου και οι α} & 15 &  &  \\
&  & 15 & \foreignlanguage{greek}{δελφοι μου και οϲ αν ποιη το θελημα του} & 7 & \textbf{35} &  \\
&  & 8 & \foreignlanguage{greek}{\textoverline{θυ} ουτοϲ μου αδελφοϲ και αδελφη και μη} & 15 &  &  \\
&  & 15 & \foreignlanguage{greek}{τηρ εϲτιν και ηρξατο παλιν διδαϲκειν προϲ} & 5 & \mygospelchapter &  \\
&  & 6 & \foreignlanguage{greek}{την θαλαϲϲαν και ϲυνηχθη προϲ αυτον} & 11 &  &  \\
&  & 12 & \foreignlanguage{greek}{οχλοϲ πλειϲτοϲ ωϲτε αυτον ειϲ το πλοιον} & 18 &  &  \\
&  & 19 & \foreignlanguage{greek}{ενβαντα καθηϲθαι παρα τον αιγιαλον} & 23 &  &  \\
&  & 24 & \foreignlanguage{greek}{και παϲ ο οχλοϲ εν τω αιγιαλω ην και εδι} & 2 & \textbf{2} &  \\
&  & 2 & \foreignlanguage{greek}{δαϲκεν αυτουϲ εν παραβολαιϲ λεγων} & 6 &  &  \\
& \textbf{3} &  & \foreignlanguage{greek}{ακουεται ιδου εξηλθεν ο ϲπειρων ϲπειραι} & 6 &  &  \\
& \textbf{4} &  & \foreignlanguage{greek}{και το μεν επεϲεν παρα την οδον και} & 8 &  &  \\
&  & 9 & \foreignlanguage{greek}{ηλθεν τα ορνεα και κατεφαγεν αυτο} & 14 &  &  \\
& \textbf{5} &  & \foreignlanguage{greek}{αλλα δε επεϲεν επι τα πετρωδη και ο} & 8 &  &  \\
&  & 8 & \foreignlanguage{greek}{τι ουκ ειχε γην πολλην ευθεωϲ ανετει} & 2 & \textbf{6} &  \\
&  & 2 & \foreignlanguage{greek}{λε ηλιου δε ανατιλαντοϲ εκαυματι} & 6 &  &  \\
&  & 6 & \foreignlanguage{greek}{ϲθη και δια το μη εχειν ριζαν εξηρανθη} & 13 &  &  \\
& \textbf{7} &  & \foreignlanguage{greek}{και αλλα επεϲεν επι ταϲ ακανθαϲ και α} & 8 &  &  \\
&  & 8 & \foreignlanguage{greek}{νεβηϲαν αι ακανθαι και ϲυνεπνιξαν αυ} & 13 &  &  \\
&  & 13 & \foreignlanguage{greek}{τα και καρπον ουκ εδωκαν και αλλα ε} & 3 & \textbf{8} &  \\
&  & 3 & \foreignlanguage{greek}{πεϲεν ειϲ την γην την καλην και εδι} & 10 &  &  \\
&  & 10 & \foreignlanguage{greek}{δει καρπον αναβαινοντα και αυξανο} & 14 &  &  \\
&  & 14 & \foreignlanguage{greek}{μενον και φερει το εν \textoverline{λ} και το εν \textoverline{ξ}} & 23 &  &  \\
&  & 24 & \foreignlanguage{greek}{και το εν \textoverline{ρ} και ελεγεν ο εχων ωτα α} & 6 & \textbf{9} &  \\
&  & 6 & \foreignlanguage{greek}{κουειν ακουετω} & 7 &  &  \\
& \textbf{10} &  & \foreignlanguage{greek}{και οτε εγενετο κατα μοναϲ επηρωτη} & 6 &  &  \\
&  & 6 & \foreignlanguage{greek}{ϲαν αυτον οι μαθηται αυτου τιϲ η παρα} & 13 &  &  \\
&  & 13 & \foreignlanguage{greek}{βολη αυτη και λεγει αυτοιϲ υμιν} & 4 & \textbf{11} &  \\
&  & 5 & \foreignlanguage{greek}{δεδοται το μυϲτηριον τηϲ βαϲιλειαϲ} & 9 &  &  \\
&  & 10 & \foreignlanguage{greek}{του \textoverline{θυ} εκεινοιϲ δε τοιϲ εξω εν παρα} & 17 &  &  \\
&  & 17 & \foreignlanguage{greek}{βολαιϲ παντα γεινεται ινα βλεποντεϲ} & 2 & \textbf{12} &  \\
[0.2em]
\cline{4-4}
\end{tabular}
\end{center}
\end{table}
}
\clearpage
\newpage
 {
 \setlength\arrayrulewidth{1pt}
\begin{table}
\begin{center}
\begin{tabular}{ccc|l|ccc}
\cline{4-4} \\ [-1em]
\multicolumn{7}{c}{\foreignlanguage{greek}{ευαγγελιον κατα μαρκον} \textbf{(\nospace{4:12})} } \\ \\ [-1em] % Si on veut ajouter les bordures latérales, remplacer {7}{c} par {7}{|c|}
\cline{4-4} \\
\cline{4-4}
&  &  & &  &  & \\ [-0.9em]
&  & 3 & \foreignlanguage{greek}{μη ιδωϲιν και ακουοντεϲ μη ϲυνωϲιν} & 8 &  &  \\
&  & 9 & \foreignlanguage{greek}{μηποτε επιϲτρεψωϲιν και αφεθη αυτοιϲ} & 13 &  &  \\
& \textbf{13} &  & \foreignlanguage{greek}{και λεγει αυτοιϲ ουκ οιδατε την παρα} & 7 &  &  \\
&  & 7 & \foreignlanguage{greek}{βολην ταυτην και πωϲ παϲαϲ ταϲ παρα} & 13 &  &  \\
&  & 13 & \foreignlanguage{greek}{βολαϲ γνωϲεϲθαι ο ϲπειρων τον λογον} & 4 & \textbf{14} &  \\
&  & 5 & \foreignlanguage{greek}{ϲπειρει ουτοι δε ειϲιν οι παρα την οδον} & 7 & \textbf{15} &  \\
&  & 8 & \foreignlanguage{greek}{οπου ϲπειρεται ο λογοϲ και οταν ακουϲω} & 14 &  &  \\
&  & 14 & \foreignlanguage{greek}{ϲιν ευθυϲ ερχεται ο ϲαταναϲ και ερει το̅} & 21 &  &  \\
&  & 22 & \foreignlanguage{greek}{λογον τον εϲπαρμενον ειϲ αυτουϲ} & 26 &  &  \\
& \textbf{16} &  & \foreignlanguage{greek}{ουτοι δε ειϲιν οι επι τα πετρωδη ϲπειρο} & 8 &  &  \\
&  & 8 & \foreignlanguage{greek}{μενοι οιτινεϲ οταν ακουϲωϲι τον λογο̅} & 13 &  &  \\
&  & 14 & \foreignlanguage{greek}{ευθεωϲ μετα χαραϲ λαμβανουϲιν αυ} & 18 &  &  \\
&  & 18 & \foreignlanguage{greek}{τον και ουκ εχουϲιν ριζαν εν εαυτοιϲ} & 6 & \textbf{17} &  \\
&  & 7 & \foreignlanguage{greek}{αλλα προϲκαιροι ειϲιν ειτα γενομε} & 11 &  &  \\
&  & 11 & \foreignlanguage{greek}{νηϲ θλιψεωϲ και διωγμου δια τον λογο̅} & 17 &  &  \\
&  & 18 & \foreignlanguage{greek}{και ευθυϲ ϲκανδαλιζονται οι δε ειϲ} & 3 & \textbf{18} &  \\
&  & 4 & \foreignlanguage{greek}{ταϲ ακανθαϲ ϲπειρομενοι ουτοι ειϲι̅} & 8 &  &  \\
&  & 9 & \foreignlanguage{greek}{οι τον λογον ακουοντεϲ και αι μερι} & 3 & \textbf{19} &  \\
&  & 3 & \foreignlanguage{greek}{μναι του βιου και απαται του πλουτου} & 9 &  &  \\
&  & 10 & \foreignlanguage{greek}{ειϲπορευομεναι ϲυνπνιγουϲι τον} & 12 &  &  \\
&  & 13 & \foreignlanguage{greek}{λογον και ακαρποι γιγνονται ουτοι} & 1 & \textbf{20} &  \\
&  & 2 & \foreignlanguage{greek}{δε ειϲιν οι επι την γην την καλην πιπτο̅} & 10 &  &  \\
&  & 10 & \foreignlanguage{greek}{τεϲ οιτινεϲ ακουουϲιν τον λογον και} & 15 &  &  \\
&  & 16 & \foreignlanguage{greek}{παραδεχονται και καρπον φερουϲιν} & 19 &  &  \\
&  & 20 & \foreignlanguage{greek}{το εν \textoverline{λ} και το εν \textoverline{ξ} και το εν \textoverline{ρ} και λεγει} & 2 & \textbf{21} &  \\
&  & 3 & \foreignlanguage{greek}{αυτοιϲ μητι ο λυχνοϲ καιεται ινα υπο} & 9 &  &  \\
&  & 10 & \foreignlanguage{greek}{τον μοδιον τεθη η υπο την κλεινην} & 16 &  &  \\
&  & 17 & \foreignlanguage{greek}{αλλ ινα επι την λυχνιαν τεθη ουδεν} & 1 & \textbf{22} &  \\
&  & 2 & \foreignlanguage{greek}{γαρ εϲτιν κρυπτον αλλ ινα φανερωθη} & 7 &  &  \\
&  & 8 & \foreignlanguage{greek}{ουδε εγενετο αποκρυφον αλλ ινα ειϲ} & 13 &  &  \\
[0.2em]
\cline{4-4}
\end{tabular}
\end{center}
\end{table}
}
\clearpage
\newpage
 {
 \setlength\arrayrulewidth{1pt}
\begin{table}
\begin{center}
\begin{tabular}{ccc|l|ccc}
\cline{4-4} \\ [-1em]
\multicolumn{7}{c}{\foreignlanguage{greek}{ευαγγελιον κατα μαρκον} \textbf{(\nospace{4:22})} } \\ \\ [-1em] % Si on veut ajouter les bordures latérales, remplacer {7}{c} par {7}{|c|}
\cline{4-4} \\
\cline{4-4}
&  &  & &  &  & \\ [-0.9em]
&  & 14 & \foreignlanguage{greek}{φανερον ελθη ει τιϲ εχει ωτα ακουειν} & 5 & \textbf{23} &  \\
&  & 6 & \foreignlanguage{greek}{ακουετω και ελεγεν αυτοιϲ βλεπε} & 4 & \textbf{24} &  \\
&  & 4 & \foreignlanguage{greek}{ται τι ακουεται εν ω μετρω μετριται} & 10 &  &  \\
&  & 11 & \foreignlanguage{greek}{μετρηθηϲεται υμιν οϲ γαρ εχει δοθηϲε} & 4 & \textbf{25} &  \\
&  & 4 & \foreignlanguage{greek}{ται αυτω και οϲ ουκ εχει και ο εχει αρθη} & 13 &  &  \\
&  & 13 & \foreignlanguage{greek}{ϲεται απ αυτου και ελεγεν ουτωϲ εϲτι̅} & 4 & \textbf{26} &  \\
&  & 5 & \foreignlanguage{greek}{η βαϲιλεια του \textoverline{θυ} ωϲ \textoverline{ανοϲ} οταν βαλη ϲπο} & 13 &  &  \\
&  & 13 & \foreignlanguage{greek}{ρον επι την γην και καθευδη και εγει} & 4 & \textbf{27} &  \\
&  & 4 & \foreignlanguage{greek}{ρεται νυκτα και ημεραν και ο ϲποροϲ} & 10 &  &  \\
&  & 11 & \foreignlanguage{greek}{βλαϲτα και μηκυνεται ωϲ ουκ οιδεν} & 16 &  &  \\
&  & 17 & \foreignlanguage{greek}{αυτοϲ αυτοματη γαρ η γη καρποφορει} & 5 & \textbf{28} &  \\
&  & 6 & \foreignlanguage{greek}{πρωτον χορτον ειτα ϲταχυν ειτα πλη} & 11 &  &  \\
&  & 11 & \foreignlanguage{greek}{ρηϲ ο ϲειτοϲ εν τω ϲταχυει οταν παρα} & 2 & \textbf{29} &  \\
&  & 2 & \foreignlanguage{greek}{δω ο καρποϲ αποϲτελλει το δρεπανον} & 7 &  &  \\
&  & 8 & \foreignlanguage{greek}{οτι παρεϲτηκεν ο θεριϲμοϲ} & 11 &  &  \\
& \textbf{30} &  & \foreignlanguage{greek}{και ελεγεν πωϲ ομοιωϲωμεν την βα} & 6 &  &  \\
&  & 6 & \foreignlanguage{greek}{ϲιλειαν του \textoverline{θυ} η εν τινι την παραβο} & 13 &  &  \\
&  & 13 & \foreignlanguage{greek}{λην δωμεν ωϲ κοκκον ϲιναπεωϲ οπο} & 4 & \textbf{31} &  \\
&  & 4 & \foreignlanguage{greek}{ταν ϲπαρη επι την γην μικροτεροϲ ω̅} & 11 &  &  \\
&  & 12 & \foreignlanguage{greek}{παντων των ϲπερματων των επι} & 16 &  &  \\
&  & 17 & \foreignlanguage{greek}{τηϲ γηϲ αυξει και γεινεται μειζον πα̅} & 5 & \textbf{32} &  \\
&  & 5 & \foreignlanguage{greek}{των των λαχανων και ποιει κλαδουϲ} & 12 &  &  \\
&  & 13 & \foreignlanguage{greek}{μεγαλουϲ ωϲτε δυναϲθαι αυτου υπο τη̅} & 18 &  &  \\
&  & 19 & \foreignlanguage{greek}{ϲκιαν τα πετινα του ουρανου καταϲκη} & 24 &  &  \\
&  & 24 & \foreignlanguage{greek}{νουν και τοιαυταιϲ παραβολαιϲ ελα} & 4 & \textbf{33} &  \\
&  & 4 & \foreignlanguage{greek}{λει αυτοιϲ τον λογον καθωϲ εδυναν} & 9 &  &  \\
&  & 9 & \foreignlanguage{greek}{το ακουειν χωριϲ δε παραβοληϲ ουκ ε} & 5 & \textbf{34} &  \\
&  & 5 & \foreignlanguage{greek}{λαλει αυτοιϲ καθ ειδιαν δε τοιϲ μα} & 11 &  &  \\
&  & 11 & \foreignlanguage{greek}{θηταιϲ αυτου επελυεν αυταϲ} & 14 &  &  \\
& \textbf{35} &  & \foreignlanguage{greek}{και λεγει αυτοιϲ εν εκεινη τη ημερα} & 7 &  &  \\
[0.2em]
\cline{4-4}
\end{tabular}
\end{center}
\end{table}
}
\clearpage
\newpage
 {
 \setlength\arrayrulewidth{1pt}
\begin{table}
\begin{center}
\begin{tabular}{ccc|l|ccc}
\cline{4-4} \\ [-1em]
\multicolumn{7}{c}{\foreignlanguage{greek}{ευαγγελιον κατα μαρκον} \textbf{(\nospace{4:35})} } \\ \\ [-1em] % Si on veut ajouter les bordures latérales, remplacer {7}{c} par {7}{|c|}
\cline{4-4} \\
\cline{4-4}
&  &  & &  &  & \\ [-0.9em]
&  & 8 & \foreignlanguage{greek}{οψειαϲ γενομενηϲ διελθωμεν ειϲ το} & 12 &  &  \\
&  & 13 & \foreignlanguage{greek}{περαν και αφιουϲιν τον οχλον και πα} & 6 & \textbf{36} &  \\
&  & 6 & \foreignlanguage{greek}{ραλαμβανουϲιν αυτον ωϲ ην εν τω} & 11 &  &  \\
&  & 12 & \foreignlanguage{greek}{πλοιω και αμα πολλοι ηϲαν μετ αυτου} & 18 &  &  \\
& \textbf{37} &  & \foreignlanguage{greek}{και γεινεται λελαψ μεγαλου ανεμου} & 5 &  &  \\
&  & 6 & \foreignlanguage{greek}{και τα κυματα ειϲεβαλλεν ειϲ το πλοι} & 12 &  &  \\
&  & 12 & \foreignlanguage{greek}{ον ωϲτε αυτο ηδη γεμιζεϲθαι} & 16 &  &  \\
& \textbf{38} &  & \foreignlanguage{greek}{και ην αυτοϲ εν τη πρυμνη επι προϲ} & 8 &  &  \\
&  & 9 & \foreignlanguage{greek}{κεφαλαιου καθευδων και διεγειρα̅} & 12 &  &  \\
&  & 12 & \foreignlanguage{greek}{τεϲ αυτον λεγουϲιν διδαϲκαλε ου με} & 17 &  &  \\
&  & 17 & \foreignlanguage{greek}{λει ϲοι οτι απολλυμεθα} & 20 &  &  \\
& \textbf{39} &  & \foreignlanguage{greek}{και εγερθειϲ επετιμηϲεν τω ανεμω} & 5 &  &  \\
&  & 6 & \foreignlanguage{greek}{και τη θαλαϲϲη και ειπεν φιμωθητι} & 11 &  &  \\
&  & 12 & \foreignlanguage{greek}{και εκοπαϲεν ο ανεμοϲ και εγενετο} & 17 &  &  \\
&  & 18 & \foreignlanguage{greek}{γαληνη και λεγει αυτοιϲ τι διλοι ε} & 6 & \textbf{40} &  \\
&  & 6 & \foreignlanguage{greek}{ϲται ουτωϲ εχεται πιϲτιν} & 9 &  &  \\
& \textbf{41} &  & \foreignlanguage{greek}{και εφοβηθηϲαν φοβον μεγαν και} & 5 &  &  \\
&  & 6 & \foreignlanguage{greek}{ελεγον προϲ αλληλουϲ τιϲ αρα ουτοϲ} & 11 &  &  \\
&  & 12 & \foreignlanguage{greek}{εϲτιν οτι η θαλαϲϲα οι ανεμοι υ} & 18 &  &  \\
&  & 18 & \foreignlanguage{greek}{πακουουϲιν αυτω και ηλθαν ειϲ το} & 4 & \mygospelchapter &  \\
&  & 5 & \foreignlanguage{greek}{περαν τηϲ θαλαϲϲηϲ ειϲ τη χωραν τω̅} & 11 &  &  \\
&  & 12 & \foreignlanguage{greek}{γεργυϲτηνων και εξελθοντων αυ} & 3 & \textbf{2} &  \\
&  & 3 & \foreignlanguage{greek}{των εκ του πλοιου απηντηϲεν αυτω} & 8 &  &  \\
&  & 9 & \foreignlanguage{greek}{\textoverline{ανοϲ} τ�οϲ εκ των μνημιων εν \textoverline{πνι} ακα} & 16 &  &  \\
&  & 16 & \foreignlanguage{greek}{θαρτω οϲ ειχεν την κατοικηϲιν εν} & 5 & \textbf{3} &  \\
&  & 6 & \foreignlanguage{greek}{τοιϲ μνημιοιϲ και ουδε αλυϲι αυτον} & 11 &  &  \\
&  & 12 & \foreignlanguage{greek}{ουκετι εδυνατο δηϲαι δια το πολ} & 3 & \textbf{4} &  \\
&  & 3 & \foreignlanguage{greek}{λακειϲ αυτον δεδεϲθαι και πεδεϲ και} & 8 &  &  \\
&  & 9 & \foreignlanguage{greek}{αλυϲεϲι διεϲπαρκεναι δε ταϲ αλυϲιϲ} & 13 &  &  \\
&  & 14 & \foreignlanguage{greek}{και ταϲ πεδαϲ ϲυντετριφεναι} & 17 &  &  \\
[0.2em]
\cline{4-4}
\end{tabular}
\end{center}
\end{table}
}
\clearpage
\newpage
 {
 \setlength\arrayrulewidth{1pt}
\begin{table}
\begin{center}
\begin{tabular}{ccc|l|ccc}
\cline{4-4} \\ [-1em]
\multicolumn{7}{c}{\foreignlanguage{greek}{ευαγγελιον κατα μαρκον} \textbf{(\nospace{5:4})} } \\ \\ [-1em] % Si on veut ajouter les bordures latérales, remplacer {7}{c} par {7}{|c|}
\cline{4-4} \\
\cline{4-4}
&  &  & &  &  & \\ [-0.9em]
&  & 18 & \foreignlanguage{greek}{μηδενα δε ιϲχυειν αυτον ετι δαμαϲαι} & 23 &  &  \\
& \textbf{5} &  & \foreignlanguage{greek}{νυκτοϲ δε και ημεραϲ δια παντοϲ ε̅} & 7 &  &  \\
&  & 8 & \foreignlanguage{greek}{τοιϲ ορεϲιν και εν τοιϲ μνημιοιϲ ην} & 14 &  &  \\
&  & 15 & \foreignlanguage{greek}{κραζων και κατακοπτων εαυτον} & 18 &  &  \\
&  & 19 & \foreignlanguage{greek}{λιθοιϲ} & 19 &  &  \\
& \textbf{6} &  & \foreignlanguage{greek}{ιδων δε τον \textoverline{ιν} μακροθεν προϲεδρα} & 6 &  &  \\
&  & 6 & \foreignlanguage{greek}{μεν και προϲεκυνηϲεν αυτω και κρα} & 2 & \textbf{7} &  \\
&  & 2 & \foreignlanguage{greek}{ξαϲ φωνη μεγαλη λεγει τι εμοι και ϲυ} & 9 &  &  \\
&  & 10 & \foreignlanguage{greek}{\textoverline{ιυ} υιε \textoverline{θυ} του υψιϲτου ορκιζω ϲε τον} & 17 &  &  \\
&  & 18 & \foreignlanguage{greek}{\textoverline{θν} μη με βαϲανιϲηϲ ελεγεν γαρ αυτω} & 3 & \textbf{8} &  \\
&  & 4 & \foreignlanguage{greek}{εξελθε το \textoverline{πνα} το ακαθαρτον εκ του} & 10 &  &  \\
&  & 11 & \foreignlanguage{greek}{\textoverline{ανου} και επηρωτα αυτον τι ονομα ϲοι} & 6 & \textbf{9} &  \\
&  & 7 & \foreignlanguage{greek}{και λεγει αυτω λεγιων ονομα μοι} & 12 &  &  \\
&  & 13 & \foreignlanguage{greek}{οτι πολλοι εϲμεν και παρεκαλει αυ} & 3 & \textbf{10} &  \\
&  & 3 & \foreignlanguage{greek}{τον πολλα ινα μη αποϲτιλη αυτον} & 8 &  &  \\
&  & 9 & \foreignlanguage{greek}{εξω τηϲ χωραϲ ην δε εκει αγελη} & 4 & \textbf{11} &  \\
&  & 5 & \foreignlanguage{greek}{χοιρων μεγαλη προϲ τω ορι βοϲκομε} & 10 &  &  \\
&  & 10 & \foreignlanguage{greek}{νη και παρεκαλεϲαντεϲ αυτον ειπα̅} & 4 & \textbf{12} &  \\
&  & 5 & \foreignlanguage{greek}{πεμψον ημαϲ ειϲ τουϲ χοιρουϲ ινα} & 10 &  &  \\
&  & 11 & \foreignlanguage{greek}{ειϲ αυτουϲ ειϲελθωμεν και επετρε} & 2 & \textbf{13} &  \\
&  & 2 & \foreignlanguage{greek}{ψεν αυτοιϲ και εξελθοντα τα \textoverline{πνα} τα} & 8 &  &  \\
&  & 9 & \foreignlanguage{greek}{ακαθαρτα ειϲηλθαν ειϲ τουϲ χοιρουϲ} & 13 &  &  \\
&  & 14 & \foreignlanguage{greek}{και ωρμηϲεν η αγελη κατα του κρη} & 20 &  &  \\
&  & 20 & \foreignlanguage{greek}{μνου ειϲ την θαλαϲϲαν ωϲ διϲχιλιοι} & 25 &  &  \\
&  & 26 & \foreignlanguage{greek}{και επνιγοντο εν τη θαλαϲϲη} & 30 &  &  \\
& \textbf{14} &  & \foreignlanguage{greek}{και οι βοϲκοντεϲ αυτουϲ εφυγον και} & 6 &  &  \\
&  & 7 & \foreignlanguage{greek}{ανηγγειλον ειϲ την πολιν και ειϲ τουϲ} & 13 &  &  \\
&  & 14 & \foreignlanguage{greek}{αγρουϲ και εξηλθον ιδειν τι εϲτιν το} & 20 &  &  \\
&  & 21 & \foreignlanguage{greek}{γεγονοϲ και ερχονται προϲ τον \textoverline{ιν}} & 5 & \textbf{15} &  \\
&  & 6 & \foreignlanguage{greek}{και ευριϲκουϲιν τον δαιμονιζομενον} & 9 &  &  \\
[0.2em]
\cline{4-4}
\end{tabular}
\end{center}
\end{table}
}
\clearpage
\newpage
 {
 \setlength\arrayrulewidth{1pt}
\begin{table}
\begin{center}
\begin{tabular}{ccc|l|ccc}
\cline{4-4} \\ [-1em]
\multicolumn{7}{c}{\foreignlanguage{greek}{ευαγγελιον κατα μαρκον} \textbf{(\nospace{5:15})} } \\ \\ [-1em] % Si on veut ajouter les bordures latérales, remplacer {7}{c} par {7}{|c|}
\cline{4-4} \\
\cline{4-4}
&  &  & &  &  & \\ [-0.9em]
&  & 10 & \foreignlanguage{greek}{ϲωφρονουντα τον εϲχηκοτα τον λεγε} & 14 &  &  \\
&  & 14 & \foreignlanguage{greek}{ωνα και εφοβηθηϲαν και διηγηϲαντο} & 2 & \textbf{16} &  \\
&  & 3 & \foreignlanguage{greek}{αυτοιϲ οι ειδοτεϲ πωϲ εγενετο τω δαι} & 9 &  &  \\
&  & 9 & \foreignlanguage{greek}{μονιζομενω και περι των χοιρων} & 13 &  &  \\
& \textbf{17} &  & \foreignlanguage{greek}{και ηρξαντο παρακαλειν αυτον απελ} & 5 &  &  \\
&  & 5 & \foreignlanguage{greek}{θειν απο των οριων αυτων και ενβε} & 2 & \textbf{18} &  \\
&  & 2 & \foreignlanguage{greek}{νοντοϲ αυτου ειϲ το πλοιον παρεκα} & 7 &  &  \\
&  & 7 & \foreignlanguage{greek}{λει αυτον ο δαιμονιϲθειϲ ινα μετ αυ} & 13 &  &  \\
&  & 13 & \foreignlanguage{greek}{του η και ουκ αφηκεν αυτον αλλα λε} & 6 & \textbf{19} &  \\
&  & 6 & \foreignlanguage{greek}{γει αυτω υπαγε ειϲ τον οικον ϲου προϲ} & 13 &  &  \\
&  & 14 & \foreignlanguage{greek}{τουϲ ϲουϲ και διαγγειλον αυτοιϲ οϲα ϲοι} & 20 &  &  \\
&  & 21 & \foreignlanguage{greek}{ο \textoverline{κϲ} πεποιηκεν και ηλεηκεν ϲε} & 26 &  &  \\
& \textbf{20} &  & \foreignlanguage{greek}{και απηλθεν και ηρξατο κηρυϲϲιν εν} & 6 &  &  \\
&  & 7 & \foreignlanguage{greek}{τη δεκαπολει οϲα εποιηϲεν αυτω ο \textoverline{ιϲ}} & 13 &  &  \\
&  & 14 & \foreignlanguage{greek}{και παντεϲ εθαυμαζον και διαπερα} & 2 & \textbf{21} &  \\
&  & 2 & \foreignlanguage{greek}{ϲαντεϲ εν τω πλοιω του \textoverline{ιυ} παλιν ειϲ το} & 10 &  &  \\
&  & 11 & \foreignlanguage{greek}{περαν ϲυνηχθη οχλοϲ πολυϲ επ αυτον} & 16 &  &  \\
&  & 17 & \foreignlanguage{greek}{και ην παρα την θαλαϲϲαν και ιδου} & 2 & \textbf{22} &  \\
&  & 3 & \foreignlanguage{greek}{ερχεται τιϲ των αρχιϲυναγωγων ω ο} & 8 &  &  \\
&  & 8 & \foreignlanguage{greek}{νομα ιαειροϲ και ειδων αυτον προϲπι} & 13 &  &  \\
&  & 13 & \foreignlanguage{greek}{πτι προϲ τουϲ ποδαϲ αυτου και παρε} & 2 & \textbf{23} &  \\
&  & 2 & \foreignlanguage{greek}{καλει αυτον πολλα λεγων οτι το θυ} & 8 &  &  \\
&  & 8 & \foreignlanguage{greek}{γατριον μου εϲχατωϲ εχει ινα ελθω̅} & 13 &  &  \\
&  & 14 & \foreignlanguage{greek}{επιθηϲ ταϲ χειραϲ αυτη ινα ϲωθη και} & 20 &  &  \\
&  & 21 & \foreignlanguage{greek}{ζηϲεται και απηλθεν μετ αυτου} & 4 & \textbf{24} &  \\
&  & 5 & \foreignlanguage{greek}{και ηκολουθει αυτω οχλοϲ πολυϲ και} & 10 &  &  \\
&  & 11 & \foreignlanguage{greek}{ϲυνεθλιβον αυτον και γυνη ουϲα} & 3 & \textbf{25} &  \\
&  & 4 & \foreignlanguage{greek}{εν ρυϲει αιματοϲ \textoverline{ιβ} ετη και πολλα πα} & 3 & \textbf{26} &  \\
&  & 3 & \foreignlanguage{greek}{θουϲα υπο πολλων ιατρων και δαπα} & 8 &  &  \\
&  & 8 & \foreignlanguage{greek}{νηϲαϲα τα εαυτηϲ παντα και μηδεν} & 13 &  &  \\
[0.2em]
\cline{4-4}
\end{tabular}
\end{center}
\end{table}
}
\clearpage
\newpage
 {
 \setlength\arrayrulewidth{1pt}
\begin{table}
\begin{center}
\begin{tabular}{ccc|l|ccc}
\cline{4-4} \\ [-1em]
\multicolumn{7}{c}{\foreignlanguage{greek}{ευαγγελιον κατα μαρκον} \textbf{(\nospace{5:26})} } \\ \\ [-1em] % Si on veut ajouter les bordures latérales, remplacer {7}{c} par {7}{|c|}
\cline{4-4} \\
\cline{4-4}
&  &  & &  &  & \\ [-0.9em]
&  & 14 & \foreignlanguage{greek}{ωφεληθειϲα αλλα μαλλον ειϲ το χειρον} & 19 &  &  \\
&  & 20 & \foreignlanguage{greek}{ελθουϲα και ακουϲαϲα περι του \textoverline{ιυ} εν τω} & 7 & \textbf{27} &  \\
&  & 8 & \foreignlanguage{greek}{οχλω οπιϲθεν ηψατο αυτου ελεγεν γαρ} & 2 & \textbf{28} &  \\
&  & 3 & \foreignlanguage{greek}{οτι καν των ιματιων αψωμαι αυτου ϲω} & 9 &  &  \\
&  & 9 & \foreignlanguage{greek}{θηϲομαι και ευθεωϲ εξηρανθη η πηγη} & 5 & \textbf{29} &  \\
&  & 6 & \foreignlanguage{greek}{του αιματοϲ αυτηϲ και εγνω τω ϲωμα} & 12 &  &  \\
&  & 12 & \foreignlanguage{greek}{τι οτι ειαθη απο τηϲ μαϲτιγοϲ και ευθε} & 2 & \textbf{30} &  \\
&  & 2 & \foreignlanguage{greek}{ωϲ ο \textoverline{ιϲ} επιγνουϲ εν εαυτω την εξ αυτου} & 10 &  &  \\
&  & 11 & \foreignlanguage{greek}{δυναμιν εξελθουϲαν επιϲτραφειϲ ε̅} & 14 &  &  \\
&  & 15 & \foreignlanguage{greek}{τω οχλω ειπεν τιϲ μου ηψατο των ιμα} & 22 &  &  \\
&  & 22 & \foreignlanguage{greek}{τιων και ελεγον αυτω οι μαθηται} & 5 & \textbf{31} &  \\
&  & 6 & \foreignlanguage{greek}{βλεπειϲ τον οχλον ϲυντριβοντα ϲε και} & 11 &  &  \\
&  & 12 & \foreignlanguage{greek}{λεγειϲ τιϲ μου ηψατο και περιεβλεπε} & 2 & \textbf{32} &  \\
&  & 2 & \foreignlanguage{greek}{το την τουτο πεποιηκυιαν η δε γυνη} & 3 & \textbf{33} &  \\
&  & 4 & \foreignlanguage{greek}{φοβηθειϲα και τρεμουϲα ιδυια ο γεγο} & 9 &  &  \\
&  & 9 & \foreignlanguage{greek}{νεν επ αυτη ηλθεν και προϲεπεϲεν} & 14 &  &  \\
&  & 15 & \foreignlanguage{greek}{αυτω και ειπεν αυτω εμπροϲθεν πα̅} & 20 &  &  \\
&  & 20 & \foreignlanguage{greek}{των παϲαν την αιτιαν αυτηϲ} & 24 &  &  \\
& \textbf{34} &  & \foreignlanguage{greek}{ο δε ειπεν αυτη θυγατηρ η πιϲτιϲ ϲου} & 8 &  &  \\
&  & 9 & \foreignlanguage{greek}{ϲεϲωκεν ϲε υπαγε ειϲ ειρηνην και ι} & 15 &  &  \\
&  & 15 & \foreignlanguage{greek}{ϲθει υγειηϲ απο τηϲ μαϲτιγοϲ ϲου} & 20 &  &  \\
& \textbf{35} &  & \foreignlanguage{greek}{ετι αυτου λαλουντοϲ ερχονται απο του} & 6 &  &  \\
&  & 7 & \foreignlanguage{greek}{αρχιϲυναγωγου λεγοντεϲ οτι η θυγα} & 11 &  &  \\
&  & 11 & \foreignlanguage{greek}{τηρ ϲου απεθανεν τι ετι ϲκυλλιϲ τον} & 17 &  &  \\
&  & 18 & \foreignlanguage{greek}{διδαϲκαλον ο δε \textoverline{ιϲ} παρακουϲαϲ το̅} & 5 & \textbf{36} &  \\
&  & 6 & \foreignlanguage{greek}{λογον λαλουμενον λεγει τω αρχιϲυν} & 10 &  &  \\
&  & 10 & \foreignlanguage{greek}{αγωγω μη φοβου μονον πιϲτευε και} & 1 & \textbf{37} &  \\
&  & 2 & \foreignlanguage{greek}{ουκ αφηκεν αυτω ουδενα παρακο} & 6 &  &  \\
&  & 6 & \foreignlanguage{greek}{λουθηϲε ει μη μονον πετρον και ια} & 12 &  &  \\
&  & 12 & \foreignlanguage{greek}{κωβον και ιωαννην τον αδελφον} & 16 &  &  \\
[0.2em]
\cline{4-4}
\end{tabular}
\end{center}
\end{table}
}
\clearpage
\newpage
 {
 \setlength\arrayrulewidth{1pt}
\begin{table}
\begin{center}
\begin{tabular}{ccc|l|ccc}
\cline{4-4} \\ [-1em]
\multicolumn{7}{c}{\foreignlanguage{greek}{ευαγγελιον κατα μαρκον} \textbf{(\nospace{5:37})} } \\ \\ [-1em] % Si on veut ajouter les bordures latérales, remplacer {7}{c} par {7}{|c|}
\cline{4-4} \\
\cline{4-4}
&  &  & &  &  & \\ [-0.9em]
&  & 17 & \foreignlanguage{greek}{ιακωβου και ερχεται ειϲ τον οικον του} & 6 & \textbf{38} &  \\
&  & 7 & \foreignlanguage{greek}{αρχιϲυναγωγου και θεωρει θορυβον} & 10 &  &  \\
&  & 11 & \foreignlanguage{greek}{και κλαιονταϲ και αλαλαζονταϲ πολλα} & 15 &  &  \\
& \textbf{39} &  & \foreignlanguage{greek}{και ειϲελθων λεγει αυτοιϲ τι θορυβι} & 6 &  &  \\
&  & 6 & \foreignlanguage{greek}{ϲθαι και κλαιετε το παιδιον ουκ απε} & 12 &  &  \\
&  & 12 & \foreignlanguage{greek}{θανεν αλλα καθευδει και κατεγελω̅} & 2 & \textbf{40} &  \\
&  & 3 & \foreignlanguage{greek}{αυτου ειδοτεϲ οτι απεθανεν} & 6 &  &  \\
&  & 7 & \foreignlanguage{greek}{ο δε εκβαλων πανταϲ παραλαμβανι} & 11 &  &  \\
&  & 12 & \foreignlanguage{greek}{τον \textoverline{πρα} του παιδιου και την μητερα} & 18 &  &  \\
&  & 19 & \foreignlanguage{greek}{και τουϲ εαυτου και ειϲπορευεται ο} & 24 &  &  \\
&  & 24 & \foreignlanguage{greek}{που ην το παιδιον κατακειμενον} & 28 &  &  \\
& \textbf{41} &  & \foreignlanguage{greek}{και κρατηϲαϲ τηϲ χειροϲ του παιδιου} & 6 &  &  \\
&  & 7 & \foreignlanguage{greek}{λεγει ταβιθα ο εϲτιν μεθερμηνευο} & 11 &  &  \\
&  & 11 & \foreignlanguage{greek}{μενον το κοραϲιον ϲοι λεγω εγειρε} & 16 &  &  \\
& \textbf{42} &  & \foreignlanguage{greek}{και ευθεωϲ ανεϲτη το κοραϲιον και πε} & 7 &  &  \\
&  & 7 & \foreignlanguage{greek}{ριεπατει ην γαρ ετων \textoverline{ιβ} και εξεϲτηϲα̅} & 13 &  &  \\
&  & 14 & \foreignlanguage{greek}{εκϲταϲει μεγαλη και διεϲτιλατο αυτοιϲ} & 3 & \textbf{43} &  \\
&  & 4 & \foreignlanguage{greek}{πολλα ινα μηδειϲ γνοι τουτο και ειπε} & 10 &  &  \\
&  & 11 & \foreignlanguage{greek}{δοθηναι αυτη φαγειν και εξηλθεν} & 2 & \mygospelchapter &  \\
&  & 3 & \foreignlanguage{greek}{ειϲ την πατριδα αυτου και ακολου} & 8 &  &  \\
&  & 8 & \foreignlanguage{greek}{θουϲιν αυτω οι μαθηται αυτου} & 12 &  &  \\
& \textbf{2} &  & \foreignlanguage{greek}{και γενομενου ϲαββατου ηρξαντο ε̅} & 5 &  &  \\
&  & 6 & \foreignlanguage{greek}{τη ϲυναγωγη διδαϲκειν και πολλοι} & 10 &  &  \\
&  & 11 & \foreignlanguage{greek}{ακουοντεϲ εξεπληϲϲοντο λεγοντεϲ} & 13 &  &  \\
&  & 14 & \foreignlanguage{greek}{ποθεν τουτω ταυτα και τιϲ η ϲοφια} & 20 &  &  \\
&  & 21 & \foreignlanguage{greek}{η δοθειϲα αυτω και δυναμιϲ τοιαυται} & 26 &  &  \\
&  & 27 & \foreignlanguage{greek}{δια των χειρων αυτου γεινονται} & 31 &  &  \\
& \textbf{3} &  & \foreignlanguage{greek}{ουχ ουτοϲ εϲτιν ο τεκτων ο υιοϲ τηϲ} & 8 &  &  \\
&  & 9 & \foreignlanguage{greek}{μαριαϲ αδελφοϲ δε ιακωβου και ιω} & 14 &  &  \\
&  & 14 & \foreignlanguage{greek}{ϲη και ιουδα και ϲιμωνοϲ και ουκ ει} & 21 &  &  \\
[0.2em]
\cline{4-4}
\end{tabular}
\end{center}
\end{table}
}
\clearpage
\newpage
 {
 \setlength\arrayrulewidth{1pt}
\begin{table}
\begin{center}
\begin{tabular}{ccc|l|ccc}
\cline{4-4} \\ [-1em]
\multicolumn{7}{c}{\foreignlanguage{greek}{ευαγγελιον κατα μαρκον} \textbf{(\nospace{6:3})} } \\ \\ [-1em] % Si on veut ajouter les bordures latérales, remplacer {7}{c} par {7}{|c|}
\cline{4-4} \\
\cline{4-4}
&  &  & &  &  & \\ [-0.9em]
&  & 21 & \foreignlanguage{greek}{ϲιν αι αδελφε αυτου ωδε προϲ ημαϲ και} & 28 &  &  \\
&  & 29 & \foreignlanguage{greek}{εϲκανδαλιζοντο εν αυτω} & 31 &  &  \\
& \textbf{4} &  & \foreignlanguage{greek}{ελεγεν δε ο \textoverline{ιϲ} οτι ουκ εϲτιν προφητηϲ} & 8 &  &  \\
&  & 9 & \foreignlanguage{greek}{ατιμοϲ ει μη εν τη πατριδι αυτου και} & 16 &  &  \\
&  & 17 & \foreignlanguage{greek}{εν τοιϲ ϲυνγενεϲιν και εν τη οικεια} & 23 &  &  \\
&  & 24 & \foreignlanguage{greek}{αυτου και ουκ εδυνατο ουκετι ποι} & 5 & \textbf{5} &  \\
&  & 5 & \foreignlanguage{greek}{ηϲαι δυναμιν ει μη ολειγοιϲ αρρωϲτοιϲ} & 10 &  &  \\
&  & 11 & \foreignlanguage{greek}{επιθειϲ ταϲ χειραϲ εθεραπευϲεν και ε} & 2 & \textbf{6} &  \\
&  & 2 & \foreignlanguage{greek}{θαυμαζεν δια την απιϲτιαν αυτων} & 6 &  &  \\
&  & 7 & \foreignlanguage{greek}{και περιηγεν ταϲ κυκλω κωμαϲ διδα} & 12 &  &  \\
&  & 12 & \foreignlanguage{greek}{ϲκων και προϲκαλειται τουϲ \textoverline{ιβ} και ηρ} & 6 & \textbf{7} &  \\
&  & 6 & \foreignlanguage{greek}{ξατο αυτουϲ αποϲτελλειν δυο δυο} & 10 &  &  \\
&  & 11 & \foreignlanguage{greek}{και εδωκεν αυτοιϲ εξουϲιαν των πνευ} & 16 &  &  \\
&  & 16 & \foreignlanguage{greek}{ματων των ακαθαρτων και παρηγ} & 2 & \textbf{8} &  \\
&  & 2 & \foreignlanguage{greek}{γελλεν αυτοιϲ ινα μηδεν αρωϲιν ειϲ} & 7 &  &  \\
&  & 7 & \foreignlanguage{greek}{οδον ει μη ραβδον μονον μη πηραν} & 13 &  &  \\
&  & 14 & \foreignlanguage{greek}{μη αρτον μη ειϲ την πηραν χαλκον} & 20 &  &  \\
& \textbf{9} &  & \foreignlanguage{greek}{αλλ υποδεδεμενουϲ ϲανδαλια} & 3 &  &  \\
&  & 4 & \foreignlanguage{greek}{και μη ενδυϲηϲθαι δυο χειτωναϲ και} & 1 & \textbf{10} &  \\
&  & 2 & \foreignlanguage{greek}{ελεγεν οπου αν ειϲελθητε ειϲ οικεια̅} & 7 &  &  \\
&  & 8 & \foreignlanguage{greek}{εκει μενετε εωϲ αν εξελθητε εκει} & 13 &  &  \\
&  & 13 & \foreignlanguage{greek}{θεν και οϲ αν τοποϲ μη δεξηται} & 6 & \textbf{11} &  \\
&  & 7 & \foreignlanguage{greek}{υμαϲ μηδε ακουϲη υμων εκπορευ} & 11 &  &  \\
&  & 11 & \foreignlanguage{greek}{ομενοι εκειθεν εκτιναξατε τον} & 14 &  &  \\
&  & 15 & \foreignlanguage{greek}{χουν τον υποκατω των ποδων υμω̅} & 20 &  &  \\
&  & 21 & \foreignlanguage{greek}{ειϲ μαρτυριον αυτων} & 23 &  &  \\
& \textbf{12} &  & \foreignlanguage{greek}{και εξελθοντεϲ εκηρυϲϲον ινα μετα} & 5 &  &  \\
&  & 5 & \foreignlanguage{greek}{νοωϲιν και δαιμονια πολλα εξεπεμ} & 4 & \textbf{13} &  \\
&  & 4 & \foreignlanguage{greek}{πον και ηλιφον ελαιω πολλουϲ αρρω} & 9 &  &  \\
&  & 9 & \foreignlanguage{greek}{ϲτουϲ και εθεραπευον αυτουϲ} & 12 &  &  \\
[0.2em]
\cline{4-4}
\end{tabular}
\end{center}
\end{table}
}
\clearpage
\newpage
 {
 \setlength\arrayrulewidth{1pt}
\begin{table}
\begin{center}
\begin{tabular}{ccc|l|ccc}
\cline{4-4} \\ [-1em]
\multicolumn{7}{c}{\foreignlanguage{greek}{ευαγγελιον κατα μαρκον} \textbf{(\nospace{6:14})} } \\ \\ [-1em] % Si on veut ajouter les bordures latérales, remplacer {7}{c} par {7}{|c|}
\cline{4-4} \\
\cline{4-4}
&  &  & &  &  & \\ [-0.9em]
& \textbf{14} &  & \foreignlanguage{greek}{και ηκουϲεν ο βαϲιλευϲ ηρωδηϲ φανε} & 6 &  &  \\
&  & 6 & \foreignlanguage{greek}{ρον γαρ εγενετο το ονομα αυτου και} & 12 &  &  \\
&  & 13 & \foreignlanguage{greek}{ελεγον οτι ιωαννηϲ ο βαπτιϲτηϲ εκ νε} & 19 &  &  \\
&  & 19 & \foreignlanguage{greek}{κρων ηγερθη και δια τουτο ενεργουϲι̅} & 24 &  &  \\
&  & 25 & \foreignlanguage{greek}{αι δυναμειϲ αυτω αλλοι δε ελεγον} & 3 & \textbf{15} &  \\
&  & 4 & \foreignlanguage{greek}{οτι ηλιαϲ εϲτιν αλλοι δε ελεγον οτι} & 10 &  &  \\
&  & 11 & \foreignlanguage{greek}{προφητηϲ ωϲ ειϲ των προφητων} & 15 &  &  \\
& \textbf{16} &  & \foreignlanguage{greek}{ακουϲαϲ δε ο ηρωδηϲ ειπεν οτι ον εγω} & 8 &  &  \\
&  & 9 & \foreignlanguage{greek}{ον απεκεφαλιϲα ιωαννην ουτοϲ ηγερ} & 13 &  &  \\
&  & 13 & \foreignlanguage{greek}{θη αυτοϲ γαρ ηρωδηϲ αποϲτιλαϲ εκρα} & 5 & \textbf{17} &  \\
&  & 5 & \foreignlanguage{greek}{τηϲε τον ιωαννην και εδηϲεν αυτο̅} & 10 &  &  \\
&  & 11 & \foreignlanguage{greek}{εν τη φυλακη δια ηρωδιαδα την γυ} & 17 &  &  \\
&  & 17 & \foreignlanguage{greek}{ναικα φιλιππου του αδελφου αυτου} & 21 &  &  \\
&  & 22 & \foreignlanguage{greek}{οτι αυτην εγαμηϲεν ελεγεν γαρ ο} & 3 & \textbf{18} &  \\
&  & 4 & \foreignlanguage{greek}{ιωαννηϲ τω ηρωδη οτι ουκ εξεϲτιν} & 9 &  &  \\
&  & 10 & \foreignlanguage{greek}{ϲοι γυναικα εχειν του αδελφου ϲου} & 15 &  &  \\
& \textbf{19} &  & \foreignlanguage{greek}{η δε ηρωδιαϲ ενειχεν αυτω και ηθε} & 7 &  &  \\
&  & 7 & \foreignlanguage{greek}{λεν αυτον αποκτειναι και ουκ ηδυ} & 12 &  &  \\
&  & 12 & \foreignlanguage{greek}{νατο ο γαρ ηρωδηϲ εφοβειτο τον ιω} & 6 & \textbf{20} &  \\
&  & 6 & \foreignlanguage{greek}{αννην ιδωϲ αυτον ανδρα δικαιον ϗ} & 11 &  &  \\
&  & 12 & \foreignlanguage{greek}{αγιον και ϲυνετηρι αυτον και ακου} & 17 &  &  \\
&  & 17 & \foreignlanguage{greek}{ϲαϲ αυτου πολλα επορειτο και ηδεωϲ} & 22 &  &  \\
&  & 23 & \foreignlanguage{greek}{αυτου ηκουεν και γενομενηϲ η} & 3 & \textbf{21} &  \\
&  & 3 & \foreignlanguage{greek}{μεραϲ ευκαιρου οτε ηρωδηϲ τοιϲ γε} & 8 &  &  \\
&  & 8 & \foreignlanguage{greek}{νεϲιοιϲ αυτου διπνον εποιηϲεν τοιϲ} & 12 &  &  \\
&  & 13 & \foreignlanguage{greek}{μεγιϲταϲιν αυτου και τοιϲ χειλιαρχοιϲ} & 17 &  &  \\
&  & 18 & \foreignlanguage{greek}{και τοιϲ πρωτοιϲ τηϲ γαλιλαιαϲ} & 22 &  &  \\
& \textbf{22} &  & \foreignlanguage{greek}{και ειϲελθουϲηϲ τηϲ θυγατροϲ αυτηϲ} & 5 &  &  \\
&  & 6 & \foreignlanguage{greek}{ηρωδιαδοϲ και ορχηϲαμενηϲ και αρε} & 10 &  &  \\
&  & 10 & \foreignlanguage{greek}{ϲαϲηϲ τω ηρωδη και τοιϲ ϲυνανακει} & 15 &  &  \\
[0.2em]
\cline{4-4}
\end{tabular}
\end{center}
\end{table}
}
\clearpage
\newpage
 {
 \setlength\arrayrulewidth{1pt}
\begin{table}
\begin{center}
\begin{tabular}{ccc|l|ccc}
\cline{4-4} \\ [-1em]
\multicolumn{7}{c}{\foreignlanguage{greek}{ευαγγελιον κατα μαρκον} \textbf{(\nospace{6:22})} } \\ \\ [-1em] % Si on veut ajouter les bordures latérales, remplacer {7}{c} par {7}{|c|}
\cline{4-4} \\
\cline{4-4}
&  &  & &  &  & \\ [-0.9em]
&  & 15 & \foreignlanguage{greek}{μενοιϲ ειπεν ο βαϲιλευϲ τω κοραϲιω} & 20 &  &  \\
&  & 21 & \foreignlanguage{greek}{ετηϲαι με ο δ αν θεληϲ και δωϲω ϲοι εωϲ} & 1 & \textbf{23} &  \\
&  & 2 & \foreignlanguage{greek}{ημιϲυ τηϲ βαϲιλειαϲ η δε εξελθουϲα ει} & 4 & \textbf{24} &  \\
&  & 4 & \foreignlanguage{greek}{πεν τη μητρι αυτηϲ τι αιτηϲωμαι} & 9 &  &  \\
&  & 10 & \foreignlanguage{greek}{η δε ειπεν αιτηϲε την κεφαλην ιωαν} & 16 &  &  \\
&  & 16 & \foreignlanguage{greek}{νου του βαπτιϲτου και ειϲελθουϲα} & 2 & \textbf{25} &  \\
&  & 3 & \foreignlanguage{greek}{ευθυϲ μετα ϲπουδηϲ θελω ινα μοι δω} & 9 &  &  \\
&  & 10 & \foreignlanguage{greek}{ϲηϲ επι πινακει την κεφαλην ιωαν} & 15 &  &  \\
&  & 15 & \foreignlanguage{greek}{νου του βαπτιϲτου και περιλυποϲ} & 2 & \textbf{26} &  \\
&  & 3 & \foreignlanguage{greek}{γενομενοϲ ο βαϲιλευϲ δια τουϲ ορκουϲ} & 8 &  &  \\
&  & 9 & \foreignlanguage{greek}{και τουϲ ανακειμενουϲ ουκ ηθεληϲε̅} & 13 &  &  \\
&  & 14 & \foreignlanguage{greek}{αυτην αθετηϲαι και ευθεωϲ απο} & 3 & \textbf{27} &  \\
&  & 3 & \foreignlanguage{greek}{ϲτιλαϲ ϲπεκουλατορα επεταξεν ενε} & 6 &  &  \\
&  & 6 & \foreignlanguage{greek}{χθηναι την κεφαλην αυτου επι πινα} & 11 &  &  \\
&  & 11 & \foreignlanguage{greek}{κει και απελθων απεκεφαλιϲεν αυ} & 15 &  &  \\
&  & 15 & \foreignlanguage{greek}{τον εν τη φυλακη και ηνεγκεν την} & 3 & \textbf{28} &  \\
&  & 4 & \foreignlanguage{greek}{κεφαλην αυτου επι πινακει και εδω} & 9 &  &  \\
&  & 9 & \foreignlanguage{greek}{κεν τω κοραϲιω και το κοραϲιον εδω} & 15 &  &  \\
&  & 15 & \foreignlanguage{greek}{κεν αυτην τη μητρι αυτηϲ} & 19 &  &  \\
& \textbf{29} &  & \foreignlanguage{greek}{και ακουϲαντεϲ οι μαθηται αυτου ηλθο̅} & 6 &  &  \\
&  & 7 & \foreignlanguage{greek}{κηδευϲαι το πτωμα αυτου και εθη} & 12 &  &  \\
&  & 12 & \foreignlanguage{greek}{καν αυτον εν μνημιω και ϲυναγον} & 2 & \textbf{30} &  \\
&  & 2 & \foreignlanguage{greek}{ται οι αποϲτολοι προϲ τον \textoverline{ιν} και απηγ} & 9 &  &  \\
&  & 9 & \foreignlanguage{greek}{γειλον αυτω παντα και οϲα εποιηϲεν} & 14 &  &  \\
&  & 15 & \foreignlanguage{greek}{και εδιδαϲκεν και ειπεν αυτοιϲ} & 3 & \textbf{31} &  \\
&  & 4 & \foreignlanguage{greek}{δευτε υμειϲ κατ ιδιαν ειϲ ερημον το} & 10 &  &  \\
&  & 10 & \foreignlanguage{greek}{πον και αναπαυεϲθαι λοιπον ηϲαν γαρ} & 15 &  &  \\
&  & 16 & \foreignlanguage{greek}{οι ερχομενοι και υπαγοντεϲ πολλοι} & 20 &  &  \\
&  & 21 & \foreignlanguage{greek}{και ουδε φαγειν ηυκερουν και απηλθο̅} & 3 & \textbf{32} &  \\
&  & 4 & \foreignlanguage{greek}{ειϲ ερημον τοπον τω πλοιω κατ ιδιαν} & 10 &  &  \\
[0.2em]
\cline{4-4}
\end{tabular}
\end{center}
\end{table}
}
\clearpage
\newpage
 {
 \setlength\arrayrulewidth{1pt}
\begin{table}
\begin{center}
\begin{tabular}{ccc|l|ccc}
\cline{4-4} \\ [-1em]
\multicolumn{7}{c}{\foreignlanguage{greek}{ευαγγελιον κατα μαρκον} \textbf{(\nospace{6:33})} } \\ \\ [-1em] % Si on veut ajouter les bordures latérales, remplacer {7}{c} par {7}{|c|}
\cline{4-4} \\
\cline{4-4}
&  &  & &  &  & \\ [-0.9em]
& \textbf{33} &  & \foreignlanguage{greek}{και ιδον υπαγοντεϲ οι οχλοι} & 5 &  &  \\
&  & 6 & \foreignlanguage{greek}{και επεγνωϲαν πολλοι και πεζη απο} & 11 &  &  \\
&  & 12 & \foreignlanguage{greek}{παϲων των πολεων ϲυνεδραμον ε} & 16 &  &  \\
&  & 16 & \foreignlanguage{greek}{κει και εξελθων ειδεν πολυν οχλον} & 5 & \textbf{34} &  \\
&  & 6 & \foreignlanguage{greek}{και εϲπλανχνιϲθη επ αυτοιϲ οτι η} & 11 &  &  \\
&  & 11 & \foreignlanguage{greek}{ϲαν ωϲ προβατα μη εχοντα ποιμενα} & 16 &  &  \\
&  & 17 & \foreignlanguage{greek}{και ηρξαντο διδαϲκειν αυτουϲ πολλα} & 21 &  &  \\
& \textbf{35} &  & \foreignlanguage{greek}{και ηδη ωραϲ πολληϲ γενομενηϲ} & 5 &  &  \\
&  & 6 & \foreignlanguage{greek}{προϲελθοντεϲ αυτω οι μαθηται λε} & 10 &  &  \\
&  & 10 & \foreignlanguage{greek}{γουϲιν οτι ερημοϲ εϲτιν ο τοποϲ και} & 16 &  &  \\
&  & 17 & \foreignlanguage{greek}{ηδη ωρα παρηλθεν απολυϲον αυ} & 2 & \textbf{36} &  \\
&  & 2 & \foreignlanguage{greek}{τουϲ ινα απελθοντεϲ ειϲ τουϲ κυκλω} & 7 &  &  \\
&  & 8 & \foreignlanguage{greek}{αγρουϲ και κωμαϲ αγοραϲωϲιν εαυ} & 12 &  &  \\
&  & 12 & \foreignlanguage{greek}{τοιϲ τι φαγωϲιν ο δε αποκριθειϲ} & 3 & \textbf{37} &  \\
&  & 4 & \foreignlanguage{greek}{ειπεν αυτοιϲ δοτε αυτοιϲ υμειϲ φα} & 9 &  &  \\
&  & 9 & \foreignlanguage{greek}{γειν και λεγουϲιν αυτω απελθον} & 13 &  &  \\
&  & 13 & \foreignlanguage{greek}{τεϲ αγοραϲωμεν δηναριων \textoverline{ρ} αρτουϲ} & 17 &  &  \\
&  & 18 & \foreignlanguage{greek}{και δωμεν αυτοιϲ φαγειν ινα εκα} & 23 &  &  \\
&  & 23 & \foreignlanguage{greek}{ϲτοϲ αυτων βραχυ τι λαβη} & 27 &  &  \\
& \textbf{38} &  & \foreignlanguage{greek}{ο δε λεγει αυτοιϲ ποϲουϲ αρτουϲ εχε} & 7 &  &  \\
&  & 7 & \foreignlanguage{greek}{τε υπαγεται ειδεται και γνοντεϲ} & 11 &  &  \\
&  & 12 & \foreignlanguage{greek}{λεγουϲιν πεντε και δυο ιχθυαϲ} & 16 &  &  \\
& \textbf{39} &  & \foreignlanguage{greek}{και επεταξεν αυτοιϲ ανακλιναι} & 4 &  &  \\
&  & 5 & \foreignlanguage{greek}{πανταϲ ϲυνποϲια επι τω χλωρω χορ} & 10 &  &  \\
&  & 10 & \foreignlanguage{greek}{τω και ανεπεϲαν πραϲιαι πραϲιαι} & 4 & \textbf{40} &  \\
&  & 5 & \foreignlanguage{greek}{ανδρεϲ \textoverline{ρ} και ανα \textoverline{ν} και λαβων τουϲ} & 3 & \textbf{41} &  \\
&  & 4 & \foreignlanguage{greek}{πεντε αρτουϲ και τουϲ δυο ιχθυαϲ} & 9 &  &  \\
&  & 10 & \foreignlanguage{greek}{αναβλεψαϲ ειϲ τον ουρανον ηυλογη} & 14 &  &  \\
&  & 14 & \foreignlanguage{greek}{ϲεν και κατεκλαϲεν τουϲ πεντε αρ} & 19 &  &  \\
&  & 19 & \foreignlanguage{greek}{τουϲ και εδιδου τοιϲ μαθηταιϲ αυτου} & 24 &  &  \\
[0.2em]
\cline{4-4}
\end{tabular}
\end{center}
\end{table}
}
\clearpage
\newpage
 {
 \setlength\arrayrulewidth{1pt}
\begin{table}
\begin{center}
\begin{tabular}{ccc|l|ccc}
\cline{4-4} \\ [-1em]
\multicolumn{7}{c}{\foreignlanguage{greek}{ευαγγελιον κατα μαρκον} \textbf{(\nospace{6:41})} } \\ \\ [-1em] % Si on veut ajouter les bordures latérales, remplacer {7}{c} par {7}{|c|}
\cline{4-4} \\
\cline{4-4}
&  &  & &  &  & \\ [-0.9em]
&  & 25 & \foreignlanguage{greek}{ινα παρατιθωϲιν αυτοιϲ και τουϲ δυο} & 30 &  &  \\
&  & 31 & \foreignlanguage{greek}{ιχθυαϲ εμεριϲε παϲιν και εφαγον} & 2 & \textbf{42} &  \\
&  & 3 & \foreignlanguage{greek}{παντεϲ και εχορταϲθηϲαν και ηραν} & 2 & \textbf{43} &  \\
&  & 3 & \foreignlanguage{greek}{κλαϲματων \textoverline{ιβ} κοφινουϲ πληρωματα} & 6 &  &  \\
&  & 7 & \foreignlanguage{greek}{και απο των ιχθυων και ηϲαν οι φαγο̅} & 4 & \textbf{44} &  \\
&  & 4 & \foreignlanguage{greek}{τεϲ πεντακειϲχειλιοι ανδρεϲ} & 6 &  &  \\
& \textbf{45} &  & \foreignlanguage{greek}{και ευθυϲ ηναγκαϲεν τουϲ μαθηταϲ} & 5 &  &  \\
&  & 6 & \foreignlanguage{greek}{αυτου ενβηναι ειϲ το πλοιον και προ} & 12 &  &  \\
&  & 12 & \foreignlanguage{greek}{αγειν προϲ βηθϲαιδαν εωϲ αν αυτοϲ} & 17 &  &  \\
&  & 18 & \foreignlanguage{greek}{απολυϲη τον οχλον και αποταξα} & 2 & \textbf{46} &  \\
&  & 2 & \foreignlanguage{greek}{μενοϲ αυτοιϲ απηλθεν ειϲ το οροϲ προϲ} & 8 &  &  \\
&  & 8 & \foreignlanguage{greek}{ευξαϲθαι και οψειαϲ γενομενηϲ} & 3 & \textbf{47} &  \\
&  & 4 & \foreignlanguage{greek}{ην το πλοιον εν μεϲω τηϲ θαλαϲϲηϲ} & 10 &  &  \\
&  & 11 & \foreignlanguage{greek}{και αυτοϲ μονοϲ επι τηϲ γηϲ} & 16 &  &  \\
& \textbf{48} &  & \foreignlanguage{greek}{και ιδων αυτουϲ βαϲανιζομενουϲ} & 4 &  &  \\
&  & 5 & \foreignlanguage{greek}{εν τω ελαυνειν ην γαρ ο ανεμοϲ ε} & 12 &  &  \\
&  & 12 & \foreignlanguage{greek}{ναντιοϲ αυτοιϲ ϲφοδρα και περι τε} & 17 &  &  \\
&  & 17 & \foreignlanguage{greek}{ταρτην φυλακην τηϲ νυκτοϲ ερχε} & 21 &  &  \\
&  & 21 & \foreignlanguage{greek}{τε περιπατων επι τηϲ θαλαϲϲηϲ και} & 26 &  &  \\
&  & 27 & \foreignlanguage{greek}{ηθελεν παρελθειν αυτουϲ} & 29 &  &  \\
& \textbf{49} &  & \foreignlanguage{greek}{οι δε ιδοντεϲ αυτον περιπατουντα} & 5 &  &  \\
&  & 6 & \foreignlanguage{greek}{επι τηϲ θαλαϲϲηϲ φανταϲμα εδοξα̅} & 10 &  &  \\
&  & 11 & \foreignlanguage{greek}{ειναι και ανεκραξαν παντεϲ γαρ αυ} & 3 & \textbf{50} &  \\
&  & 3 & \foreignlanguage{greek}{τον ειδον και εταραχθηϲαν και ευ} & 8 &  &  \\
&  & 8 & \foreignlanguage{greek}{θεωϲ ελαληϲεν μετ αυτων και λεγει} & 13 &  &  \\
&  & 14 & \foreignlanguage{greek}{αυτοιϲ θαρϲιτε μη φοβειϲθαι εγω ειμι} & 19 &  &  \\
& \textbf{51} &  & \foreignlanguage{greek}{και ανεβη προϲ αυτουϲ ειϲ το πλοιον} & 7 &  &  \\
&  & 8 & \foreignlanguage{greek}{και εκοπαϲεν ο ανεμοϲ και εκ περιϲ} & 14 &  &  \\
&  & 14 & \foreignlanguage{greek}{ϲου εν αυτοιϲ εξιϲταντο και εθαυμα} & 19 &  &  \\
&  & 19 & \foreignlanguage{greek}{ζον ου γαρ ϲυνηκον επι τοιϲ αρτοιϲ} & 6 & \textbf{52} &  \\
[0.2em]
\cline{4-4}
\end{tabular}
\end{center}
\end{table}
}
\clearpage
\newpage
 {
 \setlength\arrayrulewidth{1pt}
\begin{table}
\begin{center}
\begin{tabular}{ccc|l|ccc}
\cline{4-4} \\ [-1em]
\multicolumn{7}{c}{\foreignlanguage{greek}{ευαγγελιον κατα μαρκον} \textbf{(\nospace{6:52})} } \\ \\ [-1em] % Si on veut ajouter les bordures latérales, remplacer {7}{c} par {7}{|c|}
\cline{4-4} \\
\cline{4-4}
&  &  & &  &  & \\ [-0.9em]
&  & 7 & \foreignlanguage{greek}{ην γαρ αυτων η καρδια πεπωρωμενη} & 12 &  &  \\
& \textbf{53} &  & \foreignlanguage{greek}{και διαπεραϲαντεϲ ηλθαν επι την γην} & 6 &  &  \\
&  & 7 & \foreignlanguage{greek}{ειϲ γεννηϲαρετ και εξελθοντων αυ} & 3 & \textbf{54} &  \\
&  & 3 & \foreignlanguage{greek}{των εκ του πλοιου ευθυϲ επιγνοντεϲ} & 8 &  &  \\
&  & 9 & \foreignlanguage{greek}{αυτον οι ανδρεϲ του τοπου περιεδρα} & 1 & \textbf{55} &  \\
&  & 1 & \foreignlanguage{greek}{μον ειϲ ολην την περιχωρον εκεινην} & 6 &  &  \\
&  & 7 & \foreignlanguage{greek}{και ηρξαντο επι τοιϲ κρεβαττοιϲ τουϲ κα} & 13 &  &  \\
&  & 13 & \foreignlanguage{greek}{κωϲ εχονταϲ περιφερειν οτι ηκουον} & 17 &  &  \\
&  & 18 & \foreignlanguage{greek}{οτι εϲτιν εκει και οποταν ειϲεπορευ} & 3 & \textbf{56} &  \\
&  & 3 & \foreignlanguage{greek}{οντο ειϲ κωμαϲ η πολειϲ η αγρουϲ εν} & 10 &  &  \\
&  & 11 & \foreignlanguage{greek}{ταιϲ αγοραιϲ ετιθουν τουϲ αϲθενουνταϲ} & 15 &  &  \\
&  & 16 & \foreignlanguage{greek}{και παρεκαλουν αυτον ινα καν του} & 21 &  &  \\
&  & 22 & \foreignlanguage{greek}{κραϲπεδου του ιματιου αυτου αψωνται} & 26 &  &  \\
&  & 27 & \foreignlanguage{greek}{και οϲοι αν ηψαντο αυτου εϲωζοντο} & 32 &  &  \\
& \mygospelchapter &  & \foreignlanguage{greek}{και ϲυναγονται προϲ αυτον οι φαρι} & 6 &  &  \\
&  & 6 & \foreignlanguage{greek}{ϲαιοι και τινεϲ των γραμματεων ελ} & 11 &  &  \\
&  & 11 & \foreignlanguage{greek}{θοντεϲ απο ιεροϲολυμων και ιδοντεϲ} & 2 & \textbf{2} &  \\
&  & 3 & \foreignlanguage{greek}{τινεϲ των μαθητων αυτου τιναϲ κοι} & 8 &  &  \\
&  & 8 & \foreignlanguage{greek}{ναιϲ χερϲιν τουτ εϲτιν ανιπτοιϲ εϲθι} & 13 &  &  \\
&  & 13 & \foreignlanguage{greek}{ονταϲ τουϲ αρτουϲ εμεμψαντο} & 16 &  &  \\
& \textbf{3} &  & \foreignlanguage{greek}{οι γαρ φαριϲαιοι και παντεϲ οι ιουδαιοι} & 7 &  &  \\
&  & 8 & \foreignlanguage{greek}{εαν μη πυκνα νιψωνται ταϲ χειραϲ} & 13 &  &  \\
&  & 14 & \foreignlanguage{greek}{ουκ αιϲθιουϲιν κρατουντεϲ την πα} & 18 &  &  \\
&  & 18 & \foreignlanguage{greek}{ραδοϲιν των πρεϲβυτερων και απ α} & 3 & \textbf{4} &  \\
&  & 3 & \foreignlanguage{greek}{γοραϲ δε οταν ελθωϲιν εαν μη βαπτι} & 9 &  &  \\
&  & 9 & \foreignlanguage{greek}{ϲωνται ουκ αιϲθιουϲιν και αλλα πολ} & 14 &  &  \\
&  & 14 & \foreignlanguage{greek}{λα εϲτιν α παρελαβον κρατιν βαπτι} & 19 &  &  \\
&  & 19 & \foreignlanguage{greek}{ϲμουϲ ποτηριων και ξεϲτων και χαλ} & 24 &  &  \\
&  & 24 & \foreignlanguage{greek}{κιων και κλεινων επιτα ερωτω} & 2 & \textbf{5} &  \\
&  & 2 & \foreignlanguage{greek}{ϲιν αυτον οι φαριϲαιοι και οι γραμματιϲ} & 8 &  &  \\
[0.2em]
\cline{4-4}
\end{tabular}
\end{center}
\end{table}
}
\clearpage
\newpage
 {
 \setlength\arrayrulewidth{1pt}
\begin{table}
\begin{center}
\begin{tabular}{ccc|l|ccc}
\cline{4-4} \\ [-1em]
\multicolumn{7}{c}{\foreignlanguage{greek}{ευαγγελιον κατα μαρκον} \textbf{(\nospace{7:5})} } \\ \\ [-1em] % Si on veut ajouter les bordures latérales, remplacer {7}{c} par {7}{|c|}
\cline{4-4} \\
\cline{4-4}
&  &  & &  &  & \\ [-0.9em]
&  & 9 & \foreignlanguage{greek}{λεγοντεϲ δια τι οι μαθηται ϲου ου πε} & 16 &  &  \\
&  & 16 & \foreignlanguage{greek}{ριπατουϲιν κατα την παραδοϲιν των} & 20 &  &  \\
&  & 21 & \foreignlanguage{greek}{πρεϲβυτερων αλλα κοιναιϲ και ανιπτοιϲ χερ} & 26 &  &  \\
&  & 26 & \foreignlanguage{greek}{ϲιν αιϲθιουϲιν τον αρτον} & 29 &  &  \\
& \textbf{6} &  & \foreignlanguage{greek}{ο δε αποκριθειϲ ειπεν αυτοιϲ οτι κα} & 7 &  &  \\
&  & 7 & \foreignlanguage{greek}{λωϲ επροεφητευϲεν ηϲαιαϲ περι υμω̅} & 11 &  &  \\
&  & 12 & \foreignlanguage{greek}{των υποκριτων ωϲ γεγραπται ουτοϲ} & 16 &  &  \\
&  & 17 & \foreignlanguage{greek}{ο λαοϲ τοιϲ χειλεϲιν με αγαπα η δε καρ} & 25 &  &  \\
&  & 25 & \foreignlanguage{greek}{δια αυτων πορρω εχει απ εμου ματην} & 1 & \textbf{7} &  \\
&  & 2 & \foreignlanguage{greek}{δε ϲεβονται με διδαϲκοντεϲ διδαϲκαλιαϲ} & 6 &  &  \\
&  & 7 & \foreignlanguage{greek}{ενταλματα \textoverline{ανων} αφεντεϲ την εντο} & 3 & \textbf{8} &  \\
&  & 3 & \foreignlanguage{greek}{λην του \textoverline{θυ} κρατιτε την παραδοϲιν} & 8 &  &  \\
&  & 9 & \foreignlanguage{greek}{των \textoverline{ανων} και ελεγεν αυτοιϲ καλωϲ} & 4 & \textbf{9} &  \\
&  & 5 & \foreignlanguage{greek}{αθετειτε την εντολην του \textoverline{θυ} ινα τη̅} & 11 &  &  \\
&  & 12 & \foreignlanguage{greek}{παραδοϲιν υμων ϲτηϲηται μωυϲηϲ} & 1 & \textbf{10} &  \\
&  & 2 & \foreignlanguage{greek}{γαρ ειπεν τιμα τον \textoverline{πρα} ϲου και την \textoverline{μρα}} & 10 &  &  \\
&  & 11 & \foreignlanguage{greek}{ϲου και ο αθετων \textoverline{πρα} η \textoverline{μρα} θανατω} & 18 &  &  \\
&  & 19 & \foreignlanguage{greek}{τελευτατω υμειϲ δε λεγεται εαν} & 4 & \textbf{11} &  \\
&  & 5 & \foreignlanguage{greek}{ειπη \textoverline{ανοϲ} τω \textoverline{πρι} η τη \textoverline{μρι} κορβαν ο εϲτι̅} & 14 &  &  \\
&  & 15 & \foreignlanguage{greek}{δωρον ο αν εξ εμου ωφεληθηϲ και ου} & 2 & \textbf{12} &  \\
&  & 2 & \foreignlanguage{greek}{κετι αφιεται αυτον ουδεν ποιηϲαι τω} & 7 &  &  \\
&  & 8 & \foreignlanguage{greek}{\textoverline{πρι} η τη \textoverline{μρι} ακυρουντεϲ τον λογον} & 3 & \textbf{13} &  \\
&  & 4 & \foreignlanguage{greek}{την εντολην του \textoverline{θυ} τη παραδοϲι υμω̅} & 10 &  &  \\
&  & 11 & \foreignlanguage{greek}{η παρεδοτε και προϲκαλεϲαμε} & 2 & \textbf{14} &  \\
&  & 2 & \foreignlanguage{greek}{νοϲ παντα τον οχλον ελεγεν αυτοιϲ} & 7 &  &  \\
&  & 8 & \foreignlanguage{greek}{ακουεται μου παντεϲ και ϲυνιεται} & 12 &  &  \\
& \textbf{15} &  & \foreignlanguage{greek}{ουδεν εϲτιν εξωθεν του \textoverline{ανου} ειϲπο} & 6 &  &  \\
&  & 6 & \foreignlanguage{greek}{ρευομενον ειϲ αυτον ο δυναται αυτον} & 12 &  &  \\
&  & 13 & \foreignlanguage{greek}{κοινωϲαι αλλα τα εκ του \textoverline{ανου} εκπο} & 19 &  &  \\
&  & 19 & \foreignlanguage{greek}{ρευομενα εκεινα εϲτιν τα κοινουντα} & 23 &  &  \\
[0.2em]
\cline{4-4}
\end{tabular}
\end{center}
\end{table}
}
\clearpage
\newpage
 {
 \setlength\arrayrulewidth{1pt}
\begin{table}
\begin{center}
\begin{tabular}{ccc|l|ccc}
\cline{4-4} \\ [-1em]
\multicolumn{7}{c}{\foreignlanguage{greek}{ευαγγελιον κατα μαρκον} \textbf{(\nospace{7:15})} } \\ \\ [-1em] % Si on veut ajouter les bordures latérales, remplacer {7}{c} par {7}{|c|}
\cline{4-4} \\
\cline{4-4}
&  &  & &  &  & \\ [-0.9em]
&  & 24 & \foreignlanguage{greek}{τον \textoverline{ανον} ει τιϲ εχει ωτα ακουειν ακουετω} & 6 & \textbf{16} &  \\
& \textbf{17} &  & \foreignlanguage{greek}{και οτε ειϲηλθον ειϲ οικον απο του οχλου} & 8 &  &  \\
&  & 9 & \foreignlanguage{greek}{επηρωτων αυτον οι μαθηται αυτου} & 13 &  &  \\
&  & 14 & \foreignlanguage{greek}{περι τηϲ παραβοληϲ και λεγει αυτοιϲ} & 3 & \textbf{18} &  \\
&  & 4 & \foreignlanguage{greek}{ουτωϲ και υμειϲ αϲυνετοι εϲται ου νο} & 10 &  &  \\
&  & 10 & \foreignlanguage{greek}{ειτε οτι παν το εξωθεν ειϲπορευομε} & 15 &  &  \\
&  & 15 & \foreignlanguage{greek}{νον ειϲ τον \textoverline{ανον} ου δυναται αυτον} & 21 &  &  \\
&  & 22 & \foreignlanguage{greek}{κοινωϲε οτι ουκ ειϲπορευεται αυτου} & 4 & \textbf{19} &  \\
&  & 5 & \foreignlanguage{greek}{ειϲ την διανοιαν αλλα ειϲ την κοιλια̅} & 11 &  &  \\
&  & 12 & \foreignlanguage{greek}{και ειϲ τον αφεδρωνα χωρει καθα} & 17 &  &  \\
&  & 17 & \foreignlanguage{greek}{ριζων παντα τα βρωματα ελεγεν} & 1 & \textbf{20} &  \\
&  & 2 & \foreignlanguage{greek}{δε οτι το εκ του \textoverline{ανου} εκπορευομενο̅} & 8 &  &  \\
&  & 9 & \foreignlanguage{greek}{εκεινο κοινοι τον \textoverline{ανον} εϲωθεν γαρ} & 2 & \textbf{21} &  \\
&  & 3 & \foreignlanguage{greek}{εκ τηϲ καρδιαϲ των \textoverline{ανων} οι διαλογι} & 9 &  &  \\
&  & 9 & \foreignlanguage{greek}{ϲμοι κακοι εκπορευονται μοιχιαι} & 12 &  &  \\
&  & 13 & \foreignlanguage{greek}{πορνιαι κλοπαι φονοϲ πλεονεξια} & 1 & \textbf{22} &  \\
&  & 2 & \foreignlanguage{greek}{πονηρια δολοϲ αϲελγεια οφθαλ} & 5 &  &  \\
&  & 5 & \foreignlanguage{greek}{μοϲ πονηροϲ βλαϲφημια υπερηφα} & 8 &  &  \\
&  & 8 & \foreignlanguage{greek}{νια αφροϲυνη παντα τα πονηρα ε} & 4 & \textbf{23} &  \\
&  & 4 & \foreignlanguage{greek}{ϲωθεν εκπορευετε και κοινοι τον} & 8 &  &  \\
&  & 9 & \foreignlanguage{greek}{\textoverline{ανον} και αναϲταϲ απηλθεν ειϲ τα} & 5 & \textbf{24} &  \\
&  & 6 & \foreignlanguage{greek}{ορια τυρου και ειϲελθων ειϲ την οι} & 12 &  &  \\
&  & 12 & \foreignlanguage{greek}{κειαν ουδενα ηθελεν γνωναι και ου} & 17 &  &  \\
&  & 17 & \foreignlanguage{greek}{κ ηδυνηθη λαθειν ακουϲαϲα γαρ} & 2 & \textbf{25} &  \\
&  & 3 & \foreignlanguage{greek}{γυνη περι αυτου ηϲ ειχεν το θυγατρι} & 9 &  &  \\
&  & 9 & \foreignlanguage{greek}{ον εν \textoverline{πνι} ακαθαρτω ελθουϲα προϲεπε} & 14 &  &  \\
&  & 14 & \foreignlanguage{greek}{ϲεν προϲ τουϲ ποδαϲ αυτου η δε γυνη} & 3 & \textbf{26} &  \\
&  & 4 & \foreignlanguage{greek}{ην ελληνιϲ ϲυραφοινιϲϲα τω γενει} & 8 &  &  \\
&  & 9 & \foreignlanguage{greek}{και ηρωτα αυτον ινα το δαιμονιον εκ} & 15 &  &  \\
&  & 15 & \foreignlanguage{greek}{βαλη εκ τηϲ θυγατροϲ αυτηϲ} & 19 &  &  \\
[0.2em]
\cline{4-4}
\end{tabular}
\end{center}
\end{table}
}
\clearpage
\newpage
 {
 \setlength\arrayrulewidth{1pt}
\begin{table}
\begin{center}
\begin{tabular}{ccc|l|ccc}
\cline{4-4} \\ [-1em]
\multicolumn{7}{c}{\foreignlanguage{greek}{ευαγγελιον κατα μαρκον} \textbf{(\nospace{7:27})} } \\ \\ [-1em] % Si on veut ajouter les bordures latérales, remplacer {7}{c} par {7}{|c|}
\cline{4-4} \\
\cline{4-4}
&  &  & &  &  & \\ [-0.9em]
& \textbf{27} &  & \foreignlanguage{greek}{ο δε \textoverline{ιϲ} ειπεν αυτη αφεϲ πρωτον χορτα} & 8 &  &  \\
&  & 8 & \foreignlanguage{greek}{ϲθηναι τα τεκνα ου γαρ καλον εϲτιν λα} & 15 &  &  \\
&  & 15 & \foreignlanguage{greek}{βειν τον αρτον των τεκνων και βαλει̅} & 21 &  &  \\
&  & 22 & \foreignlanguage{greek}{τοιϲ κυναριοιϲ η δε απεκριθη αυτω} & 4 & \textbf{28} &  \\
&  & 5 & \foreignlanguage{greek}{λεγουϲα \textoverline{κε} και τα κυναρια υποκατω} & 10 &  &  \\
&  & 11 & \foreignlanguage{greek}{τηϲ τραπεζηϲ εϲθιουϲιν απο των ψιχων} & 16 &  &  \\
&  & 17 & \foreignlanguage{greek}{των παιδιων και ειπεν αυτη} & 3 & \textbf{29} &  \\
&  & 4 & \foreignlanguage{greek}{δια τουτον τον λογον υπαγε εξεληλυ} & 9 &  &  \\
&  & 9 & \foreignlanguage{greek}{θεν το δαιμονιον εκ τηϲ θυγατροϲ ϲου} & 15 &  &  \\
& \textbf{30} &  & \foreignlanguage{greek}{και απελθουϲα ειϲ τον οικον ευρεν το} & 7 &  &  \\
&  & 8 & \foreignlanguage{greek}{δαιμονιον εξεληλυθοϲ και την θυγα} & 12 &  &  \\
&  & 12 & \foreignlanguage{greek}{τερα βεβλημενην επι τηϲ κλινηϲ} & 16 &  &  \\
& \textbf{31} &  & \foreignlanguage{greek}{και παλιν εξελθων εκ των οριων τυ} & 7 &  &  \\
&  & 7 & \foreignlanguage{greek}{ρου και ϲιδωνοϲ ηλθεν ειϲ την θαλαϲϲα̅} & 13 &  &  \\
&  & 14 & \foreignlanguage{greek}{τηϲ γαλιλαιαϲ ανα μεϲον των οριων ειϲ} & 20 &  &  \\
&  & 21 & \foreignlanguage{greek}{την δεκαπολιν και φερουϲιν αυτω} & 3 & \textbf{32} &  \\
&  & 4 & \foreignlanguage{greek}{κωφον και μογγιλαλον και παρακα} & 8 &  &  \\
&  & 8 & \foreignlanguage{greek}{λουϲιν αυτον ινα επιθη αυτω την χειρα} & 14 &  &  \\
& \textbf{33} &  & \foreignlanguage{greek}{και προϲλαβομενοϲ αυτον απο του οχλου} & 6 &  &  \\
&  & 7 & \foreignlanguage{greek}{κατ ιδιαν εβαλε δακτυλουϲ αυτου πτυϲαϲ} & 12 &  &  \\
&  & 13 & \foreignlanguage{greek}{ειϲ τα ωτα και ηψατο τηϲ γλωϲ} & 19 &  &  \\
&  & 19 & \foreignlanguage{greek}{ϲαϲ αυτου και αναβλεψαϲ ειϲ τον ου} & 5 & \textbf{34} &  \\
&  & 5 & \foreignlanguage{greek}{ρανον εϲτεναξεν και λεγει αυτω} & 9 &  &  \\
&  & 10 & \foreignlanguage{greek}{εφεθθα ο εϲτιν διανυχθητι και ευ} & 2 & \textbf{35} &  \\
&  & 2 & \foreignlanguage{greek}{θεωϲ διηνυγηϲαν αυτου αι ακοαι και} & 7 &  &  \\
&  & 8 & \foreignlanguage{greek}{ελυθη ο δεϲμοϲ τηϲ γλωϲϲηϲ αυτου} & 13 &  &  \\
&  & 14 & \foreignlanguage{greek}{και ελαλει ορθωϲ και διεϲτιλατο αυ} & 3 & \textbf{36} &  \\
&  & 3 & \foreignlanguage{greek}{τοιϲ ινα μηδενι λεγωϲιν} & 6 &  &  \\
&  & 7 & \foreignlanguage{greek}{οϲω δε αυτοιϲ διεϲτελλετο μαλλον} & 11 &  &  \\
&  & 12 & \foreignlanguage{greek}{περιϲϲοτερον εκηρυϲϲον και υπερ πε} & 3 & \textbf{37} &  \\
[0.2em]
\cline{4-4}
\end{tabular}
\end{center}
\end{table}
}
\clearpage
\newpage
 {
 \setlength\arrayrulewidth{1pt}
\begin{table}
\begin{center}
\begin{tabular}{ccc|l|ccc}
\cline{4-4} \\ [-1em]
\multicolumn{7}{c}{\foreignlanguage{greek}{ευαγγελιον κατα μαρκον} \textbf{(\nospace{7:37})} } \\ \\ [-1em] % Si on veut ajouter les bordures latérales, remplacer {7}{c} par {7}{|c|}
\cline{4-4} \\
\cline{4-4}
&  &  & &  &  & \\ [-0.9em]
&  & 3 & \foreignlanguage{greek}{ριϲϲω εξεπληϲϲοντο λεγοντεϲ κα} & 6 &  &  \\
&  & 6 & \foreignlanguage{greek}{λωϲ παντα πεποιηκεν και τουϲ κω} & 11 &  &  \\
&  & 11 & \foreignlanguage{greek}{φουϲ πεποιηκεν ακουειν και λαλειν} & 15 &  &  \\
& \mygospelchapter &  & \foreignlanguage{greek}{εν εκειναιϲ δε ταιϲ ημεραιϲ παλιν πολ} & 7 &  &  \\
&  & 7 & \foreignlanguage{greek}{λου οχλου οντοϲ και μη εχοντων αυτω̅} & 13 &  &  \\
&  & 14 & \foreignlanguage{greek}{τι φαγωϲιν προϲκαλεϲαμενοϲ τουϲ} & 17 &  &  \\
&  & 18 & \foreignlanguage{greek}{μαθηταϲ αυτου λεγει ϲπλαγχνιζομε} & 1 & \textbf{2} &  \\
&  & 2 & \foreignlanguage{greek}{επι τω οχλω οτι ηδη ημερε τριϲ προϲμε} & 9 &  &  \\
&  & 9 & \foreignlanguage{greek}{νουϲιν μοι και ουκ εχουϲιν τι φαγωϲι̅} & 15 &  &  \\
& \textbf{3} &  & \foreignlanguage{greek}{και εαν απολυϲω αυτουϲ νηϲτιϲ εωϲ} & 6 &  &  \\
&  & 7 & \foreignlanguage{greek}{ειϲ οικον αυτων εκλυθηϲοντε εν τη οδω} & 13 &  &  \\
&  & 14 & \foreignlanguage{greek}{και τινεϲ αυτων απο μακροθεν ηκαϲι̅} & 19 &  &  \\
& \textbf{4} &  & \foreignlanguage{greek}{και απεκριθηϲαν αυτω οι μαθηται λε} & 6 &  &  \\
&  & 6 & \foreignlanguage{greek}{γοντεϲ ποθεν ωδε δυναϲαι αυτουϲ} & 10 &  &  \\
&  & 11 & \foreignlanguage{greek}{χορταϲαι αρτων επ ερημειαϲ} & 14 &  &  \\
& \textbf{5} &  & \foreignlanguage{greek}{ο δε ηρωτηϲεν αυτουϲ ποϲουϲ ωδε αρ} & 7 &  &  \\
&  & 7 & \foreignlanguage{greek}{τουϲ εχετε οι δε ειπαν επτα κα πα} & 2 & \textbf{6} &  \\
&  & 2 & \foreignlanguage{greek}{ρηγγειλεν τω οχλω αναπεϲιν επι τηϲ γηϲ} & 8 &  &  \\
&  & 9 & \foreignlanguage{greek}{και λαβων τουϲ \textoverline{ζ} αρτουϲ ευχαριϲτηϲαϲ} & 14 &  &  \\
&  & 15 & \foreignlanguage{greek}{εκλαϲεν και εδιδου αυτοιϲ ινα παραθωϲι̅} & 20 &  &  \\
&  & 21 & \foreignlanguage{greek}{και παρεθηκαν τω οχλω και ειχαν ιχθυ} & 3 & \textbf{7} &  \\
&  & 3 & \foreignlanguage{greek}{δια ολειγα και αυτα ευλογηϲαϲ ειπεν} & 8 &  &  \\
&  & 9 & \foreignlanguage{greek}{παραθειναι και εφαγον και εχορτα} & 4 & \textbf{8} &  \\
&  & 4 & \foreignlanguage{greek}{ϲθηϲαν και ηραν περιϲευματα \textoverline{ζ} ϲπυ} & 9 &  &  \\
&  & 9 & \foreignlanguage{greek}{ριδαϲ πληρειϲ ηϲαν δε οι φαγοντεϲ} & 4 & \textbf{9} &  \\
&  & 5 & \foreignlanguage{greek}{ωϲ τετρακειϲχειλιοι ϗ απελυϲεν αυτουϲ} & 9 &  &  \\
& \textbf{10} &  & \foreignlanguage{greek}{και ενβαϲ ευθυϲ ειϲ πλοιον μετα των μα} & 8 &  &  \\
&  & 8 & \foreignlanguage{greek}{θητων αυτου και ηλθεν προϲ το οροϲ} & 14 &  &  \\
&  & 15 & \foreignlanguage{greek}{δαλμουναι και εξηλθον οι φαριϲαιοι} & 4 & \textbf{11} &  \\
&  & 5 & \foreignlanguage{greek}{και ηρξαντο ϲυνζητειν αυτω ζητου̅} & 9 &  &  \\
[0.2em]
\cline{4-4}
\end{tabular}
\end{center}
\end{table}
}
\clearpage
\newpage
 {
 \setlength\arrayrulewidth{1pt}
\begin{table}
\begin{center}
\begin{tabular}{ccc|l|ccc}
\cline{4-4} \\ [-1em]
\multicolumn{7}{c}{\foreignlanguage{greek}{ευαγγελιον κατα μαρκον} \textbf{(\nospace{8:11})} } \\ \\ [-1em] % Si on veut ajouter les bordures latérales, remplacer {7}{c} par {7}{|c|}
\cline{4-4} \\
\cline{4-4}
&  &  & &  &  & \\ [-0.9em]
&  & 9 & \foreignlanguage{greek}{τεϲ απ αυτου ϲημιον εκ του ουρανου πειρα} & 16 &  &  \\
&  & 16 & \foreignlanguage{greek}{ζοντεϲ αυτον και αναϲτεναξαϲ τω \textoverline{πνι}} & 4 & \textbf{12} &  \\
&  & 5 & \foreignlanguage{greek}{λεγει τι η γενεα αυτη ϲημιον επιζητει} & 11 &  &  \\
&  & 12 & \foreignlanguage{greek}{αμην ου δοθηϲετε ταυτη τη γενεα ϲημιον} & 18 &  &  \\
& \textbf{13} &  & \foreignlanguage{greek}{και αφειϲ αυτουϲ παλιν ενβαϲ ειϲ το πλοιο̅} & 8 &  &  \\
&  & 9 & \foreignlanguage{greek}{απηλθεν ειϲ το περαν και απελθοντεϲ} & 2 & \textbf{14} &  \\
&  & 3 & \foreignlanguage{greek}{οι μαθηται αυτου λαβειν αρτουϲ ενα μονο} & 9 &  &  \\
&  & 10 & \foreignlanguage{greek}{εχοντεϲ αρτον μεθ εαυτων εν τω πλοιω} & 16 &  &  \\
& \textbf{15} &  & \foreignlanguage{greek}{και διεϲτελλετο αυτοιϲ λεγων ορατε βλε} & 6 &  &  \\
&  & 6 & \foreignlanguage{greek}{πεται απο τηϲ ζυμηϲ των φαριοεων και} & 12 &  &  \\
&  & 13 & \foreignlanguage{greek}{απο τηϲ ζυμηϲ των ηρωδιανων οι δε δι} & 3 & \textbf{16} &  \\
&  & 3 & \foreignlanguage{greek}{ελογιζοντο προϲ αλληλουϲ οτι αρτουϲ ου} & 8 &  &  \\
&  & 8 & \foreignlanguage{greek}{κ εχουϲιν και γνουϲ ο \textoverline{ιϲ} λεγει αυτοιϲ} & 6 & \textbf{17} &  \\
&  & 7 & \foreignlanguage{greek}{τι διαλογειζεϲθαι εν εαυτοιϲ ολιγοπιϲτοι} & 11 &  &  \\
&  & 12 & \foreignlanguage{greek}{οτι αρτουϲ ουκ εχεται ουπω νοειτε ου} & 18 &  &  \\
&  & 18 & \foreignlanguage{greek}{δε ϲυνιεται πεπωρωμενην εχεται την} & 22 &  &  \\
&  & 23 & \foreignlanguage{greek}{καρδιαν υμων οφθαλμουϲ εχετε και} & 3 & \textbf{18} &  \\
&  & 4 & \foreignlanguage{greek}{ου βλεπουϲιν και ωτα εχεται και ουκ α} & 11 &  &  \\
&  & 11 & \foreignlanguage{greek}{κουεται ου μνημονευεται οτε τουϲ πε̅} & 3 & \textbf{19} &  \\
&  & 3 & \foreignlanguage{greek}{τε αρτουϲ εκλαϲα ειϲ τουϲ πεντακιϲχειλι} & 8 &  &  \\
&  & 8 & \foreignlanguage{greek}{ουϲ ποϲουϲ κοφινουϲ πληρειϲ κλαϲματω̅} & 13 &  &  \\
&  & 14 & \foreignlanguage{greek}{ηρατε λεγουϲιν αυτω δωδεκα οτε δε} & 2 & \textbf{20} &  \\
&  & 3 & \foreignlanguage{greek}{τουϲ \textoverline{ζ} αρτουϲ ειϲ τουϲ τετρακιϲχειλιουϲ} & 8 &  &  \\
&  & 9 & \foreignlanguage{greek}{ποϲων ϲπυριδων πληρωματα ηρατε} & 12 &  &  \\
&  & 13 & \foreignlanguage{greek}{οι δε ειπαν \textoverline{ζ} και λεγει αυτοιϲ πωϲ ου} & 5 & \textbf{21} &  \\
&  & 5 & \foreignlanguage{greek}{πω ϲυνιεται και ερχονται ειϲ βηθαιδα̅} & 4 & \textbf{22} &  \\
&  & 5 & \foreignlanguage{greek}{και φερουϲιν αυτω τυφλον και παρακα} & 10 &  &  \\
&  & 10 & \foreignlanguage{greek}{λουϲιν αυτον ινα αυτου αψηται και ε} & 2 & \textbf{23} &  \\
&  & 2 & \foreignlanguage{greek}{πιλαβομενοϲ τηϲ χειροϲ αυτου εξηγαγε̅} & 6 &  &  \\
&  & 7 & \foreignlanguage{greek}{αυτον εξω τηϲ κωμηϲ και ενπτυϲαϲ ειϲ} & 13 &  &  \\
[0.2em]
\cline{4-4}
\end{tabular}
\end{center}
\end{table}
}
\clearpage
\newpage
 {
 \setlength\arrayrulewidth{1pt}
\begin{table}
\begin{center}
\begin{tabular}{ccc|l|ccc}
\cline{4-4} \\ [-1em]
\multicolumn{7}{c}{\foreignlanguage{greek}{ευαγγελιον κατα μαρκον} \textbf{(\nospace{8:23})} } \\ \\ [-1em] % Si on veut ajouter les bordures latérales, remplacer {7}{c} par {7}{|c|}
\cline{4-4} \\
\cline{4-4}
&  &  & &  &  & \\ [-0.9em]
&  & 14 & \foreignlanguage{greek}{τα ομματα αυτου και επιθειϲ ταϲ χειραϲ} & 20 &  &  \\
&  & 21 & \foreignlanguage{greek}{επ αυτω ηρωτα αυτον ει βλεπει ο δε α} & 3 & \textbf{24} &  \\
&  & 3 & \foreignlanguage{greek}{ναβλεψαϲ λεγει βλεπω τουϲ \textoverline{ανουϲ} ωϲ δε̅} & 9 &  &  \\
&  & 9 & \foreignlanguage{greek}{δρα περιπατουνταϲ ειτα παλιν επεθηκε̅} & 3 & \textbf{25} &  \\
&  & 4 & \foreignlanguage{greek}{ταϲ χειραϲ αυτου επι τουϲ οφθαλμουϲ αυ} & 10 &  &  \\
&  & 10 & \foreignlanguage{greek}{του και διεβλεψεν και απεκατεϲταθη} & 14 &  &  \\
&  & 15 & \foreignlanguage{greek}{και ενεβλεπεν παντα τηλαυγωϲ και α} & 2 & \textbf{26} &  \\
&  & 2 & \foreignlanguage{greek}{πεϲτιλεν αυτον ειϲ τον οικον αυτου λεγω̅} & 8 &  &  \\
&  & 9 & \foreignlanguage{greek}{μη ειϲ την κωμην ειϲελθηϲ} & 13 &  &  \\
& \textbf{27} &  & \foreignlanguage{greek}{και εξηλθεν ο \textoverline{ιϲ} και οι μαθηται αυτου} & 8 &  &  \\
&  & 9 & \foreignlanguage{greek}{ειϲ ταϲ κωμαϲ καιϲαριαϲ τηϲ φιλιππου} & 14 &  &  \\
&  & 15 & \foreignlanguage{greek}{και εν τη οδω τουϲ μαθηταϲ αυτου επη} & 22 &  &  \\
&  & 22 & \foreignlanguage{greek}{ρωτα λεγων αυτοιϲ τινα με λεγουϲιν} & 27 &  &  \\
&  & 28 & \foreignlanguage{greek}{οι \textoverline{ανοι} ειναι οι δε απεκριθηϲαν λεγον} & 4 & \textbf{28} &  \\
&  & 4 & \foreignlanguage{greek}{τεϲ οι μεν ιωαννην τον βαπτιϲτην} & 9 &  &  \\
&  & 10 & \foreignlanguage{greek}{αλλοι δε ηλιαν αλλοι δε ενα των προφη} & 17 &  &  \\
&  & 17 & \foreignlanguage{greek}{των λεγει αυτοιϲ υμειϲ δε τινα με} & 6 & \textbf{29} &  \\
&  & 7 & \foreignlanguage{greek}{λεγεται αποκριθειϲ δε ο πετροϲ λεγει} & 12 &  &  \\
&  & 13 & \foreignlanguage{greek}{αυτω ϲυ ει ο \textoverline{χϲ} ο υιοϲ του \textoverline{θυ} του ζωντοϲ} & 23 &  &  \\
& \textbf{30} &  & \foreignlanguage{greek}{και επετιμηϲεν αυτοιϲ ινα μηδενι λεγου} & 6 &  &  \\
&  & 6 & \foreignlanguage{greek}{ϲιν περι αυτου και απο τοτε ηρξατο δι} & 5 & \textbf{31} &  \\
&  & 5 & \foreignlanguage{greek}{δαϲκειν αυτουϲ οτι δει τον υιον του} & 11 &  &  \\
&  & 12 & \foreignlanguage{greek}{\textoverline{ανου} πολλα παθειν και αποδοκιμα} & 16 &  &  \\
&  & 16 & \foreignlanguage{greek}{ϲθηναι απο των πρεϲβυτερων και τω̅} & 21 &  &  \\
&  & 22 & \foreignlanguage{greek}{αρχιερεων και γραμματεων και απο} & 26 &  &  \\
&  & 26 & \foreignlanguage{greek}{κτανθηναι και τη τριτη ημερα αναϲτη} & 31 &  &  \\
&  & 31 & \foreignlanguage{greek}{ναι και παρηϲια τον λογον ελαλει} & 5 & \textbf{32} &  \\
&  & 6 & \foreignlanguage{greek}{και προϲλαβομενοϲ αυτον ο πετροϲ ηρ} & 11 &  &  \\
&  & 11 & \foreignlanguage{greek}{ξατο επιτιμαν αυτω ο δε επιϲτραφειϲ} & 3 & \textbf{33} &  \\
&  & 4 & \foreignlanguage{greek}{και ιδωϲ τουϲ μαθηταϲ αυτου επετιμη} & 9 &  &  \\
[0.2em]
\cline{4-4}
\end{tabular}
\end{center}
\end{table}
}
\clearpage
\newpage
 {
 \setlength\arrayrulewidth{1pt}
\begin{table}
\begin{center}
\begin{tabular}{ccc|l|ccc}
\cline{4-4} \\ [-1em]
\multicolumn{7}{c}{\foreignlanguage{greek}{ευαγγελιον κατα μαρκον} \textbf{(\nospace{8:33})} } \\ \\ [-1em] % Si on veut ajouter les bordures latérales, remplacer {7}{c} par {7}{|c|}
\cline{4-4} \\
\cline{4-4}
&  &  & &  &  & \\ [-0.9em]
&  & 9 & \foreignlanguage{greek}{ϲεν τω πετρω λεγων υπαγε οπιϲω μου} & 15 &  &  \\
&  & 16 & \foreignlanguage{greek}{ϲατανα οτι ου φρονειϲ τα του \textoverline{θυ} αλλα τα} & 24 &  &  \\
&  & 25 & \foreignlanguage{greek}{των \textoverline{ανων} και προϲκαλεϲαμενοϲ το̅} & 3 & \textbf{34} &  \\
&  & 4 & \foreignlanguage{greek}{οχλον ϲυν τοιϲ μαθηταιϲ αυτου ειπεν} & 9 &  &  \\
&  & 10 & \foreignlanguage{greek}{ει τιϲ θελει οπιϲω μου ακολουθειν απαρ} & 16 &  &  \\
&  & 16 & \foreignlanguage{greek}{νηϲαϲθω εαυτον και αραϲ τον ϲταυρον} & 21 &  &  \\
&  & 22 & \foreignlanguage{greek}{ακολουθειτω μοι οϲ γαρ αν θελη την ψυ} & 6 & \textbf{35} &  \\
&  & 6 & \foreignlanguage{greek}{χην αυτου ϲωϲαι απολεϲει αυτην} & 10 &  &  \\
&  & 11 & \foreignlanguage{greek}{οϲ δ αν απολεϲη την εαυτου ψυχην ενε} & 18 &  &  \\
&  & 18 & \foreignlanguage{greek}{κεν εμου και του ευαγγελιου ϲωϲει αυτη̅} & 24 &  &  \\
& \textbf{36} &  & \foreignlanguage{greek}{τι γαρ ωφελει τον \textoverline{ανον} εαν κερδηϲη} & 7 &  &  \\
&  & 8 & \foreignlanguage{greek}{τον κοϲμον ολον και ζημιωθη την εαυ} & 14 &  &  \\
&  & 14 & \foreignlanguage{greek}{του ψυχην τι γαρ δωϲει \textoverline{ανοϲ} ανταλλα} & 5 & \textbf{37} &  \\
&  & 5 & \foreignlanguage{greek}{γμα τηϲ ψυχηϲ αυτου οϲ γαρ αν επεϲχυ̅} & 4 & \textbf{38} &  \\
&  & 4 & \foreignlanguage{greek}{θη με και τουϲ εμουϲ εν τη γενεα τη μοι} & 13 &  &  \\
&  & 13 & \foreignlanguage{greek}{χαλιδει και αμαρτωλω και ο υιοϲ του} & 19 &  &  \\
&  & 20 & \foreignlanguage{greek}{\textoverline{ανου} επεϲχυνθηϲεται αυτον οταν ελθη} & 24 &  &  \\
&  & 25 & \foreignlanguage{greek}{εν τη δοξη του \textoverline{πρϲ} αυτου και των αγγε} & 33 &  &  \\
&  & 33 & \foreignlanguage{greek}{λων των αγιων και ελεγεν αυτοιϲ} & 3 & \mygospelchapter &  \\
&  & 4 & \foreignlanguage{greek}{αμην λεγω υμιν οτι ειϲιν τινεϲ των ωδε} & 11 &  &  \\
&  & 12 & \foreignlanguage{greek}{εϲτηκοτων οιτινεϲ ου μη γευϲωνται} & 16 &  &  \\
&  & 17 & \foreignlanguage{greek}{θανατου εωϲ ιδωϲιν την βαϲιλειαν του} & 22 &  &  \\
&  & 23 & \foreignlanguage{greek}{\textoverline{θυ} εληλυθυειαν εν δυναμει} & 26 &  &  \\
& \textbf{2} &  & \foreignlanguage{greek}{και μεθ ημεραϲ εξ παραλαμβανει ο \textoverline{ιϲ}} & 7 &  &  \\
&  & 8 & \foreignlanguage{greek}{τον πετρον και τον ιακωβον και τον ι} & 15 &  &  \\
&  & 15 & \foreignlanguage{greek}{ωαννην και αναφερει αυτουϲ ειϲ οροϲ} & 20 &  &  \\
&  & 21 & \foreignlanguage{greek}{υψηλον καθ ιδιαν μονουϲ και εν τω} & 27 &  &  \\
&  & 28 & \foreignlanguage{greek}{προϲευχεϲθαι αυτουϲ μετεμορφωθη} & 30 &  &  \\
&  & 31 & \foreignlanguage{greek}{ο \textoverline{ιϲ} εμπροϲθεν αυτων και τα ιματια} & 3 & \textbf{3} &  \\
&  & 4 & \foreignlanguage{greek}{αυτου εγενετο ϲτιλβοντα λευκα λιαν} & 8 &  &  \\
[0.2em]
\cline{4-4}
\end{tabular}
\end{center}
\end{table}
}
\clearpage
\newpage
 {
 \setlength\arrayrulewidth{1pt}
\begin{table}
\begin{center}
\begin{tabular}{ccc|l|ccc}
\cline{4-4} \\ [-1em]
\multicolumn{7}{c}{\foreignlanguage{greek}{ευαγγελιον κατα μαρκον} \textbf{(\nospace{9:3})} } \\ \\ [-1em] % Si on veut ajouter les bordures latérales, remplacer {7}{c} par {7}{|c|}
\cline{4-4} \\
\cline{4-4}
&  &  & &  &  & \\ [-0.9em]
&  & 9 & \foreignlanguage{greek}{ωϲ γναφευϲ επι τηϲ γηϲ ου δυναται λευ} & 16 &  &  \\
&  & 16 & \foreignlanguage{greek}{καναι και ιδου ωφθη αυτοϲ ηλιαϲ ϲυν} & 6 & \textbf{4} &  \\
&  & 7 & \foreignlanguage{greek}{μωυϲη και ηϲαν ϲυνλαλουντεϲ τω \textoverline{ιυ}} & 12 &  &  \\
& \textbf{5} &  & \foreignlanguage{greek}{και αποκριθειϲ ειπεν πετροϲ τω \textoverline{ιυ}} & 6 &  &  \\
&  & 7 & \foreignlanguage{greek}{ραββει καλον εϲτιν ωδε ημαϲ ειναι} & 12 &  &  \\
&  & 13 & \foreignlanguage{greek}{και θελειϲ ποιηϲω ωδε ϲκηναϲ τριϲ ϲοι} & 19 &  &  \\
&  & 20 & \foreignlanguage{greek}{μιαν και μωυϲη μιαν και ηλια μιαν} & 26 &  &  \\
& \textbf{6} &  & \foreignlanguage{greek}{ου γαρ ηδει τι λαλει ηϲαν γαρ εκφοβοι} & 8 &  &  \\
& \textbf{7} &  & \foreignlanguage{greek}{και ιδου εγενετο νεφελη επιϲκιαζουϲα} & 5 &  &  \\
&  & 6 & \foreignlanguage{greek}{αυτουϲ και φωνη εκ τηϲ νεφεληϲ λε} & 12 &  &  \\
&  & 12 & \foreignlanguage{greek}{γουϲα ουτοϲ εϲτιν ο υιοϲ μου ο αγαπη} & 19 &  &  \\
&  & 19 & \foreignlanguage{greek}{τοϲ ακουετε αυτου και εξαπινα περι} & 3 & \textbf{8} &  \\
&  & 3 & \foreignlanguage{greek}{βλεπομενοι ουκετι ουδενα ειδον αλλα} & 7 &  &  \\
&  & 8 & \foreignlanguage{greek}{τον \textoverline{ιν} μονον μεθ εαυτων} & 12 &  &  \\
& \textbf{9} &  & \foreignlanguage{greek}{καταβαινοντων δε αυτων απο του ορουϲ} & 6 &  &  \\
&  & 7 & \foreignlanguage{greek}{διεϲτιλατο αυτοιϲ ινα μηδενι α ειδον} & 12 &  &  \\
&  & 13 & \foreignlanguage{greek}{εξηγηϲονται ει μη οταν ο υιοϲ του \textoverline{ανου}} & 20 &  &  \\
&  & 21 & \foreignlanguage{greek}{εκ νεκρων αναϲτη οι δε τον λογον εκρα} & 5 & \textbf{10} &  \\
&  & 5 & \foreignlanguage{greek}{τηϲαν προϲ εαυτουϲ ϲυνζητουντεϲ τι ε} & 10 &  &  \\
&  & 10 & \foreignlanguage{greek}{ϲτιν οταν εκ νεκρων αναϲτη και επη} & 2 & \textbf{11} &  \\
&  & 2 & \foreignlanguage{greek}{ρωτηϲαν αυτον λεγοντεϲ τι ουν λεγου} & 7 &  &  \\
&  & 7 & \foreignlanguage{greek}{ϲιν οι γραμματιϲ οτι ηλιαν δει ελθειν πρω} & 14 &  &  \\
&  & 14 & \foreignlanguage{greek}{τον ο δε αποκριθειϲ ειπεν αυτοιϲ} & 5 & \textbf{12} &  \\
&  & 6 & \foreignlanguage{greek}{ηλιαϲ ελθων πρωτοϲ αποκαθιϲτανι πα̅} & 10 &  &  \\
&  & 10 & \foreignlanguage{greek}{τα και πωϲ γεγραπται επι τον υιον του} & 17 &  &  \\
&  & 18 & \foreignlanguage{greek}{\textoverline{ανου} ινα πολλα παθη και εξουθενηθη} & 23 &  &  \\
& \textbf{13} &  & \foreignlanguage{greek}{αλλα λεγω υμιν οτι ηδη ηλιαϲ ηλθεν} & 7 &  &  \\
&  & 8 & \foreignlanguage{greek}{και εποιηϲαν αυτω οϲα ηθεληϲαν καθωϲ} & 13 &  &  \\
&  & 14 & \foreignlanguage{greek}{γεγραπται επ αυτω και ελθοντεϲ προϲ} & 3 & \textbf{14} &  \\
&  & 4 & \foreignlanguage{greek}{τουϲ μαθηταϲ ιδον οχλον περι αυτουϲ} & 9 &  &  \\
[0.2em]
\cline{4-4}
\end{tabular}
\end{center}
\end{table}
}
\clearpage
\newpage
 {
 \setlength\arrayrulewidth{1pt}
\begin{table}
\begin{center}
\begin{tabular}{ccc|l|ccc}
\cline{4-4} \\ [-1em]
\multicolumn{7}{c}{\foreignlanguage{greek}{ευαγγελιον κατα μαρκον} \textbf{(\nospace{9:14})} } \\ \\ [-1em] % Si on veut ajouter les bordures latérales, remplacer {7}{c} par {7}{|c|}
\cline{4-4} \\
\cline{4-4}
&  &  & &  &  & \\ [-0.9em]
&  & 10 & \foreignlanguage{greek}{και γραμματιϲ ϲυνζητουνταϲ προϲ αυτουϲ} & 14 &  &  \\
& \textbf{15} &  & \foreignlanguage{greek}{και ευθυϲ παϲ ο οχλοϲ ιδοντεϲ αυτον εξεθαμ} & 8 &  &  \\
&  & 8 & \foreignlanguage{greek}{βηθηϲαν και προϲτρεχοντεϲ ηϲπαζοντο} & 11 &  &  \\
&  & 12 & \foreignlanguage{greek}{αυτον και επηρωτηϲεν αυτουϲ τι ϲυν} & 5 & \textbf{16} &  \\
&  & 5 & \foreignlanguage{greek}{ζητειτε προϲ εαυτουϲ και αποκριθειϲ} & 2 & \textbf{17} &  \\
&  & 3 & \foreignlanguage{greek}{εκ του οχλου ειϲ ειπεν αυτω διδαϲκαλε} & 9 &  &  \\
&  & 10 & \foreignlanguage{greek}{ηνεγκα τον υιον μου προϲ ϲε εχοντα \textoverline{πνα}} & 17 &  &  \\
&  & 18 & \foreignlanguage{greek}{αλαλον και οπου αν αυτον καταλαβη ρηϲ} & 6 & \textbf{18} &  \\
&  & 6 & \foreignlanguage{greek}{ϲει και αφριζει και τριζει τουϲ οδονταϲ} & 12 &  &  \\
&  & 13 & \foreignlanguage{greek}{και ξηρενετε και ειπα τοιϲ μαθηταιϲ ϲου} & 19 &  &  \\
&  & 20 & \foreignlanguage{greek}{ινα αυτο εκβαλωϲιν και ουκ ηδυνηθηϲα̅} & 25 &  &  \\
&  & 26 & \foreignlanguage{greek}{εκβαλειν αυτο και αποκριθειϲ αυτοιϲ} & 3 & \textbf{19} &  \\
&  & 4 & \foreignlanguage{greek}{ο \textoverline{ιϲ} λεγει ω γενεα απιϲτε και διεϲτραμ} & 11 &  &  \\
&  & 11 & \foreignlanguage{greek}{μενη εωϲ ποτε προϲ υμαϲ εϲομαι εωϲ} & 17 &  &  \\
&  & 18 & \foreignlanguage{greek}{ποτε ανεξωμαι υμων φερεται αυτο̅} & 22 &  &  \\
&  & 23 & \foreignlanguage{greek}{προϲ με και ηνεγκαν αυτον προϲ αυτο̅} & 5 & \textbf{20} &  \\
&  & 6 & \foreignlanguage{greek}{ιδων αυτον ευθεωϲ το \textoverline{πνα} εϲπαραξεν} & 11 &  &  \\
&  & 12 & \foreignlanguage{greek}{και πεϲων επι τηϲ γηϲ εκυλιετο αφριζων} & 18 &  &  \\
& \textbf{21} &  & \foreignlanguage{greek}{και επηρωτηϲεν αυτου τον \textoverline{πρα} λεγων} & 6 &  &  \\
&  & 7 & \foreignlanguage{greek}{ποϲοϲ χρονοϲ εϲτιν εξ ου τουτο γεγονεν} & 13 &  &  \\
&  & 14 & \foreignlanguage{greek}{αυτω ο δε ειπεν εκ παιδοθεν και πολ} & 2 & \textbf{22} &  \\
&  & 2 & \foreignlanguage{greek}{λακειϲ αυτον ειϲ πυρ εβαλεν και ειϲ υδα} & 9 &  &  \\
&  & 9 & \foreignlanguage{greek}{τα ινα απολεϲη αυτον αλλα ει τι δυνη} & 16 &  &  \\
&  & 17 & \foreignlanguage{greek}{βοηθηϲον ημιν ϲπλαγχνιϲθειϲ εφ ημαϲ} & 21 &  &  \\
& \textbf{23} &  & \foreignlanguage{greek}{ο δε \textoverline{ιϲ} ειπεν αυτω τουτο ει δυνη παντα} & 9 &  &  \\
&  & 10 & \foreignlanguage{greek}{δυνατα τω πιϲτευοντι και ευθεωϲ κρα} & 3 & \textbf{24} &  \\
&  & 3 & \foreignlanguage{greek}{ξαϲ το \textoverline{πνα} του παιδαριου ειπεν πιϲτευ} & 9 &  &  \\
&  & 9 & \foreignlanguage{greek}{ω βοηθηϲον μου τη απιϲτια} & 13 &  &  \\
& \textbf{25} &  & \foreignlanguage{greek}{ιδων δε ο \textoverline{ιϲ} οτι ϲυντρεχει ο οχλοϲ επετι} & 9 &  &  \\
&  & 9 & \foreignlanguage{greek}{μηϲε τω \textoverline{πνι} λεγων αυτω το αλαλον ϗ} & 16 &  &  \\
[0.2em]
\cline{4-4}
\end{tabular}
\end{center}
\end{table}
}
\clearpage
\newpage
 {
 \setlength\arrayrulewidth{1pt}
\begin{table}
\begin{center}
\begin{tabular}{ccc|l|ccc}
\cline{4-4} \\ [-1em]
\multicolumn{7}{c}{\foreignlanguage{greek}{ευαγγελιον κατα μαρκον} \textbf{(\nospace{9:25})} } \\ \\ [-1em] % Si on veut ajouter les bordures latérales, remplacer {7}{c} par {7}{|c|}
\cline{4-4} \\
\cline{4-4}
&  &  & &  &  & \\ [-0.9em]
&  & 17 & \foreignlanguage{greek}{κωφον \textoverline{πνα} εγω επιταϲϲω ϲοι εξελθε εξ} & 23 &  &  \\
&  & 24 & \foreignlanguage{greek}{αυτου και μηκετι ειϲελθηϲ ειϲ αυτον} & 29 &  &  \\
& \textbf{26} &  & \foreignlanguage{greek}{και κραξαϲ και πολλα ϲπαραξαϲ εξηλθεν} & 6 &  &  \\
&  & 7 & \foreignlanguage{greek}{και εγενετο ωϲει νεκροϲ ωϲτε πολλουϲ} & 12 &  &  \\
&  & 13 & \foreignlanguage{greek}{λεγειν οτι απεθανεν ο δε \textoverline{ιϲ} κρατηϲαϲ} & 4 & \textbf{27} &  \\
&  & 5 & \foreignlanguage{greek}{τηϲ χειροϲ ηγειρεν αυτον} & 8 &  &  \\
& \textbf{28} &  & \foreignlanguage{greek}{και ειϲελθοντοϲ αυτου ειϲ οικον προϲηλ} & 6 &  &  \\
&  & 6 & \foreignlanguage{greek}{θον αυτω οι μαθηται κατ ιδιαν και επη} & 13 &  &  \\
&  & 13 & \foreignlanguage{greek}{ρωτηϲαν αυτον λεγοντεϲ οτι ημειϲ ου} & 18 &  &  \\
&  & 18 & \foreignlanguage{greek}{κ ηδυνηθημεν εκβαλειν αυτο και ειπε̅} & 2 & \textbf{29} &  \\
&  & 3 & \foreignlanguage{greek}{αυτοιϲ τουτο το γενοϲ εν ουδενι δυνα} & 9 &  &  \\
&  & 9 & \foreignlanguage{greek}{τε εξελθειν ει μη εν προϲευχη ϗ νηϲτια} & 16 &  &  \\
& \textbf{30} &  & \foreignlanguage{greek}{και εκειθεν εξελθοντεϲ παρεπορευοντο} & 4 &  &  \\
&  & 5 & \foreignlanguage{greek}{δια τηϲ γαλιλαιαϲ και ουκ ηθελεν ινα} & 11 &  &  \\
&  & 12 & \foreignlanguage{greek}{τιϲ γνω εδιδαϲκεν γαρ τουϲ μαθηταϲ} & 4 & \textbf{31} &  \\
&  & 5 & \foreignlanguage{greek}{αυτου και λεγει αυτοιϲ οτι ο \textoverline{υϲ} του} & 12 &  &  \\
&  & 13 & \foreignlanguage{greek}{\textoverline{ανου} παραδιδοτε ειϲ χειραϲ \textoverline{ανων} και} & 18 &  &  \\
&  & 19 & \foreignlanguage{greek}{αποκτενουϲιν αυτον και αποκτανθειϲ} & 22 &  &  \\
&  & 23 & \foreignlanguage{greek}{τη τριτη ημερα εγειρεται οι δε ηγνο} & 3 & \textbf{32} &  \\
&  & 3 & \foreignlanguage{greek}{ουν το ρημα και εφοβουντο αυτον ερω} & 9 &  &  \\
&  & 9 & \foreignlanguage{greek}{τηϲαι και ηλθον ειϲ καφαρναουμ} & 4 & \textbf{33} &  \\
&  & 5 & \foreignlanguage{greek}{και εν τη οικεια γενομενοϲ επηρωτα αυ} & 11 &  &  \\
&  & 11 & \foreignlanguage{greek}{τουϲ τι εν τη οδω διελεχθητε προϲ εαυ} & 18 &  &  \\
&  & 18 & \foreignlanguage{greek}{τουϲ οι δε εϲιωπων προϲ αλληλουϲ γαρ} & 6 & \textbf{34} &  \\
&  & 7 & \foreignlanguage{greek}{διελεχθηϲαν εν τη οδω τιϲ αυτων μειζο̅} & 13 &  &  \\
&  & 14 & \foreignlanguage{greek}{ειη καθειϲαϲ εφωνηϲεν τουϲ \textoverline{ιβ} και λεγει} & 6 & \textbf{35} &  \\
&  & 7 & \foreignlanguage{greek}{αυτοιϲ ει τιϲ θελει πρωτοϲ ειναι εϲτε} & 13 &  &  \\
&  & 14 & \foreignlanguage{greek}{παντων εϲχατοϲ και παντων διακονοϲ} & 18 &  &  \\
& \textbf{36} &  & \foreignlanguage{greek}{και λαβων παιδιον εϲτηϲεν μεϲω αυτω̅} & 6 &  &  \\
&  & 7 & \foreignlanguage{greek}{και ενανκαλιϲαμενοϲ αυτο ειπεν αυτοιϲ} & 11 &  &  \\
[0.2em]
\cline{4-4}
\end{tabular}
\end{center}
\end{table}
}
\clearpage
\newpage
 {
 \setlength\arrayrulewidth{1pt}
\begin{table}
\begin{center}
\begin{tabular}{ccc|l|ccc}
\cline{4-4} \\ [-1em]
\multicolumn{7}{c}{\foreignlanguage{greek}{ευαγγελιον κατα μαρκον} \textbf{(\nospace{9:37})} } \\ \\ [-1em] % Si on veut ajouter les bordures latérales, remplacer {7}{c} par {7}{|c|}
\cline{4-4} \\
\cline{4-4}
&  &  & &  &  & \\ [-0.9em]
& \textbf{37} &  & \foreignlanguage{greek}{οϲ αν εκ των τοιουτων παιδιον δεξηται} & 7 &  &  \\
&  & 8 & \foreignlanguage{greek}{εν τω ονοματι μου εμε δεχεται και οϲ αν} & 16 &  &  \\
&  & 17 & \foreignlanguage{greek}{εμε δεξηται ουκ εμε δεχεται αλλα τον απο} & 24 &  &  \\
&  & 24 & \foreignlanguage{greek}{ϲτιλαντα με και απεκριθη αυτω ο ιω} & 5 & \textbf{38} &  \\
&  & 5 & \foreignlanguage{greek}{αννηϲ ειπεν διδαϲκαλε ειδομεν τινα} & 9 &  &  \\
&  & 10 & \foreignlanguage{greek}{εν τω ονοματι ϲου εκβαλλοντα δαιμονια} & 15 &  &  \\
&  & 16 & \foreignlanguage{greek}{οϲ ουκ ηκολουθει ημιν και εκωλυϲαμε̅} & 21 &  &  \\
&  & 22 & \foreignlanguage{greek}{αυτον ο δε ειπεν μη κωλυετε αυτον} & 6 & \textbf{39} &  \\
&  & 7 & \foreignlanguage{greek}{ουδειϲ γαρ εϲτιν οϲ ποιηϲει δυναμιν εν} & 13 &  &  \\
&  & 14 & \foreignlanguage{greek}{τω ονοματι μου και δυνηϲονται με κα} & 20 &  &  \\
&  & 20 & \foreignlanguage{greek}{κολογηϲαι οϲ γαρ ουκ εϲτιν καθ ημων} & 6 & \textbf{40} &  \\
&  & 7 & \foreignlanguage{greek}{υπερ ημων εϲτιν οϲ αν γαρ ποτιϲη υμαϲ} & 5 & \textbf{41} &  \\
&  & 6 & \foreignlanguage{greek}{ποτηριον υδατοϲ εν ονοματι μου οτι \textoverline{χρϲ}} & 12 &  &  \\
&  & 13 & \foreignlanguage{greek}{εϲται αμην λεγω υμιν οτι ου μη απολεϲη} & 20 &  &  \\
&  & 21 & \foreignlanguage{greek}{τον μιϲθον αυτου και οϲ αν ϲκανδαλι} & 4 & \textbf{42} &  \\
&  & 4 & \foreignlanguage{greek}{ϲη ενα των μικρων μου των πιϲτευοντω̅} & 10 &  &  \\
&  & 11 & \foreignlanguage{greek}{ειϲ εμε καλον εϲτιν μαλλον ει περιεκει} & 17 &  &  \\
&  & 17 & \foreignlanguage{greek}{το μυλον ονικον περι τον τραχηλον αυτου} & 23 &  &  \\
&  & 24 & \foreignlanguage{greek}{και εβληθη ειϲ την θαλαϲϲαν και εαν} & 2 & \textbf{43} &  \\
&  & 3 & \foreignlanguage{greek}{ϲκανδαλιϲη ϲε η χειρ ϲου αποκοψον αυτη̅} & 9 &  &  \\
&  & 10 & \foreignlanguage{greek}{καλον ϲοι εϲτιν ειϲ την ζωην ειϲελθει̅} & 16 &  &  \\
&  & 17 & \foreignlanguage{greek}{κυλλον η ταϲ δυο χειραϲ εχοντα απελ} & 23 &  &  \\
&  & 23 & \foreignlanguage{greek}{θειν ειϲ το πυρ το αϲβεϲτον} & 28 &  &  \\
& \textbf{45} &  & \foreignlanguage{greek}{πουϲ ϲου ϲκανδαλιϲη ϲε κοψον αυτον} & 9 &  &  \\
&  & 10 & \foreignlanguage{greek}{καλον ϲοι εϲτιν ειϲελθειν ειϲ την ζωη̅} & 16 &  &  \\
&  & 17 & \foreignlanguage{greek}{χωλον η τουϲ δυο ποδαϲ εχοντα απελθει̅} & 23 &  &  \\
&  & 24 & \foreignlanguage{greek}{ειϲ την γεενναν} & 26 &  &  \\
& \textbf{47} &  & \foreignlanguage{greek}{ϲκανδαλιϲη ϲε εκβαλε αυτον καλον} & 10 &  &  \\
&  & 11 & \foreignlanguage{greek}{εϲτιν μονοφθαλμον ειϲελθειν ειϲ την} & 15 &  &  \\
&  & 16 & \foreignlanguage{greek}{βαϲιλειαν του \textoverline{θυ} η δυο οφθαλμουϲ ε} & 22 &  &  \\
[0.2em]
\cline{4-4}
\end{tabular}
\end{center}
\end{table}
}
\clearpage
\newpage
 {
 \setlength\arrayrulewidth{1pt}
\begin{table}
\begin{center}
\begin{tabular}{ccc|l|ccc}
\cline{4-4} \\ [-1em]
\multicolumn{7}{c}{\foreignlanguage{greek}{ευαγγελιον κατα μαρκον} \textbf{(\nospace{9:47})} } \\ \\ [-1em] % Si on veut ajouter les bordures latérales, remplacer {7}{c} par {7}{|c|}
\cline{4-4} \\
\cline{4-4}
&  &  & &  &  & \\ [-0.9em]
&  & 22 & \foreignlanguage{greek}{χοντα ειϲ την γεενναν οπου ο ϲκωληξ αυ} & 4 & \textbf{48} &  \\
&  & 4 & \foreignlanguage{greek}{των ου τελευτα και το πυρ ου ϲβεννυεται} & 11 &  &  \\
& \textbf{49} &  & \foreignlanguage{greek}{παϲ γαρ πυρι αλιϲγηθηϲεται καλον το αλα} & 3 &  &  \\
&  & 4 & \foreignlanguage{greek}{εαν δε το αλα μωρανθη εν τινι αυτο αρτυ} & 12 &  &  \\
&  & 12 & \foreignlanguage{greek}{ϲηται υμειϲ ουν εν εαυτοιϲ εχεται αλα} & 18 &  &  \\
&  & 19 & \foreignlanguage{greek}{και ειρηνευεται εν αλληλοιϲ και εκειθε̅} & 2 & \mygospelchapter &  \\
&  & 3 & \foreignlanguage{greek}{αναϲταϲ ερχεται ειϲ τα ορια τηϲ ιουδαιαϲ} & 9 &  &  \\
&  & 10 & \foreignlanguage{greek}{περαν του ιορδανου και ϲυνπορευεται ο} & 15 &  &  \\
&  & 15 & \foreignlanguage{greek}{χλοϲ προϲ αυτον και ωϲ ιωθει παλιν εδι} & 22 &  &  \\
&  & 22 & \foreignlanguage{greek}{δαϲκεν αυτουϲ οι δε φαριϲαιοι προϲελ} & 4 & \textbf{2} &  \\
&  & 4 & \foreignlanguage{greek}{θοντεϲ επηρωτηϲαν αυτον ει εξεϲτιν} & 9 &  &  \\
&  & 10 & \foreignlanguage{greek}{ανδρι γυναικα απολυϲαι πειραζοντεϲ αυ} & 14 &  &  \\
&  & 14 & \foreignlanguage{greek}{τον ο δε αποκριθειϲ ειπεν αυτοιϲ τι υ} & 7 & \textbf{3} &  \\
&  & 7 & \foreignlanguage{greek}{μιν ενετιλατο μωυϲηϲ οι δε ειπαν} & 3 & \textbf{4} &  \\
&  & 4 & \foreignlanguage{greek}{μωυϲηϲ επετρεψε βιβλιον αποϲταϲιου} & 7 &  &  \\
&  & 8 & \foreignlanguage{greek}{γραψαι και απολυϲαι και αποκριθειϲ} & 2 & \textbf{5} &  \\
&  & 3 & \foreignlanguage{greek}{ο \textoverline{ιϲ} ειπεν αυτοιϲ προϲ την ϲκληροκαρ} & 9 &  &  \\
&  & 9 & \foreignlanguage{greek}{διαν υμων εγραψε την εντολην ταυτη̅} & 14 &  &  \\
& \textbf{6} &  & \foreignlanguage{greek}{απο δε αρχηϲ κτιϲεωϲ αρϲεν και θηλυ ε} & 8 &  &  \\
&  & 8 & \foreignlanguage{greek}{ποιηϲεν ο \textoverline{θϲ} και ειπεν ενεκεν τουτου} & 4 & \textbf{7} &  \\
&  & 5 & \foreignlanguage{greek}{καταλιψει εκαϲτοϲ τον \textoverline{πρα} αυτου και τη̅} & 11 &  &  \\
&  & 12 & \foreignlanguage{greek}{\textoverline{μρα} και προϲκολληθηϲεται προϲ την γυ} & 17 &  &  \\
&  & 17 & \foreignlanguage{greek}{ναικα αυτου και εϲονται οι δυο ειϲ ϲαρ} & 6 & \textbf{8} &  \\
&  & 6 & \foreignlanguage{greek}{κα μιαν ωϲτε ουκ ειϲιν δυο αλλα ϲαρξ} & 13 &  &  \\
&  & 14 & \foreignlanguage{greek}{μια ο ουν ο \textoverline{θϲ} εζευξεν \textoverline{ανοϲ} μη χωρι} & 8 & \textbf{9} &  \\
&  & 8 & \foreignlanguage{greek}{ζετω και εν τη οικεια παλιν επηρω} & 6 & \textbf{10} &  \\
&  & 6 & \foreignlanguage{greek}{τηϲαν οι μαθηται αυτου και λεγει αυ} & 3 & \textbf{11} &  \\
&  & 3 & \foreignlanguage{greek}{τοιϲ εαν απολυϲη γυνη τον ανδρα αυ} & 9 &  &  \\
&  & 9 & \foreignlanguage{greek}{τηϲ και γαμηϲη αλλον μοιχαται και} & 1 & \textbf{12} &  \\
&  & 2 & \foreignlanguage{greek}{εαν ανηρ απολυϲη την γυναικα μοιχατε} & 7 &  &  \\
[0.2em]
\cline{4-4}
\end{tabular}
\end{center}
\end{table}
}
\clearpage
\newpage
 {
 \setlength\arrayrulewidth{1pt}
\begin{table}
\begin{center}
\begin{tabular}{ccc|l|ccc}
\cline{4-4} \\ [-1em]
\multicolumn{7}{c}{\foreignlanguage{greek}{ευαγγελιον κατα μαρκον} \textbf{(\nospace{10:13})} } \\ \\ [-1em] % Si on veut ajouter les bordures latérales, remplacer {7}{c} par {7}{|c|}
\cline{4-4} \\
\cline{4-4}
&  &  & &  &  & \\ [-0.9em]
& \textbf{13} &  & \foreignlanguage{greek}{και προϲεφερον αυτω παιδια ινα αψηται αυ} & 7 &  &  \\
&  & 7 & \foreignlanguage{greek}{των οι δε μαθηται επετιμων τοιϲ προϲφε} & 13 &  &  \\
&  & 13 & \foreignlanguage{greek}{ρουϲιν ιδων δε ο \textoverline{ιϲ} ηγανακτηϲεν και επιτει} & 7 & \textbf{14} &  \\
&  & 7 & \foreignlanguage{greek}{μηϲαϲ αυτοιϲ ειπεν αφεται τα παιδια ερχε} & 13 &  &  \\
&  & 13 & \foreignlanguage{greek}{ϲθαι προϲ εμε μη κωλυεται αυτα των γαρ} & 20 &  &  \\
&  & 21 & \foreignlanguage{greek}{τοιουτων εϲτιν η βαϲιλεια των ουρανων} & 26 &  &  \\
& \textbf{15} &  & \foreignlanguage{greek}{αμην λεγω υμιν οϲ αν μη δεξηται την} & 8 &  &  \\
&  & 9 & \foreignlanguage{greek}{βαϲιλειαν του \textoverline{θυ} ωϲ παιδιον ου μη ειϲελ} & 16 &  &  \\
&  & 16 & \foreignlanguage{greek}{θη ειϲ αυτην και ενανκαλειϲαμενοϲ αυ} & 3 & \textbf{16} &  \\
&  & 3 & \foreignlanguage{greek}{τα επιτιθει ταϲ χειραϲ επ αυτα και ευλογει} & 10 &  &  \\
&  & 11 & \foreignlanguage{greek}{αυτα και εκπορευομενου αυτου ειϲ οδο̅} & 5 & \textbf{17} &  \\
&  & 6 & \foreignlanguage{greek}{ιδου τιϲ πλουϲιοϲ προϲδραμων και γονυ} & 11 &  &  \\
&  & 11 & \foreignlanguage{greek}{πετηϲαϲ επηρωτα αυτον λεγων διδαϲκα} & 15 &  &  \\
&  & 15 & \foreignlanguage{greek}{λε αγαθε τι ποιηϲω ινα ζωην αιωνιον} & 21 &  &  \\
&  & 22 & \foreignlanguage{greek}{κληρονομηϲω ο δε \textoverline{ιϲ} ειπεν αυτω} & 5 & \textbf{18} &  \\
&  & 6 & \foreignlanguage{greek}{τι με λεγειϲ αγαθον ουδειϲ αγαθοϲ ει μη} & 13 &  &  \\
&  & 14 & \foreignlanguage{greek}{ειϲ ο \textoverline{θϲ} ταϲ εντολαϲ οιδαϲ μη μοιχευ} & 5 & \textbf{19} &  \\
&  & 5 & \foreignlanguage{greek}{ϲηϲ μη φονευϲηϲ μη κλεψηϲ μη ψευ} & 11 &  &  \\
&  & 11 & \foreignlanguage{greek}{δομαρτυρηϲηϲ τιμα τον \textoverline{πρα} ϲου και τη̅} & 17 &  &  \\
&  & 18 & \foreignlanguage{greek}{\textoverline{μρα} ϲου ο δε αποκριθειϲ ειπεν αυτω} & 5 & \textbf{20} &  \\
&  & 6 & \foreignlanguage{greek}{διδαϲκαλε ταυτα παντα εφυλαξαμη̅} & 9 &  &  \\
&  & 10 & \foreignlanguage{greek}{εκ νεοτητοϲ μου τι υϲτερω ετι} & 15 &  &  \\
& \textbf{21} &  & \foreignlanguage{greek}{\textoverline{ιϲ} ενβλεψαϲ αυτω ηγαπηϲεν αυτον και ει} & 7 &  &  \\
&  & 7 & \foreignlanguage{greek}{πεν αυτω ει θελειϲ τελιοϲ ειναι εν ϲε υ} & 15 &  &  \\
&  & 15 & \foreignlanguage{greek}{ϲτερει υπαγε οϲα εχειϲ πωληϲον και δοϲ} & 21 &  &  \\
&  & 22 & \foreignlanguage{greek}{πτωχοιϲ και εξειϲ θηϲαυρον εν ουρανοιϲ} & 27 &  &  \\
&  & 28 & \foreignlanguage{greek}{και αραϲ τον ϲταυρον ϲου δευρο ακολουθι μοι} & 35 &  &  \\
& \textbf{22} &  & \foreignlanguage{greek}{ο δε ϲτυγναϲαϲ απο του λογου απηλθεν α} & 8 &  &  \\
&  & 8 & \foreignlanguage{greek}{π αυτου λυπουμενοϲ ην γαρ εχων κτη} & 14 &  &  \\
&  & 14 & \foreignlanguage{greek}{ματα πολλα και περιβλεψαμενοϲ ο \textoverline{ιϲ}} & 4 & \textbf{23} &  \\
[0.2em]
\cline{4-4}
\end{tabular}
\end{center}
\end{table}
}
\clearpage
\newpage
 {
 \setlength\arrayrulewidth{1pt}
\begin{table}
\begin{center}
\begin{tabular}{ccc|l|ccc}
\cline{4-4} \\ [-1em]
\multicolumn{7}{c}{\foreignlanguage{greek}{ευαγγελιον κατα μαρκον} \textbf{(\nospace{10:23})} } \\ \\ [-1em] % Si on veut ajouter les bordures latérales, remplacer {7}{c} par {7}{|c|}
\cline{4-4} \\
\cline{4-4}
&  &  & &  &  & \\ [-0.9em]
&  & 5 & \foreignlanguage{greek}{λεγει τοιϲ μαθηταιϲ αυτου πωϲ δυϲκο} & 10 &  &  \\
&  & 10 & \foreignlanguage{greek}{λωϲ οι τα χρηματα εχοντεϲ ειϲ την βαϲι} & 17 &  &  \\
&  & 17 & \foreignlanguage{greek}{λειαν του \textoverline{θυ} ειϲελευϲονται οι δε μαθη} & 3 & \textbf{24} &  \\
&  & 3 & \foreignlanguage{greek}{ται εθαμβουντο επι τοιϲ λογοιϲ αυτου} & 8 &  &  \\
&  & 9 & \foreignlanguage{greek}{ο δε \textoverline{ιϲ} αποκριθειϲ λεγει αυτοιϲ τεκνα} & 15 &  &  \\
&  & 16 & \foreignlanguage{greek}{πωϲ δυϲκολον εϲτιν ειϲ την βαϲιλεια̅} & 21 &  &  \\
&  & 22 & \foreignlanguage{greek}{του \textoverline{θυ} πλουϲιον ειϲελθειν ευκοπωτε} & 1 & \textbf{25} &  \\
&  & 1 & \foreignlanguage{greek}{ρον εϲτιν καμηλον δια τρωμαλιαϲ ρα} & 6 &  &  \\
&  & 6 & \foreignlanguage{greek}{φιδοϲ ειϲελθειν η ειϲ την βαϲιλειαν} & 11 &  &  \\
&  & 12 & \foreignlanguage{greek}{του \textoverline{θυ} πλουϲιον ειϲελθειν οι δε περιϲ} & 3 & \textbf{26} &  \\
&  & 3 & \foreignlanguage{greek}{ϲωϲ εξεπληϲϲοντο λεγοντεϲ προϲ εαυ} & 7 &  &  \\
&  & 7 & \foreignlanguage{greek}{τουϲ και τιϲ δυνηϲεται ϲωθηναι} & 11 &  &  \\
& \textbf{27} &  & \foreignlanguage{greek}{εμβλεψαϲ δε αυτοιϲ ο \textoverline{ιϲ} λεγει παρα μεν} & 8 &  &  \\
&  & 9 & \foreignlanguage{greek}{\textoverline{ανοιϲ} τουτο αδυνατον αλλα ου παρα τω \textoverline{θω}} & 16 &  &  \\
&  & 17 & \foreignlanguage{greek}{παντα γαρ δυνατα τω \textoverline{θω}} & 21 &  &  \\
& \textbf{28} &  & \foreignlanguage{greek}{ηρξατο αυτω λεγειν ο πετροϲ παντα α} & 7 &  &  \\
&  & 7 & \foreignlanguage{greek}{φηκαμεν και ηκολουθηκαμεν ϲοι} & 10 &  &  \\
& \textbf{29} &  & \foreignlanguage{greek}{αποκριθειϲ ο \textoverline{ιϲ} ειπεν αμην λεγω υμιν} & 7 &  &  \\
&  & 8 & \foreignlanguage{greek}{ουδειϲ εϲτιν οϲ αφηκεν οικειαν η αδελ} & 14 &  &  \\
&  & 14 & \foreignlanguage{greek}{φουϲ η αδελφαϲ η \textoverline{μρα} η \textoverline{πρα} η τεκνα} & 22 &  &  \\
&  & 23 & \foreignlanguage{greek}{η αγρουϲ ενεκεν εμου και ενεκεν του} & 29 &  &  \\
&  & 30 & \foreignlanguage{greek}{ευαγγελιου εαν μη λαβη εκατοντα} & 4 & \textbf{30} &  \\
&  & 4 & \foreignlanguage{greek}{πλαϲιονα νυν εν τω καιρω τουτω οικει} & 10 &  &  \\
&  & 10 & \foreignlanguage{greek}{αϲ και αδελφαϲ και \textoverline{μρα} και τεκνα και} & 17 &  &  \\
&  & 18 & \foreignlanguage{greek}{αγρουϲ μετα διωγμων και εν τω αιωνι} & 24 &  &  \\
&  & 25 & \foreignlanguage{greek}{τω ερχομενω ζωην αιωνιον πολλοι} & 1 & \textbf{31} &  \\
&  & 2 & \foreignlanguage{greek}{δε εϲονται πρωτοι εϲχατοι και εϲχατοι} & 7 &  &  \\
&  & 8 & \foreignlanguage{greek}{πρωτοι ηϲαν δε εν τη οδω ανα} & 6 & \textbf{32} &  \\
&  & 6 & \foreignlanguage{greek}{βαινοντεϲ ειϲ ιεροϲολυμα και ην προα} & 11 &  &  \\
&  & 11 & \foreignlanguage{greek}{γων αυτουϲ ο \textoverline{ιϲ} και εθαμβουντο ακολου} & 17 &  &  \\
[0.2em]
\cline{4-4}
\end{tabular}
\end{center}
\end{table}
}
\clearpage
\newpage
 {
 \setlength\arrayrulewidth{1pt}
\begin{table}
\begin{center}
\begin{tabular}{ccc|l|ccc}
\cline{4-4} \\ [-1em]
\multicolumn{7}{c}{\foreignlanguage{greek}{ευαγγελιον κατα μαρκον} \textbf{(\nospace{10:32})} } \\ \\ [-1em] % Si on veut ajouter les bordures latérales, remplacer {7}{c} par {7}{|c|}
\cline{4-4} \\
\cline{4-4}
&  &  & &  &  & \\ [-0.9em]
&  & 17 & \foreignlanguage{greek}{θουντεϲ αυτω και παραλαβων παλιν} & 21 &  &  \\
&  & 22 & \foreignlanguage{greek}{τουϲ \textoverline{ιβ} ηρξατο αυτοιϲ λεγειν τα μελλον} & 28 &  &  \\
&  & 28 & \foreignlanguage{greek}{τα αυτω ϲυμβαινειν οτι ιδου αναβαινο} & 3 & \textbf{33} &  \\
&  & 3 & \foreignlanguage{greek}{μεν ειϲ ιεροϲολυμα και ο υιοϲ του \textoverline{ανου} πα} & 11 &  &  \\
&  & 11 & \foreignlanguage{greek}{ραδοθηϲεται τοιϲ αρχιερευϲιν και γραμ} & 15 &  &  \\
&  & 15 & \foreignlanguage{greek}{ματευϲιν και κατακρινουϲιν αυτον θα} & 19 &  &  \\
&  & 19 & \foreignlanguage{greek}{νατω και παραδωϲουϲιν τοιϲ εθνεϲιν} & 23 &  &  \\
& \textbf{34} &  & \foreignlanguage{greek}{και ενπεξουϲιν αυτω και μαϲτιγωϲου} & 5 &  &  \\
&  & 5 & \foreignlanguage{greek}{ϲιν αυτον και ενπτυϲωϲιν αυτω και} & 10 &  &  \\
&  & 11 & \foreignlanguage{greek}{αποκτενουϲιν αυτον και τη τριτη ημερα} & 16 &  &  \\
&  & 17 & \foreignlanguage{greek}{αναϲτηϲεται και προϲηλθον αυτω} & 4 & \textbf{35} &  \\
&  & 5 & \foreignlanguage{greek}{ιακωβοϲ και ιωαννηϲ οι υιοι ζεβεδαιου} & 10 &  &  \\
&  & 11 & \foreignlanguage{greek}{λεγοντεϲ διδαϲκαλε θελωμεν ινα ο α̅} & 16 &  &  \\
&  & 17 & \foreignlanguage{greek}{ϲε αιτηϲωμεθα ποιηϲηϲ ημιν} & 20 &  &  \\
& \textbf{36} &  & \foreignlanguage{greek}{ο δε ειπεν αυτοιϲ τι θελεται με ποιηϲαι υμι̅} & 9 &  &  \\
& \textbf{37} &  & \foreignlanguage{greek}{οι δε ειπον αυτω δοϲ ημιν ινα ειϲ εκ δε} & 10 &  &  \\
&  & 10 & \foreignlanguage{greek}{ξιων ϲου και ειϲ εξ ευωνυμων καθιϲωμε̅} & 16 &  &  \\
&  & 17 & \foreignlanguage{greek}{εν βαϲιλεια τηϲ δοξηϲ} & 20 &  &  \\
& \textbf{38} &  & \foreignlanguage{greek}{ο δε \textoverline{ιϲ} αποκριθειϲ ειπεν αυτω ουκ οιδατε} & 8 &  &  \\
&  & 9 & \foreignlanguage{greek}{το αιτιϲθαι δυναϲθαι πιειν το ποτη} & 14 &  &  \\
&  & 14 & \foreignlanguage{greek}{ριον ο εγω πινω η το βαπτιϲμα ο εγω βα} & 23 &  &  \\
&  & 23 & \foreignlanguage{greek}{πτιζομαι βαπτιϲθηναι οι δε ειπαν} & 3 & \textbf{39} &  \\
&  & 4 & \foreignlanguage{greek}{δυναμεθα το μεν ποτηριον ο εγω πι} & 10 &  &  \\
&  & 10 & \foreignlanguage{greek}{νω πιεϲθαι και το βαπτιϲμα ο εγω βα} & 17 &  &  \\
&  & 17 & \foreignlanguage{greek}{πτιζομε βαπτιϲθηϲεϲθαι το δε κα} & 3 & \textbf{40} &  \\
&  & 3 & \foreignlanguage{greek}{θειϲαι εκ δεξιων μου η εξ ευωνυμων} & 9 &  &  \\
&  & 10 & \foreignlanguage{greek}{ουκ εϲτιν εμον δουναι αλλ οιϲ ητοιμα} & 16 &  &  \\
&  & 16 & \foreignlanguage{greek}{ϲται και ακουϲαντεϲ οι δεκα ηρξαν} & 5 & \textbf{41} &  \\
&  & 5 & \foreignlanguage{greek}{το αγανακτειν περι ιακωβου ϗ ιωαννου} & 10 &  &  \\
& \textbf{42} &  & \foreignlanguage{greek}{ο δε προϲκαλεϲαμενοϲ λεγει αυτοιϲ} & 5 &  &  \\
[0.2em]
\cline{4-4}
\end{tabular}
\end{center}
\end{table}
}
\clearpage
\newpage
 {
 \setlength\arrayrulewidth{1pt}
\begin{table}
\begin{center}
\begin{tabular}{ccc|l|ccc}
\cline{4-4} \\ [-1em]
\multicolumn{7}{c}{\foreignlanguage{greek}{ευαγγελιον κατα μαρκον} \textbf{(\nospace{10:42})} } \\ \\ [-1em] % Si on veut ajouter les bordures latérales, remplacer {7}{c} par {7}{|c|}
\cline{4-4} \\
\cline{4-4}
&  &  & &  &  & \\ [-0.9em]
&  & 6 & \foreignlanguage{greek}{οιδατε οτι οι δοκουντεϲ αρχειν των εθνω̅} & 12 &  &  \\
&  & 13 & \foreignlanguage{greek}{κατακυριευουϲιν αυτων και ου μεγα} & 17 &  &  \\
&  & 17 & \foreignlanguage{greek}{λοι αυτων κατεξουϲιαζουϲιν ουχ ουτωϲ} & 2 & \textbf{43} &  \\
&  & 3 & \foreignlanguage{greek}{εϲτιν εν υμιν αλλ οϲτιϲ αν θελη εν υ} & 11 &  &  \\
&  & 11 & \foreignlanguage{greek}{μιν μεγαϲ γενεϲθαι εϲται υμων διακο} & 16 &  &  \\
&  & 16 & \foreignlanguage{greek}{νοϲ και οϲ αν θελη υμων ειναι πρωτοϲ} & 7 & \textbf{44} &  \\
&  & 8 & \foreignlanguage{greek}{εϲται υμων παντων δουλοϲ και γαρ ο} & 3 & \textbf{45} &  \\
&  & 4 & \foreignlanguage{greek}{\textoverline{υιϲ} του \textoverline{ανου} ουκ ηλθεν διακονηθηναι αλ} & 10 &  &  \\
&  & 10 & \foreignlanguage{greek}{λα διακονηϲαι και δουναι την ψυχην} & 15 &  &  \\
&  & 16 & \foreignlanguage{greek}{αυτου λουτρον αντι πολλων και ερχο̅} & 2 & \textbf{46} &  \\
&  & 2 & \foreignlanguage{greek}{ται ειϲ ιεριχω και εκπορευομενου αυτου} & 7 &  &  \\
&  & 8 & \foreignlanguage{greek}{απο ιεριχω και των μαθητων αυτου και} & 14 &  &  \\
&  & 15 & \foreignlanguage{greek}{οχλου ικανου ο υιοϲ τιμαιου τυφλοϲ εκα} & 21 &  &  \\
&  & 21 & \foreignlanguage{greek}{θητο παρα την οδον προϲαιτων και ακου} & 2 & \textbf{47} &  \\
&  & 2 & \foreignlanguage{greek}{ϲαϲ οτι \textoverline{ιϲ} ο ναζαρηνοϲ εϲτιν ηρξατο κρα} & 9 &  &  \\
&  & 9 & \foreignlanguage{greek}{ζειν και λεγειν ο \textoverline{υϲ} \textoverline{δαδ} \textoverline{ιυ} ελεηϲον με} & 17 &  &  \\
& \textbf{49} &  & \foreignlanguage{greek}{και ϲταϲ ο \textoverline{ιϲ} ειπεν αυτον φωνηθηναι} & 7 &  &  \\
&  & 8 & \foreignlanguage{greek}{και φωνουϲιν τον τυφλον λεγοντεϲ θαρ} & 13 &  &  \\
&  & 13 & \foreignlanguage{greek}{ρων εγειρε φωνι ϲε ο δε αποβαλων το ιμα} & 5 & \textbf{50} &  \\
&  & 5 & \foreignlanguage{greek}{τιον αυτου αναϲταϲ ηλθεν προϲ τον \textoverline{ιν}} & 11 &  &  \\
& \textbf{51} &  & \foreignlanguage{greek}{και αποκριθειϲ λεγει αυτω ο \textoverline{ιϲ} τι θελιϲ ποι} & 9 &  &  \\
&  & 9 & \foreignlanguage{greek}{ηϲω ϲοι ο δε τυφλοϲ ειπεν αυτω ραββουνι} & 16 &  &  \\
&  & 17 & \foreignlanguage{greek}{ινα αναβλεψω ο δε ειπεν αυτω υπαγε} & 5 & \textbf{52} &  \\
&  & 6 & \foreignlanguage{greek}{η πιϲτιϲ ϲου ϲεϲωκεν ϲε και ευθεωϲ ανε} & 13 &  &  \\
&  & 13 & \foreignlanguage{greek}{βλεψεν και ηκολουθει αυτω εν τη οδω} & 19 &  &  \\
& \mygospelchapter &  & \foreignlanguage{greek}{και οτε ενγιζουϲιν ειϲ ιεροϲολυμα ειϲ βηθ} & 7 &  &  \\
&  & 7 & \foreignlanguage{greek}{φαγη και βηθανιαν προϲ το οροϲ των ελεω̅} & 14 &  &  \\
&  & 15 & \foreignlanguage{greek}{αποϲτελλει δυο των μαθητων αυτου λε} & 1 & \textbf{2} &  \\
&  & 1 & \foreignlanguage{greek}{γων αυτοιϲ υπαγεται ειϲ την κατεναν} & 6 &  &  \\
&  & 6 & \foreignlanguage{greek}{τι κωμην και ευθεωϲ ειϲπορευομενοι ειϲ} & 11 &  &  \\
[0.2em]
\cline{4-4}
\end{tabular}
\end{center}
\end{table}
}
\clearpage
\newpage
 {
 \setlength\arrayrulewidth{1pt}
\begin{table}
\begin{center}
\begin{tabular}{ccc|l|ccc}
\cline{4-4} \\ [-1em]
\multicolumn{7}{c}{\foreignlanguage{greek}{ευαγγελιον κατα μαρκον} \textbf{(\nospace{11:2})} } \\ \\ [-1em] % Si on veut ajouter les bordures latérales, remplacer {7}{c} par {7}{|c|}
\cline{4-4} \\
\cline{4-4}
&  &  & &  &  & \\ [-0.9em]
&  & 12 & \foreignlanguage{greek}{αυτην ευρηϲεται πωλον δεδεμενον ω ου} & 17 &  &  \\
&  & 17 & \foreignlanguage{greek}{πω ουδειϲ \textoverline{ανων} επικεκαθεικεν λυϲαντεϲ} & 21 &  &  \\
&  & 22 & \foreignlanguage{greek}{αυτον αγαγετε και εαν τιϲ υμιν ειπη τι} & 6 & \textbf{3} &  \\
&  & 7 & \foreignlanguage{greek}{ειπατε οτι ο \textoverline{κϲ} αυτου χρειαν εχει και ευθεωϲ} & 15 &  &  \\
&  & 16 & \foreignlanguage{greek}{αυτον αποϲτελει ωδε απηλθον δε και ευ} & 4 & \textbf{4} &  \\
&  & 4 & \foreignlanguage{greek}{ρον πωλον δεδεμενον προϲ θυραν εξω ε} & 10 &  &  \\
&  & 10 & \foreignlanguage{greek}{πι του αμφοδου και λυουϲιν αυτον} & 15 &  &  \\
& \textbf{5} &  & \foreignlanguage{greek}{τινεϲ δε των εκει εϲτωτων ελεγον αυτοιϲ} & 7 &  &  \\
&  & 8 & \foreignlanguage{greek}{τι ποιειτε λυοντεϲ τον πωλον οι δε ειπο̅} & 3 & \textbf{6} &  \\
&  & 4 & \foreignlanguage{greek}{αυτοιϲ καθωϲ ειπεν αυτοιϲ ο \textoverline{ιϲ} και αφη} & 11 &  &  \\
&  & 11 & \foreignlanguage{greek}{καν αυτουϲ και αγουϲιν τον πωλον προϲ} & 5 & \textbf{7} &  \\
&  & 6 & \foreignlanguage{greek}{τον \textoverline{ιν} και επιβαλλουϲιν αυτω τα ιματια} & 12 &  &  \\
&  & 13 & \foreignlanguage{greek}{και καθιζει επ αυτω πολλοι δε τα ιματια} & 4 & \textbf{8} &  \\
&  & 5 & \foreignlanguage{greek}{εϲτρωννυον ειϲ την οδον και οι προα} & 3 & \textbf{9} &  \\
&  & 3 & \foreignlanguage{greek}{γοντεϲ και οι ακολουθουντεϲ εκραζον} & 7 &  &  \\
&  & 8 & \foreignlanguage{greek}{λεγοντεϲ ευλογημενοϲ ο ερχομενοϲ} & 11 &  &  \\
&  & 12 & \foreignlanguage{greek}{εν ονοματι \textoverline{κυ} ευλογημενη η ερχομενη} & 3 & \textbf{10} &  \\
&  & 4 & \foreignlanguage{greek}{βαϲειλια του \textoverline{πρϲ} ημων \textoverline{δαδ} ειρηνη εν} & 10 &  &  \\
&  & 11 & \foreignlanguage{greek}{τοιϲ υψιϲτοιϲ και ειϲηλθεν ειϲ ιεροϲο} & 4 & \textbf{11} &  \\
&  & 4 & \foreignlanguage{greek}{λυμα ειϲ το ιερον και περιβλεψαμενοϲ} & 9 &  &  \\
&  & 10 & \foreignlanguage{greek}{παντα οψειαϲ ουϲηϲ τηϲ ωραϲ εξηλθεν} & 15 &  &  \\
&  & 16 & \foreignlanguage{greek}{ειϲ βηθανιαν μετα των \textoverline{ιβ} και τη αυριο̅} & 3 & \textbf{12} &  \\
&  & 4 & \foreignlanguage{greek}{εξελθοντων αυτων ειϲ βηθανιαν επι} & 8 &  &  \\
&  & 8 & \foreignlanguage{greek}{ναϲεν και ιδων απο μακροθεν ϲυκην} & 5 & \textbf{13} &  \\
&  & 6 & \foreignlanguage{greek}{εχουϲαν φυλλα ηλθεν ειϲ αυτην ει αρα} & 12 &  &  \\
&  & 13 & \foreignlanguage{greek}{τι ευρηϲει ειϲ αυτην και ελθων επ αυ} & 20 &  &  \\
&  & 20 & \foreignlanguage{greek}{την ουδεν ευρεν ει μη φυλλα μονον} & 26 &  &  \\
&  & 27 & \foreignlanguage{greek}{ου γαρ ην καιροϲ ϲυκων και αποκρι} & 2 & \textbf{14} &  \\
&  & 2 & \foreignlanguage{greek}{θειϲ ειπεν αυτη ο \textoverline{ιϲ} μηκετι ειϲ τον αιωνα} & 10 &  &  \\
&  & 11 & \foreignlanguage{greek}{εκ ϲου καρπον μηδειϲ φαγη ϗ ηκουϲαν} & 17 &  &  \\
[0.2em]
\cline{4-4}
\end{tabular}
\end{center}
\end{table}
}
\clearpage
\newpage
 {
 \setlength\arrayrulewidth{1pt}
\begin{table}
\begin{center}
\begin{tabular}{ccc|l|ccc}
\cline{4-4} \\ [-1em]
\multicolumn{7}{c}{\foreignlanguage{greek}{ευαγγελιον κατα μαρκον} \textbf{(\nospace{11:14})} } \\ \\ [-1em] % Si on veut ajouter les bordures latérales, remplacer {7}{c} par {7}{|c|}
\cline{4-4} \\
\cline{4-4}
&  &  & &  &  & \\ [-0.9em]
&  & 18 & \foreignlanguage{greek}{οι μαθηται αυτου και ερχονται ειϲ ιεροϲο} & 4 & \textbf{15} &  \\
&  & 4 & \foreignlanguage{greek}{λυμα και ειϲελθων ειϲ το ιερον ηρξατο εκ} & 11 &  &  \\
&  & 11 & \foreignlanguage{greek}{βαλλειν τουϲ πωλουνταϲ εν τω ιερω και} & 18 &  &  \\
&  & 19 & \foreignlanguage{greek}{ταϲ τραπεζαϲ των κολλυβιϲτων εξεχεε̅} & 23 &  &  \\
&  & 24 & \foreignlanguage{greek}{και ταϲ καθεδραϲ των πωλουντων ταϲ} & 29 &  &  \\
&  & 30 & \foreignlanguage{greek}{περιϲτεραϲ κατεϲτρεψεν και ουκ ηφι} & 3 & \textbf{16} &  \\
&  & 3 & \foreignlanguage{greek}{εν ινα τιϲ διενεγκη ϲκευοϲ δια του ιερου} & 10 &  &  \\
& \textbf{17} &  & \foreignlanguage{greek}{και εδιδαϲκεν λεγων αυτοιϲ ου γεγραπται} & 6 &  &  \\
&  & 7 & \foreignlanguage{greek}{οτι ο οικοϲ μου οικοϲ προϲευχηϲ κληθηϲετε} & 13 &  &  \\
&  & 14 & \foreignlanguage{greek}{παϲι τοιϲ εθνεϲι υμειϲ δε εποιηϲατε αυτο̅} & 20 &  &  \\
&  & 21 & \foreignlanguage{greek}{ϲπηλεον ληϲτων και ηκουϲαν οι αρχι} & 4 & \textbf{18} &  \\
&  & 4 & \foreignlanguage{greek}{ερειϲ και οι γραμματιϲ και εζητουν πωϲ} & 10 &  &  \\
&  & 11 & \foreignlanguage{greek}{αυτον απολεϲωϲιν εφοβουντο γαρ αυτο̅} & 15 &  &  \\
&  & 16 & \foreignlanguage{greek}{παϲ γαρ ο οχλοϲ εξεπληϲϲετο επι τη διδα} & 23 &  &  \\
&  & 23 & \foreignlanguage{greek}{χη αυτου και οταν οψε εγεινετο εξω} & 5 & \textbf{19} &  \\
&  & 6 & \foreignlanguage{greek}{τηϲ πολεωϲ εξεπορευοντο και παραπο} & 2 & \textbf{20} &  \\
&  & 2 & \foreignlanguage{greek}{ρευομενοι πρωει ειδον την ϲυκην εξη} & 7 &  &  \\
&  & 7 & \foreignlanguage{greek}{ρανμενην εκ ριζων και αναμνηϲθειϲ} & 2 & \textbf{21} &  \\
&  & 3 & \foreignlanguage{greek}{ο πετροϲ λεγει αυτω ραββει ειδε η ϲυκη} & 10 &  &  \\
&  & 11 & \foreignlanguage{greek}{ην κατηραϲω εξηρανται αποκριθειϲ} & 1 & \textbf{22} &  \\
&  & 2 & \foreignlanguage{greek}{ο \textoverline{ιϲ} λεγει αυτοιϲ εχεται πιϲτιν του \textoverline{θυ}} & 9 &  &  \\
& \textbf{23} &  & \foreignlanguage{greek}{αμην γαρ λεγω υμιν οϲ αν ειπη τω ορι} & 9 &  &  \\
&  & 10 & \foreignlanguage{greek}{τουτω αρθηναι και βληθηναι ειϲ την θα} & 16 &  &  \\
&  & 16 & \foreignlanguage{greek}{λαϲϲαν και μη διακριθη εν τη καρδια} & 22 &  &  \\
&  & 23 & \foreignlanguage{greek}{αυτου αλλα πιϲτευϲη οτι α λεγει γινεται} & 29 &  &  \\
&  & 30 & \foreignlanguage{greek}{εϲται αυτω δια τουτο λεγω υμιν παν} & 5 & \textbf{24} &  \\
&  & 5 & \foreignlanguage{greek}{τα οϲα προϲευχομενοι αιτιϲθαι πιϲτευ} & 9 &  &  \\
&  & 9 & \foreignlanguage{greek}{ετε οτι ελαβετε και εϲται υμιν και ο} & 2 & \textbf{25} &  \\
&  & 2 & \foreignlanguage{greek}{ταν ϲτηκηται προϲευχομενοι αφιετε} & 5 &  &  \\
&  & 6 & \foreignlanguage{greek}{ει τι εχετε κατα τινοϲ ινα και ο \textoverline{πηρ} υ} & 15 &  &  \\
[0.2em]
\cline{4-4}
\end{tabular}
\end{center}
\end{table}
}
\clearpage
\newpage
 {
 \setlength\arrayrulewidth{1pt}
\begin{table}
\begin{center}
\begin{tabular}{ccc|l|ccc}
\cline{4-4} \\ [-1em]
\multicolumn{7}{c}{\foreignlanguage{greek}{ευαγγελιον κατα μαρκον} \textbf{(\nospace{11:25})} } \\ \\ [-1em] % Si on veut ajouter les bordures latérales, remplacer {7}{c} par {7}{|c|}
\cline{4-4} \\
\cline{4-4}
&  &  & &  &  & \\ [-0.9em]
&  & 15 & \foreignlanguage{greek}{μων ο εν τοιϲ ουρανοιϲ ανη υμιν τα παρα} & 23 &  &  \\
&  & 23 & \foreignlanguage{greek}{πτωματα υμων} & 24 &  &  \\
& \textbf{27} &  & \foreignlanguage{greek}{ιεροϲολυμα και εν τω ιερω περιπατουν} & 10 &  &  \\
&  & 10 & \foreignlanguage{greek}{τοϲ αυτου ερχονται προϲ αυτον οι αρχιερειϲ} & 16 &  &  \\
&  & 17 & \foreignlanguage{greek}{και οι γραμματειϲ και οι πρεϲβυτεροι και} & 1 & \textbf{28} &  \\
&  & 2 & \foreignlanguage{greek}{ελεγον αυτω εν ποια εξουϲια ταυτα ποιειϲ} & 8 &  &  \\
&  & 9 & \foreignlanguage{greek}{και τιϲ ϲοι ταυτην την εξουϲιαν εδωκε̅} & 15 &  &  \\
& \textbf{29} &  & \foreignlanguage{greek}{ο δε \textoverline{ιϲ} αποκριθειϲ ειπεν αυτοιϲ επερωτω} & 7 &  &  \\
&  & 8 & \foreignlanguage{greek}{υμαϲ καγω ενα λογον αποκριθητε μοι} & 13 &  &  \\
&  & 14 & \foreignlanguage{greek}{και ερω υμιν εν τινι εξουϲια ταυτα ποιω} & 21 &  &  \\
& \textbf{30} &  & \foreignlanguage{greek}{το βαπτιϲμα ιωαννου απ ουρανου ην η εξ} & 8 &  &  \\
&  & 9 & \foreignlanguage{greek}{\textoverline{ανων} αποκριθηται μοι και διελογιζο̅} & 2 & \textbf{31} &  \\
&  & 2 & \foreignlanguage{greek}{το προϲ αυτουϲ λεγοντεϲ οτι εαν ειπω} & 8 &  &  \\
&  & 8 & \foreignlanguage{greek}{μεν εξ ουρανου ερει ημιν δια τι ουκ επι} & 16 &  &  \\
&  & 16 & \foreignlanguage{greek}{ϲτευϲατε αυτω αλλ εαν ειπωμεν εξ} & 4 & \textbf{32} &  \\
&  & 5 & \foreignlanguage{greek}{\textoverline{ανων} φοβουμεθα τον λαον παντεϲ γαρ} & 10 &  &  \\
&  & 11 & \foreignlanguage{greek}{ηδιϲαν τον ιωαννην οτι οντωϲ προφη} & 16 &  &  \\
&  & 16 & \foreignlanguage{greek}{τηϲ ην και αποκριθεντεϲ τω \textoverline{ιυ} λεγουϲι̅} & 5 & \textbf{33} &  \\
&  & 6 & \foreignlanguage{greek}{ουκ οιδομεν και ο \textoverline{ιϲ} αποκριθειϲ λε} & 12 &  &  \\
&  & 12 & \foreignlanguage{greek}{γει αυτοιϲ ουδε εγω λεγω υμιν εν ποια} & 19 &  &  \\
&  & 20 & \foreignlanguage{greek}{εξουϲια ταυτα ποιω και ηρξατο αυτοιϲ} & 3 & \mygospelchapter &  \\
&  & 4 & \foreignlanguage{greek}{εν παραβολαιϲ λαλειν \textoverline{ανοϲ} τιϲ εφυ} & 9 &  &  \\
&  & 9 & \foreignlanguage{greek}{τευϲεν αμπελωνα και περιεθηκεν αυ} & 13 &  &  \\
&  & 13 & \foreignlanguage{greek}{τω φραγμον εξωρυξεν υποληνιον} & 16 &  &  \\
&  & 17 & \foreignlanguage{greek}{και ωκοδομηϲεν πυργον και εξεδοτο} & 21 &  &  \\
&  & 22 & \foreignlanguage{greek}{αυτον γεωργοιϲ και απεδημηϲεν} & 25 &  &  \\
& \textbf{2} &  & \foreignlanguage{greek}{και απεϲτιλεν τω καιρω δουλον ινα πα} & 7 &  &  \\
&  & 7 & \foreignlanguage{greek}{ρα των γεωργων λαβη απο του καρπου} & 13 &  &  \\
&  & 14 & \foreignlanguage{greek}{του αμπελωνοϲ οι δε λαβοντεϲ αυτο̅} & 4 & \textbf{3} &  \\
&  & 5 & \foreignlanguage{greek}{εδιραν και απεκτιναν ϗ απεϲτιλαν κενο̅} & 10 &  &  \\
[0.2em]
\cline{4-4}
\end{tabular}
\end{center}
\end{table}
}
\clearpage
\newpage
 {
 \setlength\arrayrulewidth{1pt}
\begin{table}
\begin{center}
\begin{tabular}{ccc|l|ccc}
\cline{4-4} \\ [-1em]
\multicolumn{7}{c}{\foreignlanguage{greek}{ευαγγελιον κατα μαρκον} \textbf{(\nospace{12:4})} } \\ \\ [-1em] % Si on veut ajouter les bordures latérales, remplacer {7}{c} par {7}{|c|}
\cline{4-4} \\
\cline{4-4}
&  &  & &  &  & \\ [-0.9em]
& \textbf{4} &  & \foreignlanguage{greek}{και απεϲτιλεν προϲ αυτουϲ αλλον δουλον} & 6 &  &  \\
&  & 7 & \foreignlanguage{greek}{κακεινον κεφαλεωϲαντεϲ απεϲτιλαν} & 9 &  &  \\
&  & 10 & \foreignlanguage{greek}{ητιμαϲμενον και παλιν αλλον απεϲτι} & 4 & \textbf{5} &  \\
&  & 4 & \foreignlanguage{greek}{λεν και πολλουϲ αλλουϲ τουϲ δε δερον} & 10 &  &  \\
&  & 10 & \foreignlanguage{greek}{τεϲ τουϲ δε αποκτινοντεϲ υϲτερον δε} & 2 & \textbf{6} &  \\
&  & 3 & \foreignlanguage{greek}{ενα υιον εχων τον αγαπητον αυτου απε} & 9 &  &  \\
&  & 9 & \foreignlanguage{greek}{ϲτιλεν προϲ αυτουϲ εϲχατον λεγων} & 13 &  &  \\
&  & 14 & \foreignlanguage{greek}{εντραπηϲονται τον υιον μου εκεινοι} & 1 & \textbf{7} &  \\
&  & 2 & \foreignlanguage{greek}{δε οι γεωργοι προϲ εαυτουϲ ειπαν οτι ου} & 9 &  &  \\
&  & 9 & \foreignlanguage{greek}{τοϲ εϲτιν ο κληρονομοϲ δευτε αποκτι} & 14 &  &  \\
&  & 14 & \foreignlanguage{greek}{νωμεν αυτον και ημων εϲται η κληρο} & 20 &  &  \\
&  & 20 & \foreignlanguage{greek}{νομια και λαβοντεϲ αυτον απεκτινα̅} & 4 & \textbf{8} &  \\
&  & 5 & \foreignlanguage{greek}{και εξεβαλον εξω του αμπελωνοϲ} & 9 &  &  \\
& \textbf{9} &  & \foreignlanguage{greek}{τι ουν ποιηϲει ο \textoverline{κϲ} του αμπελωνοϲ ελευ} & 8 &  &  \\
&  & 8 & \foreignlanguage{greek}{ϲεται και απολεϲει τουϲ γεωργουϲ και δω} & 14 &  &  \\
&  & 14 & \foreignlanguage{greek}{ϲει τον αμπελωνα αλλοιϲ ουδε την γρα} & 3 & \textbf{10} &  \\
&  & 3 & \foreignlanguage{greek}{φην ταυτην ανεγνωκατε λιθον ον α} & 8 &  &  \\
&  & 8 & \foreignlanguage{greek}{πεδοκειμαϲαν οι οικοδομουντεϲ ουτοϲ} & 11 &  &  \\
&  & 12 & \foreignlanguage{greek}{εγενηθη ειϲ κεφαλην γωνιαϲ παρα \textoverline{κυ}} & 2 & \textbf{11} &  \\
&  & 3 & \foreignlanguage{greek}{εγενετο αυτη και εϲτιν θαυμαϲτη ε̅} & 8 &  &  \\
&  & 9 & \foreignlanguage{greek}{οφθαλμοιϲ ημων και εζητουν αυτον} & 3 & \textbf{12} &  \\
&  & 4 & \foreignlanguage{greek}{κρατηϲαι και εφοβηθηϲαν τον οχλον} & 8 &  &  \\
&  & 9 & \foreignlanguage{greek}{εγνωϲαν γαρ οτι προϲ αυτουϲ την παρα} & 15 &  &  \\
&  & 15 & \foreignlanguage{greek}{βολην ειπεν και αποϲτελλουϲιν προϲ} & 3 & \textbf{13} &  \\
&  & 4 & \foreignlanguage{greek}{αυτον τιναϲ των φαριϲαιων και των} & 9 &  &  \\
&  & 10 & \foreignlanguage{greek}{ηρωδιανων ινα αυτον αγρευϲωϲι λογω} & 14 &  &  \\
& \textbf{14} &  & \foreignlanguage{greek}{οι δε ελθοντεϲ ηρξαντο ερωταν αυτο̅} & 6 &  &  \\
&  & 7 & \foreignlanguage{greek}{εν δολω διδαϲκαλε οιδαμεν οτι α} & 12 &  &  \\
&  & 12 & \foreignlanguage{greek}{ληθηϲ ει και μελει ϲοι περι ουδενοϲ ου} & 19 &  &  \\
&  & 20 & \foreignlanguage{greek}{γαρ βλεπειϲ ειϲ προϲωπον ανθρωπων} & 24 &  &  \\
[0.2em]
\cline{4-4}
\end{tabular}
\end{center}
\end{table}
}
\clearpage
\newpage
 {
 \setlength\arrayrulewidth{1pt}
\begin{table}
\begin{center}
\begin{tabular}{ccc|l|ccc}
\cline{4-4} \\ [-1em]
\multicolumn{7}{c}{\foreignlanguage{greek}{ευαγγελιον κατα μαρκον} \textbf{(\nospace{12:14})} } \\ \\ [-1em] % Si on veut ajouter les bordures latérales, remplacer {7}{c} par {7}{|c|}
\cline{4-4} \\
\cline{4-4}
&  &  & &  &  & \\ [-0.9em]
&  & 25 & \foreignlanguage{greek}{αλλ επ αληθειαϲ την οδον του \textoverline{θυ} διδαϲκειϲ} & 32 &  &  \\
&  & 33 & \foreignlanguage{greek}{ειπον ουν ημιν εξεϲτιν δουναι κηνϲον καιϲαρι} & 39 &  &  \\
&  & 40 & \foreignlanguage{greek}{η ου δωμεν η μη δωμεν} & 45 &  &  \\
& \textbf{15} &  & \foreignlanguage{greek}{ο δε ειδωϲ αυτων την υποκριϲιν ειπεν} & 7 &  &  \\
&  & 8 & \foreignlanguage{greek}{αυτοιϲ τι με πειραζετε υποκριται φε} & 13 &  &  \\
&  & 13 & \foreignlanguage{greek}{ρετε μοι δηναριον ινα ειδω οι δε ηνεγκα̅} & 3 & \textbf{16} &  \\
&  & 4 & \foreignlanguage{greek}{και λεγει αυτοιϲ τινοϲ η εικων αυτη και} & 11 &  &  \\
&  & 12 & \foreignlanguage{greek}{η επιγραφη οι δε ειπαν καιϲαροϲ} & 17 &  &  \\
& \textbf{17} &  & \foreignlanguage{greek}{και αποκριθειϲ ειπεν αυτοιϲ τα καιϲα} & 6 &  &  \\
&  & 6 & \foreignlanguage{greek}{ροϲ αποδοτε καιϲαρι και τα του \textoverline{θυ} τω \textoverline{θω}} & 14 &  &  \\
&  & 15 & \foreignlanguage{greek}{και εθαυμαϲαν επ αυτω και ερχονται} & 2 & \textbf{18} &  \\
&  & 3 & \foreignlanguage{greek}{ϲαδδουκεοι προϲ αυτον οιτινεϲ λεγουϲι̅} & 7 &  &  \\
&  & 8 & \foreignlanguage{greek}{αναϲταϲιν μη ειναι και επηρωτηϲαν} & 12 &  &  \\
&  & 13 & \foreignlanguage{greek}{αυτον λεγοντεϲ διδαϲκαλε μωυϲηϲ} & 2 & \textbf{19} &  \\
&  & 3 & \foreignlanguage{greek}{εγραψεν ημιν οτι εαν τινοϲ αδελφοϲ α} & 9 &  &  \\
&  & 9 & \foreignlanguage{greek}{ποθανη και εχη γυναικα και τεκνον} & 14 &  &  \\
&  & 15 & \foreignlanguage{greek}{μη αφη ινα λαβη ο αδελφοϲ την γυναικα} & 22 &  &  \\
&  & 23 & \foreignlanguage{greek}{και εξαναϲτηϲη ϲπερμα τω αδελφω αυτου} & 28 &  &  \\
& \textbf{20} &  & \foreignlanguage{greek}{επτα αδελφοι ηϲαν και ο πρωτοϲ ελα} & 7 &  &  \\
&  & 7 & \foreignlanguage{greek}{βεν γυναικα και απεθανεν και ουκ α} & 13 &  &  \\
&  & 13 & \foreignlanguage{greek}{φηκεν ϲπερμα ο δευτεροϲ ελαβεν} & 3 & \textbf{21} &  \\
&  & 4 & \foreignlanguage{greek}{αυτην και ουδε αυτοϲ αφηκεν ϲπερμα} & 9 &  &  \\
&  & 10 & \foreignlanguage{greek}{ο τριτοϲ ωϲαυτωϲ οι \textoverline{ζ} και ουκ αφηκα̅} & 5 & \textbf{22} &  \\
&  & 6 & \foreignlanguage{greek}{ϲπερμα εϲχατον παντων η γυνη α} & 11 &  &  \\
&  & 11 & \foreignlanguage{greek}{πεθανεν εν τη αναϲταϲι ουν αυτω̅} & 5 & \textbf{23} &  \\
&  & 6 & \foreignlanguage{greek}{τινοϲ εϲται γυνη οι γαρ \textoverline{ζ} εϲχον αυτη̅} & 13 &  &  \\
&  & 14 & \foreignlanguage{greek}{γυναικα αποκριθειϲ δε ο \textoverline{ιϲ} ει} & 5 & \textbf{24} &  \\
&  & 5 & \foreignlanguage{greek}{πεν αυτοιϲ ου δια τουτο πλαναϲθαι} & 10 &  &  \\
&  & 11 & \foreignlanguage{greek}{μη ειδοτεϲ ταϲ γραφαϲ μηδε την δυ} & 17 &  &  \\
&  & 17 & \foreignlanguage{greek}{ναμιν του \textoverline{θυ} οταν γαρ εκ νεκρων} & 4 & \textbf{25} &  \\
[0.2em]
\cline{4-4}
\end{tabular}
\end{center}
\end{table}
}
\clearpage
\newpage
 {
 \setlength\arrayrulewidth{1pt}
\begin{table}
\begin{center}
\begin{tabular}{ccc|l|ccc}
\cline{4-4} \\ [-1em]
\multicolumn{7}{c}{\foreignlanguage{greek}{ευαγγελιον κατα μαρκον} \textbf{(\nospace{12:25})} } \\ \\ [-1em] % Si on veut ajouter les bordures latérales, remplacer {7}{c} par {7}{|c|}
\cline{4-4} \\
\cline{4-4}
&  &  & &  &  & \\ [-0.9em]
&  & 5 & \foreignlanguage{greek}{αναϲτωϲιν ουτε γαμουϲιν ουτε γαμι} & 9 &  &  \\
&  & 9 & \foreignlanguage{greek}{ϲκοντε αλλ ειϲιν ωϲ οι αγγελοι οι εν τοιϲ ου} & 18 &  &  \\
&  & 18 & \foreignlanguage{greek}{ρανοιϲ περι δε των νεκρων ει εγειρο̅} & 6 & \textbf{26} &  \\
&  & 6 & \foreignlanguage{greek}{τε ουκ ανεγνωκατε εν τη βιβλω μωυ} & 12 &  &  \\
&  & 12 & \foreignlanguage{greek}{ϲεωϲ επι τηϲ βατου ωϲ ειπεν ο \textoverline{θϲ} λεγω̅} & 20 &  &  \\
&  & 21 & \foreignlanguage{greek}{αυτω εγω \textoverline{θϲ} αβρααμ και \textoverline{θϲ} ιϲαακ} & 27 &  &  \\
&  & 28 & \foreignlanguage{greek}{και \textoverline{θϲ} ιακωβ ουκ εϲτιν \textoverline{θϲ} νεκρων} & 4 & \textbf{27} &  \\
&  & 5 & \foreignlanguage{greek}{αλλα ζωντων πολυ πλαναϲθαι} & 8 &  &  \\
& \textbf{28} &  & \foreignlanguage{greek}{και προελθων ειϲ των γραμματεων} & 5 &  &  \\
&  & 6 & \foreignlanguage{greek}{ακουων αυτων ϲυνζητουντων ιδω̅} & 9 &  &  \\
&  & 10 & \foreignlanguage{greek}{οτι καλωϲ απεκριθη αυτοιϲ επηρωτη} & 14 &  &  \\
&  & 14 & \foreignlanguage{greek}{ϲεν αυτον ποια εϲτιν πρωτη εντολη} & 19 &  &  \\
& \textbf{29} &  & \foreignlanguage{greek}{ο δε ειπεν αυτω παντων πρωτη α} & 7 &  &  \\
&  & 7 & \foreignlanguage{greek}{κουε ιϲτραηλ \textoverline{κϲ} ο \textoverline{θϲ} ημων \textoverline{κϲ} εϲτιν} & 14 &  &  \\
& \textbf{30} &  & \foreignlanguage{greek}{και αγαπηϲιϲ \textoverline{κν} τον \textoverline{θν} ϲου εξ οληϲ τηϲ} & 9 &  &  \\
&  & 10 & \foreignlanguage{greek}{καρδιαϲ ϲου και εξ οληϲ τηϲ ψυχηϲ ϲου} & 17 &  &  \\
&  & 18 & \foreignlanguage{greek}{και εξ οληϲ τηϲ διανοιαϲ ϲου και εξ οληϲ} & 26 &  &  \\
&  & 27 & \foreignlanguage{greek}{τηϲ ιϲχυοϲ ϲου αυτη πρωτη και δευτερα} & 2 & \textbf{31} &  \\
&  & 3 & \foreignlanguage{greek}{ομοιωϲ αυτη αγαπηϲιϲ τον πληϲιον} & 7 &  &  \\
&  & 8 & \foreignlanguage{greek}{ϲου ωϲ ϲεαυτον μιζων τουτων αλλη} & 13 &  &  \\
&  & 14 & \foreignlanguage{greek}{εντολη ουκ εϲτιν και ειπεν αυτω} & 3 & \textbf{32} &  \\
&  & 4 & \foreignlanguage{greek}{ο γραμματευϲ καλωϲ διδαϲκαλε επ α} & 9 &  &  \\
&  & 9 & \foreignlanguage{greek}{ληθειαϲ ειπαϲ οτι ειϲ \textoverline{θϲ} εϲτιν και ου} & 16 &  &  \\
&  & 16 & \foreignlanguage{greek}{κ εϲτιν αλλοϲ πλην αυτου και το αγα} & 3 & \textbf{33} &  \\
&  & 3 & \foreignlanguage{greek}{παν αυτον εξ οληϲ τηϲ καρδιαϲ και το εξ} & 11 &  &  \\
&  & 12 & \foreignlanguage{greek}{οληϲ τηϲ ϲυνεϲεωϲ και εξ οληϲ τηϲ ιϲχυ} & 19 &  &  \\
&  & 19 & \foreignlanguage{greek}{οϲ και το αγαπαν τον πληϲιον ϲου ωϲ ϲε} & 27 &  &  \\
&  & 27 & \foreignlanguage{greek}{αυτον πλιον εϲτιν παντων ολοκαυτω} & 31 &  &  \\
&  & 31 & \foreignlanguage{greek}{ματων και θυϲιων και ο \textoverline{ιϲ} ιδων οτι} & 5 & \textbf{34} &  \\
&  & 6 & \foreignlanguage{greek}{νουνεχωϲ απεκριθη ειπεν αυτω} & 9 &  &  \\
[0.2em]
\cline{4-4}
\end{tabular}
\end{center}
\end{table}
}
\clearpage
\newpage
 {
 \setlength\arrayrulewidth{1pt}
\begin{table}
\begin{center}
\begin{tabular}{ccc|l|ccc}
\cline{4-4} \\ [-1em]
\multicolumn{7}{c}{\foreignlanguage{greek}{ευαγγελιον κατα μαρκον} \textbf{(\nospace{12:34})} } \\ \\ [-1em] % Si on veut ajouter les bordures latérales, remplacer {7}{c} par {7}{|c|}
\cline{4-4} \\
\cline{4-4}
&  &  & &  &  & \\ [-0.9em]
&  & 10 & \foreignlanguage{greek}{οτι ου μακραν ει απο τηϲ βαϲειλιαϲ του \textoverline{θυ}} & 18 &  &  \\
&  & 19 & \foreignlanguage{greek}{και ουδειϲ ετολμα αυτον ουκετι επερωταν} & 24 &  &  \\
& \textbf{35} &  & \foreignlanguage{greek}{και αποκριθειϲ λεγει διδαϲκων εν τω ιερω} & 7 &  &  \\
&  & 8 & \foreignlanguage{greek}{πωϲ λεγουϲιν οι γραμματιϲ οτι \textoverline{χϲ} \textoverline{υϲ} εϲτιν \textoverline{δδ}} & 16 &  &  \\
& \textbf{36} &  & \foreignlanguage{greek}{αυτοϲ \textoverline{δαδ} ειπεν εν \textoverline{πνι} αγιω ειπεν ο \textoverline{κϲ} τω} & 10 &  &  \\
&  & 11 & \foreignlanguage{greek}{\textoverline{κω} μου καθου εκ δεξιων μου εωϲ αν θω τουϲ} & 20 &  &  \\
&  & 21 & \foreignlanguage{greek}{εκχθρουϲ ϲου υποκατω των ποδων ϲου} & 26 &  &  \\
& \textbf{37} &  & \foreignlanguage{greek}{αυτοϲ \textoverline{δαδ} λεγει αυτον \textoverline{κν} και πωϲ \textoverline{υϲ} αυτου} & 9 &  &  \\
&  & 10 & \foreignlanguage{greek}{εϲτιν και πολυϲ οχλοϲ ηκουεν αυτου ηδε} & 16 &  &  \\
&  & 16 & \foreignlanguage{greek}{ωϲ και ελεγεν εν τη διδαχη αυτου βλε} & 7 & \textbf{38} &  \\
&  & 7 & \foreignlanguage{greek}{πετε απο των γραμματεων των θελοντω̅} & 12 &  &  \\
&  & 13 & \foreignlanguage{greek}{εν ταιϲ ϲτολαιϲ περιπατειν και αϲπαϲμουϲ} & 18 &  &  \\
&  & 19 & \foreignlanguage{greek}{εν ταιϲ αγοραιϲ και πρωτοκαθεδριαϲ εν} & 3 & \textbf{39} &  \\
&  & 4 & \foreignlanguage{greek}{ταιϲ ϲυναγωγαιϲ και πρωτοκλιϲιαϲ εν} & 8 &  &  \\
&  & 9 & \foreignlanguage{greek}{τοιϲ διπνοιϲ οι κατεϲθιοντεϲ οικειαϲ χηρω̅} & 4 & \textbf{40} &  \\
&  & 5 & \foreignlanguage{greek}{και ορφανων και προφαϲι μακρα προϲευ} & 10 &  &  \\
&  & 10 & \foreignlanguage{greek}{χομενοι οιτινεϲ λημψονται περιϲϲον κρι} & 14 &  &  \\
&  & 14 & \foreignlanguage{greek}{μα και εϲτωϲ ο \textoverline{ιϲ} κατεναντι του γαζο} & 7 & \textbf{41} &  \\
&  & 7 & \foreignlanguage{greek}{φυλακιου εθεωρι πανταϲ πωϲ ο οχλοϲ} & 12 &  &  \\
&  & 13 & \foreignlanguage{greek}{βαλλει τον χαλκον ειϲ το γαζοφυλακιο̅} & 18 &  &  \\
&  & 19 & \foreignlanguage{greek}{και πολλοι πλουϲιοι εβαλλον πολλα} & 23 &  &  \\
& \textbf{42} &  & \foreignlanguage{greek}{και ελθουϲα μια χηρα πτωχη εβαλεν λε} & 7 &  &  \\
&  & 7 & \foreignlanguage{greek}{πτα δυο ο εϲτιν κοδραντηϲ και προϲκα} & 2 & \textbf{43} &  \\
&  & 2 & \foreignlanguage{greek}{λεϲαμενοϲ τουϲ μαθηταϲ λεγει αυτοιϲ} & 6 &  &  \\
&  & 7 & \foreignlanguage{greek}{αμην λεγω υμιν η χηρα αυτη η πτωχη} & 14 &  &  \\
&  & 15 & \foreignlanguage{greek}{πλιον παντων βεβληκεν ειϲ το γαζοφυ} & 20 &  &  \\
&  & 20 & \foreignlanguage{greek}{λακιον παντεϲ γαρ εκ του περιϲϲευμα} & 5 & \textbf{44} &  \\
&  & 5 & \foreignlanguage{greek}{τοϲ αυτων εβαλον αυτη δε εκ τηϲ υϲτε} & 12 &  &  \\
&  & 12 & \foreignlanguage{greek}{ρηϲεωϲ αυτηϲ εβαλεν ολον τον βιον αυτηϲ} & 18 &  &  \\
& \mygospelchapter &  & \foreignlanguage{greek}{και εκπορευομενου αυτου εκ του ιερου} & 6 &  &  \\
[0.2em]
\cline{4-4}
\end{tabular}
\end{center}
\end{table}
}
\clearpage
\newpage
 {
 \setlength\arrayrulewidth{1pt}
\begin{table}
\begin{center}
\begin{tabular}{ccc|l|ccc}
\cline{4-4} \\ [-1em]
\multicolumn{7}{c}{\foreignlanguage{greek}{ευαγγελιον κατα μαρκον} \textbf{(\nospace{13:1})} } \\ \\ [-1em] % Si on veut ajouter les bordures latérales, remplacer {7}{c} par {7}{|c|}
\cline{4-4} \\
\cline{4-4}
&  &  & &  &  & \\ [-0.9em]
&  & 7 & \foreignlanguage{greek}{λεγει αυτω ειϲ των μαθητων αυτου δι} & 13 &  &  \\
&  & 13 & \foreignlanguage{greek}{δαϲκαλε ποταποι λιθοι και ποταπε οικοδο} & 18 &  &  \\
&  & 18 & \foreignlanguage{greek}{μαι και αποκριθειϲ ειπεν αυτω βλεπειϲ} & 5 & \textbf{2} &  \\
&  & 6 & \foreignlanguage{greek}{ταυταϲ ταϲ μεγαλαϲ οικοδομαϲ ου μη} & 11 &  &  \\
&  & 12 & \foreignlanguage{greek}{αφεθη ωδε λιθοϲ επι λιθον οϲ ου μη αφε} & 20 &  &  \\
&  & 20 & \foreignlanguage{greek}{θη ουδε διαλυθηϲεται και δια τριων η} & 26 &  &  \\
&  & 26 & \foreignlanguage{greek}{μερων αλλοϲ αναϲτηϲεται ανευ χειρω̅} & 30 &  &  \\
& \textbf{3} &  & \foreignlanguage{greek}{καθημενου δε αυτου ειϲ το οροϲ των ε} & 8 &  &  \\
&  & 8 & \foreignlanguage{greek}{λεων κατεναντι του ιερου επηρωτα} & 12 &  &  \\
&  & 13 & \foreignlanguage{greek}{αυτον κατ ιδιαν πετροϲ και ιακωβοϲ} & 18 &  &  \\
&  & 19 & \foreignlanguage{greek}{και ιωαννηϲ και ανδρεαϲ ειπον ημιν} & 2 & \textbf{4} &  \\
&  & 3 & \foreignlanguage{greek}{ποτε ταυτα εϲται και τι το ϲημιον οτα̅} & 10 &  &  \\
&  & 11 & \foreignlanguage{greek}{μελλη ταυτα ϲυντελειϲθαι} & 13 &  &  \\
& \textbf{5} &  & \foreignlanguage{greek}{και αποκριθειϲ αυτοιϲ ο \textoverline{ιϲ} ηρξατο λεγει̅} & 7 &  &  \\
&  & 8 & \foreignlanguage{greek}{βλεπεται μη τιϲ υμαϲ πλανηϲη πολλοι} & 1 & \textbf{6} &  \\
&  & 2 & \foreignlanguage{greek}{ελευϲονται επι τω ονοματι μου λεγον} & 7 &  &  \\
&  & 7 & \foreignlanguage{greek}{τεϲ οτι εγω ειμει ο \textoverline{χϲ} και πολλουϲ πλα} & 15 &  &  \\
&  & 15 & \foreignlanguage{greek}{νηϲουϲιν οταν δε ακουϲηται πολεμουϲ} & 4 & \textbf{7} &  \\
&  & 5 & \foreignlanguage{greek}{και ακοαϲ πολεμων μη θροειϲθαι δει} & 10 &  &  \\
&  & 11 & \foreignlanguage{greek}{γενεϲθαι αλλ ουπω το τελοϲ εγερθη} & 1 & \textbf{8} &  \\
&  & 1 & \foreignlanguage{greek}{ϲεται εθνοϲ επι εθνοϲ και βαϲιλεια επι} & 7 &  &  \\
&  & 8 & \foreignlanguage{greek}{βαϲειλιαν εϲονται ϲιϲμοι κατα τοπουϲ} & 12 &  &  \\
&  & 13 & \foreignlanguage{greek}{λιμοι ταραχαι και δωϲουϲιν υμαϲ ειϲ} & 4 & \textbf{9} &  \\
&  & 5 & \foreignlanguage{greek}{ϲυνεδρια και ειϲ ϲυναγωγαϲ δαρηϲεϲθε} & 9 &  &  \\
&  & 10 & \foreignlanguage{greek}{και επι ηγεμονων και βαϲιλεων ϲταθηϲε} & 15 &  &  \\
&  & 15 & \foreignlanguage{greek}{ϲθαι ενεκεν εμου ειϲ μαρτυριον αυτοιϲ} & 20 &  &  \\
& \textbf{10} &  & \foreignlanguage{greek}{και ειϲ παντα τα εθνη πρωτον δε δει} & 8 &  &  \\
&  & 9 & \foreignlanguage{greek}{κηρυχθηναι το ευαγγελιον οταν δε αγω} & 3 & \textbf{11} &  \\
&  & 3 & \foreignlanguage{greek}{ϲιν υμαϲ παραδιδοντεϲ μη προμεριμνα} & 7 &  &  \\
&  & 7 & \foreignlanguage{greek}{τε τι λαληϲηται αλλ ο αν δοθη υμιν εν ε} & 16 &  &  \\
[0.2em]
\cline{4-4}
\end{tabular}
\end{center}
\end{table}
}
\clearpage
\newpage
 {
 \setlength\arrayrulewidth{1pt}
\begin{table}
\begin{center}
\begin{tabular}{ccc|l|ccc}
\cline{4-4} \\ [-1em]
\multicolumn{7}{c}{\foreignlanguage{greek}{ευαγγελιον κατα μαρκον} \textbf{(\nospace{13:11})} } \\ \\ [-1em] % Si on veut ajouter les bordures latérales, remplacer {7}{c} par {7}{|c|}
\cline{4-4} \\
\cline{4-4}
&  &  & &  &  & \\ [-0.9em]
&  & 16 & \foreignlanguage{greek}{κεινη τη ωρα εκεινο λαλειται ου γαρ εϲται} & 23 &  &  \\
&  & 24 & \foreignlanguage{greek}{υμειϲ οι λαλουντεϲ αλλα το \textoverline{πνα} το αγιο̅} & 31 &  &  \\
& \textbf{12} &  & \foreignlanguage{greek}{παραδωϲει δε αδελφοϲ αδελφον ειϲ θα} & 6 &  &  \\
&  & 6 & \foreignlanguage{greek}{νατον και \textoverline{πηρ} τεκνον και αναϲτηϲο̅} & 11 &  &  \\
&  & 11 & \foreignlanguage{greek}{ται τεκνα επι γονειϲ και θανατωϲουϲι̅} & 16 &  &  \\
&  & 17 & \foreignlanguage{greek}{αυτουϲ και εϲεϲθαι μιϲουμενοι υπο πα̅} & 5 & \textbf{13} &  \\
&  & 5 & \foreignlanguage{greek}{των δια το ονομα μου ο δε υπομειναϲ} & 12 &  &  \\
&  & 13 & \foreignlanguage{greek}{ειϲ τελοϲ ϲωθηϲεται οταν δε ειδηται} & 3 & \textbf{14} &  \\
&  & 4 & \foreignlanguage{greek}{το βδελυγμα τηϲ ερημωϲεωϲ ϲτηκον ο} & 9 &  &  \\
&  & 9 & \foreignlanguage{greek}{που ου δει ο αναγινωϲκων νοειτω} & 14 &  &  \\
&  & 15 & \foreignlanguage{greek}{τοτε οι εν τη ιουδαια φευγετωϲαν ειϲ τα} & 22 &  &  \\
&  & 23 & \foreignlanguage{greek}{ορη ο δε επι του δωματοϲ μη καταβατω} & 7 & \textbf{15} &  \\
&  & 8 & \foreignlanguage{greek}{ειϲ την οικειαν μηδε ειϲελθατω αρε εκ} & 14 &  &  \\
&  & 15 & \foreignlanguage{greek}{τηϲ οικειαϲ αυτου τι και ο ειϲ τον αγρο̅} & 5 & \textbf{16} &  \\
&  & 6 & \foreignlanguage{greek}{ων μη επιϲτρεψατω ειϲ τα οπιϲω αρε τα} & 13 &  &  \\
&  & 14 & \foreignlanguage{greek}{ιματια αυτου ουαι δε ταιϲ εν γαϲτρι εχου} & 6 & \textbf{17} &  \\
&  & 6 & \foreignlanguage{greek}{ϲαιϲ και θηλαζουϲαιϲ εν εκειναιϲ ταιϲ} & 11 &  &  \\
&  & 12 & \foreignlanguage{greek}{ημεραιϲ προϲευχεϲθαι δε ινα μη γενη} & 5 & \textbf{18} &  \\
&  & 5 & \foreignlanguage{greek}{ται χειμωνοϲ εϲονται γαρ αι ημεραι εκει} & 5 & \textbf{19} &  \\
&  & 5 & \foreignlanguage{greek}{ναι θλιψειϲ οια ου γεγονεν τοιαυτη απ αρ} & 12 &  &  \\
&  & 12 & \foreignlanguage{greek}{χηϲ ηϲ εκτιϲεν ο \textoverline{θϲ} εωϲ του νυν και ου} & 21 &  &  \\
&  & 22 & \foreignlanguage{greek}{μη γενηται και ει μη εκολοβωϲεν ταϲ} & 5 & \textbf{20} &  \\
&  & 6 & \foreignlanguage{greek}{ημεραϲ ουκ αν εϲωθη παϲα ϲαρξ αλλα δι} & 13 &  &  \\
&  & 13 & \foreignlanguage{greek}{α τουϲ εκλεκτουϲ ουϲ εξελεξατο εκολο} & 18 &  &  \\
&  & 18 & \foreignlanguage{greek}{βωϲεν ταϲ ημεραϲ και τοτε εαν τιϲ υ} & 5 & \textbf{21} &  \\
&  & 5 & \foreignlanguage{greek}{μιν ειπη ειδου ωδε ο \textoverline{κϲ} ειδου εκει μη πι} & 14 &  &  \\
&  & 14 & \foreignlanguage{greek}{ϲτευεται εγερθηϲονται γαρ πολλοι ψευ} & 4 & \textbf{22} &  \\
&  & 4 & \foreignlanguage{greek}{δοχριϲτοι και ψευδοπροφηται και δωϲου} & 8 &  &  \\
&  & 8 & \foreignlanguage{greek}{ϲιν ϲημια και τερατα προϲ το πλαναν} & 14 &  &  \\
&  & 15 & \foreignlanguage{greek}{ει δυνατον και τουϲ εκλεκτουϲ} & 19 &  &  \\
[0.2em]
\cline{4-4}
\end{tabular}
\end{center}
\end{table}
}
\clearpage
\newpage
 {
 \setlength\arrayrulewidth{1pt}
\begin{table}
\begin{center}
\begin{tabular}{ccc|l|ccc}
\cline{4-4} \\ [-1em]
\multicolumn{7}{c}{\foreignlanguage{greek}{ευαγγελιον κατα μαρκον} \textbf{(\nospace{13:23})} } \\ \\ [-1em] % Si on veut ajouter les bordures latérales, remplacer {7}{c} par {7}{|c|}
\cline{4-4} \\
\cline{4-4}
&  &  & &  &  & \\ [-0.9em]
& \textbf{23} &  & \foreignlanguage{greek}{υμειϲ δε βλεπετε προειρηκα υμιν παν} & 6 &  &  \\
&  & 6 & \foreignlanguage{greek}{τα αλλα εν εκειναιϲ ταιϲ ημεραιϲ μετα} & 6 & \textbf{24} &  \\
&  & 7 & \foreignlanguage{greek}{την θλιψιν εκεινην ο ηλιοϲ ϲκοτιϲθη} & 12 &  &  \\
&  & 12 & \foreignlanguage{greek}{ϲεται και η ϲεληνη ου δωϲι το φεγγοϲ} & 19 &  &  \\
&  & 20 & \foreignlanguage{greek}{αυτηϲ και οι αϲτερεϲ εκ του ουρανου πε} & 7 & \textbf{25} &  \\
&  & 7 & \foreignlanguage{greek}{ϲουντε και αι δυναμειϲ εν τοιϲ ουρανοιϲ} & 13 &  &  \\
&  & 16 & \foreignlanguage{greek}{ϲαλευθηϲονται και τοτε οψονται το̅} & 4 & \textbf{26} &  \\
&  & 5 & \foreignlanguage{greek}{υιον του \textoverline{ανου} ερχομενον εν νεφελη} & 10 &  &  \\
&  & 11 & \foreignlanguage{greek}{μετα δυναμεωϲ πολληϲ και δοξηϲ} & 15 &  &  \\
& \textbf{27} &  & \foreignlanguage{greek}{και τοτε αποϲτελει τουϲ αγγελουϲ και ε} & 7 &  &  \\
&  & 7 & \foreignlanguage{greek}{πιϲυνϲτρεψουϲιν τουϲ εκλεκτουϲ εκ τω̅} & 11 &  &  \\
&  & 12 & \foreignlanguage{greek}{τεϲϲαρων ανεμων απ ακρου τηϲ γηϲ} & 17 &  &  \\
&  & 18 & \foreignlanguage{greek}{εωϲ ακρων ουρανων απο δε τηϲ ϲυκηϲ} & 4 & \textbf{28} &  \\
&  & 5 & \foreignlanguage{greek}{μαθετε την παραβολην οταν αυτηϲ ο} & 10 &  &  \\
&  & 11 & \foreignlanguage{greek}{κλαδοϲ απαλοϲ γενηται και εκφυη τα φυλ} & 17 &  &  \\
&  & 17 & \foreignlanguage{greek}{λα γινωϲκεται οτι εγγυϲ το θεροϲ εϲτι̅} & 23 &  &  \\
& \textbf{29} &  & \foreignlanguage{greek}{ουτωϲ και υμειϲ οταν ταυτα ειδητε γει} & 7 &  &  \\
&  & 7 & \foreignlanguage{greek}{νομενα γινωϲκετε οτι εγγυϲ εϲτιν επι} & 12 &  &  \\
&  & 13 & \foreignlanguage{greek}{θυραιϲ αμην δε λεγω υμιν οτι ου μη} & 7 & \textbf{30} &  \\
&  & 8 & \foreignlanguage{greek}{παρελθη η γενεα αυτη εωϲ παντα ταυτα} & 14 &  &  \\
&  & 15 & \foreignlanguage{greek}{γενηται ο ουρανοϲ και η γη παρελευϲετε} & 6 & \textbf{31} &  \\
&  & 7 & \foreignlanguage{greek}{οι δε λογοι μου ου μη παρελθωϲιν} & 13 &  &  \\
& \textbf{32} &  & \foreignlanguage{greek}{περι δε τηϲ ημεραϲ εκεινηϲ και τηϲ ωραϲ} & 8 &  &  \\
&  & 9 & \foreignlanguage{greek}{ουδειϲ οιδεν ουδε οι αγγελοι οι εν ουρα} & 16 &  &  \\
&  & 16 & \foreignlanguage{greek}{νω ουδε ο \textoverline{υϲ} ει μη ο \textoverline{πηρ} βλεπεται δε} & 2 & \textbf{33} &  \\
&  & 3 & \foreignlanguage{greek}{αγρυπνιτε και προϲευχεϲθαι ουκ οιδα} & 7 &  &  \\
&  & 7 & \foreignlanguage{greek}{τε γαρ ει μη ο \textoverline{πηρ} και ο υιοϲ ποτε ο κεροϲ} & 18 &  &  \\
& \textbf{34} &  & \foreignlanguage{greek}{ωϲπερ γαρ \textoverline{ανοϲ} αποδημοϲ αφειϲ την οι} & 7 &  &  \\
&  & 7 & \foreignlanguage{greek}{κειαν αυτου και δουϲ τοιϲ δουλοιϲ αυτου} & 13 &  &  \\
&  & 14 & \foreignlanguage{greek}{την εξουϲιαν και εκαϲτω το εργον αυτου} & 20 &  &  \\
[0.2em]
\cline{4-4}
\end{tabular}
\end{center}
\end{table}
}
\clearpage
\newpage
 {
 \setlength\arrayrulewidth{1pt}
\begin{table}
\begin{center}
\begin{tabular}{ccc|l|ccc}
\cline{4-4} \\ [-1em]
\multicolumn{7}{c}{\foreignlanguage{greek}{ευαγγελιον κατα μαρκον} \textbf{(\nospace{13:34})} } \\ \\ [-1em] % Si on veut ajouter les bordures latérales, remplacer {7}{c} par {7}{|c|}
\cline{4-4} \\
\cline{4-4}
&  &  & &  &  & \\ [-0.9em]
&  & 21 & \foreignlanguage{greek}{και τω θυρωρω ενετιλατο ινα γρηγορη} & 26 &  &  \\
& \textbf{35} &  & \foreignlanguage{greek}{γρηγοριται ουν ουκ οιδατε γαρ ποτε ο \textoverline{κϲ}} & 8 &  &  \\
&  & 9 & \foreignlanguage{greek}{τηϲ οικειαϲ ερχεται οψε η μεϲονυκτιο̅} & 14 &  &  \\
&  & 15 & \foreignlanguage{greek}{η αλεκτοροφωνιαϲ η πρωει μη ελθω̅} & 2 & \textbf{36} &  \\
&  & 3 & \foreignlanguage{greek}{εξεφνηϲ ευρη υμαϲ καθευδονταϲ} & 6 &  &  \\
& \textbf{37} &  & \foreignlanguage{greek}{α δε υμιν λεγω παϲιν γρηγοριται} & 6 &  &  \\
& \mygospelchapter &  & \foreignlanguage{greek}{ην δε το παϲχα και τα αζυμα μετα δυο} & 9 &  &  \\
&  & 10 & \foreignlanguage{greek}{ημεραϲ και εζητουν οι αρχιερειϲ και οι} & 16 &  &  \\
&  & 17 & \foreignlanguage{greek}{φαριϲαιοι πωϲ αυτον δολω κρατηϲοντεϲ} & 21 &  &  \\
&  & 22 & \foreignlanguage{greek}{αποκτινωϲιν ελεγον δε μη εν τη εορ} & 6 & \textbf{2} &  \\
&  & 6 & \foreignlanguage{greek}{τη μηποτε θορυβοϲ εϲται του λαου} & 11 &  &  \\
& \textbf{3} &  & \foreignlanguage{greek}{και οντοϲ αυτου εν βηθανια εν τη οικεια} & 8 &  &  \\
&  & 9 & \foreignlanguage{greek}{ϲιμωνοϲ του λεπρου κατακειμενου αυ} & 13 &  &  \\
&  & 13 & \foreignlanguage{greek}{του γυνη προϲηλθεν εχουϲα αλαβα} & 17 &  &  \\
&  & 17 & \foreignlanguage{greek}{ϲτρον μυρου ναρδου πιϲτικηϲ πολυ} & 21 &  &  \\
&  & 21 & \foreignlanguage{greek}{τιμου και ϲυντριψαϲα το αλαβαϲτρο̅} & 25 &  &  \\
&  & 26 & \foreignlanguage{greek}{κατεχεεν αυτου τηϲ κεφαληϲ} & 29 &  &  \\
& \textbf{4} &  & \foreignlanguage{greek}{ηϲαν δε τινεϲ των μαθητων αγανα} & 6 &  &  \\
&  & 6 & \foreignlanguage{greek}{κτουντεϲ προϲ εαυτουϲ και λεγοντεϲ} & 10 &  &  \\
&  & 11 & \foreignlanguage{greek}{ειϲ τι η απωλεια αυτη γεγονεν εδυ} & 1 & \textbf{5} &  \\
&  & 1 & \foreignlanguage{greek}{νατο γαρ πραθηναι το μυρον επανω δη} & 7 &  &  \\
&  & 7 & \foreignlanguage{greek}{ναριων \textoverline{τ} και δοθηναι τοιϲ πτωχοιϲ} & 12 &  &  \\
&  & 13 & \foreignlanguage{greek}{και ενεβριμουντο αυτη} & 15 &  &  \\
& \textbf{6} &  & \foreignlanguage{greek}{ο δε \textoverline{ιϲ} ειπεν αυτοιϲ αφετε αυτην τι} & 8 &  &  \\
&  & 9 & \foreignlanguage{greek}{αυτη κοπον παρεχεται καλον γαρ ερ} & 14 &  &  \\
&  & 14 & \foreignlanguage{greek}{γον ηργαϲατο εν εμοι παντοτε γαρ} & 2 & \textbf{7} &  \\
&  & 3 & \foreignlanguage{greek}{τουϲ πτωχουϲ εχεται μεθ υμων και} & 8 &  &  \\
&  & 9 & \foreignlanguage{greek}{οταν θεληται δυναϲθαι αυτοιϲ ευ ποι} & 14 &  &  \\
&  & 14 & \foreignlanguage{greek}{ηϲαι εμε δε ου παντοτε εχεται ο ειχε̅} & 2 & \textbf{8} &  \\
&  & 3 & \foreignlanguage{greek}{εποιηϲεν προελαβεν μυριϲαι μου το ϲωμα} & 8 &  &  \\
[0.2em]
\cline{4-4}
\end{tabular}
\end{center}
\end{table}
}
\clearpage
\newpage
 {
 \setlength\arrayrulewidth{1pt}
\begin{table}
\begin{center}
\begin{tabular}{ccc|l|ccc}
\cline{4-4} \\ [-1em]
\multicolumn{7}{c}{\foreignlanguage{greek}{ευαγγελιον κατα μαρκον} \textbf{(\nospace{14:8})} } \\ \\ [-1em] % Si on veut ajouter les bordures latérales, remplacer {7}{c} par {7}{|c|}
\cline{4-4} \\
\cline{4-4}
&  &  & &  &  & \\ [-0.9em]
&  & 9 & \foreignlanguage{greek}{ειϲ τον ενταφιαϲμον αμην λεγω υμιν} & 3 & \textbf{9} &  \\
&  & 4 & \foreignlanguage{greek}{οτι οπου αν κηρυχθη το ευαγγελιον ειϲ} & 10 &  &  \\
&  & 11 & \foreignlanguage{greek}{ολον τον κοϲμον και ο εποιηϲεν αυτη} & 17 &  &  \\
&  & 18 & \foreignlanguage{greek}{λαληθηϲεται ειϲ μνημοϲυνον αυτηϲ} & 21 &  &  \\
& \textbf{10} &  & \foreignlanguage{greek}{και ιδου ο ιουδαϲ ο ιϲκαριωτηϲ ειϲ των \textoverline{ιβ}} & 9 &  &  \\
&  & 10 & \foreignlanguage{greek}{απηλθεν προϲ τουϲ αρχιερειϲ ινα παρα} & 15 &  &  \\
&  & 15 & \foreignlanguage{greek}{δοι αυτον οι δε ακουϲαντεϲ εχαρη} & 4 & \textbf{11} &  \\
&  & 4 & \foreignlanguage{greek}{ϲαν και επηγγειλαντο αυτω αργυριο̅} & 8 &  &  \\
&  & 9 & \foreignlanguage{greek}{δουναι και εζητι πωϲ αυτον ευκαι} & 14 &  &  \\
&  & 14 & \foreignlanguage{greek}{ρωϲ παραδοι και τη πρωτη ημερα} & 4 & \textbf{12} &  \\
&  & 5 & \foreignlanguage{greek}{των αζυμων οτε το παϲχα εθυον} & 10 &  &  \\
&  & 11 & \foreignlanguage{greek}{λεγουϲιν αυτω οι μαθηται αυτου που} & 16 &  &  \\
&  & 17 & \foreignlanguage{greek}{θελειϲ απελθοντεϲ ετοιμαϲωμεν ινα} & 20 &  &  \\
&  & 21 & \foreignlanguage{greek}{φαγηϲ το παϲχα και αποϲτιλαϲ των} & 3 & \textbf{13} &  \\
&  & 4 & \foreignlanguage{greek}{μαθητων αυτου δυο λεγει αυτοιϲ} & 8 &  &  \\
&  & 9 & \foreignlanguage{greek}{υπαγεται ειϲ την πολιν και ειϲελθο̅} & 14 &  &  \\
&  & 14 & \foreignlanguage{greek}{των υμων απαντηϲει υμιν \textoverline{ανοϲ} κερα} & 19 &  &  \\
&  & 19 & \foreignlanguage{greek}{μιον υδατοϲ βαϲταζων ακολουθηϲατε} & 22 &  &  \\
&  & 23 & \foreignlanguage{greek}{αυτω οπου αν ειϲελθη ειπατε τω οικο} & 6 & \textbf{14} &  \\
&  & 6 & \foreignlanguage{greek}{δεϲποτη οτι ο διδαϲκαλοϲ λεγει} & 10 &  &  \\
&  & 11 & \foreignlanguage{greek}{που εϲτιν το καταλυμα μου οπου το} & 17 &  &  \\
&  & 18 & \foreignlanguage{greek}{παϲχα μετα των μαθητων μου φαγο} & 23 &  &  \\
&  & 23 & \foreignlanguage{greek}{μαι και αυτοϲ υμιν δειξει αναγιον} & 5 & \textbf{15} &  \\
&  & 6 & \foreignlanguage{greek}{μεγα εϲτρωμενον ετοιμον εκει ετοι} & 10 &  &  \\
&  & 10 & \foreignlanguage{greek}{μαϲαται ημιν και εξηλθον ετοιμαϲαι} & 3 & \textbf{16} &  \\
&  & 4 & \foreignlanguage{greek}{οι μαθηται αυτου και ηλθον ειϲ την πο} & 11 &  &  \\
&  & 11 & \foreignlanguage{greek}{λιν και ευρον καθωϲ ειπεν αυτοιϲ και} & 17 &  &  \\
&  & 18 & \foreignlanguage{greek}{ητοιμαϲαν το παϲχα και οψιαϲ γενο} & 3 & \textbf{17} &  \\
&  & 3 & \foreignlanguage{greek}{μενηϲ ερχεται μετα των \textoverline{ιβ} και ανακει} & 2 & \textbf{18} &  \\
&  & 2 & \foreignlanguage{greek}{μενων αυτων και εϲθιοντων ειπεν ο \textoverline{ιϲ}} & 8 &  &  \\
[0.2em]
\cline{4-4}
\end{tabular}
\end{center}
\end{table}
}
\clearpage
\newpage
 {
 \setlength\arrayrulewidth{1pt}
\begin{table}
\begin{center}
\begin{tabular}{ccc|l|ccc}
\cline{4-4} \\ [-1em]
\multicolumn{7}{c}{\foreignlanguage{greek}{ευαγγελιον κατα μαρκον} \textbf{(\nospace{14:18})} } \\ \\ [-1em] % Si on veut ajouter les bordures latérales, remplacer {7}{c} par {7}{|c|}
\cline{4-4} \\
\cline{4-4}
&  &  & &  &  & \\ [-0.9em]
&  & 9 & \foreignlanguage{greek}{αμην λεγω υμιν οτι ειϲ εξ υμω με παρα} & 17 &  &  \\
&  & 17 & \foreignlanguage{greek}{δωϲει ο εϲθιων μετ εμου οι δε ηρξαντο} & 3 & \textbf{19} &  \\
&  & 4 & \foreignlanguage{greek}{λυπιϲθαι και λεγειν αυτω ειϲ καθ ειϲ} & 10 &  &  \\
&  & 11 & \foreignlanguage{greek}{μητι εγω ο δε αποκριθειϲ ειπεν αυτοιϲ} & 5 & \textbf{20} &  \\
&  & 6 & \foreignlanguage{greek}{ειϲ των \textoverline{ιβ} ο ενβαπτομενοϲ μετ εμου} & 12 &  &  \\
&  & 13 & \foreignlanguage{greek}{ειϲ το τρυβλιον ο μεν υιοϲ του \textoverline{ανου} πα} & 6 & \textbf{21} &  \\
&  & 6 & \foreignlanguage{greek}{ραδιδοτε υπαγει καθωϲ γεγραπται πε} & 10 &  &  \\
&  & 10 & \foreignlanguage{greek}{ρι αυτου ουαι δε τω \textoverline{ανω} εκεινω δι ου ο} & 19 &  &  \\
&  & 20 & \foreignlanguage{greek}{υιοϲ του \textoverline{ανου} παραδιδοτε καλον αυτω} & 25 &  &  \\
&  & 26 & \foreignlanguage{greek}{ει ουκ εγεννηθη ο \textoverline{ανοϲ} εκεινοϲ και εϲθι} & 2 & \textbf{22} &  \\
&  & 2 & \foreignlanguage{greek}{οντων λαβων αρτον ευλογηϲαϲ εκλαϲε̅} & 6 &  &  \\
&  & 7 & \foreignlanguage{greek}{και εδιδου αυτοιϲ και ειπεν αυτοιϲ} & 12 &  &  \\
&  & 13 & \foreignlanguage{greek}{λαβεται τουτο το ϲωμα μου και λαβω̅} & 2 & \textbf{23} &  \\
&  & 3 & \foreignlanguage{greek}{το ποτηριον ευχαριϲτηϲαϲ εδωκεν τοιϲ} & 7 &  &  \\
&  & 8 & \foreignlanguage{greek}{μαθηταιϲ και επιον εξ αυτου παντεϲ} & 13 &  &  \\
& \textbf{24} &  & \foreignlanguage{greek}{και ειπεν αυτοιϲ τουτο εϲτιν το αιμα μου} & 8 &  &  \\
&  & 9 & \foreignlanguage{greek}{το τηϲ διαθηκηϲ το υπερ πολλων εκχυ̅} & 15 &  &  \\
&  & 15 & \foreignlanguage{greek}{νομενον ειϲ αφεϲιν αμαρτιων} & 18 &  &  \\
& \textbf{25} &  & \foreignlanguage{greek}{αμην λεγω υμιν οτι ου μη πιω εκ του γε} & 10 &  &  \\
&  & 10 & \foreignlanguage{greek}{νηματοϲ τηϲ αμπελου εωϲ τηϲ ημεραϲ} & 15 &  &  \\
&  & 16 & \foreignlanguage{greek}{εκεινηϲ οταν αυτο πινω καινον εν τη} & 22 &  &  \\
&  & 23 & \foreignlanguage{greek}{βαϲιλεια του \textoverline{θυ} και υμνηϲαντεϲ εξηλ} & 3 & \textbf{26} &  \\
&  & 3 & \foreignlanguage{greek}{θον ειϲ το οροϲ των ελεων} & 8 &  &  \\
& \textbf{27} &  & \foreignlanguage{greek}{και λεγει αυτοιϲ ο \textoverline{ιϲ} οτι παντεϲ ϲκανδα} & 8 &  &  \\
&  & 8 & \foreignlanguage{greek}{λιϲθηϲεϲθαι εν εμοι εν τη νυκτι ταυτη} & 15 &  &  \\
&  & 16 & \foreignlanguage{greek}{οτι γεγραπται παταξω τον ποιμενα και} & 21 &  &  \\
&  & 22 & \foreignlanguage{greek}{τα προβατα ϲκορπιϲθηϲεται αλλα με} & 2 & \textbf{28} &  \\
&  & 2 & \foreignlanguage{greek}{τα το εγερθηναι με εκ νεκρων προαξω} & 8 &  &  \\
&  & 9 & \foreignlanguage{greek}{υμαϲ ειϲ την γαλιλαιαν} & 12 &  &  \\
& \textbf{29} &  & \foreignlanguage{greek}{ο δε πετροϲ αποκριθειϲ λεγει αυτω ει ϗ} & 8 &  &  \\
[0.2em]
\cline{4-4}
\end{tabular}
\end{center}
\end{table}
}
\clearpage
\newpage
 {
 \setlength\arrayrulewidth{1pt}
\begin{table}
\begin{center}
\begin{tabular}{ccc|l|ccc}
\cline{4-4} \\ [-1em]
\multicolumn{7}{c}{\foreignlanguage{greek}{ευαγγελιον κατα μαρκον} \textbf{(\nospace{14:29})} } \\ \\ [-1em] % Si on veut ajouter les bordures latérales, remplacer {7}{c} par {7}{|c|}
\cline{4-4} \\
\cline{4-4}
&  &  & &  &  & \\ [-0.9em]
&  & 9 & \foreignlanguage{greek}{παντεϲ ϲκανδαλιϲθηϲονται αλλ ουκ εγω} & 14 &  &  \\
& \textbf{30} &  & \foreignlanguage{greek}{και λεγει αυτω ο \textoverline{ιϲ} αμην λεγω οτι ϲυ ϲη} & 10 &  &  \\
&  & 10 & \foreignlanguage{greek}{μερον τη νυκτι ταυτη πριν αλεκτορα} & 15 &  &  \\
&  & 16 & \foreignlanguage{greek}{φωνηϲαι τριϲ με αρνηϲη ο δε πετροϲ μαλ} & 4 & \textbf{31} &  \\
&  & 4 & \foreignlanguage{greek}{λον περιϲϲωϲ ελεγεν οτι εαν με δεη ϲυ̅} & 11 &  &  \\
&  & 11 & \foreignlanguage{greek}{αποθανειν ϲοι ου μη ϲε απαρνηϲομαι} & 16 &  &  \\
&  & 17 & \foreignlanguage{greek}{ωϲαυτωϲ δε και παντεϲ ελεγον} & 21 &  &  \\
& \textbf{32} &  & \foreignlanguage{greek}{και εξερχονται ειϲ χωριον ου το ονομα} & 7 &  &  \\
&  & 8 & \foreignlanguage{greek}{γεϲϲημανιν και λεγει τοιϲ μαθηταιϲ} & 12 &  &  \\
&  & 13 & \foreignlanguage{greek}{αυτου καθειϲατε ωδε εωϲ προϲευξω} & 17 &  &  \\
&  & 17 & \foreignlanguage{greek}{μαι και παραλαμβανει τον πετρον} & 4 & \textbf{33} &  \\
&  & 5 & \foreignlanguage{greek}{και τον ιακωβον και τον ιωαννην} & 10 &  &  \\
&  & 11 & \foreignlanguage{greek}{μετ αυτου και ηρξατο εκθαμβιϲθαι} & 15 &  &  \\
&  & 16 & \foreignlanguage{greek}{και αδημονειν και λεγει αυτοιϲ πε} & 4 & \textbf{34} &  \\
&  & 4 & \foreignlanguage{greek}{ριλυποϲ εϲτιν η ψυχη μου εωϲ θανατου} & 10 &  &  \\
&  & 11 & \foreignlanguage{greek}{μιναται ωδε και γρηγορειται} & 14 &  &  \\
& \textbf{35} &  & \foreignlanguage{greek}{και προελθων μικρον επεϲεν επι την} & 6 &  &  \\
&  & 7 & \foreignlanguage{greek}{γην και προϲηυχετο ινα ει δυνατον εϲτι̅} & 13 &  &  \\
&  & 14 & \foreignlanguage{greek}{ινα παρελθη απ αυτου η ωρα και ελεγε̅} & 2 & \textbf{36} &  \\
&  & 3 & \foreignlanguage{greek}{αββα ο \textoverline{πηρ} μου παντα δυνατα ϲοι εϲτι̅} & 10 &  &  \\
&  & 11 & \foreignlanguage{greek}{παρενεγκε το ποτηριον τουτο απ εμου} & 16 &  &  \\
&  & 17 & \foreignlanguage{greek}{αλλα ου τι εγω θελω αλλα τι ϲυ} & 24 &  &  \\
& \textbf{37} &  & \foreignlanguage{greek}{και ερχεται και ευριϲκει αυτουϲ καθευ} & 6 &  &  \\
&  & 6 & \foreignlanguage{greek}{δονταϲ και λεγει τω πετρω ϲιμων} & 11 &  &  \\
&  & 12 & \foreignlanguage{greek}{καθευδειϲ ουκ ιϲχυϲαϲ μιαν ωραν} & 16 &  &  \\
&  & 17 & \foreignlanguage{greek}{γρηγορηϲαι γρηγορειται και προϲευ} & 3 & \textbf{38} &  \\
&  & 3 & \foreignlanguage{greek}{χεϲθαι ινα μη ειϲελθηται ειϲ πιραϲμο̅} & 8 &  &  \\
&  & 9 & \foreignlanguage{greek}{το μεν \textoverline{πνα} προθυμον η δε ϲαρξ αϲθε} & 16 &  &  \\
&  & 16 & \foreignlanguage{greek}{νηϲ και παλιν απελθων προϲηυξατο} & 4 & \textbf{39} &  \\
&  & 5 & \foreignlanguage{greek}{τον αυτον λογον ειπων ϗ υποϲτρεψαϲ} & 2 & \textbf{40} &  \\
[0.2em]
\cline{4-4}
\end{tabular}
\end{center}
\end{table}
}
\clearpage
\newpage
 {
 \setlength\arrayrulewidth{1pt}
\begin{table}
\begin{center}
\begin{tabular}{ccc|l|ccc}
\cline{4-4} \\ [-1em]
\multicolumn{7}{c}{\foreignlanguage{greek}{ευαγγελιον κατα μαρκον} \textbf{(\nospace{14:40})} } \\ \\ [-1em] % Si on veut ajouter les bordures latérales, remplacer {7}{c} par {7}{|c|}
\cline{4-4} \\
\cline{4-4}
&  &  & &  &  & \\ [-0.9em]
&  & 3 & \foreignlanguage{greek}{ευρεν αυτουϲ παλιν καθευδονταϲ} & 6 &  &  \\
&  & 7 & \foreignlanguage{greek}{ηϲαν γαρ οι οφθαλμοι αυτων καταβαρου} & 12 &  &  \\
&  & 12 & \foreignlanguage{greek}{μενοι και ουκ ηδιϲαν τι αυτω αποκρι} & 18 &  &  \\
&  & 18 & \foreignlanguage{greek}{θωϲιν και ερχεται το τριτον και λεγει} & 6 & \textbf{41} &  \\
&  & 7 & \foreignlanguage{greek}{αυτοιϲ καθευδεται λοιπον και αναπαυϲ} & 11 &  &  \\
&  & 11 & \foreignlanguage{greek}{εϲθαι απεχει το τελοϲ ιδου ηλθεν η ωρα} & 18 &  &  \\
&  & 19 & \foreignlanguage{greek}{και παραδιδοτε ο υιοϲ του \textoverline{ανου} ειϲ ταϲ} & 26 &  &  \\
&  & 27 & \foreignlanguage{greek}{χειραϲ των αμαρτωλων εγειρεϲθαι} & 1 & \textbf{42} &  \\
&  & 2 & \foreignlanguage{greek}{αγωμεν ιδου ο παραδιδουϲ με ηγγικεν} & 7 &  &  \\
& \textbf{43} &  & \foreignlanguage{greek}{και ετι αυτου λαλουντοϲ παραγινεται ιου} & 6 &  &  \\
&  & 6 & \foreignlanguage{greek}{δαϲ ειϲ των \textoverline{ιβ} και μετ αυτου οχλοϲ πολυϲ} & 14 &  &  \\
&  & 15 & \foreignlanguage{greek}{μετα μαχαιρων και ξυλων παρα των} & 20 &  &  \\
&  & 21 & \foreignlanguage{greek}{αρχιερεων και γραμματεων και πρεϲ} & 25 &  &  \\
&  & 25 & \foreignlanguage{greek}{βυτερων δεδωκει δε ο παραδιδουϲ} & 4 & \textbf{44} &  \\
&  & 5 & \foreignlanguage{greek}{αυτον ϲυϲημον λεγων αυτοιϲ ον αν} & 10 &  &  \\
&  & 11 & \foreignlanguage{greek}{φιληϲω αυτοϲ εϲτιν κρατηϲατε αυτο̅} & 15 &  &  \\
&  & 16 & \foreignlanguage{greek}{και απαγαγεται αϲφαλωϲ και ελθων ευ} & 3 & \textbf{45} &  \\
&  & 3 & \foreignlanguage{greek}{θεωϲ προϲελθων αυτω λεγει χαιρε} & 7 &  &  \\
&  & 8 & \foreignlanguage{greek}{ραββει και κατεφιληϲεν αυτον} & 11 &  &  \\
& \textbf{46} &  & \foreignlanguage{greek}{οι δε επεβαλον ταϲ χειραϲ αυτων και εκρα} & 8 &  &  \\
&  & 8 & \foreignlanguage{greek}{τουν αυτον και ειϲ τιϲ των παρεϲτω} & 5 & \textbf{47} &  \\
&  & 5 & \foreignlanguage{greek}{των ϲπαϲαμενοϲ μαχαιραν επεϲεν} & 8 &  &  \\
&  & 9 & \foreignlanguage{greek}{τον δουλον του αρχιερεωϲ και αφιλε̅} & 14 &  &  \\
&  & 15 & \foreignlanguage{greek}{αυτου το ωτιον και αποκριθειϲ} & 2 & \textbf{48} &  \\
&  & 3 & \foreignlanguage{greek}{ο \textoverline{ιϲ} ειπεν αυτοιϲ ωϲ επι ληϲτην εξηλ} & 10 &  &  \\
&  & 10 & \foreignlanguage{greek}{θατε μετα μαχαιρων και ξυλων ϲυνλα} & 15 &  &  \\
&  & 15 & \foreignlanguage{greek}{βειν με καθ ημεραν ημην προϲ υμαϲ} & 5 & \textbf{49} &  \\
&  & 6 & \foreignlanguage{greek}{εν τω ιερω διδαϲκων και ουκ εκρατη} & 12 &  &  \\
&  & 12 & \foreignlanguage{greek}{ϲατε με αλλ ινα πληρωθωϲιν αι γρα} & 18 &  &  \\
&  & 18 & \foreignlanguage{greek}{φαι των προφητων} & 20 &  &  \\
[0.2em]
\cline{4-4}
\end{tabular}
\end{center}
\end{table}
}
\clearpage
\newpage
 {
 \setlength\arrayrulewidth{1pt}
\begin{table}
\begin{center}
\begin{tabular}{ccc|l|ccc}
\cline{4-4} \\ [-1em]
\multicolumn{7}{c}{\foreignlanguage{greek}{ευαγγελιον κατα μαρκον} \textbf{(\nospace{14:50})} } \\ \\ [-1em] % Si on veut ajouter les bordures latérales, remplacer {7}{c} par {7}{|c|}
\cline{4-4} \\
\cline{4-4}
&  &  & &  &  & \\ [-0.9em]
& \textbf{50} &  & \foreignlanguage{greek}{τοτε οι μαθηται αυτου αφεντεϲ αυτον πα̅} & 7 &  &  \\
&  & 7 & \foreignlanguage{greek}{τεϲ εφυγον και ειϲ τιϲ νεανιϲκοϲ η} & 5 & \textbf{51} &  \\
&  & 5 & \foreignlanguage{greek}{κολουθι αυτω περιβεβλημενοϲ ϲινδονα} & 8 &  &  \\
&  & 9 & \foreignlanguage{greek}{οι δε νεανιϲκοι εκρατηϲαν αυτον ο δε} & 2 & \textbf{52} &  \\
&  & 3 & \foreignlanguage{greek}{καταλιπων την ϲινδονα γυμνοϲ εφυ} & 7 &  &  \\
&  & 7 & \foreignlanguage{greek}{γεν απ αυτων και απηγαγον τον \textoverline{ιν}} & 4 & \textbf{53} &  \\
&  & 5 & \foreignlanguage{greek}{προϲ τον αρχιερεα καιαφαν και ϲυνπο} & 10 &  &  \\
&  & 10 & \foreignlanguage{greek}{ρευονται παντεϲ οι αρχιερειϲ και οι πρεϲ} & 17 &  &  \\
&  & 17 & \foreignlanguage{greek}{βυτεροι και οι γραμματειϲ και ο πετροϲ} & 3 & \textbf{54} &  \\
&  & 4 & \foreignlanguage{greek}{απο μακροθεν ηκολουθει αυτω εωϲ ε} & 9 &  &  \\
&  & 9 & \foreignlanguage{greek}{ϲω ειϲ την αυλην του αρχιερεωϲ και ην} & 16 &  &  \\
&  & 17 & \foreignlanguage{greek}{ϲυνκαθημενοϲ μετα των υπηρετων} & 20 &  &  \\
&  & 21 & \foreignlanguage{greek}{θερμενομενοϲ προϲ το φωϲ} & 24 &  &  \\
& \textbf{55} &  & \foreignlanguage{greek}{οι δε αρχιερειϲ και ολον το ϲυνεδριον} & 7 &  &  \\
&  & 8 & \foreignlanguage{greek}{εζητουν κατα του \textoverline{ιυ} μαρτυριαν ειϲ το} & 14 &  &  \\
&  & 15 & \foreignlanguage{greek}{θανατωϲαι αυτον και ουχ ηυριϲκον} & 19 &  &  \\
& \textbf{56} &  & \foreignlanguage{greek}{πολλοι γαρ εψευδομαρτυρουν κατ αυ} & 5 &  &  \\
&  & 5 & \foreignlanguage{greek}{του λεγοντεϲ οτι ημειϲ ηκουϲαμεν} & 3 & \textbf{58} &  \\
&  & 4 & \foreignlanguage{greek}{αυτου λεγοντοϲ οτι εγω καταλυϲω το̅} & 9 &  &  \\
&  & 10 & \foreignlanguage{greek}{ναον τουτον τον χειροποιητον και δια} & 15 &  &  \\
&  & 16 & \foreignlanguage{greek}{τριων ημερων αλλον αχειροποιητον οι} & 20 &  &  \\
&  & 20 & \foreignlanguage{greek}{κοδομηϲω και ουδε ουτωϲ ην ειϲη η} & 6 & \textbf{59} &  \\
&  & 7 & \foreignlanguage{greek}{μαρτυρια αυτων και αναϲταϲ ο αρχιε} & 4 & \textbf{60} &  \\
&  & 4 & \foreignlanguage{greek}{ρευϲ ειϲ μεϲον επηρωτηϲεν τον \textoverline{ιν} λεγω̅} & 10 &  &  \\
&  & 11 & \foreignlanguage{greek}{οτι ουτοι ϲου καταμαρτυρουϲιν ο δε} & 2 & \textbf{61} &  \\
&  & 3 & \foreignlanguage{greek}{εϲιωπα και ουδεν απεκρινατο και πα} & 8 &  &  \\
&  & 8 & \foreignlanguage{greek}{λιν επηρωτα αυτον εκ δευτερου και} & 13 &  &  \\
&  & 14 & \foreignlanguage{greek}{λεγει αυτω ϲυ ει ο \textoverline{χϲ} ο υιοϲ του ευλογημε} & 23 &  &  \\
&  & 23 & \foreignlanguage{greek}{νου ο δε \textoverline{ιϲ} αποκριθειϲ ειπεν αυτω} & 6 & \textbf{62} &  \\
&  & 7 & \foreignlanguage{greek}{εγω ειμει ϗ οψεϲθαι τον υιον του \textoverline{ανου}} & 14 &  &  \\
[0.2em]
\cline{4-4}
\end{tabular}
\end{center}
\end{table}
}
\clearpage
\newpage
 {
 \setlength\arrayrulewidth{1pt}
\begin{table}
\begin{center}
\begin{tabular}{ccc|l|ccc}
\cline{4-4} \\ [-1em]
\multicolumn{7}{c}{\foreignlanguage{greek}{ευαγγελιον κατα μαρκον} \textbf{(\nospace{14:62})} } \\ \\ [-1em] % Si on veut ajouter les bordures latérales, remplacer {7}{c} par {7}{|c|}
\cline{4-4} \\
\cline{4-4}
&  &  & &  &  & \\ [-0.9em]
&  & 15 & \foreignlanguage{greek}{εκ δεξιων καθημενον τηϲ δυναμεωϲ} & 19 &  &  \\
&  & 20 & \foreignlanguage{greek}{και ερχομενον μετα τηϲ δυναμεωϲ του} & 25 &  &  \\
&  & 26 & \foreignlanguage{greek}{ουρανου ο δε αρχιερευϲ ευθυϲ διαρηξαϲ} & 5 & \textbf{63} &  \\
&  & 6 & \foreignlanguage{greek}{τουϲ χειτωναϲ αυτου λεγει τι ετι χρειαν εχο} & 13 &  &  \\
&  & 13 & \foreignlanguage{greek}{μεν μαρτυρων ηκουϲατε παντεϲ την} & 3 & \textbf{64} &  \\
&  & 4 & \foreignlanguage{greek}{βλαϲφημιαν του ϲτοματοϲ αυτου τι φαι} & 9 &  &  \\
&  & 9 & \foreignlanguage{greek}{νεται υμιν και παντεϲ κατεκριναν αυ} & 14 &  &  \\
&  & 14 & \foreignlanguage{greek}{τον ειναι ενοχον θανατου και ηρξαν} & 2 & \textbf{65} &  \\
&  & 2 & \foreignlanguage{greek}{το τινεϲ ενπτυειν αυτω και περικαλυ} & 7 &  &  \\
&  & 7 & \foreignlanguage{greek}{πτιν το προϲωπον αυτου και κολαφιζει̅} & 12 &  &  \\
&  & 13 & \foreignlanguage{greek}{αυτον και λεγειν προφητευϲον νυν \textoverline{χε}} & 18 &  &  \\
&  & 19 & \foreignlanguage{greek}{τιϲ εϲτιν ο πεϲαϲ ϲε και οι υπηρετε ρα} & 27 &  &  \\
&  & 27 & \foreignlanguage{greek}{πιϲμαϲιν αυτον ελαμβανον και οντοϲ} & 2 & \textbf{66} &  \\
&  & 3 & \foreignlanguage{greek}{πετρου εν τη αυλη κατω ερχεται μια τω̅} & 10 &  &  \\
&  & 11 & \foreignlanguage{greek}{παιδιϲκων του αρχιερεωϲ και ιδουϲα το̅} & 3 & \textbf{67} &  \\
&  & 4 & \foreignlanguage{greek}{πετρον θερμενομενον εμβλεψαϲα αυ} & 7 &  &  \\
&  & 7 & \foreignlanguage{greek}{τω λεγει και ϲυ μετα του ναζαρηνου \textoverline{ιυ} ηϲ} & 15 &  &  \\
& \textbf{68} &  & \foreignlanguage{greek}{ο δε ηρνηϲατο λεγων ουτε οιδα ουτε επι} & 8 &  &  \\
&  & 8 & \foreignlanguage{greek}{ϲταμαι ϲυ τι λεγειϲ και εξηλθεν ειϲ την} & 15 &  &  \\
&  & 16 & \foreignlanguage{greek}{εξω αυλην και η παιδιϲκη ιδουϲα αυτο̅} & 5 & \textbf{69} &  \\
&  & 6 & \foreignlanguage{greek}{ηρξατο λεγειν τοιϲ παρεϲτηκοϲιν οτι ου} & 11 &  &  \\
&  & 11 & \foreignlanguage{greek}{τοϲ εξ αυτων εϲτιν ο δε παλιν ηρνη} & 4 & \textbf{70} &  \\
&  & 4 & \foreignlanguage{greek}{ϲατο και μετα μεικρον παλιν οι παριε} & 10 &  &  \\
&  & 10 & \foreignlanguage{greek}{ϲτηκοτεϲ ελεγον τω πετρω αληθωϲ} & 14 &  &  \\
&  & 15 & \foreignlanguage{greek}{εξ αυτων ει ο δε ηρξατο αναθεματι} & 4 & \textbf{71} &  \\
&  & 4 & \foreignlanguage{greek}{ζειν και ομνυειν οτι ουκ οιδα τον \textoverline{ανον}} & 11 &  &  \\
&  & 12 & \foreignlanguage{greek}{τουτον ον λεγεται και ευθεωϲ εκ δευ} & 4 & \textbf{72} &  \\
&  & 4 & \foreignlanguage{greek}{τερου αλεκτωρ εφωνηϲεν και ανα} & 8 &  &  \\
&  & 8 & \foreignlanguage{greek}{μνηϲθειϲ ο πετροϲ του ρηματοϲ ου ειπε̅} & 14 &  &  \\
&  & 15 & \foreignlanguage{greek}{αυτω ο \textoverline{ιϲ} οτι πριν αλεκτορα φωνηϲαι} & 21 &  &  \\
[0.2em]
\cline{4-4}
\end{tabular}
\end{center}
\end{table}
}
\clearpage
\newpage
 {
 \setlength\arrayrulewidth{1pt}
\begin{table}
\begin{center}
\begin{tabular}{ccc|l|ccc}
\cline{4-4} \\ [-1em]
\multicolumn{7}{c}{\foreignlanguage{greek}{ευαγγελιον κατα μαρκον} \textbf{(\nospace{14:72})} } \\ \\ [-1em] % Si on veut ajouter les bordures latérales, remplacer {7}{c} par {7}{|c|}
\cline{4-4} \\
\cline{4-4}
&  &  & &  &  & \\ [-0.9em]
&  & 22 & \foreignlanguage{greek}{τριϲ με απαρνηϲη επιβαλων εκλαιεν} & 26 &  &  \\
& \mygospelchapter &  & \foreignlanguage{greek}{και ευθεωϲ επι το πρωει ϲυμβουλιον ποι} & 7 &  &  \\
&  & 7 & \foreignlanguage{greek}{ηϲαντεϲ οι αρχιερειϲ μετα των πρεϲβυτε} & 12 &  &  \\
&  & 12 & \foreignlanguage{greek}{ρων και των γραμματεων και ολον το ϲυ̅} & 19 &  &  \\
&  & 19 & \foreignlanguage{greek}{εδριον δηϲαντεϲ τον \textoverline{ιν} απηγαγον} & 23 &  &  \\
&  & 24 & \foreignlanguage{greek}{και παρεδωκαν αυτον τω πιλατω} & 28 &  &  \\
& \textbf{2} &  & \foreignlanguage{greek}{και επηρωτηϲεν αυτον ο πειλατοϲ λεγω̅} & 6 &  &  \\
&  & 7 & \foreignlanguage{greek}{ϲυ ει ο βαϲιλευϲ των ιουδαιων ο δε απο} & 15 &  &  \\
&  & 15 & \foreignlanguage{greek}{κριθειϲ ειπεν ϲυ λεγειϲ και κατηγορου̅} & 2 & \textbf{3} &  \\
&  & 3 & \foreignlanguage{greek}{αυτου οι αρχιερειϲ πολλα αυτοϲ δε ου} & 9 &  &  \\
&  & 9 & \foreignlanguage{greek}{δεν απεκρινατο ο δε πειλατοϲ πα} & 4 & \textbf{4} &  \\
&  & 4 & \foreignlanguage{greek}{λιν επηρωτα αυτον λεγων ουκ απο} & 9 &  &  \\
&  & 9 & \foreignlanguage{greek}{κρινη ουδεν ειδε ϲου ποϲα κατηγορου} & 14 &  &  \\
&  & 14 & \foreignlanguage{greek}{ϲιν ο δε \textoverline{ιϲ} ουκετι ουδεν απεκριθη} & 6 & \textbf{5} &  \\
&  & 7 & \foreignlanguage{greek}{ωϲτε θαυμαζειν τον πειλατον} & 10 &  &  \\
& \textbf{6} &  & \foreignlanguage{greek}{κατα δε εορτην ιωθει ο ηγεμων απολυ} & 7 &  &  \\
&  & 7 & \foreignlanguage{greek}{ειν αυτοιϲ ενα δεϲμιον ον ητουντο} & 12 &  &  \\
& \textbf{7} &  & \foreignlanguage{greek}{ην δε τοτε ο λεγομενοϲ βαρναβαϲ με} & 7 &  &  \\
&  & 7 & \foreignlanguage{greek}{τα των ϲταϲιαϲτων δεδεμενοϲ οιτι} & 11 &  &  \\
&  & 11 & \foreignlanguage{greek}{νεϲ εν τη ϲταϲι φονον πεποιηκειϲα̅} & 16 &  &  \\
& \textbf{8} &  & \foreignlanguage{greek}{και αναβοηϲαϲ ο οχλοϲ ηρξατο αιτιϲθαι} & 6 &  &  \\
&  & 7 & \foreignlanguage{greek}{καθωϲ εποιει αυτοιϲ ο δε πειλατοϲ} & 3 & \textbf{9} &  \\
&  & 4 & \foreignlanguage{greek}{απεκριθη αυτοιϲ λεγων θελεται απο} & 8 &  &  \\
&  & 8 & \foreignlanguage{greek}{λυϲω υμιν τον βαϲιλεα των ιουδαιων} & 13 &  &  \\
& \textbf{10} &  & \foreignlanguage{greek}{ηδει γαρ οτι δια φθονον παρεδωκαν αυ} & 7 &  &  \\
&  & 7 & \foreignlanguage{greek}{τον οι αρχιερειϲ οι δε αρχιερειϲ ανε} & 4 & \textbf{11} &  \\
&  & 4 & \foreignlanguage{greek}{ϲιϲαν τον οχλον ινα μαλλον τον βαρ} & 10 &  &  \\
&  & 10 & \foreignlanguage{greek}{ναβαν απολυϲη αυτοιϲ} & 12 &  &  \\
& \textbf{12} &  & \foreignlanguage{greek}{ο δε πειλατοϲ αποκριθειϲ ειπεν αυτοιϲ} & 6 &  &  \\
&  & 7 & \foreignlanguage{greek}{τι ουν ποιηϲω τον βαϲιλεα των ιουδαιω̅} & 13 &  &  \\
[0.2em]
\cline{4-4}
\end{tabular}
\end{center}
\end{table}
}
\clearpage
\newpage
 {
 \setlength\arrayrulewidth{1pt}
\begin{table}
\begin{center}
\begin{tabular}{ccc|l|ccc}
\cline{4-4} \\ [-1em]
\multicolumn{7}{c}{\foreignlanguage{greek}{ευαγγελιον κατα μαρκον} \textbf{(\nospace{15:38})} } \\ \\ [-1em] % Si on veut ajouter les bordures latérales, remplacer {7}{c} par {7}{|c|}
\cline{4-4} \\
\cline{4-4}
&  &  & &  &  & \\ [-0.9em]
& \textbf{38} &  & \foreignlanguage{greek}{απ ανωθεν εωϲ κατω ιδων δε ο κεντυ} & 4 &  &  \\
&  & 4 & \foreignlanguage{greek}{ριων παρεϲτωϲ αυτω οτι κραξαϲ εξεπνευ} & 9 &  &  \\
&  & 9 & \foreignlanguage{greek}{ϲεν ειπεν αληθωϲ ο \textoverline{ανοϲ} ουτοϲ \textoverline{υϲ} ην \textoverline{θυ}} & 17 &  &  \\
& \textbf{40} &  & \foreignlanguage{greek}{ηϲαν δε και γυναικεϲ απο μακροθεν θε} & 7 &  &  \\
&  & 7 & \foreignlanguage{greek}{ωρουϲαι εν αιϲ ην μαριαμ η μαγδαληνη} & 13 &  &  \\
&  & 14 & \foreignlanguage{greek}{και μαρια η ιακωβου μικρου και ιωϲη} & 20 &  &  \\
&  & 22 & \foreignlanguage{greek}{\textoverline{μηρ} και ϲαλωμη και οτε ην εν τη γαλιλαι} & 6 & \textbf{41} &  \\
&  & 6 & \foreignlanguage{greek}{α ηκολουθουν αυτω και διηκονουϲαν} & 10 &  &  \\
&  & 11 & \foreignlanguage{greek}{αυτω και αλλαι πολλαι ϲυναναβαϲαι αυ} & 16 &  &  \\
&  & 16 & \foreignlanguage{greek}{τω ειϲ ιεροϲολυμα και ηδη οψιαϲ γενο} & 4 & \textbf{42} &  \\
&  & 4 & \foreignlanguage{greek}{μενηϲ επι ην παραϲκευη ο εϲτιν προϲαβ} & 10 &  &  \\
&  & 10 & \foreignlanguage{greek}{βατον ελθων ιωϲηϲ ο απο αριμαθειαϲ} & 5 & \textbf{43} &  \\
&  & 6 & \foreignlanguage{greek}{ευϲχημων βουλευτηϲ οϲ και αυτοϲ ην} & 11 &  &  \\
&  & 12 & \foreignlanguage{greek}{προϲδεχομενοϲ την βαϲιλειαν του \textoverline{θυ}} & 16 &  &  \\
&  & 17 & \foreignlanguage{greek}{τολμηϲαϲ ειϲηλθεν προϲ τον πειλατο̅} & 21 &  &  \\
&  & 22 & \foreignlanguage{greek}{και ητηϲατο το ϲωμα του \textoverline{ιυ}} & 27 &  &  \\
& \textbf{44} &  & \foreignlanguage{greek}{ο δε πειλατοϲ εθαυμαϲεν ει ηδη τεθνηκε̅} & 7 &  &  \\
&  & 8 & \foreignlanguage{greek}{και προϲκαλεϲαμενοϲ τον κεντυριωνα} & 11 &  &  \\
&  & 12 & \foreignlanguage{greek}{επηρωτηϲεν αυτον ει ηδη τεθνηκεν} & 16 &  &  \\
& \textbf{45} &  & \foreignlanguage{greek}{και γνουϲ παρα του κεντυριωνοϲ εδωρη} & 6 &  &  \\
&  & 6 & \foreignlanguage{greek}{ϲατο το ϲωμα τω ιωϲη και αγοραϲαϲ ϲιν} & 3 & \textbf{46} &  \\
&  & 3 & \foreignlanguage{greek}{δονα ευθεωϲ ηνεγκεν και καθελων} & 7 &  &  \\
&  & 8 & \foreignlanguage{greek}{αυτον ενειληϲεν ειϲ την ϲινδονα και ε} & 14 &  &  \\
&  & 14 & \foreignlanguage{greek}{θηκεν αυτον εν μνημιω ο ην λελατομη} & 20 &  &  \\
&  & 20 & \foreignlanguage{greek}{μενον εκ τηϲ πετραϲ και προϲεκυλειϲε} & 25 &  &  \\
&  & 26 & \foreignlanguage{greek}{λιθον επι την θυραν του μνημιου} & 31 &  &  \\
& \textbf{47} &  & \foreignlanguage{greek}{η δε μαρια η μαγδαληνη και μαρια η ιωϲη} & 9 &  &  \\
&  & 10 & \foreignlanguage{greek}{\textoverline{μηρ} εθεωρουν που τιθεται και διαγε} & 2 & \mygospelchapter &  \\
&  & 2 & \foreignlanguage{greek}{νομενου του ϲαββατου μαρια η μαγδα} & 7 &  &  \\
&  & 7 & \foreignlanguage{greek}{ληνη και μαρια η ιακωβου κα ϲαλωμη} & 13 &  &  \\
[0.2em]
\cline{4-4}
\end{tabular}
\end{center}
\end{table}
}
\clearpage
\newpage
 {
 \setlength\arrayrulewidth{1pt}
\begin{table}
\begin{center}
\begin{tabular}{ccc|l|ccc}
\cline{4-4} \\ [-1em]
\multicolumn{7}{c}{\foreignlanguage{greek}{ευαγγελιον κατα μαρκον} \textbf{(\nospace{16:1})} } \\ \\ [-1em] % Si on veut ajouter les bordures latérales, remplacer {7}{c} par {7}{|c|}
\cline{4-4} \\
\cline{4-4}
&  &  & &  &  & \\ [-0.9em]
&  & 14 & \foreignlanguage{greek}{ηγοραϲαν αρωματα ινα ειϲελθουϲαι αλι} & 18 &  &  \\
&  & 18 & \foreignlanguage{greek}{ψωϲιν αυτον πρωει μια των ϲαββατων} & 4 & \textbf{2} &  \\
&  & 5 & \foreignlanguage{greek}{ερχονται επι το μνημα ετι ανατιλαντοϲ} & 10 &  &  \\
&  & 11 & \foreignlanguage{greek}{του ηλιου και ελεγον προϲ εαυταϲ τιϲ} & 5 & \textbf{3} &  \\
&  & 6 & \foreignlanguage{greek}{αποκυλιϲη ημιν τον λιθον απο τηϲ θυ} & 12 &  &  \\
&  & 12 & \foreignlanguage{greek}{ραϲ του μνημιου και αναβλεψαϲαι θε} & 3 & \textbf{4} &  \\
&  & 3 & \foreignlanguage{greek}{ωρουϲιν οτι αποκεκυλιϲται ο λιθοϲ} & 7 &  &  \\
&  & 8 & \foreignlanguage{greek}{ην γαρ ϲφοδρα μεγαϲ και ειϲελθου} & 2 & \textbf{5} &  \\
&  & 2 & \foreignlanguage{greek}{ϲαι ειϲ το μνημιον θεωρουϲιν νεανι} & 7 &  &  \\
&  & 7 & \foreignlanguage{greek}{ϲκον καθημενον εν τοιϲ δεξιοιϲ περι} & 12 &  &  \\
&  & 12 & \foreignlanguage{greek}{βεβλημενον ϲτολην λευκην και εξε} & 16 &  &  \\
&  & 16 & \foreignlanguage{greek}{θαμβηθηϲαν ο δε λεγει αυταιϲ μη} & 5 & \textbf{6} &  \\
&  & 6 & \foreignlanguage{greek}{φοβειϲθαι οιδα γαρ οτι \textoverline{ιν} τον ναζαρη} & 12 &  &  \\
&  & 12 & \foreignlanguage{greek}{νον ζητιται τον εϲταυρωμενον ηγερ} & 16 &  &  \\
&  & 16 & \foreignlanguage{greek}{θη ουκ εϲτιν ωδε ειδετε εκει ο τοποϲ} & 23 &  &  \\
&  & 24 & \foreignlanguage{greek}{αυτου εϲτιν οπου εθηκαν αυτον} & 28 &  &  \\
& \textbf{7} &  & \foreignlanguage{greek}{αλλα υπαγετε και ειπατε τοιϲ μαθη} & 6 &  &  \\
&  & 6 & \foreignlanguage{greek}{ταιϲ αυτου και τω πετρω οτι ιδου προ} & 13 &  &  \\
&  & 13 & \foreignlanguage{greek}{αγω υμαϲ ειϲ την γαλιλαιαν εκει αυτο̅} & 19 &  &  \\
&  & 20 & \foreignlanguage{greek}{οψεϲθαι καθωϲ ειπεν υμιν} & 23 &  &  \\
& \textbf{8} &  & \foreignlanguage{greek}{και ακουϲαϲαι εξελθον και εφυγον α} & 6 &  &  \\
&  & 6 & \foreignlanguage{greek}{πο του μνημιου εϲχεν γαρ αυταϲ φοβοϲ} & 12 &  &  \\
&  & 13 & \foreignlanguage{greek}{και εκϲταϲιϲ και ουδενι ουδεν ειπον} & 18 &  &  \\
&  & 19 & \foreignlanguage{greek}{εφοβουντο γαρ αναϲταϲ δε πρωει πρω} & 4 & \textbf{9} &  \\
&  & 4 & \foreignlanguage{greek}{τη ϲαββατου εφανη μαρια τη μαγδαλη} & 9 &  &  \\
&  & 9 & \foreignlanguage{greek}{νη παρ ηϲ εκβεβληκει επτα δαιμονια} & 14 &  &  \\
& \textbf{10} &  & \foreignlanguage{greek}{εκεινη πορευθειϲα απηγγειλεν τοιϲ μετ} & 5 &  &  \\
&  & 6 & \foreignlanguage{greek}{αυτου γενομενοιϲ πενθουϲιν κακει} & 1 & \textbf{11} &  \\
&  & 1 & \foreignlanguage{greek}{νοι ακουϲαντεϲ οτι ζη και εθεαθη υπ αυ} & 8 &  &  \\
&  & 8 & \foreignlanguage{greek}{τηϲ ηπιϲτηϲαν μετα δε ταυτα δυϲιν} & 4 & \textbf{12} &  \\
[0.2em]
\cline{4-4}
\end{tabular}
\end{center}
\end{table}
}
\clearpage
\newpage
 {
 \setlength\arrayrulewidth{1pt}
\begin{table}
\begin{center}
\begin{tabular}{ccc|l|ccc}
\cline{4-4} \\ [-1em]
\multicolumn{7}{c}{\foreignlanguage{greek}{ευαγγελιον κατα μαρκον} \textbf{(\nospace{16:12})} } \\ \\ [-1em] % Si on veut ajouter les bordures latérales, remplacer {7}{c} par {7}{|c|}
\cline{4-4} \\
\cline{4-4}
&  &  & &  &  & \\ [-0.9em]
&  & 5 & \foreignlanguage{greek}{εξ αυτων περιπατουϲιν εφανερωθη εν} & 9 &  &  \\
&  & 10 & \foreignlanguage{greek}{ετερα μορφη πορευομενοιϲ ειϲ αγρον} & 14 &  &  \\
& \textbf{13} &  & \foreignlanguage{greek}{κακεινοι απελθοντεϲ απηγγελον τοιϲ λοι} & 5 &  &  \\
&  & 5 & \foreignlanguage{greek}{ποιϲ ουδε εκεινοιϲ επιϲτευϲαν υϲτερο̅} & 1 & \textbf{14} &  \\
&  & 2 & \foreignlanguage{greek}{ανακειμενοιϲ τοιϲ \textoverline{ιβ} εφανερωθη και ω} & 7 &  &  \\
&  & 7 & \foreignlanguage{greek}{νιδιϲεν την απιϲτιαν αυτων και ϲκλη} & 12 &  &  \\
&  & 12 & \foreignlanguage{greek}{ροκαρδιαν οτι τοιϲ θεαϲαμενοιϲ αυτο̅} & 16 &  &  \\
&  & 17 & \foreignlanguage{greek}{εγηγερμενον ουκ επιϲτευϲαν} & 19 &  &  \\
&  & 20 & \foreignlanguage{greek}{κακεινοι απελογουντε λεγοντεϲ οτι ο} & 24 &  &  \\
&  & 25 & \foreignlanguage{greek}{αιων ουτοϲ τηϲ ανομιαϲ και τηϲ απιϲτιαϲ} & 31 &  &  \\
&  & 32 & \foreignlanguage{greek}{υπο τον ϲαταναν εϲτιν ο μη εων τα υπο} & 40 &  &  \\
&  & 41 & \foreignlanguage{greek}{των \textoverline{πνατων} ακαθαρτα την αληθειαν} & 45 &  &  \\
&  & 46 & \foreignlanguage{greek}{του \textoverline{θυ} καταλαβεϲθαι δυναμιν δια} & 50 &  &  \\
&  & 51 & \foreignlanguage{greek}{τουτο απολακυψον ϲου την δικαιοϲυ} & 55 &  &  \\
&  & 55 & \foreignlanguage{greek}{νην ηδη εκεινοι ελεγον τω \textoverline{χω} και ο} & 62 &  &  \\
&  & 63 & \foreignlanguage{greek}{\textoverline{χϲ} εκεινοιϲ προϲελεγεν οτι πεπληρω} & 67 &  &  \\
&  & 67 & \foreignlanguage{greek}{ται ο οροϲ των ετων τηϲ εξουϲιαϲ του} & 74 &  &  \\
&  & 75 & \foreignlanguage{greek}{ϲατανα αλλα εγγιζει αλλα δινα και υ} & 81 &  &  \\
&  & 81 & \foreignlanguage{greek}{περ ων εγω αμαρτηϲαντων παρεδοθη̅} & 85 &  &  \\
&  & 86 & \foreignlanguage{greek}{ειϲ θανατον ινα υποϲτρεψωϲιν ειϲ τη̅} & 91 &  &  \\
&  & 92 & \foreignlanguage{greek}{αληθειαν και μηκετι αμαρτηϲωϲιν} & 95 &  &  \\
&  & 96 & \foreignlanguage{greek}{ινα την εν τω ουρανω \textoverline{πνικην} και α} & 10 &  &  \\
&  & 10 & \foreignlanguage{greek}{φθαρτον τηϲ δικαιοϲυνηϲ δοξαν} & 10 &  &  \\
&  & 10 & \foreignlanguage{greek}{κληρονομηϲωϲιν αλλα πορευθεν} & 2 & \textbf{15} &  \\
&  & 2 & \foreignlanguage{greek}{τεϲ ειϲ τον κοϲμον απαντα κηρυξατε} & 7 &  &  \\
&  & 8 & \foreignlanguage{greek}{το ευαγγελιον παϲη τη κτιϲει ο πιϲτευ} & 2 & \textbf{16} &  \\
&  & 2 & \foreignlanguage{greek}{ϲαϲ και βαπτιϲθειϲ ϲωθηϲεται ο δε α} & 8 &  &  \\
&  & 8 & \foreignlanguage{greek}{πιϲτηϲαϲ κατακριθειϲ ου ϲωθηϲεται} & 11 &  &  \\
& \textbf{17} &  & \foreignlanguage{greek}{ϲημια δε τοιϲ πιϲτευϲαϲιν ταυτα παρα} & 6 &  &  \\
&  & 6 & \foreignlanguage{greek}{κολουθηϲει εν τω ονοματι μου} & 10 &  &  \\
[0.2em]
\cline{4-4}
\end{tabular}
\end{center}
\end{table}
}
\clearpage
\newpage
 {
 \setlength\arrayrulewidth{1pt}
\begin{table}
\begin{center}
\begin{tabular}{ccc|l|ccc}
\cline{4-4} \\ [-1em]
\multicolumn{7}{c}{\foreignlanguage{greek}{ευαγγελιον κατα μαρκον} \textbf{(\nospace{16:17})} } \\ \\ [-1em] % Si on veut ajouter les bordures latérales, remplacer {7}{c} par {7}{|c|}
\cline{4-4} \\
\cline{4-4}
&  &  & &  &  & \\ [-0.9em]
&  & 11 & \foreignlanguage{greek}{δαιμονια εκβαλουϲιν γλωϲϲαιϲ λαλη} & 14 &  &  \\
&  & 14 & \foreignlanguage{greek}{ϲουϲιν καινεϲ οφειϲ αρουϲιν καν θα} & 4 & \textbf{18} &  \\
&  & 4 & \foreignlanguage{greek}{ναϲιμον τι πιωϲιν ου μη αυτουϲ βλαψη} & 10 &  &  \\
&  & 11 & \foreignlanguage{greek}{επι αρρωϲτουϲ χειραϲ επιθηϲουϲιν και κα} & 16 &  &  \\
&  & 16 & \foreignlanguage{greek}{λωϲ εξουϲιν} & 17 &  &  \\
& \textbf{19} &  & \foreignlanguage{greek}{ο μεν \textoverline{κϲ} \textoverline{ιϲ} \textoverline{χϲ} μετα το λαληϲαι αυτοιϲ ανε} & 10 &  &  \\
&  & 10 & \foreignlanguage{greek}{λημφθη ειϲ τον ουρανον και εκαθειϲε̅} & 15 &  &  \\
&  & 16 & \foreignlanguage{greek}{εκ δεξιων του \textoverline{θυ} εκεινοι δε εξελθο̅} & 3 & \textbf{20} &  \\
&  & 3 & \foreignlanguage{greek}{τεϲ εκηρυξαν πανταχου του \textoverline{κυ} ϲυνερ} & 8 &  &  \\
&  & 8 & \foreignlanguage{greek}{γουντοϲ και τον λογον βεβαιουντοϲ} & 12 &  &  \\
&  & 13 & \foreignlanguage{greek}{δια των επακολουθουντων ϲημιων} & 16 &  &  \\
&  & 17 & \foreignlanguage{greek}{αμην} & 17 &  &  \\
[0.2em]
\cline{4-4}
\end{tabular}
\end{center}
\end{table}
}
\end{document}